%% -*- coding: utf-8 -*-
%% Time-stamp: <Chen Wang: 2025-12-09 20:15:27>

% ○ ◎ ‧ 「 」 『 』 々 ( ) “ ” ■ ^[一-龥]
% 【\([^】][^】][^】]+\)】 → {\\fzxk\\zihao{6}\\textcolor{red}{\1}}
% \(【评】.*\) → {\\cangkai\\zihao{5}\1}
% \(【题解】.*\) → {\\cangkai\\zihao{5}\1}
% 《\([^》]+\)》 → \\CJKunderwave{\1}
% ^\([0-9]+.[0-9]+\) → \\lettrine{\1}
% {\\fzxk\\zihao{6}\\textcolor{red}{[^o]*}}

\setlength{\parindent}{0pt}


\chapter{俭啬第二十九}




{\cangkai\zihao{5}【题解】 俭啬者,节俭与吝啬也。俭之与吝,都是舍不得钱财或消费,表面看来,事虽相近;内里却是,实质有别。勤俭持家,开源节流,同样适于治国,原属美事。春秋时秦穆公向臣下由余问国何为得失,由余曰:“常以俭得之,以奢失之。”(见\CJKunderwave{韩非子·十过}篇)这样,勤俭之道,由经济而升于政治,成为治国纲要。故李商隐\CJKunderwave{咏史}诗云:“历览前贤国与家,成由勤俭败由奢。”勤俭之道,其义甚伟。但是,真理和谬误只差一步。过分节俭,连必需的消费也予禁绝,取消了生活的一切享受和乐趣,这就性质转化,由好事变成坏事,成了货真价实的守财奴吝啬鬼。本门故事,大多是“俭啬”合称,其词倾向于“啬”义。即使单言其“俭”,也受影响,多含俭啬之义。如第一则称和峤“至俭”,就属贬义。其所谓“俭”,已转为点点滴滴也不放过的敲骨吸髓的剥削。如第3则王戎的“既富且贵”,园田周遍天下,即与其夫妇昼夜持筹算计有关,这哪有什么宰相风度!因此,本门九则故事,大多写吝啬鬼的故事及其生活教训,共占八则。只有第八则有关陶侃与庾亮啖薤留白的故事,属于节俭美德,在黑暗中略显一丝希望之光。}

\lettrine{29.1} 和峤\myidx{和峤}性至俭\footnote{和峤:字长舆,汝南西平人。晋初官侍中、中书令。至俭:很吝啬。俭:吝啬。},家有好李,王武子\myidx{王济}求之\footnote{王武子:王济字武子,太原晋阳人。浑子,和峤妻弟。官太仆卿。参前\CJKunderwave{言语}第24则注。},与不过数十。王武子因其上直\footnote{上直:入朝轮值。直,通“值”,值班。},率将少年能食之者\footnote{率将:率领。},持斧诣园,饱共啖毕\footnote{饱共啖毕:饱餐完毕。啖:吃。},伐之。送一车枝与和公,问曰:“何如君李\footnote{何如:相比如何。}?”和既得,唯笑而已。{\fzxk\zihao{6}\textcolor{red}{\CJKunderwave{晋诸公赞}曰:“峤性不通,治家富拟王公,而至俭,将有犯义之名。”\CJKunderwave{语林}曰:“峤诸弟往园中食李,而皆计核责钱。故峤妇弟王济伐之也。”}}

{\cangkai\zihao{5}【评】和峤西朝名士,政治上颇多表现,当时称为名臣。但据杜预揭发,他明显具有“钱癖”的缺陷,见\CJKunderwave{术解}第4则注。其性至俭,此所谓“俭”,是俭吝,而非节俭。其治家富敌五侯,钱是多多益善。就连自家兄弟进果园吃他几个李子,也如外人一样,“计核责钱”——只认钱而不认人,这比今天的资本家还会算计,简直是钱眼中看人而六亲不认。点滴游资尚紧抓不放,更何况是经营田庄、商铺等大规模的剥削收入呢?其发财致富的原因甚多,但“广种博收”——不让一分一厘的金钱从自己手缝中溜掉,也是奥秘之一吧!不过,王济食李伐树,也属恶作剧,但他是公主的丈夫,武帝的驸马,人们又奈他何!李树伐倒,无法起死回生,于是乎和峤一笑了之,倒也表现了政治家的风度。}

\lettrine{29.2} 王戎\myidx{王戎}俭吝\footnote{王戎:字濬冲,琅邪人,魏晋间竹林名士。官至司徒,爵安丰侯。俭吝:吝啬。},其从子婚\footnote{从子:同族侄儿。},与一单衣\footnote{单衣:没夹里的便服。},后更责之\footnote{责:讨,索回。}。{\fzxk\zihao{6}\textcolor{red}{王隐\CJKunderwave{晋书}曰:“戎性至俭,不能自奉养,财不出外,天下人谓为膏肓之疾。”}}

{\cangkai\zihao{5}【评】以下四则,主角都是王戎。送给侄儿的结婚礼物很轻——仅单衣一件,但事后却悔而索讨。这样的吝啬鬼,不怕人家笑话,可见其脸皮之厚,天下少有。见物不见人,亲情何在?魏晋名士,内心之丑陋一面,形象呈现。}

\lettrine{29.3} 司徒王戎\myidx{王戎}既贵且富,区宅、僮牧、膏田、水碓之属\footnote{区宅:房舍。僮牧:奴婢。膏田:良田。水碓:利用水力资源来劳作的作坊。},洛下无比\footnote{洛下:西晋京师洛阳。}。契疏鞅掌\footnote{契疏:契券账簿之类。鞅掌:繁忙。},每与夫人烛下散筹算计\footnote{散筹:摊开筹码。算计:算账,计算家财。}。{\fzxk\zihao{6}\textcolor{red}{\CJKunderwave{晋诸公赞}曰:“戎性简要,不治仪望,自遇甚薄,而产业过丰。论者以为台辅之望不重。”王隐\CJKunderwave{晋书}曰:“戎好治生,园田周遍天下。翁妪二人,常以象牙筹昼夜算计家资。”\CJKunderwave{晋阳秋}曰:“戎多殖财贿,常若不足。或谓戎故以此自晦也。”戴逵论之曰:“王戎晦默于危乱之际,获免忧祸,既明且哲,于是在矣。”或曰:“大臣用心,岂其然乎?”逵曰:“运有险易,时有昏明,如子之言,则蘧瑗、季札之徒,皆负责矣。自古而观,岂一王戎也哉!”}}

{\cangkai\zihao{5}【评】本门9则故事,王戎独占4则,占一半弱,成为并非光彩的当然主角。看来王戎生财有道,聚敛有方,但又“自遇甚薄”,成了古代典型的吝啬鬼守财奴。 但戎为西京清谈名士,人们为贤者讳,或谓如此敛财非其本性,而是生当乱世的自晦自全的明哲之举。偶尔一二件事,或许还可以讨论;但是再三再四,则无法以“晦默”、“免祸”之说塞议者之口。竹林之游时,阮籍斥之曰:“俗物已复来败人意。”其聚敛、俭啬之俗,年轻时已见端倪。故余嘉锡评曰:“观诸书及\CJKunderwave{世说}所言,戎之鄙吝,盖出于天性。戴逵之言,名士相为护惜,阿私所好,非公论也。”所论洞彻肺腑,使名士无遁其形。}

\lettrine{29.4} 王戎\myidx{王戎}有好李\footnote{好李:优良品种的李子。},常卖之,恐人得其种,恒钻其核\footnote{恒:总是。}。

{\cangkai\zihao{5}【评】在人与自然的关系中,为了防止果树品质的蜕化,改进优良品种,人们在长期的农业园艺实践中,不断实验,做出了不懈的努力。因此,“好李”——优良品种的李子,是人类科学实践的心血结晶。但是,王戎为了保护自家“好李”对市场的垄断,竟然钻核卖李,破坏优良品种的流传。吝啬鬼为了赚几个钱,竟然破坏人类科研成果,这是对人类文明的犯罪!}

\lettrine{29.5} 王戎\myidx{王戎}女适裴頠\myidx{裴頠}\footnote{适:嫁。裴頠:字逸民,河南闻喜人。官侍中,尚书左仆射。参前\CJKunderwave{言语}第23则注。} ,贷钱数万\footnote{贷钱:借款。}。女归,戎色不悦,女遽还钱\footnote{遽:立即,赶紧。},乃释然\footnote{释然:不快消失。}。

{\cangkai\zihao{5}【评】两晋统治者提倡“以孝治国”。王戎与和峤是西晋最著名的两大孝子。“孝”是建立在家族血缘关系之上的亲情。但若考察两人对于亲人亲族的态度,则又不能不有所怀疑。和峤对于诸弟,王戎对于侄儿,甚至是自己的亲生女儿,都一样是认钱不认人,缺乏亲情的爱惜。今人对于子女的爱怜,如果不说超过,至少是不亚于对父母的孝顺。从心理学的角度,以今推古,和峤、王戎诸名士,薄于骨肉却能孝其父母,甚至是获得“生孝”与“死孝”之美名,其内里的实质真实性应打上一个大大的问号。封建礼教的虚伪,于此可见一斑:为“名”可装扮孝顺,为钱则六亲不认,两者相形,则面目暴露无遗。}

\lettrine{29.6} 卫江州\myidx{卫展}在寻阳\footnote{卫江州:卫展。南渡初曾任元帝朝廷尉。},{\fzxk\zihao{6}\textcolor{red}{\CJKunderwave{永嘉流人名}曰:“卫展字道舒,河东安邑人。祖列,彭城护军。父韶,广平令。展,光熙初,除鹰扬将军、江州刺史。”}} 有知旧人投之\footnote{知旧人:相知旧友。投:投奔。},都不料理\footnote{料理:招待,看顾。},唯饷王不【】行一斤\footnote{饷:馈赠。王不【】行:宋本原缺一字,据\CJKunderwave{类说}卷三一引为“留”字。刘注同。王不留行,中药名,\CJKunderwave{本草纲目}曰:“此物性走而不住,虽有王命不能留其行,故名。”俗又名麦蓝菜、剪金花、兔儿草,产于山东、河北、辽宁等地。},此人得饷便命驾\footnote{命驾:命驭者驾车立即离开。}。{\fzxk\zihao{6}\textcolor{red}{\CJKunderwave{本草}曰:“王不留行生太山,治金疮、除风,久服之轻身。”}} 李弘范\myidx{李轨}闻之曰:“家舅刻薄,乃复驱使卉木\footnote{乃复:竟然。}。”{\fzxk\zihao{6}\textcolor{red}{\CJKunderwave{中兴书}曰:“李轨字弘范,江夏人。仕至尚书郎。”按:轨,刘氏之甥。此应弘度,非弘范者也。}}

{\cangkai\zihao{5}【评】人情冷暖,世态炎凉,自古已然。关系之亲疏,大多以利之大小相较。因此,结交了有用的新友,则弃无用旧知如弊履。人情淡如水,古人如此,今人又将如何?悲乎!}

\lettrine{29.7} 王丞相\myidx{王导}俭节\footnote{王丞相:王导。},帐下甘果盈溢不散\footnote{不散:不散发,不分发掉。},涉春烂败\footnote{涉春:入春,经春。}。都督白之\footnote{都督:此非军队统帅之都督,而是特指总管家务的管家。},公令舍去\footnote{舍去:丢掉。},曰:“慎不可令大郎\myidx{王悦}知!”{\fzxk\zihao{6}\textcolor{red}{王悦也。}}

{\cangkai\zihao{5}【评】王导之节俭,属于正面意义的节约,性质和和峤、王戎之俭吝有异。史称王导即使身为宰相,但其生活“简素寡欲,仓无储谷,衣无重帛”,生活相当简朴。导为官称廉洁,与贪官污吏异道而驰,但这只是个人行为,而无救于官吏腐败之横行。但其家属纳贿,如王导小妾雷氏,因此而有“雷尚书”之称,王导知否?又如王戎是其从兄,照样刻薄小民甚至是亲人以致富。只是王导俭节,有些过分。甘果盈溢,经春腐烂而不肯分发下人食用,这不就成了一毛不拔的铁公鸡了么!作为一代政治家,其心胸气魄又似乎有所欠缺。}

\lettrine{29.8} 苏峻\myidx{苏峻}之乱\footnote{苏峻之乱:咸和二年(327),庾亮辅政,征历阳太守、冠军将军苏峻入朝为大司农,苏峻率兵反,攻入京师。后陶侃、温峤等举义军剿灭之。},庾太尉\myidx{庾亮}南奔见陶公\myidx{陶侃}\footnote{庾太尉:庾亮。陶公:陶侃。按:时庾亮兵败,逃奔温峤与陶侃,共奉侃为义军盟主。},陶公雅相赏重\footnote{雅:很,非常。赏重:赏识器重。}。陶性俭吝,及食啖薤\footnote{啖:吃。薤:蔬菜名,即藠(jiào叫)头。草本植物,地下鳞茎可食,可种。靠近根部的薤头叫薤白,也省称“叫白”。},庾因留白。陶问:“用此何为?”庾云:“故可种。”于是大叹庾非唯风流,兼有治实\footnote{风流:才华横溢。治实:办事务实。}。

{\cangkai\zihao{5}【评】本门故事,唯有此则之“俭吝”,全属褒义,前第7则论王导“节俭”,仍是褒中有贬,态度有所保留。在刘义庆心目中,陶侃是魏晋勤俭理政的典范。有人以为,庾亮食薤留白,是故意做作,以投陶侃所好,是一种愚弄侃以取信任的谲诈。但若全面考察陶、庾二公关系,实属不然。侃出身寒门,自少贫贱,故生活节俭,是其本性使然。但其俭吝,不仅严于律己,而且并不因此苛刻士卒部属。史称其数十年的征战,“凡有虏获,皆分士卒,身无私焉”。甚至是流亡饥民,陶侃也“竭资振给焉”。可见他并不是一个吝啬鬼守财奴。其“俭吝”之行不仅克己奉众,同时也是出于治国施政的需要。“非唯风流,兼有治实”,侃之叹亮,实是自我形象的光辉写照。故刘辰翁评曰:“小说取笑,陶未易愚。”信然。}

\lettrine{29.9} 郗公\myidx{郗愔}大聚敛\footnote{郗公:郗愔,字方回。高平人。官至徐、兖二州刺史。卒赠司空。},有钱数千万。嘉宾\myidx{郗超}意甚不同\footnote{嘉宾:郗超小字,愔子。官至司徒左长史,参前\CJKunderwave{言语}第59则注。},常朝旦问訙(讯)\footnote{常:通“尝”,曾经。}。郗家法,子弟不坐,因倚语移时\footnote{倚语:站着说话。移时:很长时间。},遂及财货事\footnote{财货:钱财。}。郗公曰:“汝正当欲得吾钱耳\footnote{正当:只不过。}!”迺开库一日\footnote{迺:乃。},令任意用。郗公始正谓损数百万许\footnote{正谓:只是以为。},嘉宾遂一日乞与亲友\footnote{乞与:给予,赠送。},周旋略尽\footnote{周旋:应酬,打交道。}。郗公闻之,惊怪不能已已\footnote{已已:停止,休了。}。{\fzxk\zihao{6}\textcolor{red}{\CJKunderwave{中兴书}曰:“超少卓荦而不羁,有旷世之度。”}}

{\cangkai\zihao{5}【评】这则故事,作为\CJKunderwave{俭啬}门的压轴戏,有人物,有故事,有细节,是一篇非常生动的纪实小小说。其中郗愔与郗超这对父子很有意思,一个好聚敛而吝啬守财,积钱数千万以压库;一个慷慨好施,一日散尽家财而不眨一眼。父子俩的不同人生态度及其作为,矛盾冲突,形成了强烈的艺术对比,给人以刺激和艺术感染。在政治上,郗愔虽忠于司马王室,但却是一个平庸之才,暗于事机而聚敛守财,致讥世人而无所作为。郗超则党于桓温而反之,但却在政治上颇具天赋,企望大有作为。因此,他慷慨好施,收买人心,以成其大业。只要政权在握,自然财源滚滚而来,又岂在乎那区区数千万元家财!可惜他作为桓温谋主,跟错了人,终于毁灭了一颗政治新星。}





%%% Local Variables:
%%% mode: latex
%%% TeX-engine: xetex
%%% TeX-master: "../Main"
%%% End:
