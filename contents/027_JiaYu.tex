%% -*- coding: utf-8 -*-
%% Time-stamp: <Chen Wang: 2025-12-09 23:01:22>

% ○ ◎ ‧ 「 」 『 』 々 ( ) “ ” ■ ^[一-龥]
% 【\([^】][^】][^】]+\)】 → {\\fzxk\\zihao{6}\\textcolor{red}{\1}}
% \(【评】.*\) → {\\cangkai\\zihao{5}\1}
% \(【题解】.*\) → {\\cangkai\\zihao{5}\1}
% 《\([^》]+\)》 → \\CJKunderwave{\1}
% ^\([0-9]+.[0-9]+\) → \\lettrine{\1}
% {\\fzxk\\zihao{6}\\textcolor{red}{[^o]*}}

\setlength{\parindent}{0pt}


\chapter{假谲第二十七}




{\cangkai\zihao{5}【题解】 假谲者,虚伪诡诈而好以智术相欺也。从词义看,具有贬义,但也不尽然,要看具体对象而论,对敌人的真诚,就是对自己对同志的犯罪,不示以假又将如何?假谲与真诚是一对矛盾。一般而言,在漫长的人生中, 假诈者常是媚俗阿世,因而在黑暗的社会中,常能获得生存和发展的机会;相反,真诚的高士却常因背俗违众,不肯同流合污而被俗世视为眼中钉肉中刺,必拔去而后快。在社会上,毕竟是真诚者少而媚俗者众,因而假谲者自然不断地扩大其影响和市场。如唐代韩愈称,作文自志高洁而人必以为恶,“小称意人亦小怪,大称意即人必大怪之也”;相反,违心从众而“作俗下文字,下笔令人惭,及示人,则人以为好矣” 。见其\CJKunderwave{与冯宿论文书}。小俗小好,大俗大好,只见假诈媚俗之“光环”,又岂具风霜高洁之真诚!文学是生活的反映,假谲之病,已深入到生活的各个角落。为了一己之安全,为了实现自己的野心,阴谋诡计滥杀无辜,罪行令人发指。但是,生活教训令人清醒。在虚伪诡诈之风流行之时,人们对“假谲”加以改造和利用,本门许多故事,大多是赞扬以智慧欺骗敌人或对手所取得的胜利及成就。如第6则写晋明帝智探王敦军营,颂其英武明断。第9则温峤骗婚觅佳偶,第10则江虨智娶新寡女等,发自内心的善意欺诈,带来了生活的热情和幸福。这样,其所谓\CJKunderwave{假谲},又是智慧幽默之渊薮。这与\CJKunderwave{夙惠}、\CJKunderwave{排调}诸门运用智数颇多相通之处,“假谲”之中潜伏了一颗猛烈跳动的真诚之心。类似的故事具有经验教训,应予以认真研究,而不可一概抹煞。}

\lettrine{27.1} 魏武\myidx{曹操}少时\footnote{魏武:指曹操,字孟德,沛国谯人。汉末统一北方,位至丞相,封魏王。其子曹丕篡位建魏,追尊其为太祖武皇帝,故称。少时:年轻时。},尝与袁绍\myidx{袁绍}好为游侠\footnote{袁绍(?—202):字本初,汉末汝南汝阳人。官至司隶校尉。会诸侯起兵攻董卓,据冀、青、幽、并四州之地。后被曹操大败于官渡,旋病死。游侠:指不顾法制而救人危难,或是故意违法犯禁而逞其豪强。}。观人新婚,因潜入主人园中,夜叫呼云:“有偷儿贼!”青庐中人皆出观\footnote{青庐:古时婚俗搭青布帐篷举行婚礼。},魏武乃入,抽刃劫新妇。与绍还出\footnote{还出:退出。},失道,坠枳棘中\footnote{枳棘:二种有刺之木。},绍不能得动。复大叫云:“偷儿在此!”绍遑迫自掷出\footnote{遑迫:惶恐急迫。掷出:跳出。},遂以俱免。{\fzxk\zihao{6}\textcolor{red}{\CJKunderwave{曹瞒传}曰:“操小字阿瞒。少好谲诈,游放无度。”孙盛\CJKunderwave{杂语}云:“武王少好侠,放荡不修行业。尝私入常侍张让宅中,让乃手戟于庭,逾垣而出,有绝人力,故莫之能害也。”}}

{\cangkai\zihao{5}【评】一个人从小看八十。曹操年轻时即假谲多智,与其合作,当多多小心才是。后与曹操联合者大多因一时之利的诱惑,终坠入其政治圈套之中而难以自拔。操抽刃劫人新妇,智则智矣,意欲何为?为了一己之快,不惜缺德犯法而致人悲剧,不足为训。又:读此故事,则数十年后曹、袁官渡之战的胜负,已可预知。}

\lettrine{27.2} 魏武\myidx{曹操}行役\footnote{行役:行军。},失汲道\footnote{失汲道:找不到水源。},三军皆渴。乃令曰:“前有大梅林,饶子\footnote{饶子:果实累累。},甘酸可以解渴。”士卒闻之,口皆出水。乘此得及前源。

{\cangkai\zihao{5}【评】后世“望梅止渴”的成语,故事即从此出。作为三军统帅,不仅要能率兵征战,同时还必须研究士兵的心理需求,这样才容易驾驭队伍。望梅止渴,正是曹操出于急智的一种心理疗法,从而战胜了一时的生理困难,终于“乘此得及前源”,摆脱困境,通向胜利。这一假谲并非恶作剧,而是工作需要,见曹瞒智慧之闪光。}

\lettrine{27.3} 魏武\myidx{曹操}常谓人欲危己\footnote{危己:危害自己,指暗杀之类事情。},己辄心动\footnote{心动:心有感应,心跳加速。}。因语所亲小人曰:“汝怀刃密来我侧,我必说‘心动’,执汝使行刑,汝但勿言其使\footnote{勿言其使:不要说是我的指使。},无他\footnote{无他:不会有什么事情。},当厚相报。”执者信焉,不以为惧。遂斩之,此人至死不知也。左右以为实,谋逆者挫气矣\footnote{挫气:灰心丧气,泄气。}。{\fzxk\zihao{6}\textcolor{red}{\CJKunderwave{曹瞒传}曰:“操在军,廪谷不足,私语主者曰:‘何如?’主者云:‘可以小斛足之。’操曰:‘善。’后军中言操欺众,操题其主者背以徇曰:‘行小斛,盗军谷。’遂斩之,仍云:‘特当借汝死,以厌众心。’其变诈皆此类也。”}}

{\cangkai\zihao{5}【评】为一己之安全,运用智数,欺诈他人,伤害天理,虽“所亲小人”,在所不顾。“所亲”二字,令人联想多多,所亲者尚且如此,若非亲非故,其态度又将如何?想来令人战栗不已。小人临斩,头颅落地,“至死不知也”,这才是曹阿瞒高人一筹之处,但也是最令人害怕、让人恐惧的地方。刘辰翁评曰:“文字中留此,鬼当夜哭。”当时军阀混战,逐鹿中原,曹操智数才干远超孙(权)、刘(备)、袁(绍)诸雄,但终于只能接受三国鼎立的现实而无法实现其统一中国的大业。过量运其假谲之智而失信天下,似有关涉。}

\lettrine{27.4} 魏武\myidx{曹操}常云\footnote{常:通“尝”,曾经。}:“我眠中不可妄近\footnote{妄近:随意接近。},便斫人亦不自觉,左右宜深慎此!”后阳眠\footnote{阳眠:佯眠,假装睡着。阳,通“佯”。},所幸一人窃以被覆之\footnote{窃以被覆之:悄悄给他盖上被子。窃:暗中,悄悄地。},因便斫杀。自尔每眠\footnote{自尔:从此。},左右莫敢近者。

{\cangkai\zihao{5}【评】人称伴君如伴虎,自古已然,非仅宰臣股肱,即其宠幸近侍之人,同样也是在劫难逃。凌濛初评曰:“所为不良,心亦兢兢,作此多狡。”上行下效,社会风气又将如何?在一个虚伪假诈的社会中,要想生存,下面之人必然也运用其智数假谲以欺人自欺。与之相较,魏晋名士即使是傲慢之类的缺点,却显得天真烂漫,真率自然,一点也不掩饰,这就无须心理提防,给人可爱的感觉。}

\lettrine{27.5} 袁绍\myidx{袁绍}年少时\footnote{袁绍:参本门第1则注。},曾遣人夜以剑掷魏武\myidx{曹操},少下,不箸\footnote{少下不箸:稍微偏下一点没刺中。}。魏武揆之\footnote{揆之:测度,思忖一下。},其后来必高,因帖卧床上\footnote{帖:通“贴”。},剑至果高。{\fzxk\zihao{6}\textcolor{red}{按袁、曹后由鼎跱,迹始携贰。自斯以前,不闻雠隙,有何意故而剚之以剑也?}}

{\cangkai\zihao{5}【评】衡以常人常理,年少之袁、曹,如刘注所称,“不闻仇隙,有何意故而剚之以剑也”?刘孝标理解的“年少”可能指少年,也即十几岁的小男孩。实际上\CJKunderwave{世说}所称“年少”,更多的是指年轻人——也即血气方刚的青年。如本门第1则“魏武少时”,“抽刃劫新妇”者,岂是儿童所为?又如\CJKunderwave{排调}第47则“刘遵祖(爰之)少为殷中军所知,称之于庾公”,此“少”同样是指青年。不然,庾亮面谈后,讥之为“羊公鹤”,对一个小孩,会这样苛刻吗?此“年少”非少年,语序不可颠倒。一个血气方刚的青年,意气用事,相互仇杀,以今证古,并非罕见,这是一。其次,袁、曹二人,禀性奸雄,不可以常人测之。如凌濛初所评:“英雄相忌,不必有隙。”生当天下大乱之日,毫无法制,意存逐鹿之群雄,年轻时即胆大妄为,而在鼎峙之后,为争天下,反倒应该有所收敛才是。另外,愚意“剑掷”非一般人所理解的持剑直刺,而是如今之飞刀,掷以短剑,一剑不中,再飞二剑。年少魏武遇刺,亦属报应。但大难不死,又见其智数过人。}

\lettrine{27.6} 王大将军\myidx{王敦}既为逆\footnote{王大将军:指王敦。参前\CJKunderwave{言语}第20则注。逆:叛逆。},顿军姑熟\footnote{顿:屯,驻扎。姑熟:昔日军事重地,在今安徽省当涂。姑熟,他本或作“姑孰”。}。晋明帝\myidx{司马绍}以英武之才\footnote{晋明帝:东晋第二个皇帝司马绍,元帝子。},犹相猜惮\footnote{猜惮:疑惧。}。乃箸戎服\footnote{戎服:军装。},骑巴賨马\footnote{巴賨(cóng丛)马:巴郡賨人所养的良种马。},赍一金马鞭\footnote{赍(jī基):携带。},阴察军形势\footnote{阴察:暗中观察。}。未至十馀里,有一客姥居店食\footnote{客姥:客店老妪。居店食:袁本作“居店卖食”,是。},帝过愒之\footnote{愒 :通“憩”,休息。},谓姥曰:“王敦举兵图逆,猜害忠良,朝廷骇惧,社稷是忧。故劬劳晨夕,用相觇察\footnote{觇察:窥伺侦察。}。恐形迹危露\footnote{危露:败露。危:败。},或致狼狈,追迫之日,姥其匿之\footnote{匿:掩护,遮掩。}!”便与客姥马鞭而去,行敦营匝而出\footnote{匝:围绕一圈。}。军士觉,曰:“此非常人也!”敦卧心动,曰:“此必黄须鲜卑奴来\footnote{黄须鲜卑奴:明帝母荀氏,燕地鲜卑人,明帝似母,故有鲜卑人的外貌。}!”命骑追之\footnote{骑:骑兵。}。已觉多许里\footnote{觉:通“较”,相差,相去。},追士因问向姥:“不见一黄须人骑马度此邪\footnote{度:经过。}?”姥曰:“去已久矣,不可复及。”于是骑人息意而反\footnote{息意:打消追赶念头。}。{\fzxk\zihao{6}\textcolor{red}{\CJKunderwave{异苑}曰:“帝躬往姑孰,敦时昼寝,卓然惊悟,曰:‘营中有黄头鲜卑奴来,何不缚取!’帝所生母荀氏,燕国人,故貌类焉。”}}

{\cangkai\zihao{5}【评】这是一篇优秀的纪实小说。晋明帝、王敦及客姥三人的形象栩栩如生,心理描绘如画,特别是晋明帝,是一个英俊果敢的年轻帝王。“黄须鲜卑奴”,着戎服、骑巴賨马,亲自深入敌后,侦察敌情,历史上可称千古一帝。王敦卧榻心动,命骑追赶,敢把皇帝拉下马的野心毕露。凌濛初评:“老贼亦灵。”奸雄亦非等闲。客姥似为中间人物,但以智诈之言欺敌骑,则其心中自有是非,同样非常可爱。三者关系,互为因果连环,令人眼花缭乱。明帝与王敦,其心理战能力,也在经受实践的考验。}

\lettrine{27.7} 王右军\myidx{王羲之}年减十岁时\footnote{王右军:王羲之。减:不到,少于。},大将军\myidx{王敦}甚爱之\footnote{大将军:王敦。},恒置帐中眠\footnote{恒:常。}。大将军尝先出,右军犹未起。须臾,钱凤\myidx{钱凤}入\footnote{钱凤(?—324):王敦心腹之一,与沈充同为王敦叛乱谋主,敦败被诛。},屏人论事\footnote{屏:屏退,让人避开。},{\fzxk\zihao{6}\textcolor{red}{\CJKunderwave{晋阳秋}曰:“凤字世仪,吴嘉兴尉子也。奸慝好利,为敦铠曹参军。知敦有不臣心,因进说。后敦败见诛。”}} 都忘右军在帐中\footnote{都忘:全忘记。},便言逆节之谋\footnote{逆节之谋:叛乱阴谋。}。右军觉,既闻所论,知无活理,乃剔吐污头面被褥\footnote{剔吐:以手指探喉令呕吐。剔吐:袁本同,他本或作“阳吐”。朱铸禹\CJKunderwave{汇校集注}云:“然下文云‘吐唾纵横’,似吐者为唾液,非胃中食物。熟睡时流唾液,今人亦有之,羲之为装熟睡,故假作之。”亦可备一说。},诈孰眠\footnote{孰眠:熟睡。孰,通“熟”。}。敦论事造半\footnote{造半:到中途,一半。},方意右军未起\footnote{方意:方才想起。方意,沈校本作“方忆”,亦通。},相与大惊曰\footnote{相与:一起,共同。}:“不得不除之。”及开帐,乃见吐唾从横\footnote{从横:纵横。},信其实孰眠,于是得全。于时称其有智。{\fzxk\zihao{6}\textcolor{red}{案:诸书皆云王允之事,而此言羲之,疑谬。}}

{\cangkai\zihao{5}【评】刘注疑主角非右军,而是王允之事。考右军生年,史有二说,一谓303年,一谓321年。但右军大谢安十几岁,谢安生于320年,则右军不当生于是年之后。故以生于303年为是。据此推算,王敦谋逆,顿兵姑熟的明帝太宁二年(324),右军已是二十二岁的青年,不得称为“年减十岁”。故此事应该以\CJKunderwave{晋书}卷七六\CJKunderwave{王允之传}为是。称右军者,名人效应之讹也。允之字深猷,敦之族子。孩子在全无活理的困境之中,智诈欺老贼以自全,实非易事。后允之还京,以敦、凤谋逆事白父舒,“舒即与导俱启明帝”,朝廷得以早做准备。于此可见,同一琅邪王氏家族,王敦、王含等为叛逆,而王导、王舒父子等则同保国家朝廷,一分为二,态度不同。孩子之功不可没。}

\lettrine{27.8} 陶公\myidx{陶侃}自上流来赴苏峻\myidx{苏峻}之难\footnote{陶公:陶侃,字士衡(行)。时侃任荆州刺史,都督荆、雍、益、梁诸州军事,征西大将军。参前\CJKunderwave{言语}第47则注。 苏峻之难:成帝咸和二年(327),辅政庾亮削苏峻兵权,峻举兵反,攻入京师建康。赴……难:赶去拯救国家于危难之中。},令诛庾公\myidx{庾亮}\footnote{庾公:庾亮。},谓必戮庾,可以谢峻\footnote{谢峻:意谓杀亮谢罪以退峻兵。}。{\fzxk\zihao{6}\textcolor{red}{\CJKunderwave{晋阳秋}曰:“是时成帝在襁褓,太后临朝,中书令庾亮以元舅辅政,欲以风轨格政,绳御四海。而峻拥兵近甸,为逋逃薮。亮图召峻,王导、卞壸并不欲。亮曰:‘苏峻豺狼,终为祸乱。晁错所谓削亦反,不削亦反。’遂下优诏,以大司农征之。峻怒曰:‘庾亮欲诱杀我也!’遂克京邑。平南温峤关乱,号泣登舟,遣参军王愆期推征西陶侃为盟主,俱赴京师。时亮败绩奔峤,人皆尤而少之,峤愈相崇重,分兵以配给之。”}} 庾欲奔窜则不可\footnote{奔窜:奔逃。},欲会恐见执\footnote{会:见面,会晤。},进退无计。温公\myidx{温公}劝庾诣陶\footnote{温公:温峤。},曰:“卿但遥拜,必无他,我为卿保之。”庾从温言诣陶,至便拜,陶自起止之,曰:“庾元规何缘拜陶士衡\footnote{庾元规:庾亮字元规。 陶士衡:陶侃字士衡(一称“士行”)。}?”毕,又降就下坐\footnote{降就下坐:降身坐于末位。},陶又自要起同坐\footnote{要:通“邀”。 同坐:并列而坐。}。定\footnote{定:袁本作“坐定”,是。},庾乃引咎责躬\footnote{引咎责躬:引咎自责。},深相逊谢\footnote{逊谢:谦恭谢罪。},陶不觉释然\footnote{释然:怨怒之气消释顿尽。}。

{\cangkai\zihao{5}【评】此事发生于咸和三年(328)庾亮败奔寻阳时。故事生动,情节曲折,有矛盾,有高潮,戏剧性相当强烈。但更重要的是三个人物形象声吻毕肖,形象如画。加以内在心理描绘细腻,故颇富艺术感染力量。庾亮见陶公,在兵败逃亡之际,而非执政坐朝之时,苏峻之难,亮实为罪首,故“欲奔窜则不可,欲会恐见执”,处于进退维谷的两难之地,内心矛盾形象体现。温峤则从国家大局出发,对陶、庾二人内心均有深刻了解,故劝庾诣陶,以国家前途为重,而不因个人情绪退却。一个正直大臣形象树立了起来。至于陶侃,为国家与民族,捐弃前嫌,接受道歉而怨怒冰释,同心同德而“要起同坐”,正见其胸襟开阔气性豪爽的英雄性格。三人俱见政治家本色,而非使诈之人,入\CJKunderwave{假谲}门欠妥,而更合于\CJKunderwave{雅量}或\CJKunderwave{豪爽}之门。}

\lettrine{27.9} 温公\myidx{温峤}丧妇\footnote{温公:温峤,字太真,太原祁人。东晋中兴名臣。官至骠骑大将军。卒谥忠武。参前\CJKunderwave{言语}第35则注。}。从姑刘氏,家值乱离散\footnote{值乱离散:遭战乱流离失所。},唯有一女,甚有姿慧,姑以属公觅婚\footnote{属:通“嘱”,吩咐,委托。}。公密有自婚意,答云:“佳婿难得,但如峤比\footnote{如……比:像……相似。},云何\footnote{云何:怎样?}?”姑云:“丧败之馀\footnote{丧败之馀:大乱中侥幸存活之人。},乞粗存活\footnote{乞粗存活:马虎活着。},便足慰吾馀年,何敢希汝比!”却后少日\footnote{却:过后。},公报姑云:“已觅得婚处,门地粗可\footnote{粗可:大致可以。},婿身名宦\footnote{名宦:名声地位。},尽不减峤。”因下玉镜台一枚\footnote{下:下聘礼。}。姑大喜。既婚,交礼\footnote{交礼:夫妇成婚交拜之礼。},女以手披纱扇\footnote{披:拂去。},抚掌大笑,曰:“我固疑是老奴\footnote{老奴:老家伙,老东西。},果如所卜。”{\fzxk\zihao{6}\textcolor{red}{按\CJKunderwave{温氏谱},峤初取高平李暅女,中取琅邪王诩女,后取庐江何邃女,都不闻取刘氏,便为虚谬。谷口云:“刘氏,政谓其姑尔,非指其女姓刘也。孝标之注,亦未为得。”}} 玉镜台,是公为刘越石\myidx{刘琨}长史\footnote{刘越石:刘琨字越石,西晋末并州刺史,在北方坚持抗战。温峤作为长史,成为琨之谋主。参前\CJKunderwave{言语}第35则注。},北征刘聦(聪)所得\footnote{刘聪:字玄明,渊子。匈奴族的汉国君主。}。{\fzxk\zihao{6}\textcolor{red}{王隐\CJKunderwave{晋书}曰:“建兴二年,峤为刘琨假守左司马,都督上前锋诸军事,讨刘聪。”\CJKunderwave{晋阳秋}曰:“聪一名载,字玄明,屠各人。父渊,因乱起兵,死,聪嗣业。”}}

{\cangkai\zihao{5}【评】故事发生在东晋初年。在西晋末“五胡乱华”之时,温峤曾在北方刘琨麾下,为拯救祖国而战。作为爱国的志士,他同时是个极重男女情趣的人物。史称峤之品性,讲实际,尚风节,重然诺,而轻礼法不护细行,故常见讥于清议。峤敢于漠视礼法名教,在婚姻感情生活方面,他努力争取自己的幸福。当他认为缘分已到,就紧紧捕捉时机,以“诈”欺姑,智娶表妹。这正是努力把握自我命运的真诚。至于其表妹,以手披开纱扇,而不等新郎揭开头盖;“抚掌”大笑——拍着巴掌大笑起来;一声“老奴”,声调何其亲昵,从内在心理到外在动作,同样是新娘满怀信心去争取新生活而胸有成竹的智慧表现。“抚掌大笑”和一声“老奴”,典型细节何其传神!故事中男女主人公的心理刻画深刻细腻,其狡狯和智慧,声口毕肖而跃然纸上。}

\lettrine{27.10} 诸葛令\myidx{诸葛恢}女\myidx{诸葛文彪},庾氏\myidx{庾会}妇\footnote{诸葛令:诸葛恢,字道明,琅邪阳都人。东晋时官至尚书令,故称。庾氏妇:庾亮长子会娶诸葛文彪。会死于苏峻之乱,故文彪新寡。见\CJKunderwave{方正}第25则注引\CJKunderwave{庾氏谱}。刘注“父虨”为“文彪”之形讹。},既寡,誓云:“不复重出\footnote{重出:再嫁。}。”此女性甚正强\footnote{正强:刚正。},无有登车理\footnote{无有登车理:没有再嫁之理。}。{\fzxk\zihao{6}\textcolor{red}{即庾亮子会妻,父虨(文彪),已见上。}} 恢既许江思玄\myidx{江虨}婚\footnote{江思玄:江虨字思玄。陈留人。统子。官至左仆射、护军将军、领国子祭酒。},乃移家近之。初诳女云:“宜徙于是。”家人一时去\footnote{一时去:一下子都离开。},独留女在后,比其觉\footnote{比:及。},已不复得出。江郎暮来,女哭詈弥甚\footnote{哭詈(lì立):哭骂。},积日渐歇。江虨暝入宿\footnote{暝:晚上。},恒在对床上。后观其意转帖\footnote{帖:帖顺,安定。},虨乃诈厌\footnote{厌:通“魇”,做噩梦。},良久不悟,声气转急。女乃呼婢云:“唤江郎觉。”江于是跃来就之,曰:“我自是天下男子,厌何预卿事而见唤邪\footnote{预:相干。}?既尔相关\footnote{相关:关心。},不得不与人语。”女默然而惭,情义遂笃\footnote{笃:深厚。}。{\fzxk\zihao{6}\textcolor{red}{葛令之清英,江君之茂识,必不背圣人之正典,习蛮夷之秽行。康王之言,所轻多矣。}}

{\cangkai\zihao{5}【评】把寡妇改嫁之事,写得绘声绘色,这是魏晋特色。诸葛门第,堪与琅邪王氏相比,其家教门风甚为方正。文彪受家风影响,婚姻谨守传统礼教,甚至超过乃父:夫死守寡,从一而终。但魏晋名士并不以改嫁为耻,这与后世不同。当诸葛恢与亲家庾亮商量女儿改嫁之事,庾亮通情达理,回信说:“贤女尚小,故其宜也。感念亡儿,若在初没。”(\CJKunderwave{伤逝}第8则)但是好事多磨,作梗的恰是文彪脑中的封建道德流毒。这时,江思玄作为新郎官,浑身充满了青春活力,洋溢着智慧与自信,他以内心真情,向铁石心肠的妙龄女郎射出了一支爱神之箭。文彪终于为真情所感动,投怀送抱,“情义遂笃”,过着幸福的婚姻生活。“我自是天下男子!”正不必有头巾气。面对一个有血有肉的火热新郎,新娘能无动于衷吗?“女默然而惭”,正写出新娘半推半就的娇羞之态。从文学上看,把新婚夫妻圆房时的特殊心态,刻画得活灵活现。故事虽短,却也跌宕起伏,一波三折,扣人心弦。小小狡狯之诈,写出了真情的智慧之花,艺术极其动人。}

\lettrine{27.11} 愍度道人\myidx{支愍道}始欲过江\footnote{愍度道人:即支愍道,东晋高僧,创“心无义”说,著\CJKunderwave{传译经录}。过江:渡过长江到江东诸地。时在成帝之世。},与一伧道人为侣\footnote{伧道人:北方和尚。伧:南方人对北方人的蔑称。},谋曰:“用旧义往江东,恐不办得食\footnote{不办得食:没有饭吃。不办:魏晋常语,不能,无法。}。”便共立“心无义\footnote{心无义:东晋佛教般若学“六家七宗”之一的“心无宗”说,支愍度所创立。一度盛行于江南。几十年后,竺法汰会集名僧如慧远辈破之,此义转衰。}”。既而此道人不成渡。愍度果讲义积年。{\fzxk\zihao{6}\textcolor{red}{\CJKunderwave{名德沙门题目}曰:“支愍度才鉴清出。”孙绰\CJKunderwave{愍度赞}曰:“支度彬彬,好是拔新。俱禀昭见,而能越人。世重秀异,咸竞尔珍。孤桐峄阳,浮磬泗滨。”}} 后有伧人来,先道人寄语云:“为我致意愍度,‘无义’那可立?{\fzxk\zihao{6}\textcolor{red}{旧义者曰:“种智有是而能圆照。然则万累斯尽,谓之空无;常住不变,谓之妙有。”而无义者曰:“种智之体,豁如太虚。虚而能知,无而能应,居宗至极,其唯无乎!”}} 治此计,权救饥尔\footnote{权:姑且。},无为遂负如来也\footnote{无为:不要。如来:佛祖释迦牟尼的法号之一。}!”

{\cangkai\zihao{5}【评】东晋之时,北方中原佛学,与江东佛学,彼此旨趣有异。因此和尚渡江南来,若持旧义传教,将有“不办得食”之虞。其共立“心无义”,虽云“救饥”之术,实亦出自对于南方佛学实际的研究,然后创立新学说,并非纯是以智诈欺人。当日南方佛学与玄学交相影响。支愍度所立“心无义”说,并非全取般若空宗之义,而是如近人陈寅恪\CJKunderwave{支愍度学说考}所说,心无义之旨,与\CJKunderwave{老子}及\CJKunderwave{易经·系辞}之旨相符合,实取外书三玄之说,以释内典之义。此与晋人清谈为精神寄托一样,并非仅是疗饥之诈。而未过江之伧道人所寄语:“无为遂负如来也!”虽云持道甚坚,精神可嘉,但并不了解江南佛学实际。学术应从实际出发,何诈之有?}

\lettrine{27.12} 王文度\myidx{王坦之}弟阿智\myidx{王处之}\footnote{王文度:王坦之字文度。父述。阿智:王处之。字文将,坦之弟。},恶乃不翅\footnote{不翅:同“不啻”,不仅,不止。},当年长而无人与婚。孙兴公\myidx{孙绰}有一女亦僻错\footnote{孙兴公:孙绰字兴公。僻错:性情乖僻。},又无嫁娶理。因诣文度,求见阿智。既见,便阳言\footnote{阳言:假装说。阳:通“佯”。}:“此定可\footnote{定:确定,一定。},殊不如人所传,那得至今未有婚处\footnote{婚处:婚配对象。}!我有一女,乃不恶,但吾寒士\footnote{寒士:相对于高门世族而言,出身于寒微的知识分子自谦之称。},不宜与卿计,欲令阿智娶之。”文度欣然而启蓝田\myidx{王述}云\footnote{蓝田:王述,爵蓝田侯,故称。}:“兴公向来,忽言欲与阿智婚\footnote{向:刚才。}。”蓝田惊喜。既成婚,女之顽嚣\footnote{顽嚣:冥顽嚣张。},欲过阿智\footnote{欲过:要超过。}。方知兴公之诈。{\fzxk\zihao{6}\textcolor{red}{阿智,王处之小字。处之字文将,辟州别驾,不就。太原孙绰女,字阿恒。}}

{\cangkai\zihao{5}【评】王阿智配孙阿恒,不翅恶男娶僻错之女,却也是门当户对,并非虚欺,而是智诈——孙绰运其智数,成就了一对愚痴男女,可称功德无量。可怜天下父心!}

\lettrine{27.13} 范玄平\myidx{范汪}为人好用智数\footnote{范玄平:范汪字玄平,东晋颍阳人。博学多通,喜谈名理,官至徐、兖二州刺史。智数:才智心计。},而有时以多数失会\footnote{多数:过多心计。失会:丢失机会。}。尝失官居东阳,桓大司马\myidx{桓温}在南州\footnote{南州:当时或指荆州、江州及姑熟。此指姑熟。},故往投之\footnote{投:投奔。}。桓时方欲招起屈滞\footnote{屈滞:长期郁抑下位而心怀不满之人。},以倾朝廷,且玄平在京,素亦有誉。桓谓远来投己\footnote{投己:投靠自己。},喜跃非常。比入至庭\footnote{比:及至,等到。},倾身引望\footnote{倾身引望:探身伸长脖子探望。},语笑欢甚。顾谓袁虎\myidx{袁虎}曰\footnote{袁虎:袁宏小字虎,时为桓温记室参军。}:“范公且可作太常卿\footnote{范公:范汪。 太常卿:朝廷九卿之一,掌礼乐祭祀之事。}。”范裁坐\footnote{裁:通“才”。},桓便谢其远来意 。范虽实投桓,而恐以趋时损名\footnote{趋时:迎合时势。},乃曰:“虽怀朝宗\footnote{朝宗:除觐见帝王外,魏晋时礼谒上司也称朝宗。},会有亡儿瘗在此\footnote{瘗(yì意):埋葬。},故来省视。”桓怅然失望,向之虚伫\footnote{虚伫:虚心等待。},一时都尽。{\fzxk\zihao{6}\textcolor{red}{\CJKunderwave{中兴书}曰:“初,桓温请范汪为征西长史,复表为江州,并不就。还都,因求为东阳太守,温甚恨之。汪后为徐州,温北伐,令汪出梁国,失期,温挟憾奏汪为庶人。汪居吴,后至姑熟见温,语其下曰:‘玄平乃来见,当以护军起之。’汪数日辞归,温曰:‘卿适来,何以便去?’汪曰:‘数岁小儿丧,往年经乱,权瘗此境,来迎之,事竟去耳。’温愈怒之,竟不屑意。”}}

{\cangkai\zihao{5}【评】古人仕宦,实是人生大事。士人失官,犹如今之失业,生活都无着落,其痛苦可以想象。范汪企望东山再起而投奔桓温,也可以理解。但其失误,在于好名使诈,“好用智数”,可称是聪明反被聪明误。如刘辰翁所评:“真有如此强口者,\CJKunderwave{世说}虽鄙,然种种备。”深刻揭示了主人公的内心世界奥秘。这是一解。而刘注引\CJKunderwave{中兴书}则为相反的另一解。范汪是个直道而行的节义之人,为温所废多年,屏居乡里,从容讲学而不言枉直之倔强老头,岂有不保晚节而更趋奉迎之理?\CJKunderwave{晋书}汪传称\CJKunderwave{世说}诬之也。二解皆通,但详读汪子\CJKunderwave{范宁传},“终温之世,兄弟无在列位者”,则桓温与范汪仇怨甚深,似以后解为是。}

\lettrine{27.14} 谢遏\myidx{谢遏}年少时\footnote{谢遏:谢玄小字遏,奕子,安侄。参前\CJKunderwave{言语}第78则注。},好箸紫罗香囊\footnote{箸:佩戴。紫罗香囊:装有紫藤香料的香袋,古时作为佩戴饰物。},垂覆手\footnote{覆手:手巾之类。},太傅\myidx{谢安}患之\footnote{太傅:谢安。},而不欲伤其意。乃谲与赌\footnote{谲:假装,欺诈。},得即烧之。{\fzxk\zihao{6}\textcolor{red}{遏,谢玄小字。}}

{\cangkai\zihao{5}【评】陈郡阳夏谢氏家族的谢安,其长兄奕、二哥据早卒,他成老大,就是当然的家长,负有教育与培养谢氏子侄的重任。对于侄子谢玄,他的期望很大,希望把他培养成国家栋梁。但因受当日社会风气的影响,贵游子弟,大多有佩戴香囊和手巾饰物的习惯,谢玄也不能免俗。这令谢安内心不快,男人女性化,缺乏大丈夫的阳刚之气,贪图享受,跟着流行走,就会影响上进心,怎能进一步干大事业?但作为家长,谢安不是痛斥一番,以免伤害孩子的自尊心,而是运用智数之诈,巧妙与赌,然后当面烧却。一“赌”一“烧”的两个连续动作,刺激了孩子,启发了他去追问“为什么”,一旦从道理上明白,就会幡然悔悟,奋发上进。后来谢玄成为名垂青史的历史人物,当与谢安成功的特殊教育有关。只要目的正当,“假谲”也是有益的智慧。}




%%% Local Variables:
%%% mode: latex
%%% TeX-engine: xetex
%%% TeX-master: "../Main"
%%% End:
