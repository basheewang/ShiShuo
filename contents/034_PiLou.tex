%% -*- coding: utf-8 -*-
%% Time-stamp: <Chen Wang: 2025-12-09 22:24:58>

% ○ ◎ ‧ 「 」 『 』 々 ( ) “ ” ■ ^[一-龥]
% 【\([^】][^】][^】]+\)】 → {\\fzxk\\zihao{6}\\textcolor{red}{\1}}
% \(【评】.*\) → {\\cangkai\\zihao{5}\1}
% \(【题解】.*\) → {\\cangkai\\zihao{5}\1}
% 《\([^》]+\)》 → \\CJKunderwave{\1}
% ^\([0-9]+.[0-9]+\) → \\lettrine{\1}
% {\\fzxk\\zihao{6}\\textcolor{red}{[^o]*}}

\setlength{\parindent}{0pt}


\chapter{纰漏第三十四}




{\cangkai\zihao{5}【题解】 纰漏者,纰缪错误与粗疏遗漏也。与前\CJKunderwave{尤悔}门相比,尤悔也属错误罪尤,但多是其大者,如政治、军事、治国方略等大事方面的问题,其错误后果难以挽回,故生悔恨之叹。而纰漏多是生活方面的错失,对人的生活道路的发展,或多或少产生了影响。本门八则故事,专门记载当时士人在日常生活事务中言行举止所出现的过错或疏漏:或说贵族生活之骄奢,或写贵胄名士之失落,或道上层官僚之无知,或绘年轻野心家的内心欲求蠢动。这类生活纰漏,虽然一时并不直接威胁生命,但久而久之,同样后果严重。如虞啸父官拜侍中,皇帝近臣,但却不懂“献替”为何物。读书如此愚鲁,又岂能参议国政?王国宝热衷于荆州刺史的实力地位,见其贪婪野心,又岂能终老病床?纰漏之失虽小于尤悔,但因小见大,也同样形象地描绘了现实社会的方方面面及魏晋名士的内心世界。}

\lettrine{34.1} 王敦\myidx{王敦}初尚主\footnote{初尚主:刚娶公主为妻不久。},{\fzxk\zihao{6}\textcolor{red}{敦尚武帝女舞阳公主,字修祎。}} 如厕,见漆箱盛干枣,本以塞鼻,王谓厕上亦下果\footnote{下果:设果供食。},食遂至尽。既还,婢擎金澡盘盛水,琉璃碗盛澡豆\footnote{琉璃:一种有色半透明的玉石。 澡豆:供洗涤用的豆子。},因倒箸水中而饮之,谓是干饭\footnote{干饭:稠稀饭。}。群婢莫不掩口而笑之。

{\cangkai\zihao{5}【评】故事大约发生在晋武帝太康中期,王敦是二十馀岁的青年。当时武帝刚平吴统一中国不久,到处是一片歌舞升平景象,汰侈豪奢之风迅速弥漫,很快腐蚀了整个朝廷与国家。皇家厕所,讲究享受,竟然如此气派,连琅邪王家贵少王敦也是不知所措,不懂享用,因此不仅被公主视为“乡巴佬”,甚至于“群婢莫不掩口而笑之”。奴才们那轻蔑神色,比高声呵责的侮辱还要令人难受。奴婢敢于瞧不起驸马,仗恃的正是公主皇家势力。当时武帝健在,公主是君,王敦虽出高门,但仍是臣,君臣之间,犹如主与奴的关系,地位并不平等。这是政治婚姻,毫无爱情可言。王敦生活中的一个小小纰漏,却种下了日后复仇的心理。一旦形势逆转,皇家失势,于是当日青州刺史王敦轻车赴京,在兵荒马乱的情况中抛下公主生死而不顾,并把公主百馀婢女尽赐军士糟蹋。如此报复结发妻子及众多婢女,简直丧尽了人性。}

\lettrine{34.2} 元皇\myidx{司马睿}初见贺司空\myidx{贺循}\footnote{元皇:司马睿,东晋开国君主。贺司空:贺循字彦先,会稽山阴人,官至太常卿,卒赠司空,故称。},言及吴时事,问:“孙皓(晧)\myidx{孙晧}烧锯截一贺头\footnote{孙晧:祖权,东吴最后一任君主,后降晋封归义侯。},是谁?”司空未得言,元皇自忆曰:“是贺劭\myidx{贺劭}\footnote{贺劭:字兴伯,吴散骑常侍、中书令。因直言极谏被吴主孙晧所杀。}。”{\fzxk\zihao{6}\textcolor{red}{劭,即循父也。晧凶暴骄矜,劭上书切谏,晧深恨之。亲近惮劭贞正,谮云谤毁国事,被诘责,后还复职。劭中恶风,口不能言语,晧疑劭托疾,收付酒藏,考掠千数,卒无一言。遂杀之。}} 司空㳅(流)涕曰:“臣父遭遇无道,创巨痛深,无以仰答明诏\footnote{仰答明诏:回答提问。仰、明等作敬语。}。”{\fzxk\zihao{6}\textcolor{red}{\CJKunderwave{礼}云:“创巨者其日久,痛深者其愈迟。”}} 元皇愧惭,三日不出。

{\cangkai\zihao{5}【评】故事发生在西晋将亡而东晋未建之际,时元帝任安东将军而见循。元帝时虽未即位,但在王导、周顗、顾荣、贺循等中原士族及江东士族的经营拥戴下,开国江南的形势日趋明显,除此之外,难避胡骑侵逼。故事以东吴亡国之主孙晧的残暴昏聩,与一心向往开基建国的晋元帝作鲜明对比,元帝创业之际,知难而进,知错愧叹,过而能改,宅心仁厚。因小见大,知其开东晋百年基业,并非偶然。于此也见出了国运兴衰之教训。}

\lettrine{34.3} 蔡司徒\myidx{蔡谟}渡江\footnote{蔡司徒:蔡谟,字道明,陈留考城人。官至侍中、司徒,参前\CJKunderwave{方正}第40则注。},见彭蜞\footnote{彭蜞:生长水边,似蟹而小。},大喜曰:“蟹有八足,加以二螯\footnote{“蟹有八足”二句:此蔡邕\CJKunderwave{劝学章}之句取义于\CJKunderwave{荀子·劝学篇}。}。”令烹之。既食,吐下委顿\footnote{吐下:上吐下泻。委顿:困顿,狼狈。},方知非蟹。后向谢仁祖\myidx{谢尚}说此事\footnote{谢仁祖:谢尚字仁祖。},谢曰:“卿读\CJKunderwave{尔雅}不熟\footnote{\CJKunderwave{尔雅}:我国古代第一部分类词典。},几为\CJKunderwave{劝学}死\footnote{\CJKunderwave{劝学}:此指蔡邕\CJKunderwave{劝学章}。邕为谟之从曾祖,故谟熟读其文章。}。”{\fzxk\zihao{6}\textcolor{red}{\CJKunderwave{大戴礼·劝学篇}曰:“蟹二螯八足,非蛇蟺之穴,无所寄托者,用心躁也。”故蔡邕为\CJKunderwave{劝学章},取义焉。\CJKunderwave{尔雅}曰:“螖泽小者劳。”即彭蜞也,似蟹而小。今彭蜞小于蟹而大于彭螖,即\CJKunderwave{尔雅}所谓螖泽也。然此三物,皆八足二螯,而状甚相类。蔡谟不精其小大,食而致弊。故谓读\CJKunderwave{尔雅}不熟也。}}

{\cangkai\zihao{5}【评】读书自有学问。活读书则读书活,死读书则读书死。蔡谟读其祖先蔡邕的\CJKunderwave{劝学章}滚瓜烂熟,冲口而出,但却没有在现实中做实践性的考察,以致分不清蟹与蜞的分别,因此误食彭蜞而食物中毒,为死读书而付出惨痛的代价,故致谢尚之讥,宜矣。}

\lettrine{34.4} 任育长\myidx{任瞻}年少时\footnote{年少:年轻。},甚有令名\footnote{令名:美好声名。}。武帝\myidx{司马炎}崩,选百二十挽郎\footnote{挽郎:牵引灵柩唱挽歌的少年。},一时之秀彦\footnote{秀彦:隽秀杰出之士。},育长亦在其中。王安丰\myidx{王戎}选女婿\footnote{王安丰:王戎,其爵安丰侯,故称。},从挽郎搜其胜者\footnote{搜:寻找。},且择取四人,任犹在其中。童少时,神明可爱\footnote{神明:神情。},时人谓育长影亦好\footnote{影:身影,指外貌。}。自过江,便失志\footnote{失志:恍惚失神的样子。}。王丞相\myidx{王导}请先度时贤共至石头迎之\footnote{石头:城名,在京师建康西。},犹作畴日相待\footnote{畴日:昔日。},一见便觉有异。坐席竟,下饮,便问人云:“此为茶为茗\footnote{茶、茗:六朝时以早采者为茶,晚采者为茗。}?”觉有异色,乃自申明云:“向问饮为热为冷耳\footnote{为热为冷:任瞻因一时未辨而问为茶为茗,出口后知道失当,因以“茗”与“冷”韵母相同来遮盖其误。但“茶”与“热”不近,难以自圆其说。}。”尝行从棺邸下度,流涕悲哀。王丞相闻之曰:“此是有情痴。”{\fzxk\zihao{6}\textcolor{red}{\CJKunderwave{晋百官名}曰:“任瞻字育长,乐安人。父琨,少府卿。瞻历谒者仆射、都尉、天门太守。”}}

{\cangkai\zihao{5}【评】人与环境,关系甚巨。年少任瞻,神明可爱,是在正常的和平年代。但自经中原丧乱,社稷丘墟,一旦过江,就有寄人篱下而失神落魄之感。环境改变了人。其为茶为茗之问,出口即误,虽改为冷为热之辩,仍难掩盖其神情恍惚之态。此所以为纰漏也。才智之失,受环境影响甚大。但不随环境而变者,是其率性自然之“情痴”。经棺邸(棺材店)而落涕悲哀,正见其对人之生死问题的关注与思考,有助于了解魏晋名士的心态。}

\lettrine{34.5} 谢虎子\myidx{谢据}尝上屋熏鼠\footnote{谢虎子:谢据,安之二哥。},{\fzxk\zihao{6}\textcolor{red}{虎子,据小字。据字玄道,尚书裒弟(第)二子。年三十三亡。}} 胡儿\myidx{谢朗}既无由知父为此事\footnote{胡儿:谢朗,据之长子。},闻人道痴人有作此者,戏笑之\footnote{戏笑:讥讽嘲笑。},时道此非复一过\footnote{非复一过:不止一次。}。太傅\myidx{谢安}既了己之不知\footnote{了:知道。己:作第三人称代词用,指谢朗。},因其言次\footnote{言次:言语之间。},语胡儿曰:“世人以此谤中郎\footnote{中郎:指谢据。安长兄奕、次据,安为老三,故称据为中郎。},亦言我共作此。”{\fzxk\zihao{6}\textcolor{red}{中郎,据也,章仲反。按世有兄弟三人,则谓第二者为中。今谢昆弟有六,而以据为中郎,未可解。当由有三时以中为称,因仍不改也。}} 胡儿懊热\footnote{懊热:懊恼。},一月日闭斋不出\footnote{一月日:一个月。 闭斋不出:关门不出。}。太傅虚托引己之过,以相开悟\footnote{开悟:启发感悟。},可谓德教\footnote{德教:以德为教。}。

{\cangkai\zihao{5}【评】谢朗讥笑乃父上屋熏鼠之愚,偶尔一次,原是不知者不为罪。但再三再四讥笑,则可能获不孝罪名,晋时提倡以孝治国,所以叔父谢安必须让侄儿明白。陈郡谢氏过的是大家族的生活,在长兄奕、次兄据去世后,老三安即为当然的家长。他对子侄负有教育培养的责任。但面对子侄的糊涂纰漏,他不是板起家长的面孔,生硬训斥一番,进行道德说教,这样效果可能适得其反,激起孩子的逆反心理。谢安实行的是“虚脱引己之过”的启悟教育,这样更能在感情上与孩子打成一片,这就是最好的“开悟”,其以身作则的“德教”——也即言传身教,颇能引起孩子的深入思考,因为榜样的力量是无穷的。}

\lettrine{34.6} 殷仲堪\myidx{殷仲堪}父\myidx{殷师}病虚悸\footnote{殷仲堪:陈郡人,官至荆州刺史。虚悸:病名,中医以为是因气血亏损而引发的心跳慌乱。},闻床下蚁动,谓是牛斗。{\fzxk\zihao{6}\textcolor{red}{\CJKunderwave{殷氏谱}曰:“殷师字师子(‘师子’,汪藻\CJKunderwave{殷氏谱}作‘子桓’)。祖识、父融,并有名。师至骠骑咨议。生仲堪。”\CJKunderwave{续晋阳秋}曰:“仲堪父曾有失心病,仲堪腰不解带,弥年,父卒。”}} 孝武\myidx{司马曜}不知是殷公,问仲堪:“有一殷病如此不?”仲堪流涕而起曰:“臣进退唯谷\footnote{进退维谷:进退两难的困境。谷,穷也,喻困境。}。”{\fzxk\zihao{6}\textcolor{red}{\CJKunderwave{大雅}诗也。毛公注曰:“谷,穷也。”}}

{\cangkai\zihao{5}【评】孝武所问,正是仲堪之父。仲堪若正面回答,则暴父之疾,是为不孝;若拒绝回答,则是有违诏问,是为不敬,处于两难之地。故引\CJKunderwave{诗经·大雅·桑柔}诗句,做出适当的回答,既典雅又妥帖,显现了仲堪沉挚之痛及其应对急智。}

\lettrine{34.7} 虞啸父\myidx{虞啸父}为孝武\myidx{司马曜}侍中\footnote{虞啸父:光禄大夫虞潭之孙。官至会稽内史。},帝从容问曰:“卿在门下\footnote{门下:门下省,朝廷官署名。掌侍从、顾问之责。},初不闻有所献替\footnote{献替:献可替否,即提建议或直言极谏。}。”虞家富春\footnote{富春:地名,今属浙江。},近海,谓帝望其意气\footnote{意气:指馈献或进奉,亦可指馈献之物。},对曰:“天时尚暖,䱥鱼虾䱹未可致\footnote{䱥鱼虾䱹:䱥,鱼名,可制鱼酱。䱹,鱼虾的腌制品。},寻当有所上献\footnote{寻:不久。当:将。上献:敬奉,进献。}。”帝抚掌大笑。{\fzxk\zihao{6}\textcolor{red}{\CJKunderwave{中兴书}曰:“啸父,会稽人。九(光)禄潭之孙,右将军纯之子。少历显位,与王廞同废为庶人。义旗初,为会稽内史。”}}

{\cangkai\zihao{5}【评】侍中之职,“掌傧赞威仪,大驾出则次直侍中护驾。……备切问近对,拾遗补阙”(\CJKunderwave{晋书·职官志}),是皇帝左右亲近的顾问官吏,非常重要。许多军国大事、政策措施,侍中都参与讨论决策。但虞啸父任侍中,作为孝武帝宠任大臣,却一无献替——没有什么政治思考。其对孝武之问,望文生义直解“有所献替”为上献进贡珍稀物品,一心只想贿赂讨好,实是令人啼笑皆非。士大夫不读书如此,岂能治理国家?东晋孝武之后形势大乱、国祚不长,与大量启用佞人、小人有关。故刘辰翁评曰:“如此谬,子孙之羞也。”}

\lettrine{34.8} 王大\myidx{王忱}丧后\footnote{王大:王忱字元达,小字佛大。故称。坦之第四子。仕至荆州刺史。},朝论或云:“国宝\myidx{王国宝}应作荆州\footnote{国宝:王坦之第三子,忱兄。官至中书令、尚书左仆射。与从弟绪为会稽王司马道子宠任,弄权朝廷,后被杀。作荆州:任荆州刺史。}。”{\fzxk\zihao{6}\textcolor{red}{\CJKunderwave{晋安帝纪}曰:“王忱死,会稽王欲以国宝代之。孝武中诏用仲堪,乃止。”}} 国宝主簿夜函白事云\footnote{主簿:官名。掌公府文书,印鉴等。白事:报告。}:“荆州事已行\footnote{已行:已决定。}。”国宝大喜,其夜开閤唤纲纪\footnote{纲纪:公府政令大都由主簿宣布,故称主簿为纲纪。},话势虽不及作荆州\footnote{话势:谈话趋向。},而意色甚恬\footnote{恬:恬静愉快。}。晓遣参问\footnote{参问:查问,探寻。},都无此事。即唤主簿数之曰\footnote{数:数落,责备。}:“卿何以误人事邪?”

{\cangkai\zihao{5}【评】故事发生于孝武帝太元十七年(392),时荆州刺史王忱死。国宝与忱皆为坦之子,出身太原王氏名门。王坦之一代名流,与谢安齐名当朝,史称其“言不及私,惟忧国家之事”,是个忠心国家朝廷的正人君子。但国宝则反乃父之道而行之,史称其阿谀会稽王道子,弄权乱国,“贪纵聚敛,不知纪极,后房伎妾以百数,天下珍玩充满其室”,是个不折不扣的国之佞人。他一门心思在于争权夺位。当时荆州富庶,军甲占东晋之半,掌控荆州,实具问鼎朝廷之实力。为此,国宝希求荆州刺史之位已久。故朝论“或云”即某人提议,国宝即把“或”当真,以为是必然之事。其对主簿前喜后斥,细致描绘了小人内在心理的微妙变化。故凌濛初评曰:“道意色殊肖。”}








%%% Local Variables:
%%% mode: latex
%%% TeX-engine: xetex
%%% TeX-master: "../Main"
%%% End:
