%% -*- coding: utf-8 -*-
%% Time-stamp: <Chen Wang: 2025-12-07 12:42:03>

% ○ ◎ ‧ 「 」 『 』 々 ( ) “ ” ■ ^[一-龥]
% 【\([^】][^】][^】]+\)】 → {\\fzxk\\zihao{6}\\textcolor{red}{\1}}
% \(【评】.*\) → {\\cangkai\\zihao{5}\1}
% \(【题解】.*\) → {\\cangkai\\zihao{5}\1}
% 《\([^》]+\)》 → \\CJKunderwave{\1}
% ^\([0-9]+.[0-9]+\) → \\lettrine{\1}
% {\\fzxk\\zihao{6}\\textcolor{red}{[^o]*}}



\setlength{\parindent}{0pt}


\chapter{宠礼第二十二}



{\cangkai\zihao{5}【题解】 宠礼,意思是宠爱和礼遇,被宠礼者因故而获得了超越一般的特殊礼遇。至于为什么受宠礼?“因故”二字则大有文章,这就得考究其言外之意了。本篇六则故事,都发生在东晋时代。东晋一朝,社会动荡。承永嘉西晋覆亡之痛,司马南渡,建立江南的偏安小王朝。这时皇族对于中央朝廷的控制力,大大削弱,因而必须借助强大士族的支撑,通过给予权臣以超越礼仪的特殊待遇,来收买人心,以便稳固政权。同样,那些野心勃勃的高门士族,如支撑东晋政权的四大家族之一的桓家(温、玄),同样也在积极“招兵买马”,笼络人才,给予为其所用之士,以不次的超擢与礼遇。古人有云:“士为知己者死。”各级统治者希望通过宠礼这一特殊手段,延揽特殊人才,以便为自己坐江山打天下,多出一把力。“宠礼”不是一般意义上的举贤授能,其政治奥秘,并非为“义”而设,而是专为射“利”而行,究其实质,背后仍然是一种隐蔽的权、钱交易。}

\lettrine{22.1} 元帝\myidx{司马睿}正会\footnote{元帝:晋元帝司马睿,东晋开国皇帝。正会:新年元旦朝会。},引王丞相\myidx{王导}登御床\footnote{王丞相:指王导。御床:皇帝宝座。床,古时亦指坐榻。},王公固辞\footnote{王公:指王导。},中宗引之弥苦\footnote{中宗:元帝死后,庙号中宗。弥苦:更加殷勤。}。王公曰:“使太阳与万物同晖\footnote{太阳与万物:喻皇帝与臣民。使:假使。},臣下何以瞻仰?”{\fzxk\zihao{6}\textcolor{red}{\CJKunderwave{中兴书}曰:“元帝登尊号,百官陪位,诏王导升御坐,固辞然后止。”}}

{\cangkai\zihao{5}【评】元帝以皇族远支,因风云际会,开国江南,骤获大宝,实王导为之主谋。史称“导知天下已乱,遂倾心推奉,潜有兴复之志。帝(按:司马睿当时为琅邪王)雅相器重,契同友执”(\CJKunderwave{晋书·王导传})。由于王导等士族强宗的经营拥戴,司马朝廷才得以偏安南方半壁江山。故时人为之语曰:“王与马,共天下。”此“王”指琅邪王氏。元帝正会,极力拉拢王导共登御床,推其内在心理,真假参半。感激王氏拥戴,表面诚真;而作为君临万民的皇帝,引臣子共登御座,古往今来,绝无此礼。但在大庭广众隆重盛会之际,元帝一反礼制,“引之弥苦”,却是为何?或因王氏家族有大将军敦拥兵在外,随时威胁皇朝政权的生存,故有此举,以试探高门士族琅邪王家的态度如何。若是王敦,可能直登帝座;但王导非敦,他心中明白,无司马则无王氏之尊贵,故谦抑自损,回答得体,明心迹以安帝心,以便为东晋建国的稳固和发展争取生机。在现实面前,王导头脑清醒,并没有因为获宠礼殊荣而昏头转向。}

\lettrine{22.2} 桓宣武\myidx{桓温}尝请参佐入宿\footnote{桓宣武:大将军桓温,卒谥号宣武,故称。参佐:参谋僚属。入宿:住宿府衙。},袁宏\myidx{袁宏}、伏滔\myidx{伏滔}相次而至\footnote{袁宏(328—376):字伯彦,小字虎,陈郡阳夏人。曾任桓温大司马记室参军,以文才著名当世。参前\CJKunderwave{言语}第83则注。伏滔:字玄度,平昌安丘(今属山东)人。以才学桓温聘为参军。参前\CJKunderwave{言语}第72则注。相次:先后。}。涖名\footnote{涖名:核校到者名单。涖,到。},府中复有袁参军\footnote{参军:方面大员或王国诸侯的重要僚佐。},彦伯疑焉,令传教更质\footnote{传教:传达府主教令的小吏。}。传教曰:“参军是袁、伏之袁\footnote{袁、伏:时袁宏与伏滔齐名,故云。},复何所疑?”

{\cangkai\zihao{5}【评】这则故事虽称“袁、伏”,但第一主角应是被宠礼的对象袁宏,第二主角是闻其声气而并未出场的主人桓温,伏滔不过是连带提起而已。袁宏其人,文史全才,著名当世,其才思敏捷,桓温、陶胡奴、谢安屡试不爽,令人赞叹不已。当时的伏滔,和他同事桓温,一样受桓温宠礼,出则同游。但是,文人相轻,自古而然,袁宏自视一代文宗,岂能容忍他人与己并列齐名?尽管伏滔和他是关系很好的同事,但他认为受宠礼的程度不该一样。史称:“与伏滔同在温府,府中呼‘袁伏’。宏心耻之,每叹曰:‘公之厚恩未优国士,而与滔比肩,何辱之甚。’”(\CJKunderwave{晋书·文苑·袁宏传})其心胸狭隘如此。因此,当他误会还有一个袁参军时,怎能不问个明白呢?“彦伯疑焉,令传教更质”,重在内心刻画。传教的回答,实是府主桓温之言,同样妙语解人,正是针对袁宏的心理世界而发,可谓一语破的。}

\lettrine{22.3} 王珣\myidx{王珣}、郗超\myidx{郗超}并有奇才\footnote{王珣:字元琳。王导孙,洽子。参前\CJKunderwave{言语}第102则注。郗超:字景兴,一字嘉宾。参前\CJKunderwave{言语}第59则注。},为大司马\myidx{桓温}所眷拔\footnote{大司马:指桓温。眷拔:眷顾超拔。},珣为主簿\footnote{主簿:政府衙门属官,掌管簿籍、印鉴等,位居掾属之首。},超为记室参军\footnote{记室参军:相当于今之行政秘书,掌管机要。}。超为人多髯\footnote{为人:相貌。髯:两颊胡须。},珣行状短小,于时荆州为之语\footnote{于时:当时。荆州:长江中上游重镇,治所江陵(今属湖北)。当时桓温代庾翼任荆州刺史,节镇西藩。语:指民谣。}曰:“髯参军,短主簿,能令公喜,能令公怒。”{\fzxk\zihao{6}\textcolor{red}{\CJKunderwave{续晋阳秋}曰:“超有才能,珣有器望,并为温所昵。”}}

{\cangkai\zihao{5}【评】桓温是个集雄心与野心于一身的人物。为了政治需要,他注意赏拔人才。当时谢安、王珣、郗超、袁宏、习凿齿等一代英才,并集温府,兼文武斌斌之盛。故事通过民谣,刻画了王珣、郗超的艺术形象,“多髯”与“短小”状其貌,这是写实;“能令公喜,能令公怒”,则属写虚,写尽二人神气与才干。虚实结合,以貌传神,极其生动。但同为才干之士,因政治倾向不同,后来分化,郗超成为桓温立威朝廷的谋主,而王珣则逐渐转向了维护司马朝廷的利益。桓温之“眷拔”,是在动态发展的历史实际中展现的,其所爱恶,关键在于能否为我所用,政治权势之利,是其衡量标准。又,据朱铸禹\CJKunderwave{汇校集注},谓史称“珣弱冠从温辟,温已移镇姑孰,不在荆州”,因疑故事发生地点不在荆州。可备一说。}

\lettrine{22.4} 许玄度\myidx{许询}停都一月\footnote{许玄度:许询,字玄度。东晋玄学清谈名士。参前\CJKunderwave{言语}第69则注。都:京都建康。},刘尹\myidx{刘惔}无日不往\footnote{刘尹:指刘惔,字真长。东晋名士。参前\CJKunderwave{德行}第35则注。},乃叹曰:“卿复少时不去\footnote{少时:不长时日。},我成轻薄京尹\footnote{轻薄:轻浮。京尹:京兆尹,京师地区行政长官。时刘惔任丹阳尹,故称。}。”{\fzxk\zihao{6}\textcolor{red}{\CJKunderwave{语林}曰:“玄度出都,真长九日十一诣之,曰:‘卿尚不去,使我成薄德二千石。’”}}

{\cangkai\zihao{5}【评】许询是东晋玄学清谈名士,又是颇有才藻而擅长诗文的文学家,其玄言诗与孙绰齐名。他虽布衣之士,但却经常结交贵族名流,如谢安、刘惔、王濛等。许询是“民”,但他并不高视仰望做官的人;刘惔是官,却也并不卑视布衣之士。在官本位的封建社会中,能够官、“民”平等,自由交往,毫无心理障碍,这只有在魏晋这样特殊的年代中,才得以实现。因爱才而惺惺相惜,追求心灵自由,这是魏晋士人的正常心态。刘惔为见许询,差点荒废公务,为的是求得“洗尽尘滓,独存孤迥”的自由洒脱而超越世俗的精神共鸣。故凌濛初评曰:“得刘尹如此,甚难,甚难!”于此可见刘惔的真诚与可爱。}

\lettrine{22.5} 孝武\myidx{司马曜}在西堂会\footnote{孝武:指东晋孝武帝司马曜。西堂:厅堂名,在太极殿西侧。},伏滔\myidx{伏滔}预坐\footnote{伏滔:字玄度,平昌安丘人。参前\CJKunderwave{言语}第72则注。预坐:在坐。}。还,下车呼其儿,{\fzxk\zihao{6}\textcolor{red}{儿即系也。丘渊之\CJKunderwave{文章录}曰:“系字敬鲁,仕至光禄大夫。”}} 语之曰:“百人高会\footnote{高会:盛大集会。},临坐未得他语,先问:‘伏滔何在?在此不\footnote{不:通“否”。}?’此故未易得\footnote{故:实在,的确。}。为人作父如此\footnote{为人:在社会上做人。作父:在家中做父亲。},何如?”

{\cangkai\zihao{5}【评】封建知识分子中的某些人,一旦得宠得势,也常会翘尾巴,自以为附青云而扶摇直上,因而得意忘形。伏滔一旦被皇帝动问,浑身骨头轻飘飘,回家立即向自家儿女夸耀,一副小人得志之态,活灵活现。忘记了自己是附皮之毛,一旦时过境迁,皮之不存,毛将焉附!前面有袁宏以“袁、伏”并称为耻,人们不太理解,读了这一故事,方才恍然大悟,伏滔器小,该当如此。伏滔“为人作父如此,何如”一语,李贽评曰:“十分像。”栩栩如生的“这一个”,确非伏滔莫属,以此见作者之笔墨高妙。}

\lettrine{22.6} 卞范之\myidx{卞鞠}为丹阳尹\footnote{卞范之:字敬祖,又字鞠,济阴冤句(今山东菏泽西南)人。桓玄心腹。参前\CJKunderwave{伤逝}第19则注。}。羊孚\myidx{羊孚}南州暂还\footnote{羊孚:字子道,泰山人。绥子。桓玄心腹。参前\CJKunderwave{言语}第104则注。南州:京师南边州郡,此指姑孰。},往卞许\footnote{卞许:卞范之家。许,住处,处所。},云:“下官疾动\footnote{疾动:病发作。下官:做官者自谦之词。},不堪坐。”卞便开帐拂褥,羊径上大床,入被须枕\footnote{须枕:依靠枕头。}。卞回坐倾睐\footnote{回坐:扭身而坐。倾睐:倾心看视,朱铸禹\CJKunderwave{汇校集注}云:“倾心睐视,护持之,通夜中旦,未尝懈也。”},移晨达暮\footnote{移晨达暮:从早到晚。形容整天亲自护持。}。羊去,卞语曰:“我以第一理期卿\footnote{第一理:至关重要的事理。期:期望。},卿莫负我!”{\fzxk\zihao{6}\textcolor{red}{丘渊之\CJKunderwave{文章录}曰:“范之字敬祖,济阴冤句人。祖㟪,下邳太守。父循,尚书郎。相玄辅政,范之迁丹阳尹。玄败,伏诛。”}}

{\cangkai\zihao{5}【评】卞范之和羊孚,都是桓玄的心腹。羊病,径上卞床而拥被倚枕,卞不以为忤,反而“回坐倾睐,移晨达暮”而细心护持,这不仅有个人感情的关系,又为共同的政治利益所驱动。当时桓玄藉世资而雄踞荆楚,属晋失政,遂生觊觎之望,篡逆之形渐显。卞范之所称“我以第一理期卿,卿莫负我”,当即指桓玄篡逆之事。后羊孚早卒,幸免叛逆之罪。而卞范之则以附逆被诛。生命是空间性的存在,但更是时间性的“艺术”。卞范之被时间钉在了历史的耻辱柱上;而羊孚同谋,则因早亡而逃脱了历史的惩罚,何其侥幸!}






%%% Local Variables:
%%% mode: latex
%%% TeX-engine: xetex
%%% TeX-master: "../Main"
%%% End:
