%% -*- coding: utf-8 -*-
%% Time-stamp: <Chen Wang: 2025-12-06 12:09:24>

% ○ ◎ ‧ 「 」 『 』 々 ( ) “ ” ■ ^[一-龥]
% 【\([^】][^】][^】]+\)】 → {\\fzxk\\zihao{6}\\textcolor{red}{\1}}
% \(【评】.*\) → {\\cangkai\\zihao{5}\1}
% \(【题解】.*\) → {\\cangkai\\zihao{5}\1}
% 《\([^》]+\)》 → \\CJKunderwave{\1}
% ^\([0-9]+.[0-9]+\) → \\lettrine{\1}
% {\\fzxk\\zihao{6}\\textcolor{red}{[^o]*}}


\setlength{\parindent}{0pt}



\chapter{夙惠第十二}



{\cangkai\zihao{5}【题解】 夙惠,唐写本作“夙慧”,“惠”与“慧”通借。夙惠者,早慧也。幼童聪慧过人,属于天才神童的范围。由于乱世需要治国人才,因而有关“才”与“天才”的问题,在魏晋受到特别的重视。后汉三国的清议之风,至魏晋化为品目之评,对人之才性的探索,成为时代探索的重点课题。如魏朝正始年间的“四本论”,重在讨论才与性的异同,当时尚书傅嘏论才性同,中书令李丰论才性异,侍郎锺会论才性合,屯骑校尉王广论才性离。讨论的问题很多,视角各异,但都离不开对于人本身天才问题的研究。在天地人“三才”之中,“人”居中成为核心,探讨才性关系及天才问题,本身就是把研究的兴趣和重心安放在人的身上。而人类的成长,受自然和社会的影响,在漫长的历史实践中不断积累变化,日新月异,经由量变到质变的循环往复和不断飞跃,而与世界文明同步发展。每一次飞跃,都让人感到天才的存在。由于人的遗传基因、生理机体和心理机能等自然资质不同,的确会有天才的早慧儿童存在,发现神童,并在社会实践中加以培养,就会培养出合乎时代需要的栋梁之材。}

\lettrine{12.1} 宾客诣陈太丘\myidx{陈寔}宿\footnote{诣:到……去。陈太丘:陈寔,字仲弓,颍川人。曾任太丘长,故称。参《德行》第6则注。},太丘使元方\myidx{陈纪}、季方\myidx{陈谌}炊\footnote{元方:陈纪字元方,寔长子。季方:陈谌,字季方,寔少子。按,陈寔父子三人俱称后汉名士。}。客与太丘论议\footnote{论议:讨论问题。},二人进火\footnote{进火:烧火做饭。},俱委而窃听\footnote{委:丢弃。},炊忘箸箅\footnote{箅:蒸饭器具。竹箅,使米不漏锅中。},饭落釜中\footnote{釜:锅。}。太丘问:“炊何不馏\footnote{馏:蒸饭。}?”元方、季方长跪曰\footnote{长跪:直身跪地,以示敬重。}:“大人与客语,乃俱窃听,炊忘箸箅,饭今成糜\footnote{糜:粥。}。”太丘曰:“尔颇有所识不\footnote{识:记忆。}?”对曰:“仿佛志之。”二子俱说,更相易夺\footnote{易夺:补充修正。},言无遗失。太丘曰:“如此,但糜自可\footnote{但:只。},何必饭也!”

{\cangkai\zihao{5}【评】这是一篇优秀的古代儿童小说。故事生动风趣,情节跌宕起伏,有开头、发展和结尾,一波三折,耐人寻味。人物形象生动,心理刻画惟妙惟肖。孩子忘记做饭,明知犯错,本应受严父呵责,所以通过“长跪”不起一个生动细节描写,把儿童惧怕父亲责骂的战栗心理如画托出。但可贵的是,陈寔非但不骂,而是先问明原因,知因偷听论议,好学深思而出差错,就启发孩子:“尔颇有所识不?”这一问句,犹如一个优秀教师的善于启发教育,从而引出了下面“二子俱说,更相易夺,言无遗失”的生动一幕,把故事矛盾推向了高潮,引向了合理的喜剧结束。“二子俱说”,把儿童受到鼓励后的欢欣雀跃,争先恐后抢着说话的天真活泼神态,描绘得栩栩如生。而兄弟的“更相易夺”,有人译为“互相补充”,不太准确,你“易”我“夺”,正是汉人论议中激烈论难场面的克隆。小孩喜欢思考和争辩,不是天才又是什么?陈寔与孩子,一是善于启发,一是及时抓住时机,好学不倦,善于学习,成为古代士人家学中教学相长的佳话,颇有借鉴价值。}

\lettrine{12.2} 何晏\myidx{何晏}七岁\footnote{何晏:字平叔。官至尚书。与王弼同是正始玄学领袖,参\CJKunderwave{德行}第14则注。},明惠若神\footnote{惠:通“慧”。},魏武\myidx{曹操}奇爱之\footnote{魏武:曹操。},因晏在宫内\footnote{宫:魏宫,实指曹操府邸。},欲以为子。晏乃画地令方,自处其中。人问其故,答曰:“何氏之庐也\footnote{庐:房屋,喻指家。}。”魏武知之,即遣还。{\fzxk\zihao{6}\textcolor{red}{\CJKunderwave{魏略}曰:“晏父蚤亡,太祖为司空时,纳晏母,其时秦宜禄阿鳔亦随母在宫,并宠如子。常谓晏为假子也。”}}

{\cangkai\zihao{5}【评】何晏是汉末大将军何进之孙。进谋诛宦官被杀,晏父早死,曹操纳晏母尹氏为夫人,故晏幼随母养于魏宫。与晏情况相似的还有秦宜禄子朗,小名阿苏,朗母杜氏也被曹操纳为夫人。晏与朗并为曹操养子,但二人性格不同,命运各异。史称朗“性谨慎,而晏无所顾惮”。读此故事,可见一斑。晏少时,在曹操全盛时代,但晏却画地令方,自谓“何氏之庐也”。这不是强调何氏家族的地位,而是不愿阿谀曹氏,热衷独立思考,强调孩子心中的自我。这是对于曹操诸子憎恶歧视态度的心理反弹,于此见其性格之倔强。\CJKunderwave{太平御览}卷三九三引\CJKunderwave{何晏别传},曹操命晏与其诸子长幼相次,晏不从,“坐则专席,止则独止”。人问其故,答曰:“礼,异族不相贯坐位。”引经据典,见其早慧,与画地为庐言行同一性质。后晏虽贵为驸马,但在魏文、明二帝之朝,无所任事,或为冗官,备受政治压抑,与其小时强调自我之尊严有关。但在统治者眼里,人的尊严不值一文。悲乎!}

\lettrine{12.3} 晋明帝\myidx{司马绍}数岁\footnote{晋明帝:司马绍,元帝长子,东晋第二主。},坐元帝\myidx{司马睿}膝上\footnote{元帝:司马睿,东晋开国之君。}。有人从长安来,元帝问洛下消息\footnote{洛下:洛阳,西晋京师。},潸然流涕\footnote{潸然:伤心流泪的样子。}。明帝问何以致泣?具以东渡意告之\footnote{东渡:西晋永嘉之乱,中原沦丧,司马王室渡江立脚江东建康。}。因问明帝:“汝意谓长安何如日远?”答曰:“日远。不闻人从日边来\footnote{不闻人从日边来:据\CJKunderwave{太平御览}引刘昭\CJKunderwave{幼童传},此上有“只闻人从长安来”句,语义更为完整。},居然可知\footnote{居然可知:显而易见。}。”元帝异之\footnote{异:惊奇,诧异。}。明日集群臣宴会,告以此意,更重问之。乃答曰:“日近。”元帝失色,曰:“尔何故异昨日之言邪?”答曰:“举目见日,不见长安\footnote{日:喻元帝。长安:喻故国。}。”

{\cangkai\zihao{5}【评】故事发生在永嘉初年(307),元帝任安东将军,用王导计,移镇建业(建康),时明帝九岁。当时胡骑蹂躏中原,京师洛阳危急,故元帝“潸然流涕”。元、明二帝父子问答,很有意思。元帝两次所问,同一问题,考儿智力如何,要求据实回答。但明帝两次回话,第一次因是父子间私人问答,孩子根据经验,给以合乎实际的回答。长安未见,不知远近,但“有人从长安来”,行有时日,可以道里计;而日虽抬头可见,又岂可道里计乎?“不闻人从日边来”,其“日远”之判断,符合科学之推论。因而“元帝异之”,欲群臣集宴之时,让儿子再次表演,以便夸示群臣。但第二次时,面对同一问题,孩子却做出了与前相反的“日近”判断,令元帝“失色”而加以责问。但明帝“举目见日,不见长安”的回答,以日喻元帝,以长安譬故国。不见长安,兴故国沦丧之悲;但抬头见日,国家仍有新生希望。因是君臣集会场合,所以孩子转换视角,以修辞比喻来进行现实思考,合乎环境实际,其聪慧非常人能及,此所以称夙慧。}

\lettrine{12.4} 司空顾和\myidx{顾和}与时贤共清言\footnote{司空顾和:顾和字君孝,顾荣族子。官至尚书令,卒赠司空。清言:清谈玄理。}。张玄之\myidx{张玄}、顾敷\myidx{顾敷}是中外孙\footnote{张玄之:张玄,又名玄之,字祖希。官至冠军将军,会稽内史。顾敷:字祖根。见刘注。},年并七岁,{\fzxk\zihao{6}\textcolor{red}{\CJKunderwave{顾恺之家传}曰:“敷字祖根,吴郡吴人。滔然有大成之量,仕至箸作郎,二十三卒。”}} 在床边戏,于时闻语,神情如不相属\footnote{不相属:不相关。属:贯注,注意。}。瞑于灯下\footnote{瞑:通“暝”,晚上。},二儿共叙客主之言,都无遗失。顾公越席而提其耳曰\footnote{越席:离席。}:“不意衰宗,复生此宝\footnote{不意:没想到。宝:宝贝子孙。}!”

{\cangkai\zihao{5}【评】此则应与\CJKunderwave{言语}第51则故事并读同悟。此谓张、顾二人“年并七岁”,但\CJKunderwave{言语}51则谓“张年九岁,顾年七岁”,二说未知孰是。但二则并谓张、顾二人“少而聪惠(慧)”,则无不同。清谈玄理,在魏晋士林中属高雅之事,已成为当时名士的一种自觉精神追求。能否清谈,成为当时评价士人内在文化素养及其精神风度优劣的一个重要方面。魏晋士人讲究生活质量,重视精神生活,清谈玄理就是一种重要的表现。清谈是一种探索“理源所归”的理论思辨,其中必有主客双方交锋的论难场面。这对儿童来说,过于精深而枯燥。但年仅七岁的张、顾二童不同,他们“共叙客主之言,都无遗失”。如果孩子不是对于理论思辨俱有特殊的敏感和兴趣,怎么可能做到复述“都无遗失”呢?这可能与其天然资质及家学家风的耳濡目染有关。神童的诞生,与优良家风及其教学环境密切相关。}

\lettrine{12.5} 韩康伯\myidx{韩伯}数岁\footnote{韩康伯:韩伯字康伯。官丹阳尹、吏部尚书。东晋玄学名家,曾注\CJKunderwave{易传}以续王弼\CJKunderwave{周易注}。},家酷贫,至大寒\footnote{大寒:我国日历二十四节气之一,全年最冷时节。},止得襦\footnote{襦:短袄。}。母殷夫人自成之\footnote{殷夫人:韩伯之母为殷羡女,殷浩姐妹。},令康伯捉熨斗\footnote{捉:握,合手。},谓康伯曰:“且箸襦,寻作複裈\footnote{複裈(kūn昆):夹裤。}。”乃云\footnote{乃云:袁本作“儿云”,是。}:“已足,不须複裈也。”母问其故,答曰:“火在熨斗中而柄热。今既箸襦,下亦当暖,故不须耳。”母甚异之,知为国器\footnote{国器:治国栋梁。}。

{\cangkai\zihao{5}【评】吃饱穿暖,是人生的基本生理需要。嗷嗷待哺的孩子更是如此。但是,世上并非人人都可以吃饱穿暖。康伯一介寒士,年少时“家酷贫”,连穿暖御寒也做不到。但是,穷人的孩子早当家。康伯年仅数岁,却很能体会母亲的苦衷。着襦已属不易,更何况是再做夹裤呢?如果让母亲再做“複裈”,又不知要怎样艰难经营耗费心血呢!所以幼童以熨斗柄热作答,劝慰老母说:“今既箸襦,下当亦暖,故不须耳。”其善于逻辑推理,令人惊异。但更重要的是小时能体谅父母困难,主动在困境中奋斗;长大之后,必能为国家分忧解难,故其母预知其日后必为国器。母子心心相印,传为文坛佳话。}

\lettrine{12.6} 晋孝武\myidx{司马曜}年十二\footnote{晋孝武:孝武帝司马曜,简文帝子。},时冬天,昼日不箸複衣\footnote{複衣:可置絮之夹袄。},但箸单练衫五六重\footnote{单练衫:单衣。练,熟绢。},夜则累茵褥\footnote{累茵褥:几重被褥。}。谢公\myidx{谢安}谏曰\footnote{谢公:谢安。}:“圣体宜令有常\footnote{有常:有规律。},陛下昼过冷,夜过热,恐非摄养之术\footnote{摄养之术:养生之道。}。”帝曰:“昼动夜静。”{\fzxk\zihao{6}\textcolor{red}{\CJKunderwave{老子}曰:“躁胜寒,静胜暑。”此言夜静寒,宜重肃也。}} 谢公出,叹曰:“上理不减先帝\footnote{理:思理。先帝:简文帝司马昱。}。”{\fzxk\zihao{6}\textcolor{red}{简文帝喜言理也。}}

{\cangkai\zihao{5}【评】故事发生在孝武帝登基三载的宁康三年(375)。时孝武虽是一国之君,但年仅十二,仍然是个孩子。不过,史称孝武“幼称聪悟”。故事即是一例。其“昼动夜静”四字,言简意赅,说理深微。人在白天活动,血液通畅,产生热量,自具御寒功能;夜晚静息,血气收敛,难抗严寒,故须借助厚实被褥来保温御寒。同一个人,在不同时期,衣着被褥应依据实际而变化。而谢安执一之“常”,却是教条式的被动护身,而非真正的“摄养之术”。一代名士的认识,反而不及一个年仅十二的孩子。孝武夙慧,令人叹赏。可惜小时了了,大未必佳。后来孝武溺于酒色,荒废朝政,英年暴崩,正是对于早年聪悟的讽刺。}

\lettrine{12.7} 桓宣武\myidx{桓温}薨\footnote{桓宣武薨:桓温死。},桓南郡\myidx{桓玄}年五岁\footnote{桓南郡:桓玄,袭封南郡公,故称。},服始除\footnote{服始除:刚脱去丧服。},桓车骑\myidx{桓冲}与送故文武别\footnote{桓车骑:桓冲字幼子。(宋本刘注“字”下漏“幼子”二字。玄叔,指桓玄叔父。)曾任车骑将军,故称。冲为温之幼弟。送故:魏晋时州郡长官离任、升迁或亡故,僚佐为之送行或送丧的礼仪活动。},{\fzxk\zihao{6}\textcolor{red}{\CJKunderwave{桓冲别传}曰:“冲字玄叔,温弟也。累迁车骑将军,都督七州诸军事。”}} 因指语南郡:“此皆汝家故吏佐\footnote{吏佐:官员僚佐。}。”玄应声恸哭,酸感傍人。车骑每自目己坐曰:“灵宝\myidx{桓玄}成人,当以此坐还之\footnote{灵宝:桓玄小字。坐:指官位。按,温死后,冲代温居任,诏冲为中军将军、扬豫二州刺史、都督扬江豫三州军事。}。”{\fzxk\zihao{6}\textcolor{red}{灵宝,玄小字也。}} 鞠爱过于所生\footnote{鞠爱:抚育爱护。}。

{\cangkai\zihao{5}【评】桓温卒于孝武帝宁康元年,时桓玄五岁,但服丧三年,服除,当是宁康三年(375)。故\CJKunderwave{晋书}玄传称:“年七岁,温服终,府州文武辞其叔父冲,冲抚玄头曰:‘此汝家之故吏也。’玄因涕泪覆面,众并异之。”文字与\CJKunderwave{世说}稍异。据礼制,玄“年七岁”为是。较五岁幼童,七岁之玄,悲从中来,恸哭流涕。这不仅为丧失严父而泣,同时也为自己失去保护而啼。其声酸楚感人,故其叔冲“鞠爱过于所生”。不过,此则故事与“夙惠”关涉不大,改入\CJKunderwave{伤逝}门似更贴切。}





%%% Local Variables:
%%% mode: latex
%%% TeX-engine: xetex
%%% TeX-master: "../Main"
%%% End:
