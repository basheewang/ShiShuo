%% -*- coding: utf-8 -*-
%% Time-stamp: <Chen Wang: 2025-12-06 21:23:35>

% ○ ◎ ‧ 「 」 『 』 々 ( ) “ ” ■ ^[一-龥]
% 【\([^】][^】][^】]+\)】 → {\\fzxk\\zihao{6}\\textcolor{red}{\1}}
% \(【评】.*\) → {\\cangkai\\zihao{5}\1}
% \(【题解】.*\) → {\\cangkai\\zihao{5}\1}
% 《\([^》]+\)》 → \\CJKunderwave{\1}
% ^\([0-9]+.[0-9]+\) → \\lettrine{\1}
% {\\fzxk\\zihao{6}\\textcolor{red}{[^o]*}}


\setlength{\parindent}{0pt}


\chapter{栖逸第十八}



{\cangkai\zihao{5}【题解】 栖逸者,幽栖山林隐遁放逸也。\CJKunderwave{栖逸}共17则故事,描写的是魏晋士人遁迹于山水自然而远离人世喧嚣的隐居生活。在中国进入文明社会以来,世代都有隐士,传说中尧舜时代有巢父、许由,孔子时代有楚狂接舆,无论时代之治乱盛衰,都有隐士的存在。因此,讲述隐士的故事,反映其隐逸思想,就成为传统道德和政治理想的一种特殊形式的必要补充。隐士现象及其隐逸思想,是我国古代社会中所特有的一种文化现象和独特的精神存在。在表面上,它与传统的统治思想文化有矛盾和冲突,但其内在的骨子里,却为传统文化与古代政治留下了一道意味深长的醒目投影。隐逸思想,它是古代士人的人格操守、价值追求、智慧水平、人生感悟以及审美体验的复杂心灵交织的结晶。\CJKunderwave{周易}中有\CJKunderwave{遁}卦,专谈隐遁之事。“不事王侯,高尚其事”(\CJKunderwave{蛊}卦上九爻辞),“大过,君子以独立不惧,遁世无闷”(\CJKunderwave{大过}卦象辞)。在时不我利的情况下,远遁山林,保持自己的清醒头脑和独立人格,而不愿成为权势金钱的奴才和牺牲品,这是另一种有效的卫“道”措施。当然,魏晋处于动荡乱世,士夫之命,朝不保夕,因而隐逸思想乘势大兴并具有玄学时代的新特点,也是水到渠成之事。}

\lettrine{18.1} 阮步兵\myidx{阮籍}啸\footnote{阮步兵:即阮籍,籍曾任步兵校尉。啸:以口哨作歌。啸歌之时,撮口吹气以发声。据晋成公绥\CJKunderwave{啸赋},谓啸吹之时,“声不假器,用不借物,近取诸身,役心御气,动唇有曲,发口成音,触类感物,因歌随吟”。},闻数百步。苏门山中\footnote{苏门山:亦称百门山或苏岭,太行山之脉,在今河南辉县西北。},忽有真人\footnote{真人:道家或道教徒所谓炼身养性的得道高士。},樵伐者咸共传说\footnote{樵伐者:砍柴的樵夫。咸:皆,都。}。阮籍往观,见其人拥膝岩侧。籍登岭就之,箕踞相对\footnote{箕踞:大开两脚,屈膝而坐,其形似簸箕。这在古代是一种表示傲慢无礼或随意自在的一种坐姿。据其情境,阮籍当是后一种意思。}。籍商略终古\footnote{商略:商讨、评论。终古:往古历史。},上陈黄、农玄寂之道\footnote{黄、农:指传说中的黄帝及神农氏。道家视此二氏为无为而治的典范。玄寂之道:指道家的清静无为以顺其自然之道。魏晋玄学家奉为至理。},下考三代盛德之美\footnote{三代:指上古的夏、商、周三代。按:有关夏代的存在与否,今之考古界仍在讨论中。},以问之,仡然不应\footnote{仡(yì译)然不应:昂着头不予理睬。}。复叙有为之外\footnote{有为之外:“外”,诸本作“教”,是。有为之教,指儒家积极入世的有为主张。},栖神导气之术以观之\footnote{栖神导气之术:道家的修身养心方法,后世发展为内气功。},彼犹如前,凝瞩不转\footnote{凝瞩不转:凝神不动。瞩:视。}。籍因对之长啸。良久,乃笑曰:“可更作。”籍复啸。意尽,退还半岭许\footnote{半岭许:半山腰的样子。许:处所。},闻上{\HanaMinB 𪡧}然有声\footnote{{\HanaMinB 𪡧}(qiú求)然:拟声词,声音悠长的样子。},如数部鼓吹\footnote{鼓吹:以打击乐器鼓、钲及吹鼓乐器箫、笳为主的乐队演奏。原是军中之乐,乐歌雄壮,声势浩大。},林谷传响\footnote{林谷传响:山林溪谷的回声。}。顾看\footnote{顾:回顾,回头。},乃向人啸也\footnote{向人:刚才见面之人。}。{\fzxk\zihao{6}\textcolor{red}{\CJKunderwave{魏氏春秋}曰:“阮籍常率意独驾,不由径路,车迹所穷,辄恸哭而反。尝游苏门山,有隐者莫知姓名,有竹实数斛,杵臼而已。籍闻而从之,谈太古无为之道,论五帝、三皇之义,苏门先生翛然曾不眄之。籍乃嘐然长啸,韵响寥亮。苏门先生乃逌尔而笑。籍既降,先生喟然高啸,有如凤音。籍素知音,乃假苏门先生之论以寄所怀。其歌曰:‘日没不周西,月出丹渊中。阳精蔽不见,阴光代为雄。亭亭在须臾,厌厌将复隆。富贵俯仰间,贫贱何必终。’”\CJKunderwave{竹林七贤论}曰:“籍归,遂箸\CJKunderwave{大人先生论},所言皆胸怀间本趣,大意谓先生与己不异也。观其长啸相和,亦近乎目击道存矣。”}}

{\cangkai\zihao{5}【评】故事中着墨最多的阮籍,史上声名颇佳,是魏晋时代玄学家和文学家中首屈一指的领军人物。但在这里,只能是屈居第二主角,而第一主角的桂冠,却不能不让与那名不见经传的苏门山得道真人。面对苏门山真人,阮籍“箕踞相对”。在古代名教之士视之,箕踞对人是一种傲慢无礼的行为;但在得道隐者看来,却是一种不拘礼节的自由自在的表现。在这里,阮籍“箕踞相对”,取的是后一义,而并非鄙视苏门山真人。如\CJKunderwave{艺文类聚}卷一九引戴逵\CJKunderwave{竹林七贤论},谓“籍常箕踞啸歌,酣放自若”。又\CJKunderwave{任诞}第11则刘注引袁宏\CJKunderwave{名士传},谓裴楷往阮家吊丧,“遇籍方醉,散发箕踞,旁若无人”。可证阮籍箕踞,是一种重在舒适自由的习惯行为。其箕踞面对苏门山真人,并无不敬之心。相反,苏门山真人恰恰成了阮籍心目中善于栖逸隐居的理想人物。据刘注引\CJKunderwave{竹林七贤论},谓“籍归遂著\CJKunderwave{大人先生论},所言皆胸怀间本趣,大意谓先生与己不异也”。今阮集有\CJKunderwave{大人先生传},称其“养性延寿,与自然齐光。其视尧舜之所事,若手中耳。……行不赴而居不处,求乎大道而无所寓”,其为人应变顺和,“足与造化推移,故默探道德,不与世同”。这就揭示了栖逸隐居之士的最高理想,是不与虚伪的世俗礼教同流合污,专注于“求乎大道”——即探索宇宙自然以及人生的终极真理。故事中的“啸”,也与故事主旨密切配合。魏晋名士多好啸道,其啸分为歌啸、吟啸、长啸、讽啸诸类,当其“发声于丹唇,激音于皓齿,响抑扬而潜转,气冲郁而熛起……曲既终而响绝,遗馀玩而未已”,是一种“自然之至音”。究其功效之大,在于通过随心长啸,抒发心胸块垒,“精性命之至机,研道德之玄奥”(见成公绥\CJKunderwave{啸赋})。因此,在阮籍与苏门真人的长啸对答中,形象地展现了隐逸高士的求道之心,同时也生动暗示了\CJKunderwave{世说·栖逸}门的重要人生思考。}

\lettrine{18.2} 嵇康\myidx{嵇康}游于汲郡山中\footnote{汲郡:郡名,治所在今河南省汲县西南。},遇道士孙登\myidx{孙登}\footnote{孙登:字公和,汲郡共人。西晋隐士,常隐于汲郡山中,嵇康师事之。后不知所终。},遂与之游\footnote{游:游学。按:前句“嵇康游于汲郡山中”之“游”,则是旅游之义。}。康临去,登曰:“君才则高矣,保身之道不足。”{\fzxk\zihao{6}\textcolor{red}{\CJKunderwave{康集序}曰:“孙登者,不知何许人。无家,于汲郡北山土窟住。夏则编草为裳,冬则披发自覆。好读\CJKunderwave{易},鼓一弦琴。见者皆亲乐之。”\CJKunderwave{魏氏春秋}曰:“登性无喜怒,或没诸水,出而观之,登复大笑。时时出入人间,所经家设衣食者,一无所辞,去皆舍去。”\CJKunderwave{文士传}曰:“嘉平中,汲县民共入山中,见一人,所居悬岩百仞,丛林郁茂,而神明甚察。自云孙姓登名,字公和。康闻,乃从游三年,问其所图,终不答,然神谋所存良妙。康每谞然叹息。将别,谓曰:‘先生竟无言乎?’登乃曰:‘子识火乎?生而有光,而不用其光,果然在于用光;人生有才,而不用其才,果然在于用才。故用光在乎得薪,所以保其曜;用才在乎识物,所以全其年。今子才多识寡,难乎免于今之世矣。子无多求!’康不能用。及遭吕安事在狱,为诗自责云:‘昔惭下惠,今愧孙登。’”王隐\CJKunderwave{晋书}曰:“孙登即阮籍所见者也。嵇康执弟子礼而师焉。”魏、晋去就,易生嫌疑,贵贱并没,故登或嘿也。}}

{\cangkai\zihao{5}【评】据刘注引王隐\CJKunderwave{晋书},谓本则故事中,“孙登即阮籍所见者也”。这也就是说,苏门山得道真人与道士孙登同是一人,为嵇、阮二人所共见者。但余嘉锡\CJKunderwave{笺疏}承李慈铭之说而加以考证,力辩其非,其说可信。苏门山真人,阮籍作\CJKunderwave{大人先生传}时,已谓其“不知姓字”,王隐等又怎能知其名?可见苏门山真人仅是阮籍虚构的隐逸典型人物。而孙登则史上确有其人。今本\CJKunderwave{晋书·隐逸}有传。与前则故事中的阮籍一样,这里的嵇康同样作为第二主角来反衬孙登的隐逸智慧。嵇康和阮籍一样,是魏晋玄学思想家及文学家,声名垂之不朽。他在动荡乱世中,也很羡慕那自保其人格尊严的隐者生活。其\CJKunderwave{幽愤诗}自称“托好\CJKunderwave{老}、\CJKunderwave{庄},贱物贵身。志在守朴,养素全真。……仰慕严郑,乐道闲居。与世无营,神气晏如”。但作为一个见几知微的隐逸高士,孙登明确指出了嵇康“保身之道不足”的时代悲剧性,具有很高的预见性,从而启迪了后世的深刻人生思考。作为一个隐者,必须具有知几识时辨位的智慧。但嵇康才华横溢,妨碍了他的冷静思考,这也是一种性格悲剧,被孙登不幸言中。据\CJKunderwave{太平御览}卷四四七引袁宏\CJKunderwave{七贤序},云:“阮公瓌杰之量不移于俗,以获免者,岂不以虚中荦节,动无近对乎?中散(按:指嵇康)遣外之情最为高绝,不免世祸,将举体秀异,直致自高,故伤之者也!”嵇康愤世嫉俗,眼中容不下一粒沙子,如他在\CJKunderwave{与山巨源绝交书}中所描述的:“刚肠疾恶,轻肆直言,遇事便发。”而不问形势如何。在乱世中缺乏隐者知几识时之智,怎能不为世俗所害?悲乎!故嵇被囚牢中时作\CJKunderwave{幽愤诗}称:“昔惭柳惠,今愧孙登。”其思想感情之沉痛,催人泪下。}

\lettrine{18.3} 山公\myidx{山涛}将去选曹\footnote{山公:世人对山涛(字巨源)的尊称。选曹:指吏部,负责官吏的任免与选拔。},欲举嵇康\myidx{嵇康}\footnote{举:推荐。},康与书告绝\footnote{康与书告绝:绝,断交。\CJKunderwave{昭明文选}卷四三载嵇康\CJKunderwave{与山巨源绝交书},\CJKunderwave{晋书}康传则节录其文。}。{\fzxk\zihao{6}\textcolor{red}{\CJKunderwave{康别传}曰:“山巨源为吏部郎,迁散骑常侍,举康。康辞之,并与山绝。岂不识山之不以一官遇己情邪?亦欲标不屈之节,以杜举者之口耳。乃答涛书,自说不堪流俗而非薄汤、武。大将军闻而恶之。”}}

{\cangkai\zihao{5}【评】这是一则故事简短的叙述文字。表面平淡无奇。但如与嵇康\CJKunderwave{与山巨源绝交书}并读,自会感到故事的后面大有文章。山涛是嵇康的好友,他们都是竹林七贤中的人物。后来在政治上,二人做出了不同的选择。山涛靠近了司马氏集团,因而官运亨通;而嵇康则不和高倡虚伪礼教的司马氏集团合作,拒绝了当时统治者的仕途诱惑。山涛荐举嵇康做官,在官本位的封建社会中,世俗之人无不视为飞黄腾达的大好机会;但嵇康明白,一旦当了司马朝廷之官,即必须成为一个卖身投靠的奴才,从而丧失了独立的人格和做人的尊严。无拘无束的自由,比什么都宝贵,高官利禄又算什么?其\CJKunderwave{绝交书}所绝者何?读\CJKunderwave{世说}者,自有新解。其所“绝”并非针对一个人,而是一种时代的呐喊。嵇康和山涛,个人之间并无芥蒂或矛盾,二人私交甚好,甚至可以说是托以生死的挚友。嵇康临刑前,曾托孤于山涛,“谓子绍曰:‘巨源在,汝不孤矣。’”(\CJKunderwave{晋书·山涛传})后来,山涛果然在晋武帝面前提出“父子罪不相及”的理由,推荐嵇绍,起家秘书丞,成为有晋一代的忠义人物(见\CJKunderwave{晋书·忠义}绍传)。嵇康作书之后,何曾与山涛绝交?故其信中“绝”字之义,只能解释为借此机会,大作痛快文章,公开表示与司马集团的腐败虚伪绝交。其“非汤武而薄孔周”,以及不愿出仕的“二不可”“七不堪”,皆为惊世骇俗之论,如鲁迅所说,“嵇康于司马氏的办事上有了直接影响,因此就非死不可了”。如要了解魏晋易代的社会生活,读\CJKunderwave{世说}及\CJKunderwave{绝交书}可大开眼界。}

\lettrine{18.4} 李廞\myidx{李廞}是茂曾\myidx{李重}弟(第)五子\footnote{李廞(xīn欣)(?—约350):两晋之交人,字宗子,钟武(今河南信阳东南)人。茂曾:李重之字。西晋人。曾上疏陈九品中正之弊。其为官清正,安贫若素,声誉颇佳。},清贞有远操\footnote{清贞:清廉真正。远操:高尚的情操。},而少羸病\footnote{羸(léi雷)病:体弱多病。},不肯婚宦\footnote{婚宦:结婚和做官。}。居在临海\footnote{临海:郡名,治所在今浙江临海。},住兄侍中\myidx{李式}墓下\footnote{兄侍中:廞长兄李式,曾官侍中,故名。}。既有高名,王丞相\myidx{王导}欲招礼之\footnote{王丞相:指王导。招礼:招辟以礼,即礼聘为官。},故辟为府掾\footnote{辟:征辟,招聘。}。廞得牋命\footnote{牋命:授官文书。但\CJKunderwave{太平御览}卷三八六引作“板命”。晋代授官有板,板上有授官之辞。},笑曰:“茂弘乃复以一爵假人\footnote{茂弘:王导之字。乃复:竟然。以一爵假人:以一官位送人。假,假借,引申为给予、赠送。}。”{\fzxk\zihao{6}\textcolor{red}{\CJKunderwave{文字志}曰:“廞字宗子,江夏钟武人。祖康(秉),秦州刺史。父重,平阳太守,世有名望。廞好学,善草隶,与兄式齐名。躄疾不能行坐,常仰卧弹琴,读诵不辍。河间王辟太尉掾,以疾不赴。后避难,随兄南渡,司徒王导复辟之。廞曰:‘茂弘乃复以一爵加人。’永和中卒。廞尝为二府辟,故号李公府也。式字景则,廞长兄也。思理儒隐,有平素之誉。渡江,累迁临海太守、侍中。年五十四而卒。”}}

{\cangkai\zihao{5}【评】李廞的“躄疾”,也就是瘸腿,可能是从小患有严重的小儿麻痹症或其他风瘫之疾,所以不能行坐而正常生活,应该说,他是个一等残疾之人。但据刘注引\CJKunderwave{文字注},李廞是身残而志不残,他仍然顽强地学习和生活:克服病痛,“仰卧弹琴”而不改其乐;“读诵不辍”以提高自己的知识学问和精神境界;他以“善草隶”而名列书法家行列。他付出远远超出正常人的心血代价,因而李廞之贤,声名鹊起。于是官场之人,就把他的声名当作商标品牌来加以推销,两次公府征辟为官,就是企望借其声名以捞取礼贤下士的资本。但是李廞拒绝权势利禄的诱惑,他更重视的是自己的人格尊严。奔走于公府之门,他既无能为力,也不屑一顾;但他发现自己人生价值,走自己的栖逸之路,对生活仍然充满了信心。其讥讽王导“乃复以一爵假人”者以此。}

\lettrine{18.5} 何骠骑\myidx{何充}弟\myidx{何准}以高情避世\footnote{何骠骑弟:指何充之弟准。何准一生征辟不赴,隐居横门之下。其兄充则早为王导赏拔,位居宰执,权势显赫。高情避世:以隐居不仕保持其高雅情操。},而骠骑劝之令仕。答曰:“予弟五之名\footnote{弟五:弟,诸本作“第”,是。第五:何准为充之五弟,故云。},何必减骠骑\footnote{骠骑:指准兄何充,当时为骠骑将军。减:比……差,不如。}!”{\fzxk\zihao{6}\textcolor{red}{\CJKunderwave{中兴书}曰:“何准字幼道,庐江潜人。骠骑将军充第五弟也。雅好高尚,征聘一无所就。充位居宰相,权倾人主。而准散带衡门,不及世事。于时名德皆称之。年四十七卒。有女,为穆帝皇后。赠光禄大夫。子恢让不受。”}}

{\cangkai\zihao{5}【评】据何准“予第五之名,何必减骠骑”之言,则故事应当发生在晋康帝建元(343—344)之后。因为\CJKunderwave{晋书·何充传}载其“建元初,出为骠骑将军”之事,至穆帝时,皇太后临朝下诏,仍称“骠骑任重”。当时形势,较为混乱。王导卒后,庾氏家族以外戚专权,所以朝廷委何充以重任来加以平衡。后何充荐桓温代庾翼坐镇荆襄,却又逐渐形成了桓氏家族专政的时代。故\CJKunderwave{晋书}充传批评其虽位居宰相而“以社稷为己任”,但却因“信任不得其人”,缺乏“澄正改革之能”而一无所成。其权势之重,何尝增其清名?相反,其弟何准,在官本位的社会中,自负地说自己的声名不比高官显爵的哥哥差。在上流贵族社会中,既要有何充一类的人出来为官办事,又应当允许何准一类的隐者栖逸山林。作为“散带衡门”而终身不仕的隐者,何准知几识时,知其不可为而拒绝仕途,这与何充的努力化为烟云相比,表现了不同价值观念的人生取向。}

\lettrine{18.6} 阮光禄\myidx{阮裕}在东山\footnote{阮光禄:即阮裕,朝廷曾以光禄大夫征辟不赴,故称。东山:指会稽剡山,阮裕隐居之地。},萧然无事\footnote{萧然:凄冷清静貌。},常内足于怀\footnote{内足于怀:内心自满自足。}。{\fzxk\zihao{6}\textcolor{red}{\CJKunderwave{阮裕别传}曰:“裕居会稽剡山,志存肥遁。”}} 有人以问王右军\myidx{王羲之}\footnote{王右军:即王羲之,曾任右军将军,故称。},右军曰:“此君近不惊宠辱\footnote{近:接近,几乎。不惊宠辱:“宠辱不惊”的倒语,也即淡泊名利的意思。},{\fzxk\zihao{6}\textcolor{red}{\CJKunderwave{老子}曰:“宠辱若惊,得之若惊,失之若惊。”}} 虽古之沈冥\footnote{沈冥:指沉思冥想玄寂之道的隐逸之士。},何以过此!”{\fzxk\zihao{6}\textcolor{red}{\CJKunderwave{杨子}曰:“蜀庄沈冥。”李轨注曰:“沈冥,犹玄寂,泯然无迹之貌。”}}

{\cangkai\zihao{5}【评】阮裕,字思旷。\CJKunderwave{世说}作者刘义庆为避宋武帝刘裕名讳,故去其名而以官职称之。他出身于陈留阮氏家族,是名门之后。他如果想登仕途,实在可以做到取金紫如拾芥,但他却一再求隐。为什么?据\CJKunderwave{晋书}卷四九裕传,他年刚二十即被辟为太尉掾,被大将军王敦任命为主簿,甚被知遇。须知,当时“王与马,共天下”,敦专国政,权倾人主。但是,阮裕早已看出王敦的叛逆不臣的野心,于是故意“终日酣畅,以酒废职”,而被免官,终于躲过了王敦叛乱之劫难。于是世人才明白他年轻时就已产生的隐逸思想,实在具有高明的政治预见性,是一种不愿同流合污的人生智慧结晶。对于阮裕的肥遁东山,无官场应酬干扰而常感到内在精神充实得很的恬淡生活,王羲之给予宠辱不惊而与古人媲美的极高评价。当其受朝廷知遇之时,不为得宠而欢欣;当其退归东山而被禁锢终身不用之际,同样没有任何惊恐害怕,而是淡泊名利,顺其自然,体现了人性之真。他虽屡辞征聘,但却曾做二郡(临海、东阳)太守。人家问他为什么?他回答说:“虽屡辞王命,非敢为高也。吾少无宦情,兼拙于人间,既不能躬耕自活,必有所资,故曲躬二郡。岂以聘能,私计故耳。”(\CJKunderwave{晋书}卷四九裕传)公开表明自己出来做官并非高尚,而是生活“私计”所逼。其内心之真,纯洁透明,和那些“形在江湖之上,心存魏阙之下”的虚伪隐士相比,实在可爱得很。}

\lettrine{18.7} 孔车骑\myidx{孔愉}少有嘉遁意\footnote{孔车骑:即孔愉,字敬康,会稽山阴人。卒赠车骑将军,故称。嘉遁:美好之隐居生活。嘉,嘉美。},年四十馀,始应安东\myidx{司马睿}命\footnote{安东:指晋元帝司马睿,他未即帝位前,曾任安东将军镇守扬州,并于当时聘孔愉为参军。}。未仕宦时,常独寝歌自箴诲\footnote{独寝:独自起居寝处,即单独生活。歌:诸本于“歌”下增一“吹”字,是。歌吹,指啸歌。箴诲:规劝。},自称“孔郎”,游散山石\footnote{山石:袁本作“名山”,疑是。}。{\fzxk\zihao{6}\textcolor{red}{\CJKunderwave{孔愉别传}曰:“永嘉大乱,愉入临海山中,不求闻达。中宗命为参军。”}} 百姓谓有道术\footnote{道术:道教徒的养性成仙修炼之术。},为生立庙\footnote{为生立庙:在人活着时建立祠庙加以纪念。}。今犹有孔郎庙。

{\cangkai\zihao{5}【评】隐遁山林之人,并非尽皆道家之徒。据\CJKunderwave{晋书}卷七八\CJKunderwave{孔愉传}看来,孔愉是“达则兼济天下”而守正不阿,“穷则独善其身”而未凋其松柏志节,是一个儒道双修的仁人长者。西晋末年永嘉乱中,他东还会稽,栖遁新安山中,改姓以居,“以稼穑读书为务。信著乡里”。在动荡的年代里,不求闻达于诸侯而幽栖独处,并非做缩头乌龟,而是表现了一种保护生命、保存发展机会的一种远见卓识和人生智慧。在其隐居之地,老百姓为立生祠加以纪念,这是为什么?“百姓谓有道术”,认为他的生活很神奇而不同一般,因而视为神仙般的人物加以纪念;但同时他在隐居之时,不仅保存自我,而且还能善待百姓,在力所能及范围内为百姓做好事,所以本传才会称之为“信著乡里”,在当地百姓中很有威信。实际上,无论其出其处,淡泊名利,“勤抚其人,以济其艰”,以人为本,才是孔愉本意。所以他在晚年,虽然官高爵显,但只在山阴侯山买数亩地为宅,“草屋数间,便弃官居之”,公私馈赠数百万钱,“悉无所取”,表现了一个真诚隐者的心迹。}

\lettrine{18.8} 南阳刘驎之\myidx{刘驎之}\footnote{南阳:郡名,其治所宛,即今河南省南阳市。刘驎之:字子骥,一字遗民。},高率善史传\footnote{高率:高尚率真。},隐于阳岐\footnote{阳岐:濒临长江的小村,距荆州约二百里。}。于时苻坚\myidx{苻坚}临江\footnote{苻坚:前秦皇帝,曾率号称百万大军直逼东晋,企图统一中国,但在淝水大战中兵败,后为姚苌所杀。临江:前秦大军逼近长江流域。},荆州刺史桓冲\myidx{桓冲}将尽訏谟之益\footnote{桓冲:桓温弟,温卒,代其掌控东晋长江中上游兵力。驎(xū须)谟:事关国家民族的宏图远谋。},征为长史\footnote{征:征辟,聘任。长史:魏晋时三公及都督、将军府的主要属官。},遣人船往迎,赠贶甚厚\footnote{赠贶(kuànɡ况):馈赠,赠送。}。驎之闻命,便升舟,悉不受所饷\footnote{悉不受所饷:据李慈铭云:“案当作‘悉受所饷’,‘不’字疑衍。”疑是。饷,馈赠。},缘道以乞穷乏\footnote{缘道:沿路。乞:给,予。穷乏:指贫苦之人。},比至上明亦尽\footnote{比:及。上明:桓冲任荆州刺史时为抵御苻坚而筑,并迁州治于此,其地在今湖北松滋市西。}。一见冲,因陈无用,翛然而退\footnote{翛(xiāo消)然:超然自在貌。}。居阳岐积年\footnote{积年:多年。},衣食有无,常与村人共,直己匮乏\footnote{直:通“值”。},村人亦知之,甚厚为乡闾所安\footnote{甚厚:李慈铭谓“厚字疑衍”,可参考。乡闾:乡里,这里指同乡村的人。安:安适。}。{\fzxk\zihao{6}\textcolor{red}{邓粲\CJKunderwave{晋纪}曰:“驎之字子骥,南阳安众人。少尚质素,虚退寡欲,好游山泽间,志存遁逸。桓冲尝至其家,驎之方条桑,谓冲:‘使君既枉驾光临,宜先诣家君。’冲遂诣其父,父命驎之,然后乃还,拂短褐与冲言。父使驎之自持浊酒菹菜供宾,冲敕人代之,父辞曰:‘若使官人,则非野人之意也。’冲为慨然,至昏乃退。因请为长史,固辞。居阳岐,去道斥近,人士往来,必投其家。驎之身自供给,赠致无所就(受)。去家百里,有孤妪疾,将死,谓人曰:‘只有刘长史当埋我耳!’驎之身往候之,值终,为治棺殡。其仁爱皆如此。以寿卒。”}}

{\cangkai\zihao{5}【评】作为栖逸隐士,刘驎之明白自己的人生价值,并不是故意逃避现实。但前秦苻坚大军南侵,江南震动之时,荆州刺史桓冲慕其高名,邀其出山,共襄却敌大计。刘驎之并没有推托,而是“闻命便升舟”,随使者前往,不敢有丝毫的耽搁。但是,他“一见冲,因陈无用,翛然而退”。这不是隐士取名的故作姿态,而是实事求是之举。因为刘驎之对自己的才能和价值有清醒的认识。他不是政治家、军事家,长年累月在僻静农村,社会的急剧变化及其信息的传递,总是要慢了许多节拍。而军国大计,关系国家兴亡和民族命运,怎能只是依靠书本中的知识来纸上谈兵呢?试想,如果刘驎之不是自陈“无用”,而是缺乏自知之明,贪图功名富贵,任职长史,出馊主意,形势又当如何呢?其自陈“无用”,正是无用之用,是一种真正对国家和民族严肃负责的态度。这样说来,是否隐士一无价值呢?非也。据\CJKunderwave{晋书·隐逸}传,刘驎之“少尚质素,虚退寡欲……好游山泽间,志存遁逸”,作为隐士,他确认自己另有不同于政治家的人生价值。他“高率善史传”,是个认真读书、好学深思的学问家。他淡泊名利,金钱资财与乡邻朋友共,救穷济乏,是其发自心底的自愿行为。一孤姥将死,叹息谓人曰:“谁当理我,惟有刘长史(按:指刘驎之)耳!”刘驎之即赶往她家候之,为之“营棺殡送终”,其仁爱隐恻如此,可称为关心群众生活疾苦而受大家爱戴的慈善家。他又是一位地理学家,旅游探险家。陶渊明\CJKunderwave{桃花源记}曾说:晋太元中,“南阳刘子骥(即驎之),高尚士也,闻之,欣然规往。未果,寻病终,后遂无问津者”。据刘注及\CJKunderwave{晋书}本传,他与桓冲接触,不亢不卑,并未屈服于官本位的社会陋俗,于此见其对人生价值另有清醒的认识。隐者并非忘记世界,而是在生活中另走一条属于自己的道路。}

\lettrine{18.9} 南阳翟道渊\myidx{翟汤}与汝南周子南\myidx{周邵}少相友\footnote{南阳:郡名,其治所宛(今河南南阳市)。翟道渊:即翟汤。\CJKunderwave{晋书·隐逸}传称其字“道深”,是唐代作者避高祖李渊名讳而改。汝南:郡名,治所在悬瓠城(今河南汝南)。周子南:即周邵。初与翟汤共隐,后为庾亮所荐,起为镇蛮护军、西阳太守。},共隐于寻阳\footnote{寻阳:即浔阳,县名,故址在今江西九江市西。}。庾太尉\myidx{庾亮}说周以当世之务\footnote{庾太尉:指庾亮。说(shuì税):劝说。当时之务:指从政为官,走上仕途。},周遂仕。翟秉志弥固\footnote{秉志弥固:坚守自己的隐逸志趣更加坚固。}。其后周诣翟,翟不与语。{\fzxk\zihao{6}\textcolor{red}{\CJKunderwave{晋阳秋}曰:“翟汤字道渊,南阳人。汉方进之后也。笃行任素,义让廉洁,馈赠一无所受。值乱多寇,闻汤名德,皆不敢犯。”\CJKunderwave{寻阳记}曰:“初庾亮临江州,闻翟汤之风,束带蹑屐而诣焉,亮礼甚恭。汤曰:‘使君直敬其枯木朽株耳。’亮称其能言,表荐之,汤征国子博士,不赴。主簿张玄曰:‘此君卧龙,不可动也。’终于家。”}}

{\cangkai\zihao{5}【评】古之隐逸,有真有假。这个故事,通过一对朋友的不同志趣和道路来相互反衬,形象描绘了隐者的真假本质。如周邵之辈,其隐居显然处于一种“流行”的观念,认为这样可以获取高名,以便为今后的仕途打通一条新的“终南捷径”。而当时庾太尉亮又说之以“当世之务”,其与邵书云:“西阳一郡,户口差实。……今具上表,请足下无让。”(\CJKunderwave{世说·尤悔}第10则刘注引\CJKunderwave{寻阳记})用另外一种功名富贵的“流行”观念来打动他,这就不是偶然的。关键在于周邵并无真正坚守其隐逸君子人格和理想志趣的决心。故其或出或处,无不是一种受“流行”观念震撼而迅速动摇自我信念的行为。这样的人,必然为自己易于动摇付出惨重的代价。须知,入仕做官,很难保持自己的操守。周邵后来做官果然“不称意”,并没按照自己的理想一路飞黄腾达,精神郁郁而导致“发背而卒”,悲乎!故事以周邵作为反面形象,更托出真正隐士翟汤的人格之高尚。翟汤隐逸不仅是坚守自我志趣,而且同时不忘其社会责任。\CJKunderwave{晋书·隐逸}传称安西将军庾翼北征,敕有司蠲免翟汤所调,汤拒绝接受利益,“推仆使悉之乡吏”,以便为国家出力。真正高尚的隐逸君子,并非不食人间烟火!}

\lettrine{18.10} 孟万年\myidx{庾亮}说周以当世之务\footnote{庾太尉:指庾亮。说(shuì税):劝说。当时之务:指从政为官,走上仕途。},周遂仕。翟秉志弥固\footnote{秉志弥固:坚守自己的隐逸志趣更加坚固。}。其后周诣翟,翟不与语。{\fzxk\zihao{6}\textcolor{red}{\CJKunderwave{晋阳秋}曰:“翟汤字道渊,南阳人。汉方进之后也。笃行任素,义让廉洁,馈赠一无所受。值乱多寇,闻汤名德,皆不敢犯。”\CJKunderwave{寻阳记}曰:“初庾亮临江州,闻翟汤之风,束带蹑屐而诣焉,亮礼甚恭。汤曰:‘使君直敬其枯木朽株耳。’亮称其能言,表荐之,汤征国子博士,不赴。主簿张玄曰:‘此君卧龙,不可动也。’终于家。”}}

{\cangkai\zihao{5}【评】古之隐逸,有真有假。这个故事,通过一对朋友的不同志趣和道路来相互反衬,形象描绘了隐者的真假本质。如周邵之辈,其隐居显然处于一种“流行”的观念,认为这样可以获取高名,以便为今后的仕途打通一条新的“终南捷径”。而当时庾太尉亮又说之以“当世之务”,其与邵书云:“西阳一郡,户口差实。……今具上表,请足下无让。”(\CJKunderwave{世说·尤悔}第10则刘注引\CJKunderwave{寻阳记})用另外一种功名富贵的“流行”观念来打动他,这就不是偶然的。关键在于周邵并无真正坚守其隐逸君子人格和理想志趣的决心。故其或出或处,无不是一种受“流行”观念震撼而迅速动摇自我信念的行为。这样的人,必然为自己易于动摇付出惨重的代价。须知,入仕做官,很难保持自己的操守。周邵后来做官果然“不称意”,并没按照自己的理想一路飞黄腾达,精神郁郁而导致“发背而卒”,悲乎!故事以周邵作为反面形象,更托出真正隐士翟汤的人格之高尚。翟汤隐逸不仅是坚守自我志趣,而且同时不忘其社会责任。\CJKunderwave{晋书·隐逸}传称安西将军庾翼北征,敕有司蠲免翟汤所调,汤拒绝接受利益,“推仆使悉之乡吏”,以便为国家出力。真正高尚的隐逸君子,并非不食人间烟火!}

\lettrine{18.10} 孟万年\myidx{孟嘉}\footnote{孟万年:即孟嘉}及弟少孤\myidx{孟陋}\footnote{孟万年:即孟嘉。陶渊明之外祖父。渊明为作\CJKunderwave{晋故征西大将军长史孟府君传},可详参。少孤:即孟陋。\CJKunderwave{晋书·隐逸}有传。},居武昌阳新县\footnote{武昌:郡名。阳新:武昌郡属县,今属湖北省。}。万年游宦\footnote{游宦:外出做官。},有盛名当世。少孤未尝出京邑\footnote{出京邑:到京师。\CJKunderwave{世说}之“出”,六朝习惯用语,特指由隐之显的行为,京城处位显要,故进京称“出都”或“出京邑”。见吴金华\CJKunderwave{世说新语考释}。}。人士思欲见之,乃遣信报少孤云:“兄病笃\footnote{病笃:病危。}。”狼狈至都\footnote{狼狈:形容急忙赶路的情急之态。}。时贤见之者,莫不嗟重\footnote{嗟重:叹赏推重。}。因相谓曰:“少孤如此,万年可死。”{\fzxk\zihao{6}\textcolor{red}{袁宏\CJKunderwave{孟处士铭}曰:“处士名陋,字少孤。武昌阳新人。吴司空子(孟)宗后也。少而希古,布衣蔬食,栖迟蓬荜之下,绝人好(间)之事。亲族慕其孝。大将军命会稽王辟之,称疾不至。相府历年虚位,而澹然无闷,卒不降志。时人奇之。”}}

{\cangkai\zihao{5}【评】对比艺术,是\CJKunderwave{世说}创作的重要艺术手法。故事通过一对亲兄弟的对比,生动传达了作者对于隐逸之士高尚君子人格的推扬和叹赏。在叙述描写中,作者对孟嘉并没有直接加以贬损的笔墨,相反,称其因游宦而“有盛名当世”,评价还不错。但是,声名之大小及评价的好坏,是相比较而成立的。孟陋误信“兄病笃”的流言,情急之下,打破自己不愿离开隐居之地而奔赴繁华京师的原则,“狼狈至都”,这一生动的细节,相当典型,表现了对于兄长的真诚仁爱之心,令人终生难忘。隐者所逃避的是混乱的世俗,而不是充满亲情友爱的人生。对人的关爱,同样洋溢在隐者的胸臆。“少孤如此,万年可死”,这当然是一种艺术修辞的夸张手法,并非真是希望孟嘉立即死去;但却刻画了魏晋人士尊重隐逸的时代心理。不过,这一夸张说法,对于兄长孟嘉,未免残酷一些。孟氏兄弟三人,父母早逝,一家生活重担,压在长兄孟嘉身上,他不出来“游宦”养家活口,一家人都要喝西北风。至于孟陋,他也不以隐居获取高名,据\CJKunderwave{晋书·隐逸}陋传,他明白宣示真心:“亿兆之人,无官者十居其九,岂皆高士哉!我疾病不堪恭相王之命,非敢为高也。”他根据自己的身体特点和个性要求,做个隐者,安静做学问,终于成了著名的三\CJKunderwave{礼}及\CJKunderwave{论语}专家,从而表现了自己的人生价值。}

\lettrine{18.11} 康僧渊\myidx{康僧渊}在豫章\footnote{康僧渊:见前\CJKunderwave{文学}47注。西域僧人。出生长安。后渡江南下,为东晋名僧。豫章:郡名,治所在今江西南昌市。},去郭数十里立精舍\footnote{郭:外城。精舍:学舍,讲堂。这里指僧人道士修炼讲习之屋。},傍连岭,带长川,芳林列于轩庭\footnote{芳林:芳草树木。轩庭:带有回廊栏杆的庭院。},清流激于堂宇。乃闲居研讲\footnote{闲居:独居避俗。},希心理味\footnote{希心:倾心,潜心。理味:义理韵味。}。庾公\myidx{庾亮}诸人多往看之\footnote{庾公:指庾亮。},观其运用吐纳\footnote{吐纳:指说话吞吐的声调气息变化。},风流转佳\footnote{风流:仪表风度。转:更。},加处之怡然\footnote{加处之怡然:袁本“加”下增一“已”字,作“加已处之怡然”。已,通“以”。另,“已”又可能为“己”之形讹。“已”,据张万起\CJKunderwave{世说新语词典},可用作第三人称代词,犹“之”、“他”。亦通。怡然:愉悦貌。},亦有以自得\footnote{自得:自感舒适。},声名乃兴。后不堪,遂出。{\fzxk\zihao{6}\textcolor{red}{僧渊,已见。}}

{\cangkai\zihao{5}【评】栖逸山林之人,可以是儒者,或是道士,也可以是佛教僧人,康僧渊即其例。康僧渊是东晋初期名僧。他栖逸山林的生活内容,主要是以教育为中心的讲学活动。当然,随着思想形势的发展变化,为使佛学理论适应中国具体环境的需要,康僧渊吸取了当时风势正旺的玄理,以构成其佛、玄合流的新的佛学义理。据故事所描绘,康僧渊的精舍讲堂,远离尘俗,环境幽雅,又颇具规模,是主人公教授生徒和会友讲学,甚至是辩论义理的好地方,也可以说是一所具有专门化理论知识的别具一格的私立僧侣学校。故事称其“闲居研讲”,说明其精舍不仅为潜心学术的书斋,更是他与众人研讨讲习的课堂,而非是个人独自居住而自说自话的“囚狱”。其所“研讲”,当有其听众对象,不仅有一般的学生,更有学问精湛的一些士夫学者,如庾亮一辈人物。庾亮为东晋政界要人,事务纷繁,还要出城与康僧渊会面研讨义理。因此,康僧渊的义理之学,必有过人独特之处,从而吸引了大批学生和听众,他与庾亮等政要处于亦师亦友之间。其教学研讨的内容,特点是“希心理味”,已非一般教授传统儒家章句之学,也不是一般死译佛经教义。而是像他曾与王导、殷浩等玄学清谈家所讨论的那样,结合玄学义理,来对佛理进行新的阐释。于此可见,只要潜心教育与学术,隐居生活也是天地宽阔而自有乐地。后来他不耐寂寞而出山,是志趣改变的结果,那又另当别论了。}

\lettrine{18.12} 戴安道\myidx{戴逵}既厉操东山\footnote{戴安道:即戴逵,东晋画家,参前\CJKunderwave{雅量}34注。厉操:激扬操守。东山:此指其隐居地会稽剡山。},{\fzxk\zihao{6}\textcolor{red}{\CJKunderwave{续晋阳秋}曰:“逵不乐当世,以琴书自娱,隐会稽剡山。国子博士征,不就。”}} 而其兄\myidx{戴逯}欲建式遏之功\footnote{其兄:指逵兄逯,刘注引\CJKunderwave{戴氏谱}作“逯”,误。按,据\CJKunderwave{晋书·谢玄传}附从玄征伐者有戴逯,字安丘。但明谓“处士逵之弟”。故逯对谢安之问,改“家弟不改其乐”为“家兄”,与\CJKunderwave{世说}略异。式遏:典出\CJKunderwave{诗经·大雅·民劳}“式遏寇虐”之言。式,语首助词,无实义。遏,制止,遏制。原指制止对人民的侵害掠夺。这里引申为建功立业之义。}。{\fzxk\zihao{6}\textcolor{red}{\CJKunderwave{戴氏谱}曰:“逯(逯)字安丘,谯国人。祖硕,父绥,有名位。逯以武勇显,有功,封广陵侯,仕至大司农。”}} 谢太傅\myidx{谢安}曰\footnote{谢太傅:指谢安。}:“卿兄弟志业\footnote{志业:志向与事业。},何其太殊?”戴曰:“下官不堪其忧,家弟不改其乐\footnote{下官不堪其忧,家弟不改其乐:语出\CJKunderwave{论语·雍也}孔子之美弟子颜回:“贤哉回也,一箪食,一瓢饮,在陋巷,人不堪其忧,回也不改其乐。”戴逯之意,自己是世俗的一般之人,故“不堪其忧”;但其弟则是隐逸高士,如颜回一样的贤人,故“不改其乐”。}。”

{\cangkai\zihao{5}【评】戴逯对谢安问,说话生动、形象而幽默,从中透视出特定时代的诗人心理。在魏晋隐士中,戴逵是个具有贤者风范的儒家典型。史称其“性高洁,常以礼度自处,深以放达为非道”,并著论加以批判。说“乡愿似中和,所以乱德;放者似达,所以乱道”。反对“怀情丧真”的虚伪之隐。戴逯、戴逵兄弟,皆与谢安相识。谢安离开东山而出仕之后,戴逵继续隐居东山。而戴逯作为谢安属下将军,“骁果多权略”而“以武勇显”(见\CJKunderwave{晋书·谢安传}附)。兄弟二人各有不同的人生价值取向及社会贡献。逯以军功封广信侯,当然对国家安全有所贡献。但逵则在隐居的安静环境中,成为一位博学“善属文,能鼓琴,工书画,其馀巧艺靡不毕综”(见\CJKunderwave{晋书·隐逸传})的艺术名家,为人类创造了不朽的精神文化遗产。二者贡献相比,戴逯有自惭形秽之色,故有篇末谦虚之言。但逯言同时也反映出魏晋士人高尚隐逸的时代思潮。此事如果发生在官本位的时代,戴逯如果是一个严重的官本位论者,则必然会改变其口吻声调。}

\lettrine{18.13} 许玄度\myidx{许询}隐在永兴南幽穴中\footnote{许玄度:许询,字玄度,参\CJKunderwave{言语}62注。永兴:县名,晋时会稽郡属县,故城在今萧山西。幽穴:僻静清幽的山洞。},每致四方诸侯之遗\footnote{四方诸侯:各地高级官员。遗:馈赠。}。或谓许曰\footnote{或:有人。}:“尝闻箕山人\footnote{箕山人:原指尧时隐居箕山的高士许由。箕山,在今河南登封东南。传说尧让天下于许由被拒绝,故事借喻隐逸高士。},似不尔耳\footnote{尔:如此,这样。}。”许曰:“筐篚苞苴\footnote{筐篚(fěi诽):竹筐,方为筐,圆为篚。苞苴(jū狙):草包,用以裹鱼肉。按:筐篚苞苴,喻所包装礼物价值一般。},故当轻于天下之宝耳\footnote{天下之宝:天子之位。故当:自然。}。”{\fzxk\zihao{6}\textcolor{red}{郑玄\CJKunderwave{礼记注}云:“苞苴,裹肉也,或以苇,或以茅。”此言许由尚致尧帝之让,筐篚之遗,岂非轻邪?}}

{\cangkai\zihao{5}【评】魏晋隐士的生活,有三种来源:一是躬耕陇亩,自食其力;一是教授生徒,糊口度日;一是虚邀声誉,依靠高官政要的馈赠而养尊处优,悠游林泉。前二种隐士,虽然物质生活相对贫乏,但因不依赖于权门,从而获得了一定的心灵自由和精神解放。如郭瑀指翔鸿以绝诸侯之征,云:“此鸟也,安可笼哉!”遂深逃绝迹(见\CJKunderwave{晋书·隐逸传}),表现出蔑视权贵的高尚品格。相反,依靠权门政要馈赠而发财致富的“隐士”,其虚名之腾,由人为炒作而致,已日渐背离了隐居生活的本质及其君子人格,岂是甘贫乐道以求安静读书做学问之人!俗话说:“吃人家的嘴软,拿人家的手短。”歆羡富贵又不耐寂寞,又岂能不为嗟来之食而折腰权势呢?一旦违心而失却自由,又哪有独立人格可言!豢养由人,当然也就只能听人吆喝了。前\CJKunderwave{文学}第64则载许询赴都,寓于丹阳尹刘惔家中,羡慕其锦衣美食,而有“殊胜东山”隐居生活之叹。当时王羲之在座,讥之云:“令巢(父)许(由)遇稷、契,当无此言。”许色大愧。故事中许询之言,不过是一种自我解嘲的无力辩解,充分表现了其隐居的虚伪矫饰。隐者向权贵靠拢,思想渐趋合流,这在当时自有一定的代表性,所以许询终于成为统治者手中的一张金字招牌。}

\lettrine{18.14} 范宣\myidx{范宣}未尝入公门\footnote{范宣:字宣子,陈留人。东晋名儒。\CJKunderwave{晋书·儒林传}有传。公门:政府衙门或官员府邸。}。韩康伯\myidx{韩伯}与同载\footnote{韩康伯:韩伯,字康伯,颍川长社人(今河南长葛)人。时任豫章太守。},遂诱俱入郡\footnote{入郡:进入郡守府衙。},范便于车后趋下\footnote{趋:疾走。}。{\fzxk\zihao{6}\textcolor{red}{\CJKunderwave{续晋阳秋}曰:“宣少尚隐通(遁),家于豫章,以清洁自立。”}}

{\cangkai\zihao{5}【评】在魏晋栖隐高士中,范宣属于不言流行老庄之道的儒者。据前\CJKunderwave{德行}第38则,他曾在马车中半推半就地接受了豫章太守韩康伯的二丈绢,以免老婆穷得没裤子穿。如果范宣是位高权重的政要人物,可能“公关”小姐就会盯上他,此“关”一破,半推半就的二丈,进而可能是二匹、二十匹、二百匹也说不定。但幸亏范宣只是一介以“讲诵为业”的民间教书先生,大概此后再没有什么人走他的“后门”,因此他也就专心隐遁,潜心读书治学,以便教好学生。后来,他果然“博综群书”,著\CJKunderwave{礼}、\CJKunderwave{易}论难诸书行世。他的学生,当时“闻风宗仰,自远而至,讽诵之声,有若齐、鲁”,可见其私立学校效果很好,像上面提到的画家戴逵,就是他的高足。学生的造诣,就是老师的幸福和骄傲。他在隐居生活中的成就,多亏他后来悟到权势之“关”不可破的道理,拒绝了太守的官场势利之诱惑。这一“定力”很重要。所以,他虽一时上当,上了太守的马车,但前车进,后车出,回头是岸,也算及时。不然,当时可能就会落下笑话了。}

\lettrine{18.15} 郗超\myidx{郗超}每闻欲高尚隐退者\footnote{郗超:司空愔长子。曾入桓温大将军幕府,权倾一时。高尚:特指高尚隐逸之事。语出\CJKunderwave{易·蛊}卦上九爻辞:“不事王侯,高尚其事。”},辄为办百万资\footnote{辄(zhé哲):即,总是。百万资:\CJKunderwave{晋书}作“百金”,喻钱之多。},并为造立居宇。在剡\footnote{剡(shàn擅):县名,治所在今浙江嵊州市。},为戴公\myidx{戴逵}起宅甚精整\footnote{戴公:对戴逵的尊称。精整:精美齐整。}。戴始往(旧)居\footnote{戴始往旧居:\CJKunderwave{太平御览}卷一五○\CJKunderwave{逸民}引,“往”下无“旧”字。徐震堮\CJKunderwave{校笺}、朱铸禹\CJKunderwave{汇校集注}疑“旧”字衍,当删。},与所亲书曰:“近至剡如官舍\footnote{如官舍:据\CJKunderwave{御览}卷一五○作“如入官舍”。}。”郗为傅约\myidx{傅琼}亦办百万资\footnote{傅约:即傅琼。余嘉锡\CJKunderwave{笺疏}疑约为傅瑗之兄弟行。},傅隐事差互\footnote{差互:蹉跎不遂。},故不果遗\footnote{遗:赠送。}。{\fzxk\zihao{6}\textcolor{red}{约,琼小字。}}

{\cangkai\zihao{5}【评】在东晋中期桓温专擅朝政的年代里,年轻气盛的郗超,是桓温集团中的主要谋士,是个炙手可热的人物,不仅是谢安等人对他畏忌三分,就是桓温的废立大计,也是郗超为之主谋,连简文帝司马昱在他面前也只能请求与叹气。应该说郗超是个不折不扣的政要权贵了。但就是这个郗超,却偏是信奉佛教,推扬隐逸,支持隐士而毫不吝惜。照理说,隐士是高尚其事,不事王侯,有才不为朝廷所用,在政治上不与朝廷合作,在思想上颇多扞格,政要不予直接打击,已属宽容。可是,郗超却相反地予以物质支持,这是为什么?原来,郗超是个多谋多智的政治家,他自己热衷政治与权势,但在其内心深处,却也明白腥风血雨政治斗争之残酷,因此,他很想为自己的内心世界求得一方净土。自己做不到,就寄托在宗教与隐逸者身上。对于郗超个人,这是一种心理压抑的反弹;而在社会,则反映出士大夫视隐逸为高尚理想之举。隐士们多趋于安静做学问或游山玩水,对朝政并无威胁,因此可以用来为国家点缀升平。魏晋士夫心态,于此可见一斑。}

\lettrine{18.16} 许掾\myidx{许询}好游山水\footnote{许掾:许询曾以司徒掾征,故称。},而体便登陟\footnote{体便登陟:身体便捷,利于登山涉水。}。时人云:“许非徒有胜情\footnote{非徒:不只。胜情:美好心情。},实有济胜之具\footnote{济胜之具:攀涉山水胜境的身体健康条件。}。”

{\cangkai\zihao{5}【评】对于栖逸隐士来说,并非只过独闭岩穴面壁而坐的枯燥生活,游山玩水而登览胜景,也是隐逸生活的重要的内容之一,即在幽静美丽的山水自然中,寻找自我,获得心灵的安慰。但要实现这一良好的愿望,不仅要有“胜情”——即精神上的向往,还要有“胜具”——即健康的体魄。在登览江山胜景的同时,必须具有一定的探险精神,才能见人之所未见,想人之未想,从而对人生有自己的新发现。}

\lettrine{18.17} 郗尚书\myidx{郗恢}与谢居士\myidx{谢敷}善\footnote{郗尚书:郗恢(?—398),字道胤,东晋高平金乡(今属山东)人。昙子。曾以雍州刺史镇襄阳,抗击姚苌。后以尚书征,途中被殷仲堪所害。谢居士:指谢敷,终生隐居不仕。居士,居家奉佛或修道的人。},常称谢庆绪识见虽不绝人\footnote{绝人:超越人们。},可以累心处都尽\footnote{累心处:指世俗烦恼之事。}。{\fzxk\zihao{6}\textcolor{red}{尚书,郗恢也,别见。檀道鸾\CJKunderwave{续晋阳秋}曰:“谢敷字庆绪,会稽人。崇信释氏。初入太平山中十馀年,以长斋供养为业,招引同事,化纳不倦。以母老还南山若邪中。内史郗愔表荐之,征博士不就。初,月犯少微星,一名处士星。占云:‘以处士当之。’时戴逵居剡,既美才艺,而交游贵盛,先敷箸名,时人忧之。俄而敷死,会稽人士以嘲吴人云:‘吴中高士,便是求死不得。’”}}

{\cangkai\zihao{5}【评】故事借郗恢之言,表彰谢敷心胸与人格之美。因此,故事的第一主人公是谢居士。不过,郗恢的认识也代表了魏晋士人对于隐逸之风的尊重,这是一股时代习气,对于争名夺利的世俗之徒,也是一种无言的否定和批判。刘注拿戴逵与谢敷作比,而二人皆为隐逸之士,但戴氏“交游贵盛”,结识众多贵要名士,比如接受郗超馈赠的精美房舍,虽说出于郗超自愿,但是,这仍是一种无名之求,无功而受禄,经人炒作而著名,实内心有愧。相比之下,则谢敷淡泊名利,专心隐居而无求于人,此其所以高于戴氏。戴逵虽隐居于会稽郡剡山,但其拒绝朝廷国子博士之征时,曾避居于吴,故刘注称其为吴中高士。谢敷之死,甚至比戴逵之生更值得人们怀念,可见在魏晋士人心目中,不为世俗所累的真隐士具有崇高的地位。}






%%% Local Variables:
%%% mode: latex
%%% TeX-engine: xetex
%%% TeX-master: "../Main"
%%% End:
