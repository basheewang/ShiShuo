%% -*- coding: utf-8 -*-
%% Time-stamp: <Chen Wang: 2025-12-07 11:54:59>

% ○ ◎ ‧ 「 」 『 』 々 ( ) “ ” ■ ^[一-龥]
% 【\([^】][^】][^】]+\)】 → {\\fzxk\\zihao{6}\\textcolor{red}{\1}}
% \(【评】.*\) → {\\cangkai\\zihao{5}\1}
% \(【题解】.*\) → {\\cangkai\\zihao{5}\1}
% 《\([^》]+\)》 → \\CJKunderwave{\1}
% ^\([0-9]+.[0-9]+\) → \\lettrine{\1}
% {\\fzxk\\zihao{6}\\textcolor{red}{[^o]*}}


\setlength{\parindent}{0pt}


\chapter{术解第二十}



{\cangkai\zihao{5}【题解】 术解者,术谓术数方术、技术技艺之类,古时如医学、占卜、星相、风水、相马以及音律等,都属于“术”的范围;而所称解,指精解或通达事理之谓,\CJKunderwave{世说}以“术解”连称,明显是指通晓术数、精通技艺的人和事。以占卜算卦为主,医术治病次之,再次为音律、相马诸术。在古代,国之大事,唯祀与戎,祭祀与战争,是最大之事,国君不可假手于人。而不管是“祀”或“戎”,在决事之前,都必须以龟卜\CJKunderwave{易}筮来占问吉凶,以便作为国家决策的重要参考。故上古时代的占卜算卦,虽然属于巫术迷信之事,但占卦解卦之巫史,作为沟通天人之际的传人,则具有很高的文化修养及社会历史地位,这在\CJKunderwave{世说}所处的中古时期,仍然有所反映。人们对\CJKunderwave{易}卦大师郭璞的推崇赞许就是典型事例,甚至是帝王如晋明帝也“解占冢宅”,精通风水之术,欲与郭璞一比高低。占卜与风水之术,其难不在占卜而在具体解释。精明之术人,常在迷信外衣下,做出某些比较合乎科学或准科学的精解。至于医学诸术,因统治者的歧视,斥之于上流社会之外。由于这种观念性的错误认识,也影响了古代自然科学的健康发展。}

\lettrine{20.1} 荀勖\myidx{荀勖}善解音声\footnote{荀勖(xù序)(?—289):魏晋间人。字公曾,颍川颍阴(今河南许昌)人。魏时为安阳令,晋时官至中书监、侍中。参\CJKunderwave{方正}第14则注。勖因善解韵律而兼掌朝廷乐事。善解:精通。音声:音乐,兼指乐理。},时论谓之“闇解”\footnote{时论:当时舆论。闇解:自然解悟。}。遂调律吕\footnote{律吕:泛指乐律。古时乐律阴阳各有六律,阳律为黄钟、太蔟、姑洗、蕤宾、夷则、无射,阴律为大吕、夹钟、仲吕、林钟、南吕、应钟,阴阳合而为十二乐律。},正雅乐\footnote{雅乐:朝廷郊庙之乐。}。每至正会\footnote{正会:古时元旦君臣朝会之称,又名元会。},殿庭作乐,自调宫商\footnote{调宫商:以乐律校正调谐乐器之五音。古时通行宫、商、角、徵、羽五声音阶,故以“宫商”指代音律。},无不谐韵\footnote{谐韵:和谐合乐。}。阮咸\myidx{阮咸}妙赏\footnote{阮咸:字仲容,魏晋间陈留尉氏(今属河南)人,阮籍之侄,与籍同列竹林七贤。},时谓“神解”\footnote{神解:一种触动灵感而自然天成的心领神会。}。每公会作乐\footnote{公会:会事集会。},而心谓之不调\footnote{不调:指音乐演奏时音律不和谐。}。既无一言直勖\footnote{直:肯定正确。},意忌之,遂出阮为始平太守\footnote{出:外放。当时朝官位尊,故外放含贬义。始平:郡名,治所槐里(今陕西兴平市东南)。}。后有一田父耕于野\footnote{田夫:农夫。},得周时玉尺,便是天下正尺\footnote{正尺:音律的标准尺。便是:恰是。}。荀试以校己所治钟鼓、金石、丝竹,皆觉短一黍\footnote{钟:原刻作“锺”,袁本作“钟”。黍:\CJKunderwave{晋书·乐律志}作“米”。黍,古时长度单位,以一粒中等黍米的纵向长度为一分,积百黍米为一尺。短一黍,即短一分。},于是伏阮神识\footnote{伏:佩服。神识:神妙见识。}。{\fzxk\zihao{6}\textcolor{red}{\CJKunderwave{晋后略}曰:“钟律之器,自周之末废,而汉成、哀之间,诸儒修而冶(治)之。至后汉末,复隳矣。魏氏使协律知音者杜夔造之,不能考之典礼,徒依于时丝管之声、时之尺寸而制之,甚乖失礼度。于是世祖命中书监荀勖依典制,定钟律。既铸律之管,慕求古器,得周时玉律数枚,比之不差。又诸郡舍仓库,或有汉时故钟,以律命之,皆不叩而应,声音韵合,又若俱成。”\CJKunderwave{晋诸公赞}曰:“律成,散骑侍郎阮咸谓勖所造声高,高则悲。夫亡国之音哀以思,其民困。今声不合雅,惧非德政中和之音,必是古今尺有长短所致。然今钟磬是魏时杜夔所造,不与勖律相应,音声舒雅,而久不知夔所造,时人为之,不足改易。勖性自矜,乃因事左迁咸为始平太守,而病卒。后得地中古铜尺,校度勖今尺,短四分,方明咸果解音,然无能正者。”干宝\CJKunderwave{晋纪}曰:“荀勖始造\CJKunderwave{正德}、\CJKunderwave{大象}之舞,以魏杜夔所制律吕校大乐,本音不和。后汉至魏,荀乃依\CJKunderwave{周礼},积粟以起度量,以度古器,长于古四分有馀,而夔据之,是以失韵。符于本铭。遂以为式,用之郊庙。”}}

{\cangkai\zihao{5}【评】故事的主角荀勖,是西晋的开国元勋,晋武帝太康时的中诏给予很高的评价,谓其“经识天序……兼博洽之才。久典内任,著勋弘茂,……宜登大位,毗赞朝政”(\CJKunderwave{晋书}本传)。他不仅是政治官僚,同时还是个绝顶聪明颇有学问的技术官僚。荀勖精通音乐,在朝兼掌乐事,对于当时国家雅乐正声的创制和实施,负有全责。在乐事方面,他是当行里手的专家,应该说是内行领导内行了。但是,由于古代国家的郊庙雅乐,是与祭祀礼制相结合的一种重要仪式,实际化为政治权力的象征,已是政治异化之物,荀勖“调律吕,正雅乐”,着眼点正在于政治礼仪制度的需要,而非音乐艺术的娱人功能。他曾上奏朝廷,建言“去奇技,抑异说”,因循旧章而反对改革,这样,不仅社会不会进步,科技和艺术也很难发展。这是一种观念性的误导。而且,作为政治家兼技术官僚的乐事领导,他从政治出发,嫉贤妒能,不能容忍真正艺术家的意见和批评。阮咸是竹林七贤中的著名音乐家,他妙赏音乐,似有“神解”,每次朝会演奏荀勖所订雅乐,外行人虽然分辨不出,但作为一个高明的音乐家,犹如一个交响乐队指挥,当他一听到些微的不谐和音的时候,音乐的耳朵就自然加以排斥,感到难受。他凭直觉感悟,“无一言直勖”,这纯粹是从艺术角度着眼。但作为音乐内行的领导,荀勖心胸比外行更狭隘,他挟政治之威,谗贬阮咸,实是形同流放。不过,“山外青山楼外楼”,后来出土文物的考古发现,证明了荀勖的错误和阮咸的正确。等到荀勖“伏阮神识”之时,音乐天才已是客死他乡,悲乎哀哉!}

\lettrine{20.2} 荀勖\myidx{荀勖}尝在晋武帝\myidx{司马炎}坐上\footnote{晋武帝:即司马炎,昭长子,字安世。炎于公元265年,废魏开晋,史称晋武帝。坐,通“座”。},食笋进饭,谓在坐人曰:“此是劳薪炊也\footnote{劳薪:析旧车轮车轴为柴火,因车轮运转劳苦,故称“劳薪”。}。”坐者未之信,密遣问之,实用故车脚\footnote{故车脚:即旧车轮。}。

{\cangkai\zihao{5}【评】荀勖不仅是个精解乐律的专家,更是个会享受生活的美食家。刘辰翁曾批云:“薪岂知劳,而烟气亦异邪?”刘氏虽然不敢肯定,但他的怀疑是有一定道理的。\CJKunderwave{隋书·王劭传}载,劭曾上表隋文帝请变火,其表有云:“昔师旷食饭,云是劳薪所爨。晋平公使视之,果然车辋。今温酒及炙肉,用石炭、柴火、竹火、草火、麻荄火,气味各不同。”看来,烹煮食物美味,不仅在于食物本身,同时又关系到烧煮食物的柴火。柴火性质不同,化为烟火的分子结构自然各异,不同的分子,一旦通过烟气进入食物,又会间接影响食物气味。这一分析是有一定科学依据的,而非仅是怀疑和猜测。比如用松木为薪,烟火中自带一股松脂味;用樟木为柴火,又会有一股清香之气。笔者少年时曾进山采樵,有一定实践体会,故记之以供参考。荀勖之言,体会细微,也可能是经验感悟之谈。}

\lettrine{20.3} 人有相羊祜父\myidx{羊衜}墓\footnote{相:占相,这里具体指风水之术。羊祜:魏晋间人。字叔子。参前\CJKunderwave{言语}第86则注。羊祜曾建平吴之策,是西晋开国元勋,官至平南大将军,卒后追赠太傅。据\CJKunderwave{晋书}祜传,父衜,上党太守。},后应出受命君\footnote{受命君:应天受命而为帝王君主。}。祜恶其言,遂掘断墓后,以坏其势。相者立视之,曰:“犹应出折臂三公\footnote{三公:古代朝廷的最高级别官员。魏晋间三公为太尉、司徒、司空,位同宰相。}。”俄而祜坠马折臂,位果至公。{\fzxk\zihao{6}\textcolor{red}{\CJKunderwave{幽明录}曰:“羊祜工骑乘。有一儿五六岁,端明可喜。掘墓之后,儿郎亡。羊时为襄阳都督,因盘马落地,遂折臂。于时士林咸叹其忠诚。”}}

{\cangkai\zihao{5}【评】古代中国是个泛宗教的国家,经过巫术的心理渲染,甚至木石也能成为人们的崇拜偶像,因此,形形色色的巫术迷信活动,广泛深入民心。不仅是不读书的人迷信,知书达礼的读书人也迷信。无论是穷人或阔佬,甚至是贵族政要,心田中均埋藏有迷信的温床。在故事中,羊祜对于风水相术,似信非信。据\CJKunderwave{晋书}本传,羊祜十二岁丧父,其父墓地,当然不是这个年幼孩童所挑选的,但后来相师却看出了这是一块龙脉,“后应出受命君”——也就是说,今后羊祜可能夺取天下而当皇帝。风水先生的话很可能让他遭遇化吉为凶的厄运,甚至可能遭到灭族之诛。因此,不管羊祜个人是否相信风水之命,他还是采取了“掘断墓后,以坏其势”的果断措施,以明自己忠心国家朝廷而绝无政治野心。这行动既是自愿的,更多的却是一肚皮的无奈。因此,虽然羊祜贵为晋景帝司马师的妻舅,但他一生为官,史称“立身清俭”、“贞悫无私”,政治上极其谨慎,可能与此风水龙脉故事有关。这又变坏事为好事,好与坏的转换,全在自我选择,这是实实在在的生活辩证法,而非风水迷信之命。}

\lettrine{20.4} 王武子\myidx{王济}善解马性\footnote{王武子:晋初王济,字武子。参前\CJKunderwave{言语}第24则注。解:精熟,很懂。}。尝乘一马,箸连钱障泥\footnote{箸:同“著”,披挂。连钱障泥:绣有连钱花纹的障泥。障泥,即马鞯,垫在马鞍之下垂覆马背两边以遮挡泥水的马具。},前有水,终日不肯渡。王云:“此必是惜障泥。”使人解去,便径渡\footnote{径:径直,喻动作之迅捷。}。{\fzxk\zihao{6}\textcolor{red}{\CJKunderwave{语林}曰:“武子性爱马,亦甚则(别)之。故杜预道王武子有马癖,和长舆有钱癖。武帝问预:‘卿有何癖?’对曰:‘臣有\CJKunderwave{左传}癖。’”}}

{\cangkai\zihao{5}【评】在自然世界的人与动物关系中,马经过驯养,与人关系尤为密切。马通人性,人通马性,关键在于人的认识和理解。史称王济“好弓马”,因其所好而生“马癖”,在与群马长期相处中,逐渐认识并理解了马的脾性和需求,此所谓“解马性”,也即是类似知己朋友之间的理解。马惜连钱障泥而不肯渡,岂非马与人一样而有爱美惜美之心乎?但是,世界上真正“解马性”者有多少呢?推而广之,对动物充满理解与同情者又有多少呢?可谓寥若晨星。当今世界,动物在人类的文明“枪炮”威胁下生活,种类迅速减少,灭种者成千上万,人类生存环境迅速恶化,谁之过乎?希望不仅是生物学家,而是包括每个普通人在内,都应探索动物之性,对动物充满爱心,与动物为友,和谐相处,从而建立一个更新更美好的动物世界。}

\lettrine{20.5} 陈述\myidx{陈述}为大将军\myidx{王敦}掾\footnote{陈述(?—322):字嗣祖,两晋之交颍川许昌人。大将军王敦辟以为掾。掾,官府属员。},甚见爱重。及亡,郭璞\myidx{郭璞}往哭\footnote{郭璞:字景纯,河东闻喜人。避永嘉乱南渡,曾任王敦记室参军,被王敦所害。参前\CJKunderwave{文学}第76则注。}之,甚哀,乃呼曰:“嗣祖,焉知非福!”俄而大将军作乱\footnote{俄而:不久。大将军作乱:晋元帝永昌元年(322),大将军王敦以清君侧为名,起兵鄂州(今湖北武昌),挥师沿江东下,攻陷石头,诛大臣,杀名士,囚元帝。晋明帝时再次起兵,卒于军中。},如其所言。{\fzxk\zihao{6}\textcolor{red}{\CJKunderwave{陈氏谱}曰:“述字嗣祖,颍川许昌人。有美名。”}}

{\cangkai\zihao{5}【评】在中国历史上,两晋之交的郭璞是著名的算卦大师,精通\CJKunderwave{易}理。同时,他又是玄学清谈家,熟悉\CJKunderwave{老}、\CJKunderwave{庄}自然之道。\CJKunderwave{易}强调“一阴一阳之谓道,……阴阳不测之谓神”(\CJKunderwave{易传·系辞}上),认为阴阳矛盾,相生相反,如\CJKunderwave{泰}卦的乐极生悲,\CJKunderwave{否}卦的否极泰来,否泰矛盾,相反相成,揭示了阴阳矛盾在运动中相互转换的神明变化。另外,老、庄之道的祸福倚伏之理,也强化了\CJKunderwave{易}卦阴阳矛盾转换的预测之学。正因精于\CJKunderwave{易}理,所以郭璞作为\CJKunderwave{易}卦大师,其预测吉凶常有言则中的之妙。且大将军王敦的狼子野心,他在平时的观察中,早已了然于胸。陈述病卒于乱前,不预乱事,是由凶趋吉之幸;自己将身陷乱中,则是由吉趋凶,不得善终。不久,郭璞果然因反对王敦叛乱被杀,不幸而言中。故其生前吊唁陈述,有“焉知非福”之叹,实是充满\CJKunderwave{易}卦阴阳矛盾智慧的辩证之言。}

\lettrine{20.6} 晋明帝\myidx{司马绍}解占塚宅\footnote{晋明帝(299—325):即东晋的第二个皇帝司马绍,字道畿。占塚宅:占卜墓地风水之吉凶。}。闻郭璞\myidx{郭璞}为人葬,帝微服往看\footnote{微服:穿便装。},因问主人:“何以葬龙角\footnote{龙角:堪舆家有关风水的术语,认为墓塚依山为势,整体如龙,棺材葬于龙鼻龙额则吉,葬于龙角龙眼则凶,可致灭族之诛。}?此法当灭族!”主人曰:“郭云此葬龙耳,不出三年,当致天子\footnote{当:将会。致:招至。}。”帝问:“为是出天子邪\footnote{为是:说的是。为,通“谓”。}?”答曰:“非出天子,能致天子问耳。”{\fzxk\zihao{6}\textcolor{red}{青鸟子\CJKunderwave{相冢书}曰:“葬龙之角,暴富贵,后当灭门。”}}

{\cangkai\zihao{5}【评】皇帝“解占塚宅”,迷恋于堪舆风水之术,在“不语怪力乱神”的传统士人看来,已是奇事;但是晋明帝为了与臣子郭璞一争高低,进一步微服私访,实地勘探风水,这又是奇中之奇;更重要的是,他对郭璞就阴宅安葬所称“当致天子”之说,发出了“此法当灭族”的严重威胁,这才是最具本质意义的思考。明帝明确把风水堪舆之术,与维护皇权联系了起来。幸亏主人聪明,善于应对,方才巧妙地躲过了一大劫难。明帝司马绍虽贵为元帝长子,但其母荀氏,却是燕代鲜卑人。所以大将军王敦蔑称其为“黄须鲜卑奴”,可见他身上有一半具有我国北方少数民族强悍的血统。其争强好胜的鲜明个性,或多少与此有关乎?}

\lettrine{20.7} 郭景纯\myidx{郭璞}过江\footnote{郭景纯:郭璞,字景纯,河东闻喜(今属山西)人。是两晋之交的著名文学家及\CJKunderwave{易}卦占筮大师。参前\CJKunderwave{文学}第76则注。江:特指长江。},居于暨阳\footnote{暨阳:县名,晋时属毗陵郡,治所在今江苏省江阴市长寿镇东南。},墓去水不盈百步\footnote{去水:距离水边。不盈:不满。},时人以为近水。景纯曰:“将当为陆\footnote{将当:将会。}。”{\fzxk\zihao{6}\textcolor{red}{\CJKunderwave{璞别传}曰:“璞少好经术,明解卜筮。永嘉中,海内将乱,璞投策叹曰:‘黔黎将同异类矣!’便结亲昵十馀家,南渡江,居于暨阳。”}} 今沙涨,去墓数十里皆为桑田。其诗曰:“北阜烈烈\footnote{阜:土山。烈烈:山高峻貌。},巨海混混\footnote{混混(ɡǔn ɡǔn滚滚):亦作浑浑,波翻浪涌貌。}。垒垒三坟\footnote{垒垒:重迭堆积貌。},唯母与昆\footnote{昆:兄。}。”

{\cangkai\zihao{5}【评】古人对于祖先阴宅墓地的选址,极其重视,常是不惜重金,请来风水先生勘探测定,以利于子孙后福。郭璞本身即是著名的堪舆风水先生。他选中长江南岸水边作为祖坟之地,其对江岸变化的预测,似是料事如神,人或以为是他精于堪舆风水之术的缘故。其实,其“术”之妙,并非来自鬼神迷信,而是有其原因:一是他精熟地理之学,实地勘察长江两岸因江水冲击而进退消长的变化;更重要的是他精于\CJKunderwave{易}理,熟悉阴阳矛盾相反相生的生活辩证法,明白白云苍狗、沧海桑田的运动变化之理。其准确预测,来自哲学之道及生活实践。}

\lettrine{20.8} 王丞相\myidx{王导}令郭璞\myidx{郭璞}试作一卦\footnote{王丞相:指王导。作:占卜。},卦成,郭意色甚恶\footnote{意色:神气脸色。恶:难看,不好。},云:“公有震厄\footnote{震厄:雷震之灾。按:据\CJKunderwave{易},震卦为雷,故云。}。”王问:“有可消伏理不\footnote{消伏:消除。理:方法。}?”郭曰:“命驾西出数里\footnote{命驾:命令驾车。},得一柏树,截断如公长,置床上常寝处\footnote{常寝处:常睡的地方。},灾可消矣。”王从其语。数日中,果震柏粉碎。子弟皆称庆\footnote{称庆:作揖道贺。}。{\fzxk\zihao{6}\textcolor{red}{王隐\CJKunderwave{晋书}曰:“璞消灾转祸,扶厄择胜,时人咸言京、管不及。”}} 大将军云:“君乃复委罪于树木\footnote{乃复:竟然。}!”

{\cangkai\zihao{5}【评】两晋之时,颇重\CJKunderwave{易}卦占筮之术,上自皇帝丞相,下至平民百姓,纷纷求卜问卦以测吉凶命运。作为中国古代最著名的\CJKunderwave{易}卦算命大师,当时向郭氏求筮问卦者不可胜数。据\CJKunderwave{晋书}诸史及诸笔记所载,似乎非常灵验,神妙得很。这一故事即是例子。王导作为东晋的开国元勋,曾多次问卦于郭璞。如\CJKunderwave{晋书}卷六五\CJKunderwave{王导传}载:“初,导渡淮,使郭璞筮之,卦成,璞曰:‘吉,无不利。淮水绝,王氏灭。’其后子孙繁衍,竟如璞言。”又\CJKunderwave{郭璞传}载,王导深重郭璞,引参己军事,为劝晋元帝开国江东,多次命璞筮卦劝进。于此可见,\CJKunderwave{易}筮在当时的思想影响及其鼓动民心的社会作用。郭璞作为一代大师,深明\CJKunderwave{易}筮的影响和力量,因此,除一般的“玩”卦以适应人们迷信心理外,更多的是一种沟通天人之际,以便配合政治改革的一种有效手段。当然,晋时也有士大夫不相信\CJKunderwave{易}卦之筮,如颜含就拒绝了郭璞“欲为之筮”的好意,明白地说:“年在天,位在人,修己而天不与者,命也;守道而人不知者,性也。自有性命,无劳蓍龟。”(\CJKunderwave{晋书}卷八八\CJKunderwave{孝友传})王彪之更在朝廷之上,反对任命官吏以“卜术得进”的做法(\CJKunderwave{晋书}卷七六\CJKunderwave{王彪之传})。但在有晋一代,这在士大夫中毕竟是少数,多数是在信与不信之间,\CJKunderwave{易}卦算命已成当时上流贵族社会的一种风流雅事。}

\lettrine{20.9} 桓公\myidx{桓温}有主簿善别酒\footnote{桓公:指桓温。主簿:中央或地方官府中的属吏,主管文书、印鉴等。别:鉴别。},有酒辄令先尝\footnote{尝:品尝。},好者谓“青州从事\footnote{青州从事:喻美酒。}”,恶者谓“平原督邮\footnote{平原督邮:喻劣酒。}”。青州有齐郡\footnote{青州:古州名,其州治东汉在临淄,西晋在东阳,东晋则于广陵侨置青州。青州下辖济南、平原、乐安、北海、东莱、齐郡。地处今山东省东部地区。齐郡:青州属郡,治所临淄(今山东淄博市东北)。},平原有鬲县\footnote{平原:郡国名。治所平原(今县西南)。平原下辖九县:平原、高唐、般、鬲、祝阿、东陵、湿阴、安德、厌次。};“从事”言到脐\footnote{“从事”言到脐:青州从事可以到下属齐郡巡察视事。“齐”与“脐”谐音,谓美酒入口,酒力下透肚脐。从事,即从事史,州府佐官。},“督邮”言在鬲上住\footnote{“督邮”言在鬲上住:平原督邮可巡视下属鬲县,故言“鬲上住”。“鬲”与“膈”谐音,谓劣质之酒,难以下咽,其酒力只能在横膈膜上停滞不下。督邮:官名,郡守佐吏。}。

{\cangkai\zihao{5}【评】在古代,“国之大事,唯祀与戎”(\CJKunderwave{左传·成公十三年}),祭祀礼仪必有酒供奉神鬼。酒之质量优劣,不仅关乎个人爱好,更进一步关乎天地神明。因此,古代鉴别酒的质量优劣的工作,就为人们所重视,桓温手下的主簿“善别酒”,也就成了为人所敬重的品酒师傅。他不仅辨别口味,更进一步从酒力在身体中的具体感受来鉴别其质量优劣。其真知灼见出于实践的体悟。同时,主簿说话诙谐风趣,谐音比喻,生动形象,给人以深刻的印象。其品酒之妙,更有在酒味之外者,快哉,快哉!}

\lettrine{20.10} 郗愔\myidx{郗愔}信道甚精勤\footnote{郗愔:字方回,东晋高平金乡(今属山东)人。鉴子,超父。仕至领徐、兖二州刺史、都督徐兖青幽及扬州之晋陵诸军事,后解军职,还为会稽内史。卒赠司空。参前\CJKunderwave{品藻}第29则注。信道:具体指郗愔信奉天师道教。精勤:精诚勤奋。},常患腹内恶\footnote{腹内恶:腹中不舒服。},诸医不可疗。闻于法开\myidx{于法开}有名\footnote{于法开:东晋名僧,与支道林齐名。始以义理称,后隐于剡县,专精医道。\CJKunderwave{高僧传}卷四有传。参前\CJKunderwave{文学}第45则注。},往迎之。既来,便脉\footnote{脉:把脉问诊。},云:“君侯所患\footnote{君侯:原指古代列侯,后引申为对于尊贵者的敬称。},正是精进太过所致耳\footnote{精进太过:修奉道教行为太过分。精进,佛教以布施、持戒、忍辱、精进、禅定、知慧为成佛阶梯,称“六度”。精进指不懈努力、求得正果。“精进”概念,两晋南北朝时佛、道并用。致:招致。}。”合一剂汤与之\footnote{合:调配。汤:汤药。},一服即大下\footnote{大下:大泻,拉稀。},去数段许纸如拳大\footnote{去:泻。数段许:约略有几段。许,在数量词后,表示约数。},剖看,乃先所服符也\footnote{符:即道教的符箓,也称“丹书”、“墨书”或“符字”,是一种似字似画特殊符号,道教以为服之可以治病。}。{\fzxk\zihao{6}\textcolor{red}{\CJKunderwave{晋书}曰:“法开善医术,尝行,莫投主人,妻产而儿积日不堕。法开曰:‘此易治耳。’杀一肥羊,食十馀脔而针之,须臾儿下,羊膋裹儿出。其精妙如此。”}}

{\cangkai\zihao{5}【评】两晋士人,多有随顺自然而超拔世俗之智者,但聪明人却同时又受某些宗教迷信意识的影响,信奉佛道,沉迷其中而无法自拔。郗愔其人,史称其“执德存正,识怀沈敏”,“忠于王室”而坚决抵制权臣桓温的野心(\CJKunderwave{晋书}卷六七),是个较为正直的有识之士。但他同时又醉心于黄老之术,迷信天师道的符箓,结果是适得其反,招来无妄之灾,由健康之体转致沉重腹病。其服汤药而“大下”,该幡然醒悟乎?难言哉!晋穆帝病重,曾请于法开救治。他著有\CJKunderwave{议论备豫方}一卷。但以其医术之精,却难救一国之愚者。近今鲁迅、郭沫若弃医学文,以救治国人之心,是否受此故事启迪,待考。}

\lettrine{20.11} 殷中军\myidx{殷浩}妙解经脉\footnote{殷中军:指东晋名士殷浩,字渊源。见前\CJKunderwave{政事}第22则注。经脉:中医术语,人体气血运行的经络血脉,引申为诊脉治病的医术。},中年都废。有常所给使\footnote{给使:供差遣的仆役下人。},忽叩头流血。浩问其故,云:“有死事\footnote{死事:关乎性命存亡之事。},终不可说。”请问良久,乃云:“小人母年垂百岁,抱疾来久\footnote{抱疾来久:抱病在身已有很长时间。};若蒙官一脉\footnote{官:徐震堮\CJKunderwave{校笺}附\CJKunderwave{世说新语词语简释}云:“门下及属吏从称府主。”这里特指殷浩。脉:诊脉,治疗。},便有活理,讫就屠戮无恨\footnote{讫:终了,完毕。}。”浩感其至性\footnote{至性:内在诚挚的心性,此指孝心。},遂令舁来\footnote{舁(yú鱼):抬。},为诊脉处方。始服一剂汤,便愈\footnote{愈:痊愈,恢复健康。}。于是悉焚经方\footnote{经方:指古代医书。}。

{\cangkai\zihao{5}【评】人生在世,孰能不病?因此,医生救死扶伤,治病救人,是极其重要的事业。人类要健康发展,文明要不断进步,都离不开医生、医学。殷浩为一代名医,本该引为骄傲,其为人治病,原属美善之事。但他却在一汤便愈之后,“悉焚经方”以示与医学决绝,这样的乖戾行为,又是为什么?我想,这是传统偏见在作祟。如司马迁\CJKunderwave{报任安书}所说:“文史星历近乎卜祝之间,固主上所戏弄,倡优畜之,流俗之所轻也。”自古医生列入方伎类,其历史地位更在“卜祝”之下。如韩愈所说:“巫医乐师百工之人,君子不齿。”(\CJKunderwave{师说})传统思想是重道德教化而轻科技医学及艺术。特别是在魏晋门阀社会中,殷浩出身贵族士人,高自门第,但出于一念之仁,因精于医术而为下人之母治病,反为医生俗事所累,以此“悉焚经方”,诀绝医门,以示摆脱世俗。呜呼!中国科技实业及医学发展,难有长足的进步,正与此传统偏见密切相关。宋刘辰翁评殷浩云:“诊之似达,焚方又隘,无益盛德。”其实,这岂止是关乎个人“盛德”之事,轻视医道,焚烧医书,并非小事,而是事关国家、民族能否健康发展的根本大事。必须从制度与观念上加以扭转改变。}





%%% Local Variables:
%%% mode: latex
%%% TeX-engine: xetex
%%% TeX-master: "../Main"
%%% End:
