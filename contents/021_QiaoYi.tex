%% -*- coding: utf-8 -*-
%% Time-stamp: <Chen Wang: 2025-12-07 12:30:06>

% ○ ◎ ‧ 「 」 『 』 々 ( ) “ ” ■ ^[一-龥]
% 【\([^】][^】][^】]+\)】 → {\\fzxk\\zihao{6}\\textcolor{red}{\1}}
% \(【评】.*\) → {\\cangkai\\zihao{5}\1}
% \(【题解】.*\) → {\\cangkai\\zihao{5}\1}
% 《\([^》]+\)》 → \\CJKunderwave{\1}
% ^\([0-9]+.[0-9]+\) → \\lettrine{\1}
% {\\fzxk\\zihao{6}\\textcolor{red}{[^o]*}}



\setlength{\parindent}{0pt}



\chapter{巧艺第二十一}



{\cangkai\zihao{5}【题解】 推文本原意,“巧艺”是一并列结构名词:“巧”谓技巧,如\CJKunderwave{孟子·离娄上}开篇所称“离娄之明,公输子之巧”,所谓“巧”,指的是设计发明之类的技巧;“艺”则谓技艺或艺术,如古代所称射、御、书、数以及建筑、绘画之类艺术。但若引而申之,也可把“巧艺”理解为偏正结构名词,主词是“艺”,“巧”则是形容词,以形容技艺之精巧,已达出神入化的境界。如此解释,虽非原意,但却与文本精神有暗中相通的一面。本篇共收故事14则,广泛涉及棋艺、建筑、书法和绘画等诸多艺术,从中透露出魏晋时代的艺术精神。如称围棋为“坐隐”,为“手谈”,说明了当时围棋运动的升华,已进入士人精神交流的新境界,而非纯粹之娱乐。由于当时玄谈及佛禅的影响,魏晋人评价人物已不同于汉儒,“徒贵形似”之类,为当时士人(如戴逵)所不取,他们欣赏的更多是淡泊功利、脱俗自然的内在神明之美,由此而逐渐引发了艺术精神与时俱进的变化。魏末荀勖画锺繇肖像,“衣冠状貌如平生”,“如平生”云者,即平生风神之谓,而非仅是外在衣冠状貌之似。发展到东晋顾恺之,则进一步脱略形似而强调传神写照的“迁得妙想”,重在内在风神之潇洒自然,而无关乎四体之妍蚩。这一强调以形写神的艺术思想,成为魏晋美学之主流,对后世传统美学精神影响至为深远。于此可见,读\CJKunderwave{世说新语}不可以小说家言而轻忽之,生动故事中自有大学问在。}

\lettrine{21.1} 弹棋始自魏,宫内用牀(妆)奁戏\footnote{弹棋:古代的一种博戏,唐以后失传。据柳宗元\CJKunderwave{序棋}称,棋盘“隆其中而规焉,其下方以直。置棋二十有四,贵者半,贱者半,贵曰上,贱曰下,咸自第一至第十二。下者二乃敌一,用朱墨以别焉。……戏者二人,则视其贱者而贱之,贵者而贵之,其使之击触也。”于此可知其大概。牀奁:诸本作“妆奁”,是。“牀”、“妆”形近而讹。妆奁,古代妇女的梳妆盒。}。{\fzxk\zihao{6}\textcolor{red}{傅玄\CJKunderwave{弹棋赋叙}曰:“汉成帝好蹴踘,刘向以谓劳人体,竭人力,非至尊所宜御。乃因其体作弹棋。今观其道,蹴踘道也。”按玄此言,则弹棋之戏,其来久矣。且\CJKunderwave{梁冀传}云:“冀善弹棋格五。”而此云起魏世,谬矣。}} 文帝\myidx{曹丕}于此戏特妙\footnote{文帝:指魏文帝曹丕。},用手巾角拂之\footnote{手巾:古代揩脸、擦手的手帕,常系腰间以作装饰用。拂:披、拨。},无不中。有客自云能,帝使为之。客箸葛巾角\footnote{葛巾:用葛布制作的头巾。},低头拂棋,妙逾于帝。{\fzxk\zihao{6}\textcolor{red}{\CJKunderwave{典论}帝自叙曰:“戏弄之事,少所喜,唯弹棋略尽其妙,少时尝为之赋。昔京师少工有二焉,合乡侯、东方世安、张公子,常恨不得与之对也。”\CJKunderwave{博物记}曰:“帝善弹棋,能用手巾角。时有一书生,又能低头以所冠葛巾角撇棋也。”}}

{\cangkai\zihao{5}【评】曹丕虽贵为帝王,但也是性情中人,其于琴棋书画,无不精通,而不以“技艺”贱之。弹棋为戏,一般人以手指弹棋,曹丕则特加钻研,在实践中积累技能,精益求精,能够不以手指拂击,而以巾角击棋中的,故称“于此戏特妙”。但是,艺无止境,强中还有强中手。其宾客能够不以手执角巾拂棋,而是头戴葛巾披拂中的。这就使我想起了京剧\CJKunderwave{泗州城},女演员张美娟以背后靠旗一一拨击来袭红缨枪,真有出神入化之妙。如果不是在长期的艺术实践中呕心沥血,付出艰辛劳动代价,又岂能百尺竿头,再进一步,而取得“妙逾于常”的惊人成就!棋艺虽小,通于大道,信然。}

\lettrine{21.2} 陵云台楼观精巧\footnote{陵云台:楼台名,在洛阳明光殿西。据传为魏文帝曹丕所建。“陵”通“凌”。楼观:观阁楼台。},先称平众木轻重,然后造构,乃无锱铢相负揭\footnote{锱铢:喻微小。古代六铢为一锱,四锱为一两。负揭:高下失衡。}。台虽高峻,常随风摇动,而终无倾倒之理。魏明帝\myidx{曹叡}登台\footnote{魏明帝:文帝子,名睿,字仲元。},惧其势危,别以大材扶持之,楼即颓坏。论者谓轻重力偏故也\footnote{轻重力偏:失去重心。}。{\fzxk\zihao{6}\textcolor{red}{\CJKunderwave{洛阳宫殿簿}曰:“陵云台上壁方十三丈,高九尺,楼方四丈,高五丈,栋去地十三丈五尺七寸五分也。”}}

{\cangkai\zihao{5}【评】此则故事,从楼台设计之巧,体现出中国古代建筑学的发展与进步。任何事物之巧妙,都是精心设计而锱铢必较。增之太重失衡,减之太轻亦危。总之,以多寡适中为是。魏明帝不明此理,“登台惧其势危”,而别以大木材相扶衬,因而造成楼台重心失衡,立即倾塌的严重后果。这就是外行领导内行而好心办坏事,能无惧乎!}

\lettrine{21.3} 韦仲将\myidx{韦诞}能书\footnote{韦仲将:韦诞字仲将,三国魏京兆杜陵人。魏时仕至光禄大夫。善书法,师邯郸淳。}。魏明帝\myidx{曹叡}起殿\footnote{起殿:建造宫殿。},欲安榜\footnote{榜:匾额。},使仲将登梯题之\footnote{题之:在榜上书写。}。既下,头鬓皓然\footnote{头鬓皓然:鬓发尽白。},因敕儿孙勿复学书\footnote{敕:告诫。}。{\fzxk\zihao{6}\textcolor{red}{\CJKunderwave{文章叙录}曰:“韦诞字仲将,京兆杜陵人。太仆端子。有文学,善属辞。以光禄大夫卒。”卫恒\CJKunderwave{四体书势}曰:“诞善楷书,魏宫观多诞所题。明帝立陵霄观,误先钉榜,乃笼盛诞,辘轳长絙引上,使就题之。去地二十五丈,诞甚危惧。乃戒子孙绝此楷法,箸之家令。”}}

{\cangkai\zihao{5}【评】韦诞善书,在魏时是个鼎鼎大名的书法家。但因此而受名所累。魏明帝犯了先安白榜的错误。但他却不负责任,于是错就错,令韦诞高空作业,在“去地二十五丈”而凌空作书题写匾额。估计韦诞患恐高症,但君命难违,以此,书毕而须发皆白,怎一个“恐”字了得!不过因此戒子孙“勿复学书”,则又多此一举。凌濛初评曰:“岂至学书者必遭此?”反驳甚是。韦诞因个人一时得失,而发此偏激毒誓,实际是当时士人轻书贱艺传统心理的一种潜意识的反应。}

\lettrine{21.4} 锺会\myidx{锺会}是荀济北\myidx{荀勖}从舅\footnote{锺会:字士季,繇子,毓弟。参前\CJKunderwave{言语}第11则注。荀济北:荀勖(xù序),入晋后封济北郡公,故称。从舅:堂舅。},二人情好不协\footnote{情好不协:感情不好。}。荀有宝剑,可直百万\footnote{直:通“值”,价值。},常在母锺夫人许\footnote{锺夫人:荀肸妻,勖母,锺繇堂侄女。}。{\fzxk\zihao{6}\textcolor{red}{\CJKunderwave{孔氏志怪}曰:“勖以宝剑付妻。”}} 会善书,学荀手迹,作书与母取剑\footnote{作书:写信。},仍窃去不还\footnote{仍:义同“乃”,于是,就。}。{\fzxk\zihao{6}\textcolor{red}{\CJKunderwave{世语}曰:“会善学人书,伐蜀之役,于剑阁要邓艾章表,皆约其言,令词旨倨傲,多自矜伐。艾由此被收也。”}} 荀勖知是锺,而无由得也,思所以报之\footnote{报之:报复他。}。后锺兄弟以千万起一宅,始成,甚精丽,未得移住。荀极善画,乃潜往画锺门堂作太傅\myidx{锺繇}形象\footnote{潜:暗中。作太傅形象:此指绘画锺繇肖像。锺繇,毓与会之父,在魏官至太傅,故称。},衣冠状貌如平生。二锺入门\footnote{二锺:指锺毓、锺会兄弟。},便大感恸,宅遂空废。{\fzxk\zihao{6}\textcolor{red}{\CJKunderwave{孔氏志怪}曰:“于时咸谓勖之报会,过于所失数十倍。彼此书画,巧妙之极。”}}

{\cangkai\zihao{5}【评】书、画艺术,本是高人雅事,但锺(会)、荀(勖)二人,却用之作俗,成为相互报复的工具,成了坏事。这不足为训。但从另一个角度来看,善学他人手迹,以及肖像绘画栩栩如平生,也可想象当时书、画艺术所达到的高度。如果书法艺术不被士人看重,锺会何必模仿名家手迹?如果不是经过千锤百炼,荀画锺繇,何能“衣冠状貌如平生”,以至于二锺兄弟,见肖像画而恸哭废居呢?荀勖之画,虽因肖像艺术的特殊要求,不废“衣冠状貌”之形似;但“如平生”三字,则又见出魏之士人绘画,极其重视人物的内在气韵神明的传达。若无内在神似之生气,仅是外在的“衣冠状貌”,又岂能有如此感人的艺术魅力!}

\lettrine{21.5} 羊长和\myidx{羊忱}博学工书\footnote{羊长和:羊忱,字长和,西晋末年泰山平阳人。参前\CJKunderwave{方正}第19则注。工书:善书法。},{\fzxk\zihao{6}\textcolor{red}{\CJKunderwave{文字志}曰:“忱性能草书,亦善行隶,有称于一时。”}} 能骑射\footnote{骑射:骑马射箭。},善围棋。诸羊后多知书,而射、弈馀艺莫逮\footnote{弈:弈棋,这里指下围棋。馀艺:其他技艺。}。

{\cangkai\zihao{5}【评】泰山平阳羊氏,是魏晋时高门士族。由于士族的家法传承,出于“世为冠族”的羊忱,“博学工书”,多才多艺,围棋手谈,无所不精。甚至骑马射箭等切合实用的激烈运动,也是擅名一时。据前\CJKunderwave{方正}第19则故事载,赵王伦行篡逆时,曾派使者追忱,忱“不暇被马”,“帖骑而避”,“矢左右发,使者不敢进”。在光溜溜的马背上,不仅稳帖,而且还能左右开弓,准确发射而不取使者性命。于此可见其骑射之精善。羊忱于永嘉乱中遇害,一个艺术全才,终被黑暗所吞噬。羊氏家族后人,多继承家法而在书法艺术上拓展。但作为高门望族之士,却大多养尊处优,精于思理而懒于行动,因而“射、弈馀艺莫逮”,也在意料之中。由此可以想象,读书学习搞艺术,甚至是骑马射箭下围棋,没有一定条件,当然不行;但是生活条件太好,于学问、艺术和运动,也不一定是好事。中国足球,长败不起,能说是条件太差、待遇太低所致吗?读此或可深省。}

\lettrine{21.6} 戴安道\myidx{戴逵}就范宣\myidx{范宣}学\footnote{戴安道:即戴逵,字安道,东晋谯国人。见前\CJKunderwave{雅量}第34则注。范宣:字宣子,陈留人。东晋儒学教育家,与范宁并称豫章二范。参前\CJKunderwave{德行}第33则注。就:跟随,师从。},{\fzxk\zihao{6}\textcolor{red}{\CJKunderwave{中兴书}曰:“逵不远千里往豫章诣范宣,宣见逵异之,以兄女妻焉。”}} 视范所为,范读书亦读书,范抄书亦抄书。唯独好画,范以为无用,不□(宜)劳思于此\footnote{不□(宜):“不”下字残,据诸本当作“宜”。劳思:费心劳神。}。戴乃画\CJKunderwave{南都赋图}\footnote{\CJKunderwave{南都赋图}:以汉张衡\CJKunderwave{南都赋}作为题材进行创作而成的画作。},范看毕咨嗟\footnote{咨嗟:叹赏。},甚以为有益,始重画。

{\cangkai\zihao{5}【评】与魏晋士人的“世尚\CJKunderwave{老}、\CJKunderwave{庄}”不同,范宣是个传授儒家经学的教育家,这在当时实属稀罕。作为传统儒者,他轻视绘画为末艺,也在情理之中。不过,当他看到戴逵所作\CJKunderwave{南都赋图}后,却幡然有悟,“以为有益”,并且开始重视绘画创作。这在讲究师法而不可越雷池一步的汉儒眼中,是不可想象之事。但范宣具体生活在魏晋时代,虚心接受弟子的批评,在艺术观念上来了个大改变。“弟子不必不如师”,这虽是后来唐代韩愈说的话,但论其言行,范宣早已启其端。另外,从戴逵作画说服“顽固”师长,又可见东晋绘画艺术魅力之一斑。}

\lettrine{21.7} 谢太傅\myidx{谢安}\footnote{谢太傅:指谢安。}云:“顾长康\myidx{顾恺之}画\footnote{顾长康:顾恺之,字长康,东晋晋陵无锡人。参前\CJKunderwave{言语}第88则注。其为人博学多才,又精绘画,是东晋最著名的画家兼理论家。画论著作有\CJKunderwave{论画}、\CJKunderwave{魏晋胜流画赞}、\CJKunderwave{画云台山记}等。},有苍生来所无。”{\fzxk\zihao{6}\textcolor{red}{\CJKunderwave{续晋阳秋}曰:“恺之尤好丹青,妙绝于时。曾以一厨画寄桓玄,皆其绝者,深所珍惜,悉糊题其前。桓乃发厨后取之,好加理复。恺之见封题如初,而画并不存,直云:‘妙画通灵,变化而去,如人之登仙矣!’”}}

{\cangkai\zihao{5}【评】赞美绘画艺术,在\CJKunderwave{巧艺}篇的14则故事中占9则,约占三分之二的篇幅,其中,荀勖画锺繇肖像因他事带出,占1则;戴逵画事占2则;而顾恺之一人之画艺及其有关评论共占6则,在全部绘画故事的9则中,占三分之二的篇幅。从这一数字统计中,可看出绘画艺术在魏晋“巧艺”领域中的重要地位,而顾恺之在魏晋画史之中,更是独占鳌头,而有“三绝”(才绝、画绝、痴绝)之称。他不仅创作了大量的人物肖像及神仙、佛像、禽兽和山水画,如为建康瓦棺寺作\CJKunderwave{维摩诘像}壁画,轰动一时;而且在创作实践基础上,提倡“迁想妙得”、“以形写神”等美学思想,影响了千年以后的传统美学精神。其影响与贡献,岂止是魏晋美术史,可说是彪炳千秋,垂之不朽。顾恺之在当时政治斗争中,偏向桓(温)党,而不同于谢安。但谢安不因政治之异而抹煞其艺术,其论顾恺之画,谓“有苍生以来所无”,确是的论,而非溢美之辞。本篇顾氏6则故事,由谢安一语作纲领提起,构成了连续的生动组画。}

\lettrine{21.8} 戴安道\myidx{戴逵}中年画行像甚精妙\footnote{行像:据\CJKunderwave{大唐西域记},当泛指佛像。}。庾道季\myidx{庾和}看之\footnote{庾道季:庾和小字道季,太尉亮子。参前\CJKunderwave{言语}第79则注。},语戴云:“神明太俗\footnote{神明:神情风韵。},由卿世情未尽\footnote{世情:世俗的七情六欲。}。”戴云:“唯务光\myidx{务光}当免卿此语耳\footnote{务光:传说夏末时的隐士。其事可参见\CJKunderwave{庄子·让王}篇及\CJKunderwave{史记·伯夷列传}。}。”{\fzxk\zihao{6}\textcolor{red}{\CJKunderwave{列仙传}曰:“务光,夏时人也。耳长七寸,好鼓琴,服菖蒲韭根。汤将伐桀,谋于光,光曰:‘非吾事也。’汤曰:‘伊尹何如?’务光曰:‘强力忍诟,不知其它。’汤克天下,让于光。光曰:‘吾闻无道之世,不践其土,况让我乎?’负石自沈于卢水。”}}

{\cangkai\zihao{5}【评】庾和谓戴逵画佛像“神明太俗”,这批评在精通佛学的士人看来,不无道理,因为佛家主“空”,要求超越世俗,脱离苦海。而戴氏佛像之画,却明显染有浓厚的世俗生活色彩,如庾氏所言,是由于作者“世情未尽”,未能破除我执与他执。但佛像艺术不是佛学思想的机械翻版。不热爱生活,缺乏感情的人,又怎能赋予艺术以不朽的生命活力呢?因此,“由卿世情未尽”的讥评,正可从反面看出戴画艺术生命之所在,其佛像艺术之精妙,正在努力促进外来佛教艺术的中国化,并具有强烈的时代生活气息。讥评之酷,反衬其“行像精妙”。另一方面,不管是批评或辩解,皆从“神明”入手,又可见魏晋人评画及其审美趣味之所在。}

\lettrine{21.9} 顾长康\myidx{顾恺之}画裴叔则\myidx{裴楷}\footnote{裴叔则:裴楷,字叔则,西晋河东闻喜人。仕至中书令。当时的清谈名家,以清通著称。八王之乱时被害。},颊上益三毛\footnote{颊:面颊,脸的两侧。}。人问其故,顾曰:“裴楷隽朗有识具\footnote{隽朗:风神高迈,气格爽朗。识具:见识才具。},正此是其识具。”看画者寻之\footnote{寻:谈寻,玩味。},定觉益三毛如有神明\footnote{定:确实。神明:神情气韵。},殊胜未安时\footnote{殊:更,甚。安:安置。}。{\fzxk\zihao{6}\textcolor{red}{恺之历画古贤,皆为之赞也。}}

{\cangkai\zihao{5}【评】史称“楷风神高迈,容仪俊爽,博涉群书,特精理义,时人谓之玉人”,又称“见裴叔则如近玉山,映照人也”(\CJKunderwave{晋书·裴楷传})。因此,画裴楷之难,不在其可见的外貌,而在其无形的内心的神明风采。裴楷颊上原无三毛,但顾恺之凭空构结,以艺术虚构来增添了颊上三毛,于是主人翁神采焕发而“隽朗识具”尽出,从而化无形为有形,完成了以形写神的生动形象刻画。从此,“颊上三毛”就成为我国人物画史上的千古不传之秘,引申为“颊上添毫”,以喻一切艺术的润饰传神。}

\lettrine{21.10} 王中郎\myidx{王坦之}以围棋是“坐隐”\footnote{王中郎:即王坦之,字文度。述子。太原晋阳人。曾官北中郎将,故称。参前\CJKunderwave{言语}第72则注。以:以为,认为。坐隐:围棋别名,即座中隐语,指对弈者不直接说话,而借棋暗示交流,成为无语的谈话。},支公\myidx{支遁}以围棋为“手谈”\footnote{支公:即支遁,字道林,东晋名僧。参前\CJKunderwave{言语}第63则注。手谈:围棋别名,意谓不以语言而以手交谈。与“坐隐”语异而义同。}。{\fzxk\zihao{6}\textcolor{red}{\CJKunderwave{博物志}曰:“尧作围棋,以教丹朱。”\CJKunderwave{语林}曰:“王以围棋为手谈,故其在哀制中,祥后客来,方幅会戏。”}}

{\cangkai\zihao{5}【评】谓围棋为“坐隐”、为“手谈”,的确是高人雅语,反映了魏晋士人对于围棋的认识。\CJKunderwave{尹文子}曰:“以智力求者喻于弈,弈进退取与、攻劫放舍在我者。”魏王粲\CJKunderwave{围棋赋序}又称:“清宁体道,稽谟元神,围棋是也。”中国是围棋的故乡,围棋不仅是游戏,更成为一种运动和艺术,是体道稽神的智慧结晶。以“坐隐”和“手谈”称围棋,可见围棋已不仅是技艺游戏,而是已进入了人类精神交流的文明境界。}

\lettrine{21.11} 顾长康\myidx{顾恺之}好写起人形\footnote{写:图画。人形:人物肖像。},{\fzxk\zihao{6}\textcolor{red}{\CJKunderwave{续晋阳秋}曰:“恺图写特妙。”}} 欲图殷荆州\myidx{殷仲堪}\footnote{殷荆州:即殷仲堪。曾官荆州刺史,故称。参前\CJKunderwave{德行}第40则注。},殷曰:“我形恶\footnote{形恶:形貌不佳。恶:丑陋。},不烦耳。”顾曰:“明府正为眼尔\footnote{明府:汉魏以来对州牧府尹的尊称。}。{\fzxk\zihao{6}\textcolor{red}{仲堪眇目故也。}} 但明点童子\footnote{童子:即“瞳子”,瞳孔。但:只要。点:点画。},飞白拂其上\footnote{飞白:原为书法之笔,其势飞动,枯墨露白。顾恺之移之于画,谓瞳孔上布白之笔。},使如轻云之蔽日\footnote{蔽日:刘注谓“日”一作“月”,似是,\CJKunderwave{晋书·顾恺之传}称引正作“蔽月”。}。”{\fzxk\zihao{6}\textcolor{red}{“日”一作“月”。}}

{\cangkai\zihao{5}【评】艺术真实源于生活真实,但又不是现实生活的机械翻版,而应该比生活更高、更真、更美。顾恺之明白这一美学原理。从外貌看,殷仲堪眇一目,形象欠佳。但顾氏的人物画,更重视的是以形传神,画出人物的内在精神风采。史称仲堪“能清言,善属文,……谈理与韩康伯齐名,士咸爱慕之”(\CJKunderwave{晋书·殷仲堪传})。可见其外貌虽然有欠缺,但内在神明风采仍然为士林追慕,值得顾恺之图画。作为艺术家,顾恺之很聪明,他掌握了艺术真实比生活真实更美的原则,稍加虚构点画,飞白布于瞳孔,犹如轻云披拂明月,爽朗清明,神采顿现而生机勃勃,这就是化丑为美,非高者不能言其妙。}

\lettrine{21.12} 顾长康\myidx{顾恺之}画谢幼舆\myidx{谢鲲}在岩石里\footnote{谢幼舆:即谢鲲,字幼舆,陈郡阳夏人。参前\CJKunderwave{文学}第20则注。}。人问其所以\footnote{所以:为什么,原因。},顾曰:“谢云:‘一丘一壑,自谓过之\footnote{“一丘一壑”二句:事载\CJKunderwave{晋书·谢鲲传},但早见于本书\CJKunderwave{品藻}第17则,云:“明帝问谢鲲:‘君自谓何如庾亮?’答曰:‘端委庙堂,使百僚准则,臣不如亮;一丘一壑,自谓过之。’”}。’此子宜置丘壑中\footnote{置:安放。}。”

{\cangkai\zihao{5}【评】人物、佛像是顾恺之绘画的强项。谢鲲是两晋之交的玄学清谈名家,为人旷达超逸,风神隽朗,其卒,晋明帝痛彻于心,与温峤书称其“远有识志,其言……味之不倦,近未易有也”(\CJKunderwave{晋书·王廙传})。如此人物,其“识致”风神,如何下笔,方可令观者“味之不倦”呢?作为政治家,谢鲲才干不如庾亮;但若论清谈思理,体道味玄,揭示人的自然本质,则谢鲲自谓过于庾亮。所言合乎实际。对于社会和时代,并不要求人人都是政治家,文化人也自有其贡献。在两晋之交的动荡年代里,士人命不保夕,因而回归自然,纵情丘壑,在山巅水涯中排拒社会的黑暗,重在尊重独立人格的思想自由。为了突出这一精神实质,顾恺之异想天开,把谢鲲的形象安放在山水自然背景当中,的确恰到好处,更为传神。这与前面画裴楷而颊增三毛,同是“迁得妙想”、“以形写神”美学精神的成功艺术实践,这比起那些拘泥于生活细节真实而斤斤计较一手一足的画家,其艺术相差不可以道里计。}

\lettrine{21.13} 顾长康\myidx{顾恺之}画人,或数年不点目精\footnote{目睛:指眼中瞳仁。}。人问其故,顾曰:“四体妍蚩\footnote{四体:四肢。妍蚩:美丑。},本无关于妙处,传神写照\footnote{写照:画像写真。},正在阿堵中\footnote{正:只,恰在。阿堵:指示代词,这,这个。}。”

{\cangkai\zihao{5}【评】中国人物画艺术之妙,不在枝节四体之妍蚩,而在于描绘关照全局整体的无形之内在神明。他画人物肖像,“或数年不点目睛”,自有大道理在。人们常说,眼睛是人类心灵的窗户。若是“窗户”不明,又岂能洞彻内心,画出人们的奕奕精神和生命活力!顾恺之久不点目睛,正是在长期的生活和创作实践中,积极思考和探索主人翁的神明风采之所在,由渐而顿,一旦启悟,立即心追笔落,神来之笔似从天降。“传神写照,正在阿堵中”,遂成千古艺坛佳话,垂之不朽,而影响深远,中国绘画艺术民族传统的形成,与其影响息息相关。传说晋哀帝兴宁间(363—365),建康瓦棺寺初建,请善男信女捐献,士大夫捐无过十万者,但顾恺之却提笔挥洒,认捐百万。他命僧备一壁,独闭户月馀,画维摩诘像,工毕将欲点眸子,乃谓寺僧曰:“第一日观者请施十万,第二日可五万,第三日任例责施。”及开户光照一寺,施者填咽,俄而得百万钱。其点眸子艺术,可谓出神入化,可惊可叹!}

\lettrine{21.14} 顾长康\myidx{顾恺之}道:“画‘手挥五弦’易,‘目送归鸿’难\footnote{“手挥五弦”二句:见于嵇康\CJKunderwave{兄秀才公穆入军赠诗十九首}之十五:“息徒兰圃,秣马华山。流磻平皋,垂纶长川。目送归鸿,手挥五弦。俯仰自得,游心太玄。嘉彼钓叟,得鱼忘筌。郢人逝矣,谁可尽言!”目送:目光追视。归鸿:春天北归的大雁。五弦:古代弹拨乐器,似琵琶,有五弦,故又称五弦琴。有直项、曲项之别。}。”

{\cangkai\zihao{5}【评】嵇康是竹林七贤的代表人物,是当时著名的玄学家、文学家。其“目送归鸿,手挥五弦。俯仰自得,游心太玄”之诗,正可见玄学思想对文学的影响,体道味玄不一定是全然消极的影响,也有利于文学高洁人格和深邃境界形成的另一面。史称“恺之每重嵇康四言诗,因为之图”(\CJKunderwave{晋书·顾恺之传})。说明嵇康的艺术精神,是薪尽火传,经顾恺之而继续在美术界熊熊燃烧。恺之所称:“画手挥五弦易,目送归鸿难”,正是从其创作实践中来的甘苦之言,从而启迪后人。其作画重在传神写照,而神来之笔,关键在点睛之眼神。手挥五弦,因有形可写,所以为易;目送归鸿,因是心想眼追,视之无形而意在象外,很难揣摩,此所以为难。所论正是顾氏审美理论的又一生动展现。}



%%% Local Variables:
%%% mode: latex
%%% TeX-engine: xetex
%%% TeX-master: "../Main"
%%% End:
