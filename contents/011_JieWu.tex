%% -*- coding: utf-8 -*-
%% Time-stamp: <Chen Wang: 2025-12-06 11:52:24>

% ○ ◎ ‧ 「 」 『 』 々 ( ) “ ” ■ ^[一-龥]
% 【\([^】][^】][^】]+\)】 → {\\fzxk\\zihao{6}\\textcolor{red}{\1}}
% \(【评】.*\) → {\\cangkai\\zihao{5}\1}
% \(【题解】.*\) → {\\cangkai\\zihao{5}\1}
% 《\([^》]+\)》 → \\CJKunderwave{\1}
% ^\([0-9]+.[0-9]+\) → \\lettrine{\1}
% {\\fzxk\\zihao{6}\\textcolor{red}{[^o]*}}


\setlength{\parindent}{0pt}



\chapter{捷悟第十一}



{\cangkai\zihao{5}【题解】 捷悟者,捷谓敏捷迅速,悟谓反应领悟,合而组成一个二言偏正结构词组,以“捷”修饰“悟”,用来指人的机智领悟、反应敏捷之言行。其实,按照佛学之说,人的开悟有“顿”有“渐”,各人情况不同,因而思维反应则或迟或速,不能以迟速快慢来判断智慧的高低。以文学创作的构思为例,或倚马可待,千言立就;或蹙眉断须,一句始成。故陆机\CJKunderwave{文赋}称:“或操觚以率尔,或含毫而邈然。”前句喻文思之敏捷,后句则状文思迟重。文思之或迟或速,俱可出经典作品。但在魏晋时代,士人精研才性,崇拜天才,因而特别欣赏思维捷悟之人,认为这是智慧超常的表现。本门七则故事,魏之杨修一人独占四则,是当然的主角。大概因为杨修后来惨死曹操屠刀之下,人们以此特殊形式来叹惜一代天才的英年早逝吧!}

\lettrine{11.1} 杨德祖\myidx{杨修}为魏武\myidx{曹操}主簿\footnote{杨德祖:杨修。主簿:府衙重要属官,掌管文书印鉴,魏晋时多总领府事。},时作相国门\footnote{相国:魏晋时对丞相的敬称。建安中,曹操任丞相。},始构榱桷\footnote{构:建造。榱桷(cuī jué崔决):椽子,安放在檩上架瓦的木条。},魏武自出看,使人题门作“活”字\footnote{题:题写。},便去。杨见,即令坏之。既竟\footnote{竟:完毕。},曰:“门中活,阔字。王正嫌门大也\footnote{王:曹操封魏王,故称。}。”{\fzxk\zihao{6}\textcolor{red}{\CJKunderwave{文士传}曰:“杨修字德祖,弘农人。太尉彪子。少有才学思干。魏武为丞相,辟为主簿。修常白事,知必有反覆教,豫为答对数纸,以次牒之而行,敕守者曰:‘向白事必教出相反覆,若按此次第连答之。’已而风吹纸次乱,守者不别而遂错误。公怒,推问,修惭惧;然以所白甚有理,终亦是修。后为武帝所诛。”}}

{\cangkai\zihao{5}【评】杨修是个绝顶聪明的人,智算无遗策,机变人难及,连曹操也自叹不如。建安中,临淄侯曹植“以才捷爱幸”。修、植二人惺惺相惜,俱以才华横溢、思维敏捷、悟性过人著名于世。曹操于建安十三年(208)始罢三公官,自立为丞相。相府新建,故事当发生于是年,正是曹操树立权威专擅朝廷而以人才为急之时。题门作“活”,不明言态度是非,实是一种无声的智力测验。测试以杨修拔头筹。修时为曹操主簿,史称“是时军国多事,修总知内外,事皆称意”(见\CJKunderwave{三国志·陈思王植传}裴注引\CJKunderwave{典略})。修见门上题字“即令坏之”,“即令”二字,见其悟性之高,反应之敏捷。题门作“活”为“阔”,是一种字谜。“王正嫌门大也”,判断准确,见其智慧。}

\lettrine{11.2} 人饷魏武\myidx{曹操}一杯酪\footnote{饷:赠。魏武:指曹操。酪:乳酪,一般为牛、羊奶的凝固结晶。},魏武啖少许\footnote{啖:吃。},盖头上题“合”字以示众。众莫能解,次至杨修\footnote{次:依次,顺序。},修便啖,曰:“公教人啖一口也\footnote{公:指曹操。},复何疑!”

{\cangkai\zihao{5}【评】乱世治国,唯贤是举,用人之际,人才唯先,才之表现,贵在智慧。故建安十五年(210)春曹操下求贤令曰:“自古受命及中兴之君,曷尝不得贤人君子与之共治天下者乎?……今天下尚未定,此特求贤之急时也。……若必廉士而后可,则齐桓其何以霸世!今天下得无有被褐怀玉而钓于渭滨者乎?又得无盗嫂受金而未遇无知者乎?二三子其佐我明扬仄陋,唯才是举,吾得而用之。”因此,曹操经常对自己的下属进行智力考察与培养。这次设谜,用的是拆字法,在智力竞赛中,仍是杨修一马当先,最早破解。“合”者,由“人”、“一”、“口”三字组合而成,综合理解,谜底即是每人一口酪也。出题者聪明,破题者更聪明。}

\lettrine{11.3} 魏武\myidx{曹操}尝过曹娥碑下\footnote{魏武:曹操。曹娥碑:原系东汉度尚为孝女曹娥所立之碑,碑石早已不存。现通行小楷本,后题书于东晋升平二年(358),未署书者姓名。},杨修\myidx{杨修}从\footnote{杨修:参本门第1则注。}。碑背上见题作“黄绢幼妇外孙䪠臼”八字\footnote{黄绢、幼妇、外孙、䪠臼:隐语谜面,其隐语谜底是“绝妙好辞”。},魏武谓修曰:“解不?”答曰:“解。”魏武曰:“卿未可言,待我思之。”行三十里,魏武乃曰:“吾已得。”令修别记所知\footnote{别记:另外记录。}。修曰:“黄绢,色丝也,于字为‘绝’;幼妇,少女也,于字为‘妙’;外孙,女子也,于字为‘好’;䪠臼,受辛也,于字为‘辞’:所谓‘绝妙好辞’也。”魏武亦记之,与修同,乃叹曰:“我才不及卿,乃觉三十里\footnote{觉:较,相差。}。”{\fzxk\zihao{6}\textcolor{red}{\CJKunderwave{会稽典录}曰:“孝女曹娥者,上虞人。父盱,能抚节按歌,婆娑乐神。汉安二年,迎伍君神,泝涛而上,为水所淹,不得其尸。娥年十四,号慕思盱,乃投爪(瓜)江,存其父尸曰:‘父在此,爪(瓜)当沉。’旬有七日,爪(瓜)偶沉,遂自投于江而死。县长度尚悲怜其义,为之改葬,命其弟子邯郸子礼为之作碑。”桉:曹娥碑在会稽中,而魏武、杨修未尝过江也。\CJKunderwave{异苑}曰:“陈留蔡邕避难过吴,读碑文,以为诗人之作,无诡妄也。因刻石旁作八字。魏武见而不能了,以问群僚,莫有解者。有妇人浣于汾渚,曰:‘弟(第)四车解。’既而祢正平也,衡即以离合义解之。或谓此妇人即娥灵也。”}}

{\cangkai\zihao{5}【评】这是一篇洋溢着虚构想象的微型小说。曹娥碑在会稽。杨修入曹操幕府,在建安年间,这一时期,南北敌国,如刘孝标所说:“魏武、杨修未尝过江也。”古代笔记小说,出于街谈巷语道听途说,或与事实出入,不必苛责古人。虽然故事非实,但无损其艺术想象之光彩。这是一个以文字拆合之法巧妙组成隐语的故事。在这场文字游戏中,作为挟天子以令诸侯的曹操,居然乐此不疲,可见时风众尚及士人对于这一特殊形式智力竞赛的重视。与杨修的才华智慧相较,曹操自叹不如。“我才不及卿,乃觉三十里”,这对杨修来说,表面是赞赏是光荣。不过,古代君臣之间,实是广义的主子与奴才关系。主子的智慧不及奴才,心里会好受吗?主子妒恨之心,总有爆发的一天,因此,杨修在领受赞赏荣光的同时,也早就埋下了一颗人生悲剧的种子。}

\lettrine{11.4} 魏武\myidx{曹操}征袁本初\myidx{袁绍}\footnote{魏武:曹操。袁本初:袁绍字本初。},治装\footnote{治装:治理装备。},馀有数十斛竹片\footnote{馀:剩馀。斛:古代量器。},咸长数寸,众云并不堪用\footnote{咸:皆,都。},正令烧除。太祖思所以用之,谓可为竹椑楯\footnote{谓:以为。竹椑楯:用竹片做成的椭圆形盾牌。盾牌是战场上的重要防护武器。楯,通“盾”。} ,而未显其言\footnote{未显其言:并不明说。},驰使问主簿杨德祖\myidx{杨修}\footnote{驰使:疾速派出使者。杨德祖:杨修字德祖。}。应声答之\footnote{应声答之:问话刚出,立马回答。},与帝心同。众伏其辩悟\footnote{辩悟:聪慧敏悟。}。

{\cangkai\zihao{5}【评】前几则属文字游戏之智,这一则是实用智慧的快速反应。面对使者突如其来的询问,“应声答之”四字,形容杨修不假思索的敏捷思维,快速的准确判断,的确是超人的智慧表现。生当斗争错综复杂的乱世,审时度势、准确判断的快速反应,常是脱离困境而通向成功的希望。“众伏(服)其辩悟”虽是现实存在;但主人态度如何,却是背后大有文章。魏武为何“驰使”问修?是真服其智慧捷悟,还是一种防患测试?或是二者兼而有之?值得深思。刘辰翁评曰:“以上四则皆德祖之所以可惜,所以致疑也,伤哉!”所言一语中的,点破了乱世奸雄的言外之意和险恶用心。}

\lettrine{11.5} 王敦\myidx{王敦}引车垂至大桁\footnote{引车:唐本作“引军”,是。垂至:将到。大桁:横跨建康秦淮河上的浮桥,在朱雀门外,又名朱雀桁。桁,通“航”。},明帝\myidx{司马绍}自出中堂\footnote{明帝:司马绍。元帝长子,东晋第二主。}。温峤\myidx{温峤}为丹阳尹\footnote{温峤:字太真。参\CJKunderwave{言语}第35则注。丹阳尹:丹阳郡长官。东晋时期丹阳属京畿地区。},帝令断大桁,故未断\footnote{故:故然,依旧。}。帝大怒瞋目\footnote{瞋:瞪眼。},左右莫不悚惧\footnote{悚惧:害怕恐惧。}。{\fzxk\zihao{6}\textcolor{red}{案:\CJKunderwave{晋阳秋}、邓\CJKunderwave{纪}皆云:敦将至,峤烧朱雀桥以阻其兵。而云未断大桁,致帝怒,大为讹谬。一本云“帝自劝峤入”,一本作“啖饮,帝怒”,此则近也。}} 召诸公来,峤至不谢\footnote{不谢:不肯谢罪请求原谅。},但求酒炙\footnote{炙:烤肉。}。王导\myidx{王导}须臾至\footnote{须臾:一会儿。},徒跣下地\footnote{徒跣:光脚。},谢曰:“天威在颜\footnote{天威在颜:皇上天颜震怒。},遂使温峤不容得谢。”峤于是下谢,帝乃释然\footnote{释然:消气。}。诸公共叹王机悟名言\footnote{机悟:机智敏悟。}。

{\cangkai\zihao{5}【评】故事发生在晋明帝太宁二年(324),王敦第二次举兵犯阙,直逼建康,上表称诛奸雄,以温峤为首。朝廷加峤中垒将军,领兵抗敌。\CJKunderwave{晋书}峤传称,叛军“奄至都下,峤烧朱雀桥以挫其锋,帝怒之,峤曰:‘今宿卫寡弱,征兵未至,若贼豕突,危及社稷,陛下何惜一桥?’”与\CJKunderwave{世说}所述不同,二说孰是,待考。不过,温峤御敌,一时未胜,帝怒其有损天威,则是事实。其实,胜败乃兵家常事,政治是斗争与妥协的统一。但当时明帝二十几岁,温峤三十几岁,二人英年负气,为小事而各不妥协。结果是内部自耗,岂有力量团结抗敌?大敌当前,这将如何了得?在这关键时刻,王导不愧是一名老政治家,宰相肚量,立即挺身而出,“徒跣以谢”的生动细节,“天威在颜”的诚挚之言,言外之意是温峤没有机会检讨认错。这就给明帝与温峤一个台阶下,矛盾自然消释。这里以年轻人的负气之盛,不肯妥协,来衬托王导的机敏与成熟,一个忧国忧民的老政治家形象,自然浮现眼前。另外,通过细节来捕捉人物心理的艺术刻画,也很成功。如明帝之“瞋目”,温峤之索酒炙,同王导之“徒跣”,无不内心如画。}

\lettrine{11.6} 郗司空\myidx{郗愔}在北府\footnote{郗司空:郗愔。官至都督徐兖青幽扬州之晋陵诸军事、徐兖二州刺史。卒赠司空,故称。北府:晋都建康,以京口为北府,历阳为西府,姑孰为南州。皆为军府要地。},桓宣武\myidx{桓温}恶其居兵权\footnote{桓宣武:桓温,卒谥宣武,故称。恶:厌恶。居兵权:掌握兵权。}。{\fzxk\zihao{6}\textcolor{red}{\CJKunderwave{南徐州记}曰:“徐州人多劲悍,号精兵。放(故)桓温常曰:‘京口酒可饮,箕可用,兵可使。’”}} 郗于事机素暗\footnote{事机:世事机宜。素暗:一贯糊涂。},遣笺诣桓\footnote{遣笺诣桓:送信给桓温。},方欲共奖王室,修复园陵\footnote{共奖王室,修复园陵:即恢复中原故国之意。共奖,一起辅助。王室,指朝廷。园陵,帝王陵墓。西晋帝王陵墓在洛阳,已沦丧。}。世子嘉宾\myidx{郗超}出行\footnote{世子:嫡长子。嘉宾:郗超小字嘉宾,愔长子。},于道上闻信至,急取笺,视竟,寸寸毁裂。便回还,更作笺\footnote{更:重新。},自陈老病,不堪人间\footnote{不堪人间:无法承担官府剧务。},欲乞闲地自养。宣武得笺,大喜,即诏转公督五郡、会稽太守\footnote{会稽:郡名,治所山阴(今浙江绍兴)。}。{\fzxk\zihao{6}\textcolor{red}{\CJKunderwave{晋阳秋}曰:“大司马将讨慕容暐,表求申勤(劝)平北将军愔及袁真等严办。(愔)以羸疾求退,诏大司马领愔所任。”案:\CJKunderwave{中兴书},愔辞此行,温责其不从,转授会稽。\CJKunderwave{世说}为谬。}}

{\cangkai\zihao{5}【评】\CJKunderwave{晋书·废帝海西公本纪}:太和二年“秋九月,以会稽内史郗愔为都督徐兖青幽四州诸军事、平北将军、徐兖二州刺史”。太和四年(368)四月,桓温率师北伐,温传称“平北将军郗愔以疾解职,又以温领平北将军、徐兖二州刺史”。据此,则故事发生于北伐前的春夏之交。郗愔解兵,桓温独掌军权,谁人可以牵制?愔子郗超,为温腹心。\CJKunderwave{晋书}超传称“温怀不轨,欲立霸王之基,超为之谋”。在政治上,愔忠王室,超则附温,二人立场相悖。愔不明温心,信致桓氏,欲“共奖王室,修复园陵”,一片丹心,耿如日月。而超则早知桓温异志,不欲老父对抗蹈险,故急取父笺“寸寸毁裂”而另代作笺。温之“大喜”,实喜超而非愔也。故事擅长心理分析,郗超父子及桓温之内心,如画托出。当时晋统治者内部权力斗争尖锐复杂,稍有不慎,或思虑迟钝,常罹祸乱,陷于不测。而如超辈,捷悟权变,游刃有馀,虽解父困于一时,实无助于国家之大计。悲哉!}

\lettrine{11.7} 王东亭\myidx{王珣}作宣武\myidx{桓温}主簿\footnote{王东亭:王珣,字元琳,导孙。封东亭侯,故称。参\CJKunderwave{言语}第102则注。宣武:桓温卒谥宣武。主簿:幕府重要官僚。掌文书印鉴。},尝春月与石头\myidx{桓熙}兄弟乘马出郊\footnote{石头:宋本注谓名遐,误,温六子熙、济、歆、祎、伟、玄,未闻有遐者。唐写本作熙,字伯道,与\CJKunderwave{晋书}温传合,是。}。时彦同游者连镳俱进\footnote{连镳俱进:并马前行。时彦:一时俊杰胜流。},{\fzxk\zihao{6}\textcolor{red}{石头,桓遐小字。\CJKunderwave{中兴书}曰:“遐字伯道,温长子也。仁(仕)至豫州刺史。”}} 唯东亭一人常在前,觉数十步\footnote{前:杨勇\CJKunderwave{校笺}作“后”,疑是。觉:较,相差。},诸人莫之解\footnote{莫之解:不能理解。}。石头等既疲倦,俄而秉舆\footnote{俄:一会儿。秉舆:袁本作“乘舆”,是。舆,车。},向诸人皆似从官\footnote{向:刚才。从官:侍从。},唯东亭弈弈在前\footnote{弈弈:当是“奕奕”。精神焕发的样子。},其悟捷如此。

{\cangkai\zihao{5}【评】在政治派系斗争中,王珣与郗超同属桓温一党,颇得桓温之赏识与宠任。\CJKunderwave{晋书}郗超传称桓温府中语曰:“髯参军,短主簿,能令公喜,能令公怒。”超髯,珣短故也。温为大司马在哀帝兴宁元年(363),时珣为其主簿。则故事当发生于兴宁年间,正是桓温势力腾腾上升之时。但在桓党中,珣颇有个性与自信,而不像其他“时彦”唯唯诺诺,只会拍桓家公子的马屁。石头兄弟,指桓温诸子。温府中诸位“时彦”,唯石头兄弟马首是瞻,围在府主公子身边打转,其“连镳俱进”而鞍前马后者,实是追随左右而不肯落后的讨好姿态也。而王珣则出自琅邪王氏家族,高自身份,不肯附俗。一旦石头兄弟改为乘车,则诸“时彦”形似奴仆侍从而大掉身价,而王珣则因独乘在前神采焕发而不失身份。此以“时彦”从游之愚,反衬王珣独悟机智,巧妙地维护了东晋第一贵族子弟的自尊人格。}




%%% Local Variables:
%%% mode: latex
%%% TeX-engine: xetex
%%% TeX-master: "../Main"
%%% End:
