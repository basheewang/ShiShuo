%% -*- coding: utf-8 -*-
%% Time-stamp: <Chen Wang: 2025-12-01 21:30:30>

% ○ ◎ ‧ 「 」 『 』 々 ( ) “ ” ■
% 【\([^】][^】][^】]+\)】 → {\\fzxk\\zihao{6}\\textcolor{red}{\1}}
% \(【评】.*\) → {\\cangkai\\zihao{5}\1}
% \(【题解】.*\) → {\\cangkai\\zihao{5}\1}
% 《\([^》]+\)》 → \\CJKunderwave{\1}
% ^\([0-9]+.[0-9]+\) → \\lettrine{\1}
% {\\fzxk\\zihao{6}\\textcolor{red}{[^o]*}}


\setlength{\parindent}{0pt}

\chapter{文学第四}



{\cangkai\zihao{5}【题解】文学,在本篇中的意蕴为文章博学。然而依所记104则故事,前65则为清谈博学的描述,而其后诸则所记,却有着与当今意义差不多的文学意味。如果说\CJKunderwave{世说}之编撰,沿用了儒家经典意义的说法,即\CJKunderwave{论语·先进}“文学,子游,子夏。”邢昺疏解为:“若文章博学,则有子游、子夏二人也。”那么在其实际记述中,则真实地反映了由文章博学的广义杂“文学”观念,到自觉理解作为艺术而独立存在的文学,那样一段重要的历史过程。}

{\cangkai\zihao{5}本篇前65则,是传统意义上的“文学”,记述才士们研讨经典的学术活动。但所称“经典”已不专是儒家的经典了,它包括了儒、释、道,最多的还是道家、佛家经典。魏晋学者重新诠释儒家,将追问宇宙论和人生观的学问,借重道家、佛家的说法,引入了自己的视野,深入细致地辨析一些关乎人生理解的重大概念,于是在“玄学”风尚中,演绎了一幕幕智慧人生的动人片段;中国思想、哲学的崭新面貌,定格在了清言玄理的一席席精细的谈吐之中。第66则后,以“文笔”为区分,将实用文章、审美的文学之作给予自觉的理解,内容包括书、表、诔文等应用文章之撰,诗、赋等艺术之作的鉴赏,而这些,多带着玄理思辩的时代特色和意蕴,以及对自然、人生、艺术审美的自觉,使得他们的评论、鉴赏、创作具有文学自觉时代的特征。总之,\CJKunderwave{文学}是魏晋哲学思辩、审美情趣、精神风貌的生动篇章,也是挥麈而谈、精苦辩难、飞扬文翰的才士长廊。}

\lettrine{4.1} 郑玄\myidx{郑玄}在马融\myidx{马融}门下\footnote{郑玄(127—200):字康成,东汉北海高密(今属山东)人。著名经学家,毕生勤学著述,授徒讲学。晚年被汉献帝征为大司农,后又被袁绍强征随军,未至而卒。},{\fzxk\zihao{6}\textcolor{red}{\CJKunderwave{融自叙}曰:“融字季长,右扶风茂陵人。少而好问,学无常师。大将军邓骘召为舍人,弃,游武都。会羌虏起,自关以西道断。融以谓古人有言:‘左手据天下之图,而右手刎其喉,愚夫不为。’何则?生贵于天下也。岂以曲俗咫尺为羞,灭无限之身哉?因往应之,为校书郎,出为南郡太守。”}} 三年不得相见,高足弟子传授而已\footnote{高足弟子:学问精深的优秀学生。}。尝算浑天不合\footnote{浑天:我国古代解释天体的一种学说。\CJKunderwave{晋书·天文志}:“天之形状似鸟卵,地居其中。天包地外,犹卵之裹黄也,圆如弹丸,故曰‘浑天’。”算浑天,为古代有关天文的算法之一。},诸弟子莫能解。或言玄能者,融召令算,一转便决\footnote{转:转动计算用具而推算。},众咸骇服\footnote{骇服:惊叹佩服。}。及玄业成辞归,既而融有“礼乐皆东”之叹\footnote{礼乐皆东:儒家的学问传到东面去了。礼乐代指儒学。马融为扶风茂陵人(今陕西),郑玄为高密人(今山东),学成而归,故曰。}。{\fzxk\zihao{6}\textcolor{red}{\CJKunderwave{高士传}曰:“玄字康成,北海高密人。八世祖崇,汉尚书。”\CJKunderwave{玄别传}曰:“玄少好学书数,十三诵\CJKunderwave{五经},好天文、占候、风角、隐术。年十七,见大风起,诣县曰:‘某时当有火灾。’至时果然,智者异之。年二十一,博极群书,精历数图纬之言,兼精算术。遂去吏,师故兖州刺史第五元先。就东郡张恭祖受\CJKunderwave{周礼}、\CJKunderwave{礼记}、\CJKunderwave{春秋传}。周流博观,每经历山川,及接颜一见,皆终身不忘。扶风马季长以英儒著名,玄往从之,参考同异。季长后戚,嫚于待士,玄不得见,住左右,自起精庐,既因绍介得通。时涿郡卢子幹为门人冠首,季长又不解剖裂七事,玄思得五,子幹得三。季长谓子幹曰:‘吾与汝皆弗如也。’季长临别执玄手曰:‘大道东矣,子勉之!’后遇党锢,隐居著述,凡百馀万言。大将军何进辟玄,乃缝掖相见。玄长八尺馀,须眉美秀,姿容甚伟。进待以宾礼,授以几杖。玄多所匡正,不用而退。袁绍辟玄,及去,饯之城东,欲玄必醉。会者三百馀人,皆离席奉觞,自旦及暮,度玄饮三百馀杯,而温克之容,终日无怠。献帝在许都,征为大司农,行至元城卒。”}} 恐玄擅名而心忌焉\footnote{擅名:享有名声。}。玄亦疑有追,乃坐桥下,在水上据屐\footnote{屐:木鞋,底有齿。}。融果转式逐之\footnote{式:即栻盘,刻有阴阳五行,天象历法等标记。上盘圆,象天,以枫木为之;下盘方,象地,以枣心木为之,转动上盘,观上下盘标记的对应,以推算阴阳吉凶。},告左右曰:“玄在土下水上而据木,此必死矣。”遂罢追,玄竟以得免\footnote{免:指免祸。}。{\fzxk\zihao{6}\textcolor{red}{马融海内大儒,被服仁义。郑玄名列门人,亲传其业,何猜忌而行鸩毒乎?委巷之言,贼夫人之子。}}

{\cangkai\zihao{5}【评】名师高傲,学生虔诚,这是一则有关师生教学相长的生动故事。汉代经学定于一尊,今文学派重义理阐释,古文学派重历史典章、名物训诂,二者门户宗派之争激烈。一直到东汉末年,今文学派虽立于学官而占据上风,但又因羼杂谶纬迷信弄得乌烟瘴气,这就给古文学派的挑战与兴盛带来了机遇。经古文学派因马融的有力参与,而“古学遂明”,马为一代名师。而郑玄见融前,曾受业于太学,师事京兆第五元先,通\CJKunderwave{京氏易}、\CJKunderwave{公羊春秋}、\CJKunderwave{三统历}、\CJKunderwave{九章算术},又师从张恭祖学\CJKunderwave{周官}、\CJKunderwave{礼记}、\CJKunderwave{左氏春秋}、\CJKunderwave{韩诗}、\CJKunderwave{古文尚书}等,遍读儒经,打破经今、古文学的界域,学术视野宽阔,基础雄厚。因山东学者已经“无足问者”,于是西赴关中,拜马融门下,其追求真知之精神可嘉可叹。马融身为外戚豪门,本自骄贵,又为当世名儒,生性奢华,“常坐高堂,施绛纱帐,前授生徒,后列女乐,弟子以次相传,鲜有入室者”(见\CJKunderwave{后汉书·马融传})。虽郑玄天才,亦三年未见师面,而由门弟子间接传授。但郑玄却仍日夜诵习,毫无倦怠。其积学储宝日久,终于一算惊服融门,脱颖而出。融“召见于楼上,玄因从质诸疑义,问毕而归”(见\CJKunderwave{后汉书·郑玄传}),因玄为山东高密人,故马融有“礼乐皆东”之叹。这话出自名重当世的大儒之口,郑玄作为超越师门的学者形象就跃然而出了。师不必贤于弟子,弟子不必不如师,此乃古今教学相长之通义。郑玄后来成为兼综经今、古文学的一代大师,同时又精通数学、物理等自然科学,绝非偶然。善教之师鼓励学生不固守师说之限,善学者则不仅继承,而更重超越与开拓,从而日新月异,推动学术不断健康发展。这是本则故事的启示。}

{\cangkai\zihao{5}本则后半所记,孝标已斥其为荒诞的“委巷之言”,后世研究者亦多非之。刘应登说:“师友之懿如此,而谓融忌其能,使人追杀之,有此理否?玄先疑其师追之,预坐桥下,融以其在土下水上,便以为死。皆谬乱之词。”余嘉锡亦谓:“此节盖采自\CJKunderwave{语林},见\CJKunderwave{御览}三百九十三,非义庆所杜撰也……此说为晋、宋间人所盛传。然马融送别,执手殷勤,有‘礼乐皆东’之叹,其爱而赞之如此,何至转瞬之间,便欲杀害!苟非狂易丧心,恶有此事?”诸说均理据可信。马、郑对比,抑扬之间,无非是将郑玄这位大师更加神化而已,文坛学界,为造神而造谣,可悲可叹!}

\lettrine{4.2} 郑玄\myidx{郑玄}欲注\CJKunderwave{春秋传}\footnote{郑玄:见前则。\CJKunderwave{春秋传}:\CJKunderwave{春秋经}是孔子在鲁国国史的基础上编撰的一部编年史,为儒家的经典之一。传,注解、阐释经义的文字。\CJKunderwave{春秋经}在汉代著名的有三传,\CJKunderwave{公羊传}、\CJKunderwave{穀梁传}、\CJKunderwave{春秋左氏传}。此指\CJKunderwave{春秋左氏传},简称\CJKunderwave{左传},为鲁国左丘明所作。},尚未成,时行与服子慎\myidx{服虔}遇宿客舍,先未相识,服在外车上与人说已注\CJKunderwave{传}意。{\fzxk\zihao{6}\textcolor{red}{\CJKunderwave{汉南纪}曰:“服虔字子慎,河南荥阳人。少行清苦,为诸生,尤明\CJKunderwave{春秋左氏传},为作训解。举孝廉,为尚书郎、九江太守。”}} 玄听之良久,多与己同。玄就车与语曰:“吾久欲注,尚未了。听君向言\footnote{向:刚才。},多与吾同。今当尽以所注与君。”遂为\CJKunderwave{服氏注}\footnote{\CJKunderwave{服氏注}:即\CJKunderwave{春秋左氏传解谊},今有辑本。}。

{\cangkai\zihao{5}【评】\CJKunderwave{春秋}三传,\CJKunderwave{公羊传}、\CJKunderwave{穀梁传}属今文经学,显于西汉;东汉之初,由于郑兴等一批名儒的努力,作为古文经学的\CJKunderwave{春秋左氏传}才得以渐渐受到学界的重视。随着古文经学的渐兴,\CJKunderwave{春秋左氏传}便为学者所关注。郑玄志在“思整百家之不齐”(\CJKunderwave{后汉书·郑玄传}),让争议纷纭的经学有所指归,自然也会对\CJKunderwave{左传}着意关注。以他的学养和精勤,治\CJKunderwave{左传}必有所成。服虔是关注\CJKunderwave{左传}的学者之一,“少以清苦建志,入太学受业。有雅才,善著文论”(\CJKunderwave{后汉书}本传),为\CJKunderwave{左传}精思不辍,有自己的建树。本则围绕\CJKunderwave{左传}研究,记录了学者间的动人一幕。当郑玄知道服虔注\CJKunderwave{左传}有相当基础和程度,且很多想法与己相通时,就把自己的思想成果倾数赠予,以成就服虔对\CJKunderwave{左传}的深入研究,这种胸襟和风范是撼动人心的。足见追求学术真谛、弘扬大道,是学人的品德和良知,这和以学争名致禄的汉代经生的功利目的,形成了鲜明的对比,同时也沉实地注释了成为大师所必需的基本品质。本则记述虽仅为片段,却活画出了一个大师的宽阔胸怀和动人的人格形象。}

\lettrine{4.3} 郑玄\myidx{郑玄}家奴婢皆读书\footnote{郑玄(127—200):字康成,东汉北海高密(今属山东)人。著名经学家,毕生勤学著述,授徒讲学。晚年被汉献帝征为大司农,后又被袁绍强征随军,未至而卒。}。尝使一婢\footnote{使:使唤。},不称旨\footnote{称旨:符合意图。},将挞之\footnote{挞:鞭打。}。方自陈说\footnote{方自:还在。陈说:陈述原因、分辩。},玄怒,使人曳著泥中\footnote{曳:拖。}。须臾,复有一婢来,问曰:“胡为乎泥中\footnote{“胡为乎”句:怎么会在泥水中?\CJKunderwave{诗经·邶风·式微}句。}?”{\fzxk\zihao{6}\textcolor{red}{卫\CJKunderwave{式微}诗也。毛公曰:“泥中,卫邑名也。”}} 答曰:“薄言往愬,逢彼之怒\footnote{“薄言”句:去向他述说,恰赶上他发怒。\CJKunderwave{诗经·邶风·柏舟}句。刘孝标注谓两诗为\CJKunderwave{卫风},盖别有所据,今本\CJKunderwave{诗经}皆为\CJKunderwave{邶风}。}。”{\fzxk\zihao{6}\textcolor{red}{卫、邶\CJKunderwave{柏舟}之诗。}}

{\cangkai\zihao{5}【评】郑玄不乐仕途,归家著述讲学,本传一则说“学徒相随已数百千人”,再则说“自远方至者数千”,几十年间,门庭殷盛,其嘉惠后学,不问可知。有趣的是在如此师门,连家里的奴婢都受到沾溉,诵习\CJKunderwave{诗}、\CJKunderwave{书}。更了不起的是,这些奴婢不是顺口诵说几句诗文,而是活学活用,在这日常生活的场景中,引\CJKunderwave{诗}为说,一问一答,恰到好处地抒情达意。此一情景中的对话内容本身就富有喜剧性,又加之出自两个活泼婢女之口,声吻惟妙惟肖,让人读来忍俊不禁。由此可见,郑玄不仅是一代经学宗师,博学善教,而且也是一位热心而又平民化的教育家,不弃贫贱,家里的奴婢皆可读书受教育。这种惠及奴婢的教育实践,打破身份界限,已经超越了祖师孔子“有教无类”的境界。以上三则,从不同侧面,描述了一代学术大师的形象。}

\lettrine{4.4} 服虔\myidx{服虔}既善\CJKunderwave{春秋}\footnote{服虔:见本篇2刘孝标注。善:擅长。},将为注,欲参考同异\footnote{参考:比较考察。同异:相同或不同的见解。此偏指异。},闻崔烈\myidx{崔烈}集门生讲传,{\fzxk\zihao{6}\textcolor{red}{挚虞\CJKunderwave{文章志}曰:“烈字威考,高阳安平人,骃之孙,援之兄子也。灵帝时,官至司徒、太尉,封阳平亭侯。”}} 遂匿姓名,为烈门人赁作食\footnote{赁:受人雇佣。}。每当至讲时,辄窃听户壁间。既知不能逾己,稍共诸生叙其短长\footnote{稍:渐渐。共:与。}。烈闻,不测何人,然素闻虔名,意疑之。明蚤往\footnote{蚤:通“早”。},及未寤\footnote{及:趁着。寤:睡醒。},便呼:“子慎!子慎!”虔不觉惊应,遂相与友善。

{\cangkai\zihao{5}【评】服虔注\CJKunderwave{左传},可谓至精至慎,非博访通人行家、尽量穷尽搜集便不轻易写定。其寻访不惜屈苦自己,匿名受赁,为人当炊事伙计,这里虽是暗访了一位不如己者,但其诚恳、谨慎、认真的态度却是至为动人的。参见本篇第2则,郑玄以为服虔理解\CJKunderwave{左传},与自己的想法多所吻合,而将其所注尽与服虔。它们都表达了这些真正学人认真求实,绝不欺世盗名的优良风范。因此他们的学问也便可靠而被人乐于接受。服虔成\CJKunderwave{春秋左氏传解谊},在当时就广有影响,直到南朝刘宋时,范晔作\CJKunderwave{后汉书}还说“行之至今”。只是到唐代孔颖达作\CJKunderwave{春秋左传正义},专用西晋杜预的注解,其后\CJKunderwave{正义}行而诸本晦,服虔注才逐渐亡佚,今仅有辑佚本。}

\lettrine{4.5} 锺会\myidx{锺会}撰\CJKunderwave{四本论}始毕\footnote{锺会:锺毓、锺会:魏锺繇二子,颍川长社人。毓,字稚叔,官至廷尉、青州刺史,督徐州、荆州军事,死后追赠车骑将军,谥惠侯。会,字士季,官至司徒。受命伐蜀,蜀破,欲率军谋反,内部先乱,为乱军所杀。魏以谋反论其罪。令誉:美好的声誉。\CJKunderwave{四本论}文章篇名,见刘孝标注。才性:才指治国用兵之术;性指仁孝道德。},甚欲使嵇公\myidx{嵇康}一见\footnote{嵇公:嵇康,嵇康(223—262):三国时谯郡铚(今安徽亳县)人。“竹林七贤”之一。曾任中散大夫,故称嵇中散。当时著名思想家、文学家、清谈名家。因其主张越名教而任自然,抨击礼法之士,不与司马氏统治集团合作,盛年被杀。}。置怀中,既定\footnote{既定:或以为是“既诣宅”的脱文形误。},畏其难\footnote{难:驳难。},怀不敢出,于户外遥掷,便面急走\footnote{面:遮住脸。走:跑。}。{\fzxk\zihao{6}\textcolor{red}{\CJKunderwave{魏志}曰:“会论才性同异,传于世。四本者:言才性同,才性异,才性合,才性离也。尚书傅嘏论同,中书令李丰论异,侍郎锺会论合,屯骑校尉王广论离。文多不载。”}}

{\cangkai\zihao{5}【评】余嘉锡先生\CJKunderwave{笺疏}曰:“南齐王僧虔\CJKunderwave{诫子书}云:‘\CJKunderwave{才性四本}、\CJKunderwave{声无哀乐},皆言家口实。如客至之有设也,汝皆未经拂耳瞥目,岂有庖厨不修,而欲延大宾者哉?’清谈之重\CJKunderwave{四本论}如此,殆如儒佛之经典矣。”作为言家口实的\CJKunderwave{才性四本},涉及玄理,是当日才士清谈的重要论题,也是试辨才士学问、才性的试金石。这是就当时一般意义而言。可对锺会与嵇康,这一话题的内涵就没那么简单了。依陈寅恪先生说:“当魏末西晋时代即清谈之前期,其清谈乃当日政治上之实际问题,与其时士大夫之出处进退至有关系,盖借此以表示本人态度及辩护自身立场者。”(见\CJKunderwave{陶渊明思想与清谈之关系})这样说来,本则既记录了锺会的才性,也记录了他构陷嵇康的一个插曲。}

{\cangkai\zihao{5}就文章、博学的意义说,锺会有才学,十四岁即饱读儒经、史书,十五入太学问四方奇文异训,弱冠与山阳王弼并知名。他的论才性合,虽今不见其文,以其“博学精练名理”,颇获时誉推之(见\CJKunderwave{三国志·锺会传}),当可以与嵇康论辩。然而,他在与嵇康比论名理的面上文章之外或另有一层意思。司马氏以名教为纲领,以“孝”治天下,锺会的“论合”,就根本意义说,无非主张要坚持以道德名教与治国用兵之术相统一的人才标准。作为参与司马氏集团权谋机要的心腹,锺会此论是在为司马氏夺权造舆论。嵇康是主张“越名教而任自然”的。其渊源是\CJKunderwave{老}、\CJKunderwave{庄}体无的一套理论。心体乎无,存乎道,傲然忘贤愚是非,情任自然,越名任心是他的人生体悟,见诸其\CJKunderwave{声无哀乐论}等文,才(声)、性(心)明为二物,也是他的一贯主张。所以他“非汤武而薄周孔”,这明显是才、性离、异派,适与名教仁孝等儒家学说大相径庭,也就是与当权的司马氏相违背。嵇康高傲,其学问、主张乖违司马氏,更瞧不起“名公子”锺会。锺会便对这位大名士耿耿于怀,并不断用心构陷,寻找口实,他将论著遥投与嵇康回身便跑,或并不是因其欲与嵇康讨论学术而“畏其难”,更深层的原因恐如陈寅恪先生说:是“别有企图”。曾主张才性“异”的中书令李丰、主张“离”的屯骑校尉王广,都先后被司马氏所诛杀,就很能说明问题。}

{\cangkai\zihao{5}锺会四十败亡,观其一生唯恃聪明而乏敦朴厚道,成败荣辱皆由此,本则所记这一插曲,或亦为其自恃聪明之举。王世懋评点:“令人畏至如此,那得不为所中”,觉得嵇康令锺会如此畏惧,被锺会暗算,即势所必然。\CJKunderwave{世说}将这一故事选入\CJKunderwave{文学}门,还是看重魏晋人物文章、博学的一面。}

\lettrine{4.6} 何晏\myidx{何晏}为吏部尚书\footnote{何晏:何晏字平叔,\CJKunderwave{三国志}作何进孙。少有才,正始初为曹爽所用,名盛于天下。好老庄,与夏侯玄、王弼等倡导玄学,开魏晋清谈之风。},有位望\footnote{位望:地位、名望。},时谈客盈坐\footnote{时:时常。谈客:清谈的客人。时风尚清谈,常聚客而谈。},{\fzxk\zihao{6}\textcolor{red}{\CJKunderwave{文章叙录}曰:“晏能请言,而当时权势,天下谈士,多宗尚之。”\CJKunderwave{魏氏春秋}曰:“晏少有异才,善谈\CJKunderwave{易}、\CJKunderwave{老}。”}} 王弼\myidx{王弼}未弱冠往见之\footnote{未弱冠:不满二十岁。\CJKunderwave{礼记·曲礼}:“二十曰弱,冠。”古代男子年满二十行“冠礼”,表示成年。},晏闻弼名,{\fzxk\zihao{6}\textcolor{red}{\CJKunderwave{弼别传}曰:“弼字辅嗣,山阳高平人。少而察惠,十馀岁便好\CJKunderwave{庄}、\CJKunderwave{老}。通辩能言,为傅嘏所知。吏部尚书何晏甚奇之,题之曰:‘后生可畏。若斯人者,可与言天人之际矣!’以弼补台郎。弼事功雅非所长,益不留意,颇以所长笑人,故为时士所嫉。又为人浅而不识物情。初与王黎、荀融善,黎夺其黄门郎,于是恨黎,与融亦不终好。正始中以公事免。其秋遇疠疾亡,时年二十四。弼之卒也,晋景帝嗟叹之累日,曰:‘天丧予!’其为高识悼惜如此。”}} 因条向者胜理语弼\footnote{条:一一陈述。向:方才。胜理:精妙的玄理。}曰:“此理仆以为理极,可得复难不?”弼便作难,一坐人便以为屈,于是弼自为客主数番\footnote{客主:清谈时,辩难的“客”、“主”双方。“主”提出问题,陈述观点,“客”行问难。自为客主,即自陈主张,自己问难解答,以使玄理充分阐发。番:指论辩时的一个回合。},皆一坐所不及。

{\cangkai\zihao{5}【评】史称王弼“好论儒道,辞才逸辩”(见\CJKunderwave{三国志·锺会传}附)。本则即记录了尚未弱冠的天才少年,在当时清谈宗主何晏的客厅上,辩屈众人的精彩一幕。在这里,他不但超越了何晏认为的义理极限,而且愈辩愈畅,竟至于“自为客主”,数番辩难,所陈玄理,步步深入,充分阐发。余嘉锡先生\CJKunderwave{笺疏}分析时人裴徽、管辂对何、王的评价而推论:“盖晏之为人,妙于言而不足于理,宜其非王弼之敌矣。”当时“莫不宗尚玄言,唯王辅嗣妙得虚无之旨”(\CJKunderwave{经典释文序录})。王弼精于思考,擅长体悟,一切从高人一筹的哲学本根说来,这便是其超拔于时人之上,辩论无敌的真实缘故。}

{\cangkai\zihao{5}本则描摹了一位学博而思精的天才少年的思辩风采。}

\lettrine{4.7} 何平叔\myidx{何晏}注\CJKunderwave{老子}\footnote{何平叔:何晏,字平叔,何晏字平叔,\CJKunderwave{三国志}作何进孙。少有才,正始初为曹爽所用,名盛于天下。好老庄,与夏侯玄、王弼等倡导玄学,开魏晋清谈之风。\CJKunderwave{老子}:书名。即\CJKunderwave{道德经},主张自然无为。今存河上公及王弼两种注本。1973年长沙马王堆汉墓出土有帛书\CJKunderwave{老子}甲、乙本。},始成,诣王辅嗣\myidx{王弼}\footnote{王辅嗣:王弼,字辅嗣,见前则。}。见王注精奇,乃神伏曰\footnote{神伏:从心底里服气。}:“若斯人,可与论天人之际矣\footnote{天人之际:天道与人事的相互关系。}!”因以所注为\CJKunderwave{道德二论}\footnote{\CJKunderwave{道德二论}:该文\CJKunderwave{三国志·曹爽传}但云\CJKunderwave{道德论},其文已佚。余嘉锡\CJKunderwave{笺疏}谓,上篇论道,下篇论德,故为\CJKunderwave{二论}。}。{\fzxk\zihao{6}\textcolor{red}{\CJKunderwave{魏氏春秋}曰:“弼论道约美不如晏,然自然出拔过之。”}}

{\cangkai\zihao{5}【评】何晏、王弼是开创正始玄风的领袖人物,共同“祖述老、庄。立论以天下万物以‘无’为本”(\CJKunderwave{晋书·王衍传})。然而,两人的才情却有高下之分。}

{\cangkai\zihao{5}他们之前,是汉儒家天下。作为主流意识被独尊的儒家,原本更侧重“礼”、“义”,亦即看得见、摸得着的经验层面上的伦理纲常与治国之术。其理论不但不容置疑,日趋僵化,又在利禄诱导下,愈搞愈世俗、愈芜杂烦琐,已极大地束缚了个体的人和人性发展。而其理论本身,对“性与天道”这人生的终极、本原问题,不能更深刻、透辟地解释。正始时,人们便重又发掘了\CJKunderwave{老}、\CJKunderwave{庄}的价值。\CJKunderwave{老子}是从宇宙的根源来阐释“性与天道”,亦即说明人生、政治与天道的对应关系;从生生之原的宏观出发对人生的种种具体微观存在加以审视,努力回答“天人之际”的大问题。其宇宙观便是“道”,“道”体本“无”。由“儒”而“道”,由具体“礼”、“义”到本“无”的玄理探究,这种思路的转换,更深刻的内涵便是从形而下的经验层面的积累、确证到形而上之玄理的体悟与思辩的转换。但何晏在玄理体悟与思辩的才情上逊于王弼。这点在今存的材料——两人对\CJKunderwave{论语}发表的意见中即看得出来。他们援\CJKunderwave{易}、\CJKunderwave{老}而释孔子。何晏的\CJKunderwave{论语集解},试图从宇宙的深微大道去解释儒家的人生义理,而在理论思辨方面不免拘泥于汉儒;王弼的\CJKunderwave{论语释疑}将礼乐之本引向了“道”,把儒家的“天”也转换为“则天成化,道同自然”的“天道”,使儒家人生具体伦理规则的依据,落在了抽象玄理的“道”上。从这里就可以见出,王弼比何晏更具抽象玄理思辩的本领,其本传说他:“论道附会文辞不如何晏,自然有所拔得多晏也。”王弼注\CJKunderwave{易}、\CJKunderwave{老}、\CJKunderwave{论语}独见其理论思辩的天才和气魄,其关于\CJKunderwave{周易}的研究,一扫汉儒的象、数推衍,变而为思达天人之际,追问宇宙本原与人生、政治关系的“义理”新学。这都是王弼的超拔处。}

{\cangkai\zihao{5}开玄学风气,将儒学正统引向更富于思辩、更能以简驭繁,更具通识的精神独立与自由的境地,是何、王的共同志趣,而\CJKunderwave{老子}又是正始玄学的最重要的理论武器,所以,当何晏见到王注出拔精奇,非自己见识所可比的时候,便主动收起了己注。于此可见何晏作为一代玄学宗师的学术气量。}

{\cangkai\zihao{5}王弼注\CJKunderwave{老子}确不同凡响。早期汉河上公与王弼之注成为后来解\CJKunderwave{老}的祖本。河上公本近民间系统,文句简古;王弼注本,为文人系统,文笔晓畅,后世解\CJKunderwave{老}者纷纭众多,然多“依违于河上、王弼二本之间”(见朱谦之\CJKunderwave{老子校释}),可见王注的水平,也可见当时何晏的“神伏”和慧眼。}

\lettrine{4.8} 王辅嗣\myidx{王弼}弱冠诣裴徽\myidx{裴徽}\footnote{王辅嗣:王弼见。弱冠:二十岁。诣:拜访。},{\fzxk\zihao{6}\textcolor{red}{\CJKunderwave{永嘉流人名}曰:“徽字文季,河东闻喜人,太常潜少弟也。仕至冀州刺史。”}} 徽问曰:“夫无者\footnote{无:老子哲学的概念。},诚万物之所资\footnote{资:凭借。},圣人莫肯致言\footnote{圣人:指孔子。莫肯:不愿。},而老子申之无己\footnote{老子:即老聃,春秋时楚国人,曾为周藏书室史官,作\CJKunderwave{道德经}五千言(即\CJKunderwave{老子})。老子与庄子并被视为道家始祖。申:申述,阐说。},何邪?”{\fzxk\zihao{6}\textcolor{red}{\CJKunderwave{弼别传}曰:“弼父为尚书郎,裴徽为吏部郎,徽见异之,故问。”}} 弼曰:“圣人体无,无又不可以训\footnote{训:解释。},故言必及有\footnote{故言必及有:所以讲“无”的时候,必定说到“有”。“有”亦为老子哲学的概念。};老、庄未免于有\footnote{未免:不能免。},恒训其所不足\footnote{恒:常。}。”

{\cangkai\zihao{5}【评】裴徽“才理清明,能释玄言”(见\CJKunderwave{三国志·管辂传}注),王弼于玄理思辩更是独有超越。两位玄言家,就玄理的根本命题“无”的问题上,圣人孔子与老子的说法不同做了一番探讨,但其侧重的不是理论本身,而是现实问题。}

{\cangkai\zihao{5}老子的“道”是讲“无”的,“无”是宇宙的本体,由它化生一切,开物成务,所以它是“万物之资”,这点裴徽是深信不疑的。但问题来了,圣人孔子却未尝言及这一切事物的根本,只是讨论世间具体事物的存在及其关联;老子反是,并不侧重关注事物实存之有,而是反复言说“有”之上的“道”。这就出现了极其严肃的现实问题和理论问题。在现实,儒家仍为王权的意识形态,是不可忤逆的“独尊”,但这理论是不完善的,它不能解释其因果联系,无法说明从宇宙到人生的圆整世界。老子探讨了这个问题,但如果遵从老子去解释人生、世界,又将瓦解了儒家学说,这是裴徽依违两难的根本困惑。}

{\cangkai\zihao{5}王弼则思辩清明,他精巧地将老子学说移入了圣人理路。他解释:圣人原本是以无为体的,但这样玄奥幽渺的道理是无法与一般人说清的,所以发言必切实际,表面看来只是以具体的、实有的事物立论;老、庄虽说高倡以无为体,但不能避免世间之“有”,只是更侧重人们所难以琢磨、把握的“无”,总是不断训解,总之,两家思理在根本问题上有一致性。}

{\cangkai\zihao{5}凌濛初不满意王弼的解释:“皮肤耳,未是妙语。”其实这里既是玄理,也是解决现实的困惑,使老、庄说法能有合理的外衣而理直气壮地说下去。刘辰翁“看得又别”之论,则道出了王弼以老、庄视角阐释圣人的微旨别趣。本则王弼之妙,不在言语机巧,而在潜藏暗转,排除障碍,引领玄辩之风的发展。}

\lettrine{4.9} 傅嘏\myidx{傅嘏}善言虚胜\footnote{虚胜:指“道”的本体超物质存在的无形无象、虚无之理的美妙境界。},{\fzxk\zihao{6}\textcolor{red}{\CJKunderwave{魏志}曰:“嘏字兰硕,北地泥阳人,傅介子之后也。累迁河南尹、尚书。嘏尝论才性同异,锺会集而论之。”\CJKunderwave{傅子}曰:“嘏既达治好正,而有清理识要,如论才性,原本精微,鲜能及之。司隶锺会年甚少,嘏以朋知交会。”}} 荀粲\myidx{荀粲}谈尚玄远\footnote{玄远:玄奥幽远。}。{\fzxk\zihao{6}\textcolor{red}{\CJKunderwave{粲别传}曰:“粲字奉倩,颍川颍阴人,太尉或(彧)少子也。粲诸兄儒术论议各知名。粲能言玄远,常以子贡称‘夫子之言性与天道,不可得而闻也’,然则六籍虽存,固圣人之糠秕。能言者不能屈。”}} 每至共语,有争而不相喻\footnote{喻:理解、明白。}。裴冀州\myidx{裴徽}释二家之义\footnote{裴冀州:指裴徽,见前篇。徽曾任冀州刺史,故称。},通彼我之怀,常使两情皆得,彼此俱畅。{\fzxk\zihao{6}\textcolor{red}{\CJKunderwave{粲别传}曰:“粲太和初到京邑,与傅嘏谈,(嘏)善名理,而粲尚玄远,宗致虽同,仓卒时或格而不相得意。裴徽通彼我之怀,为二家释。顷之,粲与嘏善。”\CJKunderwave{管辂传}曰:“裴使君有高才逸度,善言玄妙也。”}}

{\cangkai\zihao{5}【评】傅嘏是弱冠知名的才子,“有清理识要”,擅长论述玄理“虚胜”境界——超物质存在、无形无象的“道”体,亦论才性异同,活跃于当时,颇有思辩工夫,而性亦颇自负。荀粲也是一位才子,为太尉荀彧少子,其一门父兄皆崇尚儒学,“而粲独好言道”,思存玄理,富于辩才,同时,也是一个个性颇强的人,性“简贵”,不与常人交接。两人各有千秋,皆富个性,所以在讨论玄理时,每每争持不下,谁也说服不了谁。裴徽旁观者清,看出同异,理顺观点,沟通两家,“彼此俱畅”。这里,生动地记述了其时玄言辩难的场景,三人皆富于学识,而又为学理探讨而各执己见,往复论难,一旦达成共识,便都获得了理识和心情的畅快。正始谈玄,已成为士人生命活力的重要表现。}

\lettrine{4.10} 何晏\myidx{何晏}注\CJKunderwave{老子}未毕,见王弼\myidx{王弼}自说注\CJKunderwave{老子}旨\footnote{旨:意旨,意思。}。何意多所短,不复得作声,但应之\footnote{但:只是。但应之,袁本作“但应诺诺”。诺诺,应答声。},遂不复注,因作\CJKunderwave{道德论}。{\fzxk\zihao{6}\textcolor{red}{\CJKunderwave{文章叙录}曰:“自儒者论以老子非圣人,绝礼弃学。晏说与圣人同,著论行于世也。”}}

{\cangkai\zihao{5}【评】本则可与第7则参读。余嘉锡\CJKunderwave{笺疏}谓:“此与上文‘何平叔注\CJKunderwave{老子}’条,一事两见。而一云始成,一云未毕,馀皆小异。盖本出两书,临川不能定其是非,故并存之也。”两则合观,前曰“神伏”,此曰“诺诺”,何晏之神情跃然纸上,亦可味出\CJKunderwave{世说}传神写照的精妙。}

\lettrine{4.11} 中朝时\footnote{中朝:东晋时,对西晋的称呼。},有怀道之流\footnote{怀道:信奉老、庄学说。之流:某类人。},有诣王夷甫\myidx{王衍}谘疑者\footnote{诣:拜访。王夷甫:王衍,字夷甫。王夷甫:王衍(256—311)字夷甫,见刘孝标注。“以清虚通理称”,为当时清谈名家,“妙悟若神”,“妙善玄言,唯谈\CJKunderwave{老}、\CJKunderwave{庄}为事”。为政多谋略,不以经国为念,而善思自全之计,然终为石勒所害。(见\CJKunderwave{晋书}本传)。谘:询问,请教。}。值王昨已语多,小极\footnote{小极:身体不适。小,稍微;极,疲倦。},不复相酬答,乃谓客曰:“身今少恶\footnote{身:我。少恶:有些不适。},裴逸民\myidx{裴頠}亦近在此\footnote{裴逸民:裴頠,字逸民。王夷甫:王衍(256—311)字夷甫,见刘孝标注。“以清虚通理称”,为当时清谈名家,“妙悟若神”,“妙善玄言,唯谈\CJKunderwave{老}、\CJKunderwave{庄}为事”。为政多谋略,不以经国为念,而善思自全之计,然终为石勒所害。(见\CJKunderwave{晋书}本传)。},君可往问。”{\fzxk\zihao{6}\textcolor{red}{\CJKunderwave{晋诸公赞}曰:“裴頠谈理,与王夷甫不相推下。”}}

{\cangkai\zihao{5}【评】王衍位高势重,累居显职,他“妙善玄言,唯谈\CJKunderwave{老}、\CJKunderwave{庄}为事”,长于言辩,“世号‘口中雌黄’。朝野翕然,谓之‘一世龙门’”(见\CJKunderwave{晋书·王衍传})。其倾动当世,为后进之士所景慕趋从。这里,所谓“怀道之流”,便是追踵王衍,趋附“龙门”者。其人之来,恐怕是景慕大名、咨议疑难、投其所好、干禄求官兼而有之。然而,这位贵人兼大名士,虽名为清高,然性亦颇自私,居宰辅之重,而不顾念经国,于纷乱中,常思自全之计,对这位求见的小人物,他是不会牺牲自己的健康去接见酬答的。而他指示的往问裴頠,又真是开了“怀道之流”的玩笑。裴頠崇“有”,与衍谈\CJKunderwave{老}、\CJKunderwave{庄}玄理而崇无针锋相对,“咨疑者”怀尚“无”之玄理,果去请教裴頠,彼情彼景不问可知。}

{\cangkai\zihao{5}本则透过王衍的影响,侧面映现了当时谈玄的风气。}

\lettrine{4.12} 裴成公\myidx{裴頠}作\CJKunderwave{崇有论}\footnote{裴成公:裴頠,卒后追谥“成”,故称。\CJKunderwave{崇有论}今存于\CJKunderwave{晋书}裴頠本传。},时人攻难之\footnote{攻难:反驳、辩难。},莫能折\footnote{莫:没有人。折:驳倒。}。唯王夷甫\myidx{王衍}来,如小屈\footnote{如:似乎。小屈:稍受屈折。}。时人即以王理难裴,理还复申\footnote{还复:仍然。申:展开、申发。刘孝标注“名譣”,“譣”,通“验”,纷欣阁本作“论”。}。{\fzxk\zihao{6}\textcolor{red}{\CJKunderwave{晋诸公赞}曰:“自魏太常夏侯玄、步兵校尉阮籍等,皆著\CJKunderwave{道德论}。于时侍中乐广、吏部郎刘汉亦体道而言约,尚书令王夷甫讲理而才虚,散骑常侍戴奥以学道为业,后进庾敳之徒皆希慕简旷。頠疾世俗尚虚无之理,故著\CJKunderwave{崇有}二论以折之。才博喻广,学者不能究。后乐广与頠清闲欲说理,而頠辞喻丰博,广自以体虚无,笑而不复言。”\CJKunderwave{惠帝起居注}曰:“頠箸二论以规虚诞之弊。文词精富,为世名譣。”}}

{\cangkai\zihao{5}【评】就思想学术而言,正始玄风给板结的经学注入了一股新的生气,将思想引入了自由探讨,独立思考的境界,重新标举了思想的价值、人的尊严,这是思想界辩证发展的结果。所以玄学一经何晏、王弼的倡导,迅即蔚成风气,成为一个时代的标志。但在现实生活中,某些思想家身居重位,虽长于探讨思想学术,而对实际政治事务或拙于料理,或沉迷于理论思辨而轻于实务,或竟以玄理人格藐视世俗,以至于在当时就有了将思想探讨的价值与处理实务的结果混为一谈的看法。\CJKunderwave{晋书·裴頠传}就说:“何晏、阮籍素有高名于世,口谈浮虚,不遵礼法,尸禄耽宠,仕不事事;至王衍之徒,声誉太盛,位高势重,不以物务自婴,遂相仿效,风教陵迟。”企图否定玄学在思想发展中的思想学术价值,后世更以“清谈误国”深责当时玄风。}

{\cangkai\zihao{5}裴頠是一位博学多才的学者,也是身居显位的王朝重臣,他重务实,习惯于历史经验,\CJKunderwave{言语}门记“裴頠论前言往行,衮衮可听”,作为学者,他在理论上欲矫当时玄学之蔽,以为崇“无”之论,导致了“浮虚”之弊;作为王朝重臣,他对“时俗放荡,不尊儒术”,“仕不事事”深为忧虑,因而作\CJKunderwave{崇有论},言辩生物以“有”为本,世界只能以“有”济“有”,“理既有之众,非无为之所能循也”。落到现实处,即人们不习服礼法,则无以为政,一切将乱了套路(参见\CJKunderwave{崇有论})。其\CJKunderwave{崇有论}属玄学,也是名实之争,但这一论辩的落脚点是针对现实问题,提出主张的。裴頠博学有辩才,“时人谓頠为言谈之林薮”、“頠若武库,五兵纵横,一时之杰也”(见\CJKunderwave{晋书·裴頠传}),所以,除“世号‘口中雌黄’”的辩家王衍,其馀均不是他的对手。}

{\cangkai\zihao{5}本则所记,是学养交锋的实录,述说着魏晋思想的活跃。}

\lettrine{4.13} 诸葛厷\myidx{诸葛厷}年少不肯学问\footnote{学问:学习。}。始与王夷甫\myidx{王衍}谈,便已超诣\footnote{超诣:高超的境界。}。王叹曰:“卿天才卓出,若复小加研寻\footnote{小:稍微。研寻:探究、研讨。},一无所愧。”厷后看\CJKunderwave{庄}、\CJKunderwave{老}更与王语\footnote{\CJKunderwave{庄}、\CJKunderwave{老}:\CJKunderwave{庄子}、\CJKunderwave{老子},道家经典,当时赖以清谈的基本著作。},便足相抗衡。{\fzxk\zihao{6}\textcolor{red}{王隐\CJKunderwave{晋书}曰:“厷字茂远,琅邪人,魏雍州刺史绪之子。有逸才,仕至司空主簿。”}}

{\cangkai\zihao{5}【评】诸葛厷亦名门之后,其父诸葛绪曾为雍州刺史,晋武帝时为卫尉,因而少年诸葛厷能有机会拜见显贵兼大名士王衍,并蒙他指教。诸葛厷少而颖达,不甚学言谈就可达到高超的境界,为王衍所赏识,并加指点。他研习\CJKunderwave{庄}、\CJKunderwave{老}之后,竟能与“世号‘口中雌黄’”的辩家王衍相抗衡,可见更富思辨智慧的\CJKunderwave{庄}、\CJKunderwave{老}对人才情的启迪意义,及玄谈时风对人思辨、口才的促动。\CJKunderwave{世说·文学}集此点滴斑痕,将魏晋玄风记述得细微生动。}

\lettrine{4.14} 卫玠\myidx{卫玠}总角时问乐令\myidx{乐广}“梦”\footnote{卫玠:即卫筁,官拜太子洗马,故称。惨悴:忧伤憔悴的样子。左右:身边侍从人员。总角:古代未成年前的发式,将发梳成两个髻,状如角,故称。借指童年。乐令:乐广,乐广(?—304):字彦辅,南阳淯阳(今河南南阳东南)人。少孤贫,寒素为业,与物无竞。其清谈析理,与王衍并称,卫瓘以为有正始遗风。官至尚书令,八王乱中,以故忧卒。},乐云是想。卫曰:“形神所不接而梦,岂是想邪?”乐云:“因也\footnote{因:因由,凭借。}。未尝梦乘车入鼠穴,捣齑噉铁杵,皆无想无因故也\footnote{齑(jī鸡):把菜切细或捣碎,做成酱菜或醃菜。噉:吃。}。”{\fzxk\zihao{6}\textcolor{red}{\CJKunderwave{周礼}有六梦:一曰正梦,谓无所感动平安而梦也。二曰噩梦,谓惊愕而梦也。三曰思梦,谓觉时所思念也。四曰寤梦,谓觉时道之而梦也。五曰喜梦,谓喜说而梦也。六曰惧梦,谓恐惧而梦也。按乐所言“想”者,盖思梦也。“因”者,盖正梦也。}} 卫思“因”,经日不得,遂成病。乐闻,故命驾为剖析之。卫即小差\footnote{差:差通“瘥”,病愈。}。乐叹曰:“此儿胸中当必无膏肓之疾\footnote{膏肓之疾:古代医学称心尖脂肪为“膏”,隔膜为“肓”,是药力所不及处。因以“膏肓之疾”,称不治之病。}!”{\fzxk\zihao{6}\textcolor{red}{\CJKunderwave{春秋传}曰:“晋景公有疾,求医于秦,秦伯使医缓为之。未至,公梦疾为二竖子。曰:‘彼,良医也。惧伤我焉!’其一曰:‘居肓之上,膏之下,若我何?’医至,曰:‘疾不可为也!在肓之上,膏之下,攻之不可达,刺之不可及,药不至焉。’公曰:‘良医也。’”注:“肓,鬲也。心下为膏。”}}

{\cangkai\zihao{5}【评】卫玠是个夙慧早悟的天才。\CJKunderwave{晋书}说他“年五岁,风神秀异”,其祖卫瓘说“此儿有异于众”,所谓“异”,就是他善于提问、善于思考、感悟力超凡,后终以析理入微而令当时名士“叹息绝倒”,王敦将其视为正始天才王弼一流人物,慨叹:“昔王辅嗣吐金声于中朝,此子(卫玠)复玉振于江表,微言之绪,绝而复续。”本则记这位才子童年时的一段逸事,亦足见其“异”,那就是解索问题的执着和悟性。梦之迷,是最早进入人类思维视野的大问题之一,也是人终生面对的问题。卫玠总角时就对它索解沉思,以至于因思成病,及至得高明“剖析”,有所解悟,才病况好转,这就大异于一般少儿。这样一位善思而执着于析理的胚模,正好合适于以玄言思辩为风尚的时代。因而他的早慧故事,记述下来,就如珠玉般炫目动人。}

\lettrine{4.15} 庾子嵩\myidx{庾敳}读\CJKunderwave{庄子}\footnote{庾子嵩:庾敳,见刘孝标注。},开卷一尺许便放去\footnote{开卷:魏晋时代,书籍多为竹简或缣帛。竹简用丝绳、麻绳或皮条编结成册,竹简或缣帛都为卷轴,读时执卷展开。开卷即指诵读。许:表示大约的数量。一尺许,形容所读不多。},曰:“了不异人意\footnote{了:全。人:此指自己。}。”{\fzxk\zihao{6}\textcolor{red}{\CJKunderwave{晋阳秋}曰:“庾敳字子嵩,颍州(川)人,侍中峻第三子。恢廓有度量,自谓是老、庄之徒。曰:‘昔未读此书,意尝谓至理如此。今见之,正与人意暗同。’上至豫州长史。”}}

{\cangkai\zihao{5}【评】\CJKunderwave{易}、\CJKunderwave{老}、\CJKunderwave{庄}玄家三宝,穷神知化,启人玄想思辩,这一特点,就不像过去儒教经学那样,苦诵硬记,皓首穷经。时代风气为之一变,人们问学的态度和方法也大异其趣,多“读书不甚研求,而默识其要”(见\CJKunderwave{晋书·阮瞻传})。另外,时风盛言\CJKunderwave{老}、\CJKunderwave{庄},取其学理,作为当时的话语氛围,上流人士虽未尝亲阅\CJKunderwave{老}、\CJKunderwave{庄}原著,也为时风所染,涉其道理。庾敳为当时上流士人,耳濡目染无非玄言。故其思想理路,已是玄学特色,所以他展卷即似曾相识,以为\CJKunderwave{庄子}之论,不过尔尔。其读书法看去又是当时作风,以思辩析理为追求,不执着于书卷,所以刘辰翁评说:“此自是谈\CJKunderwave{庄子}法。”然而对这一风气、这一读书法如若失去把握之度,便流于浮虚,王僧虔\CJKunderwave{诫子书}就说明了这种情况:“汝开\CJKunderwave{老子}卷头五尺许,未知辅嗣(王弼)何所道,平叔(何晏)何所说,马(融)郑(玄)何所异,\CJKunderwave{指}、\CJKunderwave{例}何所明,而便盛于麈尾,自呼谈士,此最险事。”(见\CJKunderwave{南齐书·王僧虔传})庾敳此风度,就颇类后来王僧虔所诫之者。王世懋评价他:“此本无所晓而漫为大言者,使晓人得之,便当沉湎濡首。”可见庾敳并非何晏、王弼之辈的真学者、真名士,只不过是自视了不起的虚浮狂士,所以,当他见到真懂\CJKunderwave{庄子}的郭象时就只好噤若寒蝉了。}

\lettrine{4.16} 客问乐令\myidx{乐广}“旨不至”者\footnote{乐令:乐广,乐广(?—304):字彦辅,南阳淯阳(今河南南阳东南)人。少孤贫,寒素为业,与物无竞。其清谈析理,与王衍并称,卫瓘以为有正始遗风。官至尚书令,八王乱中,以故忧卒。“旨不至”:\CJKunderwave{庄子·天下篇}载惠施之说“指不至,至不绝”,为名家学派命题,客以此发问。},乐亦不复剖析文句,直以麈尾柄确几曰\footnote{麈尾:魏晋时清谈家的雅具,执麈尾而谈是当时风尚。庾法畅:\CJKunderwave{高僧传}卷四作康法畅,所记与本则同。麈尾:\CJKunderwave{世说音释}:“鹿之大者曰麈,群鹿从之,视麈尾所传而往,故谈者挥焉。”其形制似羽扇,上圆下平,附以长毫毛。确:敲击。几:几案。}:“至不\footnote{至:达到。不:通“否”。}?”客曰:“至!”乐因又举麈尾曰:“若至者,那得去\footnote{那得:怎么、如何。去:离开。}?”{\fzxk\zihao{6}\textcolor{red}{夫藏舟潜往,交臂恒谢,一息不留,忽焉生灭。故飞鸟之影,莫见其移;驰车之轮,曾不掩地。是以去不去矣,庸有至乎?至不至矣,庸有去乎?然则前至不异后至,至名所以生;前去不异后去,去名所以立。今天下无去矣,而去者非假哉?既为假矣,而至者岂实哉?}} 于是客乃悟服。乐辞约而旨达,皆此类。

{\cangkai\zihao{5}【评】这一“旨不至”论,肇自\CJKunderwave{公孙龙子·指物论},谓“物莫非指,而指非指”,同时又源于\CJKunderwave{庄子·天下篇}所载惠施“指不至,至不绝”之说,为典型的名家命题。有关问题,直到今天仍有歧义,可见其思理玄妙,可作多种解释。余嘉锡注据陆德明\CJKunderwave{经典释文}引司马彪云:“夫指之取物,不能自至,要假物,故至也。然假物由指不绝也。一云指之取火以钳刺鼠以锥。故假于物指是不至也。”又论说:“夫理涉玄门,贵乎妙悟,稍参迹象,便落言筌。司马所注,诚不如乐令之超脱。今姑录之,以存古义。其他家所释,咸无取焉。”嘉锡又案:“乐令未闻学佛,又晋时禅学未兴,然此与禅家机锋,抑何神似?盖老、佛同源,其顿悟固有相类者也。”如乐广之类的玄家清谈,辞约而旨达,注意揭示那潜藏在语言文字背后的精微之旨,其说理重在启人思维的顿悟,以便激发听者积极思索的主观能动性,听者与说者共同完成了理论命题的探讨。刘辰翁评曰:“此我辈禅也,在达摩前。”王世懋亦云:“此皆禅机转语。”后来禅家的机锋,并非从天而降,而是多少受到魏晋玄家清谈的影响。其实玄学的佛学化与佛学的玄化,佛玄融合而相互促进,是中国思想理论的又一大发展。}

\lettrine{4.17} 初注\CJKunderwave{庄子}者数十家,莫能究其旨要\footnote{旨要:要领、主旨。}。向秀\myidx{向秀}于旧注外为解义,妙析奇致\footnote{奇致:精奇的旨趣。},大畅玄风\footnote{畅:弘扬。玄风:谈玄的风气。}。{\fzxk\zihao{6}\textcolor{red}{\CJKunderwave{秀别传}曰:“秀与嵇康、吕安为友,趣舍不同。嵇康傲世不羁,安放逸迈俗,而秀雅好读书。二子颇以此嗤之。后秀将注\CJKunderwave{庄子},先以告康、安。康、安咸曰:‘书讵复须注?徒弃人作乐事耳!’及成,以示二子。康曰:‘尔故复胜不?’安乃惊曰:‘庄周不死矣!’后注\CJKunderwave{周易},大义可观,而与汉世诸儒互有彼此,未若隐\CJKunderwave{庄}之绝伦也。”秀本传或言:秀游托数贤,萧屑卒岁,都无注述。唯好\CJKunderwave{庄子},聊隐崔譔所注,以备遗忘云。\CJKunderwave{竹林七贤论}云:“秀为此义,读之者无不超然,若已出尘埃而窥绝冥,始了视听之表。有神德玄哲,能遗天下,外万物。虽复使动竞之人顾观所徇,皆怅然自有振拔之情矣。”}} 唯\CJKunderwave{秋水}、\CJKunderwave{至乐}二篇未竟而秀卒\footnote{\CJKunderwave{秋水}、\CJKunderwave{至乐}:\CJKunderwave{庄子}中的篇名。}。秀子幼,义遂零落\footnote{零落:散佚。},然犹有别本\footnote{别本:副本。}。郭象\myidx{郭象}者,为人薄行\footnote{薄行:操行轻薄。},有俊才。{\fzxk\zihao{6}\textcolor{red}{\CJKunderwave{文士传}曰:“象字子玄,河南人。少有才理,慕道好学,托志\CJKunderwave{老}、\CJKunderwave{庄}。时人咸以为王弼之亚,辟司空掾、太学博士。”}} 见秀义不传于世,遂窃以为己注。乃自注\CJKunderwave{秋水}、\CJKunderwave{至乐}二篇,又易\CJKunderwave{马蹄}一篇\footnote{\CJKunderwave{马蹄}:\CJKunderwave{庄子}中的篇名。},其馀众篇,或定点文句而已\footnote{定点:修改、删定。}。{\fzxk\zihao{6}\textcolor{red}{\CJKunderwave{文士传}曰:“象作\CJKunderwave{庄子注},最有清辞遒旨。”}} 后秀义别本出,故今有向、郭二\CJKunderwave{庄},其义一也。

{\cangkai\zihao{5}【评】随着玄学的展开,作为道家经典之一的\CJKunderwave{庄子},很受谈家的关心,因此注家蜂起,但能探到经典的旨要,注解传达精髓,却并非易事。向秀“雅好读书”,“清悟有远识”,和嵇康、阮籍、山涛等同为高超名家,对经典有深刻而独到的体会,所以他的注释,深得嵇康、吕安等叹许,谓“庄周不死矣。”因他的注解“发明奇趣”,将\CJKunderwave{庄子}的精神阐扬恢宏,“振起玄风”,使“读之者超然心悟,莫不自足一时”(见\CJKunderwave{晋书·向秀传})。向秀深得\CJKunderwave{庄子}旨要,在当时的注家中特立杰出,将\CJKunderwave{庄子}真正推入了此后的玄学视野,而此前,何晏、王弼只谈\CJKunderwave{易}、\CJKunderwave{老},秀之注,表现了一位优秀思想者的成就,同时他也成了魏晋之际,推动谈玄思辩风气的重要人物。}

{\cangkai\zihao{5}然而其著“未传于世”,相传为郭象窃用。事实与否,迄今无定论。余嘉锡先生考证:“向秀\CJKunderwave{庄子注}今已不传,无以考见向、郭异同。\CJKunderwave{四库总目}146\CJKunderwave{庄子提要}尝就\CJKunderwave{列子}张湛\CJKunderwave{注}、陆氏\CJKunderwave{释文}所引秀义,以校郭\CJKunderwave{注}。有向有郭无者,有绝不相同者,有互相出入者,有郭与向全同者,有郭增减字句大同小异者。知郭点定文句,殆非无证。”(见\CJKunderwave{世说新语笺疏})这样说,是看到郭象运用了向秀成果,并曾受向启迪,然而,郭象亦是“好\CJKunderwave{老}、\CJKunderwave{庄},能清言”的才子,他能解悟向秀之注,而且“他有他自己的见解,有他自己的哲学体系。他注\CJKunderwave{庄子}并不是为注而注,而是借\CJKunderwave{庄子}这部书发挥他自己的哲学见解”(见冯友兰\CJKunderwave{中国哲学史新编})。但本则要说明的是,郭象将向注据为己有而不加任何说明,这种掠他人之美的做法,有违德行操守,实是“薄行”之举。这与本书前面郑玄之于\CJKunderwave{左传}注解、何晏之于\CJKunderwave{老子}注,形成了鲜明的对照(参见本篇2、7两则),书中将当时文学名士的形象做了生动的展演,让后人领略着魏晋玄学舞台上才士的丰富面容。}

{\cangkai\zihao{5}从故事针砭郭象掠美行为的倾向,可以见出时人对向秀这位“竹林七贤”名家之一的倾服,也可见向秀其人在当时的魅力。}

\lettrine{4.18} 阮宣子\myidx{阮修}有令闻\footnote{令闻:美誉。},太尉王夷甫\myidx{王衍}见而问曰\footnote{太尉:官名,魏晋时为三公之一。王夷甫:王衍,王夷甫:王衍(256—311)字夷甫,见刘孝标注。“以清虚通理称”,为当时清谈名家,“妙悟若神”,“妙善玄言,唯谈\CJKunderwave{老}、\CJKunderwave{庄}为事”。为政多谋略,不以经国为念,而善思自全之计,然终为石勒所害。(见\CJKunderwave{晋书}本传)。}:“老、庄与圣教同异\footnote{圣教:周公、孔子之教,指儒家学说。同异:相同或是不同。}?”对曰:“将无同\footnote{将无同:莫非相同。}。”太尉善其言,辟之为掾\footnote{辟:征召入仕。掾:属官的通称。据\CJKunderwave{晋书·职官志},太尉有西曹掾、东曹掾各一人。}。世谓“三语掾”。卫玠嘲之曰\footnote{卫玠:即卫筁,官拜太子洗马,故称。惨悴:忧伤憔悴的样子。左右:身边侍从人员。}:“一言可辟,何假于三\footnote{假:凭借。}?”宣子曰:“苟是天下人望\footnote{苟:假如。人望:众望所归的人。},亦可无言而辟,复何假一?”遂相与为友。{\fzxk\zihao{6}\textcolor{red}{\CJKunderwave{名士传}曰:“阮修字宣子,陈留尉氏人。好\CJKunderwave{老}、\CJKunderwave{易},能言理,不喜见俗人。时误相逢,即舍去。傲然无营,家无担石之储,晏如也。琅邪王处仲为鸿胪卿,谓曰:‘鸿胪丞差有禄,卿常无食,能作不?’修曰:‘为复可耳。’遂为鸿胪丞、太子洗马。”}}

{\cangkai\zihao{5}【评】其事\CJKunderwave{晋书·阮瞻传}记为“瞻见司徒王戎,戎问曰:‘圣人贵名教,老、庄任自然,其旨异同?’瞻曰:‘将无同?’”与\CJKunderwave{世说}此记不同。余嘉锡谓:“唐修\CJKunderwave{晋书}喜用\CJKunderwave{世说},此独与\CJKunderwave{世说}不同,知其必有所考矣。”(见\CJKunderwave{世说新语笺疏})所记虽有不同,但并不影响我们透过故事,窥见当时风尚和名士面貌。魏晋玄学家何晏、王弼即以道论儒,何晏\CJKunderwave{论语集解},以道家观点解释儒家名实礼教,王弼更是将名教之本说成是体现自然,调和名教与自然(参见本篇7、8则),使崇尚庄、老具有合法的外衣,于是玄风大煽。因此,当回答“将无同”时,确是善得清谈“旨要”。杨慎评曰:“晋人语言务简,且为两可之词。‘将无’疑言毕竟同也,悟此言筌,千载如面也。”“将无同”一语真是栩栩如生地绘出了说话人的玄家面貌。}

{\cangkai\zihao{5}故事的后半段,亦为精彩佚事。“一言可辟,何假于三?”“得意忘言”,言为筌,意既得矣,三语都显得多,正是标准的玄家之风。}

\lettrine{4.19} 裴散骑\myidx{裴遐}娶王太尉\myidx{王衍}女\footnote{裴散骑:裴遐。曾任散骑侍郎,故称。},婚后三日,诸婿大会\footnote{大会:此为宴集亲朋。}。{\fzxk\zihao{6}\textcolor{red}{\CJKunderwave{晋诸公赞}曰:“裴遐字叔道,河东人。父纬,长水校尉。遐少有理称,辟司空掾、散骑郎。”\CJKunderwave{永嘉流人名}:“衍字夷甫,第四女适遐也。”}} 当时名士王、裴子弟皆悉集。郭子玄\myidx{郭象}在坐\footnote{郭子玄:郭象,字子玄。},挑与裴谈\footnote{挑:挑逗、引发。谈:辨辩析玄理。}。子玄才甚丰赡\footnote{丰赡:富足、充盈。},始数交未快,郭陈张甚盛,裴徐理前语,理致甚微\footnote{理致:义理情致。微:精微深奥。},四坐咨嗟称快\footnote{咨嗟:赞叹。}。{\fzxk\zihao{6}\textcolor{red}{邓粲\CJKunderwave{晋纪}曰:“遐以辩论为业,善叙名理,辞气清畅,泠然若琴,闻其言者,知与不知,无不叹服。”}} 王亦以为奇,谓诸人曰:“君辈勿为尔,将受困寡人女婿\footnote{寡人:谦辞,意谓寡德之人,古代君主、王侯用以自称,当时有地位的士大夫间或也自称寡人。}。”

{\cangkai\zihao{5}【评】本则是在王衍家的玄谈群英表演。王衍、裴遐、郭象皆清谈名士,这里,除他们在此场合按清谈的程式,逞才斗智之外,引人注意的,还有裴遐之谈令四座叹服称快的原因。刘孝标注引邓粲说,“遐以辩论为业,善叙名理,辞气清畅,泠然若琴”,余嘉锡就此点考究:“晋、宋人清谈,不惟善言名理,其音响轻重疾徐,皆自有一种风韵。\CJKunderwave{宋书·张敷传}云:‘善持音仪,尽详缓之致。与人别,执手曰:念相闻。馀音久之不绝。’裴遐之‘泠然若琴瑟’,亦若此而已。”(见\CJKunderwave{世说新语笺疏})这样,在群英表演中,让我们看到了魏晋风度的又一神韵——辞气泠然清畅之雅,也见到了时人对形式美的注重。形式不止为反映内容服务,其本身之美亦独具价值,为人叹赏。}

{\cangkai\zihao{5}这里也形象地描画了当时对谈玄的热衷、崇尚的风气。本来婚宴是喜庆、娱人的场景,主人翁王衍却在这里安排了一场“辩论会”,并且大家都尽兴、尽欢,如后世听了堂会一样满足,可见名理的辨析,抽象思维的较量,智慧的碰撞,是当时名士所乐于享受的最高乐事之一。}

\lettrine{4.20} 卫玠\myidx{卫玠}始度江\footnote{卫玠:即卫筁,官拜太子洗马,故称。惨悴:忧伤憔悴的样子。左右:身边侍从人员。度:通“渡”。},见王大将军\myidx{王敦}\footnote{王大将军:王敦,王敦:字处仲,晋琅邪临沂(今属山东)人,王导堂兄。妻为晋武帝女襄城公主,拜驸马都尉。晋室东迁,与王导一起辅佐元帝,任要职,握重兵,镇守扬州、荆州等重镇。公元322 年起兵谋反,入京都建康。王含:见刘孝标注。光禄勋:官名,九卿之一,领管光禄、大中、中散、谏议等大夫及羽林郎、五官、虎贲、左右等中郎将。}。{\fzxk\zihao{6}\textcolor{red}{\CJKunderwave{敦别传}曰:“敦字处仲,琅邪临沂人。少有名理,累迁青州刺史。避地江左,历侍中、丞相、大将军、扬州牧。以罪伏诛。”}} 因夜坐,大将军命谢幼舆\myidx{谢鲲}\footnote{命:召唤。谢幼舆:谢鲲,谢豫章:谢鲲,曾作豫章太守。刘孝标注“鲲子别见”,“子”字衍。将:携,谓携之送客。自:已经。参:参与、进入。上流:上等、上品。}。{\fzxk\zihao{6}\textcolor{red}{\CJKunderwave{晋阳秋}曰:“谢鲲字幼舆,陈郡人。父衡,晋硕儒。鲲性通简,好\CJKunderwave{老}、\CJKunderwave{易},善音乐,以琴书为业。避乱江东,为豫章太守,王敦引为长史。”\CJKunderwave{鲲别传}曰:“鲲四十三卒,赠太常。”}}玠见谢,甚悦之,都不复顾王,遂达旦微言\footnote{微言:精微深妙的言辞。指玄谈。}。王永夕不得豫\footnote{永夕:整夜、终夜。豫:参与。}。玠体素羸\footnote{素:素来、一向。羸(léi雷):身体瘦弱多病。},恒为母所禁。尔夕忽极\footnote{尔夕:那夜。忽:突然。极:疲劳过度。},于此病笃,遂不起\footnote{病笃:病重。不起:犹言“去世”、“死去”。}。{\fzxk\zihao{6}\textcolor{red}{\CJKunderwave{玠别传}曰:“玠少有名理,善\CJKunderwave{易}、\CJKunderwave{老},自抱羸疾,初不于外擅相酬对。时友叹曰:‘卫君不言,言必入冥。’武昌见大将军王敦,敦与谈论,咨嗟不能自已。”}}

{\cangkai\zihao{5}【评】卫玠从形象面貌到精神气质都是一个润如美玉的人才,谢鲲情怀远畅,恬于荣辱,是一丘一壑间人;在这两人面前,“有问鼎之心”的王敦,便为俗物了。所以,三人在座,而两人声气相投,王敦惨遭冷落,只好眼巴巴地看着这两位才士,纵情谈玄析理,敦平日虽也“雅尚清谈”,可这时却无可置喙。而他们两人,沉浸于精神世界,愈谈愈深入、愈忘情,竟把身边这位“心怀刚忍”、性喜咄咄逼人的权要,忘得一干二净,似乎并无他人一样,这真是当时才子风流、名士境界的生动演绎。王敦插不上嘴,亦不觉尴尬,无愠怒之意,在此场合出奇的平静、宽和,如观圣手对弈,从这一侧面,也见出当时谈玄的特点。}

{\cangkai\zihao{5}黄辉评曰:“当日玠喜而不寐,神情宛然。”以析理至审见称的天才卫玠,能喜而不寐,以至劳瘁病笃,可见遇到了旗鼓相当的谈友,这里不仅卫玠神情宛然,谢鲲也被烘托而出。}

\lettrine{4.21} 旧云\footnote{旧云:从前传说。}:“王丞相\myidx{王导}过江左\footnote{王丞相:王导。江左:江东,指东晋。},止道声无哀乐\footnote{止:只。道:讲说。}、{\fzxk\zihao{6}\textcolor{red}{嵇康\CJKunderwave{声无哀乐论}略曰:“夫他方异俗,歌笑不同。使错而用之,或闻哭而欢,或听歌而戚,然哀乐之情均也。今用均同之情,发万殊之声,斯非音声之无常乎?”}} 养生、{\fzxk\zihao{6}\textcolor{red}{嵇叔夜\CJKunderwave{养生论}曰:“夫虱著头而黑,麝得柏而香,颈处噞而瘿,齿居晋而黄。岂唯蒸之使重无使轻,芬之使香勿使延哉?诚能蒸以灵芝,润以醴泉,无为自得,体妙心玄,庶与羡门北寿、王乔争年,何为不可养生哉?”}} 言尽意{\fzxk\zihao{6}\textcolor{red}{欧阳坚石\CJKunderwave{言尽意论}略曰:“天理得于心,非言不畅。物定于彼,非名不辨。名逐物而迁,言因理而变,不得相与为二矣。苟无其二,言无不尽矣。”}} 三理而已\footnote{理:义理。}。然宛转关生\footnote{宛转:变化。},无所不入\footnote{入:涉及、运用。}。”

{\cangkai\zihao{5}【评】王导是一位政治家、玄学家。但作为政治家的成功却和他的学问分不开,王导式的政治手法,恰是他深谙学问的必然和成功表达,在群英名相中,只有他成功地顺应了当时规律,创立东晋百年基业,因而他不只是一个有修养的清谈家,还是一个了不起的实践家。}

{\cangkai\zihao{5}作为玄学家,王导所谈的“三理”,都是玄学中的根本命题。}

{\cangkai\zihao{5}嵇康著\CJKunderwave{声无哀乐论},辨析名实之理,认为音乐本身是客观的音声之和,它并不存在什么主观的哀乐之情,正如酒之性甘,它能令人大怒和狂欢,但不能说酒具有怒与欢之理,哀乐是人的主观情感,并非音声、酒本身所固有,因而反对传统儒教根据统治的需要而对音乐艺术做出机械教条的解释。这是他“越名教而任自然”主张的另一角度说法,也是名实之辩的一理。但在这里,他认识到了音乐的主要性质——“和”。也因为说明了音声本无哀乐之义,哀乐在人之情感,它强调了人作为审美主体的重要意义。嵇康又著\CJKunderwave{养生论}、\CJKunderwave{答难养生论},通过名实之辩,解决宇宙观的问题。他主张,人要忘却“所欲”,懂得审辨贵贱,达到“混乎与万物并行”,宠辱皆忘,不肆志于荣华而超越世俗,具有任自然,“行不违乎道”的精神境界,方可谈养生。这样就把握了人与宇宙、自然以及社会的关系,从而达到“养生”的目的。这一说,开启了对于人自身生命价值的探索。“言不尽意”与“言尽意”也是当时争论的重要的名实之理。针对普遍崇尚的“言不尽意”论,欧阳建著\CJKunderwave{言尽意论},认为人可以通过“名”所代表的概念,去认识事物的内部联系、规律,获得判断,而判断是可以用“言”来表达的。“名”与“言”都是人思维必不可少的工具。对于事物认识的“理”可得,“理得于心,非言不畅”,它说明了言语与思维的一致性,这就将“贵无”之说引到了“崇有”的境地,在精神实质上与裴頠的\CJKunderwave{崇有论}共相旨趣了。它注重了务实的一面。}

{\cangkai\zihao{5}王导将这三玄理既作为理论问题去谈,也作为人生态度去用,遂心自适,涉及社会人生各个领域。他谈活了“理”,也用活了“理”,辅政期间,既务实际,抓住要害,全神贯注地笼络江东士族,为朝廷的安身立命打下了坚实的基础;又务虚超脱,抓大放小,不纠缠于细节,以至人们误以为他是个“愦愦”的糊涂丞相。他能成功地协调各种因素、各派势力,开创江东稳定局面,正见出这种理论修养和人生妙悟在他政治生涯中的重要作用。}

\lettrine{4.22} 殷中军\myidx{殷浩}为庾公\myidx{庾亮}长史\footnote{殷中军:殷浩,(?—356):见刘孝标注。浩善谈玄,负盛名,简文执政时惧桓温势盛,引浩为建武将军、扬州刺史,以对抗桓温。后因北征许洛败绩,为桓温所弹,废为庶人。庾公:庾亮,庾亮(289—340)的敬称。他历仕东晋元、明、成三朝,作为外戚,曾执国政,显赫于朝。的卢:传说中的凶马之名,骑之不利主人。长史:官名。},{\fzxk\zihao{6}\textcolor{red}{按\CJKunderwave{庾亮僚属名}及\CJKunderwave{中兴书},浩为亮司马,非为长史也。}} 下都\footnote{下都:顺江而下,到京师建康。殷浩随庾亮在武昌,到建康须沿长江上游东下,故称“下都”。},王丞相\myidx{王导}为之集\footnote{王丞相:王导。集:集会。},桓公\myidx{桓温}、王长史\myidx{王濛}、王蓝田\myidx{王述}、{\fzxk\zihao{6}\textcolor{red}{\CJKunderwave{王述别传}曰:“述字怀祖,太原晋阳人。祖湛,父承,并有高名。述蚤孤,事亲孝谨,箪瓢陋巷,宴安永日。由是为有识所知,袭爵蓝田侯。”}} 谢镇西\myidx{谢尚}并在\footnote{桓公:桓温,桓公北征:桓温曾有三次北征,刘盼遂\CJKunderwave{世说新语校笺}考订,此次当为太和四年(369)之征。时桓温已58岁。王长史:王濛。谢镇西:谢尚,谢豫章:谢鲲,曾作豫章太守。刘孝标注“鲲子别见”,“子”字衍。将:携,谓携之送客。自:已经。参:参与、进入。上流:上等、上品。并:全、都。}。丞相自起解帐带麈尾\footnote{解帐带麈尾,麈尾悬于帐带,欲清谈,故自起解之。麈尾,庾法畅:\CJKunderwave{高僧传}卷四作康法畅,所记与本则同。麈尾:\CJKunderwave{世说音释}:“鹿之大者曰麈,群鹿从之,视麈尾所传而往,故谈者挥焉。”其形制似羽扇,上圆下平,附以长毫毛。},语殷曰:“身今日当与君共谈析理\footnote{身:晋人自称,犹言“我”。}。”既共清言,遂达三更。丞相与殷共相往反\footnote{往反:往复辩难。反,同“返”。},其馀诸贤,略无所关\footnote{略无所关:其他人无法参与谈论。关:参与、涉及。}。既彼我相尽,丞相乃叹曰:“向来语\footnote{向来:刚才。乃竟:竟然。理源:义理的本源。归:归向。},乃竟未知理源所归,至于辞喻不相负\footnote{辞喻:言辞和比喻。负:违背。指言辞丰赡,比喻精妙,顺畅而达意。}。正始之音,正当尔耳\footnote{正始之音:正始(三国魏齐王芳年号240—249)年间,何晏、王弼等人开创的清谈玄学。后人称当时的风尚为“正始之音”。正当:只能、不过。尔耳:如此。}!”明旦,桓宣武\myidx{桓温}语人曰:“昨夜听殷、王清言甚佳\footnote{清言:清谈。},仁祖亦不寂寞,我亦时复造心\footnote{造心:心有所悟。},顾看两王掾\footnote{王掾:指王濛和王述,两人均为王导属官。},{\fzxk\zihao{6}\textcolor{red}{王濛、王述,并为王导所辟}} 。辄翣如生母狗声(馨)\footnote{辄:总是。翣(shà厦):很,极。生:活的。声,诸本作“馨”。馨,样、似的,为晋人口语。}。”

{\cangkai\zihao{5}【评】本则第一主角是王导,故事具体刻画了其谈玄风采和当时清谈场景。}

{\cangkai\zihao{5}王导邀集的都是当时大名鼎鼎的清谈巨子。在这场合下,王导从仪容风度到问题的提出,都俨然是清谈领袖。此场景反映了挥麈而谈,是清谈不可或缺的表演方式,演绎的是名士的风度;“析理”是清谈的要件,即将所谈\CJKunderwave{易}、\CJKunderwave{老}、\CJKunderwave{庄}的道理,辨析毫厘,深入到微妙之处,这是对玄理的理解程度、思维水平、论辩才能的比试,是学术研讨、理论思辨的展现。殷浩长于\CJKunderwave{易}、\CJKunderwave{老},为当时“风流谈论者所宗”(\CJKunderwave{晋书·殷浩传}),这里以王导、殷浩为主客对垒辩难,其馀诸贤皆为陪客观战。“向来语,乃竟未知理源所归”,一个政治领袖,当众承认自己的理论欠缺,体现了王导的谦虚好学精神,同时也见其水平非凡。谈论下来已达三更,情状是“辞喻不相负”——论辞、比喻不仅丰赡达意,而且逻辑清晰,具有论辩的说服力,“正始之音”的动人,也不过如此。论者争胜而听者如聆妙响,时有会心,竟然把两个大名家王濛、王述听得呆如活脱脱的母狗一般,以此衬托出主人的精神风采。这种学识和智慧的博弈,不仅使人们获得享受而且推动着哲学思考的进展。}

{\cangkai\zihao{5}本则场景如绘,清谈的状态,人物的声情风貌一一展现尽致。}

\lettrine{4.23} 殷中军\myidx{殷浩}见佛经云\footnote{殷中军:殷浩,(?—356):见刘孝标注。浩善谈玄,负盛名,简文执政时惧桓温势盛,引浩为建武将军、扬州刺史,以对抗桓温。后因北征许洛败绩,为桓温所弹,废为庶人。}:“理亦应阿堵上\footnote{理:名理、义理。阿堵:晋人口语,意为“这个”、“这”。}。”{\fzxk\zihao{6}\textcolor{red}{佛经之行中国尚矣,莫详其始。\CJKunderwave{牟子}曰:“汉明帝夜梦神人,身有日光,明日,博问群臣。通人传毅对曰:‘臣闻天竺有道者号曰佛,轻举能飞,身有日光,殆将其神也。’于是遣羽林将军秦景,博士弟子王遵等十二人之大月氏国,写取佛经四十二部,在兰台石室。”刘子政\CJKunderwave{列仙传}曰:“历观百家之中,以相检验,得仙者百四十六人,其七十四人已在佛经,撰得七十。可以多闻博识者遐观焉。”如此即汉成、哀之间,已有经矣。与\CJKunderwave{牟子}传记便为不同。\CJKunderwave{魏略·西戎传}曰:“天竺城中有临儿国。\CJKunderwave{浮屠经}云:其国王生浮图。浮图者,太子也。父曰屑头邪,母曰莫邪。浮图者,身服色黄,发如青丝,爪如铜。其母梦白象而孕。及生,从右胁出,而有髻,坠地能行七步。天竺又有神人日沙律。昔汉哀帝元寿元年,博士弟子景虑,受大月氏王使伊存口传\CJKunderwave{浮屠经}。曰复豆者,其人也。”\CJKunderwave{汉武故事}曰:“昆邪王杀休屠王,以其众来降,得其金人之神,置之甘泉宫。金人皆长丈馀,其祭不用牛羊,唯烧香礼拜。上使依其国俗祀之。”此神全类于佛,岂当汉武之时,其经未行于中土,而但神明事之耳。故验刘向、鱼豢之说,佛至自哀、成之世明矣。然则牟传所言四十二者,其文今存非妄。盖明帝遣使广求异闻,非是时无经也。}}

{\cangkai\zihao{5}【评】佛教传至东晋,开始繁荣兴盛起来了,但仍属佛教哲学与中土思想的互渗磨合的初始阶段。当时,学人们以本土哲学的知识背景去理解佛学,佛学学者讲佛家经典,也把佛学的概念转译成中国哲学的术语来表达,这样听者才好理解。\CJKunderwave{高僧传}就记载,东晋名僧慧远,讲解佛经引用\CJKunderwave{庄子}为说。时人几乎把佛学等同于玄学,玄、佛概念互用、互换。本则所记,殷浩对佛经的理解,就反映了这种情况。作为玄学家,他深谙\CJKunderwave{易}、\CJKunderwave{老},见到佛经,那“名言”(佛学将概念称之为“名言”)、佛理之辩正复与玄辩相似,虽然更显精微、邃密,其境更其玄深,但思维、理路并不隔阂,于是便将玄、佛之理打通理解。外来佛学通过玄学阐释而中国化,更易为中土消化吸收,从而为中国古代的思辨哲学注入了一股清新活力,其意义不可低估。而“理亦应阿堵上”,则正是殷浩对佛经的会心得意之论,由此也见其善于思辩的玄家性情。}

\lettrine{4.24} 谢安\myidx{胜}年少时\footnote{谢安:(?—358):字无奕,谢安长兄,陈郡阳夏谢氏家族在东晋初期的代表人物之一。},请阮光禄\myidx{阮裕}道\CJKunderwave{白马论}\footnote{阮光禄:阮裕,即阮裕,曾以金紫大夫征,故称。\CJKunderwave{世说}作者刘义庆为避宋武帝刘裕名讳,从不称阮裕之名。剡(shàn 善):古县名,在今浙江嵊州。}。{\fzxk\zihao{6}\textcolor{red}{\CJKunderwave{孔丛子}曰:“赵人公孙龙云:‘白马非马。马者所以命形,白者所以命色。夫命色者非命形,故曰白马非马也。’”}} 为论以示谢,于时谢不即解阮语,重相咨尽\footnote{重(chónɡ虫):反复。咨尽:问得彻底明白。咨:询问。}。阮乃叹曰:“非但能言人不可得\footnote{非但:不仅。},正索解人亦不可得\footnote{索:寻求。解人:能理解的人。}!”{\fzxk\zihao{6}\textcolor{red}{\CJKunderwave{中兴书}曰:“裕甚精论难。”}}

{\cangkai\zihao{5}【评】\CJKunderwave{晋书}本传说,谢安总角即“神识沉敏,风宇条畅”,甚喜清言,而有名于时。本则记其年少时,就善于提问,敢于探问思辩精深的\CJKunderwave{白马论}。作为先秦名家的代表人物之一,公孙龙的诸种命题,包括“白马非马”,都是向人们习以为常,日用而不知的常识进行挑战,而欲辩明这些命题,又是对人们思维能力的考验。“白马非马”涉及了辩证法中的同一与差别、一般与个别的关系问题,也涉及了逻辑学中概念的内涵与外延的关系问题。公孙龙用抽象化、绝对化的办法,把“白”与“马”割裂开来,否定了一般思维中所必然具有的所谓“马”,只能存在于白马、黑马……一切个别的马之中;马必然有白的、黑的……舍此,则绝无抽象的“马”。他还把“马”所指的本质属性和“白马”所代表的概念两者间的差异区别开来,向人们旧有的思维习惯挑战。凡此若无切合思维规律的精审思辩是会愈辩愈糊涂的。少年谢安向“论难甚精”,属文“精义入微”的行家里手阮裕(\CJKunderwave{晋书·阮裕传})请教,并且刨根问底,非弄明白不可。可见谢安的早慧,也可见当时崇尚理论、智慧的社会风习。}

{\cangkai\zihao{5}对本则所记的谢安反复求解,王世懋叹曰:“谢公犹然,况他人乎?”}

{\cangkai\zihao{5}对本则中阮裕的慨叹,王世贞评曰:“‘文章千古事,得失寸心知’,亦谓此耳。夫刿鉥心胸,指摘造化,如探大海出珊瑚,奈何令逐臭吠声之士轻读之也。至于有美必赏,如响之应,连城隐璜,卞生动容,流水离弦,钟子抚心。古人重知己,而薄感恩,夫岂欺我!”玄理奥妙,真如探骊龙之珠,索得一解实为难事,而能得解人更是求遇知音的庆幸。}

\lettrine{4.25} 褚季野\myidx{褚裒}语孙安国\myidx{孙盛}{\fzxk\zihao{6}\textcolor{red}{褚裒、孙盛并已见。}} 云\footnote{褚季野:褚裒,对褚裒的敬称。褚裒(póu 抔)(303—349),晋康帝皇后之父,朝廷议以“不臣之礼”,力辞执政,而赴外镇。官征北大将军。曾率军三万北伐,败后上疏自贬,忧慨发愤而卒。见\CJKunderwave{晋书·外戚传}。孙安国:孙盛。}:“北人学问,渊综广博\footnote{北人:指黄河以北的人。渊综:渊深综括。}。”孙答曰:“南人学问,清通简要\footnote{清通:清明通达。简要:简明切要。}。”支道林\myidx{支遁}闻之曰\footnote{支道林:为东晋名僧,善玄理,是当时佛学“般若学”的代表人物,多才艺,长于草隶。与王洽、刘惔、殷浩、许询、郗超、王羲之、谢安等名流游好。常:同“尝”,曾经。}:“圣贤固所忘言\footnote{忘言:即“得意忘言”的缩略语。\CJKunderwave{庄子·外物}:“言者所以在意,得意而忘言。”}。自中人以还\footnote{中人:中等智力的人,与“圣贤”相对比而言。以还:以下。},北人看书,如显处视月\footnote{显处视月:在轩敞处看月亮,比喻所见广博,但重点不突出。};南人学问,如牖中窥日\footnote{牖中窥日:从窗户中看太阳,比喻所见狭隘,但重点突出。}。”{\fzxk\zihao{6}\textcolor{red}{支所言,但譬成孙、褚之理也。然则学广则难周,难周则识暗,故如显处视月;学寡则易覈,易覈,则智明,故如牖中窥日也。}}

{\cangkai\zihao{5}【评】本则记三位学者讨论南北学风,他们俱学有造诣,而所论亦切中肯綮,从中映现出各人的精神风采。}

{\cangkai\zihao{5}褚裒北人,有“皮里阳秋”之称,谢安雅重之,“恒云:‘裒虽不言,而四时之气亦备矣。’”(见\CJKunderwave{晋书·褚裒传}),他是一位有见识的干才。这里评论北人之学,虽有偏誉倾向,但将北人的为学特点讲出来了,就中也透露着褚裒作为中原学士的自负。孙盛是著名的学者,不仅善言名理,与殷浩擅名一时,而且是一位著名的史家,其史学著作\CJKunderwave{晋阳秋}“词直而理正,咸称良史焉”(见\CJKunderwave{晋书·孙盛传})。他十岁即来江南,其学浸染的是南人学风,所以对南方学人的善得要领,清明通达深有体会。他们两人是就学风而论的,概括举要,颇似学究在讨论学术史,俨然是学问家的风范。而支道林则不然,借学风谈玄理,空灵摇曳,是方外人的气质风度。支道林是名僧,少时人们就把他比作早慧天才王弼、卫玠,说他“造微之功,不减辅嗣”;神情俊彻,几乎就是卫玠再世(见\CJKunderwave{高僧传})。可见不是凡品,他又早悟佛理,二十五岁出家事佛,领会佛经,卓焉独拔。在这里,他的评论也表达着“慧根”。他先把圣贤和一般人区别开来,圣贤之质,无所谓南北,都得意忘言入造化之境,而一般人,才会有南北学风的差异,虽特点不同,各有千秋,但毕竟是“中人以还”,所得“显处见月”也罢,“牖中窥日”也罢,都是一隅之识。支道林的聪慧、风度和骨子里的自高就在这表白中洋溢而出了。}

\lettrine{4.26} 刘真长\myidx{刘惔}与殷渊源\myidx{殷浩}谈\footnote{刘真长:刘惔,字真长,曾任丹阳尹,故称。谢安妻兄,尚明帝女庐陵公主。会稽王司马昱为相,与王濛并为其座上清谈之客。性简贵自重,与王羲之友善。卒年三十六。殷渊源:殷浩,(?—356):见刘孝标注。浩善谈玄,负盛名,简文执政时惧桓温势盛,引浩为建武将军、扬州刺史,以对抗桓温。后因北征许洛败绩,为桓温所弹,废为庶人。谈:辩论。},刘理如小屈\footnote{小屈:稍显劣势。},殷曰:“恶\footnote{恶(wù误):叹词,表示慨叹。}!卿不欲作将善云梯仰攻\footnote{作将:制作。善:良好。}?”{\fzxk\zihao{6}\textcolor{red}{\CJKunderwave{墨子}曰:“公输般为高云梯,欲以攻宋。墨子闻之,自鲁往,裂裳裹足,日夜不休,十日十夜而至于郢。见楚王曰:‘闻大王将攻宋,有之乎?’王曰:‘然。’墨子曰:‘请令公输般设攻宋之具,臣请试守之。’于是公输般设攻宋之计,墨子萦带守之。输九攻之,而墨子九却之,不能入,遂辍兵。”}}

{\cangkai\zihao{5}【评】刘惔与殷浩都自视颇高,是声名赫赫的清谈家。孙盛作\CJKunderwave{易象妙于见形论},在简文帝处,帝令殷浩与孙盛就此辩难,而浩败于盛,复请惔与盛辩,惔“辞甚简至,盛理遂屈”(见\CJKunderwave{晋书·刘惔传})。此番,等于殷浩败给了刘惔。本则却是殷浩使刘惔“小屈”,便颇显得意,王世懋评曰:“此言戏刘虽善攻,不能当己之墨守也。”对清谈玄家论辩胜负的关注,正可见士人风气及其精神需求。}

\lettrine{4.27} 殷中军\myidx{殷浩}云\footnote{殷中军:殷浩,(?—356):见刘孝标注。浩善谈玄,负盛名,简文执政时惧桓温势盛,引浩为建武将军、扬州刺史,以对抗桓温。后因北征许洛败绩,为桓温所弹,废为庶人。}:“康伯\myidx{韩康伯}未得我牙后惠\footnote{康伯:韩康伯。“牙后惠”句:言莫非得我馀惠,即似我之意。说见朱铸禹\CJKunderwave{世说新语汇校集注}。}。”{\fzxk\zihao{6}\textcolor{red}{\CJKunderwave{浩别传}曰:“浩善\CJKunderwave{老}、\CJKunderwave{易},能清言。康伯,浩甥也,甚爱之。”}}

{\cangkai\zihao{5}【评】韩康伯长于\CJKunderwave{周易},其\CJKunderwave{周易注}与王弼之注并传于世。在当时,康伯即为名家,与大名家殷仲堪并称,\CJKunderwave{世说·品藻}记时人评价,说他“义理所得”与殷不相上下。他的如此造诣,或受到其舅父、著名玄学家殷浩的影响。本传就记他深为舅氏所推许:“康伯能自标置,居然是出群之器。”本则殷浩不无自得地说康伯受惠于他,当并非虚语。在尚智逞才的魏晋风气中,对才子的敏感和珍视也是那一时期的动人之处,更何况舅甥之间呢?}

\lettrine{4.28} 谢镇西\myidx{谢尚}少时\footnote{谢镇西:谢尚,谢豫章:谢鲲,曾作豫章太守。刘孝标注“鲲子别见”,“子”字衍。将:携,谓携之送客。自:已经。参:参与、进入。上流:上等、上品。},闻殷浩\myidx{殷浩}能清言\footnote{殷浩:(?—356):见刘孝标注。浩善谈玄,负盛名,简文执政时惧桓温势盛,引浩为建武将军、扬州刺史,以对抗桓温。后因北征许洛败绩,为桓温所弹,废为庶人。清言:清谈。},故往造之\footnote{造:拜会。}。殷未过有所通\footnote{通:阐发。},为谢标榜诸义\footnote{标榜:揭示。},作数百语,既有佳致\footnote{佳致:美好的情趣。},兼辞条丰蔚\footnote{辞条丰蔚:文辞条理丰富多彩。},甚足以动心骇听\footnote{动心骇听:动人心弦,骇人听闻。}。谢注神倾意,不觉流汗交面。殷徐语左右:“取手巾与谢郎拭面。”{\fzxk\zihao{6}\textcolor{red}{按殷浩大谢尚三岁,便是时流,或当贵其胜致,故为之挥汗。}}

{\cangkai\zihao{5}【评】本则可见,魏晋士人是何等地注重义理辩难,把它看作人生的价值等第,讲得从容出色,可以得意非凡,意气洋洋;而稍有逊色,则汗颜难堪。}

{\cangkai\zihao{5}谢尚聪颖特达,“辨悟绝伦,脱略细行,不为流俗之事”,“善音律,博综群艺”(见\CJKunderwave{晋书·谢尚传}),是个多才多艺的才子兼性情中人。他慕名拜访殷浩,浩果为名家,不仅义理特达,而且辞采丰蔚,辞理并茂,风流动人。如此风采,令谢尚动心动情,这位年少时就曾被誉为“一坐之颜回”的才子,见到了当世高明,听讲后不觉汗颜;而大师般的殷浩也风度从容,让人关照这位“后学”,取巾拭汗。一则故事,将当时清谈名家的推崇思辩和人格魅力烘托而出。}

\lettrine{4.29} 宣武\myidx{桓温}集诸名胜讲\CJKunderwave{易}\footnote{宣武:桓温,桓公北征:桓温曾有三次北征,刘盼遂\CJKunderwave{世说新语校笺}考订,此次当为太和四年(369)之征。时桓温已58岁。名胜:名流,名士。},{\fzxk\zihao{6}\textcolor{red}{\CJKunderwave{易乾凿度}曰:“孔子曰:易者,易也,变易也,不易也。三(成)德为道,苟为(包籥)者,易也。其德也,光明四通,日月星辰布,八卦序,四时和也。变也者,天地不变,不能成朝;夫妇不变,不能成家。不易者,其位也。天在上,地在下;君南面,臣北面;父坐子伏,此其不易也。故易者,天、地、人道也。”郑玄序\CJKunderwave{易}曰:“易之为名也,一言而函三义,简易一也,变易二也,不易三也。”\CJKunderwave{系辞}曰:“乾坤,\CJKunderwave{易}之蕴也,\CJKunderwave{易}之门户也。”又曰:“\CJKunderwave{乾},确然示人易矣;\CJKunderwave{坤},隤然示人简矣。易则易知,简则易从。”此言其简易法则也。又曰:“其为道也屡迁,变动不居周流六虚,上下无常,刚柔相易。不可以为典要,唯变所适。”此则言其从时出入移动也。又曰:“天尊地卑,乾坤定矣。卑高以陈,贵贱位矣。动静有常,刚柔断矣。”此则言其张设布列不易也。据此三义,而说易之道,广矣,大矣。}} 日说一卦。简文\myidx{司马昱}欲听\footnote{简文:晋简文帝司马昱,指晋简文帝司马昱(320—372),穆帝年幼即位,昱任抚军大将军总理政务。后来大将军桓温专擅朝政,先废海西公,后立司马昱为帝,第二年崩。},闻此便还,曰:“义自当有难易,其以一卦为限邪\footnote{其:通“岂”,怎么。}!”

{\cangkai\zihao{5}【评】桓温邀集名流讲论\CJKunderwave{周易},限定日说一卦。简文颇有个性,闻其日说一卦,便扫兴而回。按说\CJKunderwave{周易}六十四卦为一个完整的系统,每一卦都离不开这个系统,很难割裂开来,仅就某卦说某卦。更何况,\CJKunderwave{文言}、\CJKunderwave{彖传}、\CJKunderwave{象传}、\CJKunderwave{系辞传}、\CJKunderwave{说卦传}、\CJKunderwave{序卦传}、\CJKunderwave{杂卦传}等“十翼”和六十四卦象一起构成了一个严整的符号象征的哲学体系,它们相互渗透、彼此关联,以一个完整的有机体去述说天道、人事,展示着它的辩证思维的魅力。所以,“日说一卦”,实在是难以顾及全面,必定会挂一漏万的。但这系统中的每一卦,又独具自己的内容和个性,如果综合\CJKunderwave{周易}系统去讲论每一具体的卦,又会有其独特的个性和魅力,更何况是“名胜”讲论,定会问题迭出,异彩纷呈。简文喜欢通论,孙盛在他那里讲\CJKunderwave{易象妙于见形论}等\CJKunderwave{易}之通理,他就乐之不疲,津津有味;而以一卦为限,不及义理之全面,他就认为无论如何是讲不好的,不值得听。本来已往,闻此便还,可见简文虽贵为帝王,却自有其名士风度。而桓温虽然权倾朝野,日理万机,但仍广召名流,研讨、日讲\CJKunderwave{周易},又可见当时思辨哲学在士人心目中的崇高地位。}

\lettrine{4.30} 有北来道人好才理\footnote{道人:魏晋时称僧人。才理:哲理。},与林公\myidx{支遁}相遇于瓦官寺\footnote{林公:支遁,为东晋名僧,善玄理,是当时佛学“般若学”的代表人物,多才艺,长于草隶。与王洽、刘惔、殷浩、许询、郗超、王羲之、谢安等名流游好。常:同“尝”,曾经。瓦官寺:佛寺名。东晋哀帝兴宁二年(364)造,初名慧方寺,寺有瓦官阁,在建康城西南隅。},讲\CJKunderwave{小品}\footnote{\CJKunderwave{小品}:佛典\CJKunderwave{般若波罗蜜经}的略本。}。于时竺法深\myidx{法深}、孙兴公\myidx{孙绰}悉共听\footnote{竺法深:见\CJKunderwave{德行}30。孙兴公:孙绰。}。此道人语,屡设疑难,林公辩答清析,辞气俱爽。此道人每辄摧屈。孙问深公:“上人常是逆风家\footnote{上人:尊称有造诣的和尚,此指竺法深。常:袁本作“当”。逆风家:顶风前进的人。此指竺法深,言其辩论有才力。},向来何以都不言\footnote{向来:刚才。都:全。}?”{\fzxk\zihao{6}\textcolor{red}{庾法畅\CJKunderwave{人物论}曰:“法深学义渊博,名声蚤者(著),弘道法师也。”}} 深公笑而不答。林公曰:“白旃檀非不馥\footnote{旃(zhān沾)檀:即檀香,名贵香木名。有赤、白两种。},焉能逆风?”{\fzxk\zihao{6}\textcolor{red}{\CJKunderwave{成实论}曰:“波利质多天树,其香则逆风而闻。”}} 深公得此义,夷然不屑\footnote{夷然:安然、泰然。不屑:不在意,不理睬。}。

{\cangkai\zihao{5}【评】本篇第四十二则刘孝标注曰:“释氏辨空,经有详者焉,有略者焉。详者为\CJKunderwave{大品},略者为\CJKunderwave{小品}。”都是讲述佛家义理的经典。佛理辨空,尤须精审思维。支遁名僧,对佛理“卓焉独拔,得自天心”(见\CJKunderwave{高僧传}),自然能“辩答清析,辞气俱爽”,屡屈北来道人。然而竺法深也不含糊,年十八出家,二十四岁即“讲\CJKunderwave{法华}、\CJKunderwave{大品},既蕴深解,复能善说。故观风味道者常数盈五百”。其声望非凡,被认为是“道俗标令”(见\CJKunderwave{高僧传}),孙绰说他是“逆风家”并非虚誉。正因为如此,两名僧各不相让,王世懋云:“林公意谓波利质多天树才能逆风闻香;白旃檀非天树比,焉能逆风。以天树自比,以白旃檀比深公,故深公不屑。”支遁自视甚高,深公于心未许,观两人风貌,皆非悟空道人,逞才斗气,俨然是飘逸当时的风流名士。}

\lettrine{4.31} 孙安国\myidx{孙盛}往殷中军\myidx{殷浩}许共论\footnote{孙安国:孙盛。殷中军:殷浩,(?—356):见刘孝标注。浩善谈玄,负盛名,简文执政时惧桓温势盛,引浩为建武将军、扬州刺史,以对抗桓温。后因北征许洛败绩,为桓温所弹,废为庶人。许:处所。论:清谈。},往反精苦\footnote{往返:反复论辩。精苦:精深艰难而激烈。},客主无间\footnote{无间:无间隙。言论辩紧张激烈。}。左右进食,冷而复煗(暖)者数四。彼我奋掷麈尾\footnote{奋掷麈尾:奋力挥动麈尾。麈尾:庾法畅,\CJKunderwave{高僧传}卷四作康法畅,所记与本则同。麈尾:\CJKunderwave{世说音释}:“鹿之大者曰麈,群鹿从之,视麈尾所传而往,故谈者挥焉。”其形制似羽扇,上圆下平,附以长毫毛。},悉脱落,满餐饭中。宾主遂至莫忘食\footnote{莫:暮的本字。}。殷乃语孙曰:“卿莫作强口马\footnote{强口马:犟口不受约束之马。},我当穿卿鼻。”孙曰:“卿不见决鼻牛\footnote{决鼻牛:豁鼻子牛。},人当穿卿颊\footnote{人:我。}。”{\fzxk\zihao{6}\textcolor{red}{\CJKunderwave{续晋阳秋}曰:“孙盛善理义。时中军将军殷浩擅名一时,能与剧谈相抗者,唯盛而已。”}}

{\cangkai\zihao{5}【评】殷浩是一时清谈名家,孙盛名与相埒。二人曾在简文处谈\CJKunderwave{周易},殷小屈于孙。这里记另一场清谈聚会,则是旗鼓相当,辩论精苦,以至忘餐。情理所至处,奋挥麈尾,真是全身心的投入。此情此景,见出当时谈家风格的精彩,及其追求真知的精神。刘辰翁评曰:“亦是何等往复,传之后世!”但是末了,两人却有失风度,离开了玄理主题,世俗般对骂起来了。评家王世懋也不能理解,说是:“何至相对骂?”但对骂也见出机巧。殷浩情急之中违背了常识,常理是马带嚼,牛穿鼻,可他明指孙盛为马,却说要“穿卿鼻”。这一疏失,让反应机敏的孙盛瞧出破绽,说你就是那想决鼻而逃的犟牛,现在我要穿你的面颊,看你还能怎么逃?言下你是败了还不服输——这是以其人之道,还治其人之身。用机巧的世俗之喻回敬、针砭了殷浩。}

{\cangkai\zihao{5}李贽欣赏他们的才情,不管是面红耳赤的苦论玄理,还是唇枪舌剑的谈骂,总归是学识、智慧的较量,因说:“剧谈固一乐事。”}

\lettrine{4.32}  \CJKunderwave{庄子·逍遥篇}\footnote{\CJKunderwave{庄子·逍遥篇}:即\CJKunderwave{庄子}书中的\CJKunderwave{逍遥游}。},旧是难处\footnote{旧:长久。},诸名贤所可钻味\footnote{钻味:钻研品味。},而不能拔理于郭\myidx{郭象}、向\myidx{向秀}之外\footnote{拔:超出。郭、向:郭象、向秀,二人皆以注\CJKunderwave{庄子}闻名,参见本篇17则。}。支道林\myidx{支遁}在白马寺中,将冯太常\myidx{冯怀}共语\footnote{支道林:支遁,为东晋名僧,善玄理,是当时佛学“般若学”的代表人物,多才艺,长于草隶。与王洽、刘惔、殷浩、许询、郗超、王羲之、谢安等名流游好。常:同“尝”,曾经。将: 与。},{\fzxk\zihao{6}\textcolor{red}{\CJKunderwave{冯氏谱}曰:“冯怀字祖思,长乐人。历太常、护军将军。”}} 因及\CJKunderwave{逍遥}。支卓然标新理于二家之表\footnote{卓然: 高超的样子。标新: 揭示新义。},立异义于众贤之外\footnote{立异:提出不同见解。},皆是诸名贤寻味之所不得\footnote{寻味:寻求体味。}。后遂用支理。{\fzxk\zihao{6}\textcolor{red}{向子期、郭子玄\CJKunderwave{逍遥义}曰:“夫大鹏之上九万尺,鷃之起榆枋,小大虽差,各任其性。苟当其分,逍遥一也。然物之芸芸,同资有待,得其所待,然后逍遥耳。唯圣人与物冥而循大变,为能无待而常通,岂独自通而已。又从有待者不失其所待,不失则同于大道矣。”支氏\CJKunderwave{逍遥论}曰:“夫逍遥者,明至人之心也。庄生建言大道,而寄指鹏、鷃。鹏以营生之路旷,故失适于体外;鷃以在近而笑远,有矜伐于心内。至人乘天三(正)而高兴,游无穷于放浪,物物而不物于物,则遥然不我得,玄感不为不疾而速,则逍然靡不适。此所以为逍遥也。若夫有欲当其所足,足于所足,怏然有似天真。犹饥者一饱,渴者一盈,岂忘烝尝于糗粮,绝觞爵于醪醴哉?苟非至足,岂所以逍遥乎?”此向、郭之\CJKunderwave{注}所未尽。}}

{\cangkai\zihao{5}【评】魏晋将玄理和佛理打通来理解,因此名僧一如名士气质,除诵习佛典外,同时也钻研玄学典籍,参与玄学论争。支道林是一位典型的名僧兼名士的人物。\CJKunderwave{逍遥游}在庄子理论体系中,具有主旨纲要般意义,对该文的理解就涉及对庄子思想核心的把握,所以支道林用心勤苦,于当时名注向、郭义之外,标新立异,也因见解独到而愈发享有声名。}

{\cangkai\zihao{5}庄子的\CJKunderwave{逍遥游}大旨为:无己无待,任性自然,获得个人精神的超越,因悟道而达到逍遥自得的自由境界。向秀有注,他的\CJKunderwave{庄子注}今佚,但部分内容化入今存的郭象注中,所谓向、郭义,大多要看郭象注。郭象以\CJKunderwave{庄子}为蓝本,阐发了自己的一套哲学体系。郭象不是“贵无”派,他认为万物“自生”,也就是“独化”,本来就有。“造物者无主,而物各自造,物各自造而无所待焉,此天地之正也。”没有造物主,物“独化于玄冥”,这就是“自然”。秉于独化之自然,于是就各有其才,各有其分,圣人、臣妾;大鹏、尺鷃皆为独化之自然。而有臣妾之才的,就要安于臣妾之自然,如果相逾,就是过分,过分非但不能得福,还会遭灾。只有明了这个理、顺了这个自然之性才会达到“无心”,“无心”而安于性分,就不计较高下、优劣,也就达到“无待”之境而逍遥了。不难看出,这位玄家、名士的主体倾向强调的是名教与自然的统一,他的解说正是士族现实存在的理论。而在此则故事中,向、郭只是陪衬,支遁才是主角。据刘孝标引支遁说,见出与郭象确有不同。他所理解的逍遥,旨在“至人之心”,而这个心,却不是郭象的“无心”。支遁在当时的佛教传播中,是“即色”派,主张心、性之类皆是空的。一切可见、可感的东西都是因缘和合之假有,于是这原本就不真有的心性,便能随万物而化,又不为物所累。即“物物而不物于物”。它是对这些存在物的超越,因主观上的无所为而达于精神的无处不自足,这就超出了郭象指认的“独化”和安于性分。“支理”是越过因执着于“有”,而受到的性分之累,进而达到无所不适,应变无穷的自由逍遥境地。他标榜,此种境界才合于天然本性。这和郭象安于性分的意见就有了明显的差异。两者的不同,实际是反映了各自主体倾向的差异——一个是为现实政治立论的哲学,一个是追求精神解脱的人生哲学。支遁义中,隐在背后的东西早已注入了另一种思想的因子——佛学思想,它更容易激起人们对现世主体的超越,这似乎更像庄子的超越与自由。这样在形貌上与谈\CJKunderwave{庄}不异,在义理上也超拔了郭象旧解,因而得到名贤叹赏。支遁本人也因有这样的见识和精神,而远超时贤,更具名士魅力。}

\lettrine{4.33} 殷中军\myidx{殷浩}{\fzxk\zihao{6}\textcolor{red}{浩}}尝至刘尹\myidx{刘惔}所清言\footnote{殷中军:殷浩,(?—356):见刘孝标注。浩善谈玄,负盛名,简文执政时惧桓温势盛,引浩为建武将军、扬州刺史,以对抗桓温。后因北征许洛败绩,为桓温所弹,废为庶人。刘尹:刘惔,字真长,曾任丹阳尹,故称。谢安妻兄,尚明帝女庐陵公主。会稽王司马昱为相,与王濛并为其座上清谈之客。性简贵自重,与王羲之友善。卒年三十六。清言:清谈。}。良久,殷理小屈,游辞不已\footnote{游辞:虚浮不切义理的话。已:止。},刘亦不复答。殷去后,乃云:“田舍儿强学人作尔馨语\footnote{田舍儿:乡巴佬。谓土气无知。尔馨:这样、这般。}!”{\fzxk\zihao{6}\textcolor{red}{刘惔已见。}}

{\cangkai\zihao{5}【评】参见本篇二十六则,殷浩、刘惔也是不相上下的清谈敌手。该则刘“如小屈”,被殷浩不失时机地嘲笑了一番。本则刚好反转过来,殷理不仅小屈,而且“游辞不已”,思维切不进所谈义理,找不到妥帖的词汇来加以准确表达,游辞漂浮而不知所止。凌濛初说:“真长前,岂可露此破绽伎俩!”殷浩终于让刘惔抓住破绽,着实嘲讽了一番。两事不论孰先孰后,对举起来,让我们看到了名士的自尊、自负和清谈比试的认真、执着。两个场景、两人形象,正从这一侧面让我们窥见了清谈风貌和当时名士的另一抹剪影。}

\lettrine{4.34} 殷中军\myidx{殷浩}虽思虑通长\footnote{殷中军:殷浩,(?—356):见刘孝标注。浩善谈玄,负盛名,简文执政时惧桓温势盛,引浩为建武将军、扬州刺史,以对抗桓温。后因北征许洛败绩,为桓温所弹,废为庶人。通长:全都擅长。},然于才性偏精\footnote{才性:即“才性论”,是关于才、性内涵及其关系的理论,也是名实之论,为魏晋玄学的重要命题之一。偏精:特别精通。},忽言及\CJKunderwave{四本}\footnote{忽:若。\CJKunderwave{四本}:即锺会撰\CJKunderwave{四本论},参见本篇第5则。},便若汤池铁城\footnote{汤池铁城:汤池谓护城河皆沸水,不可逾越;铁城谓以铁铸就的城墙,喻坚不可摧。},无可攻之势。{\fzxk\zihao{6}\textcolor{red}{\CJKunderwave{神农书}曰:“夫有石城十仞,汤池百步,带甲百万而无粟者,不能自固也。”}}

{\cangkai\zihao{5}【评】从刘劭\CJKunderwave{人物志}研究怎样识别人物,发展为才性问题的激烈争论,魏晋之际许多名士都参与其中,主张才性同、才性异、才性合、才性离各有其人,锺会还专门作了\CJKunderwave{四本论}加以概括和阐发。对才、性这一问题,冯友兰先生说:“从一些现存的残缺材料看起来,所谓才、性,有两个方面的意义。一方面,所谓性,是指人的道德品质,所谓才,是指人的才能。在这一方面说,所谓才、性问题,就是‘德’和‘才’的关系问题。另一方面,所谓才,是指人的才能;所谓性,是指人的才能所根据的天赋的本质。在这个方面,所谓才、性问题就是一个认识论的问题:人的才能主要是由一种天赋本质所决定的,还是主要从学习得来的;是先天所有的,还是后天获得的。”(见\CJKunderwave{中国哲学史新编})这无疑是一个艰深的哲学问题,殷浩对这一问题能论证周详,无懈可击,见出其“偏精”的功夫和天才。殷浩论\CJKunderwave{四本}今不得见,倘真如故事所说的“汤池铁城”水平,则其谈论玄理的思辨哲学当在锺会之上,从而见出江左名士谈玄的进展。}

{\cangkai\zihao{5}惜乎,名士们多谈以为快事,而以著述为苦,向秀注\CJKunderwave{庄},嵇康就以为不如口谈为乐事。今天只能在\CJKunderwave{世说}所记当中,来体味这些谈士的乐趣了。}

\lettrine{4.35} 支道林\myidx{支遁}造\CJKunderwave{即色论}\footnote{支道林:支遁,为东晋名僧,善玄理,是当时佛学“般若学”的代表人物,多才艺,长于草隶。与王洽、刘惔、殷浩、许询、郗超、王羲之、谢安等名流游好。常:同“尝”,曾经。造:作。},{\fzxk\zihao{6}\textcolor{red}{\CJKunderwave{支道林集妙观章}云:“夫色之性也,不自有色,色不自有,虽色而空。故曰:‘色即为空,色复异空。’”}} 论成,示王中郎\myidx{王坦之},{\fzxk\zihao{6}\textcolor{red}{王坦之,已见。}} 中郎都无言\footnote{王中郎:王坦之。}。支曰:“嘿而识之乎\footnote{嘿而识之乎:嘿,同“默”。识(zhì志),记住。句意谓把所见所闻默默地记在心里。}?”{\fzxk\zihao{6}\textcolor{red}{\CJKunderwave{论语}曰:“嘿而识之,诲人不倦,何有于我哉?”}} 王曰:“既无文殊\footnote{文殊:佛教菩萨名。},谁能见赏?”{\fzxk\zihao{6}\textcolor{red}{\CJKunderwave{维摩诘经}曰:“文殊师利问维摩诘云:‘何者是菩萨入不二法门?’时维摩诘嘿然无言,文殊师利叹曰:‘是真入不二法门者也。’”}}

{\cangkai\zihao{5}【评】这里两个人对话,实际是在说佛家话了。}

{\cangkai\zihao{5}佛家所用的认识方法是直接感悟,它重视使认识和思维向直观性和情绪性方面发展。佛家的最高智慧“般若”之智,实质上就是体悟万物性空的直观、直觉。佛家认为,即使是佛的立文字、说法也是幻有,“一切有为法,如梦幻泡影,如露亦如电,应作如是观”(见\CJKunderwave{金刚经})。一切要靠直观、直觉的“悟”。\CJKunderwave{维摩诘经·入不二法门品}说得明确:“无言无说,无示无识,离诸问答,是为入不二法门。”这里,“都无言”、“默而识之”正好是佛家法门。于此可见,两人之谈,是在佛家语境里对话。支遁智巧,词面上用了\CJKunderwave{论语}的现成话,词底却用了佛门故事(见刘孝标注引\CJKunderwave{维摩诘经}),言下之意:你认可了我的妙论而“嘿而识之”吗?中郎回敬:(包括你在内)世上没有文殊这样的高明智者,怎么会有人懂得欣赏此时“嘿然无言”的我呢?这场景,见出王坦之对佛家经典、教义的熟悉,尤其是稔熟\CJKunderwave{维摩诘经}。这说明了佛教哲学对江左士族文人的感召力,具有佛理修养也成了名士风度的内涵之一。}

{\cangkai\zihao{5}见于\CJKunderwave{世说·轻诋},两位是“绝不相能”的一对名士,然则此情此景,就更见王坦之如此回敬中表达的名士性格了。}

\lettrine{4.36} 王逸少\myidx{王羲之}作会稽\footnote{王逸少:王羲之。作会稽:做会稽内史(太守)。},初至,支道林\myidx{支遁}在焉\footnote{支道林:支遁,为东晋名僧,善玄理,是当时佛学“般若学”的代表人物,多才艺,长于草隶。与王洽、刘惔、殷浩、许询、郗超、王羲之、谢安等名流游好。常:同“尝”,曾经。}。孙兴公\myidx{孙绰}谓王曰\footnote{孙兴公:孙绰。}:“支道林拔新领异\footnote{拔新领异:同“标新立异”,见解新奇高妙。},胸怀所及乃自佳\footnote{不:同“否”。},卿欣见不\footnote{胸怀:胸襟。乃自:确实。}?”王本自有一往隽气\footnote{一往:一腔,满腹。隽气:俊逸之气。隽,同“俊”。},殊自轻之。后孙与支共载往王许\footnote{共载:同车。许:处所。},王都领域\footnote{都领域:深相自守、闭拒。},不与交言。须臾支退,后正值王当行,车已在门。支语王曰:“君未可去,贫道与君小语\footnote{贫道:贫僧,时人谓僧为“道”。小语:稍微谈谈。}。”因论\CJKunderwave{庄子·逍遥游}。支作数千言,才藻新奇\footnote{才藻:才思文采。},花烂映发\footnote{花烂映发:如灿烂的鲜花般相映生辉。喻才气纵横,文才绚烂。}。王遂披襟解带\footnote{披襟解带:敞开衣襟,解开衣带。喻胸臆畅然。},流连不能已\footnote{流连:留恋、醉心。}。{\fzxk\zihao{6}\textcolor{red}{\CJKunderwave{支法师传}曰:“法师研十地,则知顿悟于七住;寻庄周,则辩圣人之逍遥。当时名胜,咸味其音旨。”\CJKunderwave{道贤论}以七沙门比竹林七贤。一(支)比向秀,雅尚\CJKunderwave{庄}、\CJKunderwave{老}。二子异时,风尚玄同也。}}

{\cangkai\zihao{5}【评】据\CJKunderwave{高僧传}载:王羲之素闻支遁名,但并不相信他有高才,就特意拜访了他,请他说\CJKunderwave{逍遥}义,支于是标揭新理、才藻惊绝,王不禁为之披襟解带,流连不能已。与本则略有出入,或后世僧人化被动为主动,故意抬高支遁而然。王羲之确实与支遁友好,优游山阴,那是后来的事。这里记其初识情景,与\CJKunderwave{高僧传}比,似更真实而见神韵。王羲之出身华胄,又是冠世才子,并且“以骨鲠称”(见\CJKunderwave{晋书·王羲之传}),就连当世太尉郗鉴求婿他都不在乎,颇有自傲自得的风流,本则所记情形,恰映现着他的一贯性格。就支遁说,作为名僧却颇类游说之士,喜干谒权门,结交名流。这里支遁执着求见,一面有在大才子、大名士面前展现才华,求其赏识,以通交游之好的意思;另一面也未尝没有求得有权、有势、有声望的这位内史抬举,以广其声名的意思,不然他完全可以隐居沙门,诵读他的经典。结果是喜剧性的,客观上是两位才士的相识相知,有如双璧辉映。另外,本则寥寥数笔,将门第高华的王羲之倨傲自负,目空当世名士的风格,及才子爱才的复杂矛盾心理写得真实灵动;也把支遁心怀高见,满腹才华而执着求售的情态写得如在目前。陈梦槐云:“此则叙致风华,宜亟赏。”}

\lettrine{4.37} 三乘佛家滞义\footnote{滞义:含义晦涩难懂。},支道林\myidx{支遁}分判\footnote{支道林:支遁,为东晋名僧,善玄理,是当时佛学“般若学”的代表人物,多才艺,长于草隶。与王洽、刘惔、殷浩、许询、郗超、王羲之、谢安等名流游好。常:同“尝”,曾经。分判:辨别剖析。},使三乘炳然\footnote{炳然:明白、显明。}。诸人在下坐听,皆云可通。支下坐\footnote{下坐:离开坐。},自共说\footnote{自:各自。共说:同时互相讲论。},正当得两\footnote{正当:只能。得:领会。两:两乘。},入三便乱。今义弟子虽传\footnote{弟子:佛门的受业门徒。},犹不尽得\footnote{尽得:全部领悟。}。{\fzxk\zihao{6}\textcolor{red}{\CJKunderwave{法华经}曰:“三乘者:一曰声闻乘,二曰缘觉乘,三曰菩萨乘。声闻者,悟四谛而得道也。缘觉者,悟因缘而得道也。菩萨者,行六度而得道也。然则罗汉得道,全由佛教,故以声闻为名也。辟支佛得道,或闻因缘而解,或听环佩而得悟。神能独达,故以缘觉为名也。菩萨者,大道之人也。方便则止行六度,真教则通修万善,功不为己,悉皆(袁本作‘志存’)广济,故以大道为名也。”}}

{\cangkai\zihao{5}【评】佛家“四圣谛”,对人生之价值给予了一个基本判断——“苦”,在现世人生中,具有意义的行为,即是苦炼修行,修得正果,渡越此岸苦海,而佛门的存在价值就是发大慈悲,自渡、渡人。三乘义便是佛家修行解脱,自渡、渡人的三种途径和境界。刘孝标引\CJKunderwave{法华经}说明了声闻乘、缘觉乘、菩萨乘的不同。对于佛教三义,支遁深有修习与感悟,他曾做过\CJKunderwave{辩三乘论},加以他“得自天心”的慧根及“才藻惊绝”的口辩,于是升堂讲析,果然是教义“炳然”。听者对此却只懂了半截,“悟四谛”的“声闻”,“悟因缘”的“缘觉”,这多半涉及自悟自渡的道理是可以听懂的,自相讨论也不含糊;可是到了“行六度”的“菩萨”,这志存广济,通万善、渡众生的大道,就似懂非懂了。诸名士所悟,属世俗谛;而支遁之悟,入菩萨乘,二者性质判然有别。众愚而僧慧,两相对照,故事将高僧支道林深湛的学养、智慧根器的卓焉独拔,及风流神采烘托而出,凌濛初评说:“惟支能三乘炳然,诸人辄混矣。”}

\lettrine{4.38} 许掾\myidx{许询}{\fzxk\zihao{6}\textcolor{red}{询}}年少时,人以比王苟子\myidx{王循}\footnote{许掾:许询,见\CJKunderwave{言语}69。刘注中王循,袁本作“王脩”,是。},{\fzxk\zihao{6}\textcolor{red}{苟子,王循(袁本作“脩”)之小字也。\CJKunderwave{文字志}曰:“循(袁本作‘脩’)字敬仁,太原晋阳人。父濛,司徒左长史。循明秀有美称,善隶行书,号曰‘流弈清举’。起家著作佐郎,琅邪王文学,转中军司马,未拜而卒,时年二十四。昔王弼之殁,与循同年,故循弟熙叹曰:‘无愧于古人,而年与之齐也。’”}} 许大不平。时诸人士及林法师\myidx{支遁}并在会稽西寺讲\footnote{林法师:即支道林,为东晋名僧,善玄理,是当时佛学“般若学”的代表人物,多才艺,长于草隶。与王洽、刘惔、殷浩、许询、郗超、王羲之、谢安等名流游好。常:同“尝”,曾经。法师,对和尚的敬称。西寺:寺院名,即光相寺,在会稽城西南。讲:讲说、论辩。},王亦在焉。许意甚忿\footnote{意:情绪。忿:恼怒。},便往西寺与王论理,共决优劣。苦相折挫\footnote{苦:极力、尽力。折挫:反驳摧挫。},王遂大屈\footnote{大屈:大败。}。许复执王理,王执许理,更相覆疏\footnote{更相:交互、互相。覆疏:颠倒过来梳理阐发。即执对方观点陈述。},王复屈。许谓支法师曰:“弟子向语何似\footnote{弟子:此为俗家人在僧人面前的谦称。向语:刚才说的话。何似:何如、怎么样。}?”支从容曰:“君语佳则佳矣,何至相苦邪\footnote{相苦:让别人尴尬、窘困。}?岂是求理中之谈哉\footnote{理中:得理之中,即折中、不过分。}!”

{\cangkai\zihao{5}【评】许询后来和孙绰、李充、支遁等成为一个品流的人物,“皆以文义冠世”(见\CJKunderwave{晋书·王羲之传}),在会稽与谢安、王羲之游。这样的大名士,个性自少时就与人不同,其自视、自负,颇异于人。人比之于名士王修,他却认为修不如己,这种比附伤了他的自尊,因此而“意甚忿”。而解“忿”挽回自尊的方法就是较量。当得到机会的时候,他果然不同凡响,将王修击败,证明了自己。其实,读本则,这点似乎并不要紧,生动的是,这位后来的大名士全然不讲“中庸”,而是“任性自然”,淋漓尽致地抒愤,非用足自己的才能就不肯罢休,以至于旁观者支遁都感到过分。这则图画,让我们看到了魏晋名士个性的侧面——如庄子,嬉笑怒骂,尽其天性。}

{\cangkai\zihao{5}许询的行为也说明,他未脱尘俗之气。其逞能而自鸣得意,想得到名僧的认可,抬举自身价值——这便展演了俗相,故致支遁之讥。}

\lettrine{4.39} 林道人\myidx{支遁}诣谢公\myidx{谢安}\footnote{林道人:即支道林,为东晋名僧,善玄理,是当时佛学“般若学”的代表人物,多才艺,长于草隶。与王洽、刘惔、殷浩、许询、郗超、王羲之、谢安等名流游好。常:同“尝”,曾经。谢公:谢安,(?—358):字无奕,谢安长兄,陈郡阳夏谢氏家族在东晋初期的代表人物之一。},东阳\myidx{谢朗}时始总角\footnote{东阳:谢朗。总角:未成年时束发为两小髻,状如角。借指童年时期。},新病起,体未堪劳。与林公讲论,遂至相苦\footnote{遂:至于。相苦:互相辩论激烈。}。{\fzxk\zihao{6}\textcolor{red}{东阳,谢朗也,已见。\CJKunderwave{中兴书}曰:“朗博涉有逸才,善言玄理。”}} 母王夫人在壁后听之,再遣信令还\footnote{信:使者,传话的人。},而太傅\myidx{谢安}留之\footnote{太傅:指谢安。}。王夫人因自出云:“新妇少遭家难\footnote{新妇:当时已婚妇女自称新妇。少遭家难:指早年守寡。其夫谢据早卒。},一生所寄唯在此儿。”因流涕抱儿以归。谢公语同坐曰:“家嫂辞情慷慨\footnote{慷慨:激昂感慨。},致可传述\footnote{致:通“至”,极。可:值得。传述:传扬称颂。},恨不使朝士见\footnote{恨:遗憾。朝士:朝中官员。}。”{\fzxk\zihao{6}\textcolor{red}{\CJKunderwave{谢氏谱}曰:“朗父据,取太原王韬女,名绥。”}}

{\cangkai\zihao{5}【评】支道林为怀道高僧,谢朗虽夙具慧根,但毕竟是总角小儿,与林道人讲论辩难激烈投入,怎不耗尽精神?加之新病起,体力消耗可想而知。过去,年轻的卫玠就是在玄理辩难中耗尽精力而殒命的,所以,其母忧虑不无道理。更何况孀居早寡,此儿是唯一的依靠,王夫人又怎不焦急?一位母亲的爱子之心,脱颖而出。至于谢安,一向重视自家子弟的教育,也孜孜欣赏自家子弟的才情,其中又特别器重谢玄与谢朗两侄,所以很投入地欣赏谢朗与满是灵气的林道人论辩的智慧才能。他忽略了嫂子的感受,惹得嫂子亲自出面,抱怨责备。故事中的谢安同样形象生动。以其一贯的雅量器局,应对家庭里的这点尴尬,是轻而易举的。他的应对,顾左右而言他,不解释、不致歉而是将嫂子大大地表扬了一通,当此情景,既解了王夫人的气,又免了当事人支道林的尴尬,还解嘲般地给自己下了台阶,真是一幕轻喜剧。谢安的智慧和雅量,在这小小的细节中,跃然而出,其人亦在这数行墨迹中,风采焕然。}

\lettrine{4.40} 支道林\myidx{支遁}、许掾\myidx{许询}诸人共在会稽王\myidx{司马昱}{\fzxk\zihao{6}\textcolor{red}{简文}}斋头\footnote{支道林:支遁,为东晋名僧,善玄理,是当时佛学“般若学”的代表人物,多才艺,长于草隶。与王洽、刘惔、殷浩、许询、郗超、王羲之、谢安等名流游好。常:同“尝”,曾经。许掾:许询,见\CJKunderwave{言语}69。斋头:书室中。}。支为法师,许为都讲\footnote{法师、都讲:当时讲经,一人唱经问难,一人主讲阐释,唱者为“都讲”,释经者为“法师”。}。{\fzxk\zihao{6}\textcolor{red}{\CJKunderwave{高逸沙门传}曰:“道林时讲\CJKunderwave{维摩诘经}。”}} 支通一义\footnote{通:阐述。},四坐莫不厌心\footnote{厌心:心满意足,倾倒悦服。}。许送一难\footnote{送难:传出一个诘问,即唱出一段经文,提出问题,令法师解释。},众人莫不抃舞\footnote{抃(biàn卞)舞:手舞足蹈。抃,拍手。}。但共嗟咏二家之美\footnote{嗟咏:赞叹。},不辩其理之所在。

{\cangkai\zihao{5}【评】\CJKunderwave{世说音释}引\CJKunderwave{僧史}曰:“支遁至会稽,王内史请讲\CJKunderwave{维摩},许询为都讲。许发一问,众谓支难以答。支答一义,众谓询无以难。如是问答,连环不尽。”可与本则参读。一个故事,描绘二家之美,而美不胜收。一面是音韵清切的唱经,一面是妙语连珠的应声回答,竟然令听者忽略了听经的实质内容——理之所在。美妙的情景,真的让人迷恋发狂,手舞足蹈。这里是当时风尚的记趣,也是名士非凡神采的展现。于是,本则便有了一个特别醒目之处:优美形式对人的刺激、感召,如同佛理对人的感召一样令人着迷。人们沉迷于形式美的品味、享受,以至于激动,这恰表达了魏晋风情的另一面,在审美自觉中,将文学语言的音乐性这一形式美的要素,单独标扬而出。}

{\cangkai\zihao{5}这里还见出佛事的发展情景。“都讲”唱经,其音韵如何,一般人皆可感受,而“法师”释义,怕不是连环贯珠之类的流畅就可征服听者的,其中还是要伴随听者的一些判断,至少是令人不费深虑就可听懂的,所以必须切合经义的常理,这样才能使听者获得刺激而叫好。倘说者言不及义,或让人根本听不懂,抃舞嗟咏就不再是一个生动场面,而是乱哄哄的闹剧了。由此也大致可以看到,听者或为熟悉讲经的基本群体,如今日的京剧票友,不然他们何以一闻而“厌心”呢?而“不辩”是不遑深究苦索,已被优美的形式所征服。因此,本则也客观地映现了,佛事对当时社会文化渗透的程度。}

\lettrine{4.41} 谢车骑\myidx{谢玄}在安西\myidx{谢奕}艰中\footnote{谢车骑:谢玄。安西:谢奕,谢玄父,谢安之兄,(?—358):字无奕,谢安长兄,陈郡阳夏谢氏家族在东晋初期的代表人物之一。艰:指父母之丧。},{\fzxk\zihao{6}\textcolor{red}{安西,谢弈(奕)。已见。}} 林道人\myidx{支遁}往就语\footnote{林道人:支遁,为东晋名僧,善玄理,是当时佛学“般若学”的代表人物,多才艺,长于草隶。与王洽、刘惔、殷浩、许询、郗超、王羲之、谢安等名流游好。常:同“尝”,曾经。},将夕乃退。有人道上见者,问云:“公何处来?”答云:“今日与谢孝剧谈一出来\footnote{谢孝:犹谢孝子,指居丧之谢玄。剧谈:畅谈。一出:一番。}。”{\fzxk\zihao{6}\textcolor{red}{\CJKunderwave{玄别传}曰:“玄能清言,善名理。”}}

{\cangkai\zihao{5}【评】参见\CJKunderwave{德行}有关记载,守丧尽孝、尽哀,是魏晋时极为注重的礼教,涉及对人德行、名声的评价,影响到其人在社会中的形象、地位。谢玄在丧服期间,与支遁长谈玄理,而且谈得十分投入、热烈。所谓剧谈,为谈玄之一种,就是双方互不相让,穷辞尽理,反复辩难交锋,本篇31则的孙盛与殷浩,两人在玄谈中奋掷麈尾,即可见剧谈中双方动情、热烈之一斑。支遁方外之人,循佛家之理,一切空无幻有,可以不问俗家孝道,而谢玄之家为名门贵族,是断不能不顾及社会影响的。可谢玄却不顾居丧之哀,畅言清谈,见出当时谈玄风气之盛,让人感受到魏晋名士越名教而任自然的个性风貌。谢玄此举,颇有庄子丧妻鼓盆而歌的风神,从中体悟出了纵化自然的飘逸,与儒家的礼义大异其趣。}

\lettrine{4.42} 支道林\myidx{支遁}初从东出\footnote{支道林:支遁,为东晋名僧,善玄理,是当时佛学“般若学”的代表人物,多才艺,长于草隶。与王洽、刘惔、殷浩、许询、郗超、王羲之、谢安等名流游好。常:同“尝”,曾经。东:东边,此指会稽郡。会稽在京师之东,故曰从东出。},住东安寺中\footnote{东安寺:佛寺名,在建康。}。{\fzxk\zihao{6}\textcolor{red}{\CJKunderwave{高逸少(沙)门传}曰:“遁居会稽,晋哀帝钦其风味,遣中使至东迎之。遁遂辞丘壑,高步天邑。”}} 王长史\myidx{王濛}宿构精理\footnote{王长史:王濛。宿构精理:预先构思的精深的义理。},并撰其才藻\footnote{撰:准备。才藻:才思文采。},往与支语,不大当对。王叙致作数百语,自谓是名理奇藻。支徐徐谓曰:“身与君别多年\footnote{身:我。},君义言了不长进\footnote{了不:全不。}。”王大惭而退。

{\cangkai\zihao{5}【评】文章写作,有正衬、反衬之法,本则运用的为正衬法。王濛“性和畅,能言理,辞简而有会”(见\CJKunderwave{晋书·王濛传}),是当时谈玄名士的代表,本则记其用心对付支遁,“宿构精理,并撰才藻”,结果还是惨败给了支遁。以王濛为衬托,见出了支遁的名理才情之高超,远出此辈名士之上。可僧人支遁的风格,却不类看穿一切的高僧,而更像俗间名士。他在本则的风貌,一如本篇38则所记的许询,占先而不给人留馀地,说话尖刻,弄得有头有脸的王长史,无地自容,“大惭而退”,为后世留下了一个污点,刘辰翁说:“岂无此等,亦秽清流。”有味哉,斯言。}

\lettrine{4.43} 殷中军\myidx{殷浩}读\CJKunderwave{小品}\footnote{殷中军:殷浩,(?—356):见刘孝标注。浩善谈玄,负盛名,简文执政时惧桓温势盛,引浩为建武将军、扬州刺史,以对抗桓温。后因北征许洛败绩,为桓温所弹,废为庶人。\CJKunderwave{小品}:佛经的简本。},{\fzxk\zihao{6}\textcolor{red}{\CJKunderwave{释氏辩空经},有详者焉,有略者焉,详者为\CJKunderwave{大品},略者为\CJKunderwave{小品}。}} 下二百签\footnote{签:书签,读书有疑难或心得处,加签为志。},皆是精微\footnote{精微:精妙隐微。},世之幽滞\footnote{幽滞:深奥难懂的地方。}。尝欲与支道林\myidx{支遁}辩之,竟不得。今\CJKunderwave{小品}犹存。{\fzxk\zihao{6}\textcolor{red}{\CJKunderwave{高逸沙门传}曰:“殷浩能言名理,自以有所不达,欲访之于遁。遂邂逅不遇,深以为恨。其为名识赏重,如此之至焉。”\CJKunderwave{语林}曰:“浩于佛经有所不了,故遣人迎林公,林乃虚怀欲往。王右军驻之曰:‘渊源思致渊富,既未易为敌,且己所不解,上人未必能通。纵复服从,亦名不益高。若佻脱不合,便丧十年所保。可不须往。’林公亦以为然,遂止。”}}

{\cangkai\zihao{5}【评】\CJKunderwave{小品}作为佛家的典要,在汉灵帝之时就有竺佛朔的经译,魏晋时期,高僧多诵读讲习(参见\CJKunderwave{高僧传}),支遁也讲习\CJKunderwave{小品}。时风之下,一向以“识度清远”、“尤善玄言”(见\CJKunderwave{晋书·殷浩传})为特点的殷浩,也对其痛下功夫,难解难辨之义,做标志、苦思索,留下疑难准备与讲\CJKunderwave{小品}的支遁辩析。但“竟不得”——未能如愿。依刘孝标注,是因为支遁听了王羲之劝说,为了保持名声而有意回避。这里显现了殷浩的才气更在名僧支遁之上。参见前则,名士们比才量力,激扬时风,演绎着才子风流,光景煞是好看。对本则,诸评家还见出另几层风致。凌濛初说:“惜哉逸少一阻,遂令妙义用绝。”又曰:“犹是救饥术工,啖名念重。”以殷浩之识度与支遁的卓拔,其智慧碰撞,一定会留下许多动人、精彩的思想遗迹,可惜“竟不得”,就是殷浩的标识、札记也见不到了。王羲之的一阻,见出当时对名气看的是何等的重要,甚至不惜以虚伪的手法来保持。刘辰翁曰:“逸少护林公如此,还称沙门,然传之贻笑。”其实这里护的只是“名”,它反映出来的是名士心里底层所深深藏护着的,对名气的认真、执着,“沙门”意义早让位于名气。}

\lettrine{4.44} 佛经以为,祛练神明\footnote{祛练:去除杂念,净化磨炼。神明:精神。},则圣人可致\footnote{圣人:指佛。}。{\fzxk\zihao{6}\textcolor{red}{\CJKunderwave{释氏经}曰:“一切众生,皆有佛性。但能修智慧,断烦恼,万行具足,便成佛也。”}} 简文\myidx{司马昱}云\footnote{简文:东晋简文帝司马昱,指晋简文帝司马昱(320—372),穆帝年幼即位,昱任抚军大将军总理政务。后来大将军桓温专擅朝政,先废海西公,后立司马昱为帝,第二年崩。}:“不知便可登峰造极不\footnote{登峰造极:登山达到顶点,比喻修炼到无以复加的境界。此指成佛。}?然陶练之功\footnote{陶练:陶冶修炼。},尚不可诬\footnote{诬:抹杀。}。”

{\cangkai\zihao{5}【评】去除杂念,澡雪精神,一心修炼而成圣、成佛,这是儒家和佛家一致的修行观念。儒家的境界是成为君子、成为圣人,也重视苦练修行的途径。儒经\CJKunderwave{周易}就列了\CJKunderwave{大壮}、\CJKunderwave{升}等卦强调修养成君子品格的不可间断、积小以高大的“陶练”意义。对此,朱熹概括得浅切明白:“木一日不长,便将枯瘁;学者之于学,不可一日少懈。”(\CJKunderwave{朱子语类})儒家大师强调这种修行,孔子讲“吾日三省吾身”、“慎独”;孟子强调专心致志地去“尽心知性”,绝不能一曝十寒等等,皆是“陶练”。佛家境界是识得自性,悟空即佛。“陶练”要由戒生定,由定生慧,修习禅定,证成正果。儒家文化与西来佛国文化不同的是佛家作为宗教,修行成了实实在在的清规戒律,儒家仅停留在观念形态的说教,但在观念层面上说,二者是有共同特点的。所以谙熟儒家经典的简文,其说法正是会心之言。能否登峰造极,是不是可以成佛,又自当别论,佛家的“陶练”之功是大可借鉴的。本则典型地说明着,当时士大夫对佛教的理解是以中土文化为根基的,佛教立住脚跟的过程,正是其与中国传统文化不断磨合的本土化过程。而这过程,在魏晋时已经特征明显了,简文的形象恰说明中土文化与佛教文化在士人身上的濡染、修养,于是儒雅的简文,便在这片语之中活跃了起来。}

\lettrine{4.45} 于法开\myidx{于法开}始与支公\myidx{支遁}争名\footnote{于法开:东晋名僧,精通佛法,兼擅医术,后隐居剡县(今浙江嵊州)。支公:支遁,为东晋名僧,善玄理,是当时佛学“般若学”的代表人物,多才艺,长于草隶。与王洽、刘惔、殷浩、许询、郗超、王羲之、谢安等名流游好。常:同“尝”,曾经。},后情渐归支\footnote{情:指众人情意。},意甚不分\footnote{不分:不服气。分,通“忿”,纷欣阁本作“忿”。},遂遁迹剡下\footnote{剡下:剡县一带。}。遣弟子出都\footnote{出:赴、往。都:京都。},语使过会稽。于时支公正讲\CJKunderwave{小品}\footnote{\CJKunderwave{小品}:佛经的简本。}。开戒弟子:“道林讲,比汝至\footnote{比:等到。},当在某品中。”因示语攻难数十番\footnote{示:演示。攻难:进攻、诘难。番:辩论一个回合。},云:“旧此中不可复通。”弟子如言诣支公。正值讲,因谨述开意。往反多时\footnote{往反:反复辩难。},林公遂屈,厉声曰:“君何足复受人寄载来\footnote{何足:何必。寄载:传言、授意。}!”{\fzxk\zihao{6}\textcolor{red}{\CJKunderwave{名德沙门题目}曰:“于法开才辩从横,以数术弘教。”\CJKunderwave{高逸沙门传}曰:“者开初以义学法者(著)名,后与支遁有竞,故遁居剡县,更学医术。”}}

{\cangkai\zihao{5}【评】\CJKunderwave{高僧传}载,于法开的特点是“深思孤发,独见言表”,并且“才辩纵横”,对佛经\CJKunderwave{小品}也深有研究和感悟;支遁讲\CJKunderwave{小品},发挥其长于论谈,辞藻映发的优势,所以在义理方面于法开不以支遁为高,而支遁独得高名,于法开便“意甚不分”,令弟子特意“过会稽”与之辩难。支遁之窘,见出其对于\CJKunderwave{小品}果有未通处,但因此失态,则有失名僧风度。王世懋评点:“此亦岂是求理于谈?”}

{\cangkai\zihao{5}余嘉锡先生评本则:“本篇云支公讲\CJKunderwave{小品},于法开戒弟子示语攻难数十番,云‘旧此中不可复通’,弟子如言,往反多时,林公遂屈。渊源(按,渊源,殷浩字,事见本篇四十三则)所签世之幽滞,必有即法开所谓‘旧不可通’者。然则渊源之所不解者,道林亦未必尽解也。右军惧其败名,可谓‘爱人以德’,林公遂不复往,亦庶乎知难而退矣。”}

{\cangkai\zihao{5}无论从哪个角度看,它都反映了当时名士重名的世风,僧俗皆然,而本则便是佛门争名的生动一例。}

\lettrine{4.46} 殷中军\myidx{殷浩}问\footnote{殷中军:殷浩,(?—356):见刘孝标注。浩善谈玄,负盛名,简文执政时惧桓温势盛,引浩为建武将军、扬州刺史,以对抗桓温。后因北征许洛败绩,为桓温所弹,废为庶人。}:“自然无心于禀受\footnote{自然:大自然;上天。禀受:赋予。受,通“授”。},何以正善人少\footnote{正:只。},恶人多?”诸人莫有言者,刘尹\myidx{刘惔}答曰\footnote{刘尹:刘惔,字真长,曾任丹阳尹,故称。谢安妻兄,尚明帝女庐陵公主。会稽王司马昱为相,与王濛并为其座上清谈之客。性简贵自重,与王羲之友善。卒年三十六。}:“譬如写水着地\footnote{写:即泻,倾泻。},正自纵横流漫,略无正方圆者\footnote{略无:完全没有。}。”一时绝叹,以为名通\footnote{通:解说义理,使之通畅。}。{\fzxk\zihao{6}\textcolor{red}{\CJKunderwave{庄子}曰:“天籁者,吹万不同,而使其自己也。”郭子玄\CJKunderwave{注}曰:“无既无矣,则不能生有。有之未生,又不能为生。然则生生者谁哉?块然而自生耳,非我生也。我不生物,物不生我,则自然而已,然谓之天然。天然非为也,故以天言之,所以明其自然故也。”}}

{\cangkai\zihao{5}【评】殷浩之问,是在一本正经地谈玄理名实问题,并采用郭象的\CJKunderwave{庄子}解说。郭象的观点是“独化”论,即万物之生,没有什么造物主,一切禀受自然,是物自生、自己使之然的,自己生成什么样子就是什么样子,没有外部因素对他起作用(参见刘孝标注)。既然如此,殷浩的问题来了,怎么自生出来的多是“恶人”呢?刘惔用一个精妙的比喻解说了此问的义理,正如泻水自流,没有正好是方或圆的形状,因而,“善人”、“恶人”既是自生,怎么可以期望他合于你主观上的一定的规范呢?玄深的义理,被一个妙喻说得深入浅出,活灵活现。本则让人清晰地看到,刘惔对\CJKunderwave{老}、\CJKunderwave{庄}的修养程度,及辩理之才,使其面对玄理问题,显得举重若轻、挥洒自如,谈士的理趣、情趣,也因此而生动地展示出来。}

\lettrine{4.47} 康僧渊\myidx{康僧渊}初过江\footnote{康僧渊:东晋高僧,本西域人,生于长安,晋成帝时南渡。后在豫章立寺讲经,以精于佛理著名于世。},未有知者,恒周旋市肆\footnote{恒:常。周旋:盘桓;来往。市肆:集市。},乞索以自营\footnote{乞索:乞讨。自营:自己维持生计。}。忽往殷渊源\myidx{殷浩}许\footnote{殷渊源:殷浩,(?—356):见刘孝标注。浩善谈玄,负盛名,简文执政时惧桓温势盛,引浩为建武将军、扬州刺史,以对抗桓温。后因北征许洛败绩,为桓温所弹,废为庶人。许:处所。},值盛有宾客,殷使坐,粗与寒温\footnote{粗:略。寒温:寒暄。},遂及义理\footnote{义理:玄学道理。}。语言辞旨,曾无愧色\footnote{曾:竟。}。领略粗举\footnote{领略:理会,解悟。粗举:大略阐释。},一往参诣\footnote{参诣:进入并达到了高深的境界。}。由是知之。{\fzxk\zihao{6}\textcolor{red}{僧渊氏族所出,未详。疑是胡人。尚书令沈约撰\CJKunderwave{晋书},亦称其有义学。}}

{\cangkai\zihao{5}【评】\CJKunderwave{高僧传}说康僧渊,“容止详正,志业弘深”,在南渡之前,便深究佛理,诵\CJKunderwave{放光}、\CJKunderwave{道行}二\CJKunderwave{般若}(即\CJKunderwave{大品}、\CJKunderwave{小品})。其性“以清约自处”,经常是乞食自资,“人未之识”。看来,他对义理思辨是早有准备的,并非凡品;但其性行却容易使人以为不高贵而予以漠视,如沧海遗珠。正因为如此,享有高名的殷浩,初对之不加礼遇,略事寒暄,就测验义理,而这位高僧毫不含糊,快捷领会、概括义理命题,并直入高深境界。这里又刻画了另一类名僧的风采,所谓“由是知之”,便是折服了举座宾客,也包括主人殷浩在内。主人态度的前倨后恭,说明了当时士人重思辨、重才情的时代风习。}

\lettrine{4.48} 殷\myidx{殷浩}、谢\myidx{谢安}诸人共集\footnote{殷:殷浩,(?—356):见刘孝标注。浩善谈玄,负盛名,简文执政时惧桓温势盛,引浩为建武将军、扬州刺史,以对抗桓温。后因北征许洛败绩,为桓温所弹,废为庶人。谢:谢安,(?—358):字无奕,谢安长兄,陈郡阳夏谢氏家族在东晋初期的代表人物之一。集:集会。}。{\fzxk\zihao{6}\textcolor{red}{殷浩、谢安。}} 谢因问殷:“眼往属万形\footnote{属:跟随、接触。万形:万物。刘孝标注引\CJKunderwave{成实论}为印度佛学中的小乘佛学经典,认为:人没有永恒的独立实体,本是空的;宇宙万有也是空的,即人空,“法”亦空。},万形入眼不?”{\fzxk\zihao{6}\textcolor{red}{\CJKunderwave{成实论}曰:“眼识不待到而知,虚尘假空与明,故得见色。若眼到色到,色闻则无空明。如眼触目,则不能见色。当知眼识不到而知。”依如此说,则眼不往,形不入,遥属而见也。谢有问,而殷无答,疑阙文。}}

{\cangkai\zihao{5}【评】佛教到了东晋大为兴盛,对清谈名士说来,佛学的“性空”理论,是一种比落实到现实王权的本土玄学理论,更具有思辩理趣、对人们智慧更富有挑战意味的认识论,很合于清谈家们的口味。这里谢安追问的就是佛理,辨析“空”、“有”问题,而不是眼睛看万物,万物定入眼的常识。依佛学理论:宇宙万有,都是因缘和合而生的假相,其暂时的因缘,必将散成空幻。万有本身并无真实存在的“自性”,一切都是“空”的。一切被人们所见的“有”,都是虚假的,刹那生灭的。那么,谢安追问,“万形”能来入眼么?眼前的“万形”都是空的,虚假的。刘注说:“谢有问,殷无答,疑阙文。”他们究竟讨论到了什么程度,不得其详了,但玄家的思维进入空灵、缥缈之境的那种情态,却在这里生动表现了出来。}

\lettrine{4.49} 人有问殷中军\myidx{殷浩}\footnote{殷中军:殷浩,(?—356):见刘孝标注。浩善谈玄,负盛名,简文执政时惧桓温势盛,引浩为建武将军、扬州刺史,以对抗桓温。后因北征许洛败绩,为桓温所弹,废为庶人。}:“何以将得位而梦棺器\footnote{得位:得到官位。棺器:棺材。},将得财而梦屎秽\footnote{矢秽:粪便秽物。}?”殷曰:“官本是臭腐,所以将得而梦棺尸;财本是粪土,所以将得而梦秽污。”时人以为名通\footnote{通:本为解说义理,使之通畅,此“名通”意谓至理名言。}。

{\cangkai\zihao{5}【评】余嘉锡先生引\CJKunderwave{晋书·索𬘘传}:“索充初梦天上有二棺落充前。𬘘曰:‘棺者,职也。当有京师贵人举君,二官者,频再迁。’俄而司徒王戎书属太守,使举充。太守先属充功曹,而举孝廉。”并谓:“此即所谓将得位而梦棺也。”得位梦棺、得财梦矢秽是民间迷信观念,本无可论证。但殷浩乐辩名实,于是当作名理给分析了一回。词面近乎调侃,词底表白了一种清高,同时又寓意精深。故“时人以为名通”,则亦是喜剧意味。殷浩的前期亦果如其言,朝廷累征不就,隐居“几将十年”,可见所谈所行,颇具清雅名士风格。}

\lettrine{4.50} 殷中军\myidx{殷浩}被废东阳\footnote{殷中军:殷浩,(?—356):见刘孝标注。浩善谈玄,负盛名,简文执政时惧桓温势盛,引浩为建武将军、扬州刺史,以对抗桓温。后因北征许洛败绩,为桓温所弹,废为庶人。被废东阳:殷浩因北伐失败,被桓温弹劾罢职,居东阳(今浙江金华)为民。},{\fzxk\zihao{6}\textcolor{red}{浩黜废事,别见。}} 始看佛经。初视\CJKunderwave{维摩诘}\footnote{\CJKunderwave{维摩诘}:即\CJKunderwave{维摩诘所说经},是佛家大乘教义的经典。},{\fzxk\zihao{6}\textcolor{red}{僧肇注\CJKunderwave{维摩经}曰:“维摩诘者,秦言净名,盖法身之大士,见居此土,以弘道也。”}} 疑“般若波罗蜜”太多,后见\CJKunderwave{小品}\footnote{\CJKunderwave{小品}:佛经简本。},恨此语少\footnote{恨:遗憾。}。{\fzxk\zihao{6}\textcolor{red}{波罗蜜,此言到彼岸也。\CJKunderwave{经}云:“到者有六焉:一曰檀,檀者,施也。二曰毗黎,毗黎者,持戒也。三曰羼提,羼提者,忍辱也。四曰尸罗,尸罗者,精进也。五曰禅,禅者,定也。六曰般若,般若者,智慧也。然则五者为舟,般若为导,导则为绝有相之流,升无相之彼岸也。故曰波罗蜜也。”渊源未畅其致,少而疑其多;已而究其宗,多而患其少也。}}

{\cangkai\zihao{5}【评】殷浩废居东阳为民,这是他平生所经历的最惨痛的一段心灵苦难。殷浩充当了简文司马昱对抗桓温的马前卒而不自知,王羲之劝他与桓温改善关系,他也没理会,终于在穆帝永和九年(353),北伐失败的时候,桓温上疏弹劾,将他废为庶人。而司马昱为了与颇有野心而早具威势的桓温保持关系,竟舍弃了殷浩这个马前卒。殷浩恨不已:简文把自己推上了百尺楼,却撤了梯子。(参见\CJKunderwave{世说·黜免})对发生的一切,他百思不得其解,终日以指划空写“咄咄怪事!”在这样的背景中,他向佛经求取解脱。“识度清远”,对玄学经典深有领会的他,很快就进入了佛学境界。}

{\cangkai\zihao{5}“般若”是达成佛境,超越一切经验、知识,体悟万物性空的最高智慧。是成佛的直观、直觉。“波罗蜜”是解脱此岸苦海,达于彼岸的途径、方法,亦为达于彼岸,证成正果。大乘般若学的经典,皆引人修行“般若波罗蜜”,普度众生求取解脱登彼岸。因此,殷浩初读对“般若波罗蜜”感受不深,疑其太多;悟入经典,便恨其少。殷浩的聪颖与苦难,就在这一“疑”、一“恨”的过程中生动起来了。玄学已经救治不了他的苦难心灵了,于是他舍舟登岸,归向了当时盛行的“般若”之学,以其“识度清远”的天资,实现了最后一段生命里程的价值。}

\lettrine{4.51} 支道林\myidx{支遁}、殷渊源\myidx{殷浩}俱在相王\myidx{司马昱}许\footnote{支道林:支遁,为东晋名僧,善玄理,是当时佛学“般若学”的代表人物,多才艺,长于草隶。与王洽、刘惔、殷浩、许询、郗超、王羲之、谢安等名流游好。常:同“尝”,曾经。殷渊源:殷浩,(?—356):见刘孝标注。浩善谈玄,负盛名,简文执政时惧桓温势盛,引浩为建武将军、扬州刺史,以对抗桓温。后因北征许洛败绩,为桓温所弹,废为庶人。相王:东晋简文帝司马昱,曾以会稽王任丞相职,故称。许:处所。}。{\fzxk\zihao{6}\textcolor{red}{简文。}} 相王谓二人:“可试一交言\footnote{交言:此指清谈。}。而‘才性’殆是渊源崤、函之固\footnote{才性:即“才性论”,是关于才、性内涵及其关系的理论,也是名实之论,为魏晋玄学的重要部分之一。},{\fzxk\zihao{6}\textcolor{red}{崤,谓二陵之地。函,函谷关也。并秦之险塞,王者之居。左思\CJKunderwave{魏都赋}曰:“崤、函帝王之宅。”}} 君其慎焉!”支初作,改辙远之\footnote{改辙:此谓改变论题,远避“才性”。},数四交,不觉入其玄中\footnote{玄中:玄理之中。}。相王抚肩笑曰:“此自是其胜场\footnote{胜场:胜过别人的地方。},安可争锋!”

{\cangkai\zihao{5}【评】李慈铭\CJKunderwave{世说新语批注}:“此谓殷之‘才性’无人可敌,如崤、函之固。即前所云殷中军于‘才性’偏精也。”简文的比喻,准确而有震撼力。\CJKunderwave{左传}记秦、晋崤之战,透露了崤、函之险,秦国大军于此全军覆没,将帅被擒。这是难以渡越的雄关、险隘。殷浩守“才性”之论的险、固如是,在他面前便都是败将。以辞理知名的支遁和殷浩相遇于“才性”,败得着实狼狈。本则以名僧支遁作正衬,突显了殷浩的才辩风貌,生动描绘了他作为一时谈宗的真才实学和风流雅望。}

\lettrine{4.52} 谢公\myidx{谢安}因子弟集聚\footnote{谢公:谢安,(?—358):字无奕,谢安长兄,陈郡阳夏谢氏家族在东晋初期的代表人物之一。因:趁。},问\CJKunderwave{毛诗}何句最佳\footnote{\CJKunderwave{毛诗}:汉代四家传授\CJKunderwave{诗经},其中毛亨所传称\CJKunderwave{毛诗},即今本\CJKunderwave{诗经},属经古文学派。}?遏称\myidx{谢玄}曰:{\fzxk\zihao{6}\textcolor{red}{谢玄小字。已见。}} “昔我往矣,杨柳依依;今我来思,雨雪霏霏\footnote{昔我往矣:\CJKunderwave{诗经·小雅·采薇}诗句。通过景物描写,表达了戍卒征战久久不能还家的哀苦。依依:茂盛貌。思:语末助词。霏霏:雪大貌。}。”公曰:“訏谟定命,远猷辰告\footnote{“訏谟”二句:\CJKunderwave{诗经·大雅·抑}诗句。写王朝的宰辅之臣应当是:用深谋远虑来确定大计方针,将长远的国策及时遍告群臣。訏,大。谟,谋。猷,谋略。辰,时。}。”{\fzxk\zihao{6}\textcolor{red}{\CJKunderwave{大雅}诗也。毛苌\CJKunderwave{注}曰:“訏,大也。谟,谋也。辰,时也。”郑玄\CJKunderwave{注}曰:“猷,图也。大谋定命,谓正月始和,布政于邦国都鄙。”}} 谓此句偏有雅人深致\footnote{偏:最。雅人:志趣高尚之人。深致:深远的情致。}。

{\cangkai\zihao{5}【评】本则生动展现士族名门教育家族子弟、注重文化艺术熏陶的风貌。谢安深爱家族子弟,\CJKunderwave{世说}中记述很多。本则谈论经典、艺文,则特见出他对后辈的关怀,就是让他们有深厚扎实的修养,成为生于阶庭的芝兰玉树。\CJKunderwave{诗经}既是儒家经典,也是优美艺文。作为经典,它主要的特点是给人以性情陶冶,即所谓“思无邪”;作为艺文,它可说是后代诗家之祖,锻炼人驾驭语言的能力。本则在这似乎平常的生活一幕中,给人以深刻印象的,是陈郡谢氏的雅致家风。他们讨论的是\CJKunderwave{雅}诗部分,谢玄指认的佳句,表现了那一时代和家风赋予他的艺术感悟力。在谢玄之前是经学解\CJKunderwave{诗},从艺术角度体会妙处,他是很早的一位,因此被鉴赏家们公认为名句,千载之下,不断评说。该句写物态、慰人情,于情中写景,景中见情,倍增哀乐的表现力,确实是古典诗歌中第一流的。谢玄以此句为佳,当是从画面境象之美和对人情的感动来体认的。这是时风重美、重情的必然反响,同时也看到了其人的性情。谢安所认为的佳句,说是“雅人深致”,后人从艺术角度看,颇不以为然。但这并不意味谢安鉴赏能力差,而是他的个性、阅历使然。这句出自\CJKunderwave{大雅}的句子,深沉有气度,是宰相之才的境界。如诗篇所言,辅佐君王,修明政治,平息纷乱、怨艾,使天下、宗族和靖,这正是宰相之才最完美的表达。两人鉴赏倾向不同,深层次的东西,是此时两人的阅历、身份所决定的对人生感受的差异,而绝非纯粹文艺批评意义上的价值判断。}

\lettrine{4.53} 张凭\myidx{张凭}举孝廉\footnote{孝廉:汉代察举人才的科目,魏晋仍沿此制。意为孝悌、廉洁,由乡议荐至郡国,再推举到朝廷,考核后授以官职。},出都\footnote{出都:到京城。},负其才气,谓必参时彦\footnote{彦:才能杰出的人。}。欲诣刘尹\myidx{刘惔}\footnote{诣:拜访。刘尹:刘惔,字真长,曾任丹阳尹,故称。谢安妻兄,尚明帝女庐陵公主。会稽王司马昱为相,与王濛并为其座上清谈之客。性简贵自重,与王羲之友善。卒年三十六。},乡里及同举者共笑之。张遂诣刘。(刘)洗濯料事\footnote{洗濯:清洗、洗理。料事:处理事物。袁本“洗濯”前增一“刘”字,是。},处之下坐,唯通寒暑,神意不接。张欲自发无端\footnote{无端:无由。}。顷之,长史\myidx{王濛}诸贤来清言\footnote{长史:即王濛。},客主有不通处,张乃遥于末坐判之,言约旨远,足畅彼我之怀,一坐皆惊。真长\myidx{刘惔}延之上坐,清言弥日,因留宿至晓。张退,刘曰:“卿且去,正当取卿共诣抚军\myidx{司马昱}\footnote{抚军:将军称号。此指简文帝司马昱,曾做抚军将军,掌国政。}。”张还船,同侣问何处宿,张笑而不答。须臾,真长遣传教觅张孝廉船\footnote{传教:此指持信幡传达教令的官吏。教,王、侯大臣发布的命令。},同侣惋愕\footnote{惋愕:感叹惊诧。}。即同载诣抚军,至门,刘前进谓抚军曰:“下官今日为公得一太常博士妙选\footnote{太常博士:官名。太常的属官, 执掌引导乘舆, 议定王公以下谥号等。妙选: 最佳人选。}。”既前,抚军与之话言,咨嗟称善,曰:“张凭勃窣为理窟\footnote{勃窣:晋人口语,形容才气纵横,辞采丰富。理窟:义理的渊薮。}。”即用为太常博士。{\fzxk\zihao{6}\textcolor{red}{宋明帝\CJKunderwave{文章志}曰:“凭字长宗,吴郡人。有意气,为卿闾所称。学尚所得,敏而有文。太守以才选举孝廉,试策高第,为惔所举,补太常博士,累迁吏部郎、御史中丞。”}}

{\cangkai\zihao{5}【评】魏晋重才情,因而有才气的人都颇自信;才能也使人在社会上获得崇高的地位,因而名士成了时代最醒目的亮点。张凭自负其才,急欲干时彦脱颖而出;同伴对他欲参时彦之非笑怀疑,出于名士与一般人社会地位的霄壤之别,这些都说明了当时的风尚,所以急欲出名、不惜一切地保持名声,就成了当时士人的一个“情结”。张凭成功,来自他的自信,也来自他的执着。自信使他勇于干谒时彦、权门;执着使他耐得住冷遇。以刘惔之自负、自傲性格和他的地位,对无名之辈的傲慢是顺理成章的,而张凭居下坐,不失时机地展露才气,则见出他对挤入“时彦”行列的急切。他的才气和执着,使他戏剧性地获得了机遇。这机遇,一是抚军正急欲求贤,虚位待人,二是刘惔爱才。于是,成就了张凭的前程,众人也因此敬服刘惔“知人”(见\CJKunderwave{晋书·刘惔传})。一则故事,将当时重才、重名的社会风情描摹得清晰如画。}

\lettrine{4.54} 汰法师\myidx{竺法汰}云:“六通三明同归\footnote{同归:旨趣相同。},正异名耳\footnote{正:只。}。”{\fzxk\zihao{6}\textcolor{red}{\CJKunderwave{安法师传}曰:“竺法汰者,体器弘简,道情冥到,法师友而善焉。”一说法汰,即安公弟子也。\CJKunderwave{经}云:“六通者,三乘之功德也。一曰天眼通,见远方之色;二曰天耳通,闻障外之声;三曰身通,飞行隐显;四曰他心通,水镜万虑;五曰宿命通,神知已往;六曰漏尽通,慧解累世。三明者,解脱在心,朗照三世者也。”然则天眼、天耳、身通、他心、漏尽此五者,皆见在心之明也。宿命则过去心之明也。因天眼发未来之智,则未来心之明也。同归异名,义在斯矣。}}

{\cangkai\zihao{5}【评】竺法汰活跃于东晋中期,据\CJKunderwave{高僧传}记:他在建康瓦官寺讲\CJKunderwave{放光经},开题大会,简文帝亲临听讲,王侯公卿,莫不毕集。开讲之日,这位“流名四远”的名僧,吸引得士女成群。正是这些僧人的活跃、王公贵族的支持,才使得佛教的大乘般若学,盛行于东晋。竺法汰这里说的,就是大乘佛学所描绘的,因禅定而获得的般若智慧。具体的样态,就是这种智慧具有通神之力,即“六通”,参见刘孝标注。六通之力,可以透过一切现世人所认为的不可逾越的障碍,洞见天上、人间、地下的一切,慧解“真空”,脱去业惑,达于成佛境界。而“三明”,天眼明能知来世;宿命明能知前世;漏尽明能断烦恼,和“六通”是名不同而实不异。本则的动人处,不在所说的佛家教义,而是竺法汰像谈玄一样,说佛法辨析名实,可见佛家高僧很会顺应中土的文化心理结构去输入佛家教义,这是佛教能够在中国盛行的重要法宝。高僧片语,透露了个中信息。}

\lettrine{4.55} 支道林\myidx{支遁}、许\myidx{许询}、谢盛德\myidx{谢安}共集王\myidx{王濛}家\footnote{支道林:支遁,为东晋名僧,善玄理,是当时佛学“般若学”的代表人物,多才艺,长于草隶。与王洽、刘惔、殷浩、许询、郗超、王羲之、谢安等名流游好。常:同“尝”,曾经。许:许询,见\CJKunderwave{言语}69。谢:谢安,(?—358):字无奕,谢安长兄,陈郡阳夏谢氏家族在东晋初期的代表人物之一。盛德:有德行声望的人。王:王濛。},{\fzxk\zihao{6}\textcolor{red}{许询、谢安、王濛。}} 顾谓诸人:“今日可谓彦会\footnote{彦:才能杰出的人。会:集会。“顾谓”上袁本增“谢”字。},时既不可留,此集固亦难常。当共言咏\footnote{言咏:畅谈吟咏,此指畅谈。},以写其怀\footnote{写:抒发。}。”许便问主人:“有\CJKunderwave{庄子}不?”正得\CJKunderwave{渔父}一篇\footnote{\CJKunderwave{渔父}:\CJKunderwave{庄子}篇名。}。{\fzxk\zihao{6}\textcolor{red}{\CJKunderwave{庄子}曰:“孔子游乎缁帷之林,休坐乎杏坛之上。孔子弦歌鼓琴,奏曲未半,有渔者下船而来,须眉交白,被发揄袂,行原以(上),距陆而止。左手据膝,右手持颐以听。曲(终),而招子贡、子路,语曰:‘彼何为者也?’曰:‘孔氏。’曰:‘孔氏何治?’子贡曰:‘服忠信,行仁义,饰礼乐,选人伦,孔氏之所治也。’曰:‘有土之君欤?’曰:‘非也。’渔父曰:‘仁则仁矣,恐不免其身。’孔子闻而求问之,遂言八疵、四病以诚(诫)孔子。”}} 谢看题,便各使四坐通\footnote{通:阐释义理,使通畅。}。支道林先通,作七百许语,叙致精丽\footnote{叙致:陈说、叙述。},才藻奇拔\footnote{才藻:才思辞藻。},众咸称善。于是四坐各言怀。言毕,谢问曰:“卿等尽不?”皆曰:“今日之言,少不自竭。”谢后粗难,因自叙其意,作万馀语,才峰秀逸。{\fzxk\zihao{6}\textcolor{red}{\CJKunderwave{文字志}曰:“安神情秀悟,善谈玄远。”}} 既自难干\footnote{干:犯,此为企及。},加意气拟托\footnote{意气:志向、气概。拟托:比拟、寄托。},萧然自得\footnote{萧然:洒脱。},四坐莫不厌心\footnote{厌心:满足于心,心悦诚服。}。支谓谢曰:“君一往奔诣\footnote{一往:一直。奔诣:奔赴精深境界。诣,学养的造诣、境界。},故复自佳耳\footnote{故自复:确实是。}。”

{\cangkai\zihao{5}【评】陈梦槐评曰:“有此叙致,一日风流,千载可怀。”风流雅望毕集一室,正所谓“彦会”;所谈题目亦足风流。\CJKunderwave{庄子·渔父}绘声绘色地描写了执着于礼义、天下的孔子,听了渔父的高论,而愀然若有所失。文章表达了弃绝礼义,返璞归真,以求明哲保身的道理。“时彦”们以此为话题,通辩义理,同时尽情地表现了各自的才华,确是“一日风流”。观此情形,可与王羲之\CJKunderwave{兰亭集序}对读,当时的风流俊彦的集会,不仅是才华风度之雅,也有对人生哲理的深切感悟,对生命况味的细腻体会。因此,它不像\CJKunderwave{论语·子路、曾晳、冉有、公西华侍坐章}那样“彦会”所表达的“圣贤气象”,而是魏晋名士的风流。}

{\cangkai\zihao{5}就清谈形式说,本则也让我们看到了当时清谈的另一典型场景。}

\lettrine{4.56} 殷中军\myidx{殷浩}、孙安国\myidx{孙盛}、王\myidx{王濛}、谢\myidx{谢尚}能言诸贤\footnote{殷中军:殷浩,(?—356):见刘孝标注。浩善谈玄,负盛名,简文执政时惧桓温势盛,引浩为建武将军、扬州刺史,以对抗桓温。后因北征许洛败绩,为桓温所弹,废为庶人。孙安国:孙盛。王:王濛。谢:谢尚,谢豫章:谢鲲,曾作豫章太守。刘孝标注“鲲子别见”,“子”字衍。将:携,谓携之送客。自:已经。参:参与、进入。上流:上等、上品。能言:长于清谈。},悉在会稽王\myidx{司马昱}许\footnote{会稽王:晋简文帝司马昱,曾封会稽王,指晋简文帝司马昱(320—372),穆帝年幼即位,昱任抚军大将军总理政务。后来大将军桓温专擅朝政,先废海西公,后立司马昱为帝,第二年崩。许:处所。}。殷与孙共论“易象妙于见形”,{\fzxk\zihao{6}\textcolor{red}{其论略曰:“圣人知观器不足以达变,故表圆应于蓍龟;圆应不可为典要,故寄妙迹于六爻。周流唯化所适,故虽一画而吉凶并彰,微一则失之矣。拟器托象而庆咎交著,系器则失之矣。故设八卦者,盖缘化之影迹也。天下者,寄见之一形也。圆影备未备之象,一形兼未形之形。故尽二仪之道,不与\CJKunderwave{乾}、\CJKunderwave{坤}齐妙;风雨之变,不与\CJKunderwave{巽}、\CJKunderwave{坎}同体矣。”}} 孙语道合\footnote{道合:将观点阐发得圆融无间。},意气干云\footnote{干云:直上云霄。谓气旺神足。},一坐咸不安孙理\footnote{安:满意。},而辞不能屈\footnote{屈:驳倒。}。会稽王慨然叹曰:“使真长\myidx{刘惔}来\footnote{真长:刘惔,字真长,曾任丹阳尹,故称。谢安妻兄,尚明帝女庐陵公主。会稽王司马昱为相,与王濛并为其座上清谈之客。性简贵自重,与王羲之友善。卒年三十六。},故应有以制彼。”即迎真长,孙意已不如\footnote{不如:不及。意谓义理阐释已不如刚才。}。真长既至,先令孙自叙本理,孙粗说己语,亦觉绝不及向。刘便作二百许语,辞难简切\footnote{词难简切:用词语辩诘,简明切要。},孙理遂屈。一坐同时拊掌而笑\footnote{拊掌:鼓掌。},称美良久。

{\cangkai\zihao{5}【评】\CJKunderwave{晋书·刘惔传}移录本则,说孙盛作了\CJKunderwave{易象妙于见形论}在简文处讨论。刘注称引“其论略曰”一段,其作者究竟是孙盛还是殷浩,历来看法不同。\CJKunderwave{晋书·刘惔传}以为作者是孙盛,而严可均则认为是殷浩。详味刘注,合于王弼\CJKunderwave{易}注精神,应是殷浩等玄家之言。孙盛\CJKunderwave{易象妙于见形论},原文已佚。这则故事,活脱脱地画出了清谈时诸家争鸣的热闹场面。故事中除孙盛思想近于儒家外,其馀诸人皆为清一色的玄学名家,争辩双方力量极不均衡。玄家\CJKunderwave{易}论,祖祧王弼\CJKunderwave{易}注,其\CJKunderwave{周易略例·明象},提出了“得象而忘言”,进一步达于“得意而忘象”的形而上境界,对于形而下之“器”,如象数之类,则尽皆摈落而不惜。王弼开启了玄学中的言、意之辩。这一理论为两晋玄家所继承,成为当时的学术主流。而从思想体系看,孙盛继承的是汉儒的象数\CJKunderwave{易}学,属儒学系统。据\CJKunderwave{广弘明集}卷五,孙盛曾撰\CJKunderwave{老聃非大圣论}、\CJKunderwave{老子疑问反讯}诸论,明确批判王弼\CJKunderwave{易}注及玄家之言的“笼统玄旨”,所论“皆妄”,认为玄家抛弃汉儒象数\CJKunderwave{易}说,虽然“丽辞溢目”,但却“泥夫大道”。(其言见于\CJKunderwave{三国志·魏书·锺会传}裴注称引)可见双方辩家观点的势不两立。但在这场论战中,他不怕孤立,迎战诸玄家围攻而毫无惧色,读来颇令人想到后来\CJKunderwave{三国演义}诸葛亮舌战群儒的风采。}

{\cangkai\zihao{5}本则故事,依出场顺序,第一主角当为孤军奋战的孙盛。在此思辨性的思想论争中,任何权势都不起作用,就是会稽王司马昱也只好坐在观众席,孙盛之勇全恃其学养、辩才,一座玄家都“不能屈”,可见其才学之锐。第二主角是刘惔。其自负为第一流清谈家(见\CJKunderwave{晋书·刘惔传}),果然不同凡响。尚未出场,就如见其人,如闻其声,已自震慑了论辩双方。一旦正式交锋,便“孙理遂屈”,顺理成章地刻画出了刘惔的杰出辩才。故事着墨不多,但正衬、反衬之法,使得刘惔形象具先声夺人之妙。凌濛初成以“形态逼真”评论故事人物形象之鲜活灵动,可谓一语中的。}

\lettrine{4.57} 僧意\myidx{僧意}在瓦官寺中\footnote{瓦官寺:佛寺名。东晋哀帝兴宁二年(364)造,初名慧方寺,寺有瓦官阁,在建康城西南隅。},{\fzxk\zihao{6}\textcolor{red}{未详僧意氏族所出。}} 王苟子\myidx{王循}来,{\fzxk\zihao{6}\textcolor{red}{苟子,王循(修)小字。}} 与共语,便使其唱理\footnote{唱理:倡言义理。即首先陈说义理。}。便谓王曰:“圣人有情不?”王曰:“无。”重问曰:“圣人如柱邪?”王曰:“如筹算\footnote{筹算:古代计算用的竹刻筹码。}。虽无情,运之者有情。”僧意云:“谁运圣人邪?”苟子不得答而去。{\fzxk\zihao{6}\textcolor{red}{诸本无僧意最后一句,意疑其阙。广校众本皆然,唯一书有之,故取以成其义。然王循(修)善言理,如此论,特不近人情,犹疑斯文为谬也。}}

{\cangkai\zihao{5}【评】\CJKunderwave{三国志·锺会传}注引何劭的\CJKunderwave{王弼传}说:“何晏以为圣人无喜、怒、哀、乐,其论甚精,锺会等述之。王弼与不同。”这是魏晋玄家“有”、“无”之辩的一个话题。此说者认为圣人“与无同体”,自然无情,无情比有情更其高超。这里王修同意何晏的说法,坚持“圣人无情”说,但他的比喻留下了漏洞,意在圣人如同算学上的筹码,运化无穷,并不受到世俗人情的左右,只是运用他的人有情。但“谁运圣人邪?”此一追问,王修之论的漏洞就暴露出来了,他不能自圆其说,“不答而去”。从何晏谈到了王修,这一说还未能圆满,加之论说者辩才有限,故事就塑造出了一个一辩即败的喜剧清谈家的形象。而僧意虽名不见经传,但他反应机敏,思维深邃,借力发力,再三追问,又显得风趣无穷。}

\lettrine{4.58} 司马太傅\myidx{司马道子}问谢车骑\myidx{谢玄}\footnote{司马太傅:司马道子(364—403),简文帝子。孝武帝时,官太子太傅、扬州刺史、都督中外诸军事,宰执朝政。谢车骑:谢玄。}:“惠子\myidx{惠子}其书五车\footnote{惠子:惠施,战国时著名的名家代表人物。},何以无一言入玄\footnote{玄:玄理。}?”谢曰:“故当是其妙处不传\footnote{故当:可能、或许,表示推测。}。”{\fzxk\zihao{6}\textcolor{red}{\CJKunderwave{庄子}曰:“惠施多方,其书五车,其道舛駮,其言不中。谓卵有毛,鸡三足,马有卵,犬可为羊,火不热,目不见,龟长于蛇,丁子有尾,曰(白)狗黑,连环可解。能胜人之口,不能服人之心。盖辩者之囿也。”}}

{\cangkai\zihao{5}【评】惠子和公孙龙子一样,都是杰出的辩者。\CJKunderwave{庄子}书中多记庄周与惠子交游谈辩。然惠子文章无传,在\CJKunderwave{庄子·天下篇}谈论天下各派学说时,言及惠子所提出的辩题(参见刘注所引),亦皆晦涩难懂。司马道子不解的是,既然\CJKunderwave{庄子}为玄理重要蓝本,而庄子又多与惠子辩,惠子博学善辩,怎么\CJKunderwave{庄子}里没提到惠子论证玄理,他自己也无任何著作流传呢?善名实之辩的名家人物,见解不传,真是遗憾。谢玄轻描淡写地回应了他,一方面是惠子名理之学的确难懂,另一方面,司马道子也不是一个热衷玄学的清谈家,整天酣歌醉饮,结党弄权,并无兴趣谈玄论道。所以二人问答,一方是偶一问之,一方轻拂而过,彼此都没有那种非辩难究诘,谈出结果不可的兴味和激情。本则可见,谈玄极富挑战性,是很有意味的精神活动,但却与世间俗人无缘。}

\lettrine{4.59} 殷中军\myidx{殷浩}被废\footnote{殷中军被废:殷浩,参见本篇第50则注、评。},徙东阳,大读佛经,皆精解\footnote{精解:深入透彻地理解。}。唯至“事数”处不解。{\fzxk\zihao{6}\textcolor{red}{事数:谓若五阴、十二入、四谛、十二因缘,五根、五力、七觉之属。}} 遇见一道人\footnote{道人:当时对僧人的称谓。},问所签\footnote{签:读书有疑难而作的标记。},便释然\footnote{释然:疑虑消解的样子。}。

{\cangkai\zihao{5}【评】殷浩学佛,目的在解脱痛苦,以平衡被贬废后的心理。但佛理艰深,非苦心孤诣,则难入其门。数处“不解”,必为障碍,因而苦觅其解。所谓“事数”:五阴、十二入、四谛、十二因缘、五根、五力、七觉之属,都是佛学理论的基础,不解这些基本说法,则“般若”之说就无以附着。当殷浩发宏愿,要“精解”佛理的时候,自然要搞清这些问题,必问明而后止。本则与第43则及50则,互为发明,足见殷浩在政治失败后转入学问思辨的精诚勤苦。}

\lettrine{4.60} 殷仲堪\myidx{殷仲堪}精覈玄论\footnote{殷仲堪:(?—399):善清谈,当时与韩康伯齐名。精覈:深入考索。玄论:玄理。},人谓莫不研究。殷乃叹曰:“使我解\CJKunderwave{四本}\footnote{四本:即锺会撰\CJKunderwave{四本论}。},谈不翅尔\footnote{不翅:同“不啻",不仅、不止。尔:这样。}。”{\fzxk\zihao{6}\textcolor{red}{周祇\CJKunderwave{隆安记}曰:“仲堪好学而有理思也。”}}

{\cangkai\zihao{5}【评】殷仲堪“能清言,善属文”,“其谈理与韩康伯齐名,士咸爱慕之”(见\CJKunderwave{晋书·殷仲堪传}),是深有造诣,广有影响的清谈家,但于“四本”仍有未尽其妙处,故对实际问题深入不下去。在这点上,他便和精于“才性”、“四本”的殷浩(参见本篇51则评)形成了鲜明的对比和档次差异。可见,“才性”、“四本”是测试玄家理论思辨水平的一块试金石,殷仲堪面对它的感喟、遗憾,生动地说明了这个问题。}

\lettrine{4.61} 殷荆州\myidx{殷仲堪}曾问远公\myidx{释惠远}\footnote{殷荆州: 即殷仲堪, 曾任荆州刺史, (?—399):善清谈,当时与韩康伯齐名。}:{\fzxk\zihao{6}\textcolor{red}{张野\CJKunderwave{远法师铭}曰:“沙门释惠远,雁门楼烦人。本姓贾氏,世为冠族,年十二,随舅令狐氏游学许、洛。年二十一,欲南渡,就范宣子学,道阻不通,遇释道安以为师。抽簪落发,研求法藏。释昙翼每资以灯烛之费。识鉴淹远,高悟冥赜。安常叹曰:‘道流东国,其在远乎?’襄阳既没,振锡南游,结字(宇)灵岳。自年六十,不复出山。名被流沙,彼国僧众,皆称汉地有大乘沙门。每至然香礼拜,辄东向致敬。年八十三而终。”}} “\CJKunderwave{易}以何为体\footnote{体:本体,根本。}?”答曰:“\CJKunderwave{易}以感为体\footnote{感:感应。}。”殷曰:“铜山西崩,灵钟东应,便是\CJKunderwave{易}耶?”{\fzxk\zihao{6}\textcolor{red}{\CJKunderwave{东方朔传}曰:“汉武皇帝时,未央宫前殿钟无故自鸣,三日三夜不止。诏问太史待诏王朔,朔言恐有兵气。更问东方朔,朔曰:‘臣闻铜者山之子,山者铜之母,以阴阳气类言之,子母相感,山恐有崩㢮者,故钟先鸣。\CJKunderwave{易}曰:“鸣鹤在阴,其子和之。”精之至也。其应在后五日内。’居三日,南郡太守上书言山崩,延袤二十馀里。”\CJKunderwave{樊英别传}曰:“汉顺帝时,殿下钟鸣,问英。对曰:‘蜀、㟭山崩。山于铜为母,母崩子鸣,非圣朝灾。’后蜀果上山崩,日月相应。”二说微异,故并载之。}} 远公笑而不答。

{\cangkai\zihao{5}【评】\CJKunderwave{易}强调“感”,\CJKunderwave{咸·彖传}揭明其义:“天地感而万物化生,圣人感人心而天下和平。观其所感,而天地万物之情可见矣。”没有阴阳交感的互动就没了事物,也就没了变化无穷的\CJKunderwave{易},因而体会“\CJKunderwave{易}以感为体”,是把握\CJKunderwave{易}的关键。但\CJKunderwave{易}之“感”,讲的是天地万物的规律,是无微而不至的,万事万物皆有“感”,无“感”则不通,不通则病。殷仲堪所言是一个具体的事例,机械比附,未尽\CJKunderwave{易}理。所以王世懋评说:“按\CJKunderwave{易}理精微广大,谓此非\CJKunderwave{易}不可,执此言\CJKunderwave{易}又不可,远公所以笑而不答。”}

{\cangkai\zihao{5}慧远少时即“博综六经,尤善\CJKunderwave{庄}、\CJKunderwave{老}”,后闻\CJKunderwave{般若经},“豁然而悟,乃叹曰:‘儒道九流,皆糠秕耳。’”(见\CJKunderwave{高僧传})既是玄家,也是高僧,他的笑而不答,颇有居高临下的意味。一是回答清楚,要花一番辨析的工夫;二是殷仲堪如此解\CJKunderwave{易},糊涂可笑。慧远一“笑”,其无穷意韵,尽在不言中。}

\lettrine{4.62} 羊孚\myidx{羊孚}弟\myidx{羊辅}娶王永言\myidx{王讷之}女\footnote{羊孚:见刘孝标注。羊后投桓玄,玄用为记室参军,为桓心腹。泰山人,官历太学博士、兖州别驾、太尉记室参军。}。{\fzxk\zihao{6}\textcolor{red}{孚弟,辅也。\CJKunderwave{羊氏谱}曰:“辅字幼仁,太山人。祖楷,尚书郎。父绥,中书郎。辅仕至卫军功曹。娶琅邪王讷之女,字僧首。”}} 及王家见婿,孚送弟俱往。时永言父东阳\myidx{王临之}尚在\footnote{东阳:指东阳太守王临之。},{\fzxk\zihao{6}\textcolor{red}{\CJKunderwave{王氏谱}曰:“讷之字永言,琅邪人。祖彪之,光禄大夫。父临之,东阳太守。讷之历尚书左丞、御史中丞。”}} 殷仲堪\myidx{殷仲堪}是东阳女婿\footnote{殷仲堪:(?—399):善清谈,当时与韩康伯齐名。},亦在坐。{\fzxk\zihao{6}\textcolor{red}{\CJKunderwave{殷氏谱}曰:“仲堪娶琅邪王临之女,字英彦。”}} 孚雅善理义,乃与仲堪道\CJKunderwave{齐物}\footnote{\CJKunderwave{齐物}:即\CJKunderwave{庄子·齐物论}。}。{\fzxk\zihao{6}\textcolor{red}{\CJKunderwave{庄子}篇也。}} 殷难之,羊云:“君四番后\footnote{番:次,回合。当得:一定会。},当得见同\footnote{见同:与我相一致。}。”殷笑曰:“乃可得尽,何必相同?”乃至四番后一通\footnote{一通:一致相通。}。殷咨嗟曰:“仆便无以相异。”叹为新拔者久之\footnote{新拔:新颖特出。}。

{\cangkai\zihao{5}【评】\CJKunderwave{齐物论}是庄子另一篇著名的代表作品,他用相对论来说明天道自然,创造出齐生死,一是非的论断。本来寿夭、美丑、是非等,是客观存在的差异,要把它们论证成没有对立,没有差异,符合庄子的说法,从而证明天道自然,是需要一些辩才的。殷仲堪以为难,所以拿它为题来发难,其自负名家,藐视对手,在此场合显露一下风流的用心是显而易见的。然而具有太学博士水平的羊孚也不含糊,许下你四个回合,便与我的见解一致,你必定跟着我的思路走。果然殷仲堪被征服了。“仆便无以相异”,是叹服,只能被动俯就,无可逃离。与羊孚的思维水平相比,殷仲堪明显相形见绌,只能认输叹服。照说,在羊孚面前殷仲堪是长一辈的人,并且地位、声望皆高于羊孚,但他并不以为丢面子,还将对手大加褒扬,可见清谈本是一桩雅事,也可见殷仲堪颇为温厚的一面。}

{\cangkai\zihao{5}如果将殷仲堪与本篇前面所记的诸如殷浩、支道林等等名士的风流才情相比,他作为玄家之才情风貌便显得清晰如绘了。}

\lettrine{4.63} 殷仲堪\myidx{殷仲堪}云\footnote{殷仲堪:(?—399):善清谈,当时与韩康伯齐名。}:“三日不读\CJKunderwave{道德经}\footnote{\CJKunderwave{道德经}:即\CJKunderwave{老子}。},便觉舌本间强\footnote{舌本:舌根。间:处。强:僵硬。}。”{\fzxk\zihao{6}\textcolor{red}{\CJKunderwave{晋安帝纪}曰:“仲堪有思理,能清言。”}}

{\cangkai\zihao{5}【评】\CJKunderwave{老子}为古代哲学经典,以无为本构建了宇宙观、人生观,其间丰富的辩证法思想和作为宇宙观、人生观的“道”,启人深思。无论是谈玄的需要,还是作为好学深思的读书人的习性,\CJKunderwave{老子}都是磨砺思维的最好工具,也是一个理趣的渊薮。殷仲堪本“好学而有理思”,又是一个以“能清言”著名的人物(见\CJKunderwave{晋书·殷仲堪传}),所以研习\CJKunderwave{老子}孜孜不倦。一席话正道出了玄家欲锋利其谈辩、好学者欲得深思理趣的感受和心情。\CJKunderwave{晋书}录此,前加“每云”二字,愈见神韵。}

\lettrine{4.64} 提婆\myidx{僧伽提婆}初至\footnote{提婆:僧伽提婆,西域罽宾国(今克什米尔)高僧,前秦苻坚建元十七年(381)到长安传经。东晋孝武帝太元十六年(391)到庐山,译\CJKunderwave{阿毗昙心论},经慧远校定,共四卷。},为东亭\myidx{王珣}第讲\CJKunderwave{阿毗昙}\footnote{东亭:王珣,封东亭侯。第:宅邸。}。{\fzxk\zihao{6}\textcolor{red}{\CJKunderwave{出经叙}曰:“僧伽提婆,罽宾人,姓瞿昙氏,俊朗有深鉴。符(苻)坚至长安,出诸经。后渡江,远法师请译\CJKunderwave{阿毗昙}。”远法师\CJKunderwave{阿毗昙叙}曰:“\CJKunderwave{阿毗昙心}者,三藏之要颂(领),咏歌之微言。源流广大,管综众经,领其宗会,故作者以心为名焉。有出家开士、字法胜,以\CJKunderwave{阿毗昙}源流广大,卒难寻究,别撰斯部,凡二百五十偈,以为要解,号之曰‘心’。罽宾沙门僧伽提婆少翫斯文,因请令译焉。”\CJKunderwave{阿毗昙}者,晋言大法也。道摽法师曰:“\CJKunderwave{阿毗昙}者,秦言无比法也。”}} 始发讲,坐裁半\footnote{裁:通“才”。},僧弥\myidx{王珉}便云\footnote{僧弥:王珉小字,王珣弟。}:“都已晓。”即于坐分数四有意道人\footnote{数四:三四个。有意:有才识。道人:僧人。},更就馀屋自讲。提婆讲竟,东亭问法冈\myidx{法冈}道人曰\footnote{法冈:\CJKunderwave{高僧传}做“法纲”,东晋僧人。}:{\fzxk\zihao{6}\textcolor{red}{法冈,未详氏族。}} “弟子都未解,阿弥那得已解\footnote{阿弥:王珉。那得:怎么。}?所得云何\footnote{云何:如何。}?”曰:“大略全是\footnote{大略:大要,大体。},故当小未精覈耳\footnote{故当:只是。小:稍微。精覈:深入考索。}。”{\fzxk\zihao{6}\textcolor{red}{\CJKunderwave{出经叙}曰:“提婆以隆安初游京师,东亭侯王珣迎至舍,讲\CJKunderwave{阿毗昙}。提婆宗致既明,振发义奥,王僧弥一听,便自讲,其明义易启人心如此。未详年卒。”}}

{\cangkai\zihao{5}【评】这则故事,\CJKunderwave{晋书}、\CJKunderwave{高僧传}均移用,是东晋佛坛的一大佳话。一是“外国”来名僧提婆,在庐山译了重要经典\CJKunderwave{阿毗昙心论},“至隆安元年(397)来游京师,晋朝王公及风流名士,莫不造席致敬”(\CJKunderwave{高僧传})。这种轰动,说明了东晋王朝对远来高僧的尊奉和提婆的名声影响;二是卫军东亭侯王珣亲自将其延请至府邸讲经说法,而且“名僧毕至”。盛况之下,故事摄取了一个生动细节来描绘。\CJKunderwave{阿毗昙心论}是印度小乘佛学“说一切有部”的经典,但在东晋,对待西来佛学,感受的是基本原理。所以尽管是说\CJKunderwave{阿毗昙心论},王珉只听一半,就以为不过而尔,跑到一边另开讲席,这说明他平日对佛学原理耳熟。法冈的所谓“小未精覈”,说的就是基本原理大旨未错,至于各派细说尚未能究其详。而欲究\CJKunderwave{阿毗昙心论}的细节,就是僧人专家也颇废心神,因为它重逻辑论证,带有经院哲学的色彩,对大乘、小乘佛理的条分缕析,不是王珉,甚至不是一般僧人一席讲座,就能轻易办到的。故事的妙笔就在于勾画出王氏兄弟的不同表现:“时尚幼”的王珉一闻便以为知,不耐烦去听个究竟;而“神情朗悟,经史明澈”(\CJKunderwave{晋书·王珣传})的王珣则颇生疑惑。它生动地传达了当时士人对佛学——这个与玄学互补,甚至在学理上比本土经典\CJKunderwave{易}、\CJKunderwave{老}、\CJKunderwave{庄}在认识论上有更多玄妙的理论的兴奋和欲求。而这一切,也正是魏晋时代突破汉代经学,寻求理论新途径的必然惯性,这也可能是佛学盛传于时的一个重要原因。}

\lettrine{4.65} 桓南郡\myidx{桓玄}与殷荆州\myidx{殷仲堪}共谈\footnote{桓南郡:桓玄,桓南郡:指桓玄(369—404),袭父温之爵南郡公,故称。安帝时任江州刺史、都督荆州八郡诸军事,率军东下,篡晋自立,建国号楚。旋被刘裕击败,斩首京师。殷荆州:殷仲堪,曾任荆州刺史,(?—399):善清谈,当时与韩康伯齐名。},每相攻难\footnote{攻难:攻辩诘难。}。年馀后,但一两番\footnote{番:次、回合。}。桓自叹才思转退\footnote{转:渐渐。}。殷云:“此乃是君转解。”{\fzxk\zihao{6}\textcolor{red}{周祇\CJKunderwave{隆安记}曰:“玄善言理,弃郡还国,常与殷荆州仲堪终日谈论不辍。”}}

{\cangkai\zihao{5}【评】殷仲堪是清谈人物,而桓玄“风神疏朗,博综艺术,善属文。常自负其才地,以豪雄自处”(\CJKunderwave{晋书·桓玄传}),虽然亦善言理,但他是一个雄豪霸道的野心家,而非真正的清谈家。刘注说他“弃郡还国,常与殷荆州仲堪终日谈论不辍”。此时他正“郁郁不得志”,无以展其雄豪霸才,故以清谈聊慰压抑的愁怀。本则的“叹才思转退”,其实也是他当时落寞心境的表达,一笔点染,神情宛然。殷仲堪的回答,余嘉锡先生解云:“言彼此共谈既久,玄于己所言转能了解,故攻难渐少,非才退也。”词面的确是这样的婉转之意,颇有劝慰情味。而词底却是“荆州刺史殷仲堪(对桓玄)甚敬惮之”(\CJKunderwave{晋书·桓玄传}),殷仲堪与桓玄周旋,一直在玄的阴影之下,最终还是死于桓玄之手,此是后话。此时,在这个雄豪人物面前的婉转之词,是他“敬惮”之心的自然反应,一句回答,也是神情毕现。}

{\cangkai\zihao{5}本则的清谈,不是正常意义上的谈玄究理,在桓玄是因“在荆州积年,优游无事”的解闷,在殷仲堪是与虎狼之人周旋,不得不把握分寸。寥寥数语,将两人的形象描摹得准确细腻,委婉入神。}

\lettrine{4.66} 文帝\myidx{曹丕}常令东阿王\myidx{曹植}七步中作诗\footnote{文帝:魏文帝曹丕,见\CJKunderwave{言语}10。东阿王:曹植(192—232),字子建,曹操第三子,曹丕同母弟。丕即帝位后,对植屡加贬抑,明帝曹叡亦不用植,后郁闷而死。因封东阿王,故称。},不成者行大法\footnote{大法:极刑、死刑。}。应声便为诗曰\footnote{应声:随声。}:“煮豆持作羹,漉䜵(豉)以为汁\footnote{漉:过滤。豉:煮熟发酵后做成的豆制品。诸本“豉”作“菽”,菽:豆类总称。}。箕在釜下燃\footnote{箕:诸本作“萁”,豆秸。釜:锅。},豆在釜中泣。本自同根生\footnote{本自:本来。},相煎何太急。”帝深有惭色\footnote{惭:羞愧。}。{\fzxk\zihao{6}\textcolor{red}{\CJKunderwave{魏志}曰:“陈思王植字子建,文帝同母弟也。年十馀岁诵诗论及辞赋数万言。善属文,太祖尝视其文曰:‘汝倩人耶?’植跪曰:‘出言为论,下笔成章,顾当面试,奈何倩人?’”时邺铜雀台新成,太祖悉将诸子登之,使各为赋。植援笔立成,可观。性简易,不治威仪,舆马服饰,不尚华丽。每见难问,应声而答,太祖宠爱之,几为太子者数矣。文帝即位,封鄄城侯,后徙雍丘,复封东阿。植每求试不得,而国亟迁易,汲汲无欢。年四十一薨。}}

{\cangkai\zihao{5}【评】对于本篇,李慈铭的\CJKunderwave{世说新语批注}云:“案临川之意分此以上为学,此下为文。然其所谓学者,清言、释、老而已。”就是说,以本则为分界,前此即“学”,为讨论学术的记述;后此为“文”,确为当今意义上的文学活动,于此可窥其时对“文”认识的自觉。}

{\cangkai\zihao{5}这首“七步”之诗,因植本集未载,后人疑为附会。其实早在齐、梁时任昉的\CJKunderwave{齐竟陵文宣王行状}文中就说:“陈思见称于七步。”这首诗作,让人惊诧于曹植敏捷的才思。在七步之内的瞬间应声成韵,并且以一个相当新颖而鲜明的喻式,极准确、深刻地描绘出自己内心的哀痛,生发出震颤人心的抗争,并揭示了来自人性深处的扭曲与丑恶,这种才思的敏捷和人格的奇崛,无论如何是令人叹服、称奇的。}

{\cangkai\zihao{5}因为皇权,曹植屡为曹丕父子逼迫,最后郁郁而死,这是史有明证的。至于曹丕能否悍然地以作诗为由,用“大法”逼迫亲弟,观其“御之以术,矫情自饰”(见\CJKunderwave{三国志}曹植本传)的习性,想来或不至愚蠢到这步田地。但故事恰反映了人们对植的同情与热爱,对文学之美的崇尚。在重才情的时风下,曹植才冠当世,人称“绣虎”,锺嵘\CJKunderwave{诗品},评为上上之品,誉之为“建安之杰”,他的诗文达到了“骨气奇高”与“词采华茂”的完美统一,创造了文学史上又一奇峰,这是令世人瞩目折腰的。\CJKunderwave{宋书·谢灵运传论}说:“子建、仲宣以气质为体,并标能擅美,独映当时。是以一世之士,各相慕习。”此说真实反映了当世心理,人们羡慕他的才情,同情他的遭际,所以在故事中突现了其胞兄的狰狞和诗人的才气。这样描写曹丕,未必公允。但两相对比,皆形象鲜明,尤其是曹植,不仅是才子,而且是备受压抑的志士,故事塑造的这种形象是尤为动人的。}

\lettrine{4.67} 魏朝封晋文王\myidx{司马昭}为公\footnote{晋文王:司马昭,晋武帝司马炎废魏立晋,追尊父昭为文皇帝。见\CJKunderwave{德行}15。},备礼九锡\footnote{九锡:古代天子对诸侯、大臣的非常礼遇,所赐有九:车马、衣服、乐则、朱户、纳陛、虎贲、弓矢、斧钺、秬鬯(见\CJKunderwave{左传·庄公元年})。},文王固让不受。公卿将校当诣府,敦喻\footnote{公卿将校:三公九卿,高级武官,即朝中的文武大臣。敦喻:敦请劝说。}。司空郑冲\myidx{郑冲}{\fzxk\zihao{6}\textcolor{red}{冲已见。}} 驰遣信就阮籍\myidx{阮籍}求文\footnote{司空:官名,三公之一。}。籍时在袁孝尼\myidx{袁准}家,{\fzxk\zihao{6}\textcolor{red}{\CJKunderwave{袁氏世纪}曰:“准字孝尼,陈郡(阳)夏人。父涣,魏郎中令。准忠信居正,不耻下问,唯恐人不胜己也。世事多险,故治(恬)退不敢求进。著书十馀万言。”荀绰\CJKunderwave{兖准(按:“准”字衍)州记}曰:“准有俊才,太始中,位给事中。”}} 宿醉扶起\footnote{宿醉:隔夜馀醉。},书札为之\footnote{书札:写在木札上。},无所点定\footnote{点定:涂改定稿。},乃写付使。时人以为神笔\footnote{神笔:高妙的文章。笔:指无韵的散文。}。{\fzxk\zihao{6}\textcolor{red}{顾恺之\CJKunderwave{晋文章记}曰:“阮籍\CJKunderwave{劝进},落落有宏致,至转说徐而摄之也。”一本注阮籍\CJKunderwave{劝进文}略曰:“窃闻明公固让,冲等眷眷,实怀愚心。以为圣王作制,百代同风,褒德赏功,其来久矣。周公籍已成之业,据既安之势,光宅曲阜,奄有龟蒙。明公宜奉圣旨,受兹介福也。”}}

{\cangkai\zihao{5}【评】建安十八年,汉献帝加曹操“九锡”,是曹操自己导演的取代汉室的一幕序曲,曾几何时,司马昭又重演了这一幕,这是历史的讽刺。对此“九锡”及“晋公”爵号,史称司马昭“九让,乃止”(\CJKunderwave{晋书·文帝纪})。每一“让”,都需满朝文武和皇帝本人,群起相劝。这一过程,就变成了宣传和演戏,是强化其声望、地位的过程,也成为观察异己的过程,它实在是一个险恶的政治风云的际会。阮籍“本有济世志,属魏晋之际,天下多故,名士少有全者,籍由是不问世事,遂酣饮为常”(\CJKunderwave{晋书·阮籍传})。他在曹魏与司马氏残酷争斗中间,难于进退,因此用酣饮沉醉来保全自己,其内心的孤独与苦闷,幽愤和哀伤都表达在\CJKunderwave{咏怀}八十二首诗作之中。恰在这样的时刻,向他索劝进文,本以沉醉而没参与“诣府敦喻”,现在则无法推诿,挥就文章。人们叹为“神笔”,虽见阮籍不同凡响的高才,但文章不过为不得已的应景之作。凌濛初评:“今读其文,首援伊、周,末称支、许。文王隐衷,悉为勘破,若知有他日者,毛发可竖,何云惭笔,于古昧目,致疑豪杰。”\CJKunderwave{文选}卷四十载阮籍此作,题为\CJKunderwave{为郑冲劝晋王笺},内容、形式与\CJKunderwave{三国志}武帝、文帝纪,裴松之注所引若干劝进文,几相仿佛,不循此套路,则不足以为“劝进”,若别有用意、措辞不慎,不仅阮籍,怕是连郑冲等亦躲不过司马氏的屠刀。且该文未有苦心孤诣之深论,点到而止。如果玩味阮籍\CJKunderwave{咏怀}诸诗,及其处世之尴尬,则此表面文章似可理解。如若他表现为刚烈、悻直,则不待写此文,而早被诛除了。于史而言,他是一个心怀道德宏志而被险恶政治所蹂躏的悲剧人物。}

{\cangkai\zihao{5}但本则故事的描画是精彩的,阮籍醉而扶起,下笔成文,且“无所点定”,这种才情,是令人惊服的。本则渲染的是文学才士的非凡形象。}

\lettrine{4.68} 左太冲\myidx{左思}作\CJKunderwave{三都赋}初成\footnote{左太冲:左思,字太冲,晋文学家,见刘孝标注。\CJKunderwave{三都赋}:左思所作\CJKunderwave{魏都赋}、\CJKunderwave{吴都赋}、\CJKunderwave{蜀都赋}的合称。魏都邺城,今河北临漳西南;吴都建业,今江苏南京市;蜀都成都,今四川成都。其文今见\CJKunderwave{昭明文选}。},{\fzxk\zihao{6}\textcolor{red}{\CJKunderwave{思别传}曰:“思字太冲,齐国临淄人。父雍,起于笔札,多所掌练,为殿中御史。思少孤,不甚教其书学。及长,博览名文,遍阅百家。司空张华辟为祭酒,贾谧举为秘书郎。谧诛,归乡里,专思著述。齐王冏请为记室参军,不起。时为\CJKunderwave{三都赋}未成也。后数年疾终。其\CJKunderwave{三都赋}改定,至终乃止。初作\CJKunderwave{蜀都赋}云:‘金马电发于高冈,碧鸡振翼而云披。鬼弹飞丸以礌礉,火井腾光以赫曦。’今无鬼弹,故其赋往往不同。思为人无吏干而有文才,又颇以椒房自矜,故齐人不重也。”}} 时人互有讥訾\footnote{讥訾:讥讽非毁。},思意不惬\footnote{不惬:不愉快。},后示张公\myidx{张华}\footnote{张公:张华,范阳方城(今河北固安西北)人。博学多才,贯通今古,以诗赋文章称名于世。为晋武帝筹设灭吴方略,一统天下。惠帝时官至司空,死于八王之乱。}。{\fzxk\zihao{6}\textcolor{red}{张华已见。}} 张曰:“此\CJKunderwave{二京}可三\footnote{\CJKunderwave{二京}:指东汉张衡所作\CJKunderwave{西京赋}、\CJKunderwave{东京赋}。可三:可以与之并列为三。},然君文未重于世,宜以经高名之士。”思乃询求于皇甫谧\myidx{皇甫谧}。{\fzxk\zihao{6}\textcolor{red}{王隐\CJKunderwave{晋书}曰:“谧字士彦,安定朝那人,汉太尉嵩曾孙也。祖叔献,灞陵令。父叔侯,举孝廉。谧族从皆累世富贵,独守寒素。所养叔母叹曰:‘昔孟母以三徙成子,曾父以烹豕存教,岂我居不十(卜)邻,何尔曹之甚乎?修身笃学,自汝得之,于我何有?’因对之流涕,谧乃感激。年二十馀,就乡里席坦受书,遭人而问,少有宁日。武帝借其书二车,遂博览。太子中庶子、议郎征,并不就,终于家。”}} 谧见之嗟叹,遂为作\CJKunderwave{叙}\footnote{叙:同“序”。}。于是先相非贰者\footnote{非贰:非议怀疑。},莫不敛衽赞述焉\footnote{敛衽:提起衣襟夹于带间,表示敬意。赞述:赞美称道。}。{\fzxk\zihao{6}\textcolor{red}{\CJKunderwave{思别传}曰:“思造张载,问㟭、蜀事,交接亦疏。皇甫谧西州高士,挚仲治宿儒知名,非思伦疋。刘渊林、卫伯舆并蚤终,皆不为思\CJKunderwave{赋}序注也。凡诸注解,皆思自为,欲重其名,故假时人名姓也。”}}

{\cangkai\zihao{5}【评】赋是两汉文学家呈才能、见学识的重要文体,这一价值观念延至魏晋而不衰,\CJKunderwave{文选}六十卷,其中十九卷半皆是“赋”。非博学多识、卓有才情是驾驭不了这一文体的,而一篇杰出的“赋”,就可以奠定一位文学家在文坛中的地位,所以左思发宏愿,毕生经营他的\CJKunderwave{三都赋},而并没有以他最有价值的\CJKunderwave{咏史诗}为得意。本则所反映的其实就是“赋”在当时的地位以及由此而给诗人带来的声誉价值。}

{\cangkai\zihao{5}张衡的\CJKunderwave{二京赋}是东汉文坛的力作,雄浑宏阔,富有力度。西京长安、东京洛阳跨山带河,经两汉经营,建筑、朝堂、物产、市廛、国威等等,都被作家在皇皇大赋中,渲染描绘得壮丽非常。不仅壮人心怀,而且表现了作者广博的学识、精妙的构思和宏丽的文采。左思追踵其后,欲作蜀、吴、魏\CJKunderwave{三都赋},其难度可想而知,难怪陆机笑而不信,与弟书曰:“此间有伧父,欲作\CJKunderwave{三都赋},须其成,当以覆酒瓮耳。”但是,皇天不负苦心人,左思\CJKunderwave{三都赋}果然不同凡响,其典丽凝重或不如张衡,而其恢宏壮阔,清雅婉致,自有一番境象。而且对蜀、吴、魏的描绘不局限于其都邑,而是以三国之地理、物产、风习及帝都的壮美为对象,三都之描绘不相雷同,各有品质与风采,同时赋中所显现的博学重彩亦不下张衡,所谓“非夫研考者不能练其旨,非夫博物者不能统其异”。所以,当世文坛盟主张华,一观而评其可与\CJKunderwave{二京}媲美,又经皇甫谧荐拔,终于“豪贵之家竞相传写,洛阳为之纸贵”。就是陆机见到此赋,也“绝叹服,以为不能加也”(上引均见\CJKunderwave{晋书·左思传})。本则故事,实际描写了一出作家得遇知音的动人戏剧,也从这里,表达了“赋”在当时的崇高地位,左思这一才士的形象也因此特别动人。}

{\cangkai\zihao{5}何子充评本则:“文章定价,本自明白,而时势耳目不足取信如此。士君子中蕴内晦,虽出而未试者,欲以求知皮相之士,岂不难哉!”其感慨于本则,因左思英俊沉下僚而难为俗士所鉴赏的悲哀,亦良足动人心怀。}

\lettrine{4.69} 刘伶\myidx{刘伶}著\CJKunderwave{酒德颂},意气所寄\footnote{意气:志趣。}。{\fzxk\zihao{6}\textcolor{red}{\CJKunderwave{名士传}曰:“伶字伯伦,沛郡人。肆意放荡,以宇宙为狭。常乘鹿车,携一壶酒,使人荷锸随之,‘死便掘地以埋’。土木形骸,遨游一世。”\CJKunderwave{竹林士(七)贤论}曰:“伶处天地间,悠悠荡荡,无所用心。尝与俗士相迕,其人攘袂而起,欲必筑之。伶和其色曰:‘鸡肋岂足以当尊拳!’其人不觉废然而返。未尝措意文章,终其世,凡著\CJKunderwave{酒德颂}一篇而已。其辞曰:‘有大人先生者,以天地为一朝,万期为须臾,日月为扃牖,八荒为庭衢。行无轨迹,居无室庐,幕天席地,纵意所如。行则操卮执觚,动则挈榼提壶,唯酒是务,焉知其馀?有贵介公子,缙绅处士,闻吾风声,议其所以。乃奋袂攘襟,怒目切齿,陈说礼法,是非锋起。先生于是方捧罂承槽,衔杯漱醪,奋髯踑踞,枕麴藉糟。无思无虑,其乐陶陶。兀然而醉,慌尔而醒,静听不闻雷霆之声,熟视不见太山之形,不觉寒暑之切肌,利欲之感情。俯观万物之扰扰,如江汉之载浮萍。二豪侍侧焉,如蜾蠃之与螟蛉。’”}}

{\cangkai\zihao{5}【评】刘伶以醉酒而千古留名,逸事除他“常乘鹿车,携酒一壶,使人荷锸而随之,谓曰:‘死便埋我。’”之外,就是他的\CJKunderwave{酒德颂}了。其醉之所以动人,人不以丑陋酒鬼视之,就是因为他的饮酒是以酒精之麻醉而使自己忘情一切,抗拒礼俗,回归自然,展演的是魏晋风流。其性“放情肆志,常以细宇宙齐万物为心”(见\CJKunderwave{晋书·刘伶传})。其\CJKunderwave{酒德}之颂,正是越名教而任自然的宣言,视俗情俗礼为敝屣,泠然超迈,纵意所如,回归到超凡脱俗的自由境界。不只是\CJKunderwave{酒德颂},即其全人,一生形迹,都是“意气所寄”,所以其饮其醉,如诗如歌。}

\lettrine{4.70} 乐令\myidx{乐广}善于清言\footnote{乐令:乐广,乐广(?—304):字彦辅,南阳淯阳(今河南南阳东南)人。少孤贫,寒素为业,与物无竞。其清谈析理,与王衍并称,卫瓘以为有正始遗风。官至尚书令,八王乱中,以故忧卒。清言:清谈。},而不长于手笔\footnote{手笔:指撰写文章。}。将让河南尹\footnote{让:辞去官职。河南尹:河南郡最高行政长官。},请潘岳\myidx{潘岳}为表。{\fzxk\zihao{6}\textcolor{red}{\CJKunderwave{晋阳秋}曰:“岳字安仁,荥阳人。夙以才颖发名。善属又(文),清绮绝世,蔡邕未能过也。仕至黄门侍郎,为孙秀所害。”}} 潘云:“可作耳。要当得君意\footnote{要:但是。当:必须。}。”乐为述己所以为让,标位(作)二百许语\footnote{位:余嘉锡谓:“‘位’盖‘作’之误,后人不识,因妄改为‘位’。”标作,即写出、揭示。}。潘直取错综\footnote{直:只是。错综:此指组织整理。},便成名笔。时人咸云:“若乐不假潘之文,潘不取乐之旨,则无以成斯矣。”

{\cangkai\zihao{5}【评】潘岳是当世有名的文学家,一部\CJKunderwave{文选},给予其诗、赋以可观的席位,而其赋作,煌煌如\CJKunderwave{西征赋},清绮如\CJKunderwave{秋兴赋}、\CJKunderwave{怀旧赋}等皆见其“才颖”和文章情采思理的动人;乐广“有远识”,对玄理、人生长于体味、思考而文词简淡冲约,潘、乐合作,恰好能表现思理辞采。本则故事为\CJKunderwave{晋书·乐广传}移用,在末句加一“美”字,作“无以成斯美也”,则更有神采,它同时强调着两人优长相结所成就的珠联璧合之美文、美事。而潘岳善解“远识”的乐广之旨,并且将原文稍加调整、董理就点铁成金,这见出潘的聪颖和艺术才气。}

\lettrine{4.71} 夏侯湛\myidx{夏侯湛}作\CJKunderwave{周诗}成,{\fzxk\zihao{6}\textcolor{red}{\CJKunderwave{文士传}曰:“湛字孝若,谯国人,魏征西将军夏侯渊曾孙也。有盛才,文章巧思,善补雅词,名亚潘岳。历中书侍郎。”湛集载其叙曰:“\CJKunderwave{周诗}者,\CJKunderwave{南陔}、\CJKunderwave{曰(白)华}、\CJKunderwave{华黍}、\CJKunderwave{由庚}、\CJKunderwave{崇丘}、\CJKunderwave{由仪}六篇,有其义而亡其辞,湛续其亡,故云\CJKunderwave{周诗}也。”}} 示潘安仁\myidx{潘岳}\footnote{潘安仁:潘岳,字安仁,见\CJKunderwave{言语}107。},安仁曰:“此非徒温雅\footnote{非徒:不仅。温雅:温文尔雅。},乃别见孝悌之性\footnote{孝悌:孝顺父母,尊敬兄长。}。”{\fzxk\zihao{6}\textcolor{red}{其诗曰:“既殷斯虔,仰说洪恩。夕定辰省,奉朝侍昏。宵中告退,鸡鸣在门。孳孳恭诲,夙夜是敦。”}} 潘因此遂作\CJKunderwave{家风诗}。{\fzxk\zihao{6}\textcolor{red}{岳\CJKunderwave{家风诗},载其宗祖之德,及自戒也。}}

{\cangkai\zihao{5}【评】\CJKunderwave{晋书·夏侯湛传}说:“湛幼有盛才,文章宏富,善构新词。”他作\CJKunderwave{周诗},确实是其才气性格使然。\CJKunderwave{诗经·小雅}的\CJKunderwave{南陔}等六诗是“笙诗”,学者研究证明,它们是用笙演奏的乐曲,本来就有目无辞。可\CJKunderwave{毛诗序}说,它们是目存词亡,并序其诗曰:“\CJKunderwave{南陔},孝子相诫以养也。\CJKunderwave{白华},孝子之洁白也……”魏晋时还没有人怀疑\CJKunderwave{毛诗序}的说法,所以夏侯湛要以其才能补上这个遗憾,并且以孝悌为主题。所补之诗,今见\CJKunderwave{夏侯湛集},也只有刘孝标注所引的这些。诗有\CJKunderwave{小雅}韵味,温文尔雅,见其修养、才气,因而打动了他好友潘岳。\CJKunderwave{艺文类聚}卷二十三载有潘岳的\CJKunderwave{家风诗},其诗拟\CJKunderwave{小雅·采薇}语式,应和夏侯湛\CJKunderwave{周诗}主题,亦颇有意趣:“绾发绾发,发亦鬓止。日祗日祗,敬亦慎止。靡专靡有,受之父母。鸣鹤匪和,析薪弗荷。隐忧孔疚,我堂靡构。义方既训,家道颖颖。岂敢荒宁,一日三省。”故事描述了当日文坛佳话,不仅同时记录了两人同有美观的形貌,而且才气亦如“连璧”。\CJKunderwave{世说}记此,可见他们两人是当时颇引人注目的一道景观。}

\lettrine{4.72} 孙子荆\myidx{孙楚}除妇服\footnote{孙子荆:孙楚,字子荆,亦当时豪爽之士,\CJKunderwave{晋书}卷五十六本传,言其才藻卓绝,爽迈不群,多所陵傲,缺乡曲之誉。年四十馀始仕。与王济相知甚深。除妇服:为妻服丧期满,除去丧服。古代夫为妻服丧之礼为一年。},作诗以示王武子\myidx{王济}\footnote{王武子:王济,字武子,亦当时豪爽之士,\CJKunderwave{晋书}卷五十六本传,言其才藻卓绝,爽迈不群,多所陵傲,缺乡曲之誉。年四十馀始仕。与王济相知甚深。}。{\fzxk\zihao{6}\textcolor{red}{孙楚集云:“妇,胡毋氏也。”其诗曰:“时迈不停,日月电流。神爽登遐,忽已一周。礼制有叙,告除灵丘。临祠感痛,中心若抽。”}} 王曰:“未知文生于情,情生于文?”{\fzxk\zihao{6}\textcolor{red}{一作“文于情生,情于文生”。}} 览之凄然,增伉俪之重\footnote{伉俪:夫妻。}。

{\cangkai\zihao{5}【评】王世贞评:“此语极有致。文生于情,世所恒晓。情生于文,则未易论。盖有出之者偶然,览之者实际也。吾平生时遇此境,亦见同调中有此。”这是文学家的感受、经验谈,语颇中肯。然就人们对文学的认识看,文生于情,到魏晋始张大其说,这是当时玄学人格对文学认识的反映,它切近了文学的规律,形成了文学的自觉。稍后的刘勰就奋力倡导“为情造文”,认为“情”是文之经(见\CJKunderwave{文心雕龙·情采}),其理论,生于当时,又成为此后强有力的导向。本则就清晰、醒目地表达了当时人们对“情”与“文”的真切感受。在孙楚是为情以造文;在王济是披文以入情,感动了他们的都是一“情”字。而王济之问,正是其深入了情感境界的情景。作为知己好友,王济很能体会出孙楚的感受,同时孙诗的情感又具有人情的普遍性,前面一问和后面一句夫妻情重的感喟,映现了王济沉浸在情感体味中的状态。此是出之者必然,而览之者,所得到的是比实际生活体验更多的审美感受。本则说的是论文,却客观地表现了当时那种人情、人性之美。}

\lettrine{4.73} 太叔广\myidx{太叔广}甚辩给\footnote{辩给:口才敏捷,善于言辩。},而挚仲治\myidx{挚虞}长于翰墨\footnote{翰墨:笔墨,指写文章、文辞。},俱为列卿\footnote{列卿:在九卿之列。}。每至公坐,广谈,仲治不能对;退,箸笔难广\footnote{箸笔:撰写文章。当时以韵文为“文”,无韵为“笔”。箸,同“著”。按:\CJKunderwave{世说}中“著”、“箸”、“者”因正俗字的关系,常通用之。},广又不能答。{\fzxk\zihao{6}\textcolor{red}{王隐\CJKunderwave{晋书}曰:“广字季思,东平人。拜成都王为太弟,欲使诣洛。广子孙多在洛,虑害,乃自杀。挚虞字仲治,京兆长安人,祖茂,秀才。父模,太仆卿。虞少好学,师事皇甫谧,善校练文义,多所箸述。历秘书监、太常卿,从惠帝至长安,遂流离鄠、杜间。性好博古,而文籍荡尽。永嘉五年,洛中大饥,遂饿而死。”虞与广名位略同,广长口才,虞长笔才,俱少政事。众坐广谈,虞不能对;虞退,笔难广,广不能合(答)。于是更相嗤笑,纷然于士(世)。广无可记,虞多所录,于斯为胜也。}}

{\cangkai\zihao{5}【评】古今才士中,口才、文章俱胜者有之,讷于言而长于笔者有之,口才敏捷而拙于文翰者亦有之。本则的太叔广和挚虞相对,可算是口辩、文翰之才各有千秋的一个夸张表现。太叔广能言,但口辩落实到文章,尚须有为文的训练和才能,在翰墨方面,他逊色于挚虞;挚虞“少事皇甫谧,才学通博”(\CJKunderwave{晋书·挚虞传}),长于著述,口辩却不如太叔广,于是就演绎了这段名士斗才的故事。本则的生动处,在于表现了其时重文章、重才情的彬彬之盛,这才有两才子的动人景象。}

\lettrine{4.74} 江左殷太常父\myidx{殷融}子\myidx{殷浩}并能言理\footnote{殷融:字洪远曾官太常卿,故称。江左:古人以东为左,故称长江下游一带地区为江左。此指东晋。父子:古人称叔侄亦曰父子,此即其例。},亦有辩讷之异\footnote{辩:口才敏捷。讷:言语笨拙。}。汤州口谈至剧\footnote{汤州:袁本作“扬州”,是。扬州,指殷浩,曾任扬州刺史。口谈:言谈。至剧:极敏捷。},太常辄云:“汝更思吾论。”{\fzxk\zihao{6}\textcolor{red}{\CJKunderwave{中兴书}曰:“殷融字洪远,陈郡人。桓彝有人伦鉴,见融,甚叹美之。著\CJKunderwave{象不尽意}、\CJKunderwave{大贤须易论},理义精微,谈者称焉。兄子浩,亦能清言,每与浩谈,有时而屈。退而著论,融更居长。为司徒左西属。饮酒善舞,终日啸咏,未尝以出务自婴。累迁吏部尚书、太常卿,卒。”}}

{\cangkai\zihao{5}【评】刘应登云:“浩长于谈,融长于笔也。”参见前则,这里也是口辩之才与文章之才的差异而造成的戏剧性场景。殷浩是著名的谈玄家,清谈领袖人物,能将风云一时的名士们辩得狼狈不堪(参见本篇有关殷浩诸条评析),可想而知,在他的“至剧”口谈之下,讷于口辩的殷融的处境。但融以退为进,避浩锋芒而著笔问难,这是以己之长而攻人之短,故有“汝更思吾论”之言。一句回答,互不服输的声情毕现,是本则的动人处。}

\lettrine{4.75} 庾子嵩\myidx{庾敳}作\CJKunderwave{意赋}成\footnote{庾子嵩:庾敳,字子嵩。}。{\fzxk\zihao{6}\textcolor{red}{\CJKunderwave{晋阳秋}曰:“敳永嘉中为石勒所害。先是,敳见王室多难,知终婴其祸,乃作\CJKunderwave{意赋}以寄怀。”}} 从子文康\myidx{庾亮}见\footnote{从子:侄子。文康:庾亮,死谥“文康”。},问曰:“若有意邪,非赋之所尽;若无意邪,复何所赋?”答曰:“正在有意无意之间。”

{\cangkai\zihao{5}【评】魏晋士人沉浸在玄理之中,论文也用谈玄话语。王世贞说:“料子嵩文,必不能佳,然有意无意之间,却是文章妙用。”庾亮对叔父的文章,不便指点评价,用了一句谈玄话语,依玄理,当为“言不尽意”,所以从这个意义上说,不必究此赋作的表意如何,质量高下。想是庾敳敏感地觉出了这种态度,用了绝聪明的回答,其所谓“有意无意之间”,却正道出了作家的神思,妙在文字之外的无尽意念和情思。这暗合于文学的形象思维规律。不过,本则词面是叙写了论文学的情景,而在这场景中,却也活画了两个灵动的人物,这种摇曳的空明聪颖,标记着魏晋的才士风情。}

\lettrine{4.76} 郭景纯\myidx{郭璞}诗云:“林无静树,川无停流\footnote{“林无静树”二句:树欲静而风不止,流难驻而逝不息,故孔子兴川上之叹。}。”{\fzxk\zihao{6}\textcolor{red}{王隐\CJKunderwave{晋书}曰:“郭璞字景纯,河东闻喜人。父瑗,建平太守。”\CJKunderwave{璞别传}曰:“璞奇博多通,文藻粲丽,才学赏豫,足参上流。其诗赋诔颂,并传于世。而讷于言,造次咏语,常人无异。又不持仪检,形质颓索,纵情嫚惰,时有醉饱之失。友人干令升戒之曰:‘此伐性之斧也。’璞曰:‘吾所受有分,恒恐用之不尽,岂酒色之能害?’王敦取为参军。敦纵兵都辇,乃谘以大事,璞极言成败,不为回屈。敦忌而害之。”诗,璞\CJKunderwave{幽思篇}者。}} 阮孚\myidx{阮孚}云\footnote{阮孚:字遥集,阮咸次子,晋元帝世为安东参军,历侍中、吏部尚书、丹阳尹、广州刺史等。}:{\fzxk\zihao{6}\textcolor{red}{阮孚别见。}} “泓峥萧瑟\footnote{泓:水深而清。峥:山高切云。萧瑟:风吹林木声。},实不可言。每读此文\footnote{文:诗。当时以有韵的诗、赋为“文”,无韵的散文为“笔”。},辄觉神超形越。”

{\cangkai\zihao{5}【评】郭璞博学多识,尤精\CJKunderwave{周易},而一部\CJKunderwave{周易},给人注入了最深刻的理性自觉,就是宇宙、万物的气化流行,生生不息的运动。在六十四卦的系统中,在讲述恒久的\CJKunderwave{恒}卦当中,强调的是运动变化所展现的“刚柔相摩,八卦相荡”,鼓雷霆,润风雨(\CJKunderwave{周易·系辞上传})的动感情景。郭璞将其对自然、人生的感受和学问、修养的陶冶,融而为诗。诗句的表现形态,并没有书卷气,而是对所描摹的对象遗形取神,将自然界中永无消歇的运动内力与外观景象都富有神韵地表达出来。所见是树欲静而风不止,逝者如斯不舍昼夜,而所感则是一种生命的力量和生灭律动的永恒,足以让人“神超形越”。这也是深入玄境,唤起生命、个性、生机感想的审美境界,艺术品位如斯,不唯阮孚,就是千载之下,人们诵读、玩味这话,也会“神超形越”,迁想不已。}

{\cangkai\zihao{5}这种感受和说法,其实不止郭璞,在玄学背景下,文士多有同感。殷仲文著“表”,也表达了同样意思:“洪波振壑,川无恬鳞;惊飚拂野,林无静柯。何者?势弱受制于巨力,质弱无以自保。”他把现象界背后的巨大力量表达出来了。在大自然生生不息的伟力作用之下,现象界的东西,不过是弱势、弱质而已。郭璞之诗,含蓄优美;殷仲文作文,直白浅切。不过两相对读,可以更深切地感受魏晋人所感受到的造化之功那种震撼人心的伟力,和他们任运自然的心理基础,以及由此而生发出的具有震撼力、令人玩味不已的体悟生命的乐章。}

\lettrine{4.77} 庾阐\myidx{庾阐}始作\CJKunderwave{扬都赋}\footnote{\CJKunderwave{扬都赋}:赋名,东晋庾阐所作。},道温\myidx{温峤}、庾\myidx{庾亮}云\footnote{温:温峤,当时追从姨夫刘琨,在并州为谋主,“琨所凭恃焉”(\CJKunderwave{晋书·温峤传})。建武元年(317)奉刘琨命出使江南,拥戴司马睿即帝位,建立东晋王朝。受司马睿重用,留为散骑常侍,后官至中书令,为东晋名臣。庾:庾亮,庾亮(289—340)的敬称。他历仕东晋元、明、成三朝,作为外戚,曾执国政,显赫于朝。的卢:传说中的凶马之名,骑之不利主人。}:“温挺义之标\footnote{挺:举,伸张。标:楷模。},庾作民之望\footnote{望:所仰望的人。}。方响则金声,比德则玉亮。”庾公闻赋成,求看,兼赠贶之\footnote{赠贶:馈赠财物。}。阐更改“望”为“隽”,以“亮”为“润”云\footnote{“阐更”二句:改“亮”为“润”是避庾亮的名讳。改“望”为“隽”是为了与“润”押韵。}。{\fzxk\zihao{6}\textcolor{red}{\CJKunderwave{中兴书}曰:“阐字仲初,颍川人,太尉亮之族也。少孤,九岁便能属文。迁散骑侍郎,领大著作,为\CJKunderwave{扬都赋},邈绝当时。五十四卒。”}}

{\cangkai\zihao{5}【评】据余嘉锡先生\CJKunderwave{笺疏}引\CJKunderwave{类林杂说},知庾阐作\CJKunderwave{扬都赋},用功甚苦:令其妻“于午夜以燃灯于瓮中。仲初思至,速火来,即为出灯。因此赋成,流于后世。”赋作描写扬都,亦颇壮阔,追摹张衡\CJKunderwave{二京赋}、左思\CJKunderwave{三都赋}的形制规模,叙写山川湖泽之壮美,铺陈物产建筑之丰盛等,构思俪辞之勤苦尽现其中。庾阐本人又有名于当时,所以赋未出人们即有所闻,有所期待。赋成,内有颂温峤、庾亮等胜流人物的内容,故引得庾亮求观。而赠贶之举,则不止出于赋中颂美了庾亮,也因庾亮对赋作成功的肯定、推崇,可见该赋在当时的影响力。故事记述庾阐的加工、修改,也见出作者的腹笥和用功,一字之移,用心良苦。透过故事的描绘,看出当时赋仍在文坛上居主流位置,故文士趋之若鹜。后来昭明\CJKunderwave{文选}分类选文,以赋居首,仍可见其影迹。}

\lettrine{4.78} 孙兴公\myidx{孙绰}作\CJKunderwave{庾公诔}\footnote{孙兴公:孙绰。诔:叙述死者生平德行的哀悼性文章。},袁羊\myidx{袁乔}曰\footnote{袁羊:袁乔。}:“见此张缓\footnote{张缓:指文章张弛有度。}。”于时以为名赏\footnote{名赏:出色的鉴赏、评价。}。{\fzxk\zihao{6}\textcolor{red}{\CJKunderwave{袁氏家传}曰:“乔有文才。”}}

{\cangkai\zihao{5}【评】诔作为叙述死者生平德行的哀悼性文章,在当时既是一种实用文体,也是表现文士才情的特殊载体,因为哀悼昔贤的功德,实际表现了魏晋文人对于当代生活及未来生命的关怀。孙绰的\CJKunderwave{庾公诔},被名士袁乔品评,就说明人们对这种文体的关注,甚至把它作为一种文学作品来欣赏。“于时以为名赏”,正反映着魏晋士人的共同审美心理。}

\lettrine{4.79} 庾仲初\myidx{庾仲初}作\CJKunderwave{扬都赋}成\footnote{庾仲初作\CJKunderwave{扬都赋}:赋名,东晋庾阐所作。},以呈庾亮\myidx{庾亮}\footnote{庾亮:庾亮(289—340)的敬称。他历仕东晋元、明、成三朝,作为外戚,曾执国政,显赫于朝。的卢:传说中的凶马之名,骑之不利主人。},亮以亲族之怀\footnote{亲族:亲近的同族人。怀:心理、心情。},大为其名价\footnote{名价:评价、推赞。},云可三\CJKunderwave{二京}、四\CJKunderwave{三都}\footnote{三\CJKunderwave{二京}、四\CJKunderwave{三都}:可与张衡的\CJKunderwave{二京赋}并列为三;与左思的\CJKunderwave{三都赋}并列为四。\CJKunderwave{二京}、\CJKunderwave{三都}:\CJKunderwave{二京}指东汉张衡所作\CJKunderwave{西京赋}、\CJKunderwave{东京赋}。可三:可以与之并列为三。}。于此人人竞写\footnote{于此:因此。竞写:竞相抄写。},都下纸为之贵\footnote{都下:京城(建康)。}。谢太傅myidx{\}云\footnote{谢太傅:谢安,(?—358):字无奕,谢安长兄,陈郡阳夏谢氏家族在东晋初期的代表人物之一。}:“不得尔,此是屋下架屋耳\footnote{屋下架屋:在房屋里面构架房屋,比喻模仿而无创新的多馀之举。},事事拟学,而不免俭狭\footnote{俭狭:贫乏狭隘。}。”{\fzxk\zihao{6}\textcolor{red}{王隐论杨雄\CJKunderwave{太玄经}曰:“玄经虽妙,非益也,是以古人谓其屋下架屋。”}}

{\cangkai\zihao{5}【评】赋表现文士的学识、才力,而写胜境壮丽之赋,更属不易,前有描写宏阔、掷地有声的名赋\CJKunderwave{二京}、\CJKunderwave{三都},令作者享誉文坛,庾阐亦追踵其事,成\CJKunderwave{扬都赋}。然其作品,\CJKunderwave{文选}未录;\CJKunderwave{艺文类聚}卷六十一,仅节录了其描写扬州居处形胜部分。可见它虽名动当时,但却经不起历史的检验。个中原因,在本则故事里透露了消息。庾阐用功精勤(参见本篇77则评)、富有文才,是其赋有所成就的原因,所以赋出,因皇皇巨制而引人注目,又因身居显位的庾亮推崇而颇扬声誉,以至人们争相传写,“都下为之纸贵”。但其才力尚不能与张衡、左思媲美,原创力不强,这点被谢安揭破。庾亮本“善谈论,性好\CJKunderwave{庄}、\CJKunderwave{老}”(见\CJKunderwave{晋书·庾亮传})有学识,善鉴赏,但因其以“亲族之怀”来品评文学作品,带了功利色彩,所以不免走了样,并非的评;而谢安是雅有修养的文学鉴赏家,敏锐地指出了\CJKunderwave{扬都赋}作为“拟学”之作,不能和\CJKunderwave{二京}、\CJKunderwave{三都}相提并论;于文学创作而言,他倡导了独立创造的精神,表达着魏晋的艺术崇尚。凌濛初有感谢安眼光,评曰:“太傅阳秋,纸当减价。”}

{\cangkai\zihao{5}本则的故事虽小,却也说明着一个铁的事实,即文学作品的生命力在其自身的价值,任何人为的炒作或毁誉都无济于事。}

\lettrine{4.80} 习凿齿\myidx{习凿齿}史才不常\footnote{习凿齿:见\CJKunderwave{言语}2.72。不常:不寻常。},宣武\myidx{桓温}甚器之\footnote{宣武:桓温,桓公北征:桓温曾有三次北征,刘盼遂\CJKunderwave{世说新语校笺}考订,此次当为太和四年(369)之征。时桓温已58岁。器:器重。},未三十,便用为荆州治中\footnote{治中:官名,即“治中从事史”,汉代始置,为州刺史的助理,主管文书案卷。}。凿齿谢笺亦云\footnote{谢笺:感谢信。笺,多用于下对上的书信文体。}:“不遇明公,荆州老从事耳!”后至都见简文\myidx{司马昱},返命\footnote{返命:复命。},宣武问:“见相王何如\footnote{相王:指司马昱,其以会稽王居相位。}?”答云:“一生不曾见此人。”从此迕旨\footnote{迕旨:违逆旨意。},出为荣(荥)阳郡\footnote{荣阳:别有宋本作“荥阳”,朱铸禹\CJKunderwave{世说新语汇校集注}考:“\CJKunderwave{晋书}卷八十二\CJKunderwave{习凿齿传}亦作‘荥’,考荥阳属司州,自穆帝已陷没,至太元间始复,温时不得守,亦别无侨郡,当作‘衡阳’为是。”袁本作“衡阳”。衡阳郡,东晋郡名,治所在今湖南湘潭西。},性理遂错\footnote{性理:神智。}。于病中犹作\CJKunderwave{汉晋春秋},品评卓逸\footnote{品评:评价、议论。卓逸:高妙、卓越。}。{\fzxk\zihao{6}\textcolor{red}{\CJKunderwave{续晋阳秋}曰:“凿齿少而博学,才情秀逸,温甚奇之。自州从事,岁中三转,至治中。后以迕旨,左迁户曹参军、衡阳太守。在郡著\CJKunderwave{晋汉春秋},斥温觊觎之心也。”凿齿集载其论,略曰:“静汉末累世之交争,廓九域之蒙晦,大定千载之盛功者,皆司马氏也。若以魏有代王之德,则不足;有静乱之功,则孙、刘鼎立。共王秦政犹不见叙于帝王,况蹔制数州之众哉。且汉有系周之业,则晋无所承魏之迹矣。春秋之时,吴、楚称王,若推有德,彼必自系于周,不推吴、楚□也。况长辔庙堂,吴、蜀两定,天下之功也。”}}

{\cangkai\zihao{5}【评】\CJKunderwave{晋书}说习凿齿“少有志气,博学洽闻,以文笔著称”,他在桓温处曾深受赏识器重。桓温出征,常委以亲重之任,“或从或守,所在任职,每处机要”,他本人也不负知遇之恩,尽职尽忠,“莅事有绩”。然而,桓温独断霸道“以雄武专朝,觊觎非望”(见\CJKunderwave{晋书·桓温传}),其志由来已久,并且一直挟制简文,他怎么能容忍二心于己而对简文由衷赞美的人呢?桓温打击的是不能认同自己野心的任何人,要为自己实现目标扫除障碍,所以他先将习凿齿“左迁户曹参军”,再放逐至边远衡阳。故事叙写了桓温这个被权力欲望异化了的野心家;习凿齿则是一个旧史家、文士的典型,认可以往历史所标示的逻辑,忠心于司马王朝。其实从简文懦弱苟且的为政气质说,他并不具有政治家的风范,而习凿齿深加叹美,以习凿齿作为史家的“史识”,似不当如此,其深层次因素,或是过去的历史逻辑成了他心理障碍,同时他有士为知己而用的士人品格,挟此诸品格而遭遇桓温,是悲剧的。故事虽短却颇精彩,片言只语而写活了两个生动的具有个性的人物的恩怨,于此见其点睛之妙。}

\lettrine{4.81} 孙兴公\myidx{孙绰}云\footnote{孙兴公:孙绰。}:“\CJKunderwave{三都}、\CJKunderwave{二京}\footnote{\CJKunderwave{三都}、\CJKunderwave{二京}:皆赋名。},五经鼓吹\footnote{五经:指\CJKunderwave{周易}、\CJKunderwave{尚书}、\CJKunderwave{诗经}、\CJKunderwave{仪礼}、\CJKunderwave{春秋}等五部儒家经典。鼓吹:宣扬、宣传。}。”{\fzxk\zihao{6}\textcolor{red}{言此五赋,是经典之羽翼。}}

{\cangkai\zihao{5}【评】张衡、左思都是崇尚儒家正统思想的学者,其对社会的理解,是期望儒家制度理想化的实现,因此在他们的\CJKunderwave{二京赋}、\CJKunderwave{三都赋}的精神气质上都是儒家的。作品中描绘的宫室建构、军国之制、朝臣仪方、民生状态等等,都体现了儒家礼乐思想的彬彬之盛,气势恢宏,大国风范。孙绰崇尚道家,是当时玄言诗的代表作家,锺嵘\CJKunderwave{诗品}说他“弥善恬淡之词”,即指其作品,比一般人更善于表达\CJKunderwave{庄}、\CJKunderwave{老}思想。所以与玄、道相比较,孙绰敏感地意识到\CJKunderwave{三都}、\CJKunderwave{二京}是儒家经典的形象化、理想化的表达,是“五经鼓吹”。由玄言诗的代表作家孙绰来“鼓吹”\CJKunderwave{三都}、\CJKunderwave{二京},正见魏晋思想之兼容并包,又反映了文学家的宽广胸怀。}

\lettrine{4.82} 谢太傅\myidx{谢安}问主簿陆退\myidx{陆退}\footnote{谢太傅:谢安,(?—358):字无奕,谢安长兄,陈郡阳夏谢氏家族在东晋初期的代表人物之一。主簿:官名,中央和地方郡县所设属官,主管文书簿籍,掌印鉴。}:{\fzxk\zihao{6}\textcolor{red}{\CJKunderwave{陆氏谱}曰:“退字黎民,吴郡人。高祖凯,吴丞相。祖仰,吏部郎。父伊,州主簿。退仕至光禄大夫。”}} “张凭\myidx{张凭}何以作母诔\footnote{诔:叙述死者生平德行的哀悼性文章。},而不作父诔?”退答曰:“故当是丈夫之德,表于事行\footnote{故当:当然。丈夫:男子。表:体现,显现。};妇人之美,非诔不显。”{\fzxk\zihao{6}\textcolor{red}{\CJKunderwave{陆氏谱}曰:“退,凭婿也。”}}

{\cangkai\zihao{5}【评】“诔”为褒人生荣死哀之文,显祖德、述形迹、致哀悼。在男权天下的封建社会,它几乎就是男子的专利,是对逝去者一生的价值确认,而女子的价值是“无攸遂,在中馈”(见\CJKunderwave{周易·家人}),不显现于社会,只在家中料理好内庭家事,因而无所谓社会价值的“荣”或“哀”,这是谢安发问的现实和心理背景,在当时是顺理成章的。陆退倒是开明的,和张凭之作母诔一样,肯定了女性的价值,并且可以使用堂堂皇皇的诔文来张扬。故事记述的是论文情景,而在其背后却是魏晋时风中,那种对人、人情、人性的肯定与张扬。}

\lettrine{4.83} 王敬仁\myidx{王修}年十三,作\CJKunderwave{贤人论}\footnote{王敬仁:王修,长史王濛子。\CJKunderwave{贤人论}文章名,见刘孝标注。\CJKunderwave{晋书}修传作\CJKunderwave{贤全论}。}。长史\myidx{王濛}送示真长\myidx{刘惔}\footnote{真长:刘惔。},真长答云:“见敬仁所作论,便足参微言\footnote{真长:刘惔,字真长,曾任丹阳尹,故称。谢安妻兄,尚明帝女庐陵公主。会稽王司马昱为相,与王濛并为其座上清谈之客。性简贵自重,与王羲之友善。卒年三十六。微言:精神玄妙的言辞,即玄言清谈。}。”{\fzxk\zihao{6}\textcolor{red}{\CJKunderwave{修集}载其论曰:“或问:‘\CJKunderwave{易}称贤人,黄裳元吉,苟未能暗与理会,何得不求通?求通则有损,有损则元吉之称将虚设乎?’答曰:‘贤人诚未能暗与理会,当居然体从,比之理肃(尽),犹一豪之领一梁。一豪之领一梁,虽于理有损,不足以挠梁。贤有情之至寡,豪有形之至小,豪不至挠梁,于贤人何有损之者哉!’”}}

{\cangkai\zihao{5}【评】长史王濛舐犊之情可感,欣喜自家子弟有才,十三岁而能论,并将其文送给好友大名士刘惔鉴赏,欲其题拂称扬之心可鉴。人以文名,是当时风气,欲子弟脱颖而出,固当以文彰显。故事以这一小事、细节,记录了时人心理和当时风气,颇为生动。}

{\cangkai\zihao{5}至于其\CJKunderwave{贤人论}如何,王世懋评价:“此等论,在今世未免抚掌,当时所谓名理乃尔,文章一大厄也。”余嘉锡亦谓:“此论所言,浅薄无取。‘一豪之领一梁’云云,尤晦涩难通。晋人之所谓微言,如此而已。”王、余之论似过苛酷。少年心性,能潜心钻研、写作,无论文章深浅,都表明其好学深思,崇尚思理才情,本自动人;至于其后来的才智不及胜流则又另当别论,不能以此印象,苛责少年王修。}

\lettrine{4.84} 孙兴公\myidx{孙绰}云\footnote{孙兴公:孙绰。}:“潘\myidx{潘岳}文烂若披锦\footnote{潘:潘岳,见\CJKunderwave{言语}107。文:诗。烂:灿烂。指文采华美。},无处不善。{\fzxk\zihao{6}\textcolor{red}{\CJKunderwave{续文章志}曰:“岳为文,选言简章,清绮绝伦。”}} 陆\myidx{陆机}文若排沙简金\footnote{陆:陆机,参\CJKunderwave{晋书}本传,其为吴郡吴县华亭(今上海松江)人,当时著名的文学家。吴亡入晋后,累迁太子洗马、著作郎。曾任平原内史,故称“陆平原”。事成都王颖,颖兴兵攻掌权于洛阳的长沙王司马乂时,任陆机为后将军、河北大都督。机兵败遭谗,与弟陆云同为颖所杀。排沙简金:拨开沙砾,挑选金子。比喻在芜杂中选取精华。},往往见宝。”{\fzxk\zihao{6}\textcolor{red}{\CJKunderwave{文章传}曰:“机善属文,司空张华见其文章,篇篇称善,犹讥其作文大治。谓曰:‘人之作文,患于不才,至子为文,乃患太多也。’”}}

{\cangkai\zihao{5}【评】陆机、潘岳是西晋太康时期,文坛上两颗耀眼的明星,格外引起时辈的关注。东晋孙绰的评价,尽管西晋风气未泯,更艳羡其文采,但还是比较准确地说出了他们诗风的特色。潘岳诗歌突出的特点,就是刻意追求华艳的词采,只有少数诗,如\CJKunderwave{悼亡诗}以言情见长,高于陆机。陆机也极为讲究辞藻,甚而不惜流于堆砌繁冗。但他在语言艺术的创造力的发展上,也具有相当的贡献,如意象描绘的工巧细致,表现着诗人感受的敏锐和刻炼之功。孙绰此评,所谓“往往见宝”,恐不只是对其珠玉词彩的感受,也包括了对其新巧意象创造的评价。陆、潘之文,孙绰之评,所有这些,都是当时文坛的价值崇尚。所谓“采缛于正始,力柔于建安”(\CJKunderwave{文心雕龙·明诗})是后人站在文学发展的角度,去审视当时的创作,而时人并不觉得。孙绰评论,恰表达了时人注重形式,虽江左崇尚玄言,但对辞藻刻炼雕镂,依旧欣赏、沉迷。本则故事的意趣,也就在于它相当真实地传达着当时文士的那种唯美心态。}

{\cangkai\zihao{5}王世贞曰:“然则陆之文,病在多而芜也。余不以为然,陆病不在多而在模拟,寡自然之致。”“陆翩翩藻秀,颇见才致,无奈佻弱何。潘气力胜之,旨趣不足。”古今评价的不同,正见时代审美风尚的变异。}

\lettrine{4.85} 简文\myidx{司马昱}称许掾\myidx{许询}云\footnote{简文:简文帝司马昱,指晋简文帝司马昱(320—372),穆帝年幼即位,昱任抚军大将军总理政务。后来大将军桓温专擅朝政,先废海西公,后立司马昱为帝,第二年崩。许掾:许询,字玄度,见\CJKunderwave{言语}69。}:“玄度五言诗,可谓妙绝时人\footnote{妙绝时人:精妙至极,超越同时代的人。}。”{\fzxk\zihao{6}\textcolor{red}{\CJKunderwave{续晋阳秋}曰:“询有才藻,善属文。自司马相如、王褒、杨雄诸贤,世尚赋颂,皆体则\CJKunderwave{诗}、\CJKunderwave{骚},傍综百家之言。及至建安,而诗章大盛。逮乎西朝之末,潘、陆之徒,虽时有质文,而宗归不异也。正始中,王弼、何晏好\CJKunderwave{庄}、\CJKunderwave{老}玄胜之谈,而世遂贵焉。至过江,佛理尤盛,故郭璞五言始会合道家之言而韵之。询及太原孙绰转相祖尚,又如(袁本作‘加’)以三世之辞,而\CJKunderwave{诗}、\CJKunderwave{骚}之体尽矣。询、绰并为一时文宗,自此作者悉体之。至义熙中,谢混始改。”}}

{\cangkai\zihao{5}【评】简文之性“清虚寡欲,尤善玄言”(\CJKunderwave{晋书·简文帝纪}),并且对王濛、刘惔、许询等能玄言的人悉心叹赏。王、刘是口谈玄理,许询则以玄理制诗,在这里,简文如欣赏王、刘的清谈一样,对许诗评价极高。而文学批评家锺嵘,在其\CJKunderwave{诗品}中,却将玄言诗及其代表作家孙绰、许询特置之下品,其自序里评价他们的作品:“理过其辞,淡乎寡味”,“诗皆平典,似\CJKunderwave{道德论}”。\CJKunderwave{文选}亦未录许诗。许询诗作,据余嘉锡先生\CJKunderwave{笺疏},今存仅如下:\CJKunderwave{艺文类聚}记其竹扇词四句:“良工眇方林,妙思触物骋。篾疑秋蝉翼,团取望舒景。”\CJKunderwave{初学记}引其诗两句:“青松凝素髓,秋菊落芳英。”\CJKunderwave{文选}注引其\CJKunderwave{农里诗}两句:“亹亹玄思得,濯濯情累除。”所谓尝鼎一脔,足知其味。这些诗句,或雕镂字句,步陆机、潘岳后尘,或谈玄理,表清虚之致,其情致皆不足“妙绝时人”。由此可见,简文评的是其玄理兴味,他的激赏,摇曳的是自家性情。这是玄学兴盛的产物,后来锺嵘站在文学家的立场,提出批评,立场不同,评价自然各异旨趣。}

\lettrine{4.86} 孙兴公\myidx{孙绰}作\CJKunderwave{天台赋}成\footnote{孙兴公:孙绰。天台山:在今浙江天台、临海两县境。},以示范荣期\myidx{范启},{\fzxk\zihao{6}\textcolor{red}{\CJKunderwave{中兴书}曰:“范启字荣期,慎阳人。父坚,护军。启以才义显于世,仕至黄门郎。”}} 云:“卿试掷地,要作金石声\footnote{要:应当。金石声:指钟、磬一类乐器的乐音,激越铿锵。此喻文章优美。}。”范曰:“恐子之金石,非宫商中声\footnote{宫商:乐律五音中宫、商二音,代指乐律。}。”然每至佳句,{\fzxk\zihao{6}\textcolor{red}{“赤城霞起而建标,瀑布飞流而界道。”此赋之佳处。}} 辄云:“应是我辈语\footnote{辄:总是。应:的确。}。”

{\cangkai\zihao{5}【评】孙绰“博学善属文,少与高阳许询俱有高尚之志。居于会稽,游放山水,十有馀年”(\CJKunderwave{晋书·孙绰传})。作为清谈家,他的博学是以玄理、释道之学为主要内容的;因其“高尚之志”,十馀年,舍朝堂而游山水,对山水游涉既深,所见所感自与常人不同。这篇\CJKunderwave{天台赋},\CJKunderwave{文选}卷十一名为\CJKunderwave{游天台山赋},写的是“释域中之常恋,畅超然之高情”为标榜的山水玄言之赋。}

{\cangkai\zihao{5}孙绰对此作之所以非常得意,一是高谈了玄理之人生体悟,犹如一回别开生面的清谈;二是属文谨严,描摹入神,词采清丽。他大约是体悟、描写天台山色的第一人,也是最早置身山色,不做静态、旁观摹写的赋家。他把美不胜收的山水,领悟为大道之融化,一路观览、攀缘、聆听都是在“大道”中游,故多佛老之意,但客观上却写出了天台山清新、壮美、生机无限的神韵,画卷随他的游览而一路连续展开,令阅读者有亲历之感。如此规模的山水画卷,在汉魏以来的赋中是第一次出现,自然令人耳目一新。其词彩,一反过去赋中所追求的典重博学,而显得既严谨工致,又平易流畅,给人耳目一新之感。篇末纯讲玄理,今天看来是累赘多馀之笔,而在孙绰则是以玄理至道总合了全赋,且深入了玄理的讲论,是其得意之笔。所以,他因此赋而自负,是自有其道理的。故事就把他神采飞扬的得意形象,描摹得跃然纸上。范启也是“以才义显于当世”(\CJKunderwave{晋书·范启传})的清谈家,对孙绰的自夸将信将疑,然而一旦读来却被征服。}

\lettrine{4.87} 桓公\myidx{桓温}见谢安石\myidx{谢安}作简文\myidx{司马昱}谥议\footnote{桓公:桓温,桓公北征:桓温曾有三次北征,刘盼遂\CJKunderwave{世说新语校笺}考订,此次当为太和四年(369)之征。时桓温已58岁。安石:谢安,字安石,(?—358):字无奕,谢安长兄,陈郡阳夏谢氏家族在东晋初期的代表人物之一。谥:帝王、贵族、大夫死后,按他的一生事迹,依“谥法”给予褒贬的称号。帝王之谥,由礼官议上,官员之谥,由朝廷赐予。},看竟,掷与坐上诸客曰:“此是安石碎金\footnote{竟:毕。碎金:比喻零篇佳作。}。”{\fzxk\zihao{6}\textcolor{red}{刘谦之\CJKunderwave{晋纪}载安议曰:“谨案:\CJKunderwave{谥法}:‘一德不懈曰简,道德博闻曰文。’\CJKunderwave{易}简而天下之理得,观乎人文,化成天下,仪之景行,犹有仿佛。宜尊号曰太宗,谥曰简文。”}}

{\cangkai\zihao{5}【评】本则故事,有两点颇富意趣。一是桓温活灵活现的形貌,二是晋人风尚。}

{\cangkai\zihao{5}桓温废司马奕而立简文,挟天子而令诸侯,实是由自己经营天下,意在取代晋室。本以为简文病重,机会降临,“温初望简文临终禅位于己,不尔便为周公居摄”。但等他从外面赶回来,发现早已安排了谢安、王坦之,并命自己如诸葛亮、王导辅佐少主故事,于是“事既不副所望,故甚愤怨”(见\CJKunderwave{晋书·桓温传})。这就是本则故事桓温的心理背景。此时他一“掷”,毫不掩饰其怨愤轻蔑情绪,也毫不掩饰他一贯雄豪霸道的面貌。对谢安也没客气,表面给了一点肯定,实则等于说,这不过是安石的小把戏。其面貌的霸道,口吻的冷峭,使其形象声容毕肖。}

{\cangkai\zihao{5}周代制“谥法”,本极敦朴凝重,给死去帝王一个定评,给后来皇帝一个楷模或警示。如“一德不懈曰‘简’”、“平易不訾曰‘简’”;“经纬天地曰‘文’”、“道德博闻曰‘文’”等,都很实际。而这里,依刘孝标注,则依玄理论定这位“大行皇帝”,可见谈玄的风尚,无处不在,连议“谥”这样典重的事情,也注入了玄理,这是时风使然。}

\lettrine{4.88} 袁虎\myidx{袁宏}少贫\footnote{袁宏,小字虎。},{\fzxk\zihao{6}\textcolor{red}{虎,袁宏小字也。}} 尝为人佣载运租\footnote{佣:雇佣。}。谢镇西\myidx{谢尚}经船行\footnote{谢镇西:谢尚,谢豫章:谢鲲,曾作豫章太守。刘孝标注“鲲子别见”,“子”字衍。将:携,谓携之送客。自:已经。参:参与、进入。上流:上等、上品。},其夜清风朗月,闻江渚间估客船上有咏诗声\footnote{江渚:江中小洲。估客船:商贩船。},甚有情致\footnote{情致:情味韵致。}。所诵五言,又其所未尝闻,叹美不能已。即遣委曲讯问\footnote{委曲:详细、详尽。},乃是袁自咏其所作\CJKunderwave{咏史诗}。因此相要\footnote{要:通“邀”,邀请。},大相赏得\footnote{赏:赏识。得:满意、亲近。}。{\fzxk\zihao{6}\textcolor{red}{\CJKunderwave{续晋阳秋}曰:“虎少有逸才,文章绝丽,曾为\CJKunderwave{咏史诗},是其风情所寄。少孤而贫,以运租为业。镇西谢尚,时镇牛渚,乘秋佳风月,率尔与左右微服泛江。会虎在运租船中讽咏,声既清会,辞又藻拔。非尚所曾闻,遂住听之,乃遣问讯。答曰:‘是袁临汝郎诵诗,即其\CJKunderwave{咏史}之作也。’尚佳其率有胜致,即遣要迎,谈话申旦。自此名誉日茂。”}}

{\cangkai\zihao{5}【评】故事描写士人风情雅致颇为动人。袁宏为名门之后,“有逸才,文章绝美”,但“少孤贫,以运租自业”(\CJKunderwave{晋书·袁宏传}),经谢尚的称扬提携,才转变命运,成为当世名流。本则记述了其命运转折的这一幕。其\CJKunderwave{咏史诗}曾博得赞誉,锺嵘\CJKunderwave{诗品}列其诗入“中品”,评为“虽文体未遒,而鲜明紧健,去凡俗远矣。”\CJKunderwave{文心雕龙·才略}评论“袁宏发轸以高骧,故卓出而多偏”,这些批评家即使是在纵向、横向的群才比较下,也是颇多赞誉。可见当时文坛对袁宏的认可。谢尚出身门第高华,更兼皇亲国戚,在门阀社会中,眼里何尝有人?但是他爱才,又喜好文学,不仅自己风流儒雅,兴致非凡,而且慧眼识才,在邂逅中敏感地发现了一个默无声闻却卓有才情的文士,这种才士会心,认可他人美才的风范,就尤其动人了。故事对其敏感、心细,为才情所吸引的痴劲,描绘得很真实、生动,在这一过程中展示了其生动形象。这幕情景及谢尚形象,令人一读难忘。}

{\cangkai\zihao{5}至于袁宏的\CJKunderwave{咏史诗},今仅存二首,写的是对历史人物的感受,即所谓“风情所寄”,大致和他\CJKunderwave{三国名臣颂}(见\CJKunderwave{晋书}所录)差不多,就史论史,缺乏自己更深刻的感慨寄托,没有左思\CJKunderwave{咏史诗}的骨气风力,故锺嵘\CJKunderwave{诗品}置左思上品,而袁宏则落中品,这是公允的评价。}

\lettrine{4.89} 孙兴公\myidx{孙绰}云\footnote{孙兴公:孙绰。}:“潘\myidx{潘岳}文浅而净\footnote{潘:潘岳,见\CJKunderwave{言语}107。净:纯净。},陆\myidx{陆机}文深而芜\footnote{陆:陆机,参\CJKunderwave{晋书}本传,其为吴郡吴县华亭(今上海松江)人,当时著名的文学家。吴亡入晋后,累迁太子洗马、著作郎。曾任平原内史,故称“陆平原”。事成都王颖,颖兴兵攻掌权于洛阳的长沙王司马乂时,任陆机为后将军、河北大都督。机兵败遭谗,与弟陆云同为颖所杀。芜:芜杂。}。”

{\cangkai\zihao{5}【评】参见本篇84则,刘应登曰:“此二语又自作‘披锦’、‘排沙’注脚。”}

\lettrine{4.90} 裴郎\myidx{裴启}作\CJKunderwave{语林},始出\footnote{裴郎:指裴启。\CJKunderwave{语林},晋裴启撰,又名\CJKunderwave{裴子},原书十卷,记载汉、魏、晋人物的佚事、言论。\CJKunderwave{世说新语}多取材于此书,亦为唐人修\CJKunderwave{晋书}所取。原书已佚,今有鲁迅\CJKunderwave{古小说钩沉}辑本。},大为远近所传。时流年少,无不传写,各有一通\footnote{一通:一篇。}。载王东亭\myidx{王珣}作\CJKunderwave{经王(黄)公酒垆下赋}\footnote{王东亭:王珣。“王公”当作“黄公”。},甚有才情。{\fzxk\zihao{6}\textcolor{red}{\CJKunderwave{裴氏家传}曰:“裴荣字荣期,河东人。父稚,丰城令。荣期少有风姿才气,好论古今人物。撰\CJKunderwave{语林}数卷,号曰\CJKunderwave{裴子}。”檀道鸾谓裴松之,以为启作\CJKunderwave{语林},荣傥别名启乎?}}

{\cangkai\zihao{5}【评】本书\CJKunderwave{轻诋篇}注引\CJKunderwave{续晋阳秋}曰:“晋隆安中,河东裴启撰汉魏以来迄于今时,言语应对之可称者,谓之\CJKunderwave{语林}。时人多好其事,文遂流行。”时人崇尚风流雅望,所以裴启撰\CJKunderwave{语林},叙魏晋名流言语、佚事,很能刺激当世。特别是年轻人,憧憬着未来,希望如同昔日的风流人物,活得有声有色,故“无不传写”,津津欣赏。透过纸面,可以想见,这些年轻人对\CJKunderwave{语林}中名流的崇拜、痴迷,颇类今日之“追星族”。本则就记录了这一风尚和年轻人特有的心理。}

{\cangkai\zihao{5}故事中的王东亭赋,见\CJKunderwave{伤逝篇}:“王濬冲为尚书令,经黄公酒垆下过。顾谓后车客:‘吾昔与嵇叔夜、阮嗣宗共酣饮于此垆。今日视此虽近,邈若山河。’”又\CJKunderwave{轻诋篇}:庾道季“陈东亭\CJKunderwave{经酒垆下赋}”。此赋今不传,余嘉锡\CJKunderwave{笺疏}认为:“东亭正赋此事耳。”则\CJKunderwave{经王(黄)公酒垆下赋}是王珣所作的一篇睹物思人的伤逝感怀之作。人们特别欣赏,除去嵇康、阮籍等为一代名士,自具动人的魅力之外,也可见魏晋时风中的动人情怀。}

\lettrine{4.91} 谢万\myidx{谢万}作\CJKunderwave{八贤论}\footnote{谢万:太傅谢安弟。\CJKunderwave{八贤论}文篇名。},与孙兴公\myidx{孙绰}往反\footnote{孙兴公:孙绰。往反:辩论。},小有利钝\footnote{利钝:锐利迟钝。此为偏义复词,指钝,即词锋滞碍。}。{\fzxk\zihao{6}\textcolor{red}{\CJKunderwave{中兴书}曰:“万善属文,能谈论。”万\CJKunderwave{集}载其叙四隐四显,为八贤之论,谓渔父、屈原、季主、贾谊、楚老、龚胜、孙登、嵇康也。其旨以处者为优,出者为劣。孙绰难之,以谓体玄识远者,出处同归。文多不载。}} 谢后出以示顾君齐\myidx{顾夷},{\fzxk\zihao{6}\textcolor{red}{\CJKunderwave{顾氏谱}曰:“夷字君齐,吴郡人。祖廞,孝廉。父霸,少府卿。夷辟州主簿,不就。”}} 顾曰:“我亦作,知卿当无所名\footnote{名:成名。}。”

{\cangkai\zihao{5}【评】谢万聪明而轻浅,喜自我炫耀,“矜豪傲物,尝以啸咏自高”(\CJKunderwave{晋书·谢万传})。他作\CJKunderwave{八贤论},表达他对人物出、处的见识。自己感觉颇好,拿到名士孙绰那里讨论,又示与顾君齐,体会其心态,大有炫耀意味,结果是两处都碰了钉子,让人感受到谢万兴致勃勃又到处碰钉子的尴尬形象。故事的描写,确实是含蓄而深有意味的。}

{\cangkai\zihao{5}然而,本则更值得注意的是,时人对“出处”问题的认识。当时的共同审美风气,是以“绝俗”为雅,以隐逸为高,这是事物的一面,另一面,依刘孝标注,孙绰之辩难的内容是“体玄识远者,出处同归”。这与王羲之诫勉谢万的话,同一声口:“所谓通识,正自当随事行藏,乃为远耳。”(见\CJKunderwave{晋书·王羲之传}载诫万书)孙绰、王羲之均优游会稽山水,研究玄理,后领职受任,出、处从容。在王羲之是琅邪王氏之后,为王朝倚重的豪族;孙绰为孙楚之后,父祖兄弟颇历要职,以他们的认知背景,对出处的领悟如其宣言:并不拘执于隐逸为高,而是“随事行藏”。这与汉武帝时东方朔的“大隐隐于朝”论不同,少了一些以滑稽、游戏态度对抗强权的意味,多了一份玄理妙悟,也就是理论上的自觉,人生态度的逍遥。这比谢万夸张、偏执之轻狂更有深度。王羲之劝诫谢万是出于世交友好对这个轻狂子弟的爱护;孙绰是和狂士一论高下,而两家所执的认识相同,讲出了当时新的隐逸论,并以此指导人生践履。这是本则故事背后更有意味的东西。}

\lettrine{4.92} 桓宣武\myidx{桓温}命袁彦伯\myidx{袁宏}作\CJKunderwave{北征赋}\footnote{桓宣武:桓温,桓公北征:桓温曾有三次北征,刘盼遂\CJKunderwave{世说新语校笺}考订,此次当为太和四年(369)之征。时桓温已58岁。袁彦伯:袁宏。\CJKunderwave{北征赋}:文篇名。},{\fzxk\zihao{6}\textcolor{red}{\CJKunderwave{续晋阳秋}曰:“宏从温征鲜卑,故作\CJKunderwave{北征赋},宏文之高者。”}} 既成,公与时贤共看\footnote{公:尊称桓温。时贤:当世名流,贤才。},咸嗟叹之。时王珣\myidx{王珣}在坐云\footnote{王珣。}:“恨少一句\footnote{恨:憾。},得‘写’字足韵\footnote{足韵:补足韵脚。},当佳。”袁即于坐揽笔益云\footnote{揽:拿来。益:增加。}:“感不绝于余心,泝流风而独写\footnote{“感不绝”二句:在我心中感慨联翩不能断绝,追溯先贤遗风而独自抒发情怀。}。”公谓王曰:“当今不得不以此事推袁。”{\fzxk\zihao{6}\textcolor{red}{宏\CJKunderwave{集}载其赋云:“闻所闻于相传,云获麟于此野。诞灵物以瑞德,奚授体于虞者。悲尼父之恸泣,似实恸而非假。岂一物之足伤,实致伤于天下。感不绝于余心,遡流风而独写。”\CJKunderwave{晋阳秋}曰:“宏尝与王珣、伏滔同侍温坐,温令滔续其赋,至‘致伤于天下’,于此改韵。云:‘所咏慨深千载。今于“天下”之后便移韵,于写送之致,如为未尽。’滔乃云:‘得益“写”一句,或当小胜。’桓公语宏:‘卿试思益之,’宏应声而益,王、伏称善。”}}

{\cangkai\zihao{5}【评】述征纪行之赋,是东汉以来常见作品。此类赋作的通例,是由所历所睹之场所、景象的具体空间,与古往今来之兴亡旧事融合起来,浮想联翩,将成败兴衰的经验教训和对豪杰寇贼人物的褒贬抒写出来,兴感喟,下针砭,加之鸿篇巨制的载体,往往给人带来震撼。这种赋作,甚见作者的才华与见识。袁宏是当世才子,于太和四年(369)随桓温征前燕,其时桓温正负其才力,怀大志,对袁宏文才甚为知重,所以命作\CJKunderwave{北征赋},在桓温可见其雄豪之气,在袁宏为一展才华。本则记才士鉴赏此赋,王珣建议以“写”字足韵。该赋今仅存片段,不睹全貌,但就孝标所录情形看,前文为昔日获麟于野,孔子哀叹天下动乱,此是客观叙写,如果于此再加一“写”韵,文章的情形就不同了。“写”是抒泄个人主观感受,融自身于历史,不仅使文章内涵拓展了,而且以我之感受动人,文章更见活力。袁宏从善如流,应之如响,文章果然大获赞美。刘辰翁甚看重此点,评曰:“谈文有法,补句自佳。”}

\lettrine{4.93} 孙兴公\myidx{孙绰}道曹辅佐\myidx{曹毗}才如白地明光锦\footnote{孙兴公:孙绰。道:评论。白地:白色的底子。},{\fzxk\zihao{6}\textcolor{red}{\CJKunderwave{中兴书}曰:“曹毗字辅佐,谯国人,魏大司马休曾孙也。好文籍,能属辞,累迁太学博士、尚书郎、光禄勋。”}} 裁为负版绔\footnote{负版绔:隶役人穿的裤子。},{\fzxk\zihao{6}\textcolor{red}{\CJKunderwave{论语}曰:“孔子式负版者。”郑氏\CJKunderwave{注}曰:“版,谓邦国籍也。负之者,贱隶人也。”}} 非无文采,酷无裁制\footnote{酷:极,甚。裁制:剪裁制作。}。

{\cangkai\zihao{5}【评】朱铸禹先生述本则:“意谓以锦制负版之绔,用之极不得当,似喻曹之才美而用不得当也。”(见\CJKunderwave{世说新语汇校集注})依\CJKunderwave{晋书},孙绰既是才为当世之冠的名士,也是一位颇具幽默性格的人,本传说他“性通率,好讥调”。这里所记,正是他以幽默的口吻,评议当时文士的情趣、风貌。看来在孙绰眼里,这位曹辅佐对他的美才,常用之极不得当,他用了一个极夸张的比喻,用高华绚丽的“光明锦”,裁制成做粗活的工作服“负版绔”,便构成了一个天大的玩笑。如此,则曹辅佐之才和其用才也便具有喜剧色彩,让人感受到了这一评论,在幽默之中隐含着的尖刻嘲讽。}

\lettrine{4.94} 袁伯彦\myidx{袁宏}作\CJKunderwave{名士传}成\footnote{袁彦伯:袁宏。\CJKunderwave{名士传}:书名。},{\fzxk\zihao{6}\textcolor{red}{宏以夏侯太初、何平叔、王辅嗣为正始名士,阮嗣宗、嵇叔夜、山巨源、向子期、刘伯伦、阮仲客(容)、王濬冲为竹林名士,裴叔则、乐彦辅、王夷甫、庾子嵩、王安期、阮千里、卫叔宝。谢幼舆为中朝名士。}} 见谢公\myidx{谢安},公笑曰\footnote{谢公:谢安,(?—358):字无奕,谢安长兄,陈郡阳夏谢氏家族在东晋初期的代表人物之一。}:“我尝与诸人道江北事\footnote{江北事:长江下游以北地区。},特作狡狯耳\footnote{特:只,不过。狡狯:此指戏谑,谈笑间的随意之言。},彦伯遂以箸书\footnote{箸:同“著”。}。”

{\cangkai\zihao{5}【评】本则入“文学门”,是在表彰着袁宏的文才,但作为故事,却给人另一番意趣。}

{\cangkai\zihao{5}\CJKunderwave{晋书}本传载有袁宏的\CJKunderwave{三国名臣颂},评论三国时著名的文臣武将,见出他对名人的浓厚兴趣。依刘注,\CJKunderwave{名士传}专记名动当时的风流名士,在袁宏是对这些人物的欣赏、仰慕。\CJKunderwave{名士传}的来源,当然不只是谢安的言说,但谢安却下以“狡狯”二字,恰从另一面道出自己对\CJKunderwave{名士传}的贡献。两人形象,一个代表了当时崇尚名流的风尚,一个则是活脱名士风度。}

\lettrine{4.95} 王东亭\myidx{王珣}到桓公\myidx{桓温}吏\footnote{王东亭:王珣。桓公:桓温,桓公北征:桓温曾有三次北征,刘盼遂\CJKunderwave{世说新语校笺}考订,此次当为太和四年(369)之征。时桓温已58岁。},既伏閤下,桓令人窃取其白事\footnote{白事:报告的文书。}。东亭即于閤下更作,无复向一字\footnote{向:刚才。}。{\fzxk\zihao{6}\textcolor{red}{\CJKunderwave{续晋阳秋}曰:“珣学涉通敏,文高当世。”}}

{\cangkai\zihao{5}【评】王珣“弱冠与陈郡谢玄为桓温掾,俱为桓温所敬重”(\CJKunderwave{晋书·王珣传}),本则表现了他敏捷的才思。他报告文书的原本被“窃”走,又在特定的场合,有限的时间内重写,他能挥笔而就,并且一字不与前本重复,这见出了他的才思。故意不与前本有一字重复,也可见他是情知桓温用心的,若将写过的报告复述一遍,这并不难,而无一字重复,则需真才实学了。此情此景,等于是桓温对他临场考试,他交出了令人满意的答卷。故事里跃动出一个活生生的才子形象。}

\lettrine{4.96} 桓宣武\myidx{桓温}北征\footnote{桓宣武:桓温,桓公北征:桓温曾有三次北征,刘盼遂\CJKunderwave{世说新语校笺}考订,此次当为太和四年(369)之征。时桓温已58岁。},{\fzxk\zihao{6}\textcolor{red}{\CJKunderwave{温别传}曰:“温以太和四年上疏,自征鲜卑。”}} 袁虎\myidx{袁宏}时从\footnote{袁虎:袁宏。},被责免官。会须露布文\footnote{会:正碰见。露布文:指公告、檄文之类,不需要封缄,迅速公布四方的文书。},唤袁倚马前令作。手不辍笔\footnote{辍:停。},俄得七纸\footnote{俄:一会儿。},绝可观。东亭\myidx{王珣}在侧\footnote{东亭:王珣。},极叹其才。袁虎云:“当令齿舌间得利\footnote{齿舌:言语辞令。}。”

{\cangkai\zihao{5}【评】\CJKunderwave{晋书·桓温传}载:桓温北伐,过淮、泗,践北境,与僚属登楼望中原,将亡国之责委之于崇尚清谈的西晋宰辅重臣王衍,对此袁宏发表不同意见:“运有兴废,岂必诸人之过?”桓温作色,讲了一个杀蠢牛的故事比况袁宏,“坐中皆失色”,本书\CJKunderwave{轻诋}第11则记此事说“袁亦失色”,可见桓温当时的气急败坏。本则云“被责免官”,当即前述背景。在这样的心理压力下,袁宏能倚马为文,一挥而就,“得七纸”之多,可见他腹笥富厚与才思敏捷。刘辰翁云:“谓露布流传,须剪裁浏亮可称诵”,见其文才非凡。故事用简洁的笔墨描写了下笔千言、倚马可待的才子风流,灵动可爱。}

{\cangkai\zihao{5}至于王珣的评论,王世懋说:“按此语最深难解。言袁有此才,而官不利,徒得东亭叹赏齿舌得利而已,何益于事?自古文人同恨。”而刘应登则曰:“王批固明,虽然,才宁独以官为利耶?正难得知己赏识耳。一言赞叹,重于九迁。袁是欣语,非愤语。亦是自信语,非不足语。”(见朱铸禹\CJKunderwave{世说新语汇校集注})二说相较,刘解更为深刻,合乎袁宏的名家风度。}

\lettrine{4.97} 袁宏\myidx{袁宏}始作\CJKunderwave{东征赋}\footnote{袁宏。\CJKunderwave{东征赋}:文篇名。},都不道陶公\myidx{陶侃}\footnote{陶公:陶侃。}。胡奴\myidx{陶范}诱之狭室中\footnote{胡奴:侃子陶范,字道则,小字胡奴。官历乌程令、光禄勋。},临以白刃\footnote{白刃:雪亮的刀、剑。},{\fzxk\zihao{6}\textcolor{red}{胡奴,陶范。别见。}} 曰:“先公勋业如是\footnote{先公:指死去的父亲。},君作\CJKunderwave{东征赋},云何相忽略\footnote{云何:为什么。}?”宏窘蹙无计,便答:“我大道公,何以云无?因诵曰:‘精金百炼,在割能断\footnote{“精金”句:好钢经过百炼,切物一割即断。此喻陶侃如百炼之钢,确为干才。},功则治人,职思靖乱\footnote{职:居官任职。靖:平定。}。长沙之勋\footnote{长沙:指陶侃,被封长沙郡公。},为史所赞\footnote{赞:赞美,称颂。}。’”{\fzxk\zihao{6}\textcolor{red}{\CJKunderwave{续晋阳秋}曰:“宏为大司马记室参军,复为\CJKunderwave{东征赋},悉称过江诸名望。时桓温在南州,宏语众云:‘我决不及桓宣武。’时伏滔在温府,与宏善,苦谏之,宏笑而不答。滔密以启温,温甚忿,以宏一时文宗,又闻此赋有声,不欲令人显问之。后游青山饮酌,既归,公命宏同载,众为危惧。行数里,问宏曰:‘闻君作\CJKunderwave{东征赋},多称先贤,何故不及家君?’宏答云:‘尊公称谓,自非下官所敢专,故未呈启,不敢显之耳。’温乃云:‘君欲为何辞?’宏即答云:‘风鉴散朗,或搜或引。身虽可亡,道不可陨。则宣城之节,信为允也。’温泫然而止。”二说不同,故详载焉。}}

{\cangkai\zihao{5}【评】袁宏作\CJKunderwave{东征赋}不道陶侃,余嘉锡先生分析:“陶侃为庾亮所忌,于其身后奏废其子夏,又杀其子称,由是陶氏不显于晋。当宏作赋时,陶氏式微已甚。其孙虽嗣爵,而名宦不达。陶范虽存,复不为名氏所与。观\CJKunderwave{方正篇}载王修龄却陶胡奴送米,厌恶之情可见。非必胡奴之为人得罪于清议也,直以其家出自寒门,摈之不以为气类,以示流品之严而已。宏之不道陶公,亦犹是耳。”(见\CJKunderwave{世说新语笺疏})如是,则袁宏从时人之习,本未想写陶公,所以面对陶胡奴的突然袭击,没有任何心理准备。在白刃当前、性命攸关的紧急情况下,他临机应变,张口成诵,并且毫无破绽,如同宿构,这就不是一般的文思敏捷了,让人读来有奇才之感。本则突现了袁宏非凡的急智与文才。}

\lettrine{4.98} 或问顾长康\myidx{顾恺之}\footnote{顾长康:顾恺之,字长康,小字虎头。}:“君\CJKunderwave{筝赋}何如嵇康\myidx{嵇康}\CJKunderwave{琴赋}\footnote{\CJKunderwave{筝赋}:文篇名,今不传。 嵇康: 嵇康(223—262):三国时谯郡铚(今安徽亳县)人。“竹林七贤”之一。曾任中散大夫,故称嵇中散。当时著名思想家、文学家、清谈名家。因其主张越名教而任自然,抨击礼法之士,不与司马氏统治集团合作,盛年被杀。\CJKunderwave{琴赋}文篇名, 见\CJKunderwave{文选}卷十八。}?”顾曰:“不赏者作后出相遗\footnote{赏:赏识。遗:舍弃。},深识者亦以高奇见贵\footnote{贵:重视。}。”{\fzxk\zihao{6}\textcolor{red}{\CJKunderwave{中兴书}曰:“凯(恺)之博学有才气,为人迟钝而自矜尚,为时所笑。”宋明帝\CJKunderwave{文章志}曰:“桓温云:‘顾长康体中痴黠各半,合而论之,正平平耳。’世云有三绝:画绝、文绝、痴绝。”\CJKunderwave{续晋阳秋}曰:“恺之矜伐过实,诸年少因相称誉以为戏弄。为散骑常侍,与谢瞻连省,夜于月下长咏,自云得先贤风制。瞻每遥赞之,恺之得此,弥自力忘倦。瞻将眠,语槌脚人令代,恺之不觉有异,遂几申之(旦)而后止。”}}

{\cangkai\zihao{5}【评】顾恺之“三绝”(文绝、画绝、痴绝)之誉,为当世才子。然其\CJKunderwave{筝赋}今不见,而\CJKunderwave{文选}有嵇康的\CJKunderwave{琴赋}。\CJKunderwave{文选}将东汉以来的咏乐器、乐舞诸赋归为“音乐”类,如果\CJKunderwave{筝赋}入选,亦当属此类。这类辞赋有一个基本的章法、套路,即先赋乐器材质之出产环境,再赋其音乐特色,后讲其音乐的感人作用,并且赋家亦有通该乐器者。如马融赋长笛,他自己就“有俊才,好吹笛”(\CJKunderwave{文选}李善注);嵇康赋琴,自己就是古琴演奏家。所以,尽管同一章法,因有切身体会,故所赋皆入神。顾恺之有才,然史书未记其于音乐有何特长。其\CJKunderwave{筝赋}如何,能否比肩嵇康\CJKunderwave{琴赋},待考。不过顾恺之的回答却颇富神采。凌濛初曰:“后出相遗,人人然,古亦然,今亦然。”人们往往因先入为主的习惯,所以顾恺之强调因后出而被人忽略,这种回答很入理;而识者称贵之说,又很俏皮,一面表现了他的自信、自负,是其性格中“矜伐过实”特点的反映,一面也说明了他机敏过人,是其性格中“黠”的反映,凡不“以高奇见贵”者,皆不是识家。读本则,最动人之处是将顾恺之性格特点都形象表现出来,令人感到生动可爱,至于其\CJKunderwave{筝赋}与嵇康\CJKunderwave{琴赋}孰高孰低,便不重要了。}

\lettrine{4.99} 殷仲文\myidx{殷仲文}天才宏瞻\footnote{宏瞻:宏大丰富。瞻,义当是“赡”字。},{\fzxk\zihao{6}\textcolor{red}{\CJKunderwave{续晋阳秋}曰:“仲文雅有才藻,著文数十篇。”}} 而读书不甚广博,亮\myidx{傅亮}叹曰\footnote{亮:傅亮,字季友。东晋官中书黄门侍郎,入宋官至尚书令、光禄大夫。}:{\fzxk\zihao{6}\textcolor{red}{亮,别见。}} “若使殷仲文读书半袁豹\myidx{袁豹}\footnote{半袁豹:有袁豹的一半。},{\fzxk\zihao{6}\textcolor{red}{丘渊之\CJKunderwave{文章叙}曰:“豹字士蔚,陈郡人。祖耽,历阳太守。父质,琅邪内史。豹隆安中著作佐郎,累迁太尉长史、丹阳尹。义熙九年卒。”}} 才不减班固。”{\fzxk\zihao{6}\textcolor{red}{\CJKunderwave{续汉书}曰:“固字孟坚,右扶风人。幼有隽才,学无常师。善属文,经传无不究览。”}}

{\cangkai\zihao{5}【评】殷仲文文名甚著,桓玄为乱,就令他总领诏命,玄加九锡,文是“仲文之辞”,当时权要人物也都爱重其文才。本则和\CJKunderwave{晋书}都拿他和袁豹相比,说他文多而读书少。袁豹是以“好学博文,多览典籍”著名的,并“善言雅俗,每商较古今,兼以诵咏,听者忘疲”(见\CJKunderwave{宋书·袁豹传}),其才美也为当世所知重。这里论者之意:两相比较,袁豹学富,仲文才隽,各有千秋。但对仲文的少读书,颇有叹恨,时人认为,若殷能学,以其宏赡之才,可追班固。\CJKunderwave{文学篇}里选录了此事,反映的是当时人们对才学的崇尚,其中又含着对才与学之间必然关系的关注和理解。在重才风气之下,人们也并不盲目,依然看重学养的意义。而由此观察殷仲文,则是一个很值得玩味的现象了。其才“宏赡”,而其人并非读书、识理的种子,慕浮华,且嗜财极欲,特重尘想,并无操守,就其人的风格说来,骨子里缺乏魏晋风流,所以时人拿他和班固比才,实对殷仲文深入骨子的一种嘲讽。故事的精彩,恰是在简短的对比、评价当中,把殷仲文的面貌,入木三分地揭示出来了。同时,也见出了时人对才士理解的一个尺度,即才与学并重。}

\lettrine{4.100} 羊孚\myidx{羊孚}作\CJKunderwave{雪赞}云\footnote{羊孚:见刘孝标注。羊后投桓玄,玄用为记室参军,为桓心腹。\CJKunderwave{雪赞}:文篇名,今存\CJKunderwave{艺文类聚}卷二。}:“资清以化\footnote{资:凭,依靠。清:清冷。化:成形。},乘气以霏\footnote{乘:驾驭。霏:形容雪的联翩之盛。}。遇象能鲜\footnote{象:物象。},即洁成辉\footnote{即:接触。}。”桓胤\myidx{桓胤}遂以书扇\footnote{书扇:写在扇子上。}。{\fzxk\zihao{6}\textcolor{red}{\CJKunderwave{中兴书}曰:“胤字茂祖,谯国人。祖冲,太尉。父嗣,江州刺史。胤少有清操,以恬退见称。仕至中书令。玄败,徙安成郡,后见诛。”}}

{\cangkai\zihao{5}【评】羊孚的\CJKunderwave{雪赞}和郭璞的诗句(参见本篇76则)一样,都是在玄学的背景下,对自然的解读和欣赏。他把雪的面貌、姿态、神采,简洁而生动地描绘出来了,给人一种清新感,也让人寻味这面貌之下的自然之理,所以这省净、鲜活的画面很是动人。桓胤就其性格说来,是个性情中人,“少有清操,虽奕世华贵,甚以恬退见称”(\CJKunderwave{晋书}本传)。所以,\CJKunderwave{雪赞}情景很容易打动他,书之于扇,扇挥于夏,犹如冰雪怡人,暑热自消。其审美功能,值得欣赏、玩味。羊孚\CJKunderwave{雪赞},桓胤书扇,均见清操之性。}

\lettrine{4.101} 王孝伯\myidx{王恭}在京\footnote{王孝伯:王恭,即王忱,因小字佛大,故称。},行散至其弟王睹\myidx{王爽}户前\footnote{行散:魏晋士大夫有服五石散风习,该药服后,须漫步行走以散发药性,此称为“行散”。},{\fzxk\zihao{6}\textcolor{red}{睹,王爽小字也。\CJKunderwave{中兴书}曰:“爽字季明,恭第四弟也。仕至侍中。恭事败,赠太常。”}} 问:“古诗中何句为最?”睹思未答。孝伯咏“所遇无故物,焉得不速老\footnote{“所遇”二句:\CJKunderwave{古诗十九首·回车驾言迈}诗句。}”:“此句为佳。”

{\cangkai\zihao{5}【评】本则反映了服药之风下深层次的心理状态,颇有价值。}

{\cangkai\zihao{5}自何晏服五石散,宣扬“神明开朗”以来,此风“大行于世,服者相寻”(参何晏字平叔,\CJKunderwave{三国志}作何进孙。少有才,正始初为曹爽所用,名盛于天下。好老庄,与夏侯玄、王弼等倡导玄学,开魏晋清谈之风)。在这风习蔓延中,王恭对服药心理的表达,可说是此风之行的一个十分重要的注脚。}

{\cangkai\zihao{5}王恭是在服药行散过程中谈及“古诗”的,服药的感觉与对“古诗”的感觉,在这种情形下有着一种通感。对“古诗”王恭感受最深,或者说对他最富有刺激的是“所遇无故物,焉得不速老”,它是流贯于“古诗十九首”中的最醒目的主题之一——生命的迅疾,转瞬即逝,不能把握。这也是汉代以来,最令人惊心动魄的问题之一。面对现实人生,“古诗”的态度,一是“服食求神仙”,企望长生;一是及时行乐,不负了这短暂的生命。服五石散的感觉是“神明开朗”——顿有舒畅振奋之感,即使不能长生不死,获得眼下这份爽适、愉悦,也如同饮美酒一样,在暂时的满足与刺激中,获得值得珍惜的生命体验。于是服药之行散与古诗之吟哦,就在这里重叠成了一个意义相通的完整的画面。强烈的生命意识、悲剧意识与及时行乐的抗争意识就成了且行且吟的底色,也就是服药之风深层次心理状态的形象表达。}

\lettrine{4.102} 桓玄\myidx{桓玄}常登江陵城南楼\footnote{桓玄、桓南郡:指桓玄(369—404),袭父温之爵南郡公,故称。安帝时任江州刺史、都督荆州八郡诸军事,率军东下,篡晋自立,建国号楚。旋被刘裕击败,斩首京师。杨广(?—399):曾官淮南太守,南蛮校尉,后与弟佺期俱被桓玄攻杀。殷荆州:指殷仲堪。江陵:南郡治所,今湖北江陵。},云:“我今欲为王孝伯\myidx{王恭}作诔\footnote{王孝伯:王恭,即王忱,因小字佛大,故称。诔:叙述死者生平德行的哀悼性文章。}。”因吟啸良久,随而下笔,一坐之间\footnote{一座之间:满座人谈论之间。},诔以之成。{\fzxk\zihao{6}\textcolor{red}{\CJKunderwave{晋安帝纪}曰:“玄文翰之美,高于一世。”玄集载其诔叙曰:“隆安二年九月十七日,前将军青、兖二州刺史太原王孝伯薨。川岳降补,哲人是育。既爽其灵,不贻其福。天道芒昧,孰测倚伏?犬为反噬,犲狼翘陆。岭摧高梧,林残松竹。人之云亡,邦国丧牧。于以诔之,爰旌芳郁。”文多不载书(袁本作“文多不尽载”)。}}

{\cangkai\zihao{5}【评】对于东晋王朝说来,王恭算是一个忠直的朝臣。本传说他“性抗直,深存义节,读\CJKunderwave{左传}至‘奉王命讨不庭’,每辍卷而叹”。在司马道子总揽朝政的时候,王朝确实政治昏乱,他自己整日蓬发昏目,纵酒取乐,多倒行逆施,又任用王国宝等佞小,希望削弱方镇,集中权力。在司马道子手中,东晋王朝已经走向了末路。王恭不仅在朝不畏权臣,直指司马道子的过愆,而且联络桓玄、殷仲堪等起兵,志在匡辅王朝。其为人也颇清简,无贪欲,号称“恭作人无长物”(即王忱,因小字佛大,故称),以国舅和王朝重臣之贵,死时却“家无财帛,唯书籍而已,为识者所伤”。其人还“美姿仪,人多爱悦,或目之云:‘濯濯如春月柳。’”(\CJKunderwave{晋书·王恭传})他二次起兵清君侧,虽然败丧,但这是由于他未谙政治,战略及权谋非其之长所致,犹如荆轲刺秦王,“惜哉剑术疏”(陶潜\CJKunderwave{咏荆轲}),虽败犹荣,其特立独行,人多叹惜。桓玄的“吟啸良久”,是在品味、怀念王恭其人,胸间郁积,不吐不快。本则辞面上是侧重于对桓玄文翰之才的摹写,就中也客观表达了纵是被正统价值标准判断为逆臣贼子的桓玄,他作为一个活生生的人,其个性也是丰富的,也有着重情、爱才的一面,因而在这短短的描述中,故事主人公的形象才会生动起来。}

\lettrine{4.103} 桓玄\myidx{桓玄}初并西夏\footnote{桓玄:桓南郡:指桓玄(369—404),袭父温之爵南郡公,故称。安帝时任江州刺史、都督荆州八郡诸军事,率军东下,篡晋自立,建国号楚。旋被刘裕击败,斩首京师。杨广(?—399):曾官淮南太守,南蛮校尉,后与弟佺期俱被桓玄攻杀。殷荆州:指殷仲堪。并:吞并。西夏:指中原的西部,六朝时以荆楚地区为西夏。},岭(领)荆江二州、二府、一国\footnote{岭:诸本为“领”,是。领,统领。二府:八州都督府和后将军府。一国:指南郡公的封国。}。{\fzxk\zihao{6}\textcolor{red}{\CJKunderwave{玄别传}曰:“玄既克殷仲堪后,扬(杨)佺期遣使讽朝廷,朝廷以玄都督八州,领江州、荆州二刺史。”}} 于时始雪,五处俱贺,五版并入\footnote{版:简牍。}。玄在厅事上\footnote{厅事:厅堂。},版至,即答版后,皆粲然成\footnote{粲然:文辞华美灿烂。}章,不相揉杂\footnote{揉杂:混杂。}。

{\cangkai\zihao{5}【评】余嘉锡先生\CJKunderwave{笺疏}引程炎震云:“隆安三年十二月,桓玄袭江陵,荆州刺史殷仲堪、南蛮校尉杨佺期并遇害。盖玄以南郡公为广州,并殷得荆州,并杨得雍州,又争得桓修之江州,故有五处俱贺之事。”晋安帝隆安三年(399),桓玄占领了殷仲堪、杨佺期、桓修等所领州郡,旋都督荆、江八州及扬、豫八郡,加后将军、开府,就有了五处俱贺的事情。故事突出了桓玄的“文翰之美,高于一世”的文才。在厅堂中“五版并入”,他能从容答谢,而且“粲然成章,不相揉杂”,依各版的具体情况而一一回复,这的确显得文思敏捷,素有才情。\CJKunderwave{晋书}记其隆安四年(400)被斩,“时年三十六”,那么本则所记,就见出桓玄正当三十馀岁的壮盛之年,精力充沛,才情英发的情形。魏晋时人大多是不以成败论英雄。作为故事的主人公,本则写活了桓玄其人的才情风貌。故事体现的是叹美人物才情的时代风尚。}

\lettrine{4.104} 桓玄\myidx{桓玄}下都\footnote{桓玄:桓南郡:指桓玄(369—404),袭父温之爵南郡公,故称。安帝时任江州刺史、都督荆州八郡诸军事,率军东下,篡晋自立,建国号楚。旋被刘裕击败,斩首京师。杨广(?—399):曾官淮南太守,南蛮校尉,后与弟佺期俱被桓玄攻杀。殷荆州:指殷仲堪。下都:到京都。晋安帝元兴元年(402),桓玄反,率军攻入京城建康。},羊孚\myidx{羊孚}时为兖州别驾\footnote{羊孚:见刘孝标注。羊后投桓玄,玄用为记室参军,为桓心腹。兖州:此指东晋时在京口(今镇江)所置的侨郡,史称南兖州。别驾:官名,州刺史的重要佐吏。},从京来诣门\footnote{京:京口。},笺云\footnote{笺:拜笺,拜帖。}:“自顷世故睽离\footnote{世故:世事。睽离:背离。},心事纶蕰\footnote{纶蕰:隐藏,郁结。} 。明公启晨光于积晦,澄百流以一源\footnote{“明公”二句:你能开启晨光于黑暗之中,澄清百流而统一水源。意谓带来光明,治理时局。}。”桓见笺,驰唤前云:“子道,子道,来何迟!”即用为记室参军\footnote{用:任用。记室参军:诸王、三公、将军所置属官,掌表章、文书等。}。孟昶\myidx{孟昶}{\fzxk\zihao{6}\textcolor{red}{别见。}} 为刘牢之\myidx{刘牢之}主簿\footnote{孟昶:字彦达,东晋平昌(今山东安丘南)人,曾官丹阳尹,后卢循攻石头,他饮鸩而死。},{\fzxk\zihao{6}\textcolor{red}{\CJKunderwave{续晋阳秋}曰:“牢之字道坚,彭城人,世以将显。父遁,征虏将军。牢之沈毅多计数,为谢玄参军。符(苻)坚之役,以骁猛成功。及平王恭,转徐州刺史。桓玄下都,以牢之为前锋,行征西将军。玄至,归降,用为会稽内史。欲解其兵,奔而缢死。”}} 诣门谢,见云:“羊侯,羊侯,百口赖卿\footnote{百口:指全家。}。”

{\cangkai\zihao{5}【评】羊孚官历太学博士、兖州别驾,是个富有文才的士人,对桓玄这名动天下的人物颇为钦敬,他笺牍所叙不是违心的阿谀逢迎,文辞简约,却圆满陈述了对桓玄的敬服之情。桓玄也深相敬重,马上用为腹心之任的记室参军。但这里还是以羊孚为衬托,刻意突出了桓玄的形象。桓玄的性格结构中,本自有一段魏晋才子的情痴癖性,这在\CJKunderwave{晋书}本传,及\CJKunderwave{世说}中都有表达,如其在兄桓伟的丧服期,因公除服,便着急听音乐,“初奏,玄抚节恸哭,既而收泪尽欢”,又如其痴迷书画的癖好等等,都反映了他性格的这一侧面。本则“子道,子道,来何迟?”促语疾呼,将桓玄的渴求之心,写得声情如绘,很有一些爱才痴情的生动。一个细节描绘,写出了一个人物的性格侧面,使得人们对他印象深刻,这是本则的成功之笔。至于后半段,不过是更加烘托出桓玄当时的能量、地位,因他信用羊孚,人们就可通过羊孚来保护自己,从中也说明着桓玄对羊孚的信赖,真正要描写的,还是桓玄的爱才痴情之性。整个故事既有直接的正面描写,又有烘云托月的侧面曲笔,简短的一则记述,可称是神驰笔追,尽其妙致。}




%%% Local Variables:
%%% mode: latex
%%% TeX-engine: xetex
%%% TeX-master: "../Main"
%%% End:
