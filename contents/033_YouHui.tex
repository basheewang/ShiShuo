%% -*- coding: utf-8 -*-
%% Time-stamp: <Chen Wang: 2025-12-09 21:58:38>

% ○ ◎ ‧ 「 」 『 』 々 ( ) “ ” ■ ^[一-龥]
% 【\([^】][^】][^】]+\)】 → {\\fzxk\\zihao{6}\\textcolor{red}{\1}}
% \(【评】.*\) → {\\cangkai\\zihao{5}\1}
% \(【题解】.*\) → {\\cangkai\\zihao{5}\1}
% 《\([^》]+\)》 → \\CJKunderwave{\1}
% ^\([0-9]+.[0-9]+\) → \\lettrine{\1}
% {\\fzxk\\zihao{6}\\textcolor{red}{[^o]*}}

\setlength{\parindent}{0pt}


\chapter{尤悔第三十三}




{\cangkai\zihao{5}【题解】 尤悔者,罪尤过错与懊恼悔恨也。\CJKunderwave{尤悔}门所记载的是魏晋士人因其言行过失所产生的悔恨并以此警醒世人。典出\CJKunderwave{论语·为政}篇孔子之言:“多闻阙疑,慎言其馀,则寡尤;多见阙殆,慎行其馀,则寡悔。言寡尤,行寡悔,禄在其中矣。”夫子教导说,只要言行谨慎,少犯错误,仕途自会顺利。这是正面教育的理想境界。实际上,人的一生,无论是在官场中混,还是在家庭生活中,谁能不犯错误?如能做到孔子所说的“寡尤”、“寡悔”——尽量减少错误和悔恨,已属不易。一般人会犯上许许多多的错误过失,关键在于知错能改以及错误的性质和罪过的大小。人生过失寻常事,时有懊悔不奇怪。但有时错误很大,或是时机因缘凑泊,一失足成千古恨,终于追悔莫及。在历史发展中,社会斗争极其复杂,因此产生尤悔之事千千万万。本门共录十七则故事,不仅刻画了魏晋士人的内在心理,同时形象反映了门阀制度、权力斗争、尔虞我诈黑暗社会的诸多方面,很有参考价值。}

\lettrine{33.1} 魏文帝\myidx{曹丕}忌弟任城王\myidx{曹彰}骁壮\footnote{魏文帝:曹丕,操次子。公元220年篡汉称魏,崩谥号文皇帝,故称。任城王:曹彰。操子,丕弟。封任城王,故称。骁壮:勇猛强壮。},因在卞太后\myidx{卞太后}閤共围棋\footnote{卞太后:曹操之妾。原为倡家,生丕、彰、植,丕称帝后,尊为太后。},并啖枣,文帝以毒置诸枣蒂中\footnote{蒂:瓜果与枝藤相连处。},自选可食者而进。王弗悟\footnote{弗悟:不明白,不知道。},遂杂进之。既中毒,太后索水救之,帝预敕左右毁瓶罐\footnote{预敕:事先命令。}。太后徒跣趋井\footnote{徒跣:赤脚而走,不及穿鞋袜。表示仓猝。},无以汲,须臾遂卒\footnote{须臾:一会儿。}。{\fzxk\zihao{6}\textcolor{red}{\CJKunderwave{魏略}曰:“任城威王彰,字子文,太祖卞太后弟(第)二子。性刚勇而黄须。北讨代郡,独与麾下百馀人突虏而走。太祖闻曰:‘我黄须可用也。’”\CJKunderwave{魏志(氏)春秋}曰:“黄初三年,彰来朝。初,彰问玺绶,将有异志,故来朝不即得见,有此忿惧而暴薨。”}} 复欲害东阿\myidx{曹植}\footnote{东阿:曹植曾封东阿王,故称。},太后曰:“汝已杀我任城,不得复杀我东阿!”{\fzxk\zihao{6}\textcolor{red}{\CJKunderwave{魏志·方伎传}曰:“文帝问占梦周宣:‘吾梦磨钱文,欲灭而愈更明,何谓?’宣怅然不对。帝固问之,宣曰:‘陛下家事,虽欲尔,而太后不听,是以欲灭更明耳。’帝欲治弟植之罪,逼于太后,但加贬爵。”}}

{\cangkai\zihao{5}【评】据\CJKunderwave{三国志}彰传及曹植\CJKunderwave{赠白马王彪诗序},故事发生于黄初四年(223)五月朝京师洛阳时,而非三年。诗有云“太息将何为?天命与我违。奈何念同生,一往形不归。孤魂翔故域,灵柩寄京师。……仓卒骨肉情,能不怀苦辛!”表现了失去亲人的真挚的悲痛之情。丕、彰、植为同母兄弟,但是,为了权力,兄弟如同水火。丕毒杀彰,与植之哭彰,形成了鲜明的对比,给人以巨大的感情冲击。彰死后,丕复欲杀植,其内在阴暗心理,在于维护自己的权力中心及无情报复的情绪驱动。权力是个魔鬼,使人疯狂,帝王更是如此。为权力和报复而杀害亲兄弟,是以牺牲血缘亲情为代价,以抛弃优良传统道德为代价的。不幸生于帝王家,灭绝人性如此,悲乎哀哉!}

\lettrine{33.2} 王浑\myidx{王浑(太原)}后妻,琅邪颜氏女\footnote{王浑:魏晋之际有二王浑,一是琅邪王浑,一是太原王浑。此指后者。字玄冲,官至侍中、尚书左仆射、司徒。}。王时为徐州刺史,交礼拜讫\footnote{交礼:新婚时夫妻交拜之礼。拜讫:拜毕。按:此指颜氏女交拜礼毕。},王将答拜,观者咸曰:“王侯州将\footnote{王侯州将:王浑袭爵京陵侯,州将称刺史,时浑任徐州刺史,故称。},新妇州民\footnote{州民:普通百姓。新妇是琅邪人,实属徐州管辖,故称。},恐无由答拜\footnote{无由:没有理由。}。”王乃止。武子\myidx{王济}以其父不答拜不成礼\footnote{武子:王济字武子,浑第二子。参前\CJKunderwave{言语}第24则注。},恐非夫妇,不为之拜,谓为“颜妾”。颜氏耻之,以其门贵,终不敢离。{\fzxk\zihao{6}\textcolor{red}{婚姻之礼,人道之大,岂由一不拜而遂为妾媵者乎?\CJKunderwave{世说}之言,于是乎纰缪。}}

{\cangkai\zihao{5}【评】故事当发生于晋武帝受禅的太始元年(265)以后,时王浑任徐州刺史。魏晋门阀制度的阴影,遍布各个角落,即在家庭夫妇,也不能免其影响。古时妻与妾别,地位犹如主与奴。因为新妇是“州民”,王浑“不答拜不成礼”,可见当时门第等级森严,夫妻之间,也讲出身。浑子济称后母为“颜妾”,更是直接侮辱,成为终生之耻。但颜氏及其家族却因王家“门贵”,只能忍辱受屈而“终不敢离”。当时妇女感情生活的痛苦,于此可见一斑。贵族之家悔恨如此,民间妇女更是无可如何。}

\lettrine{33.3} 陆平原\myidx{陆机}沙(河)桥败\footnote{陆平原:陆机字士衡,吴郡人。祖逊,父抗,东吴一代将相。曾任平原内史,故称。参前\CJKunderwave{言语}第26则注。沙桥:袁本作“河桥”。沙桥与河桥均为桥名,沙桥在江陵,河桥在朝歌附近。据\CJKunderwave{晋书}机传,“列军自朝歌至于河桥”,则作“河桥”,是。},为卢志\myidx{卢志}所谗\footnote{卢志:字子通,范阳人。祖毓,父珽,一代名公。历成都王长史,卫尉卿,尚书郎。},被诛。{\fzxk\zihao{6}\textcolor{red}{王隐\CJKunderwave{晋书}曰:“成都王颖讨长沙王乂,使陆为都督前锋诸军事。”\CJKunderwave{机别传}曰:“成都王长史卢志,与机弟云趣舍不同。又黄门孟玖求为邯郸令于颖,颖教付云,云时为左司马,曰:‘刑馀之人,不可以君民。’玖闻此怨云,与志谗构日至。及机于七里涧大败,玖诬机谋反所致。颖乃使牵秀斩机。先是,夕梦黑慢(幔)绕车,手决不开,恶之。明旦,秀兵奄至。机索戎服,箸衣幍。见秀,容貌自若,遂见害,时年四十三。 军士莫不㳅(流)涕。是日天地雾合,大风折木,平地尺雪。”干宝\CJKunderwave{晋纪}曰:“初,陆抗诛步阐,百口皆尽,有识尤之。及机、云见害,三族无遗。”}} 临刑叹曰:“欲闻华亭鹤唳\footnote{华亭鹤唳:陆机陆云兄弟于吴亡入洛之前,在家乡华亭闭门读书十年。据称华亭出鹤,有鹤巢。此喻陆机生前依恋旧地景物,叹出仕被害之痛。},可复得乎!”{\fzxk\zihao{6}\textcolor{red}{\CJKunderwave{八王故事}曰:“华亭,吴由拳县郊外墅也,有清泉茂林。吴平后,陆机兄弟共游于此十馀年。”\CJKunderwave{语林}曰:“机为河北都督,闻警角之声,谓孙丞曰:‘闻此,不如华亭鹤唳。’”故临刑而有此叹。}}

{\cangkai\zihao{5}【评】故事发生在晋惠帝太安二年(303)十月。当时八王之乱降临中华大地,犹如一个绞肉机在吞噬无数的生命。陆机河桥兵败,说是“尤”——即错失,的确如此;但若按之军法受到严惩,又何“悔”之有?其实,陆机之悔在军事原因之外。一是他不听友人顾荣、戴若思等劝告,在中原行将大乱之际,不急流勇退,而是“负其才望,而志匡世难”,盲目而主动地投入了八王之乱的非正义战争中。其次,是中原士族对江南士人的歧视与偏见。卢志曾于众坐,辱及陆机父祖,故机回骂之为“鬼子敢尔”,参见\CJKunderwave{方正}第18则故事。被谗遇害,正是受到打击报复,祸根早已埋下。南北士人对抗之激烈,思此能无悔乎!还有,就是主子成都王颖的昏庸,麾下中原将领的不听指挥,并诬其“造反”,一个三军统帅,无法调动军队,失败是必然的,思之能无悔乎?总之,作为一个忠心国家、勤于事业的一代名流,稍一不慎,立即粉身碎骨。做人难,做名人更难。但当他参透人生天机之际,却是为时已晚,早已人头落地。人生至此,能无悔乎?这是华亭鹤唳留给人们的深刻的历史教训,值得深思玩味。}

\lettrine{33.4} 刘琨\myidx{刘琨}善能招延\footnote{刘琨:字越石,中山魏昌人。官尚书右丞、并州刺史。参前\CJKunderwave{言语}第35则注。招延:招引延致。},而拙于抚御\footnote{抚御:抚慰驾驭。}。一日虽有数千人归投\footnote{归投:投奔归附。},其逃散而去,亦复如此,所以卒无所建\footnote{卒无所建:最终无所建树。}。{\fzxk\zihao{6}\textcolor{red}{邓粲\CJKunderwave{晋纪}曰:“琨为并州牧,纠合齐盟,驱率戎旅,而内不抚其民,遂至丧军失士,无成功也。”敬胤按:琨以永嘉元年为并州,于时晋阳空城,寇盗四攻,而能收合士众,抗行渊、勒,十年之中,败而能振。不能抚御,其得如此乎?凶荒之日,千里无烟,岂一日有数千人归之!若一日数千人去之,又安得一纪之间以对大难乎?}}

{\cangkai\zihao{5}【评】晋怀帝元嘉元年(307),刘琨临危受命,出任并州刺史,在北方艰苦抗战,抵御强胡。是时兵祸连结,荒年饥岁,饿殍遍地。中原士族大多南迁渡江。而刘琨却志存恢复,不惜独抗强敌。“善能招延,而拙于抚御”,对于一代豪放诗人,或是事实。史称其“在官未期,流人稍复,鸡犬之音复相接矣”,以民众之归心,而坚持十年抗战,其志节干云,气贯长虹。惜其拙于谋略,加以粮尽乏食,士卒离散,时有发生。其母曾批评琨曰:“汝不能弘经略,驾豪杰,专欲除胜己以自安,当何以得济!”故其败亡,虽然外逼强敌,内实“拙于抚御”,由于队伍并不团结,因内乱而自取其败。故刘辰翁评曰:“意气不足持,须是规模宏远,甚可鉴也。”此琨所以兴“功业未及建,夕阳忽西流”(\CJKunderwave{重赠卢谌})之悔叹也。}

\lettrine{33.5} 王平子\myidx{王澄}始下\footnote{王平子:王澄字平子。死前任荆州刺史,被乱兵所败。始下:从长江中游顺流而下。},丞相\myidx{王导}语大将军\myidx{王敦}\footnote{丞相:王导。大将军:王敦。}:“不可复使羌人东行\footnote{羌人:古代我国西部的一个少数民族。按:此实指王澄。东行:王澄应元帝召自荆州东下赴建康。}。”平子面似羌。{\fzxk\zihao{6}\textcolor{red}{按王澄自为王敦所害,丞相名德,岂应有斯言也!}}

{\cangkai\zihao{5}【评】从人品和道德观念而言,如刘孝标所称,“丞相名德,岂应有斯言”!王导岂会随意劝敦杀掉自己的族叔?而且,澄是在经豫章时被江州刺史王敦所杀,时王导在建康,空间距离有千万里,当时没有电报电话和电脑,怎能及时传信劝敦杀澄呢?明显不合事实。}

{\cangkai\zihao{5}但从另一角度考虑,权力是个魔鬼。王澄、王敦与王导,虽然同属琅邪王氏,但澄、戎一支,与敦、导一支较为疏远。当时中原已乱,王导与敦,协助元帝大力经营江东,准备开基立国,故当时民谣有“王与马,共天下”之言。而澄为西京名士,一旦东下建康,是否会对敦、导实权构成威胁呢?史称澄东下时,名出敦右,素为敦所惮。“澄犹以旧意侮敦”,令敦愤怒不堪。敦之杀澄,亦在料中。而当时敦、导一体,正在全力构建琅邪王氏的新权力中心。导平昔劝敦防澄,不令东下,自也可能。因为权力比亲情更重要。}

\lettrine{33.6} 王大将军\myidx{王敦}起事\footnote{王大将军:王敦。起事:指起兵反对东晋朝廷。按:晋元帝永昌元年(322),大将军王敦以清君侧诛刘隗、刁协为名,起兵武昌,攻陷拱卫京师建康的石头城,控制朝廷。},丞相\myidx{王导}兄弟诣阙谢\footnote{丞相兄弟:指王导兄弟。诣阙谢:到朝廷谢罪。}。周侯\myidx{周顗}深忧诸王\footnote{周侯:周顗,字伯仁。袭父浚爵为武城侯,故称。时官尚书左仆射。参前\CJKunderwave{言语}第30则注。},始入,甚有忧色。丞相呼周侯曰:“百口委卿\footnote{百口:百口之家。委:托付。}!”周直过不应。既入,苦相存救。既释,周大说,饮酒\footnote{说:通“悦”。}。及出,诸王故在门\footnote{故:仍在。}。周曰:“今年杀诸贼奴\footnote{今年:此“今年”不是与“去年”、“明年”相对的严格时间概念,在这里是现在进行式的一次性具体行动时间概念,犹言这回,这次,这一下子。诸贼奴:指王敦叛军。},当取金印如斗大,系肘后。”大将军至石头,问丞相曰:“周侯可为三公不\footnote{三公:朝廷最高官位,指太尉、司徒、司空。}?”丞相不答。又问:“可为尚书令不\footnote{尚书令:朝廷处理政事的长官。}?”又不应。因云:“如此,唯当杀之耳!”复默然。逮周侯被害\footnote{逮:及,等到。},丞相后知周侯救己,叹曰:“我不杀周侯,周侯由我而死,幽冥中负此人\footnote{幽冥:阴间,地下。}!”{\fzxk\zihao{6}\textcolor{red}{虞预\CJKunderwave{晋书}曰:“敦克京邑,参军吕漪说敦曰:‘周顗、戴渊,皆有名望,足以惑。视近日之言,无惭惧之色。若不除之,役将未歇也。’敦即然之,遂害渊、顗。初,漪为台郎,渊既上官,素有高气,以漪小器待之,故售其说焉。”}}

{\cangkai\zihao{5}【评】这则故事发生在元帝永昌元年(322)王敦兵下石头之时,既反映了魏晋时代皇室与世家豪族既联合又斗争的复杂关系,同时也暴露了王导这个东晋开国名相灵魂中虚伪丑恶的一面。在门阀社会中,皇室与高门士族共同掌握政权。对士大夫来说,维护家族利益,甚至比忠于朝廷更重要。东晋之初,形成了“王与马,共天下”的局面。琅邪王氏家族的大将军王敦率兵在外,丞相王导辅政于内。但政治天平的暂时平衡很快被打破。晋元帝对琅邪王氏颇多忌惮,于是身边的刘隗、刁协劝其根除王氏势力,连开国元勋王导也不放过。为了家族利益及其个人野心,王敦以清君侧为由兴兵向阙,此事王导默然认同。但敦欲废帝,导不同意而作罢。后明帝时,王敦再次叛逆,王导从王舒处得知消息,迅速告知明帝而预做准备。可见在与朝廷的权力斗争中,琅邪王氏也非铁板一块。但刘隗等并不讲“统战”,劝元帝“悉诛王氏”,以此激化矛盾。为了家族利益,也为了身家性命,王导内心的矛盾和痛苦可知。他与王敦,是亲近的堂兄弟,不待罪又将如何?但当敦得势时,他又默认王敦诛杀周顗等忠义之士,这在当时并不奇怪,是家族利益在起作用。但当他从档案中明白周顗救护自己的态度后,作为一个政治家,良心发现而有“幽冥中负此人”之悔叹,可惜为时已晚。杨慎评云:“是借剑于敦而杀顗也,非敦反乃导反也。”讥评严苛,启人深思。}

\lettrine{33.7} 王导\myidx{王导}、温峤\myidx{温峤}俱见明帝\myidx{司马绍}\footnote{王导:见前注。时为司徒。温峤:见前注。时为中书令。二人俱为辅政大臣。明帝:司马绍,元帝子。},帝问温前世所以得天下之由。温未答,顷,王曰:“温峤年少未谙\footnote{未谙:不熟悉。},臣为陛下陈之。”王乃具叙宣王\myidx{司马懿}创业之始\footnote{宣王:指司马懿。},诛夷名族\footnote{诛夷名族:司马懿集团于正始十年发动政变夺权,诛杀曹爽、何晏、王凌等。},宠树同己\footnote{宠树同己:培植亲信党羽。},及文王\myidx{司马昭}之末高贵乡公\myidx{曹髦}事\footnote{文王:指司马昭,时任魏大将军,甘露五年(260),发动政变,废立自专,弑魏帝曹髦(此前正始五年封郯县高贵乡公)。立曹奂为帝。高贵乡公:曹髦,在甘露政变中被杀。}。{\fzxk\zihao{6}\textcolor{red}{宣王创业,诛曹爽、任蒋济之㳅(流)者是也。高贵乡公之事,已见上。}} 明帝闻之,覆面箸床曰\footnote{床:坐榻。}:“若如公言,祚安得长\footnote{祚:皇位,国统。}!”

{\cangkai\zihao{5}【评】王导以辅政老臣的身份和年轻的明帝说话,显然含有教诫的口气,于此可见,他已走出了王敦事件的阴影,努力在恢复琅邪王氏的作用与影响。其讲话主要强调二点:一是反对政治上的“诛夷名族”,强调高门士族对于朝廷支持的重要性;一是批判本朝先帝“文王”司马昭弑高贵乡公事。在魏之朝,高贵乡公曹髦是帝,是君,司马昭虽然大权在握,但仍然是臣子,身份尚未改变。司马昭指挥贾充等弑主,后又装腔作势地猫哭老鼠,都改变不了不忠的罪名。故西晋提倡以孝治国而羞言“忠”字。王导针对东晋形势,提出批评,实际是为了巩固国家和朝廷,重新强调发扬“忠”的精神。晋明帝听后“覆面箸床”,其愧恨之心态,正是对于祖先罪行的一种思想清算,也可说是时过境迁后的良心发现。}

\lettrine{33.8} 王大将军\myidx{王敦}于众坐中曰:“诸周由来未有作三公者\footnote{诸周:指周顗及其父亲兄弟。父浚安东将军,弟嵩从事中郎,弟谟中护军,顗尚书左仆射。由来:历来,从来。三公:太尉、司徒、司空,朝廷品阶最高的官员。}。”有人答曰:“唯周侯\myidx{周顗}邑五马领头而不克\footnote{周侯:周顗。邑:李慈铭疑“邑”当作“已”,疑是。五马领头而不克:以樗博之戏为喻。五马领头喻局势大好。不克者,惜其不能最后取胜。此借喻周顗被杀事。}。”丈(大)将军曰:“我与周洛下相遇\footnote{洛下:指洛阳。},一面顿尽\footnote{一面顿尽:一见面即诚心相待。}。值世纷纭\footnote{纷纭:混乱。},遂至于此!”因为流涕。{\fzxk\zihao{6}\textcolor{red}{邓粲\CJKunderwave{晋纪}曰:“王敦参军有于敦坐樗蒱,临当成者(都),马头被杀,因谓曰:‘周家奕世令望,而位不至三公。伯仁垂作而不果,有似下官此马。’敦慨然㳅(流)涕曰:‘伯仁总角时,与于东宫相遇,一面披衿,便许之三司。何图不幸,王法所裁,凄怆之深,言何能尽!’”}}

{\cangkai\zihao{5}【评】对王敦而言,周顗忠义害事,挡住自己去路,所以非杀不可。但在杀人之后,眼泪不妨流淌,惺惺作态,虚伪矫饰,以便迷乱人眼,于此方见政治家的本色。王导不简单,王敦也非等闲。这是一对难兄难弟的绝妙表演。}

\lettrine{33.9} 温公\myidx{温峤}初受刘司空\myidx{刘琨}使劝进\footnote{温公:温峤。刘司空:刘琨。按:劝进之时,刘琨为并州刺史,在北方力抗强胡。时温峤为其右司马。永嘉南渡后,峤奉琨命南下劝进。劝进:劝登帝位。特指晋元帝即位。},母崔氏固驻之\footnote{固驻:坚决阻止。},峤绝裾而去\footnote{绝裾:断绝衣袖。}。{\fzxk\zihao{6}\textcolor{red}{\CJKunderwave{温氏谱}曰:“峤父襜,娶清河崔参女。”}} 迄于崇贵,乡品犹不过也\footnote{乡品:魏晋时实行九品中正制,州郡有大小中正官,政府根据乡里舆论品评的高低做参考授官职。},每爵皆发诏。{\fzxk\zihao{6}\textcolor{red}{虞预\CJKunderwave{晋书}曰:“元帝即位,以温峤为散骑侍郎。峤以母亡,逼贼,不得往临葬,固辞。诏曰:‘峤以未葬,朝议又颇有异同,故不拜。其令八坐议,吾将折其衷。’”}}

{\cangkai\zihao{5}【评】魏晋之世,实行九品中正官人法。“尊世胄,卑寒士,权归右姓已”(见\CJKunderwave{新唐书·儒学·柳冲传})。州、郡中正官皆取士族大姓担任,以定品第,藻绘人物。士庶贵贱,不可易也。当时出身血统之贵贱,在门阀制度下产生了恶劣的历史影响,阻碍了社会的进步。温峤是东晋的开国功臣,勋望卓著。为劝进大业,恢复之计,他绝裾南下,为国忘家,但乡品却不予原谅,视为不孝。每次升官,都必须皇帝下特诏。“乡品不过”云云,时过境迁,后世不解,故有“不知绝裾之是非”之言(刘辰翁评)。而一旦置于历史,其义不言自明。中正乡品怪胎,志士仁人何悔?温峤亦然。}

\lettrine{33.10} 庾公\myidx{庾亮}欲起周子南\myidx{周邵}\footnote{庾公:庾亮。起:起用,任用。},子南执辞愈固。庾每诣周,庾从南门入,周从后门出。庾尝一往奄至\footnote{一往奄至:径直前往突然而至。},周不及去,相对终日。庾从周索食,周出蔬食\footnote{蔬食:粗饭素食。},庾亦彊饭\footnote{彊饭:勉强而食。彊,通“强”。},极欢;并语世故\footnote{世故:人情世事。},约相推引\footnote{推引:推荐引进。},同佐世之任\footnote{佐世:辅助朝廷。}。既仕,至将军、二千石\footnote{二千石:指郡守一类的官。汉时郡守禄二千石,故称。},{\fzxk\zihao{6}\textcolor{red}{\CJKunderwave{寻阳记}曰:“周邵,字子南。与南阳翟汤隐于寻阳庐山。庾亮临江州,闻翟、周之风,束带蹑履而诣焉。闻庾至,转避之。亮复密往,值邵弹鸟于林,因前与语。还,便云:‘此人可起。’即拔为镇蛮护军、西阳太守。”其集载与邵书曰:“西阳一郡,户口差实。非履道真纯,何以镇其流遁?询之朝野,佥曰足下。 今具上表,请足下临之,无让。”}} 而不称意, 中宵慨然曰\footnote{中宵:半夜。}:“丈夫乃为庾元规所卖\footnote{庾元规:庾亮字元规。}!”一叹,遂发背而卒\footnote{发背而卒:背生疽病而死。}。

{\cangkai\zihao{5}【评】此则应与\CJKunderwave{栖逸}第9则并读体味。魏晋之时,隐逸成风,栖遁山林,玉辉冰洁,修身无闷,悔吝弗生。生当动荡混浊之世,此乐何如!但真正隐逸者不多,倒是栖隐待聘者比比皆是,如\CJKunderwave{晋书·隐逸传}所称,“征聘之礼贲于岩穴,玉帛之贽委于窐衡”,这实是仕途上的另一终南捷径。周邵与翟汤原本同隐于庐山。翟真高隐之人,“不屑世事,耕而后食”,他人馈赠一无所受,官府征聘坚决拒绝,是个依靠自己劳动生活而淡泊富贵之人。但周邵内心则期望以栖隐博取大富贵。故在庾亮说以“当世之务”后,立即出仕,引发翟汤不满,与之绝交。周仕翟隐,形成鲜明对比。但庾亮量才给官,与周邵内心期望值相差甚远。故周生发被庾亮出卖之叹,其发背疽卒,正是其虚伪矫饰所付出的生命代价。刘辰翁讥其“二千石不自足,以躁死”,一语中其外似淡泊而内欲富贵之病。}

\lettrine{33.11} 阮思旷\myidx{阮裕}奉大法\footnote{阮思旷:阮裕字思旷。参前\CJKunderwave{德行}第32则注。奉:信奉。大法:佛法。},敬信甚至。大儿年未弱冠\footnote{年未弱冠:年龄不到二十岁。弱冠,古时男子二十成人而行加冠礼。},忽被笃疾\footnote{笃疾:重病。}。{\fzxk\zihao{6}\textcolor{red}{\CJKunderwave{阮氏谱}曰:“牖(佣)字彦伦,裕长子也。仕至州主簿。”}} 儿既是偏所爱重,为之祈请三宝\footnote{三宝:佛教称佛、法、僧为三宝。},昼夜不懈,谓至诚有感者\footnote{有感者:有情识之人。},必当蒙祐。而儿遂不济\footnote{不济:无救,喻死。}。于是结恨释氏\footnote{释氏:指佛教。},宿命都除\footnote{宿命:佛教以为人之命运由其前生善恶所定,即宿命论。}。{\fzxk\zihao{6}\textcolor{red}{以阮公智识,必无此弊。脱此非谬,何其惑欤!夫文王期尽,圣子不能驻其年;释种诛夷,神力无以延其命。故业有定限,报不可移。若请祷而望其灵,匪验而忽其道,固陋之徒耳,岂可与言神明之智者哉!}}

{\cangkai\zihao{5}【评】阮裕出于陈留阮氏这一高门士族,本身一代名士,颇富理识,思辨清晰,为谢安讲解\CJKunderwave{白马论},时人所难。但就是这么一个智识精英,却有其智识盲区,从盲目迷信佛法,到一概否定佛法,祈请既惑,感恨尤误,全凭长子之生死牵动,而缺乏理性的思考与认识。刘孝标为之辩诬,以为“阮公智识,必无此弊”。但王世懋反驳说:“注理高,但人情未必。”所论甚是。裕痛爱子女,因亲情之痛而一时丧失理智,并非不可能,这正写出了一个有血有肉有感情有缺点的名士全貌。}

\lettrine{33.12} 桓宣武\myidx{桓温}对简文帝\myidx{司马昱}\footnote{桓宣武:桓温。 简文帝:司马昱。},不甚得语\footnote{不甚得语:谓语不投机,很少说话。}。废海西\myidx{司马奕}后\footnote{废海西:太和六年(371),桓温北征枋头大败后,以废立树威朝廷,废帝司马奕为海西公,立简文为帝。},宜自申叙,乃豫撰数百语\footnote{豫:预先。},陈废立之意。既见简文,简文便泣下数十行。宣武矜愧\footnote{矜愧:矜怜愧疚。},不得一言。

{\cangkai\zihao{5}【评】桓温与简文,一豪族军阀,一文人皇帝,既互相利用,又彼此斗争,明显是一对矛盾。温一代枭雄,自负才力,早已不满足于“桓与马,共天下”的局面,他是久怀异志,觊觎帝座,欲先立功河朔,还受九锡。但太和四年(369)北伐,于枋头为燕所败。问计郗超,于是在太和六年(371)仓促行废立之计,废帝司马奕为海西公,立简文帝司马昱。一切军权政权,全在桓氏掌控之中,简文仅是傀儡皇帝,日夜忧心司马氏国祚不长,对此能无泣乎?“泣下数十行”,自然奔涌,真情迸发。桓温欺上吓下,又不得不依例“陈废立之意”,矫饰之伪,情态毕现,欺人孤寡,意在夺人江山,虽尚未实现,但内心能无愧乎!此所以面对简文而“不得一言”也。}

\lettrine{33.13} 桓公\myidx{桓温}卧语曰:“作此寂寂,将为文\myidx{司马昭}、景\myidx{司马师}所笑\footnote{文景:指晋文王司马昭、晋景王司马师。懿子,兄弟二人为司马氏篡魏开晋奠定了必要的基础。}。”既而屈起坐曰\footnote{屈起:勃然坐起。}:“既不能㳅(流)芳后世,亦不足复遗臭万载邪\footnote{不足:不值得。}?”{\fzxk\zihao{6}\textcolor{red}{\CJKunderwave{续晋阳秋}曰:“桓温既以雄武专朝,任兼将相,其不臣之心,形于音迹。曾卧对亲僚,抚枕而起曰:‘为尔寂寂,为文、景所笑。’众莫敢对。”}}

{\cangkai\zihao{5}【评】故事发生在桓温废立自专的晚年。垂垂老翁,来日无多,时不我待,不臣野心,不能不孤注一掷。其行废立者,为将来桓氏帝国之开基作铺垫也。但作为久经斗争的政治家,他明知这在政治上是极险的一步棋。冒险之事,胜王败寇,有侥幸成功的机会,更有失败后遭人唾骂的可能。故勃然“屈起”,而兴不能流芳后世,亦当“遗臭万年”之叹。其所咏叹,出语惊天动地,如王世懋所评:“曲尽奸雄语态,自非常人语。”奈何时运不济,不久即一命呜呼,功业尽化云烟。前秦皇帝苻坚听说桓温废立事,即公开评论说:“温前败灞上,后败枋头,十五年间,再倾国师。六十岁公举动如此,不能思愆免退,以谢百姓,方废君以自悦,将如四海何!谚云‘怒其室而作色于父’者,其桓温之谓乎!”早已料其必败。}

\lettrine{33.14} 谢太傅\myidx{谢安}于东船行\footnote{谢太傅:谢安。东:此指谢氏家居的会稽,在京东面,故云。},小人引船\footnote{引船:摇船。},或迟或疾,或停或待,又放船从横\footnote{从横:纵横。从,通“纵”。},撞人触岸,公初不何谴\footnote{何谴:任何斥责。按:袁本作“呵谴”,“何”读为“呵”,亦通。},人谓公常无嗔喜\footnote{嗔:怒。}。曾送兄征西\myidx{谢弈}葬还\footnote{征西:谢安兄奕卒于安西将军、豫州刺史任上,卒赠镇西将军。},{\fzxk\zihao{6}\textcolor{red}{征西,谢弈(奕)。}} 日暮雨驶\footnote{雨驶:雨猛。驶,迅猛疾速貌。},小人皆醉,不可处分\footnote{处分:安排。}。公乃于车中手取车柱撞驭人\footnote{车柱:车停时作支撑的木棍。驭人:驾车人。},声色甚厉。夫以水性沈柔,入隘奔激,方之人情,固知迫隘之地\footnote{迫隘:窘迫狭隘。无得:不能。},无得保其夷粹\footnote{夷粹:平和美好。}。{\fzxk\zihao{6}\textcolor{red}{\CJKunderwave{孟子}曰:“湍水决之东则东,决之西则西。搏而跃之,可使过颡,激而行之,可使在山。岂水之性哉?人可使为不善,性亦犹是也。”}}

{\cangkai\zihao{5}【评】据\CJKunderwave{晋书·穆帝纪},奕卒于升平三年(359)八月,则故事发生于是年暮秋。故事称安“常无嗔喜”,即喜怒不形于色,其性“夷粹”。这与其生活习性有关。谢安是在兄奕卒、弟万败后的升平四年(360)出山踏入仕途的。此前的四十馀年,多在故乡会稽隐居,读书学习,谈玄说理,游山玩水,生活悠闲,故性平和。但是人的性格和心态,随环境不同而变化,平和之人,也有紧急呼叫之时。这里写出了谢安性格的另一方面,帮助读者全面地看人。故事忽以一段议论作结,行文自是跌宕可喜。但内容与\CJKunderwave{尤悔}无涉,改入\CJKunderwave{忿狷}门似更妥帖。}

\lettrine{33.15} 简文\myidx{司马昱}见田稻不识\footnote{简文:简文帝司马昱。田稻:水田之稻,水稻。},问是何草,左右答是稻。简文还,三日不出,云:“宁有赖其末而不识其本\footnote{本、末:原指根部与末梢,这里喻植株和果实。水稻之实指稻谷。}!”{\fzxk\zihao{6}\textcolor{red}{文公种菜,曾子牧羊,纵不识稻,何所多悔?此言必虚。}}

{\cangkai\zihao{5}【评】刘注“文公种菜,曾子牧马”,注欠通顺。据余嘉锡\CJKunderwave{笺疏}引\CJKunderwave{淮南子·泰族训},作“文公种米,曾子架(驾)羊”,刘向\CJKunderwave{说苑·杂言}亦作“文公种米,曾子驾羊”。种米驾羊,人知为愚。此喻当务其大者而忘其小。治国者“纵不识稻,何所多悔”?此刘氏之辩也。但这是对雄才大略的大政治家及圣贤而言。常人则不可以此借口而自解。故王世懋评驳刘注曰:“简文生长富贵,不知稼穑艰难,此愧,大是良心,而注驳之何居?”此又一胜解也。}

\lettrine{33.16} 桓车骑\myidx{桓冲}在上明政(畋)猎\footnote{桓车骑:桓冲,字幼子,温弟。曾官车骑将军,故称。时任荆州刺史。上明:地名,当时桓冲为抗前秦苻坚,迁荆州治所于长江南岸之上明城(故址在今湖北松滋市西)。政猎:袁本作“畋猎”,是。},东信至,传淮上大捷\footnote{东信:从东边京城来的信使。 淮上大捷:指太元八年(383)谢玄等率东晋军大败前秦苻坚百万之师于淝水之上。},语左右云:“群谢年少大破贼\footnote{群谢年少:指谢安之子侄辈谢玄、谢琰等年轻将领。大破贼:指淝水之战大捷。}!”因发病薨。谈者以为此死,贤于让扬之荆\footnote{让扬之荆:桓冲原为扬、豫二州刺史,于康宁三年(375),以扬州刺史让谢安而改任徐州刺史,后又调任荆州刺史。}。{\fzxk\zihao{6}\textcolor{red}{\CJKunderwave{续晋阳秋}曰:“桓冲本以将相异宜,才用不同。忖己德量不及谢安,故解扬州以让安,自谓少经军镇。及为荆州,闻苻坚自出淮、淝,深以根本为虑,遣其随身精兵三千人赴京师。时安已遣诸军,且欲外示门(闲)暇,因令冲军还。冲大惊,曰:‘谢安乃有庙堂之量,不闲将略。吾量贼必破襄阳而并力淮、淝。今大敌果至,方游谈示暇,遣诸不经事年少,而实寡弱,天下谁知?吾其左衽矣!’俄闻大勋克举,惭慨而薨。”}}

{\cangkai\zihao{5}【评】淝水大捷,并非偶然。谢安与桓冲,文武将相,一内一外,精诚团结,尽忠国家,故能事半功倍,以少胜多。如无桓冲保卫长江中上游的安全,减轻下游京师的军事压力,则不可能有“群谢年少”的淝水之捷。淝水战前,前秦左仆射权翼曾向苻坚直谏曰:“今晋道虽微,未闻丧德,君臣和睦,上下同心。谢安、桓冲,江表伟才,可谓晋有人焉……未可图也。”(见\CJKunderwave{晋书·苻坚载记})从敌人所言,知淝水大捷,桓冲也有一份功劳与贡献,何愧何恨之有?但冲为宿将,因“诸谢年少”骤然大胜,与自己判断不合,因此愧悔发病而死,一方面见其心胸之不宽广,另一方面可能担忧陈郡谢氏骤兴而桓氏家族走向衰落,心理矛盾很复杂,其痛苦难以言表。}

\lettrine{33.17} 桓公\myidx{桓玄}初报破殷荆州\myidx{殷仲堪}\footnote{桓公:特指桓玄,非玄父温。初报:刚刚接到报告。破:击败。殷荆州:殷仲堪,时任荆州刺史。},{\fzxk\zihao{6}\textcolor{red}{周祇\CJKunderwave{隆安记}曰:“仲堪以人情注于玄,疑朝廷欲以玄代己,遣道人竺僧𠎝赍宝物遗相王宠幸媒尼左右,以罪状玄。玄知其谋而击灭之。”}} 曾讲\CJKunderwave{论语},至“富与贵是人之所欲,不以其道得之不处\footnote{不处:不接受,不愿享用。}”,{\fzxk\zihao{6}\textcolor{red}{孔安国注曰:“不以其道得富贵,则仁者不处。”}} 玄意色甚恶\footnote{意色甚恶:脸色很难看。意色,神情。恶,坏。}。

{\cangkai\zihao{5}【评】故事发生于隆安三年(399)桓玄举兵袭江陵击灭殷仲堪之时。在严重的军阀混战中,野心勃勃的桓玄,因其夺取荆州而实力大增,地位腾腾直上。但在戎马倥偬之际,桓玄却仍在军中讲读\CJKunderwave{论语},以示闲暇,其学习热情似甚高涨。玄之才情文理,不减乃父,并非仅是不读书的一介武夫。但史称玄之为人,“好逞伪辞”,好以读书谈理来高自标榜,以资号召,为自己的政治资本增添筹码。玄读\CJKunderwave{论语}之类儒典,弃其仁义之心,仅作矫饰之用。在魏晋篡弑相继的军阀无义战中,意在夺人江山,与“君子无终食之间违仁”的圣人精神背道而驰。读\CJKunderwave{论语}“不以其道得之”而“意色甚恶”,正是面对历史嘲讽时其虚伪内在心理的形象刻画。}







%%% Local Variables:
%%% mode: latex
%%% TeX-engine: xetex
%%% TeX-master: "../Main"
%%% End:
