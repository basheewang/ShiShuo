%% -*- coding: utf-8 -*-
%% Time-stamp: <Chen Wang: 2025-12-09 22:38:51>

% ○ ◎ ‧ 「 」 『 』 々 ( ) “ ” ■ ^[一-龥]
% 【\([^】][^】][^】]+\)】 → {\\fzxk\\zihao{6}\\textcolor{red}{\1}}
% \(【评】.*\) → {\\cangkai\\zihao{5}\1}
% \(【题解】.*\) → {\\cangkai\\zihao{5}\1}
% 《\([^》]+\)》 → \\CJKunderwave{\1}
% ^\([0-9]+.[0-9]+\) → \\lettrine{\1}
% {\\fzxk\\zihao{6}\\textcolor{red}{[^o]*}}

\setlength{\parindent}{0pt}


\chapter{惑溺第三十五}



{\cangkai\zihao{5}【题解】 惑溺者,迷恋惑乱与沉溺不返也,意谓当时士人因情因事而迷惑心志,沉溺其中,犹如中了蛊毒一般,难以挽救。本门共七则故事,大多写魏晋士大夫沉溺女色的故事,作者采取的是否定的态度,以批判的眼光来看问题。如曹操父子为娶甄氏而屠邺,把人丁兴旺的邺城杀得个鸡犬不留。又如贾充后妻郭槐,因妒生恨,任意残杀乳母,以致亲生儿子夭亡。因贪男欢女爱而杀人解恨,实极残忍,作者批判了统治者因沉溺色欲而迷惑疯狂,态度正确。但本门另有一半的故事,却是突破了封建礼教的桎梏,真实描绘了当时士人仕女的真挚爱情故事,写来情趣盎然,很有生活气息,读后令人感动、令人赞叹。如荀粲在冰雪中裸身取冷以熨帖高烧不退的病妻,韩寿与贾午偷香窃玉的故事,都是笔触细腻,写得楚楚有致。这从另一侧面反映了魏晋士人仕女在男女婚恋观念方面,具有比较开放的意识。真挚的爱情是可贵的,不该受批评而应予颂扬。}

\lettrine{35.1} 魏甄后惠而有色\footnote{甄后:原为袁熙妻,曹操破邺,曹丕纳之,生明帝。后因故被废赐死。明帝时追尊生母为文昭皇后。惠:通“慧”,聪慧。},先为袁熙\myidx{袁熙}妻\footnote{袁熙:袁绍次子,绍任之为幽州刺史。},甚获宠。曹公\myidx{曹操}之屠邺也\footnote{曹公:曹操。 屠邺:攻破袁绍接班人袁尚据守的邺城。},令疾召甄。左右白:“五官中郎\myidx{曹丕}已将去\footnote{五官中郎:曹丕曾任五官中郎将,故称。将去:带走,取走。}。”公曰:“今年破贼,正为奴\footnote{奴:相当于“她”,指甄氏。}。”{\fzxk\zihao{6}\textcolor{red}{\CJKunderwave{魏略}曰:“建安中,袁绍为中子熙娶甄会(逸)女。绍死,熙出任幽州,甄留侍姑。及邺城破,五宫(官)将从而入绍舍,见为怖,以头伏姑䣛(膝)上。五宫(官)将谓绍妻袁(刘)夫人扶甄令举头,见其色非凡,称叹之。太祖闻其意,遂为迎娶,擅室数岁。”\CJKunderwave{世语}曰:“太祖下邺,文帝先入袁尚府,见妇人被发如垂涕,立绍妻刘后,文帝问,知是熙妻,使令揽发,以神(袖)拭面,姿貌绝伦。既过,刘谓甄曰:‘不复死矣!’遂纳之,有宠。”\CJKunderwave{魏氏春秋}曰:“五官将纳熙妾也,孔融与太祖书曰:‘武王伐纣,以妲己赐周公。’太祖以融博学,真谓\CJKunderwave{书传}所记。后见融问之,对曰:‘以今度古,想其然也。’”}}

{\cangkai\zihao{5}【评】故事发生在汉献帝建安九年(204),曹操率军大破袁尚,攻取邺城(今河北临漳西南)。汉末军阀混战,诸侯逐鹿中原。曹军屠邺,操“令疾召甄”,看来,这个好色之徒,早已做好了信息调查,一旦入城,即刻动手夺人妻女,以供自己享乐腐化。曹操这种不道德的行为,其实并非一次,而是习惯性的动作。如操攻吕布于下邳时,关羽求娶布将秦宜禄妻杜氏,操疑其有色,城破后自纳之,即秦朗之母。见\CJKunderwave{三国志·明帝纪}裴注引\CJKunderwave{魏氏春秋}。又娶何进儿媳尹氏为如夫人,即何晏之母。见\CJKunderwave{三国志·魏书·曹爽传}附何晏传。但是,有其父必有其子。曹丕比乃父有过之而无不及,父子同争甄氏,而丕捷足先登。父子好色,共争一女,贻笑万年。惜丕用情不专,宠幸几年,喜新厌旧,红颜薄命,不久甄后赐死,悲乎哀哉!}

\lettrine{35.2} 荀奉倩\myidx{荀粲}与妇至笃\footnote{荀奉倩:荀粲字奉倩。颍川颍阴(今河南许昌)人。父彧,曹操的重要谋士。其言玄远,知名于世。至笃:感情深厚。},冬月妇病热\footnote{病热:生热病,发高烧。},乃出中庭自取冷\footnote{出中庭:到庭院中。},还以自熨之\footnote{熨:贴。}。妇亡,奉倩后少时亦卒。以是获讥于世。{\fzxk\zihao{6}\textcolor{red}{\CJKunderwave{粲别传}曰:“粲常以妇人才智不足论,自宣(宜)以色为至。骠骑将军曹洪女有色,粲于是兴(聘)焉。容服帷帐甚丽,专房燕婉。历年后,妇病亡。未殡,傅嘏往喭粲,粲不哭而神伤。嘏问曰:‘妇人才色并茂为难。子之聘也,遗才存色,非难遇也。何哀之甚?’粲曰:‘佳人难再得。顾逝者不能有倾城之异,然未可易遇也。’痛悼不能已已,岁馀亦亡。亡时年二十九。粲简贵,不与常人交接,所交者一时俊杰。至葬夕,赴期者裁十馀人,悉同年相知名士也。哭之,感恸路人。粲虽褊隘,以燕婉自丧,然有识犹追惜其能言。”}} 奉倩曰:“妇人德不足称,当以色为主。”裴令\myidx{裴楷}闻之\footnote{裴令:裴楷,字叔则,河东闻喜人,曾官尚书令,故称。},曰:“此乃是兴到之事,非盛德言,冀后人未昧此语\footnote{未昧此语:不被此语所蒙蔽。}。”{\fzxk\zihao{6}\textcolor{red}{何劭论粲曰:“仲尼称‘有德者有言’,而荀粲减于是,内顾所言有馀,而识不足。”}}

{\cangkai\zihao{5}【评】在婚姻男女问题上,魏晋名士时有惊人骇俗之论,荀粲的“妇人德不足称,当以色为主”的“唯色”论,哪朝哪代有这样乖背礼法名教的言论出现?在传统礼教中,统治者提倡的当然是“唯德”论,妇德是唯一的衡量标准。但何谓“德”?古人早有“女子无才便是德”之言。男与女相对,男人是天是主,女人是地是奴,因此女人只有卑顺男人才是有德的表现。这不是吃人礼教是什么?但在传统礼教的阴影下,人们的神经已经麻木。这时,精熟玄学的荀粲突然宣扬“唯色”论以相对抗,一石激起千层浪,打破长年的沉寂。故其“获讥于世”,也是势在必然。故事生动地刻画了荀粲夫妻的真挚爱情,表达了对“情”有独钟的深刻认识。其所称“色”,作为对抗封建妇“德”的武器,对荀粲而言,实是兼指“至笃”深情基础而言的。如果不是因为感情深厚,丈夫会“出中庭”挨冻以熨帖妻子吗?荀粲是色见于外,而情动于中,他把女人之“色”作为一种生活之美来加以欣赏,这和今人所说“爱美是人的天性”意思相近。最终,荀粲真正为情而死,似乎比\CJKunderwave{红楼梦}中的贾宝玉还要痴情!}

\lettrine{35.3} 贾公闾\myidx{贾充}{\fzxk\zihao{6}\textcolor{red}{\CJKunderwave{充别传}曰:“充父逵,晚有子,故名曰充,字公闾,言后必有充闾之异。”}} 后妻郭氏\myidx{郭槐}酷妒\footnote{贾公闾:贾充字公闾,晋朝开国元勋。官至尚书令,很受武帝宠任,权倾一时。郭氏:即郭槐,又名玉璜。贾充后妻,郭配女,贾后母。封广城君,卒前改称宣城君。}。有男儿名黎民,生载周\footnote{载周:周岁。载:始。},充自外还,乳母抱儿在中庭\footnote{中庭:庭院中。},儿见充喜踊\footnote{喜踊:喜欢跃动。},充就乳母手中呜之\footnote{呜:亲吻。}。郭遥望见,谓充爱乳母,即杀之。儿悲思啼泣,不饮他乳,遂死。郭后终无子。{\fzxk\zihao{6}\textcolor{red}{\CJKunderwave{晋诸公赞}云:“郭氏,即贾后母也。为性高朗,知后无子,甚忧,爱愍怀,每劝厉之。临亡,诲贾后令尽意于太子,言甚切至。赵充华及贾谧母,并勿令出入宫中。又曰:‘此皆乱汝事。’后不能用,终至诛夷。”臣按:傅畅此言,则郭氏贤明妇人也。向令贾后抚爱愍怀,岂当纵其妒悍,自毙其子?然则物我不同,或老壮情异乎?}}

{\cangkai\zihao{5}【评】在男女关系问题上,女人受害很深。在男性中心社会里,男人可以三妻四妾,女人却必须从一而终,未婚夫死,女人也必须守望门寡,了其残生。在封建礼教的重压下,男女发生矛盾之时,女人无法合理抗争,剩下唯一维护女权的武器就是“妒”。贾充后妻郭氏之妒,本来不该受指责。但魏晋贵族仕女在享受较多生活开放和自由的同时,却又具有随意杀奴的权力而不受法律制约。郭槐“酷妒”,妒到了无法容忍其他女人接近丈夫的地步,并因此而随意杀害了乳母,这简直是骇人听闻的灭绝人性的行为。为了保护自己的专宠地位,竟然想杀人就杀人!女人如此心理变态,如果一旦掌权,也是极其可怕的。后来其女贾后发扬“妒”风,更胜娘亲,她借皇威,性极酷妒,史称“或以戟掷孕妾,子随刃坠地”,充华赵粲为之辩解曰:“贾妃年少,妒是夫人之情耳!”酷妒之大,愈烧愈热,甚至会危及国家。这样的“酷妒”,实已异化变质,而绝非维护女权的合理行动。}

\lettrine{35.4} 孙秀\myidx{孙秀}降晋\footnote{孙秀:字彦才。吴郡人。孙吴宗室。按:此与西晋的琅邪孙秀字俊忠者别是一人。},晋武帝\myidx{司马炎}厚存宠之\footnote{厚存宠之:非常关怀宠爱他。}。{\fzxk\zihao{6}\textcolor{red}{\CJKunderwave{太原郭氏录}曰:“季(秀)字彦才,吴郡吴人。为下口督,甚有威恩。孙皓(晧)惮欲除之,遣将军何定溯江而上,辞以捕鹿三千口供厨。秀豫知谋,遂来归化。世祖喜之,以为骠骑将军、交州牧。”}} 妻以姨妹蒯氏,室家甚笃\footnote{室家甚笃:夫妻感情深厚。}。妻尝妒,乃骂秀为貉子\footnote{貉子:魏晋时北人对南人的轻诋之称。}。{\fzxk\zihao{6}\textcolor{red}{\CJKunderwave{晋阳秋}曰:“蒯氏,襄阳人。祖良,吏部尚书。父钧,南阳太守。”}} 秀大不平,遂不复入。蒯氏大自悔责,请救于帝。时大赦,群臣咸见。既出,帝独留秀,从容谓曰\footnote{从容:委婉。}:“天下旷荡\footnote{旷荡:宽大。},蒯夫人可得从其例不\footnote{从其例:谓获得宽大原谅。}?”秀免冠而谢,遂为夫妇如初。

{\cangkai\zihao{5}【评】据\CJKunderwave{晋书·武帝纪},泰始六年(270)十二月,“吴夏口督,前将军孙秀率众来奔”,又泰始八年(272)六月“大赦”。则故事应当发生于泰始八年六月。是时蜀亡多年,而东吴处于将平未平之际,但晋统一中国已成大势所趋。孙秀奔晋,在此关键当口,所以会受到武帝的“存宠”。不过,孙秀出身将军,颇有个性。当妻子蒯氏因妒骂其“貉子”时,秀血气上涌,“遂不复入”。因为当时北方中原之士,多以“貉子”诬南人,其义近于亡国奴。血性男儿,怎能受此污辱?但当妻子“大自悔责”,又通过武帝缓颊,借大赦天下之际,戏称望“从其例”。在无形的皇权面前,同时更因昔日夫妻感情“甚笃”,孙秀颇通人情,于是夫妇和好如初。经过一番波折之后,大家都接受教训。夫妇相互尊重,幸福美满的家庭生活基础更加稳固。}

\lettrine{35.5} 韩寿\myidx{韩寿}美姿容\footnote{韩寿:字德真,堵阳人。妻贾午,充女。},贾充\myidx{贾充}辟以为掾\footnote{贾充:字公闾,官至尚书令。见前注。辟:征辟,召。掾:僚属,属官。}。充每聚会,贾女于青琐中看\footnote{青琐:窗格。},见寿悦之,恒怀存想\footnote{存想:思念。},发于吟咏。后婢往寿家,具述如此,并言女光丽\footnote{光丽:光鲜艳丽。}。寿闻之心动,遂请婢潜修音问\footnote{潜修音问:暗中传递消息。},及期往宿。寿蹻捷绝人,逾墙而入,家中莫知。{\fzxk\zihao{6}\textcolor{red}{\CJKunderwave{晋诸公赞}曰:“寿字德真,南阳赭(堵)阳人。曾祖暨,魏司徒,有高行。”寿敦家风,性忠厚,岂有若斯之事?诸书无闻,唯见\CJKunderwave{世说},自未可信。}} 自是充觉女盛自拂拭\footnote{拂拭:修饰,打扮。},说畼(畅)有异于常\footnote{说畼:同“悦畅”,欢喜舒畅。}。后会诸吏,闻寿有奇香之气,是外国所贡,一箸人,则历月不歇。{\fzxk\zihao{6}\textcolor{red}{\CJKunderwave{十洲记}曰:“汉武帝时,西域月氏国王遣使献香四两,大如雀卵,黑如桑椹,烧之,芳气经三月不歇。”盖此香也。}} 充计武帝唯赐己及陈骞\myidx{陈骞},馀家无此香,疑寿与女通,而垣墙重密\footnote{重密:严密。},门閤急峻\footnote{门閤:门户。},何由得尔?乃托言有盗,令人修墙。使反\footnote{使反:差人返回。},曰:“其馀无异,唯东北角如有人迹,而墙高非人所逾。”充乃取女左右婢考问,即以状对\footnote{状:实状,实情。}。充秘之,以女妻寿。{\fzxk\zihao{6}\textcolor{red}{\CJKunderwave{(郭)子}谓与韩寿通者,乃是陈骞女,即以妻寿,未婚而女亡。寿因娶贾氏,故世因传是充女。}}

{\cangkai\zihao{5}【评】这是一篇优秀的古代爱情小小说。文字不长,但故事内容,人物形象,情节波澜,细节描绘,无不生动如画,堪称上乘之作。在封建卫道者看来,此属风格轻佻,思想儇薄的作品,故入\CJKunderwave{惑溺}门。但安置在魏晋这一特定的历史文化背景中,则其意义不可轻估:它是以男欢女爱之情愫,来对抗传统礼教的重压。那久被压抑的情和欲,一旦从禁锢森严的礼教魔瓶中释放出来,就很难控制,它自由自在地涌动,生命活力骚动于中,青春气息洋溢于外,以偷香窃玉的形式,传达了以“情”抗“礼”的内容,正是突破传统礼教的一次大胆尝试。韩寿与贾午这对年轻男女,不仅是男人心动“逾墙而入”,“及期往宿”,这样幽会极其危险,身入侯门相府,一旦失风,死无葬身之地,将为此付出生命的代价;而女人主动追求,挑选丈夫,不顾父母之命、媒妁之言,以致“恒怀存想,发于吟咏”,为情所动,以至于斯,对于一个大家闺秀,确实难能可贵。男女双方,为情驱动,选择了自己的道路,过着夫妻和谐的感情生活。这可能与时代的开放,及玄学思潮追求自然率性的影响有关。就是贾充,也还算人性尚未全泯,不然的话,将是一场棒打鸳鸯两分散的悲剧。}

\lettrine{35.6} 王安丰妇常卿安丰\myidx{王戎}\footnote{王安丰:王戎字濬冲。琅邪人。竹林七贤之一。官至司徒。爵安丰侯,故称。卿:第二人称代名词,侪辈之间称“君”,年爵较尊称“公”,上对下,尊对卑,贵对贱则称“卿”,侪辈间亲昵也可称“卿”。},安丰曰:“妇人卿婿,于礼为不敬,后勿复尔。”妇曰:“亲卿爱卿,是以卿卿。我不卿卿,谁当卿卿?”遂恒听之\footnote{恒:恒常,永久。}。

{\cangkai\zihao{5}【评】在夫妻关系方面,除了“妒”这一常见的变相之爱以外,魏晋仕女还有直接表露亲昵情爱的坦荡方式,如本则故事中王戎之妻的言行举止,在别的封建时代中是很难想象的。“卿”是当时人口语,是以上对下之辞,“夫呼妻为卿则无词,妻呼夫为卿则不可”(见徐震堮\CJKunderwave{校笺})。王戎妻一旦称夫为“卿”,戎即以违背礼法禁其“后勿复尔”。王戎是竹林名士,尚且不敢违背礼法,但其妻不管这一传统礼教的紧箍咒,而是充分发挥一个年轻女人的个性和魅力,在夫妻关系中表现出力求相亲相爱的主动性。卿卿丈夫而不置,活用口语,声吻毕肖,一个活泼耍刁颇会“发嗲”的贵族少妇形象,跃然纸上。她以“卿”夫的言行,表示了不甘于男尊女卑的教条,力争夫妻平等的地位。这是暗中对传统礼教的挑战。称谓的改变,潜藏了思想观念的进步与变化。}

35. 王丞相\myidx{王导}有幸妾姓雷\footnote{王丞相:王导。幸妾:宠爱的妾。},颇预政事\footnote{颇预政事:颇多干预政事。},纳货\footnote{纳货:受贿。}。蔡公\myidx{蔡谟}谓之“雷尚书\footnote{蔡公:蔡谟字道明。济阳考城人。官至录尚书事、扬州刺史。卒赠司空。}”。{\fzxk\zihao{6}\textcolor{red}{\CJKunderwave{语林}曰:“雷有宠,生洽、恬。”}}

{\cangkai\zihao{5}【评】此则如与前\CJKunderwave{轻诋}第6则参读当更有味。大概蔡谟等名士,对于东晋初“王与马,共天下”的局面颇为不满,损害了其他上品士族的利益。为此,他以开玩笑形式,多次委婉批评了王导的言行。这次讥讽王妾雷氏为“雷尚书”,也是系列批评之一,表现了对琅邪王氏长期执政的不满。王导是东晋开国的一代名相,号称贤臣。史称“导为政务在清净,每劝帝克己励节,匡主宁邦”,对于西晋和平时期,“公卿世族,豪侈相高”的腐败政治,公开批判,予以纠正。本人生活也是“简素寡欲,仓无储谷,衣不重帛”,提倡廉洁之风。但他管得了自己,却不知约束家属及其身边的亲爱者,其宠妾雷氏公然纳贿而干预政事。雷氏何官何职?何权何势?行贿者还不是看宰相王导的权杖跳舞?这同样是王导变相的腐败行为,不仅有损其一世清名,而且会大失士心民心,在位者能不思乎?}







%%% Local Variables:
%%% mode: latex
%%% TeX-engine: xetex
%%% TeX-master: "../Main"
%%% End:
