%% -*- coding: utf-8 -*-
%% Time-stamp: <Chen Wang: 2025-12-07 11:28:10>

% ○ ◎ ‧ 「 」 『 』 々 ( ) “ ” ■ ^[一-龥]
% 【\([^】][^】][^】]+\)】 → {\\fzxk\\zihao{6}\\textcolor{red}{\1}}
% \(【评】.*\) → {\\cangkai\\zihao{5}\1}
% \(【题解】.*\) → {\\cangkai\\zihao{5}\1}
% 《\([^》]+\)》 → \\CJKunderwave{\1}
% ^\([0-9]+.[0-9]+\) → \\lettrine{\1}
% {\\fzxk\\zihao{6}\\textcolor{red}{[^o]*}}


\setlength{\parindent}{0pt}


\chapter{贤媛第十九}




{\cangkai\zihao{5}【题解】 贤媛,贤淑美善之女性。媛原为美女,但这里取的是美善贤德之义,主要是从内在精神道德及其心灵智慧方面来考虑的。“贤媛”一门,共收三十二则故事,最早的是秦末汉初的陈婴之母,及西汉元帝时的王昭君、汉成帝时班婕妤,其馀皆为三国魏晋故事。应该承认,写的主要是魏晋人心目中贤淑妇女中的佼佼者。当然,和其他时代一样,妇女所受压迫最为深重,从揭露社会罪恶方面,如赵母嫁女故事,母敕准新娘“慎勿为好”,女孩儿误认可在夫家“为恶”,母曰:“好尚不可为,其况恶乎!”正见出魏晋妇女左右难做人的窘态。但在更多的场合,\CJKunderwave{世说}着重写出了当时妇女的时代特点和新的面貌,以及为争取与男子平等命运时所作的努力和智慧。汉代以前的传统礼教,要求妇女必须具有四德,即妇德、妇言、妇容、妇功。这是套在妇女身上的沉重枷锁,其基本精神是“女子无才便是德”,做伺候男人的顺从奴仆。但魏晋妇女则做种种努力,企望挣脱或减轻传统枷锁的束缚。虽然不可能真正实现,但作为一种尝试,却值得肯定。晋葛洪\CJKunderwave{抱朴子}外篇\CJKunderwave{疾谬}云:“今俗妇女,……舍中馈之事,修周旋之好,更相从诣,之适亲戚。承星举火,不已于行。……或宿于他门,或冒夜而返。游戏佛寺,观视渔畋。登高临水,去境庆吊。开车褰帏,周章城邑。杯觞路酌,弦歌行奏。转相高尚,习非成俗。”这从反面看出,因受时代玄学思潮影响,魏晋的贵族妇女,具有相对的自由和解放,并为此做出了自己的努力与贡献。}

\lettrine{19.1} 陈婴\myidx{陈婴}者\footnote{陈婴:秦汉之际东阳人。原为项梁将,后归汉,封堂邑侯。事见\CJKunderwave{史记·项羽本纪}记述。},东阳人\footnote{东阳:县名,在淮水南,故城在楚州盱眙县东七十里。}。少修德行,箸称乡党\footnote{乡党:即乡里。}。秦末大乱,东阳人欲奉婴为主\footnote{奉:拥戴。为主:作为主宰或领袖。}。母曰:“不可。自我为汝家妇,少见贫贱,一旦富贵\footnote{一旦:一朝,一下子。},不祥。不如以兵属人\footnote{属:托给,交付。},事成,少受其利,不成,祸有所归。”{\fzxk\zihao{6}\textcolor{red}{\CJKunderwave{史记}曰:“婴故东阳令史,居县,素信,为长者。东阳人欲立长,乃请婴。婴母谏之,乃以兵属项梁,梁以婴为上柱国。”}}

{\cangkai\zihao{5}【评】陈婴作为县衙令史,虽然多少读点书,但其所受教育,则基本上是农民意识。其母之言,即是农民智慧之结晶。秦末,陈胜、吴广揭竿而起,天下动荡,群雄逐鹿中原,个个野心勃勃。比如项羽见秦始皇巡游车驾之盛而叹曰:“彼可取而代也!”代表了旧贵族后裔的雄心。(\CJKunderwave{史记·项羽本纪})刘邦在咸阳见秦始皇车驾,喟然太息曰:“嗟乎,大丈夫当如此也!”(\CJKunderwave{史记·高祖本纪})代表了流氓无产者的冒险意识。但作为思想较为保守的一般农民,则不敢作争天下、坐天下之想,因为这是冒杀头灭族的危险的。陈婴之母所言,一旦暴发富贵则不祥,说明了普通农民求稳以保平安的心理。“事成,少受其利;不成,祸有所归。”缺乏风险意识,重在求其实利,这正是一种典型的农民“狡狯”之智。}

\lettrine{19.2} 汉元帝\myidx{刘奭}宫人既多\footnote{汉元帝(前76—前33):刘奭,西汉第八代皇帝。在位十六年(前48—前33)。},乃令画工图之\footnote{图之:画像。},欲有呼者,辄披图召之\footnote{披:翻阅。}。其中常者\footnote{中常者:姿貌平常一般之女。},皆行货赂\footnote{货赂:行贿。}。王明君\myidx{王嫱}姿容甚丽\footnote{王明君:即王嫱,字昭君。因避晋文王司马昭名讳,改“昭”为“明”。其和亲事,见载于\CJKunderwave{汉书·元帝纪}及\CJKunderwave{匈奴传}下,另见载于\CJKunderwave{后汉书·南匈奴传}。王昭君入匈奴后,被呼韩邪单于封为宁胡阏氏(yān zhī焉支)。卒葬于匈奴,墓称“青冢”,在今内蒙古呼和浩特市南十公里处。},志不苟求,工遂毁为其状\footnote{工遂毁为其状:据徐震堮\CJKunderwave{校笺}引李评曰:“‘志不苟求’二句,\CJKunderwave{御览}作‘志不可苟求,工遂毁为甚丑’,当从\CJKunderwave{御览},否则今本必去‘为’字,方令人解。”可参考。}。后匈奴来和\footnote{匈奴:古代中国北方的少数民族,中原人蔑称之为“胡”,以游牧为生,秦汉时强横塞外,边事频仍。},求美女于汉帝,帝以明君充行\footnote{充行:冒充公主身份出嫁匈奴。}。既召见而惜之,但名字已去\footnote{名字已去:名字已报送匈奴。},不欲中改,于是遂行。{\fzxk\zihao{6}\textcolor{red}{\CJKunderwave{汉书·匈奴传}曰:“竟宁元年,呼韩邪单于来朝,自言愿婿汉氏以自亲。元帝以后宫良家子王嫱字明君赐之。单于欢喜,上书愿保塞。”文颖曰:“昭君,本蜀郡秭归人也。”\CJKunderwave{琴操}曰:“王昭君者,齐国王穰女也。年十七,仪形绝丽,以节闻国中。长者求之者,王皆不许,乃献汉元帝。帝造次不能别房帷,昭君恚怒久。会单于遣使,帝令宫人装出,使者请一女,帝乃谓宫中曰:‘欲至单于者起。’昭君喟然越席而起,帝视之,大惊悔。是时使者并见,不得止,乃赐单于。单于大悦,献诸珍物。昭君有子曰世违。单于死,世违继立。凡为胡者,父死,妻母。昭君问世违曰:‘汝为汉也,为胡也?’世违曰:‘欲为胡耳。’昭君乃吞药自杀。”石季伦曰:“昭以触文帝讳,故改为明。”}}

{\cangkai\zihao{5}【评】昭君和亲的故事,在中国影响很大,传为千古佳话。\CJKunderwave{世说}所载故事,具体细节是否皆为真实,尚可商榷。但就其大概来说,见于史书记载,确有其事。故事发生在汉元帝竟宁元年(前33),汉宫之中,一个男性皇帝,霸占了万千美丽女性,以致宠幸之前,无法见面,而必须“披图召之”。万千宫女,被皇帝召幸的机会,只能万分之一,可说是微乎其微。对皇帝来说,是独霸与垄断,是对女人的占有和奴役;而对万千宫女来说,则是严重的阴阳失调,是一种根本违背人性的摧残,也是变相的永远的守活寡。故事首先针对帝王及朝廷制度,揭露了社会的罪恶,并为妇女所受深重灾难鸣不平。历代专门吟咏昭君故事的诗词共有七百馀篇,但受传统观念影响,往往感叹王昭君被逼北上匈奴和亲时的痛苦而自叹“红颜薄命”,抒写昭君的无限悲怨哀愁,即如宋代王安石\CJKunderwave{明妃曲},也有“明妃初出汉宫时,泪湿春风鬓脚垂”之句,一副悲啼哭泣的可怜相。这实是一种误解。据史考证,王昭君因不满汉宫凄冷孤寂生活,主动请缨,远赴匈奴和亲。自此以后,汉与匈奴,边境相安数十百年。铁骑百万,安然不动,民族和睦,友好安定。王昭君不仅是为个人争幸福,做个真正的女人;而且客观上为中国这个多民族的大家庭,做出了自己的历史贡献。清朝女诗人郭润玉,力破传统偏见,重写\CJKunderwave{明妃曲},云:“漫道黄金误此身,朔风吹散马头尘。琵琶一曲干戈靖,论到边功是美人。”诗人以其女性的敏感和细腻笔触,认为巾帼不让须眉,高度评价了昭君出塞的历史功绩,歌颂了昭君在民族团结与融合的洪流中所起的卓越作用。}

\lettrine{19.3} 汉成帝\myidx{刘骜}幸赵飞燕\myidx{赵飞燕}\footnote{汉成帝:刘骜(前51—前7),西汉第九代皇帝。公元前32年至前7年在位二十六年。在位时,外戚王氏集团擅政,故国势日衰。幸:宠幸,喜爱。赵飞燕(?—前1):曾谮废许皇后,与妹昭仪专宠后宫十馀年。成帝暴死后尊为太后,但于平帝时被废为庶人而自杀。},飞燕谗班婕妤\myidx{班婕妤}祝诅\footnote{谗:诬陷,言语中伤。祝诅(zhòu zǔ咒祖):祈求鬼神降祸于仇人。班婕妤(jié yú捷予):名未详,西汉雁门楼烦班况女,班彪姑母。原为成帝宠姬,后失宠,曾作诗赋自伤。婕妤,宫中女官名,皇帝妃嫔的一种称号。},于是考问\footnote{考问:审问。},辞曰\footnote{辞:此特指审讯供辞。}:“妾闻死生有命,富贵在天\footnote{“死生有命”二句:语出\CJKunderwave{论语·颜渊}子夏之口。}。修善尚不蒙福,为邪欲以何望?若鬼神有知,不受邪佞之诉\footnote{邪佞(nìnɡ宁):巧言善辩的邪媚之辞。};若其无知,诉之何益?故不为也。”{\fzxk\zihao{6}\textcolor{red}{\CJKunderwave{汉书·外戚传}曰:“成帝赵皇后,本长安宫人。初生,父母不举,三日不死,乃收养之。及壮,属河阳主家,学歌舞,号曰飞燕。帝微行过主,见而悦之,召入宫,大得幸,立为后。班婕妤者,雁门人。成帝初选入宫,大得幸,立为婕妤。帝游后庭,尝欲与同辇,婕妤辞之。赵飞燕谮许皇后及婕妤,婕妤对有辞致,上怜之,赐黄金百斤。飞燕娇妬,婕妤恐见危,中求供养太后于长信宫。帝崩,婕妤充奉园陵。薨,葬园中。”}}

{\cangkai\zihao{5}【评】孔子曾说:“唯女子与小人为难养也。”(\CJKunderwave{论语·阳货})其轻视之心,经后儒变本加厉,发为“女子无才便是德”之言。但\CJKunderwave{世说}作者,对于这一传统偏见,似乎有所修正。他所收集编撰的“贤媛”故事,大多集中在女人的智慧闪光——也即是“才”的方面。这正是魏晋以来思想较为解放年代的一种突破。班婕妤是个贵族妇女,但仍难逃悲惨的命运。最后是“充奉园陵”——为死去的帝王空守坟墓而终其一生。但是,如果她在受谗之后,没有一篇自明心迹的绝妙辩护词,则连一天都活不下来,很可能立即死无葬身之地。一篇好文章,从理论根基来看,是牢不可破;从逻辑层次来看,是无懈可击的;从感情角度来看,是真情感人,令人不得不信服的。引用\CJKunderwave{论语}中的大道理,既合乎汉儒的要求;同时更说明了女人的文化修养及其智慧。其退处东宫时所作自伤悼赋,今见载于\CJKunderwave{汉书}卷九七\CJKunderwave{外戚传}下。与昏庸好色的汉成帝相比,女人的生命价值和历史内涵,岂非更有意义?后来雁门楼烦班氏家族,又出了一个班昭,帮助其兄班固续完\CJKunderwave{汉书}大业,作为一个史学世家,除了男人以外,女人也做出了重要贡献,岂是偶然!}

\lettrine{19.4} 魏武帝\myidx{曹操}崩\footnote{魏武帝:曹操卒谥武王,曹丕篡汉开魏之后,史尊为魏武帝。},文帝\myidx{曹丕}悉取武帝宫人自侍\footnote{自侍:服侍自己。}。及帝病困\footnote{病困:病危。},卞后\myidx{卞后}出看疾\footnote{卞后(160—240):原为倡家女,操纳为妾,后扶正为继室,即魏文帝曹丕之母,故又称太后。看疾:探视病人。}。太后入户,见直侍并是昔日所爱幸者\footnote{直侍:当值侍候的宫人。直,通“值”。}。太后问:“何时来邪?”云:“正伏魄时过\footnote{伏魄:招魂。古人迷信,以为人刚死时,持其衣物登高北面呼叫,令其魂魄归来。伏,通“复”;“伏魄”即“复魄”。}。”因不复前而叹曰:“狗鼠不食汝馀\footnote{狗鼠不食汝馀:古代俗语,喻其轻贱,连狗鼠畜生都予以唾弃。},死故应尔\footnote{死故应尔:确实该死。故,的确。}。”至山陵\footnote{山陵:山陵原指帝王陵墓,这里名词动化,指帝王葬礼。},亦竟不临\footnote{临:临穴哭吊。}。{\fzxk\zihao{6}\textcolor{red}{\CJKunderwave{魏书}曰:“武宣卞皇后,琅邪开阳人。以汉延熹三年生齐郡白亭,有黄气满室移日。父敬侯怪之,以问卜者王越。越曰:‘此吉祥也。’年二十,太祖纳于谯。性约俭,不尚华丽,有母仪德行。”}}

{\cangkai\zihao{5}【评】故事揭露了帝王生活的荒淫腐朽及封建礼教的虚伪性。曹操是个法家,原不太相信汉儒的一套礼教。但他又是个实用主义的政治家,为了对付政敌,他有时又偏祭起了忠孝节义的礼教法宝,坚决把“不孝”的孔融杀掉。可是到了他的儿子曹丕手里,又把“孝”字抛到爪洼国去了。父亲刚死,还在招魂的时候,他已急不可耐地霸占了昔日父亲爱幸过的宫女侍妾。以传统的眼光视之,这是为人不齿的畜生行为,败坏人伦纲纪的无耻之尤。可他偏偏是个至高无上的皇帝,忠孝节义之类的礼教,是他随心打扮的女孩子,招之即来,挥之即去,充分显示了帝王的虚伪本性。“狗鼠不食汝馀,死故应尔”,这话出自一个亲生母亲之口,说得沉痛之极,促人三思。}

\lettrine{19.5} 赵母嫁女\footnote{赵母(?—243):三国时吴人,夫虞韪没,孙权诏入宫,称赵姬。著书作赋,具有较深的文化素养。},女临去,敕之曰\footnote{敕:告诫。}:“慎勿为好\footnote{慎勿为好:意谓做善事有名声,易招人嫉妒而受害。}!”女曰:“不为好,可为恶邪?”母曰:“好尚不可为,其况恶乎\footnote{其况:岂况,何况。}!”{\fzxk\zihao{6}\textcolor{red}{\CJKunderwave{列女传}曰:“赵姬者,桐乡令东郡虞韪妻,颍川赵氏女也。才敏多览。韪既没,文皇帝敬其文才,诏入宫省。上欲自征公孙渊,姬上疏以谏。作\CJKunderwave{列女传解},号赵母注。赋数十万言。赤乌六年卒。”\CJKunderwave{淮南子}曰:“人有嫁其女而教之者,曰:‘尔为善,善人疾之。’对曰:‘然则当为不善乎?’曰:‘善尚不可为,而况不善乎?’”景献羊皇后曰:“此言虽鄙,可以命世人。”}}

{\cangkai\zihao{5}【评】做人难,做女人更难。人活世上很累,必须前瞻后顾侧目观察,调整好各种复杂的人事关系。故\CJKunderwave{红楼梦}中有对联云:“世事洞明皆学问,人情练达即文章。”这岂是一朝一夕之功?必须在现实生活中跌打滚爬,方可明白一些。但等到人们有所觉悟,其大半生已经匆匆过去了。赵母嫁女时的肺腑之言,是其一生做女人的经验总结,也可说是其智慧结晶。明王世懋谓其智“何必减\CJKunderwave{庄子}”,甚是。因此,为了生存和发展,争取做人的权利,女人一路走去,战战兢兢,如临深渊,如履薄冰,稍有不慎,就有可能跌入万丈深渊而万劫不复。做女人必须比男人多付出几倍的心血和代价。做个新娘,原该是喜气洋洋笑容满脸才好,但因环境改变,为善易招人嫉,为恶则遭报复,集矢加身,无所措其手足,奈何奈何!做女人只能无善无恶,浑浑噩噩,无才是德,随便男人吆喝摆布,才有资格做个精神“残疾”之人。其实,不仅女人如此,扩而大之,昔日凡是不能独立之人,不分男女,无不如此,悲乎!}

\lettrine{19.6} 许允\myidx{许允}妇是阮卫尉\myidx{阮共}女\footnote{许允(?—254):参\CJKunderwave{赏誉}139注。允与中书令李丰、太常夏侯玄友善,为专擅朝政的司马师所害。阮卫尉:即阮共,出自陈留阮氏家族。},德如\myidx{阮侃}妹\footnote{德如:即阮侃,共之少子。}。{\fzxk\zihao{6}\textcolor{red}{\CJKunderwave{魏略}曰:“允字士宗,高阳人。少与清河崔赞,俱发名于冀州。仕至领军将军。”\CJKunderwave{陈留志名}曰:“阮共字伯彦,尉氏人。清真守道,动以礼让。仕魏至卫尉卿。少子侃,字德如,有俊才,而饬以名理,风仪雅润。与嵇康为友。仕至河内太守。”}} 奇丑\footnote{奇:非常。}。交礼竟\footnote{交礼:结婚时的交拜礼仪。竟:终,结束。},允无复入理,家人深以为忧。会允有客至\footnote{会:恰巧。},妇令婢视之,还,答曰:“是桓郎\myidx{桓范}\footnote{桓郎:指桓范,字元则,有文才,曾编\CJKunderwave{皇览},又杂抄\CJKunderwave{汉事}诸事作\CJKunderwave{世要论}以讽世。郎:古代对青年男子的美称。}。”桓郎者,相(桓)范也\footnote{相范:诸本“相”作“桓”,是。桓范,魏末官至大司农,后因心存魏室为司马懿诛杀。}。{\fzxk\zihao{6}\textcolor{red}{\CJKunderwave{魏略}曰:“范字允明(元则),沛郡人。仕至大司农。为宣王所诛。”}} 妇云:“无忧\footnote{无:通“毋”,不必,无须。},桓必劝入。”桓果语许云:“阮家既嫁丑女与卿,故当有意\footnote{故当:肯定,当然。},卿宜察之。”许使回入内,既见妇,即欲出。妇料其此出无复入理,便捉裾停之\footnote{捉裾:抓住衣襟。}。许因谓曰:“妇有四德,卿有其几\footnote{四德:封建礼教对于妇女的四项要求,即品德、言语、容仪、女功。几:多少,几项。}?”{\fzxk\zihao{6}\textcolor{red}{\CJKunderwave{周礼}:“九嫔掌妇学之法,以教九御:妇德、妇言、妇容、妇功。”郑注曰:“德谓贞顺,言谓辞令,容谓婉娩,功谓丝泉(枲)。”}} 妇曰:“新妇所乏唯容尔。然士有百行,君有几?”许云:“皆备。”妇曰:“夫百行以德为首\footnote{百行:各种品行。“百”泛称其多。},君好色不好德,何谓皆备?”允有惭色,遂相敬重。

{\cangkai\zihao{5}【评】故事的人物形象生动,心理刻画细腻入微。“捉裾停之”,是一典型细节,不仅新郎不忘,就是读者也印象深刻。因为新妇一旦丧失机会,日后纵有千般本事万般智慧,也难以挽回,其所“捉裾”,不仅是情急之行,更是一种准确判断、当机立断的智慧结晶。男人责女人以德,其所谓“德”,实是无才便是“德”,如许允新妇之言之行,则不合乎传统古训。但她以深思熟虑的问答,引诱丈夫入其彀中,而不能不推服认输,其逻辑推理之力,已超过以文才智慧闻世的新郎,所以令人叹服。其夫妻关系之和谐幸福,是在矛盾斗争中形成,是女人依靠自己的智慧主动争取而获得的。}

\lettrine{19.7} 许允\myidx{许允}为吏部郎\footnote{吏部郎:魏晋时中央朝廷吏部的副长官,主持官吏的选拔黜降工作。},多用其乡里\footnote{乡里:同乡之人。},魏明帝\myidx{曹叡}遣虎贲收之\footnote{虎贲(bēn奔):以猛虎喻武士。贲,通“奔”。}。其妇出诫允曰:“明主可以理夺\footnote{理夺:用道理来争取说服。},难以情求。”既至,帝覈问之\footnote{覈问:审问核实。},允对曰:“‘举尔所知\footnote{举尔所知:孔子之言,出自\CJKunderwave{论语·子路}。}’,臣之乡人,臣所知也。陛下检校为称职与不\footnote{检校:检查核实。},若不称职,臣受其罪。”既检校,皆官得其人,于是乃释。允衣服败坏,诏赐新衣。初允被收,举家号哭\footnote{举家:全家。}。阮新妇自若,云:“勿忧,寻还\footnote{寻:不久。}。”作粟粥待,顷之允至\footnote{顷:一会儿。}。{\fzxk\zihao{6}\textcolor{red}{\CJKunderwave{魏氏春秋}曰:“初,允为吏部,选迁郡守。明帝疑其所用非次,将加其罪。允妻阮氏洗(跣)出谓曰:‘明主可以理夺,不可以情求。’允颔之而入。帝怒诘之,允对曰:‘某郡太守虽限满,文书先至,年限在后,(某守虽后),日限在前。’帝前取事视之,乃释然。遣出,望其衣败,曰:‘清吏也。’”}}

{\cangkai\zihao{5}【评】这则故事,既揭露了帝王的刻薄寡恩,威福独擅;同时又以此反衬了许允之妻推理判断之准确,以明其智慧。在历史上,魏明帝曹睿是儒法并行不悖,史称其“沉毅断识,任心而行”,\CJKunderwave{三国志·明帝纪}裴注更引\CJKunderwave{魏晋春秋}云:“时明帝喜发举,数有以轻微而致大辟者”,实行特务统治,因而人人自危。许允新妇虽在深闺,但却关心国家大事,对于帝王的嗜好及其心理,了若指掌。其研究之精细,似乎超越作为朝官的丈夫。丈夫被捕时相当突然,新妇甚至来不及穿鞋袜而“跣出”告诫丈夫:“明主可以理夺,难以情求。”正是针对明帝防范臣下的苛察之心而发,可说是一语中的,救了丈夫的性命,并保住其前途。于此可见,巾帼之智,何让须眉。}

\lettrine{19.8} 许允\myidx{许允}为晋景王\myidx{司马师}所诛\footnote{晋景王:即司马师,懿子。魏末任大将军,与弟昭共擅朝政,参\CJKunderwave{言语}16则注。},门生走入告其妇\footnote{门生:投靠世家大族的门客,地位高于仆众,也可以入仕。}。妇正在机中\footnote{机:织布机。},神色不变,曰:“蚤知尔耳\footnote{蚤知尔耳:早知如此。“蚤”通“早”。}。”{\fzxk\zihao{6}\textcolor{red}{\CJKunderwave{魏志}曰:“初,领军与夏侯玄、李丰亲善。有诈作尺一诏书,以玄为大将军,允为太尉,共录尚书事。无何,有人天未明乘马以诏版付允门吏。曰:‘有诏。’因便驱走,允投书烧之,不以关呈景王。”\CJKunderwave{魏略}曰:“明年,李丰被收,允欲往见大将军。已出门,允回遑不走(定),中道还取绔。大将军闻而怪之,曰:‘我自收李丰,士大木(夫)何为匆匆乎?’会镇北将军刘静卒,以允代静。大将军与允书曰:‘镇北虽少事,而都典一方。念足下震华鼓,建朱节,历本州,此所谓箸绣昼行也。’会有司奏允前擅以厨钱谷乞诸俳及其官属,减死徙边,道死。”\CJKunderwave{魏氏春秋}曰:“允之为镇此(北),喜谓其妻曰:‘吾知免矣。’妻曰:‘祸见于此,何免之有?’”\CJKunderwave{晋诸公赞}曰:“允有王(正)情,与文帝不平,遂幽杀之。”\CJKunderwave{妇人集}载阮氏与允书,陈允祸患所起,辞甚酸怆。文多不录。}} 门人欲藏其儿,妇曰:“无豫诸儿事\footnote{豫:关涉。}。”后徙居墓所。景王遣锺会\myidx{锺会}看之\footnote{锺会:字士季,官至魏司徒,是司马集团中要人。伐蜀功成,谋反被杀。参\CJKunderwave{言语}11则注。},若才流及父\footnote{才流:才智流品。},当收\footnote{收:拘捕。}。儿以咨母\footnote{咨:咨询,请教。},母曰:“汝等虽佳,才具不多\footnote{才具:才干。},率胸怀与语\footnote{率胸怀与语:坦开胸怀与之交谈。},便无所忧。不须极哀,会止便止。又可少问朝事\footnote{少:稍。又可少问朝事:余嘉锡\CJKunderwave{笺疏}曰:“阳为愚不晓事,不知会之侦己,无所疑惧也。”}。”儿从之。会反\footnote{反:通“返”。},以状对,卒免。{\fzxk\zihao{6}\textcolor{red}{\CJKunderwave{世语}曰:“允二子,奇字子太,猛字子豹,并有治理。”\CJKunderwave{晋诸公赞}曰:“奇,泰始中为太常丞。世祖尝祠庙,奇应行事。朝廷以奇受害之门,不令接近,出为长史。世祖下诏,述允宿望,又称奇才,擢为尚书祠部郎。猛礼学儒博,加有才识,为幽州刺史。”}}

{\cangkai\zihao{5}【评】本门第6、7、8三则故事,犹如现存的电视连续剧,形象地描绘了许允之妻从初作新妇,到丈夫被杀,教育两个年轻儿子许奇、许猛,机智地避免了重蹈父亲覆辙,数十年间的智慧闪光轨迹。史称许允被贬为镇北将军时,曾喜谓其妻,有“吾知免矣”之言,以为已经逃过劫难。但妻子却明确指出:“祸见于此,何免之有?”让丈夫在路上做好必要的防范准备。但许允之智,终在妻子之下,以此遇害。这是侥幸心理害死了他。因此,门客报允死讯时,其妻神色不变而有“早知尔耳”之言,其料事如神犹如女诸葛一般。据刘注引\CJKunderwave{妇人集},载其与允书,“陈祸患之所起,辞甚酸怆”,但因文多不录,惜哉!另,以智慧增女人之光彩,又可见魏晋人不同于传统观念的思想认识。}

\lettrine{19.9} 王公渊\myidx{王广}娶诸葛诞\myidx{诸葛诞}女\footnote{王公渊:王广(?—251年)。字公渊,三国魏王凌子。当时名士。与父陵因拥曹魏而反对司马氏集团被杀。诸葛诞:字公休,原为魏扬州刺史、镇东将军、司空。后据寿春反魏,兵败被杀。参前\CJKunderwave{品藻}第4则注。}。入室,言语始交,王谓妇曰:“新妇神色卑下,殊不似公休\footnote{殊:颇、甚。公休:指诸葛诞。}。”妇曰:“大丈夫不能仿佛彦云\myidx{王凌}\footnote{仿佛彦云:效仿王凌。彦云,王陵字。陵在魏曾官司空、太尉、征东将军,封南乡侯,后被司马懿所杀。按:据王利器校,王陵当作“王凌”,故字彦云。},而令妇人比踪英杰\footnote{比踪英杰:与英雄豪杰比肩看齐。}。”{\fzxk\zihao{6}\textcolor{red}{\CJKunderwave{魏氏春秋}曰:“王广字公渊,王陵子也。有风量才学,名重当世。与傅嘏等论才性同异,行于世。”\CJKunderwave{魏志}曰:“广有志尚学行,陵诛,并死。”臣谓王广名士,岂以妻父为戏,此言非也。}}

{\cangkai\zihao{5}【评】入木三分的心理刻画,细致生动。王广出身于太原王氏家族,名门之后,本身又是名士,曾与傅嘏、锺会、李丰等论人之才性异同而著称于世,成为魏晋玄学的重要命题之一,其才华智慧非同凡响。有此资本,因而盛气凌人,在新婚的第一天,就想给出身于琅邪诸葛家族的妻子摆谱,其得意忘形的自高自大,竟然不顾礼仪,直呼老丈人的字讳,来嘲讽新婚妻子。但是,魔高一尺,道高一丈,新娘子面对忘乎所以的新郎,毫不退缩,而是以其人之道还治其人之身,针锋相对地借公公名讳来嘲讽丈夫。这不是为了决裂,而是以斗争求团结的有效手段,在男人面前争取做个女人的平等权利。新婚之夜,新郎新娘斗智,二人的内在心理,昭然若画。新妇的急智,令自负的新郎不得不低下那高傲的头颅。}

\lettrine{19.10} 王经\myidx{王经}少贫苦\footnote{王经(?—260):字彦纬(一作伟)。三国时魏之清河(今属河北)人。与高阳许允并称冀州名士,官至尚书,高贵乡公之难,为司马昭所杀。},仕至二千石\footnote{二千石:郡守郎将的俸禄等级。}。母语之曰:“汝本寒家子,仕至二千石,此可以止乎!”经不能用。为尚书\footnote{尚书:官名,指朝廷尚书省各部长官。},助魏,不忠于晋,被收\footnote{被收:被逮捕。}。涕泣辞母,曰:“不从母敕\footnote{敕:告诫。},以至今日。”母都无慽容,语之曰:“为子则孝,为臣则忠。有孝有忠,何负吾邪?”{\fzxk\zihao{6}\textcolor{red}{\CJKunderwave{世语}曰:“经字彦伟,清河人。高贵乡公之难,王沈、王业驰告文王。经以正直不出,因沉(沈)业申意。后诛经及其母。”\CJKunderwave{晋诸公赞}曰:“沈、业将出,呼经,不从,曰:‘吾子行矣!’”\CJKunderwave{汉晋春秋}曰:“初,曹髦将自讨司马昭,经谏曰:‘昔鲁昭不忍季氏,败走失国,为天下笑。今权在其门久矣,朝廷四方,皆为之致死,不顾逆顺之理,非一日也。且宿卫空阙,寸刃无有,陛下何所资用?而一旦如此,无乃欲除疾而更深之邪!’髦不听。后杀经,并及其母。将死,垂泣谢母,母颜色不变,哭而谓曰:‘人谁不死?往所以止汝者,恐不得其所也。以此并命,何恨之有!’”干宝\CJKunderwave{晋纪}曰:“经正直,不忠于我,故诛之。”桉:傅畅、干宝所记,则是经实忠贞于魏,而\CJKunderwave{世语}既谓其正直,复云因沈、业申意,何其相反乎?故二家之言深得之。}}

{\cangkai\zihao{5}【评】英雄豪杰的诞生,离不开伟大母亲的培育。忠孝节义,封建士人视为荦荦大节。不忠不孝,谓为士耻,亦称国耻。和王沈、王业卖主求荣之流相比,王经忠孝兼备、慷慨赴死以救国难,其大仁大义,青史流芳、千古不朽,成为士人之楷模。王经之死节,肇自其母平素之谆谆教导。王经满门为司马昭所杀,临刑前,王经为牵累无辜老母而泣,母曰:“为子则孝,为臣有忠。有孝有忠,何负吾邪?”死前面不改色,其言掷地有声,其大智大勇,义薄云天。明王世懋评曰:“读史至王章妻、王经母,未尝不流涕也。”母对儿言,感天动地。“天下兴亡,匹夫有责”,昔日只论男子汉大丈夫。其实,王经之母的临刑誓词,充分说明了巾帼何让须眉!}

\lettrine{19.11} 山公\myidx{山涛}与嵇、阮一面\footnote{山公:对山涛(210—283)的敬称。参前\CJKunderwave{言语}第78则及\CJKunderwave{政事}第5则注。一面:即一见如故,或译注为只见一面,误。},契若金兰\footnote{契若金兰:喻朋友相知,情投意合。语本\CJKunderwave{易·系辞上}:“二人同心,其利断金;同心之言,其臭如兰。”}。山妻韩氏,觉公与二人异于常交,问公,公曰:“我当年可以为友者,唯此二生耳。”妻曰:“负羁之妻,亦亲观狐、赵\footnote{“负羁之妻”二句:据\CJKunderwave{左传·僖公二十三年}载,晋公子重耳(按:后来的晋文公)流亡路经曹国时,为曹共公所辱。曹国之臣僖负羁之妻劝夫以礼相待,曰:“吾观晋公子之从者,皆足以相国。”其“从者”即指狐偃、赵衰(cuī崔)等人。狐偃、赵衰后来助重耳返晋,是为晋文公,成为春秋五霸之一。};意欲窥之\footnote{窥:暗中偷视。},可乎?”他日,二人来,妻劝公止之宿,具酒肉。夜穿墉(牗)以视之\footnote{墉:\CJKunderwave{御览}卷四○九称引作“牗”,于义较胜。墉,北面之墙。牗,窗户。},达且(旦)忘反\footnote{达且:诸本“且”作“旦”,是。}。公入曰:“二人何如?”妻曰:“君才殊不如,正当以识度相友耳\footnote{识度:见识、器度。}。”公曰:“伊辈亦常以我度为胜\footnote{伊辈:他们。度:气量、器度。}。”{\fzxk\zihao{6}\textcolor{red}{\CJKunderwave{晋阳秋}曰:“涛雅量恢达,度量弘远,心存事外,而与时俯仰。尝与阮籍、嵇康诸人,箸忘言之契。至于群子屯蹇于世,涛独保浩然之度。”王隐\CJKunderwave{晋书}曰:“韩氏有才识,涛未仕时,戏之曰:‘忍寒,我当作三公,不知卿堪为夫人不耳!’”}}

{\cangkai\zihao{5}【评】成功男人的背后,常有女人的支持。山涛事业的成就,其中或许有一半是妻子的功劳。朱铸禹\CJKunderwave{汇校集注}曾评曰:“\CJKunderwave{山公启事}想俱由山婆鉴定耶?”所论虽借笑话而发,但却多少道出了其中之奥妙。一个人才,当观其才、胆、识、力四个方面,四者之中,又当以“识”为主,识是判断、认识,是世界观的核心。“使无识,则三者俱无所托”(叶燮\CJKunderwave{原诗})。有胆无识则卤妄无知;有力无识,则坚僻妄为;有才无识,则虽才华横溢,议论纵横,而是非混淆,黑白颠倒,反而为才所累。山妻在赞颂嵇康、阮籍才情之时,又肯定了丈夫的“识度”,一方面表达了隐藏内心的对丈夫的挚情真爱,一方面又为丈夫的为人做事指明了前进的方向。}

\lettrine{19.12} 王浑\myidx{王浑}妻锺氏生女令淑\footnote{王浑(223—297):字玄冲,魏晋间太原晋阳(今山西太原)人。昶子。以平吴之功,晋爵为公,官征东大将军,迁司徒。令淑:美丽贤淑。}。{\fzxk\zihao{6}\textcolor{red}{虞预\CJKunderwave{晋书}曰:“浑字玄冲,太原晋阳人。魏司徒昶子,仕至司徒。”}} 武子\myidx{王济}为妹求简美对而未得\footnote{武子:即王济,字武子。浑中子。参前\CJKunderwave{言语}第24则注。求简美对:寻找选择美好正配。简,检。}。有兵家子\footnote{兵家子:军人之子。},有隽才\footnote{隽才:优秀才能。},欲以妹妻之,乃白母。{\fzxk\zihao{6}\textcolor{red}{\CJKunderwave{王氏谱}曰:“锺夫人名琰之,太傅繇之〔曾〕孙。”}} 曰:“诚是才者,其地可遗\footnote{其地可遗:其出身门第可以不论。},然要令我见。”武子乃令兵儿与群小杂处\footnote{群小:众多“小人”,特指身份较低的普通百姓。},使母帷中察之\footnote{帷:帐幕。}。既而,母谓武子曰:“如此衣、形者,是汝所拟者,非邪\footnote{“如此衣、形者”三句:穿着这样衣服这样形状的,是你所选中的人,是不是?}?”武子曰:“是也。”母曰:“此才足以拔萃\footnote{拔萃:出类拔萃。},然地寒\footnote{地寒:出身于寒微之家。},不有长年\footnote{长年:长寿。},不得申其才用\footnote{申:施展。}。观其形骨必不寿\footnote{形骨:形貌骨相。不寿:短命,夭折。},不可与婚。”武子从之。兵儿数年果亡。

{\cangkai\zihao{5}【评】锺琰,晋初人,王浑之妻、王济之母。她是出身于颍川长社锺氏家族的名门闺秀。在\CJKunderwave{世说}中,锺琰有三则故事,其馀二则是本门第16则,\CJKunderwave{排调}第8则。从这些故事看来,锺夫人知书达礼,很有修养,\CJKunderwave{隋书·经籍志}录有\CJKunderwave{锺夫人集}一卷可证,但却不拘泥于教条,而颇有自己的个性和主张。其夫王浑出将入相。其子王济,晋初名士而文武兼备。济想为妹觅佳婿,找到一个“有隽才”的兵家子——在魏晋时代,兵家子出身低贱,令人轻视,即使东晋贵如桓温,独擅朝政,世家望族照样瞧不起他的兵家子门第而不愿与桓家结亲。当时太原王氏,高门士族,与普通的兵家子,门第悬隔,天渊之别。但锺夫人却明白宣示:“诚是才者,其地可遗。”打破门当户对的传统婚姻偏见,在实行九品中正制的魏晋时代,其思想认识高人一筹。但可贵的是,她比儿子更聪明更有智慧,因而对儿女婚姻的选择也更全面,看中才华,但不唯才是论,还要看健康。不寿短命之人,岂可与之结婚?从而避免了女儿的婚姻悲剧命运。锺夫人的智慧之光,来自对实际生活的考察。}

\lettrine{19.13} 贾充\myidx{贾充}前妇是李丰\myidx{李丰}女\myidx{李婉}\footnote{贾充(217—282):字公闾,平阳襄陵(今山西襄汾东北)人。曹魏时任大将军司马、廷尉,是司马氏集团的核心骨干。入晋任司空、侍中、尚书令,备受宠信。参前\CJKunderwave{政事}第3则注。前妇:前妻。按,即李婉,字淑文,李丰女。李丰(?—254):参前\CJKunderwave{容止}第4则注。}。丰被诛,离婚徙边\footnote{徙边:犯罪被流放边疆服劳役。}。{\fzxk\zihao{6}\textcolor{red}{\CJKunderwave{妇人集}曰:“充妻李氏,名婉,字淑文。丰诛,徙乐浪。”}} 后遇赦得还,充先已取郭配\myidx{郭配}女\footnote{郭配:郭淮之弟,官城阳太守。其女名玉璜,一名槐,嫁贾充为继室。李详谓郭氏一名扶,误。}。{\fzxk\zihao{6}\textcolor{red}{\CJKunderwave{贾氏谱}曰:“郭氏名王(玉)璜,即广宣君也。”}} 武帝\myidx{司马炎}特听置左右夫人\footnote{特听:特地批准。左右夫人:即第一、第二夫人,皆为正室。}。李氏别住外,不肯还充舍。{\fzxk\zihao{6}\textcolor{red}{\CJKunderwave{晋诸公赞}曰:“世祖践阼,李氏赦还。而齐献王妃欲令充遣郭氏,更纳其母。充不许,为李氏筑宅而不往来。充母柳氏将亡,充问所欲言者,柳曰:‘我教汝迎李新妇尚不肯,安问他事!’”}} 郭氏语充,欲就省李\footnote{省:看望。}。充曰:“彼刚介有才气\footnote{刚介:刚直耿介。},卿往不如不去\footnote{卿往不如不去:你想看她,还是不去为好。}。”{\fzxk\zihao{6}\textcolor{red}{\CJKunderwave{充别传}曰:“李氏有淑性令才也。”}} 郭氏于是盛威仪,多将侍婢\footnote{将:率领。}。既至,入户,李氏起迎,郭不觉脚自屈,因跪再拜。既反,语充。充曰:“语卿道何物\footnote{语卿道何物:余嘉锡\CJKunderwave{笺疏}引吴承仕曰:“以今语译之,当云:‘我告诉你什么来着?’何物即什么,么即物的声转。”按:何物,魏晋时口语。}?”{\fzxk\zihao{6}\textcolor{red}{按\CJKunderwave{晋诸公赞}曰:“世祖以李丰得罪晋室,又郭氏是太子妃母,无离绝之理,乃下诏敕断,不得往还。”而王隐\CJKunderwave{晋书}亦云:“充既与李绝婚,更取城阳太守郭配女,名槐。李禁锢解,诏充置左右夫人。充母柳亦敕充迎李。槐怒,攘臂责充曰:‘刊定律令,为佐命之功,我有其分。李那得与我并!’充乃架屋永年里中以安李。槐晚乃知,充出,辄使人寻充。诏许充置左右夫人。充答诏,以谦让不敢当盛礼。”\CJKunderwave{晋赞}既云:“世祖下诏,不遣李还。”而王隐\CJKunderwave{晋书}及\CJKunderwave{充别传}并言:诏听置立左右夫人,充惮郭氏,不敢迎李。三家之说并不同,木(未)详孰是。然李氏不还,别有馀故。而\CJKunderwave{世说}云自不肯还,谬矣。且郭槐彊很,岂能就李而为之拜乎?皆为虚也。}}

{\cangkai\zihao{5}【评】故事的主角是贾充的前妻李婉,但却着墨不多,主要是通过贾充及其后妻郭槐的言行来加以艺术衬托。郭槐悍妒有馀,才气不足。她往见李氏,“盛威仪,多将侍婢”,莫非想给李氏一个下马威,令其屈膝臣服,以便获得内心的满足。贾充了解前妻,劝郭不见为好,但郭槐不听,果然是自取其辱。郭、李相见的细节描绘,具体生动,“李氏起迎”,不失风度;“郭不觉脚自屈,因跪再拜”,郭之威风顿然消失。这生动地托出了李氏的刚介之性及其自爱自尊,形象高出郭氏许多。故事称李氏赦还后,“不肯还充舍”,刘注谓谬。其实,这正是李氏耿介性格的自然发展所致,谬误者应是刘孝标自己。}

\lettrine{19.14} 贾充\myidx{贾充}妻李氏\myidx{李婉}作\CJKunderwave{女训}行于世\footnote{\CJKunderwave{女训}:书名,作者李婉,其书已佚。}。李氏女\myidx{贾荃},齐献王妃\footnote{李氏女:指李婉与贾充所生女贾荃,一名褒,事附见\CJKunderwave{晋书·贾充传}。齐献王:司马攸,字大猷,司马昭子,晋武帝炎弟,封齐王,因武帝疑惧,被贬而忧卒,谥献。},郭氏女\myidx{贾南风}\footnote{郭氏女:指郭槐与贾充所生女贾南风。\CJKunderwave{晋书·后妃}有传。晋惠帝时,专擅朝政,直接开启八王之乱,被赵王所诛。},惠帝后\footnote{惠帝:指痴呆皇帝司马衷,武帝子。}。充卒,李、郭女各欲令其母合葬,经年不决\footnote{经年:历年,多年。}。贾后废,李氏乃祔葬\footnote{祔:(fù附)葬:合葬。},遂定。{\fzxk\zihao{6}\textcolor{red}{\CJKunderwave{晋诸公赞}曰:“李氏有才德,世称\CJKunderwave{李夫人训}者。生女合(荃),亦才明,即齐王妃。”\CJKunderwave{妇人集}曰:“李氏至乐浪,遗二女\CJKunderwave{典式(戒)}八篇。”王隐\CJKunderwave{晋书}曰:“贾后,字南风,为赵王所诛。”}}

{\cangkai\zihao{5}【评】故事应该发生于太康三年(282)四月贾充卒后。当时,李、郭二位夫人先充而卒,故有争合葬事。据礼,夫妇合葬。但充前妻李氏,因是犯官李丰之女,徙边赦还,终武帝时,不入充门,未被正式承认,所以无法恢复她的合法地位。但是,贾后被诛之后,政治天平倾向李氏,于是祔葬之礼遂定。其合乎礼数与否,实与当时的政治斗争密切相关。透过祔葬之争,正可见出西晋一朝政教人心之混乱。据\CJKunderwave{隋书·经籍志}载,充妻李氏有集一卷,又有\CJKunderwave{女训}行于世,可见其文化素养及其才华,但生当乱世,渡过被世抛弃的悲剧人生,冤哉枉哉!}

\lettrine{19.15} 王汝南\myidx{王湛}少无婚\footnote{王汝南:王湛(249—295),字处冲,出于太原王氏世家望族。曾官汝南内史,故称。好读书而冲素简淡,人以为痴。参前\CJKunderwave{赏誉}第17则注。无婚:未婚。},自求郝普\myidx{郝普}女\footnote{郝普:刘注谓其门第孤陋,当是出身于庶族寒门。}。{\fzxk\zihao{6}\textcolor{red}{\CJKunderwave{郝氏谱}曰:“普字道匡,太原襄城人。仕至洛阳太守。”}} 司空\myidx{王昶}以其痴\footnote{司空:指王昶(?—259),字文郐。王浑、王湛之父。有才智谋略,官至魏司空,故称。},会无婚处\footnote{会:恰巧。无婚处:没有婚配对象。},任其意,便许之。{\fzxk\zihao{6}\textcolor{red}{\CJKunderwave{魏氏志}曰:“王昶字文舒,仕至司空。”}} 既婚,果有令姿淑德,生东海\myidx{王承}\footnote{东海:指王承,字安期,湛子。晋之名士。官至东海太守,故称。},遂为王氏母仪\footnote{母仪:贤妻良母之典范。}。或问汝南:“何以知之?”曰:“尝见井上取水,举动容止不失常,未尝忤观\footnote{忤(wǔ五)观:随心东张西望。},以此知之。”{\fzxk\zihao{6}\textcolor{red}{\CJKunderwave{汝南别传}曰:“襄城郝仲将,门至孤陋,非其所偶也。君尝见其女,便求聘焉。果高朗英迈,母仪冠族。其通识馀裕皆此类。”}}

{\cangkai\zihao{5}【评】这则故事,虽写王湛之妻“有令姿淑德,生东海,遂为王氏母仪”,但作为妇女楷模,事仅如此,似乎并非描写的中心人物。其实,如把此则故事移至\CJKunderwave{识鉴}门,可能更合适。其主角人物是“痴汉”王湛,据前\CJKunderwave{赏誉}第十七则故事,当时不仅是家族成员认为湛痴,就连晋武帝也当面调侃王济,问道:“卿家痴叔死未?”实际上,王湛是大智若愚,好读深思,默默似痴,他精于\CJKunderwave{易}理,在王济等清谈名家前曾一展风采。故其临事之智,源于平素沉潜之思,何痴之有?他对婚姻,相信自己的选择,而不以“父母之命、媒妁之言”为据,在冲破传统习惯方面,甚为大胆,这是其“痴”——也即可爱之一;其妻出于郝家,刘孝标注谓妻父郝普门第孤陋,妻未出嫁时,又亲上井台打水,可见出于庶族寒门,与太原王氏高门士族,门第悬隔,在魏晋门阀社会中,本是难以逾越的婚姻障碍。但王湛却主动打破门第偏见,其“痴”之可爱二也;不以妇德审查未婚妻,而却相信自己的直观判断,以郝女眼不“忤观”,眼睛是心灵的窗户,判断其为人之正派,在实践中来完成考察,其“痴”可爱之三也。}

\lettrine{19.16} 王司徒\myidx{王浑}妇,锺氏女,太傅\myidx{锺繇}曾孙\footnote{王司徒:指王浑,字玄冲,官至司徒,故称。锺氏女,太傅曾孙:指王浑妻锺氏,字琰,魏太傅锺繇曾孙女,父徽为黄门侍郎。事载\CJKunderwave{晋书}卷九六\CJKunderwave{列女传}。据此,则刘注有误。}。{\fzxk\zihao{6}\textcolor{red}{\CJKunderwave{王氏谱}曰:“夫人,黄门侍郎锺琰(徽)女。”}} 亦有俊才女德\footnote{俊才:杰出才智。}。{\fzxk\zihao{6}\textcolor{red}{\CJKunderwave{妇人集}曰:“夫人有文才,其诗、赋、颂、诔行于世。”}} 锺、郝为娣姒\footnote{锺、郝为娣姒(dì sì第似):锺夫人夫王浑,郝夫人夫王湛,是亲兄弟。娣姒,即妯娌,兄弟间妻子之间的称呼。},雅相亲重:锺不以贵陵郝\footnote{陵:通“凌”,欺凌,侮辱。},郝亦不以贱下锺\footnote{下:低下。“下锺”即比锺氏低下一等。}。东海家内\footnote{东海:湛子王承。},则郝夫人之法\footnote{则:以……为准则。“则”是准则,名词动化。};京陵家内\footnote{京陵:指京陵侯王浑。},范锺夫人之礼\footnote{范:以……为榜样或典范。名词动化。}。

{\cangkai\zihao{5}【评】人的门第出身,爹娘无法选择。这在魏晋门阀社会中,关系重大,不仅男人因士庶之别,仕途自然悬殊;就是女人,也因士庶之隔,婚姻遂有大家小户等难以逾越的障碍。可贵的是,锺、郝二位夫人,一是颍川长社锺氏世家望族,一是孤陋寒门之女,出身虽有士庶之别,但却毫无传统的门第偏见。出身高门的锺夫人,不以富贵凌人,已属不易;而出身寒微的郝夫人,不因贫贱而自卑地低下头颅,更是气度非凡,难能可贵。因为她们都明白自己的人生价值,尊重自己,见其真情,而毫无虚矫之饰。这正是魏晋妇女的可爱之处。}

\lettrine{19.17} 李平阳\myidx{李重}\footnote{李平阳:即李重,字茂曾,江夏钟武人。曾官平阳太守,故称。参\CJKunderwave{品藻}第46则注。},秦州\myidx{李秉}子\footnote{秦州:指重父李秉,刘注引\CJKunderwave{永嘉流人名}作李康,误。曾任魏秦州刺史,故称。},{\fzxk\zihao{6}\textcolor{red}{李重,已见。\CJKunderwave{永嘉流人名}曰:“康(秉)字玄胄,江夏人。魏秦州刺史。”}} 中夏名士\footnote{中夏:指中原地区。},于时以比王夷甫\myidx{王衍}\footnote{王夷甫:王衍字夷甫,晋初名臣,清谈领袖人物。参前\CJKunderwave{言语}第23则注。}。孙秀\myidx{孙秀}初欲立威权\footnote{孙秀(?—301):字俊忠,赵王伦篡位称帝后任中书令,专擅朝政,后被齐王冏所杀。按:晋初有两孙秀,另一为江东孙秀,字彦才。},咸云:“乐令\myidx{乐广}氏(民)望\footnote{乐令:指西晋名臣乐广,字彦辅,与王衍并称清谈领袖。参前\CJKunderwave{德行}第23则注。},不可杀;减李重者,又不足杀\footnote{不足杀:不值得杀。}。”{\fzxk\zihao{6}\textcolor{red}{\CJKunderwave{晋诸公赞}曰:“孙秀字俊忠,琅邪人。初,赵王伦封琅邪,秀给为近职小吏。伦数使秀作书疏,文才称伦意。伦封赵,秀徙户为赵人,用为侍郎,信任之。”\CJKunderwave{晋阳秋}曰:“伦篡位,秀为中书令,事皆决于秀。为齐王所诛。”}} 遂逼重自裁\footnote{自裁:自杀。}。初\footnote{初:当初,先前。},重在家,有人走从门入\footnote{走:奔走,跑。},出髻中疏示重\footnote{疏:疏状,条陈。},重看之色动\footnote{色动:面孔变色。}。入内示其女,女直叫“绝\footnote{绝:完了,没希望。}”。了其意\footnote{了:明白,清楚。},出则自裁。{\fzxk\zihao{6}\textcolor{red}{案:书皆云:“重知赵王伦作乱,有疾不治,遂以致卒。”而此书乃言自裁,甚乖谬。且伦、秀凶虐,动加诛夷,欲立威权,自当显戮,何为逼令自裁?}} 此女甚高明,重每咨焉\footnote{咨:咨询。}。

{\cangkai\zihao{5}【评】\CJKunderwave{晋书·李重传}谓重卒于永康初赵王伦当政之时。查晋惠帝永康仅一年,即公元300年,则故事发生于该年。当时贾后杀太子,直接诱发了西晋的八王之乱,局势动荡。在故事中,李重之女一看条陈,立即明白了形势的严重性,直叫完了。于此可见其准确判断的政治智慧。“此女甚高明,重每咨焉”,篇末画龙点睛,神韵秀出。按:史称李重多次上疏,如批评九品中正官人法云:“九品始于丧乱,军中之政,诚非经国不刊之法也。且其检防转碎,征刑(形)失实,故朝野之论,佥谓驱动风俗,为弊已甚。”表现了一个清醒政治家的远见卓识。有其父必有其女,但女儿之慧,却无用武之地。惜哉!}

\lettrine{19.18} 周浚\myidx{周浚}作安东时\footnote{周浚:字开林,汝南安成(今河南正阳)人。曹魏时官拜折冲将军、扬州刺史,封射阳侯。平吴时以功封成武侯,拜安东将军、都督扬州诸军事。},行猎,值暴雨\footnote{值:遭遇,碰到。},过汝南李氏\footnote{汝南:郡名,晋时治所在悬瓠城(今河南汝南)。李氏:李家。}。李氏富足,而男子不在。有女名络秀\footnote{络秀:李伯宗女,嫁周浚。},闻外有贵人\footnote{贵人:显贵之人,指公卿大夫之辈。},与一婢于内宰猪羊,作数十人饮食,事事精办,不闻有人声。密觇之\footnote{觇(chān搀):窥视,暗中观看。},独见一女子,状貌非常。浚因求为妾,父兄不许。络秀曰:“门户殄瘁\footnote{门户:门第。殄瘁(tiǎn cuì忝粹):衰败。},何惜一女!若连姻贵族,将来或大益。”父兄从之。{\fzxk\zihao{6}\textcolor{red}{\CJKunderwave{八王故事}曰:“浚字开林,汝南安城(成)人。少有才名。太康初,平吴,自御史中丞出为扬州刺史。元康初,加安东将军。”}} 遂生伯仁\myidx{周顗}兄弟\footnote{生伯仁兄弟:李络秀为周浚生三子:周顗,字伯仁;周嵩,字仲智;周谟。三子在东晋时皆居显职,甚有声名。}。络秀语伯仁等:“我所以屈节为汝家作妾\footnote{屈节:谓违背自己感情而降身相从。},门户计耳\footnote{门户计耳:意谓为娘家门第打算而与周家联婚。}。{\fzxk\zihao{6}\textcolor{red}{案:\CJKunderwave{周氏谱},浚取同郡李伯宗女,此云为妾,妄耳。}} 汝若不与吾家作亲亲者\footnote{亲亲:亲戚。},吾亦不惜馀年!”伯仁等悉从命。由此李氏在世得方幅齿遇\footnote{方幅:魏晋时口语,本义为形体方整,这里引申为公开、“正当”或“光明正大”之义,见徐震堮\CJKunderwave{校笺·词语简释}。}。

{\cangkai\zihao{5}【评】故事开篇谓“周浚作安东时”过汝南李家而迎娶李络秀,时间有误。周浚作安东将军在平吴之役后,平吴之役在晋初太康元年(280),其时浚已入晚年,垂垂老翁,何来新婚之喜?其与络秀所生长子周顗,据\CJKunderwave{晋书}本传,永昌元年(322)死于王敦刀下,“时年五十四”,上推生年,为晋武帝泰始五年(269),则平吴之后,已是十四五岁英俊少年。据此推断,“安东”应是“折冲”之讹。周浚盛年时任折冲将军。又\CJKunderwave{晋书·周浚传},浚曾娶史曜之妹为妻,据此,则李络秀屈节为妾,当是事实,史家因其子贤而为之讳,刘孝标驳难甚谬。}

{\cangkai\zihao{5}古代门户之立,重在男人,光宗耀祖之事,非男莫属。但汝南李氏,本是乡间一土财主,出身微贱,不入流品,与上流社会无涉。这在魏晋门阀社会中是正常现象,也即络秀所说的“门户殄瘁”。络秀为挽救家庭,牺牲感情,屈节为妾,生周顗贤兄弟,李氏娘家因此得“方幅齿遇”。故明李卓吾叹曰:“好女便立家,何必男子!”信然。}

\lettrine{19.19} 陶公\myidx{陶侃}少有大志\footnote{陶公(259—334):指东晋初名臣陶侃,参前\CJKunderwave{言语}第47则注。},家酷贫\footnote{酷:非常,极其。},与母湛氏同居\footnote{湛(zhàn战)氏:陶侃母,以贤惠称。}。同郡范逵\myidx{范逵}素知名\footnote{范逵:鄱阳孝廉,侃友。},举孝廉\footnote{孝廉:汉时选举科目,孝指孝道,廉指廉洁,合称孝廉,由郡贡于朝廷。魏晋因之。},{\fzxk\zihao{6}\textcolor{red}{逵,未详。}} 投侃宿。于时冰雪积日,侃室如悬磬\footnote{室如悬磬:意谓家中一贫如洗。悬磬,谓府库空虚。},而逵马仆甚多。侃母湛氏语侃曰:“汝但出外留客,吾自为计。”湛头发委地\footnote{委地:下垂及地。},下为二髲\footnote{髲(bì币):假发。},{\fzxk\zihao{6}\textcolor{red}{一作“髢”。}} 卖得数斛米\footnote{斛:容器名,古时十斗为一斛。}。斫诸屋柱,悉割半为薪,剉诸荐以为马草\footnote{剉(cuò):铡碎。荐:草垫、卧席。}。日夕,遂设精食,从者皆无所乏。逵既叹其才辩,又深愧其厚意\footnote{愧:感谢。}。明旦去,侃追送不已,且百里许\footnote{且:将近。许:表约数之意。百里许,即大约百把里路。}。逵曰:“路已远,君宜还。”侃犹不返。逵曰:“卿可去矣。至洛阳,当相为美谈\footnote{当:一定。相为:为你。美谈:说好话。}。”侃乃返。逵及洛\footnote{洛:西晋国都洛阳。},遂称之于羊晫\myidx{羊晫}、顾荣\myidx{顾荣}诸人\footnote{羊晫:\CJKunderwave{晋书·陶侃传}作“杨晫”,历豫章郎中令、十郡大中正。举荐陶侃甚力。顾荣(?—312):字彦先,吴县(今属江苏)人。江东士族领袖人物。参前\CJKunderwave{德行}第26则注。},大获美誉。{\fzxk\zihao{6}\textcolor{red}{\CJKunderwave{晋阳秋}曰:“侃父丹,娶新淦湛氏女,生侃。湛虔恭有智算,以陶氏贫贱,纺绩以资给侃,使交结胜己。侃少为寻阳吏,鄱阳孝廉范逵尝过侃宿。时大雪,侃家无草,湛彻所卧荐剉给,阴截发,卖以供调。逵闻之叹息。逵去,侃追送之。逵曰:‘岂欲仕乎?’侃曰:‘有仕郡意。’逵曰:‘当相谈致。’过庐江,向太守张夔称之。召补吏,举孝廉,除郎中。时豫章顾荣或责羊晫曰:‘君奈何与小人同舆?’晫曰:‘此寒俊也。’”王隐\CJKunderwave{晋书}曰:“侃母既截发供客,闻者叹曰:‘非此母不生此子。’乃进之于张逵(夔)。羊晫亦简之。后晫为十郡中正,举侃为鄱阳小中正,始得上品也。”}}

{\cangkai\zihao{5}【评】古代女人才智纵能治国安邦,仁义道德足为乡国之典范,但因受封建礼教的束缚,无法入仕从政,所有才智皆付之东流。因此,她们只能作为贤妻良母,相夫教子,把自己的生命活力,转移到丈夫儿子身上。陶侃作为东晋一代名臣,正是其母亲精心培育的结果。陶侃能从一个庶族寒门,迅速上升到二品士族之门,湛氏的远见卓识,给儿子以智慧的启迪。陶母不惜牺牲自己的委地美发,正是为儿子赢来美好的前途。吃小亏占大便宜,看到远期投资的长远利益所在,其眼光之深邃,何减须眉!}

\lettrine{19.20} 陶公\myidx{陶侃}少时作鱼梁吏\footnote{鱼梁吏:管理水堤闸口捕鱼的官吏。鱼梁,一种捕鱼用的堤堰,以土石断水截流,中有缺口,置竹篓顺流捕鱼。},尝以坩鲊饷母\footnote{坩(gān甘):盛物陶器。鲊(zhǎ扎):一作“䱹”,异体字,经腌制加工而成的鱼类食品。}。母封鲊付使,反书责侃\footnote{反书:回信,“反”通“返”。}曰:“汝为吏,以官物见饷,非唯不益,乃增吾忧也\footnote{乃:却,只是。}。”{\fzxk\zihao{6}\textcolor{red}{\CJKunderwave{侃别传}曰:“母湛氏,贤明有法训。侃在武昌,与佐吏从容饮燕,常有饮限。或劝犹可少进,侃凄然良久,曰:‘昔年少,曾有酒失,二亲见约,故不敢踰限。’及侃丁母忧,在墓下,忽有二客来吊,不哭而退,仪服鲜异。知非常人,遣随视之,但见双鹤冲天而去。”\CJKunderwave{幽明录}曰:“陶公在寻阳西南一塞取鱼,自谓其池曰鹤门。”按吴司徒孟宗为雷池监,以鲊饷母,母不受,非侃也。疑后人因孟假为此说。}}

{\cangkai\zihao{5}【评】陶侃母湛氏,如刘辰翁所评,是“真陶母”也,也就是说,这是一个真正伟大的母亲。做母亲的一般心理,自己的儿女即便是癞痢头,也是好的。因此,常因从小溺爱而生护短心理。但是,溺爱并非真爱,故古人有严父慈母之说,一个孩子的成长,不仅要有母爱的关怀,更要有严父的教训。但陶侃少年失怙,严父见背而依恃寡母养育。湛氏在陶家身兼严父慈母之责。以此,为了孩子的前途,她对陶侃,除了一般女人的母爱之外,更是严加管教而绝不护短,即使陶侃已经长大成人,已提升为“干部”时也是如此。陶侃做鱼梁吏而送一坩鱼鲊,事情很小,不值几文,并见其奉母之心,出于至孝天性,应该说是好事。但湛氏的思想认识却比常人深入一层,一坩鱼鲊虽小,但却有监守自盗之嫌。为了做一个清清白白的廉洁官吏,能不引为警惕吗?贪污腐败之风不止,或许是从小贪一步步发展为大贪,最后弥漫开来,无可救药。以古鉴今,能无惧乎!}

\lettrine{19.21} 桓宣武\myidx{桓温}平蜀\footnote{桓宣武:指东晋中期权臣桓温,卒谥宣武,故称。参前\CJKunderwave{言语}第55则注。平蜀:平定蜀地李氏成汉小朝廷。时在晋穆帝永和三年(347),桓温时任征西大将军。},以李势\myidx{李势}妹为妾\footnote{李势:字子仁,十六国成汉国主。永和三年被桓温所灭,降晋封归义侯。参前\CJKunderwave{识鉴}第20则注。},甚有宠\footnote{宠:宠爱。},常箸斋后\footnote{箸:安置,安排。斋:此指书斋。}。主始不知\footnote{主:公主。此特指南康长公主,即桓温之妻。},既闻,与数十婢拔白刃袭之\footnote{白刃:指刀、剑一类武器。}。{\fzxk\zihao{6}\textcolor{red}{\CJKunderwave{续晋阳秋}曰:“温尚明帝女南康长公主。”}} 正值李梳头,发委藉地\footnote{委:下垂。藉地:铺地,席地。},肤色玉曜\footnote{玉曜:指皮肤洁白温润如玉。},不为动容,徐曰:“国破家亡,无心至此,今日若能见杀,乃是本怀。”主惭而退\footnote{惭:惭愧。}。{\fzxk\zihao{6}\textcolor{red}{\CJKunderwave{妒记}曰:“温平蜀,以李势女为妾。郡主凶妒,不即知之。后知,乃拔刃往李所,因欲斫之。见李在窗梳头,姿貌端丽,徐徐结发,敛手向主,神色闲正,辞甚凄惋。主于是掷刀前抱之,曰:‘阿子,我见汝亦怜,何况老奴。’遂善之。”}}

{\cangkai\zihao{5}【评】故事发生在东晋穆帝永和三年(347)桓温伐蜀胜利之后不久,写了两个女人的不幸遭遇。作为征西大将军、都督荆梁诸军事的统帅,桓温处于三十五岁的盛年,势力正在腾腾上升,所以虽然嫡妻是南康长公主,但却是宠妾成群,因而产生了感情危机,公主之“凶妒”,实是保护自己的一种手段。但因桓氏军事集团,当时朝廷倚重,气盖朝野,公主对驸马也无可奈何,而只能迁怒于另一弱女子。这就从正确滑向了错误的道路。公主仗其皇家之威,拔白刃以临弱女子,实在不应该。被掳李势之妹,故事叙述其梳头时“发委藉地,肤色玉曜”,来形容其美丽;另一方面,以“徐曰”二字,形容她在白刃之下不为动容,有泰山崩于前而不变色的镇定与从容,说明她是有充分的心理准备。“国破家亡,无心至此”,描绘了她那内在高昂的节概及其坚强的灵魂,而绝不向高贵与强暴势力低头。故事选入\CJKunderwave{贤媛}门,主角当然是李势之妹,而不是高贵的公主。}

\lettrine{19.22} 庾玉台\myidx{庾友}\footnote{庾玉台:庾友,字惠彦,小字玉台,庾冰第三子。},希\myidx{庾希}之弟也\footnote{希:指庾希,字始彦,庾冰长子,官至徐兖二州刺史,因其贵盛,为桓温所忌而诛杀。}。希诛,将戮玉台。{\fzxk\zihao{6}\textcolor{red}{希,已见。玉台,庾友小字。\CJKunderwave{庾氏谱}曰:“友字惠彦,司空冰弟(第)三子。历中书郎、东阳太守。”}} 玉台子\myidx{庾宣}妇\footnote{玉台子妇:指庾友长子宣之妻桓女幼。},宣武\myidx{桓温}弟桓豁\myidx{桓豁}女也\footnote{桓豁:字朗子,官征西大将军。桓温弟。},{\fzxk\zihao{6}\textcolor{red}{\CJKunderwave{庾氏谱}曰:“友字弘之,长子宣,娶宣武弟桓豁之女,字女幼。”}} 徒跣求进\footnote{徒跣:因情急赤足而行。},阍禁不内\footnote{阍:门房,守门人。内:通“纳”。不纳,即不让进门。},女厉声曰:“是何小人?我伯父门\footnote{伯父:特指桓温。},不听我前!”因突入,号泣请曰:“庾玉台常因人\footnote{因人:依随他人。},脚短三寸\footnote{脚短三寸:朱铸禹\CJKunderwave{汇校集注}云:“脚短三寸,比喻之辞,犹今俗语,‘比人短一头’或‘赶不上人脚后跟’之类。”},当复能作贼不\footnote{作贼:造反,叛乱。}?”宣武笑曰:“婿故自急\footnote{婿:指庾宣。急:危急。“婿故自急”,朱铸禹谓有两解皆可通:“一谓婿固当自急;一谓婿乃自情急,以明己实无欲诛戮之意。”}。”遂原玉台一门。{\fzxk\zihao{6}\textcolor{red}{\CJKunderwave{中兴书}曰:“桓温杀庾希弟倩,希闻难而逃。希弟友当伏诛,子妇桓氏女溸(请)温,得宥。”}}

{\cangkai\zihao{5}【评】封建时代的政治斗争,胜者为王败为寇,败者动辄株连九族,极其残酷。东晋时与朝廷分权执政的王、庾、桓、谢四大家族,轮流上台。先是琅邪王家当政,继而庾氏家族借王敦事件,挤兑王家而执政;后是桓氏家族借助军事实力独擅朝政,觊觎帝位;然后是桓温死,谢安所代表的谢氏家族执政。故事当发生在桓温专擅朝政的时期。当时一门显贵的庾氏子弟被桓温借故诛戮殆尽。桓女幼的丈夫庾宣情况岌岌可危显而易见。如果不是因为形势危急,桓女怎会“徒跣求进”呢?在大庭广众之下,连鞋袜也来不及穿,不顾闺秀小姐的身份,呵斥门卫,突入号泣,一连串的叙事动作,极其紧张、生动,把矛盾及人物心理斗争,推向高潮。公公小字脱口而出,顾不了平常礼节,正见其窘急之状。“庾玉台常因人,脚短三寸,当复能作贼不?”通过口语,刻画比喻,既形象生动,又道理深刻,一语破除了桓温的顾虑,——只要对其执政没威胁,答应侄女请求,有何不可?小女子重情;桓温则注重的是政权安全。}

\lettrine{19.23} 谢公\myidx{谢安}夫人帏诸婢\footnote{谢公夫人:谢安妻刘氏,沛国刘耽女,兄惔,当时玄学清谈名士。帏:帷帐,这里名词动化,作以帷帐遮隔。},使在前作伎\footnote{伎:原指歌儿舞女。这里名词动化。作伎,即表演歌舞或演奏音乐。},使太傅暂见便下帏\footnote{太傅:指谢安,卒赠太傅,故称。下帏:降下帐帏,犹今舞台之闭幕。}。太傅索更开\footnote{索:要求。},夫人云:“恐伤盛德\footnote{盛德:大德美名。}。”{\fzxk\zihao{6}\textcolor{red}{刘夫人,已见。}}

{\cangkai\zihao{5}【评】这则故事入\CJKunderwave{贤媛}门,初读不知所谓,观看歌舞,欣赏丝竹,何伤盛德?但细加品味咀嚼,自然明白个中奥秘。刘夫人出身名门,具有相当的文化素养,又颇有个性。丈夫谢安虽为一代名流,人称贤圣,但她在尽其妻子责任之时,却又能常常和丈夫平等商量,特别是在可能干扰其家庭感情生活方面,约束丈夫而寸步不让。这与现代的“女权主义”者有几分相似。原来,古代实行的是围绕男人为中心的一夫一妻多妾制。魏晋贵族,畜伎畜妾现象严重破坏了婚姻家庭生活的质量。古代常是“妓(伎)妾”连称,如\CJKunderwave{晋书·王国宝传}云:“后房伎妾以百数。”\CJKunderwave{世说·言语}第106则刘注引\CJKunderwave{续晋阳秋},称殷仲文“后房妓妾数十,丝竹不绝音”。当时畜妓畜妾,只是名分有别,实际功能相似。在这方面,刘夫人之“妒”,正是在夫妻关系的敏感点上的自然反映。对谢安来说,虽为一代贤相,但他同时也是个有血有肉、有七情六欲的男人,加以当时魏晋社会,广置女乐伎妾,不仅是一种生活享受,更是一种身份体现,何乐而不为?但从女人的视角来看,则刘夫人讥谢“恐有伤盛德”云者,正是一种维护自身、维护家庭、维护妇女正当权益的表现。王世懋评云:“此直妒耳,何足称贤?”实是不明就里的误解。}

\lettrine{19.24} 桓车骑\myidx{桓冲}不好箸新衣\footnote{桓车骑:指桓冲(328—384),字幼子,桓温弟。曾任车骑将军,故称。参前\CJKunderwave{夙惠}第6则注。箸:穿。},浴后,妇故送新衣与\footnote{妇:桓冲妻王女宗。出琅邪王氏家族。}。{\fzxk\zihao{6}\textcolor{red}{\CJKunderwave{桓氏谱}曰:“冲娶琅邪王恬安(女),字女也(宗)。”}} 车骑大怒,催使持去。妇更持还\footnote{更:再次,又。},传语云:“衣不经新,何由而故?”桓公大笑,箸之。

{\cangkai\zihao{5}【评】故事幽默、风趣而生动,洋溢着温馨的家庭生活气氛。桓冲不喜欢穿新衣,可能与其年幼时的艰苦生活有关,勤俭朴素原本是好的作风,但如执着过甚,非旧衣不穿,则又有偏执之弊。其妻王氏非常懂得生活的辩证法。新与旧是一对矛盾,没有新,又怎会有旧?衣服也是一样的道理。“衣不经新,何由而故?”妻子的话,合情合理,又体贴入微,她总希望把自己的丈夫打扮得漂漂亮亮,其爱夫之心,由心底飞到脸上。桓冲也是一代名流,很快感受到爱的关怀,因而在得意的大笑声中,披上新装。是妻子王氏的智慧和爱,改变了丈夫的生活习惯。}

\lettrine{19.25} 王右军\myidx{王羲之}郗夫人\myidx{郗璿}谓二弟司空\myidx{郗愔}、中郎\myidx{郗昙}\footnote{王右军郗夫人:王羲之妻郗璿,字子房,郗鉴女。司空:指鉴子郗愔,字方回。曾征拜司空,故称。中郎:指鉴子郗昙,注称字重渊,但据\CJKunderwave{晋书}愔传,则字重熙。曾官北中郎将,故称。愔、昙另参前注。}曰:{\fzxk\zihao{6}\textcolor{red}{司空,愔。已见。\CJKunderwave{郗昙别传}曰:“昙,字重渊,鉴少子。性韵方质,和正沉简。累迁丹阳尹,北中郎将,徐、兖二州刺史。”}} “王家见二谢\footnote{二谢:指谢安、谢万。万(328—约369),参前\CJKunderwave{言语}第77则注。},倾筐倒屣\footnote{倾筐倒屣:“屣”,袁本作“庋”,是。屣,鞋也。“倒屣”于义难通。而“庋”者,藏物品之木板或架子。倾筐倒庋,即倾家所有,热情招待。},{\fzxk\zihao{6}\textcolor{red}{二谢,安、万。}} 见女辈来\footnote{女辈:你们。“女”通“汝”。},平平尔\footnote{平平尔:普通平常罢了。}。汝可无烦复往\footnote{无烦复往:不必再去。}。”

{\cangkai\zihao{5}【评】王右军夫人郗璿,不仅是个活了九十馀岁的女寿星,更兼智慧与文才。魏晋是个门阀贵族统治的社会,出身门第是否高贵,关系到社会地位及人际关系。当时高平郗家与琅邪王家做姻亲。东晋时代,王、谢家族是高门士族,朝廷支柱,地位极高。当时郗氏家族,虽然也是高官满门的高门士族,但与王、谢二氏相比,地位影响都略逊一筹。同样是士族,其间仍有亲疏贵贱之别。王羲之家之所以待谢安等特别热情,不仅因个人情谊,还因谢家门第高贵,地位正在上升,所以努力联络以便扩大王家影响。至于郗家,王羲之及其子弟认为门第不如己之高贵,故流露轻视之色。郗璿在生活实践中,早已悟透这层道理。故劝二弟“无烦复往”,以免自讨没趣。她强调的是人的尊严。刘辰翁云:“语悉世情,可以有省。”所评甚是,人情关系,世态炎凉,实是社会大学中的一门重要学问。}

\lettrine{19.26} 王凝之\myidx{王凝之}谢夫人\myidx{谢道韫}既往王氏\footnote{王凝之:羲之次子,字叔平。参前\CJKunderwave{言语}第71则注。谢夫人:即谢道韫,谢奕女,谢安侄女。或谓道韫为字,名韫元,可备一说。参前\CJKunderwave{言语}第71则注。往:嫁。},大薄凝之\footnote{薄:轻视,瞧不起。}。既还谢家,意大不悦\footnote{意:心情,情绪。}。太傅\myidx{谢安}慰释之\footnote{太傅:指谢安。慰释:安慰开释。},曰:“王郎\footnote{王郎:指王凝之。郎,对青年男子的美称。},逸少之子\footnote{逸少:王羲之字。},人身亦不恶\footnote{人身:人才。不恶:不差,不错。},汝何以恨乃尔\footnote{恨:怨恨,遗憾。乃尔:如此,这样。}?”答曰:“一门叔父\footnote{门:家门,家族。},则有阿大\myidx{谢尚}、中郎\myidx{谢据}\footnote{阿大:指谢安从兄谢尚,字仁祖。谢鲲子,官拜尚书仆射,进号镇西将军。参前\CJKunderwave{言语}第46则注。中郎:指谢据,字玄通,小字虎子,谢安二兄。};群从兄弟,则有封\myidx{谢韶}、胡\myidx{谢朗}、遏\myidx{谢玄}、末\myidx{谢渊}\footnote{封、胡、遏、末:封,谢韶小字,字穆度,官车骑将军。胡,朗小字,字长度,官东阳太守。遏,玄小字,字幼度,卒赠车骑将军。末,渊小字,字叔度,官义兴太守。见\CJKunderwave{晋书·谢万传}:“时谢氏尤彦秀者,称封、胡、羯(遏)、末。封谓韶,胡谓朗,羯谓玄,末谓川(渊)。”}。{\fzxk\zihao{6}\textcolor{red}{封胡,谢韶小字。遏末,谢渊小字。韶字穆度,万子,车骑司马。渊字叔度,奕弟(第)二字(子),义兴太守,时人称其尤彦秀者。或曰封、胡、遏、末。封谓朗,遏谓玄,末谓韶。“朗、玄、渊”一作“胡谓渊,遏谓玄,末谓韶”也。}} 不意天壤之中\footnote{不意:没想到。天壤:天地之间。},乃有王郎\footnote{乃有王郎:竟然有这样的王郎!是极度轻蔑之意。}!”

{\cangkai\zihao{5}【评】这则故事,把一个贵族少妇埋怨丈夫的声调口吻,描绘得栩栩如生。东晋时代,是以男性为中心的门阀社会,高门士族是统治集团的核心,到处弥漫着傲慢与偏见的空气,令人窒息。年轻的女主角谢道韫就生活在这样的贵族上流社会中。她大概在琅邪王家受足了气,备受压抑,因此一回到娘家,就埋怨这门亲事,抒泄怨恨。但是,叔父谢安从维护王、谢家族之间的政治联盟出发,不希望其婚姻破裂,因而一方面对侄女加以“慰释”,另一方面又加以批评。“王郎,逸少之子,人身亦不恶”,言外之意,能嫁给王羲之的儿子这样的人才,还有什么可埋怨的呢?在这里,谢安是以政治家的眼光来看问题。作为高门士族子弟,王凝之仕途一帆风顺,这比什么都重要。但是,谢道韫是从一个年轻妻子的女性眼光来看丈夫,夫妻之间,关键在于感情的融洽。谢道韫是一个很有思想、颇有文学才华的女子,是“未若柳絮因风起”的主人。但其丈夫王凝之,却因其琅邪王家的声名门第,眼中很少有人,就连他的舅父郗家,他也瞧不起,因而被舅父骂为“鼠辈敢尔”!他又迷信天师道,在做会稽内史时,孙恩部队攻城,他作为守土有责的长官统帅,居然迷信天兵天将而不设防,以致城破身死、二儿被杀,令妻子终生过着以泪洗面的生活。王凝之之狂妄与偏执,在日常家庭生活中一定会有所表现,谢道韫和他天天生活在一起,因而有所觉察而埋怨,是生活的必然。其感情体验,即使是一代贤相的叔父谢安,也无法理解。作为魏晋贵族妇女智慧的代表人物,谢道韫只能敛尽才华光芒,眼看男人的愚言蠢行而徒唤奈何,让自己的生命在时光中无声流逝,女人不幸,悲乎哀哉!}

\lettrine{19.27} 韩康伯\myidx{韩伯}母隐古凡(几)毁坏\footnote{韩康伯母:名士殷浩之妹。参前\CJKunderwave{德行}第47则注。韩伯,字康伯,颍川长社(今河南长葛)人。隐:凭,倚。},卞鞠\myidx{卞鞠}见凡(几)恶欲易之\footnote{卞鞠(?—405):卞范之,字敬祖,小字鞠。桓玄篡晋,倚为心腹,事败被杀。参前\CJKunderwave{伤逝}第19则注。恶:劣,坏。}。{\fzxk\zihao{6}\textcolor{red}{鞠,卞范之,母之外孙也。}} 答曰:“我若不隐此,汝何以得见古物?”

{\cangkai\zihao{5}【评】韩康伯的母亲是一个聪慧而富有同情心的贵族妇女。魏晋贵族生活竞为豪奢,故\CJKunderwave{世说}专立\CJKunderwave{汰侈}门。而韩母则反其道而行之,以其凭倚古几而不换新的行为,来提倡勤俭朴素,物尽其用。但她对于不同意见,特别是下辈子孙的好意,不是倚老卖老,一味呵斥而引起反感。相反,她是运用智慧,以幽默的语言来化解矛盾,泯灭代沟,令人自然感悟其中的道理。这样,家庭生活就充满了趣味和色彩,而不是呆滞死板的上下辈也即新与老的对抗。}

\lettrine{19.28} 王江州\myidx{王凝之}夫人语谢遏\myidx{谢玄}曰\footnote{王江州夫人:指王凝之夫人谢道韫。王凝之曾任江州刺史,故称。谢遏:谢玄,小字遏。按:刘注谓道韫为“玄之妹”,误。据史,当为玄之姐。}:“汝何以都不复进\footnote{都:完全,全然。不复过:不再进步。}?{\fzxk\zihao{6}\textcolor{red}{夫人,玄之妹(姐)。}} 为是尘务经心\footnote{尘务:世俗之事。经心:烦心。},天分有限\footnote{天分:天资。}?”

{\cangkai\zihao{5}【评】谢道韫虽为女性,但论其智慧——即知识、学问与文学才华,实是魏晋仕女之精英。刘孝标称其“有文才,所著诗、赋、诔、颂传于世”(见\CJKunderwave{言语}第71则注)。其玄学清谈及哲理思辨,水平更在其小叔子王献之这个名士之上。在其晚年嫠居会稽之时,太守刘柳请见,曾叹美云:“实顷所示见,瞻察言气,使人心形俱服。”(以上见\CJKunderwave{晋书·列女传})于此可见其才华。她对弟弟谢玄的批评,说他读书学习不认真努力,文才学植没有进步,于此又可见其居高临下的告诫口气,说明她对教育和学习的重视。一个人只有不断学习,才能有不断的进步。谢玄后来成长为一代风流儒将名臣,在淝水之战中声威赫赫而名标青史,当与叔父谢安及其姐道韫的关心、教育分不开。论智慧才华,道韫不让须眉。但是,正如她在\CJKunderwave{拟嵇中散诗}中所悲叹的:“时哉不我与!”作为女性,对于社会与时代,又完全是无可奈何,悲乎!}

\lettrine{19.29} 郗嘉宾\myidx{郗超}丧\footnote{郗嘉宾:郗超(336—377),字嘉宾,一字景兴(或作敬舆),高平金乡(今属山东)人。愔子。参前\CJKunderwave{言语}第52则注。},妇兄弟欲迎妹还\footnote{妇:指郗超之妻周马头。},终不肯归,{\fzxk\zihao{6}\textcolor{red}{\CJKunderwave{郗氏谱}曰:“超娶汝南周闵女,名马头。”}} 曰:“生纵不得与郗郎同室,死宁不同穴\footnote{“生纵”二句:语出\CJKunderwave{诗经·王风·大车}:“穀则异室,死则同穴。”穀,活着。原是描写一位女子热恋时的爱情诗。郗超之妻,则借其热恋誓词以表明自己的决心。}?”{\fzxk\zihao{6}\textcolor{red}{\CJKunderwave{毛诗}曰:“穀则异室,死则同穴。”郑玄注曰:“穴,谓圹中墟也。”}}

{\cangkai\zihao{5}【评】郗超是桓温谋主,在政治上受人诟病,连他的父亲郗愔都很生气。但如撇开政治上的党派之别,则日常生活中的郗超是一个具有远见卓识而才华出众的智囊人物,深受妻子的热爱。郗超死,周家兄弟为免其寡居之苦,准备迎周马头回娘家。这说明魏晋时寡妇可以改嫁,不像宋明之后受到舆论歧视。但周马头因为自己的感情原因,坚决拒绝了兄弟的好意。故事中的“生纵”二句,铮铮誓言,掷地有声,借古喻今,以表明自己对于死去丈夫忠贞不贰的爱情。所言颇有文化修养,感情热烈奔放,愈增其感人力量。}

\lettrine{19.30} 谢遏\myidx{谢玄}绝重其姊\myidx{谢道韫}\footnote{谢遏姊:指谢道韫,详参本门第26则注。谢遏,谢玄小名遏。重:敬重,尊重。},张玄\myidx{张玄}常称其妹\footnote{张玄:一名张玄之,字祖希。少以学显。官吴兴太守、吏部尚书。时与谢玄齐名,誉为“南北二玄”。参前\CJKunderwave{言语}第51则注。张玄妹佚名。},欲以敌之。有济尼者,并游张、谢二家\footnote{游:交游、交往。},人问其优劣,答曰:“王夫人神情散朗\footnote{王夫人:指王凝之夫人谢道韫。散朗:洒脱,潇散疏朗。},故有林下风气\footnote{林下风气:竹林名士的风韵气概。};顾家妇清心玉映\footnote{顾家妇:指张玄妹,嫁与顾氏,故称。清心玉映:心胸明净,如玉辉映。},自是闺房之秀\footnote{闺房之秀:妇女中之秀出皎然者。}。”

{\cangkai\zihao{5}【评】东晋中期的张玄与谢玄,时人并称“南北二玄”,他们一样敬重称赞其家中姐妹。二玄皆为当时名士。谢玄曾受其姐道韫的关爱与教导,因而“绝重其姐”,则是事实之必然,一片诚挚之心。至于张玄之称其妹,虽然其妹可能天资美善有足称者,但因其听说谢重其姐而“欲以敌之”,如佛家\CJKunderwave{金刚经}所说的是一种“住于相”的执着,有意而为,已落第二义。当时之济尼姑,聪明对应,其比较二家优劣,有讨好二家之心,又自有其高下评判,其识见有过人之处。如余嘉锡\CJKunderwave{笺疏}所评:“道韫以一女子而有林下风气,足见其为女中名士。至称顾家妇为闺中之秀,不过妇人中之秀出者而已。不言其优劣,而高下自见,此晋人措词妙处。”所论甚是。以模糊语言应对,愈增其评判之佳妙,魏晋之尼,亦不简单。}

\lettrine{19.31} 王尚书惠\myidx{王惠}尝看王右军\myidx{王羲之}夫人\footnote{王尚书惠:王惠,字令明。出于琅邪王氏,王羲之族孙。曾任吏部尚书,故称。王右军夫人:指王羲之夫人郗璿。参前注。},{\fzxk\zihao{6}\textcolor{red}{\CJKunderwave{宋书}曰:“惠字令明,琅邪人。历吏部尚书,赠太常卿。”}} 问:“眼耳未觉恶不\footnote{恶:劣,差。此指视听能力的衰退。}?”{\fzxk\zihao{6}\textcolor{red}{\CJKunderwave{妇人集}载谢表曰:“妾年九十,孤骸独存。愿蒙哀矜,赐其鞠养。”}} 答曰:“发白齿落,属乎形骸\footnote{形骸:形体躯壳。};至于眼耳,关于神明\footnote{神明:精神。},那可便与人隔!”

{\cangkai\zihao{5}【评】如前所述,王羲之妻郗夫人是个颇富睿智的性情中人,观前教诫二弟勿往王家,即是体悉世情炎凉的金玉良言。她活了九十馀岁,一阵又一阵的白发人送黑发人,不仅丈夫,就是子女也相继谢世,此情此景,能无恸乎?其“发白齿落”,乃自然之数;但眼耳视听,“关于神明”,人的精神超乎形骸之外,“那可便与人隔”?作为魏晋知识妇女的出色人物,与当时士人一样,关心的是人类心灵的交流。其对话言简意赅,颇见思辨色彩。}

\lettrine{19.32} 韩康伯\myidx{韩伯}母殷\footnote{韩康伯母殷:豫章太守殷羡之女。韩康伯,即韩伯,参前注。},随孙绘之\myidx{韩绘之}之衡阳\footnote{绘之:字季伦,韩伯子。在桓景叛乱中被害。衡阳:郡名,晋时治所在湘乡。之:往,到。},{\fzxk\zihao{6}\textcolor{red}{\CJKunderwave{韩氏谱}曰:“绘之,字季伦。父康伯,太常卿。绘之仕至衡阳太守。”}} 于阖庐洲中逢桓南郡\myidx{桓玄}\footnote{阖庐洲:长江中小洲名。桓南郡:指桓玄,字敬道。桓温子,袭封为南郡公,故称。后篡位兵败被杀。参前\CJKunderwave{德行}第41则注。}。卞鞠\myidx{卞鞠}是其外孙\footnote{卞鞠:卞鞠是桓玄心腹。},时来问讯\footnote{问讯:问候,问安。}。谓鞠曰:“我不死,见此竖二世作贼\footnote{竖:竖子、小子。二世:指桓温、玄父子二代。作贼:叛乱,造反。}。”在衡阳数年,绘之遇桓景真\myidx{桓亮}之难也\footnote{桓景真:桓亮,字景真。温孙,玄侄。桓玄篡位败亡,亮举兵反叛,被杀。}。{\fzxk\zihao{6}\textcolor{red}{\CJKunderwave{续晋阳秋}曰:“桓亮,字景真,大司马温之孙。父济,给事中。叔父玄,篡逆见诛。亮聚众于长沙,自号湘州刺史,杀太宰甄恭、衡阳前太守韩绘之等十馀人。为刘毅军人郭珍斩之。”}} 殷抚尸哭曰:“汝父昔罢豫章,征书朝至夕发\footnote{征书:朝廷征召官吏的文件。}。汝去郡邑数年,为物不得动\footnote{为物不得动:指韩绘之调离衡阳太守之职,因种种人事牵缠,迟迟不离开湖湘地区。},遂及于难,夫复何言!”

{\cangkai\zihao{5}【评】韩伯之母殷氏,内蕴睿智,外具性情,乃一代妇女之英。其政治敏感及洞察力,具远见卓识,非常人能及。卞鞠虽其外孙,但更是桓玄的死党与心腹。殷氏当其面直斥桓温桓玄父子,谓“此竖二世作贼”。在魏晋时,因篡弑相继,故统治者提倡“以孝治国”,从而绕开了“忠”义之道。因此,不少魏晋贵族对于“忠”君思想,淡漠得很,更有人萌发了取而代之的政治野心。桓氏父子即是如此,这给东晋国家与人民带来了深重的灾难。韩母殷氏虽为老妇,但关心国事,其斥“此竖”之言,口语生动,见其节概。至于孙绘之死难,又抚尸恸哭,批评其罢官去郡数年,“为物不得动”,缺乏淡泊名利之心及政治预见性,故有“夫复何言”之叹。此情此景,真性情中人,巾帼何减须眉!}






%%% Local Variables:
%%% mode: latex
%%% TeX-engine: xetex
%%% TeX-master: "../Main"
%%% End:
