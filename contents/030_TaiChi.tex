%% -*- coding: utf-8 -*-
%% Time-stamp: <Chen Wang: 2025-12-09 20:43:59>

% ○ ◎ ‧ 「 」 『 』 々 ( ) “ ” ■ ^[一-龥]
% 【\([^】][^】][^】]+\)】 → {\\fzxk\\zihao{6}\\textcolor{red}{\1}}
% \(【评】.*\) → {\\cangkai\\zihao{5}\1}
% \(【题解】.*\) → {\\cangkai\\zihao{5}\1}
% 《\([^》]+\)》 → \\CJKunderwave{\1}
% ^\([0-9]+.[0-9]+\) → \\lettrine{\1}
% {\\fzxk\\zihao{6}\\textcolor{red}{[^o]*}}

\setlength{\parindent}{0pt}


\chapter{汰侈第三十}




{\cangkai\zihao{5}【题解】 汰侈者,骄奢无度、放纵享乐之谓也。\CJKunderwave{汰侈}一门,所写的是统治者骄纵奢侈、荒淫无度的生活故事。\CJKunderwave{汰侈}与\CJKunderwave{俭啬},一个是骄纵享乐,恨不得一日费尽天下财;一个是贪鄙守财、拔一毛以利天下而不为。但若论其本质,却是异中见同,都是极端的利己主义在作怪。“俭啬”者钱财不许使用,“汰侈”者则聚敛无数钱财专为自己享用,实际核心所在都是“为我”二字。魏晋门阀社会,贵族豪姓衣钵相传,在少数贵族手中,搜刮聚集了巨亿的财富,因而他们有条件骄奢极欲,相沿成风。西晋之迅速沦亡,与此腐败之风不无关系。故\CJKunderwave{晋书·五行志中}云:“武帝初,何曾薄太官御膳,自取私食,子劭又过之,而王恺又过劭。王恺、羊琇之俦,盛致声色,穷珍极丽。至元康中,夸恣成俗,转相高尚,石崇之侈,遂兼王、何,而俪人主矣。崇既诛死,天下寻亦沦丧。”这一批又一批骄奢成风的贵族,其享受生活的资本从哪儿来呢?来自名门望族的高官厚禄及其政治经济特权,来自对于全国百姓小民敲骨吸髓的剥削掠夺。一人餐费数万于上,万人饥贫受苦于下,这样的社会,失掉了民心,而不是以人为本,怎能不速亡呢?此门12则故事,应细读并汲取历史教训。}

\lettrine{30.1} 石崇\myidx{石崇}每要客燕集\footnote{石崇(249—300):字季伦,渤海南皮人。官至侍中、征虏将军、荆州刺史。参见本门第8则注。要:通“邀”。燕集:宴会。},常令美人行酒\footnote{行酒:持觞劝酒。},客饮酒不尽者,使黄门交斩美人\footnote{黄门:魏晋贵族供内庭驱使的阉人。}。王丞相\myidx{王导}与大将军\myidx{王敦}尝共诣崇\footnote{王丞相:王导。大将军:王敦。二人皆出琅邪王氏家族。},丞相素不能饮,辄自勉强,至于沉醉。每至大将军,固不饮以观其变\footnote{固:坚决,坚持。},已斩三人,颜色如故,尚不肯饮。丞相让之\footnote{让:责备。},大将军曰:“自杀伊家人\footnote{伊:彼,他。},何预卿事?”{\fzxk\zihao{6}\textcolor{red}{王隐\CJKunderwave{晋书}曰:“石崇为荆州刺史,劫夺杀人,以致巨富。”\CJKunderwave{王丞相德音记}曰:“丞相素为诸父所重,王君夫问王敦:‘闻君从弟佳人,又解音律,欲一作妓,可与共来。’遂往。吹笛人有小忘,君夫闻,使黄门阶下打杀之,颜色不变。丞相还曰:‘恐此君处世,当有如此事。’”两说不同,故详录。}}

{\cangkai\zihao{5}【评】故事发生在晋武帝统一中国的歌舞升平时期,其时开国功臣及贵族大姓竞相汰侈豪奢。故\CJKunderwave{世说}专立\CJKunderwave{汰侈}门以示诫。王济用人乳喂猪,石崇作锦步障五十里以胜王恺紫丝步障四十里。这类物质上的骄奢,已是骇人听闻;而这则故事的权贵,竟然以残杀美人取乐,并以此夸豪斗富,则是精神上的汰侈,更是旷古未闻,可称是魏晋门阀社会的“特产”。故事的人物主角是石崇和王敦,是残杀美人的侩子手。汉时法律规定,“杀人者死”,汉光武时又下诏曰:“天地之性,人为贵。其杀奴婢,不得减罪。”见\CJKunderwave{后汉书·光武皇帝本纪}。魏晋时贵族妇女的生活,相对开放与自由。但奴婢等下层妇女则相反,因其人身依附,生命系于主人之手而毫无保障。如曹操杀女乐,司马懿妻杀女婢,贾后戟掷孕妇,上行下效,故石崇、王恺、王敦等敢于酒宴上公开杀美人。这不是偶然现象,而是门阀制度使然。女伎美人也是人。据\CJKunderwave{晋书·王敦传},美人行酒时,“悲惧失色”,其瑟缩战栗之态,写尽了内在心理的恐慌。但王敦却“颜色如故”,他是把少女的头颅当瓜果来品尝,把美人的鲜血当红花来欣赏。如此灭绝人性,是可忍,孰不可忍!故王世懋评曰:“无论处仲(王敦字)忍人,观此事,晋那得不乱!”}

\lettrine{30.2} 石崇\myidx{石崇}厕常有十馀婢侍列,皆丽服藻饰\footnote{丽服藻饰:衣服华丽,妆饰讲究。},置甲煎粉、沈香汁之属\footnote{甲煎粉:后世胭脂唇膏一类的化妆品。},无不毕备。又与新衣箸令出。客多羞不能如厕\footnote{入厕:上厕所。如:入。}。王大将军\myidx{王敦}往\footnote{王大将军:王敦。},脱故衣,箸新衣,神色傲然。群婢相谓曰:“此客必能作贼\footnote{作贼: 魏晋习惯用语,指造反。}!”{\fzxk\zihao{6}\textcolor{red}{\CJKunderwave{语林}曰:“刘寔诣石崇,如厕,见有绛纱帐大床,茵蓐甚丽,两婢持锦香囊。寔遽反走,即谓崇曰:‘向误入卿室内。’崇曰:‘是厕耳。’”}}

{\cangkai\zihao{5}【评】故事的时间与上则相近。人物主角是石崇,但与上则一样,第一主角应是王敦。看来,当时贵族是想尽各种办法来享乐,就是上厕所也要讲排场讲享受,不惜以公开“隐私”来夸豪斗富。石崇家的厕所具有时代的“超前性”。笔者曾到某集团公司数十层的高楼参观,整幢大楼最豪华最气派的地方,是小会议厅里的厕所。其装修之考究,服务之细致周到,令人叹为观止。可是这一切,比一千多年前的石家厕所,似乎远为逊色,因为古人还有十几个美女环列伺候。骄奢如此,西晋怎能不亡!一般士大夫尚有点羞耻感,但王敦则纵情享用,“脱故衣,箸新衣,神色傲然”,不以为耻。无耻之徒,不做贼造反才怪呢!后来果然不幸言中,王敦叛乱败亡。故凌濛初评曰:“何物婢子乃知人!”实践说明,卑贱者自有眼光。}

\lettrine{30.3} 武帝\myidx{司马炎}常降王武子\myidx{王济}家\footnote{武帝:晋武帝司马炎。王武子:王济,字武子,官至侍中,太仆卿。娶武帝女儿常山公主。},武子供馔\footnote{供馔:供奉食品。},并用璢璃器\footnote{璢即琉,琉璃器:用半透明矿石制造的器皿,当时极珍贵。}。婢子百馀人,皆绫罗袴𧟌\footnote{绫罗袴:用丝绸绫绢一类裁制的衣裤。袴:裤。𧟌:上衣。},以手擎饮食。蒸㹠肥美\footnote{蒸㹠:蒸乳猪。},异于常味。帝怪而问之,答曰:“以人乳饮㹠\footnote{以人乳饮㹠:用人奶喂养小猪。}。”帝甚不平,食未毕,便去。王\myidx{王恺}、石\myidx{石崇}所未知作\footnote{王、石:指王恺、石崇等贵族。未知作:不知道这种做法。}。{\fzxk\zihao{6}\textcolor{red}{𧟌,一作襬。}}

{\cangkai\zihao{5}【评】王济生活之侈华腐败,比于王(恺)、石(崇),连皇帝也“甚不平”,自叹不如。王济是武帝女婿,看在女儿常山公主分上,皇帝降临驸马爷家。为了迎驾,王济家在“硬件”上也极尽铺张,打扮得花枝招展的如花美女百馀人,“手擎饮食”,不过这不一定能与皇家排场相比;但其蒸乳猪,“以人乳饮㹠”,连皇帝也叹息,这就在精神骄奢方面,气势盖过了皇家。用人乳来喂猪,只是为了贵族一餐口食,又该有多少母亲被迫弃其呱呱待哺的婴儿而不顾!这样毫无人性,简直荒唐到举世无双的地步。但是,透过灯红酒绿豪华生活的外表,却可窥见魏晋贵族那无法把握自我命运而及时行乐的阴暗心理,悲乎!}

\lettrine{30.4} 王君夫\myidx{王恺}以𥹋糒澳釜\footnote{王君夫:王恺字君夫。晋武帝的司马炎的舅父。𥹋:同“饴”, 麦芽糖。糒:干饭。按:以干饭涮锅,义欠通。\CJKunderwave{晋书·石崇传}作“以𥹋澳釜”,无“糒”字,疑“糒”字衍。澳釜:涮锅。},石季伦\myidx{石崇}用蜡烛作炊\footnote{作炊:煮饭,做饭。};君夫作紫丝布步障碧绫裹四十里\footnote{步障:夹道屏障。},石崇作锦步障五十里以敌之\footnote{敌:与……对抗。};石以椒为泥\footnote{椒:花椒。以椒和泥涂壁,取其香气,兼喻多子。},王以赤石脂泥壁\footnote{赤石脂:风化石粉末的一种,红色,细腻,原是道家丹料,又可用来装饰墙壁。}。{\fzxk\zihao{6}\textcolor{red}{\CJKunderwave{晋诸公赞}曰:“王恺字君夫,东海人,王肃子也。虽无检行,而少以才力见名,有在军(公)之称。既自以外戚,晋氏政宽,又性至豪。旧制:鸩不得过江,为其羽栎酒中,必杀人。恺为翊军时,得鸩于石崇而养之,其大如鹅,喙长尺馀,纯食蛇虺。司隶奏按恺、崇,诏悉原之,即烧于都街。恺肆其意色,无所忌惮。为后军将军。卒,谥曰丑。”}}

{\cangkai\zihao{5}【评】这是一场古代的斗富大赛,双方无不穷奢极欲,各显神通。如在今日,当入世界吉尼斯纪录。但浪费多少民脂民膏,只为博取贵族斗奇争胜之名,令人扼腕叹息。王恺谥“丑”,石崇见诛,实是自我作孽,无可救药。其时,不仅个人,而且整个国家,也因骄奢汰侈之风的腐蚀,内里早就腐肠烂肚,很快天下大乱而中原沦丧,悲哉!}

\lettrine{30.5} 石崇\myidx{石崇}为客作豆粥,咄嗟便办\footnote{咄嗟:顷刻之间。喻其速。}。恒冬天得韭蓱䪡\footnote{韭蓱䪡:细碎的韭菜和艾蒿做成的腌菜。蓱,蓱萧,即艾蒿。䪡,捣碎研末的酱菜。}。又牛形状气力不胜王恺牛,而与恺\myidx{王恺}出游,极晚发,争入洛城,崇牛数十步后迅若飞禽,恺牛绝走不能及\footnote{绝走:奔跑。}。每以此三事为搤腕\footnote{搤腕:一手握另一手腕,愤怒不平貌。}。乃密货崇帐下都督及御车人\footnote{都督:总管家。},问所以\footnote{问所以:问其中原因。}。都督曰:“豆至难煮,唯豫作熟末\footnote{豫:通“预”,预先。熟末:煮熟研细。},客至,作白粥以投之。韭䪡是捣韭根,杂以麦苗尔。”复问驭人:“牛所以驶\footnote{驶:迅速飞奔。}?”驭人云:“牛本不迟,由将车人不及制之尔\footnote{不及制之尔:不善控制。不及,魏晋口语,不懂,不理解。}。急时听偏辕\footnote{偏辕:双辕之车,偏辕令车重心倾向一轮,以减少摩擦力而提速奔驰。},则驶矣。”恺悉从之,遂争长\footnote{争长:夺得胜利。}。石崇后闻,皆杀告者。{\fzxk\zihao{6}\textcolor{red}{\CJKunderwave{晋诸公赞}曰:“崇性好侠,与王恺竞相夸眩也。”}}

{\cangkai\zihao{5}【评】为夸富争胜而各极心思,也算是小智小慧。故事重心在牛车竞赛一节。御车人所说,很有道理。牛原健走,但主人驭不得法,紧拉缰绳,牛有力气而难于奔跑,一旦松绳,则自由飞奔,自然胜利夺标。用牛如此,用人亦然。管理者应充分发挥被用者的主观积极性,自然可收事半功倍的效益。此说颇有启迪。但石崇因竞赛失利而“皆杀告者”,则暴露了贵族丧失人性的残忍。}

\lettrine{30.6} 王君夫\myidx{王恺}有牛名八百里駮\footnote{王君夫:王恺,参前注。八百里駮:牛名。駮,同“驳”,原指黑白相间之马,这里借喻牛,意谓日行八百里的良种牛。},常莹其蹄角\footnote{莹其蹄角:用玉石装饰头角和牛脚。}。王武子\myidx{王济}语君夫\footnote{王武子:王济。参前注。}:“我射不如卿,今指赌卿牛,以千万对之\footnote{以千万对之:以千万钱与名牛价值相抵以赌。}。”君夫既恃手快\footnote{手快:动作敏捷,技术熟练。},且谓骏物无有杀理,便相然可\footnote{然可:应允,答应。},令武子先射。武子一起便破的\footnote{破的:射中靶心。},却据胡床\footnote{胡床:一种可折叠的轻便交椅。却:退回来。},叱左右速探牛心来\footnote{探:探取,掏。}。须臾,炙至\footnote{炙:烤肉。},一脔便去\footnote{脔:切块的肉。}。{\fzxk\zihao{6}\textcolor{red}{\CJKunderwave{相牛经}曰:“\CJKunderwave{牛经}出宁戚,传百里奚。汉世河西薛公得其书,以相牛,千百不失。本以负重致远,未服辎軿,故文不传。至魏世,高堂生又传以与晋宣帝,其后王恺得其书焉。”臣按其\CJKunderwave{柏(相)经}云:“阴虹属颈,千里。”注曰:“阴虹者,双筋白尾骨属颈。宁戚所饭者也。”恺之牛亦有阴红(虹)也。宁戚\CJKunderwave{经}曰:“棰头欲得高,百体欲得紧,大膁疏肋难齝,龙头突目欲好跳。又角欲得细,身欲促,形欲得如卷。”}}

{\cangkai\zihao{5}【评】在门阀社会的庇荫下,高门望族政治经济特权世代相袭。这样,这批贵族子弟无须竞争而照样升官发财,在养尊处优的生活中寻求种种刺激打破平淡无聊之心绪。一场豪赌就这样随意地发生了。据前\CJKunderwave{俭啬}门第5则,王戎“贷钱数万”与女儿,即脸色不悦,可见数万不少。又第9则“郗公大聚敛,有钱数千万”,数千万是一个大富豪终生积敛的数目。以此作为参照,则数千万之数为巨富之蓄,是千百普通人家财产的总和。但是二王所赌,“以千万对之”,更是虚荣腐化心理的角胜。八百里駮是人类经过长期努力培养的优良品种,属人类文明进步的体现,但王济杀之不眨一眼,探心烤炙,仅食一脔,即扬长而去。这对王恺不仅是物质的损失,更是巨大的心理打击。但是,破坏人类科学结晶的良种牛,王济这个纨绔子弟负有不可推卸的罪责。}

\lettrine{30.7} 王君夫\myidx{王恺}尝责一人无服馀袒(衵)\footnote{王君夫:王恺。尝:曾经。无服馀袒:没有穿贴身内衣。袒,他本皆作“衵”,是。衵(旧读nì昵,今读rì日),贴身内衣。朱铸禹\CJKunderwave{汇校集注}云:“此盖褫服以责之,但馀贴身小衣耳。”另备一说。},因直\footnote{直:通“值”,入朝值班,当时一值五日。},内箸曲閤重闺里\footnote{内:通“纳”。曲閤重闺:阁道回廊曲折回复,后庭闺房幽深重叠。},不听人将出\footnote{不听:不允许。}。遂饥经日,迷不知何处去。后因缘相为\footnote{因缘相为:朱铸禹注曰:“似谓偶值机缘,经人相助之意。”另有释“因缘”为亲近小吏,或指朋友、同伙。鄙意以前说为是。},垂死,乃得出。

{\cangkai\zihao{5}【评】王恺豪宅之广大深邃,犹如迷宫一般,致令侍者迷失方向,几乎饿毙。从宅居建筑,可见其穷奢极侈生活之一斑。王恺敢于在公开宴会之上,扑杀女乐,血溅当场;又因小失而禁锢侍者垂死,也属人命关天,于此又添罪证一桩。如此丧尽天良者占据要津,国家岂不危乎殆哉!}

\lettrine{30.8} 石崇\myidx{石崇}与王恺\myidx{王恺}争豪\footnote{争豪:竞夸斗富。},并穷绮丽以饰舆服\footnote{绮丽:华美艳丽。舆服:车服、衣冠章服之总称。古时制度,以舆服表等级。}。{\fzxk\zihao{6}\textcolor{red}{\CJKunderwave{续文章志}曰:“崇资产累巨万金。宅室舆马,僭拟王者。庖膳必穷水陆之珍。后房百数,皆曳纨绣,珥金翠,而丝竹之艺,尽一世之选。筑榭开沼,殚极人巧。与贵戚羊琇、王恺之徒,竞相高以侈靡,而崇为居最之首。琇等每愧羡,以为不及也。”}} 武帝\myidx{司马炎},恺之舅(甥)也,每助恺\footnote{每:常。}。尝以一珊瑚树高二尺许赐恺\footnote{珊瑚:海洋中珊瑚虫的石灰质骨髓所形成的物质,状如树枝,故称珊瑚树,可供赏玩。},枝柯扶疏\footnote{扶疏:繁茂纷披。},世罕其比。恺以示崇,崇视讫,以铁如意击之\footnote{铁如意:铁质如意。如意:柄端作手指形,用以搔背,尽如人意,故称。},应手而碎。恺既惋惜,又以为疾己之宝,声色甚厉。崇曰:“不足恨,今还卿。”乃命左右悉取珊瑚树,有三尺、四尺,条干绝世,光采溢目者六七枚,如恺许比甚众\footnote{许:处。}。恺惘然自失\footnote{惘然:失意或失神的样子。}。{\fzxk\zihao{6}\textcolor{red}{\CJKunderwave{南州异物志}曰:“珊瑚生大秦国,有洲在涨海中,距其国七八百里,名珊瑚树洲,底有盘石,水深二十馀丈,珊瑚生于石上。初生白,软弱似菌,国人乘大船,载铁网,先没在水下,一年,便生网目中。其色尚黄,枝柯交错,高三四尺,大者围尺馀。三年色赤,便以铁钞发其根,系铁网于船,绞车举网,还,裁凿恣意所作。若过时不凿,便枯索虫蛊。其大者输之王府,细者卖之。”\CJKunderwave{广志}曰:“珊瑚,大者可为车轴。”}}

{\cangkai\zihao{5}【评】在\CJKunderwave{汰侈}门12则故事中,王恺占5则,如果连第一则刘注所记故事,则为6则,是与石崇并列第一的主角。石、王二人多次夸豪斗富,为西晋上游贵族社会的骄奢之风推波助澜。石崇以铁如意击碎王恺之宝——皇帝赐予国舅爷的二尺许珊瑚树,随手击发,毫无怜惜之心。这一细节很典型生动,说明石崇不仅是富比王侯,甚至在某些方面,超过了皇宫。其搜刮民脂民膏,借官势大肆掠夺,手段直截了当。如史所称,“在荆州,劫远使商客,致富不赀”,这是执法犯法的伎俩。而王恺之奢华腐败,则直接来自最高统治者的支持与赏赐,后台大老板是皇帝。上自皇族,至于高门世家,无不竞相豪侈,以此为荣,朝廷如此腐败,国家岂能长久!西晋之亡,表面是“五胡乱华”,八王之乱,但若究其根底,则是自我作孽、内里腐败所致。须知,夸豪斗富,大失民心,而失民心者,必失天下。这就是历史的教训。}

\lettrine{30.9} 王武子\myidx{王济}被责\footnote{王武子:王济。责:斥责,责罚。},移第北芒下\footnote{移第:移家。北芒:即洛阳东北的北邙山。芒,通“邙”。}。{\fzxk\zihao{6}\textcolor{red}{\CJKunderwave{晋诸公赞}曰:“济与从兄恬(佑)不平。济为河南尹,未拜,行过王宫,吏不时下道,济于车前鞭之,有司奏免官。论者以济为不长者。寻转太仆,而王恬(佑)已见委任,济遂斥外。”}} 于时人多地贵,济好马射,买地作埒\footnote{\xpinyin*{埒}:矮墙。此指马场周围的界墙。},编钱匝地竟埒\footnote{编钱:穿联铜钱。匝(zā):环绕一周。},时人号曰“金沟”。

{\cangkai\zihao{5}【评】史称王济风姿英爽,气盖一时。“好弓马,勇力绝人”。生平善马,杜预谓其有“马癖”。故事称其好马射而买地作埒,为便于纵横驰骋,马场之大,可以想象。此马场若是设在荒僻的农村,地广人稀,尽可供其驰骋。但王济的马场,是设在人口密集、寸土寸金的京师洛阳贵族聚居的北邙山下,情况则大不相同。其所开销,已非一般贵族所能承受。但为显豪富,满足夸胜于时的心理,竟然“编钱匝地竟埒”,编串金钱竟绕场一周,生怕人家没看见,人们以此称之为“金沟”——金光炫耀于外,精神败絮其中。“金沟”之谓,非颂而实讥也,惜哉王济不醒。}

\lettrine{30.10} 石崇\myidx{石崇}每与王敦\myidx{王敦}入学戏\footnote{入学戏:到学校游玩。每:常。},见颜、原象{\fzxk\zihao{6}\textcolor{red}{\CJKunderwave{家语}曰:“颜回,字子渊,鲁人。少孔子二十九岁而发白,三十二岁早死。”原宪已见。}} 而叹曰\footnote{颜、原:指孔子七十弟子中的颜回和原宪。二人皆安贫乐道而好学。}:“若与同升孔堂\footnote{同升孔堂:一起成为孔子的学生。},去人何必有间\footnote{去人何必有间:和七十子之徒没有什么不同。}!”王曰:“不知馀人云何\footnote{馀人:其他人。云何:怎么样。},子贡去卿差近\footnote{子贡:端木赐字子贡,孔子高徒。其人能言词,善经商,富埒王侯。差近:差不多,相似。}。”{\fzxk\zihao{6}\textcolor{red}{\CJKunderwave{史记}曰:“端木赐,字子贡,卫人。尝相鲁,家累千金。终于齐。”}} 石正色云:“士当令身名俱泰\footnote{身名俱泰:地位名声全都亨通显达而安泰。},何至以瓮牖语人\footnote{何至:岂有。瓮牖:以破瓮为窗户,喻贫寒。语见\CJKunderwave{庄子·让王}:“原宪居鲁,环堵之室,茨以生草,蓬户不完,桑以为枢而瓮牖。”}!”{\fzxk\zihao{6}\textcolor{red}{原宪以瓮为户牖。}}

{\cangkai\zihao{5}【评】王敦讥石崇,只知追求富贵而不知道德学问,怎能与七十子之徒同升孔子之堂呢?石则“正色”作答,极其认真,“身名俱泰”是个理想境界,做人应是名声地位和金钱学问两不误,臻达通泰之境,以便享受人生。只知安贫乐道,困窘瓮牖之下,学而不达,完全不合时宜。石崇所言,正是当时骄奢汰侈之风流行的指导思想。今之学人,经商致富,从政掌权,身名腾达,列于世界名人排行榜之上,其追求似近石崇,而与颜回、原宪之学所追求的精神世界背道而驰。孰优孰劣,何去何从,读者自辨。}

\lettrine{30.11} 彭城王\myidx{司马权}有快牛\footnote{彭城王:指司马权,馗子。馗为司马懿弟。则权与师、昭为嫡堂兄弟。按:余嘉锡引程炎震曰:“权薨于咸宁元年,衍才二十岁。此彭城王,未必定是权。”程说疑是。此彭城王疑为权子植,其年辈与衍相仿。},至爱惜之。{\fzxk\zihao{6}\textcolor{red}{朱凤\CJKunderwave{晋书}曰:“彭城穆王权,字子舆,宣帝弟馗子。太始元年封。”}} 王太尉\myidx{王衍}与射\footnote{王太尉:王衍,曾官太尉,故称。参前\CJKunderwave{言语}第23则注。射:此指博射,魏晋人士的一种赌博游戏,而与实战的兵射不同。},赌得之。彭城王曰:“君欲自乘,则不论;若欲啖者\footnote{啖:吃。},当以二十肥者代之。既不废啖,又存所爱\footnote{存:保全。}。”王遂杀啖\footnote{王遂杀啖:王衍终于把快牛杀了吃掉。}。

{\cangkai\zihao{5}【评】彭城王之快牛,大概与王恺的八百里駮相似,都是人类精心培育的优良种牛。射赌输却,自然无话可说。但若当肉牛吃掉,则与常牛无异,令人顿生爱惜之心,故彭城愿以二十肥牛易之。从经济效益来说,赢者大为合算。但王衍却弃物质不论,坚持“杀啖”而毁名牛,这不仅是对于爱牛者的严重心理打击,论其实际效果,更是对人类文明的挑战。名士之“名”,以残害文明获取,岂不悲哉!}

\lettrine{30.12} 王右军\myidx{王羲之}少时\footnote{王右军:王羲之。},在周侯\myidx{周顗}末坐\footnote{周侯:周顗字伯仁,爵武城侯。东晋中兴名臣。参前\CJKunderwave{言语}第30则注。末坐:座席末位。},割牛心啖之\footnote{牛心:此指牛心炙。},于此改观\footnote{于此改观:从此改变了对羲之的看法。}。{\fzxk\zihao{6}\textcolor{red}{俗以牛心为贵,故羲之先食之。}}

{\cangkai\zihao{5}【评】据\CJKunderwave{晋书}羲之本传,称其“年十三,尝谒周顗。时重牛心炙。坐客未吃,顗先割牛心啖羲之。于是始知名。”周顗此举,一是因为羲之年少,对孩子优先照顾,以示不拘礼节;一是因为他真正喜欢这个孩子。但是,这些都属于正常,为何收入\CJKunderwave{汰侈}门呢?张万起、刘尚慈评云:“大约并非食之先后问题,而是特意宰牛割心给他吃,以此奢侈之举提高他的地位和名望。”所论不无道理,但若改入\CJKunderwave{赏誉}或\CJKunderwave{识鉴}门,则更为妥帖。}





%%% Local Variables:
%%% mode: latex
%%% TeX-engine: xetex
%%% TeX-master: "../Main"
%%% End:
