%% -*- coding: utf-8 -*-
%% Time-stamp: <Chen Wang: 2025-11-25 23:52:37>

% ○ ◎ ‧ 「 」 『 』 々 ( ) “ ”
% 【\([^】][^】][^】]+\)】 → {\\fzxk\\zihao{6}\\textcolor{red}{\1}}
% \(【评】.*\) → {\\cangkai\\zihao{5}\1}
% \(【题解】.*\) → {\\cangkai\\zihao{5}\1}
% 《\([^》]+\)》 → \\CJKunderwave{\1}
% ^\([0-9]+.[0-9]+\) → \\lettrine{\1}
%  [[0-9]+]

\setlength{\parindent}{0pt}

\chapter{言语第二}

{\cangkai\zihao{5}【题解】 “言语”位列第二,循孔门四科之例。可在\CJKunderwave{世说}这里,已是旧瓶装新酒。在孔门,其言语之杰,宰我、子贡所标志的,言语是为政事服务的工具,注重的是春秋时行人辞令之大用,“利口巧辞”,出使四方,长于专对,不辱使命。一句话,孔门的言语之教,是将其作为从政的利器去重视的。而\CJKunderwave{世说新语}已经淡化了前述作用,它标志的是魏晋才士的风貌、神采。一门“言语”,精妙纷呈:或玄虚简远,言近而旨遥;或巧慧机敏,辩俊而味长;或深蕴学养,儒雅而切理;或感物精审,片言而入微……每一谈吐,都摇曳着说话人的智趣才情,展现着他们那自由的精神,独立的人格,深湛的学养,鲜明的个性,表达着辞令艺术的成熟。这里的108则各具面貌,然而皆隽永精彩,风神奕奕。}

\lettrine{2.1} 边文礼\myidx{边让}见袁奉高\myidx{袁阆}\footnote{边文礼:边让,字文礼,汉末陈留人。有才名,能辞赋,甚得当时名士推重,官至九江太守。建安中,为曹操所杀。袁奉高:即袁阆。刘注作“闳”,误。袁奉高:袁阆字奉高,汉末汝南慎阳人。刘注引\CJKunderwave{汝南先贤传}作“袁闳”,余嘉锡\CJKunderwave{笺疏}证其误。按:袁宏字夏甫,袁安玄孙,史称安为汝阳人。},{\fzxk\zihao{6}\textcolor{red}{闳(阆)也。}} 失次序\footnote{失次序:谓举止失措,不得体。}。{\fzxk\zihao{6}\textcolor{red}{\CJKunderwave{文士传}曰:“边让字文礼,陈留人。才隽辩逸。大将军何进闻其名,召署令史,以礼见之。让占对闲雅,声气如流,坐客皆慕之。让出就曹,时孔融、王朗等并前为掾,共书刺从让,让平衡与交接。后为九江太守,为魏武帝所杀。”}} 奉高曰:“昔尧聘许由\myidx{许由},面无怍色\footnote{怍(zuò作):惭愧。}。{\fzxk\zihao{6}\textcolor{red}{皇甫谧曰:“由字武仲,阳城槐里人也。尧、舜皆师而学事焉。后隐于沛泽之中,尧乃致天下而让焉。由为人据义履方,邪席不坐,邪膳不食,闻尧让而去。其友巢父闻由为尧所让,以为污己,乃临池洗耳。池主怒曰:‘何以污我水?’由于是遁耕于中岳颍水之阳,箕山之下,终身无经天下色,死葬箕山之颠,在阳城之南十里。尧因就其墓,号曰箕山公神,以配食五岳,世世奉祀,至令(今)不绝也。”}} 先生何为颠倒衣裳\footnote{颠倒衣裳:语出\CJKunderwave{诗经·齐风·东方未明}:“东方未明,颠倒衣裳。颠之倒之,自公召之。”衣,上衣;裳,下裳。诗谓情急之中,将衣、裳倒置,无法穿上,手忙脚乱。此借为匆忙失措。}?”文礼答曰:“明府初临\footnote{明府:“明府君”的省称。汉代人称郡太守为“府君”或“明府君”。余嘉锡注引程炎震语“汉时吏民通称守相为明府”。袁奉高为太尉掾,并未出守过汝南或陈留,此“明府”之称,不过是当时吏民对地位高的长官之习称。},尧德未彰\footnote{尧德:尧之谦让美德。彰:显现。},是以贱民颠倒衣裳耳!”{\fzxk\zihao{6}\textcolor{red}{按袁闳卒于太尉掾,未尝为汝南,斯说谬矣。}}

{\cangkai\zihao{5}【评】袁奉高为太尉掾,身当要津,有荐才之贤;边文礼“心性通达,口辩辞长”(\CJKunderwave{后汉书·边让传}),正怀才求沽,所以一见袁奉高,生怕失去机会,心理压力之大,竟至于无所措手足,乱了阵脚,亦在情理之中。史称边让“少(年少时)辩博”,本则记其在情急之中亦显出“占对闲雅,声气如流”的才干,一席对答,顺势而出,于当下情景中既解释了“失次序”的缘由,又表白了对袁的崇敬,确是活画出了边让的口才和机智。此情此景,急湍转流,展演了才士的风采。

刘辰翁认为袁出言以尧聘许由自况是“不足道”的,准确揭示了这位权高势重的贵族之狂妄。与边的闲雅应对相比,袁阆相形见绌,露出其迂拙、浅俗。这也印证了郭泰对他的评论:“奉高之器,譬诸氾滥,虽清而易挹。”(\CJKunderwave{后汉书·黄宪传})}

\lettrine{2.2} 徐孺子\myidx{徐穉}{\fzxk\zihao{6}\textcolor{red}{穉也。}} 年九岁\footnote{徐孺子:徐穉字孺子,汉末隐者,有“南州高士”之称。},尝月下戏。人语之曰:“若令月中无物\footnote{令:使。月中物:传说月中的黑影为蟾蜍。},当极明邪\footnote{当:一定。}?”{\fzxk\zihao{6}\textcolor{red}{\CJKunderwave{五经通议}曰:“月中有兔、蟾蜍者何?月,阴也;蟾蜍,亦阴也,而与兔并明,阴系于阳也。”}} 徐曰:“不然,譬如人眼中有瞳子\footnote{瞳子:瞳仁。},无此必不明。”

{\cangkai\zihao{5}【评】此记徐孺子幼童时节,一段充满童趣的问对。前面一问,对孩童说来似显深奥,但却发人遐想;后面一答,“近取诸身,远取诸物”,类比思维,结论妙趣横生而又不无道理。一问一对,见出徐孺子的聪明机敏。正是这种敏于事的天资,使他后来能博通经史、洞达世情,在后汉晚期天下欲崩之际,看到“大树将颠,非一绳所维,何为栖栖不遑宁处”,于是,屡被推举征辟,皆不仕。隐居晏然,家贫而“常自耕稼,非其力不食。恭俭义让,所居服其德”,成为口碑很好的“南州高士”。(见\CJKunderwave{后汉书·徐穉传})}

\lettrine{2.3} 孔文举\myidx{孔融}{\fzxk\zihao{6}\textcolor{red}{融也。}} 年十岁\footnote{孔文举:孔融(153—208)字文举,鲁国(今山东曲阜)人,孔子二十代孙(\CJKunderwave{后汉书}),汉末著名文学家。献帝时为北海相,故人称“孔北海”,后任少府、太中大夫等官职。为人性刚直,喜直言,因触怒曹操为操所杀。},随父到洛。时李元礼\myidx{李膺}有盛名,为司隶校尉\footnote{李元礼(110—169):李膺字元礼,汉末颍川襄城(今属河南)人。在朝清议领袖之一,与杜密并称“李杜”。因反对宦官专政,被太学生称为“天下模楷”。后遭党锢之祸,死于狱中。司隶校尉,官名,掌管纠察京师百官及京师近郡犯法者,“秩比二千石”,官俸、职能,皆与郡太守相当。}。诣门者,皆隽才清称及中表亲戚乃通\footnote{诣(yì义):前往,到。隽才:才智出众的人。清称:有声望的人。中表:古代称父亲的姐妹(姑母)之子为外兄弟;称母亲的兄弟(舅父)姐妹(姨母)之子为内兄弟,外为表,内为中,合称“中表兄弟”。通:通报,引见。}。文举至门,谓吏曰:“我是李府君亲\footnote{李府君:司隶校尉与郡太守相当,因称李元礼为“李府君”。}。”既通,前坐。元礼问曰:“君与仆有何亲\footnote{仆:古代谦称,即指“我”。}?”对曰:“昔先君仲尼与君先人伯阳有师资之尊,是仆与君弈世为通好也\footnote{先君:先人,指祖先。仲尼:孔子名丘,字仲尼。春秋鲁国(今山东曲阜)人,儒家学派创始人,古代著名的思想家,教育家。伯阳:\CJKunderwave{史记}称,老子姓李,名耳,字伯阳。道家学派创始人,古代著名的思想家。师资:老师。\CJKunderwave{史记}载孔子曾向老子请教问“礼”,故此称“有师资之尊”。弈世:累世,代代。通好:交好,交谊深厚。汉魏以师友为通家。}。”元礼及宾客莫不奇之。太中大夫陈韪\myidx{陈韪}后至\footnote{太中大夫:官名,掌议论,秩比千石。陈韪(wěi委):\CJKunderwave{后汉书}作陈炜,生平不详。},人以其语语之。韪曰:“小时了了\footnote{了了:聪明。},大未必佳。”文举曰:“想君小时,必当了了。”韪大踧踖\footnote{踧踖(cùjí促急):局促不安的样子。}。{\fzxk\zihao{6}\textcolor{red}{\CJKunderwave{续汉书}曰:“孔融字文举,鲁国人,孔子二十四(二十四,当为二十)世孙也。高祖父尚,钜鹿太守。父宙,泰山都尉。”\CJKunderwave{融别传}曰:“融四岁,与元(兄)食梨,辄尉小者。人问其故,答曰:“小儿,法当取小者。”年十岁,随父诣京师。河南君(尹)李膺有重名,融欲观其为人,遂造之。膺问:“高明父祖尝与仆周旋乎?”融曰:“然。先君孔子与君先人李老君同德比义,而相师友,则融与君累世通家也。”众坐莫不叹息,佥曰:“异童子也!”太中大夫陈韪后至,同坐以告,韪曰:“人小时了了者,长大未必能奇。”融应声曰:“即如所言,君之幼时,岂实慧乎?”膺大笑,顾谓融曰:“长大必为伟器。”}}

{\cangkai\zihao{5}【评】在汉末,孔融以刚直敏慧、道德文章而名播天下。本则故事说明,他的刚直敏慧与口才在少年时代就条发颖竖,有大过人之处,确乎令人拍案称奇。

文中的李元礼是个大人物,名重当时,一言可使人跃登龙门,身价百倍。十岁的孔融想到要拜访他,一则是慕名,二则也未尝不存有请其提拂,获得声名的愿望。能有这样想法的少年本已是早慧,而一席妙语,不但有出人意表的机巧,更见出他勤学的底蕴,所以出语即征服了举座名士,博得满堂叹赏。这片段真可谓精彩至极,以至跨越千载,仍馀馨不竭。

可是自恃聪明,也会误事。接下来的一个片段,同样精彩,但正如凌濛初评价:“机锋太迅,太自佳,惟不免祸耳。”机敏迅捷,然而不留馀地,逼得人“大踧踖”,这便是忘了祸从口出而自取其祸的根苗了。后来孔融被杀,原因复杂,但其性格惯性也是导火索。\CJKunderwave{后汉书}本传说他多次弄得曹操尴尬,令其“数不能堪”。有一显例,很能说明这个问题:击败了袁绍的曹操,听任儿子曹丕将袁绍貌美的儿媳据为己有,孔融当面嘲讽这事:“武王伐纣,以妲己赐周公。”曹操问“出何经典”,融对曰:“以今度之,想当然耳。”这一说便把曹氏欺世盗名,又掠人妻女的丑行彰露无遗。以此,一旦有人故意给孔融构罪,曹操一旦抓了把柄,就毫不犹豫地将这位大名士满门抄斩了。}

\lettrine{2.4} 孔文举\myidx{孔融}有二子,大者六岁,小者五岁。昼日父眠\footnote{昼日:白天。},小者床头盗酒饮之\footnote{盗:偷。}。大儿谓曰:“何以不拜?”答曰:“偷,那得行礼\footnote{那得:怎能。}!”

{\cangkai\zihao{5}【评】这里与本门第九则内容雷同,刘辰翁说:“后锺毓、锺会事同,疑只一事,讹而二之。后者是。”无论孰为本事,孰为讹增,这种辗转流传皆可看作是魏晋饮酒风气的反映。名士健饮,所以家家备酒,这由孔融名言“坐上客恒满,尊(樽)中酒不空,吾无忧矣”(\CJKunderwave{后汉书·孔融传})可见一斑。既然备酒,又有家君的楷模,便不由得不发生小儿好奇,模仿大人饮酒的事情。小儿乘大人不防而偷酒喝,这事本来就滑稽有趣,再加上这场景中的如此对话,便更显得伶俐可爱。“酒以成礼”,综观\CJKunderwave{仪礼},每一庄重礼仪,皆有酒赞其成,酒与礼实不可分。在\CJKunderwave{诗}、\CJKunderwave{礼}传家的士大夫门庭,孩童自有濡染,知酒为行礼之物,所以出现了这番对话。

王世懋说本则:“可称‘夙慧’,未足当‘言语’。”或许小儿对话,用老成人腔调,越显机巧可爱,因而编入此门。}

\lettrine{2.5} 孔融\myidx{孔融}被收,中外惶怖\footnote{孔融被收:曹操因对孔融“积嫌忌”而早有灭融之心,这时,与融旧有不睦的御史大夫郗虑又“构成其罪”,曹操遂责令他的军谋祭酒路粹周纳成文,“枉状奏融”,终以“谤讪朝廷”等罪名,逮捕孔融,下狱弃市,“妻子皆被诛”。事在汉献帝建安十三年。收:逮捕。中外:朝廷内外。惶怖:惶恐。}。时融儿大者九岁,小者八岁,二儿故琢钉戏\footnote{故:依旧,照样。琢钉戏:余嘉锡注引周亮工\CJKunderwave{因树屋书影}说明此戏,是古代用钉为玩具的一种儿童游戏。},了无遽容\footnote{了无:毫无,完全没有。遽容:恐惧的脸色。}。融谓使者曰:“冀罪止于身\footnote{冀:希望。},二儿可得全不?”儿徐进曰:“大人岂见覆巢之下,复有完卵乎\footnote{大人:对长辈的敬称。覆巢:打翻的鸟巢。}?”寻亦收至\footnote{寻:不久,随即。}。{\fzxk\zihao{6}\textcolor{red}{\CJKunderwave{魏氏春秋}曰:“融对孙权使有讪谤之言,坐弃市。二子方八岁、九岁。融见收,弈棋端坐不起。左右曰:‘而父见执。’二子曰:‘安有巢毁而卵不破者哉?’遂俱见杀。”\CJKunderwave{世语}曰:“魏太祖以岁俭禁酒,融谓‘酒以成礼,不宜禁’,由是惑众,太祖收法焉。二子髫龀,见收,顾谓二子曰:‘何以不辟?’二子曰:‘父尚如此,复何所辟!’”裴松之以为\CJKunderwave{世语}云融儿不辟,知必俱死,犹差可安,孙盛之言,诚所未譬。八岁小儿,能悬了祸患,聪明特达,卓然既远,则其忧乐之情,固亦有过成人矣!安有见父被执,而无变容,弈棋不起,若在暇豫者乎?昔申生就命,言不忘父,不以己之将死,而废念父之情也。父安尚犹若兹,而况颠沛哉!盛以此为美谈,无乃贼夫人之子与!盖由好奇情多,而不知言之伤理也。}}

{\cangkai\zihao{5}【评】孔融十六岁即与边让等名士“齐声称”,后以刚正不阿,敢于面对贪官势力和凶恶的宦官阵营,揭发其罪恶而名重天下,几乎成为清流士人的代表。连大将军何进、穷凶极恶的董卓也不敢轻易对之加害。随献帝迁都许昌,孔融仍在朝廷引领议论,可见其声望地位。此番曹操真的将他逮捕,朝廷内外怎不震惊、恐惧。而在这种情形下,融二幼子却安之若素,照旧游戏。这并非是小儿因无知而无畏,恰恰相反,他们心如明镜。一句透辟的比喻,形象地描绘了孩子从容面对破家灭族的灾难,说明了政治的严酷。这精妙深刻的比喻成为活跃千古的经典语句。}

\lettrine{2.6} 颍川太守髡陈仲弓\myidx{陈寔}\footnote{髡(kūn昆):古代剃去头发的刑罚。陈寔(104—187)字仲弓,汉末颍川许昌(今属河南)人。曾任太丘长,故云。其治政清明,百姓安业,以公正直名闻世。时人评云:“宁为刑罚所加,不为陈君所短。”党锢祸起,自请系狱。卒时远近赴吊,刊石立碑,谥文范。荀朗陵:荀淑曾任郎陵侯相,故云。此云“颍川太守髡陈仲弓”,考诸史传,未见陈仲弓被髡事。仲弓曾两次被捕,一是县吏疑其杀人而逮捕,后“考掠无实”,很快就被放了出来;二是党锢祸起,他为伸张士人的正气,令天下对正气有所依归,便自己“请囚”入狱,又“遇赦得出”。本则之说,程炎震认为,大约是因前面被囚的事情而“增饰之”,即或本诸曾有之事而传闻异辞,未见得如孝标所说的齐东野语,纯是胡编滥造。}。{\fzxk\zihao{6}\textcolor{red}{案寔之在乡里,州郡有疑狱不能决者,皆将诣寔。或到而情首,或中途改辞,或托狂悸,皆曰:“宁为刑戮所苦,不为陈君所非。”岂有盛德感人,若斯之甚,而不自卫,反招刑辟,殆不然乎?此所谓东野之言耳!}} 客有问元方\myidx{陈纪}\footnote{元方:陈纪,字元方。陈寔长子。以“至德”扬名,正直敢言,最终官拜大鸿胪。}:“府君何如?”元方曰:“高明之君也。”“足下家君何如\footnote{家君:父亲的尊称。}?”曰:“忠臣孝子也。”客曰:“易称:‘二人同心,其利断金;同心之言,其臭如兰\footnote{“二人同心”句:语出\CJKunderwave{周易·系辞上}。谓二人能同心同德,其力量之锐利,可以切断金属;同心相应的话,其气味如同香兰一样可意动人。臭,嗅的本字。}。’{\fzxk\zihao{6}\textcolor{red}{王廙注\CJKunderwave{系辞}曰:“金至坚矣,同心者其利无不入。兰芳物也,无不乐者。言其同心者,物无不乐也。”}} 何有高明之君,而刑忠臣孝子者乎\footnote{刑:此用如动词,用刑之意。}?”元方曰:“足下言何其谬也!故不相答。”客曰:“足下但因伛为恭而不能答\footnote{因伛为恭:伛(yǔ与),脊梁弯曲之病,即驼背。此句谓,病伛的人不得不弯腰弓背,貌似恭敬,其实内心并不然。}。”元方曰:“昔高宗\myidx{武丁}放孝子孝己\footnote{放:放逐。}{\fzxk\zihao{6}\textcolor{red}{,\CJKunderwave{帝王世纪}曰:“殷高宗武丁有贤子孝己,其母早死,高宗惑后妻之言,放之而死,天下哀之。”}} 尹吉甫\myidx{尹吉甫}放孝子伯奇,{\fzxk\zihao{6}\textcolor{red}{\CJKunderwave{琴操}曰:“尹吉甫,周卿也,有子伯奇,母死,更娶,后妻生子曰伯邽,乃谮伯奇于吉甫。于是放伯奇于野。宣王出游,吉甫从,伯奇乃作歌,以言感之。宣王闻之,曰:‘此孝子之辞也!’吉甫乃求伯奇于野,而射杀后妻。”}} 董仲舒\myidx{董仲舒}放孝子符起\footnote{董仲舒:汉代大儒。}。{\fzxk\zihao{6}\textcolor{red}{未详。}} 唯此三君,高明之君;唯此三子,忠臣孝子。”客惭而退。

{\cangkai\zihao{5}【评】\CJKunderwave{周易·系辞上}说:“言行,君子之枢机,枢机之发,荣辱之主也。言行,君子之所以动天地也,可不慎乎?”言语辞说表现了一个人的判断力,也说明着论辩水平,在价值天平上,确实关乎荣辱。这里主客之辩,就表明了两人判断力的不同,荣辱之高下。客拘于常识、习惯性思维,以高明之人必办高明之事,很少犯错误为出发点,并据此而凌厉其论辩攻势,甚而断言,陈纪方犹病伛者,拿驼背当恭敬,其实是无法做答。机锋劲健,似乎逼对手到了死地。可对手的确比“客”高明。他跳出这种单一、极端的思维方式,以更深刻的社会认识、更开阔的眼界为依据,如凌濛初言:“刑辟之招,政未必不在盛德。世途足慰,是非何常。”尤其在汉末纷纭复杂的政治环境下,单纯的常识已不能保证一个人的判断力了,对于亲历汉末政治环境磨砺的士人来说,这个道理很简单,所以陈纪方先是不屑做答,后被纠缠,不得不沛然以对。他的这番话,就常识说,高明之人也难免有过错,而更深一层是说,生活中的“是非何常”是影响人正常判断的更复杂的因素。这就比“客”的思维方式更深刻、更具广角性,其判断水平也就不言而喻了。他的回答也映照出了“客”的浅薄轻率,“客”只能“惭而退”。客之败是由较低层次的判断力所支撑的论辩水平,自取其辱,理所当然。也由此可见,当时所崇尚的“言语”,是以智慧、学识为根基的口才,于是士人的形象便显得更深刻、更灵动,也更富有魅力。}

\lettrine{2.7} 荀慈明\myidx{荀爽}与汝南袁阆\myidx{袁阆}相见\footnote{荀慈明:荀爽,字慈明,荀淑第六子,当世硕儒。慈明:荀爽(128—190)字,一名谞,淑第六子。幼而好学,早通经传,征辟不应。在荀淑八子中,人称“荀氏八龙,慈明无双”。著\CJKunderwave{诗传}、\CJKunderwave{易传}等。后官至司空。与司徒王允谋诛董卓,事未行而病卒。行酒:巡行劝酒。行,汉魏时常用语,犹赐也,即按客一一分送物品。袁阆,字奉高,东汉汝南人,官至太尉掾,袁奉高:袁阆字奉高,汉末汝南慎阳人。刘注引\CJKunderwave{汝南先贤传}作“袁闳”,余嘉锡\CJKunderwave{笺疏}证其误。按:袁宏字夏甫,袁安玄孙,史称安为汝阳人。},{\fzxk\zihao{6}\textcolor{red}{荀爽一名谞。\CJKunderwave{汉南纪}曰:“谞文章典籍无不涉,时人谚曰:‘荀氏八龙,慈明无双。’潜处笃志,征聘无所就。”张璠\CJKunderwave{汉纪}曰:“董卓秉政,复征爽,爽欲遁去,吏持之急,起布衣,九十五日而至三公。”}} 问颍川人士\footnote{颍川:魏晋郡名,治所在许昌(今属河南)。汉魏之际,当地人才济济,居中原之冠。如荀淑、陈寔等家族,均以德行著称,为人师表而图画百城。},慈明先及诸兄\footnote{及:谈及。}。阆笑曰:“士但可因亲旧而已乎\footnote{但:只、仅。亲旧:亲戚朋友。此句意谓,推举士人,不可仅仅因为是亲朋故旧。}?”慈明曰:“足下相难,依据者何因\footnote{相难:责问。因:余嘉锡校订作“经”。}?”阆曰:“方问国士而及诸兄,是以尤之耳\footnote{尤:责怪、指责。}!”慈明曰:“昔者祁奚\myidx{祁奚}内举不失其子,外举不失其雠,以为至公\footnote{至公:最公正。}。{\fzxk\zihao{6}\textcolor{red}{\CJKunderwave{春秋传}曰:“祁奚为中军,请老,晋侯问嗣焉。称解狐,其雠也,将立之而卒。又问焉,对曰:‘午也可。’其子也。君子谓祁奚可谓能举善矣。称其雠,不为谄;立其子,不为比。”}} 公旦\myidx{周公}\CJKunderwave{文王}之诗,不论尧、舜之德而颂文、武者,亲亲之义也\footnote{公旦:即周公,姓姬名旦,周文王子,武王弟。助武王开国建立周朝,后辅佐幼主成王,为周初政治家。文王:此指\CJKunderwave{诗经·大雅·文王之什},十篇作品,皆颂美文王、武王的至德、业绩。亲亲之义:亲近亲族是仁义的根本。\CJKunderwave{礼记·中庸}:“仁者,人也,亲亲为大。”\CJKunderwave{孟子·尽心上}:“亲亲,仁也。”}。\CJKunderwave{春秋}之义,内其国而外诸夏\footnote{\CJKunderwave{春秋}:我国最早的一部编年体史书,亦为儒家经典之一。传为孔子据鲁国史书删述而成。内其国:内,亲近。国,自己的国家。外:疏远。诸夏:周天子所分封的其他诸侯国。\CJKunderwave{公羊传·成公十五年}:“\CJKunderwave{春秋}内其国而外诸夏,内诸夏而外夷狄。”}。且不爱其亲而爱他人者,不为悖德乎\footnote{“且不”句:\CJKunderwave{孝经·圣治章}:“父子之道,天性也,君臣之义也。父母生之,续莫大焉,君亲临之,厚莫重焉。故不爱其亲而爱他人者,谓之悖德;不敬其亲而敬他人者,谓之悖理。”悖,违背。}?”

{\cangkai\zihao{5}【评】袁阆问颍川“国士”,荀爽竟把自己的哥哥们给抬举了一通。对此,不仅袁阆,怕是所有的俗士都会瞠目结舌,口议腹诽。于是有了荀爽的辩难,他句句援引经典,雄强有力,无可辩驳。文章也戛然而止,想必袁阆已无辞以对。但荀爽的硕学与口才恐还不是入载本书\CJKunderwave{言语}门的真正理由。汉武帝“罢黜百家,独尊儒术”,并以此选官,诱以利禄之途,儒家经典遂成汉代最大的学问,但也同时沦为“干禄”之具,流衍下来,章句之学遂成风气,经学渐渐变得枯朽沉闷,作为名缰利锁而戕害着人性,读书人株守“师法”、“家法”而皓首穷经,“说五字之文,至于二三万言,……幼童守一艺,白首而后能言;安其所习,毁所不见,终以自蔽”(\CJKunderwave{汉书·艺文志})。此习渐至汉末,愈演愈烈。荀爽之论把儒经原本生动存在的人性成分,入情入理地拈入当下情景。一席对答扫尽腐儒的章句之习,表现了傲然自视的鲜明个性,在当时是震撼人心的。这种人性、个性的展露,正是他辩难中闪耀着光彩的地方,同时开魏晋士人风气之先,所以入载本书而耀其光华。}

\lettrine{2.8} 祢衡\myidx{祢衡}被魏武\myidx{曹操}谪为鼓吏\footnote{祢衡:汉末名士,博闻强记,擅长文章,但“尚气刚傲,好矫时慢物”,逆曹操,忤荆州牧刘表,触怒江夏守黄祖,终为祖所杀。参见刘孝标注,及\CJKunderwave{后汉书}本传。魏武,即曹操,其子曹丕代汉称帝,追尊操为太祖武皇帝。鼓吏,掌鼓的小吏。},正月半试鼓\footnote{正月半:正月十五。},衡扬枹为\CJKunderwave{渔阳掺挝}\footnote{扬:举起。枹(fú浮):鼓槌。渔阳掺挝(càn zhuā惨抓),鼓谱名。},渊渊有金石声\footnote{渊渊:形容鼓声。\CJKunderwave{诗经·小雅·采芑}“伐鼓渊渊”。金石声:钟磬类乐器发出的铿锵、清越之声。},四座为之改容\footnote{改容:脸色改变,指感动。}。{\fzxk\zihao{6}\textcolor{red}{\CJKunderwave{典略}曰:“衡字正平,平原般人也。”\CJKunderwave{文士传}曰:“衡,不知先所出,逸才飘举。少与孔融作尔汝之交,时衡未满二十,融已五十,敬衡才秀,共结殷勤,不能相违。以建安初比游,或劝其诣京师贵游者,衡怀一刺,遂至漫灭,竟无所诣。融数与武帝笺,称其才,帝倾心欲见,衡称疾不肯往,而数有言论。帝甚忿之,以其才名不杀,图欲辱之,乃令录为鼓吏。后至八月朝会大阅试鼓节,作三重阁,列坐宾客。以帛绢制衣,作一岑牟、一单绞及小裈。鼓吏度者,皆当脱其故衣,箸此新衣。次传衡,衡击鼓为\CJKunderwave{渔阳掺挝},蹋地来前,蹑馺脚足,容态不常,鼓声甚悲,音节殊妙。坐客莫不忼慨,知必衡也。既度,不肯易衣。吏呵之曰:‘鼓吏,何独不易服?’衡便止,当武帝前,先脱裈,次脱馀衣,裸身而立,徐徐乃箸岑牟,次箸单绞,后乃箸裈。毕,复击鼓,掺槌(挝)而去,颜色无怍。武帝笑谓四坐曰:‘本欲辱衡,衡反辱孤。’至今有\CJKunderwave{渔阳掺挝},自衡造也。为黄祖所杀。”}} 孔融\myidx{孔融}曰:“祢衡罪同胥靡,不能发明王之梦\footnote{胥靡:古代服劳役的刑徒。相传商王武丁夜梦得圣人,便使人依梦中形象图画出来,命四处寻找,终于在傅岩找到了正在服劳役的傅说(yuè月),遂举以为相,因而出现了殷商的中兴。}。”{\fzxk\zihao{6}\textcolor{red}{皇甫谧\CJKunderwave{帝王世纪}曰:“武丁梦天赐己贤人,使百工写其像,求诸天下。见筑者胥靡衣褐于傅岩之野,是谓傅说。”张晏曰:“胥靡,刑名。胥,相也。靡,从也。谓相从坐轻刑也。”}} 魏武惭而赦之。

{\cangkai\zihao{5}【评】支撑着人生信念的聪慧和才情,构成了狂士自信、自怜的性格基础。他们更在乎的是自己英发的生命品性,给生命过程带来的如霞光彩,因“淑质贞亮,英才卓砾……思若有神”而傲视一切俗物。祢衡便是这一群体的典型代表。他“始达颍川,乃阴怀一刺,既无所之适,至于刺字漫灭”(\CJKunderwave{后汉书·祢衡传})。年轻的才士,无视于人才济济、高明往来的颍川衣冠,竟觉得没哪一个值得他趋门拜访。这与汉末栖栖遑遑奔走权门,企仰高明品题奖掖的士风,形成了鲜明的对照。这也就毫不奇怪,曹操谪其为鼓吏,辱之同伶优,他仍以超越俗辈的表演令人改容,展露奇才而傲视这玩天下于股掌的当世雄豪。以祢衡的聪明,他何尝不懂得在曹操面前针锋挑战的后果,但他要表达的是自我生命的飞扬闪光,不同凡响。谁都别想让他逊顺,他就是要以聪慧、才情和个性,张扬他生命的价值。在这里,孔融的辞说可谓与祢衡异曲同工。在谦和、恭顺的比喻辞面之下,又藏着一个尖锐的对比——祢衡才同傅说,而曹操“明王”之“明”无法比肩武丁,因此就不要梦想,得而驾驭如此贤才了。两个难茹难吐的芒刺,直逼得曹操以羞“惭”作罢。}

\lettrine{2.9} 南郡庞士元\myidx{庞统}闻司马德操\myidx{司马徽}在颍川\footnote{南郡:郡名,治所在今湖北江陵。庞士元:见刘孝标注。又,士元清识善谈,有谋略,深为刘备器重,与诸葛亮并为军师中郎将。随刘备入蜀,出计策,收益州,后攻洛城中流矢,卒。司马德操:见刘孝标注,汉末名士。颍川:郡名,治所在许昌。},故二千里候之\footnote{故:特意。候:拜访。}。至,遇德操采桑,士元从车中谓曰:“吾闻丈夫处世,当带金佩紫,焉有屈洪流之量,而执丝妇之事\footnote{带金佩紫:金指金印,紫指紫绶。汉代三公、将军、列侯带金印,佩紫绶。此指高官显爵。洪流之量:指气概非凡。洪流,大水;量,气度。丝妇:妇女做的采桑养蚕之事。}?”{\fzxk\zihao{6}\textcolor{red}{\CJKunderwave{蜀志}曰:“庞统字士元,襄阳人。少时朴钝未有识者。颍州司马徽有知人之鉴,士元弱冠往见徽,徽采桑树上,坐士元树下,共语自昼至夜。徽异之,曰:‘生当为南州士人之冠冕。’由是渐显。”\CJKunderwave{襄阳记}曰:“士元,德公之从子也。年少,未有识者,唯德公重之。年十八,使往见德操,与语,叹曰:‘德公诚知人,实盛德也!’后刘备访燕(世)事于德操,德操曰:‘俗士岂识时务,此间自有伏龙、凤鶵。’谓诸葛孔明与士元也。”\CJKunderwave{华阳国志}曰:“刘备引士元为军帅(师)中郎将,从攻洛,为流矢所中,卒,时年三十八。”}} 德操曰:{\fzxk\zihao{6}\textcolor{red}{\CJKunderwave{司马徽别传}曰:“徽字德操,颍川阳翟人。有人伦鉴识。居荆州,知刘表性暗,必害善人,乃括囊不谈议。时人有以人物门(问)徽者,初不辨其高下,每辄言‘佳’。其妇谏曰:‘人质所疑,君宜辩论,而一皆言“佳”,岂人所以咨君之意乎!’徽曰:‘如君所言,亦复佳。’其婉约逊遁如此。尝有妄认徽猪者,便推与之,后得其猪,叩头来还,徽又厚辞谢之。刘表子琮往候徽,遣问在不。会徽自锄园。琮左右问:‘司马君在耶?’徽曰:‘我是也。’琮左右见其丑陋,骂曰:‘死庸!将军诸郎欲求见司马君,汝何等用(田)奴,而自称是邪!’徽归,刈头箸帻出见琮。左右见徽,故是向老翁,恐,向琮道之。琮起,叩头辞谢。徽乃谓曰:‘卿真不可。然吾甚羞之,此自锄园,唯卿知之耳。’有人临蚕求蔟箔者,徽自弃其蚕而与之。或曰:‘凡人损己以赡人者,谓彼急我缓也,今彼此正等,何为与人?’徽曰:‘人未尝求己,求之不与,将惭。何有以财物令人惭者?’人谓刘表曰:‘司马德操,奇士也,但未遇耳。’表后见之,曰:‘世间人为妄语,此直小书生耳!’其智而能愚皆此类。荆州破,为曹操所得,操欲大用,会其病死。”}} “子且下车。子适知邪径之速\footnote{适:仅仅,只。邪径:小路,便道。},不虑失道之迷。昔伯成\myidx{伯成}耦耕\footnote{伯成耦耕:见刘孝标注,伯成不慕诸侯之荣,安于耕种。耦耕,二人并排耕作,此泛指耕田。},不慕诸侯之荣;{\fzxk\zihao{6}\textcolor{red}{\CJKunderwave{庄子}曰:“尧治天下,伯成子高立为诸侯。禹为天子,伯成辞诸侯而耕于野。禹往见之,趍就下风而问焉,子高曰:‘昔尧治天下,不赏而民劝,不罚而民畏。今子赏罚而民且不仁,德自此衰,刑自此立。夫子盍行邪?毋落吾事!’”}} 原宪\myidx{原宪}桑枢,不易有官之宅\footnote{原宪:见刘孝标注。桑枢:用桑木作门轴,喻简陋贫寒。}。{\fzxk\zihao{6}\textcolor{red}{\CJKunderwave{家语}曰:“原宪字子思,宋人,孔子弟子。居鲁,环堵之室,茨以生草,蓬户不完,桑枢而瓮牖,上漏下湿,坐而弦歌。子贡轩车不容巷,往见之,曰:‘先生何病也?’宪曰:‘宪闻无财谓之贫,学而不能行谓之病。今宪贫也,非病也。夫希世而行,比周而友。学以为人,教以为己。仁义之慝,舆马之饰,宪不忍为也。’”}} 何有坐则华屋,行则肥马,侍女数十,然后为奇?此乃许\myidx{许由}、父\myidx{巢父}\footnote{许、父:即许由、巢父,皆尧时的隐士。尧让君位给许由,由不受,又召其为九州长,由不欲闻,洗耳于颍滨。巢父饮牛,见其洗耳,问其故,而云:“污我犊口。”牵牛上流而饮。见皇甫谧\CJKunderwave{高士传}。}{\fzxk\zihao{6}\textcolor{red}{许由、巢父。}} 所以慷慨,夷、齐所以长叹\footnote{夷、齐:伯夷、叔齐。商末孤竹君二子,反对武王伐纣,认为是以暴易暴,非行仁义,耻食周菽,饿死于首阳山。}。{\fzxk\zihao{6}\textcolor{red}{\CJKunderwave{孟子}曰:“伯夷、叔齐目不视恶色,耳不听恶声,与乡人居,若在涂炭。盖圣人之清也。”}} 虽有窃秦之爵,千驷之富\footnote{窃秦之爵:指吕不韦。千驷之富:指春秋时齐景公。见刘孝标注。},{\fzxk\zihao{6}\textcolor{red}{\CJKunderwave{古史考}曰:“吕不韦为秦子楚行千金货于华阳夫人,请立子楚为嗣。及子楚立,封不韦洛阳十万户,号文信侯。”以诈获爵,故曰窃也。\CJKunderwave{论语}曰:“齐景公有马千驷,民无德而称焉。”孔安国曰:“千驷,四千匹。”}} 不足贵也。”士元曰:“仆生出边垂,寡见大义,若不一叩洪钟、伐雷鼓,则不识其音响也\footnote{边垂:即边陲,边疆,边远之地。伐:敲击。雷鼓:古祀天用的响鼓。}!”

{\cangkai\zihao{5}【评】余嘉锡认为:以士元通家子的身份和其人伦修养,不应安坐车中呼而语,对德操如此不恭,且出言鄙陋,又:“观其问答,盖仿(扬雄)\CJKunderwave{客难}、\CJKunderwave{解嘲}之体,特缩大篇为短章耳。此必晋代文士所拟作,非事实也。”就此篇内容和形制看,余说诚为的论。然而,虽可看作晋人拟作,篇中的精神却真实地反映着当时士人的风采。德操有雅望贤能而从容耕桑,此旌表隐士在浊世、浊政中急流勇退,全人格,远祸患,表现出超凡的智慧。而能躬行践履,又需有特立不移的操守。其言其行,确为“洪钟”、“雷鼓”。从中也见出司马徽战国才士般的论辩雄风。}

\lettrine{2.10} 刘公幹\myidx{刘桢}以失敬罹罪\footnote{刘公幹:刘桢,字公幹,东平(今属山东)人。当时著名诗人,建安七子之一。曾任曹操丞相掾属。罹罪:获罪。事见刘孝标注。},{\fzxk\zihao{6}\textcolor{red}{\CJKunderwave{典略}曰:“刘桢字公幹,东平宁阳人。建安十六年,世子为五官中郎将,妙选文学,使桢随侍世子。酒酣坐欢,乃使夫人甄氏出拜,坐上客多伏,而桢独平视。他日公闻,乃收桢,减死,输作部。”\CJKunderwave{文士传}曰:“桢性辩捷,所问应声而答。坐平视甄夫人,配输作部,使磨石。武帝至尚方观作者,见桢匡坐正色磨石。武帝问曰:‘石何如?’桢因得喻己自理,跪而对曰:‘石出荆山悬岩之巅,外有五色之文,内含卞氏之珍。磨之不加莹,雕之不增文,禀气坚贞,受之自然。顾其理枉屈纡绕而不得申。’帝顾左右大笑,即日赦之。”}} 文帝\myidx{曹丕}问曰:“卿何以不谨于文宪\footnote{文宪:法律。}?”桢答曰:“臣诚庸短,亦由陛下纲目不踈\footnote{庸短:平庸短浅。纲目:王先谦本作“网目”,指法网。不踈:即太密。踈,“疏”的异体字。}。”{\fzxk\zihao{6}\textcolor{red}{\CJKunderwave{魏志}曰:“帝讳丕,字子桓,受汉禅。”案:诸书或(咸)云桢被刑魏武之世,建安二十[二]年病亡。后七年文帝乃即位,而谓桢得罪黄初之时,谬矣。}}

{\cangkai\zihao{5}【评】刘桢其人,在\CJKunderwave{诗品}里列在当时最杰出的诗人曹植之后。之所以如此,就是特推重他作品的“骨气”。锺嵘评其诗:“仗气爱奇,动多振绝,真骨凌霜,高风跨俗。”此等风骨和才气,表达着一种不能无端屈挠的才士品格。其才性使之在举座当中独敢“平视”曹丕夫人,凌濛初叹曰:“平视自佳。”刘辰翁评点:“狂宜有此。”后答曹丕之问,愈见出其骨气。丕发问的主题词是“文宪”,公幹婉转其词,退中反刺,等于一是冒犯太子尊颜,二是婉刺曹氏多科条,密网令人损性。其言虽为申辩,实则反驳,篇中旨趣,便是对个人尊严与价值的顽强护卫。一个场景,活画出刘公幹辩捷巧对的才士风情。}

\lettrine{2.11} 锺毓\myidx{锺毓}、锺会\myidx{锺会}少有令誉\footnote{锺毓、锺会:魏锺繇二子,颍川长社人。毓,字稚叔,官至廷尉、青州刺史,督徐州、荆州军事,死后追赠车骑将军,谥惠侯。会,字士季,官至司徒。受命伐蜀,蜀破,欲率军谋反,内部先乱,为乱军所杀。魏以谋反论其罪。令誉:美好的声誉。},{\fzxk\zihao{6}\textcolor{red}{\CJKunderwave{魏书}曰:“毓字稚叔,颍川长社人,相国繇长子也。年十四,为散骑侍郎,机捷谈笑有父风,仕至车骑将军。”}} 年十三,魏文帝\myidx{曹丕}闻之,语其父锺繇\myidx{锺繇}\footnote{锺繇:字元常,汉末举孝廉为郎,多才艺,是当时著名书法家。其为官勤谨,历侍魏武、文帝、明帝,官至太尉,死谥成侯。}{\fzxk\zihao{6}\textcolor{red}{\CJKunderwave{魏志}曰:“繇字元常,家贫好学,为\CJKunderwave{周易}、\CJKunderwave{老子}训。历大理、相国,迁太傅。”}} 曰:“可令二子来。”于是敕见\footnote{敕(chì赤)见:皇帝下令晋见。}。毓面有汗,帝曰:“卿面何以汗?”毓对曰:“战战惶惶,汗出如浆\footnote{战战惶惶:发抖恐惧的样子。浆:水。}。”复问会:“卿何以不汗?”对曰:“战战栗栗\footnote{栗栗:义同“战战”。},汗不敢出。”

{\cangkai\zihao{5}【评】裴松之\CJKunderwave{三国志}卷二十八注引锺会为其母所作的\CJKunderwave{传}云:“黄初六年,生会。”\CJKunderwave{三国志}文帝本传载,丕于黄初七年去世,是文帝去世,会始两岁。又,\CJKunderwave{三国志}锺毓本传说,毓年十四为散骑侍郎,“太和初,蜀相诸葛亮围祁山,明帝欲西征,毓上疏……”太和初即文帝去世次年。则文帝时,毓或及十三岁而会远未及。此条云,毓、会十三见文帝,后注\CJKunderwave{世说}者颇疑“此条记年皆有误”。(朱铸禹\CJKunderwave{世说新语汇校评注})

显然,如果据史考证,则本条客观事实的真实性是大可怀疑的,然而,作为笔记小说的艺术构思,却反映了那一时代风尚的真实性。依据传闻,撮凑这样一则故事,并去欣赏、沉潜于其中的言语巧对、人物风貌,是当时那种唯美心理和珍视生命精神的真实体现。况且,故事生动描摹出两个少年性情、品格之不同,不失为一则意味隽永的美文。

同是出生于书香之家,同沐浴着锺繇儒雅练达的父风,可二子性情差异甚大。毓“机捷谈笑,有父风”(\CJKunderwave{三国志·锺毓传}),一生严谨、忠诚,历仕明帝、齐王芳、曹髦、元帝朝,屡上疏匡谏,甚至不惜得罪炙手可热的大将军曹爽,表现了对司马集团的耿耿之忠。这则故事里的“战战惶惶,汗出如浆”,如点睛一样,活现出其人朴实而机捷的性格特征。锺会则不然,机捷善辩但不诚实,失之于浅浮。就生理而言,汗不似泪,可以主观控制。真的惶惧,汗是不由自主的,绝非“敢”与“不敢”。“战战栗栗,汗不敢出”,孩童的这一谎言,由于机巧敏捷而貌似玲珑可爱,可它却映射着其人的性格走向。后来的锺会,多巧智事人,以“精练策数”、多谋而显名,时比之于汉初智谋之士张良,人谓为“子房”。可他在深谙世情处却远逊于子房,所以破蜀后盲目自信,“我自淮南以来,画无遗策,四海所共知也”(\CJKunderwave{三国志·锺会传}),欲仗恃自己的智谋和十几万大军,废掉早已在朝廷内外经营成熟、盘根错节、实力雄厚、老于计谋的司马氏,终于以“谋反”罪殒命。其实,从小看八十,锺会晚年的悲剧,早已在少年时代埋下了根苗。}

\lettrine{2.12} 锺毓兄弟\myidx{锺毓}\myidx{锺会}小时,值父\myidx{锺繇}昼寝,因共偷服药酒\footnote{药酒:\CJKunderwave{北堂书钞}卷八十五\CJKunderwave{续谈助}引\CJKunderwave{小说}作“散酒”。徐震堮谓:“‘散’即‘五石散’之类。”(见\CJKunderwave{世说新语校笺})}。其父时觉,且托寐以观之\footnote{时觉:当时即醒。寐:装睡。}。毓拜而后饮,会饮而不拜。{\fzxk\zihao{6}\textcolor{red}{\CJKunderwave{魏志}曰:“会字士季,繇少子也。敏惠夙成。中护军蒋济著论,谓‘观其眸子,足以知人’。会年五岁,繇遣见济,济甚异之,曰:‘非常人也!’及壮,有才数,精练名理,累迁黄门侍郎。诸葛诞反,文王征之,会谋居多,时人谓之‘子房’。拜镇西将军,伐蜀。蜀平,进位司徒。自谓功名盖世,不可复为人下,谓所亲曰:‘我淮南已来,画无遗策,四海共知,将此欲安归乎?’遂谋反,见诛,时年四十。”}} 既而,问毓何以拜?毓曰:“酒以成礼,不敢不拜。”又问会何以不拜?会曰:“偷本非礼,所以不拜。”

{\cangkai\zihao{5}【评】关于这一则故事,余嘉锡谓“此与本篇孔文举二子盗酒事略同。盖即一事而传闻异辞”。性质同前篇一样,或属传闻,但虽事同一辙,其摹写人物却皆符合各自的性格。本篇写一拜一不拜,参见前篇,都映现出毓、会兄弟二人的不同秉性,所以也不失真实、生动,不能完全看作同一模本的传闻。锺繇这样的名流,家中常备药酒,从中可见魏晋饮酒、服药风习之一斑。}

\lettrine{2.13} 魏明帝\myidx{曹叡}为外祖母筑馆于甄氏\footnote{魏明帝:曹叡,曹丕之子,魏第二代君主,死谥明皇帝。外祖母:文帝甄皇后之母。明帝即位,追封外祖父甄逸,谥敬侯,令其嫡孙甄象袭爵。甄氏有侯府之尊。筑馆于甄氏:即在明帝舅氏甄府建造馆舍。},{\fzxk\zihao{6}\textcolor{red}{\CJKunderwave{魏末传}曰:“帝讳叡,字元仲,文帝太子,以甘(其)母废,未立为嗣。文帝与俱猎,见子母鹿,文帝射其母,应弦而倒。复令帝射其子,帝置弓,泣曰:‘陛下已杀其母,臣不忍复杀其子。’文帝曰:‘好语动人心!’遂定为嗣,是为明帝。”\CJKunderwave{魏书}曰:“文昭甄皇后,明帝母也。父逸,上蔡令。烈宗即位,追封上蔡君。嫡孙象袭爵,象薨,子畅嗣,起大第,车驾亲自临之。”}} 既成,自行视,谓左右曰:“馆当以何为名?”侍中\footnote{侍中:官名,侍从皇帝左右,出入禁宫,应对顾问。}缪袭\myidx{缪袭}曰:{\fzxk\zihao{6}\textcolor{red}{\CJKunderwave{文帝(章)叙录}曰:“袭字熙伯,东海兰陵人。有才学,累迁侍中、光禄勋。”}} “陛下圣思齐于哲王,罔极过于曾\myidx{曾参}、闵\myidx{闵子骞}\footnote{圣思:圣明的想法。齐:等同。哲王:圣明的君主。罔极:没有穷尽。\CJKunderwave{诗经·小雅·蓼莪}叙孝子思亲报德,历说父母养育之恩及自己无从报答的痛苦。诗中有“欲报之德,昊天罔极”,后多用“罔极”隐含子女大孝。曾、闵:即孔子的弟子曾参和闵子骞,均以孝行著称。\CJKunderwave{史记·仲尼弟子列传}载,孔子因曾参能通孝道而收为门徒,其孝行动人,并说\CJKunderwave{孝经}即是他所作。\CJKunderwave{论语·先进}载,孔子曾赞叹闵子骞之孝:“孝哉闵子骞!人不间于其父母昆弟之言。”};此馆之兴,情钟舅氏,宜以‘渭阳’为名\footnote{渭阳:用\CJKunderwave{诗经·秦风·渭阳}典,秦康公见舅思母,送别舅氏晋文公,诗曰:“我送舅氏,曰至渭阳。何以赠之,路车乘黄。我送舅氏,悠悠我思。何以赠之,琼瑰佩玉。”}。”{\fzxk\zihao{6}\textcolor{red}{\CJKunderwave{秦诗}曰:“\CJKunderwave{渭阳},康公念母也。康公之母,晋献公之女。文公遭骊姬之难,未反而秦姬卒。穆公纳文公,康公时为太子,赠送文公于渭之阳,念母之不见也。我见舅氏,如母存焉。”案\CJKunderwave{魏书},帝于后园为象母起观,名其里曰“渭阳”,然则象母即帝之舅母,非外祖母也。且“渭阳”为馆名,亦乖旧史也。}}

{\cangkai\zihao{5}【评】魏明帝母子可谓“患难母子”。其母甄氏,曹丕得之于袁熙,以貌美而恩宠有加,生曹叡和东乡公主。后曹丕登基,有山阳公奉送二女,又有诸贵人,丕恩幸别移,甄氏因失宠而生怨艾,竟“遣使赐死”。其祸殃及嫡长子曹叡,迟迟不能立嗣。裴松之\CJKunderwave{三国志}注引\CJKunderwave{魏略}说,文帝曾一度欲立他姬子为太子。经历了这般变故,母子之情非同于一般。所以,曹叡一登基就追谥其母为“文昭皇后”,并将外公、舅氏一并抬举。其中不仅仅是秉承旧制,深怀那段患难也是重要因素。此番筑馆于舅氏之家,个中背景深为侍中缪袭所理解,他的话不同于一般的阿谀奉承,而是带有动人的真意。其为馆所命之名,实亦妥帖。凌濛初评曰:“‘渭阳’一名,侍中腹笥可测。”\CJKunderwave{渭阳}小诗,平易朴素,情深意挚,见舅思母,馀味绵长,契合曹叡思亲情景,从而引发丰富的联想。故事无论是否有“乖旧史”,这一描写,于情于理都令人有真实之感,侍中命名之妙也如画龙点睛。}

\lettrine{2.14} 何平叔\myidx{何晏}云\footnote{何平叔:何晏字平叔,\CJKunderwave{三国志}作何进孙。少有才,正始初为曹爽所用,名盛于天下。好老庄,与夏侯玄、王弼等倡导玄学,开魏晋清谈之风。}:“服五石散,非唯治病,亦觉神明开朗\footnote{五石散:由石钟乳、硫黄、白石英、紫石英、赤石脂五种矿物药研成的粉末。因服后身体烦热,必须“寒衣、寒饮、寒食、寒卧,极寒益善”又称“寒食散”。神明:指人的精神。开朗:爽朗。}。”{\fzxk\zihao{6}\textcolor{red}{\CJKunderwave{魏略}曰:“何晏字平叔,南阳宛人,汉大将军进孙也。或云何苗孙也。尚主,又好色,故黄初时无所仕。正始中,曹爽用为中书,主选举,宿旧者多得济拔。为司马宣王所诛。”秦丞相(秦承祖)\CJKunderwave{寒食散论}曰:“寒食散之方,虽出汉代,而用之者寡,靡有传焉。魏尚书何晏首获神效,由是大行于世,服者相寻也。”}}

{\cangkai\zihao{5}【评】何晏“少以才秀知名”(\CJKunderwave{三国志}本传),读书甚多,著述也颇丰。作为当时的思想家,他性好老庄,立言以体“无”为本,与天才少年王弼一起,倡树了旨在通识达变,个人体悟的正始之音、玄学之风。这一风尚的现实意义,就是冲决了以往最庄严的儒家传统名教的道德准则,淡化了群体——社会认同的价值标准,而崇尚个人精神的独立和自由。在生活上,曹操因纳其母而收养他,自小长养在宫中,“见宠如公子”,后又“尚公主”,成了驸马爷,风流落拓,其性“无所顾惮”,我行我素。这便形成了何晏颇具个性的精神气质,此则之言,便是这种气质的生动表达。他有条件去研究、尝试“五石散”,也有兴趣去体认服用“五石散”给人带来的气血畅旺的异常感受,于是出此自得之言。他开魏晋玄风之先,也是服药之风的始作俑者。}

\lettrine{2.15} 嵇中散\myidx{嵇康}语赵景真\myidx{赵至}\footnote{嵇中散:嵇康(223—262):三国时谯郡铚 (今安徽亳县)人。“竹林七贤”之一。曾任中散大夫,故称嵇中散。当时著名思想家、文学家、清谈名家。因其主张越名教而任自然,抨击礼法之士,不与司马氏统治集团合作,盛年被杀。赵景真:赵至字景真,代郡(今河北)人,有才学,与嵇康相熟。晋太康中以良吏入洛都,知母亡,呕血而卒,年三十七。}:{\fzxk\zihao{6}\textcolor{red}{嵇绍\CJKunderwave{赵至叙}曰:“至字景真,代郡人。汉末,其祖流宕,客缑氏。令新之官,至年十二,与母共道傍看。母曰:‘汝先世非微贱家也,汝后能如此不?’至曰:‘可尔耳!’归便就师诵书。早闻父耕叱牛声,释书而泣。师问之,答曰:‘自伤不能致荣华,而使老父不免勤苦。’年十四,入太学观,时先君在学写石经古文,事讫,去,遂随车问先君姓名。先君曰:‘年少何以问我?’至曰:‘观君风器非常,故问耳。’先君具告之。至年十五,佯病,数数狂走五里三里,为家追得。又灸身体十数处。年十六,遂亡命。径至洛阳,求索先君不得。至邺,沛国史仲和,是魏领军史涣孙也,至便依之,遂名翼,字阳和。先君到邺,至具道太学中事,便逐先君归山阳,经年。至长七尺三寸,洁白,黑发,赤唇、明目,鬓(须)不多,闲详安谛,体若不胜衣。先君常谓之曰:‘卿头小而锐,瞳子白黑分明,视瞻停谛,有白起风。’至论议清辩,有从横才,然亦不以自长也。孟元基辟为辽东从事,在郡断九狱,见称清当。自痛弃亲远游,母亡不见,吐血发病,服未竟而亡。”}} “卿瞳子白黑分明,有白起\myidx{白起}之风\footnote{白起:战国末年秦国名将,屡败韩、魏、赵、楚等大军,拔七十馀城,封武安君。后被秦昭王赐死。恨:遗憾。}。{\fzxk\zihao{6}\textcolor{red}{严尤\CJKunderwave{三将叙}曰:“白起,平原君劝赵孝成王受冯亭,王曰:‘受之,秦兵必至,武安君必将,谁能当之者乎?’对曰:‘渑池之会,臣察武安君小头而面锐,瞳子白黑分明,视瞻不转。小头而面锐者,敢断决也;瞳子白黑分明者,见事明也;视瞻不转者,执志强也。可与持久,难与争锋。廉颇为人,勇鸷而爱士,知难而忍耻。与之野战则不如,持守足以当之。’王从其计。”}} 恨量小狭。”赵云:“尺表能审玑衡之度\footnote{尺表:尺许长,用来测日影以记时的标杆。审:测定。玑衡:观测四时天象的仪器。},{\fzxk\zihao{6}\textcolor{red}{\CJKunderwave{周髀}曰:“夏至北方六千里,冬至南方十三万五千里,日中树表则无影矣。周髀长八尺,夏至,日晷尺六寸。髀,股也。晷,勾也。正南千里,勾尺五寸;正北千里,勾尺七寸。”\CJKunderwave{周髀}之书也。}} 寸管能测往复之气\footnote{寸管:九寸竹管,用于测定音律。}。{\fzxk\zihao{6}\textcolor{red}{\CJKunderwave{吕氏春秋}曰:“黄帝使伶伦自大夏之西,昆仑之阴,取竹之嶰谷生,其窍厚薄均者,断两节间而吹之,以为黄钟之管,制十二筩,以听凤凰之鸣;雄鸣六,雌亦六,以为律吕。”\CJKunderwave{续汉书·律历志}曰:“十二律之变,至于六十,以律候气。候气之法,为室三重,户闭,涂衅必周,密布缇幔,以木为案,加律其上,以葭莩灰抑其内,为气所动者,其灰散也。以此候之。”}} 何必在大,但问识如何耳\footnote{但:只,仅。}。”

{\cangkai\zihao{5}【评】刘辰翁评曰:“本语量狭,文采支离,可恨尔。”辰翁觉得,以器物之长短来说心胸气量之大小,不伦不类,缺乏正常的逻辑关联,辞面便支离破碎,无统一流畅之美。其实,在此语境,两人皆心照不宣,又恰显出景真“议论精辩,有纵横才气”。(\CJKunderwave{晋书·赵至传})

司马迁评白起:“鄙谚云:‘尺有所短,寸有所长。’白起料敌合变,出奇无穷,声震天下,然不能救患于应侯。”白起天才将军,长于横扫天下,短于应对朝廷内部的争斗。终为应侯范雎所算,死于非命;因此,太史公以“鄙谚”的尺寸短长之喻,来慨叹惋惜这位天才。嵇康以白起比况景真,景真脱口而答,既见敏捷精辩才气,又见学问修养。其气量之句,承太史公言而来,又将尺寸长短,转入“尺表”、“寸管”功能,并将这一转落在“识”上。孔子曾把“学”与“思”结合起来,同等强调,“学”而“思”,得之即为“识”。这是学问修养的不二法门。可是至汉代的经学末流,将“学”庸俗为利禄之具,便忽略了“识”——独立思考,自成见识的重要性,这便忽略了个体人格的独立与价值。讨论至此,一扫汉代腐儒积习,以“识”为高标。在景真看来,只有“识”才是一个人的价值准的,“量”倒在其次,所以,有白起之风,虽乏气量,并不失为具有风流雅望的天才。然而,景真确实心胸不宽,经不住未尽奉养之孝而母亡又未及面别的打击,呕血而卒。他也有白起才长量狭的特点。由此看来,嵇康识人的确深刻。}

\lettrine{2.16} 司马景王\myidx{司马师}东征\footnote{司马景王:即司马师,见刘孝标注。东征:魏高贵乡公正元二年(255)镇东大将军毌丘俭与扬州刺史文钦,矫太后诏,历数大将军司马景王罪状,举兵反于淮南。司马景王统大军征讨。},{\fzxk\zihao{6}\textcolor{red}{\CJKunderwave{魏书}曰:“司马师字子元,相国宣文侯长子也。以道德清粹,重于朝廷,为大将军,录尚书事。毌丘俭反,师自征之。薨,谥景王。”}} 取上党李喜\myidx{李喜}以为从事中郎\footnote{上党:郡名,秦置,治所在壶关(今山西长治北)。李喜:\CJKunderwave{晋书}本传作李憙。从事中郎:魏晋时三公、郡王、州郡所置之佐吏。}。因问喜曰:“昔先公辟君不就,今孤召君,何以来\footnote{先公:指司马懿。辟:征召。孤:王侯自称。}?”喜对曰:“先公以礼见待,故得以礼进退;明公以法见绳,喜畏法而至耳\footnote{见待:对待我。绳:约束、整治。见绳,约束我。}。”{\fzxk\zihao{6}\textcolor{red}{\CJKunderwave{晋诸公赞}曰:“喜字季和,上党铜鞮人也。少有高行,研精艺学。宣帝为相国,辟喜,喜固辞疾。景帝辅政,为从事中郎,累迁光禄大夫、特进,赠太保。”}}

{\cangkai\zihao{5}【评】徐震堮\CJKunderwave{世说新语校笺}引\CJKunderwave{后汉书·周黄徐姜申屠列传}事,与此事相类。显宗与荀恁戏曰:“先帝征君不至,骠骑辟君而来,何也?”对曰:“先帝秉德以惠下,故臣可得不来。骠骑执法以检下,故臣不敢不至。”\CJKunderwave{晋书·宣帝纪}载:曹操为司空时,征辟司马懿,懿“知汉运方微,不欲屈曹氏”,称疾谢绝,后曹操为丞相,再征,并命使者:“若复盘桓,便收之”,懿“惧而就职”。此类出处为封建专制时代常情。专制利剑之下,臣子原无个体自由可言。

本则故事,看似调侃幽默的问对,实际表现了在下者的无奈,故隽永而耐人寻味。}

\lettrine{2.17} 邓艾\myidx{邓艾}口吃,语称“艾艾”\footnote{邓艾:见刘孝标注。又\CJKunderwave{三国志}本传载,邓艾不仅能征善战,而且有战略眼光,司马懿大兴军屯就颇多接受邓艾的建议,他自己驻军处“荒野开辟,军民并丰”,这些业绩使他成为在曹魏当时,贡献、影响最大的人物之一。但平蜀之后,他“深自矜伐”,不从司马文王之命,最后实际上是为文王所害。艾艾:古时同别人说话时称自己的名,以示谦恭,邓艾口吃,所以别人听来,迭称艾艾。}。{\fzxk\zihao{6}\textcolor{red}{\CJKunderwave{魏志}曰:“艾字士载,棘阳人。少为农人养犊。年十二,随母至颍川,读故太丘长碑,文曰:‘言为世范,行为士则。’遂名范,字士则,后宗族有同者,故改焉。每见高山大泽,辄规度指画军营处所,时人多笑焉。后见司马宣帝(王),王辟为掾。累迁征西将军,伐蜀。蜀平,进位太尉。为卫瓘所害。”}} 晋文王\myidx{司马昭}戏之曰\footnote{晋文王:即司马昭,封晋王,死谥文王。}:“卿云‘艾艾’,定是几艾?”对曰:“凤兮凤兮,故是一凤\footnote{凤兮:\CJKunderwave{论语·微子}:“楚狂接舆歌而过孔子曰:‘凤兮,凤兮!何德之衰!往者不可谏,来者犹可追。已而已而,今之从政者殆而!”故:本来。}。”{\fzxk\zihao{6}\textcolor{red}{朱凤\CJKunderwave{晋纪}曰:“文王讳昭,字子上,宣帝次子也。”\CJKunderwave{列仙传}曰:“陆通者,楚狂接舆也,好养性,游诸名山,尝遇孔子而歌曰:‘凤兮凤兮,何德之衰!往者不可谏,来者犹可追。’后入蜀,在峨嵋山中也。”}}

{\cangkai\zihao{5}【评】本则记趣,谐谑幽默。口吃的邓艾,面对文王问,竟能一口气,流畅地说出了完整的句子,并且巧妙使用了\CJKunderwave{论语}故实,确实满纸风趣,令人欢然而笑,同时又儒雅不俗,王世懋赞曰:“仓卒对,乃妙绝。”}

\lettrine{2.18} 嵇中散\myidx{嵇康}既被诛\footnote{嵇中散:即嵇康,魏曹奂景元四年(263),因锺会构陷,被司马昭所杀。},向子期\myidx{向秀}举郡计入洛\footnote{向子期:见刘孝标注。举郡计:被郡守选任为上计吏。计,即上计吏。汉魏时考核地方官吏的做法,年终,县将本地方的户口、垦田、钱、粮、盗贼、狱讼等事统计汇编为计簿,上报至郡;郡再汇编成计簿由郡守遣吏将副本上报朝廷。郡级执行此任务的人员皆称“上计吏”,或简称“计吏”。洛:洛阳,魏都城。},文王\myidx{司马昭}引进\footnote{文王:即司马昭。},问曰:“闻君有箕山之志,何以在此\footnote{箕山之志:隐居不仕的志向。箕山,相传为尧时许由隐居之地,后以喻隐居不仕。}?”对曰:“巢\myidx{巢父}、许\myidx{许由}狷介之士,不足多慕\footnote{巢、许:巢父、许由。狷介:拘谨偏激。}。”王大咨嗟\footnote{咨嗟:赞叹,叹赏。}。{\fzxk\zihao{6}\textcolor{red}{\CJKunderwave{向秀别传}曰:“秀字子期,河内人。少为同郡山涛所知,又与谯国嵇康、东平吕安友善,并有拔俗之韵,其进止无固必‘固必’诸本作‘不同’),而造事营生业,亦不异常。与嵇康偶(耦)锻于洛邑,与吕安灌园于山阳。不虑家人有无,外物不足怫其心。弱冠著\CJKunderwave{儒道论},弃而不录。好事者或存之。或云:‘是其族人所作,困于不行,乃告秀,欲假其名。’笑曰:‘何复尔耳?’后康被诛,秀遂夫(失)图。乃应岁举,到京师,诣大将军司马文王,文王问曰:‘闻君有箕山之志,何能自屈?’秀曰:‘常谓彼人不达尧意,本非所慕也。’一坐皆悦。随次转至黄门侍郎、散骑常侍。”}}

{\cangkai\zihao{5}【评】刘辰翁评:“向之此语,如负叔夜。”朱铸禹案曰:“注引\CJKunderwave{别传}所云,向为人殆一无定见者流,其对司马氏之言‘彼人’,实隐指叔夜。若是,则大负流水之奏矣。”并谓嵇康枉引向秀为同调,秀本无操守,负友求荣。细按向子期,觉此论未必尽然。

秀为当时玄谈名家,深好老庄之学,解喻老庄,“发明奇趣,振起玄风,读之者超然心悟,莫不自足一时也”(\CJKunderwave{晋书·向秀传})。此秀之答司马昭,或可从其所悟玄理求之。向秀之\CJKunderwave{庄子}注,今大体亡佚,而郭象注俱存。据\CJKunderwave{晋书},郭象注是发挥了向秀见解的,“今有向、郭二\CJKunderwave{庄},其义一也”。郭象注\CJKunderwave{庄}的一个基本观念,即是强调“物任其性,事称其能,各当其分”。明于性分之适,脱却俗累,就会得到自然之真,而得到自然之真,就会“遗其所寄”。山林、庙堂就其形式说并无两样,心任自然,丢弃此外壳,即可获得自由境界。而“狷介”是太拘执了,于不可为之时,守持自己的价值观,张扬不苟同流俗的个性,这并非老庄的达观之境,从体悟玄理说,它是“不足多慕”的。从世俗层面上看,其客观效果确似投司马昭所好。然而,读向秀\CJKunderwave{思旧赋}和本传,他后来“在朝不任职,容迹而已”,则秀实非“无定见”,更非属卖友求荣者。此问对,就深层次说,是描绘了一段玄言风采。}

\lettrine{2.19} 晋武帝\myidx{司马炎}始登祚,探策得一\footnote{晋武帝:司马炎,(236—290)。刘仲雄(?—285):刘毅字仲雄。魏晋时东莱掖(今山东莱州市)人。官至尚书左仆射。性方正謇忠。曾当面讥晋武帝为汉之桓、灵二帝。主张废九品中正制度,未果。登祚:即皇帝位。祚,朝堂前东面的台阶。皇帝即位登祚阶以主持祭祀。探策:用蓍草占卜。}{\fzxk\zihao{6}\textcolor{red}{。\CJKunderwave{晋世谱}曰:“世祖讳炎,字安宇(世)。咸熙二年受魏禅。”}} 王者世数,系此多少\footnote{世数:王位传承世代的数目。系:取决于。}。帝既不悦,群臣失色,莫能有言者。侍中裴楷\myidx{裴楷}进曰\footnote{裴楷:裴令公:即裴楷,曾官中书令,故云,又称“裴令”。善\CJKunderwave{老}、\CJKunderwave{易},当时著名清谈名家。二国租钱:指从梁、赵二国税收所获钱财。}:“臣闻天得一以清,地得一以宁,侯王得一以为天下贞\footnote{“臣闻”三句:\CJKunderwave{老子}第三十九章:“天得一以清,地得一以宁,神得一以灵,万物得一以生,侯王得一以为天下贞。”一,即“道”。贞,正。}。”帝悦,群臣叹服。{\fzxk\zihao{6}\textcolor{red}{王弼\CJKunderwave{老子注}云:“一者,数之始,物之极也。各是一物之所以为主也。各以其一,致此清、宁、贞。”}}

{\cangkai\zihao{5}【评】王世懋评曰:“此故自应至此。魏篡汉无几而亡,晋篡魏亦应无几而亡。”探策得一,预兆晋祚不永,这使武帝殊感别扭,因而闹得一时气氛紧张。裴楷之论“取绝捷”,虽“供奉语”亦“不妨雅致”(王世贞),打破了“群臣失色”的僵局。

探策——\CJKunderwave{易}筮,以五十蓍草反复抽数推算才可得出一爻一卦。此推演“探策得一”,依汉儒象数\CJKunderwave{易}学的解释,\CJKunderwave{乾}卦位列第一,象征天、龙、帝,是尽善尽美的吉卦。\CJKunderwave{乾}卦六爻,初九爻第一。初九讲“潜龙勿用”,而此时的武帝,篡位登祚,已是“飞龙在天”。武帝问的又是“世数”,“得一”者,不是一年,就是一代。依象数\CJKunderwave{易},是大大的凶兆。但裴楷引玄入\CJKunderwave{易},以义理解之,说武帝是得道之君,可以使天下安宁。这一解释别开生面,解除了紧张气氛。虽不免谄谀之嫌,但其援\CJKunderwave{老}释\CJKunderwave{易},言趣相似,思理如一,表现了当时玄家义理的新观念,确是别具神采。他的急中生智,源于学问修养,在此情景,又见出这位名士的儒雅风度。}

\lettrine{2.20} 满奋\myidx{满奋}畏风\footnote{满奋:见刘孝标注。史有满奋身长八尺,体貌“丰肥,肤肉溃裂,每至暑夏,辄膏汗流溢”(见\CJKunderwave{三国志·满宠传}裴注;\CJKunderwave{太平御览}卷三七八)的记载,未详其何以“畏风”。}。在晋武帝\myidx{司马炎}坐,北窗作琉璃扇屏风\footnote{琉璃:一种有色而半透明的矿物制品,近似玻璃,汉代自西域传入。扇:指成板状、片状的材料。有本作“扉”,误。句谓晋武帝的北窗是用一块块琉璃镶嵌而成。},实密似疏,奋有难色。帝笑之。{\fzxk\zihao{6}\textcolor{red}{荀绰\CJKunderwave{冀州记}曰:“奋字武秋,高平人,魏太尉宠之孙也。性清平有识,自吏部郎出为冀州刺史。”\CJKunderwave{晋诸公赞}曰:“奋体量清雅,有曾(‘曾’字衍)祖宠之风,迁尚书令,为荀顗(‘荀顗’当为‘苗愿’之形讹)所害。”}} 奋答曰:“臣犹吴牛,见月而喘。”{\fzxk\zihao{6}\textcolor{red}{今之水牛,唯生江淮间,故谓之吴牛也。南土多暑,而此牛畏热,见月疑是日,所以见月则喘。}}

{\cangkai\zihao{5}【评】\CJKunderwave{事类赋}卷一引\CJKunderwave{风俗通}:“吴牛望见月则喘,使之苦于日月,怖而喘焉。”(见余嘉锡笺疏)该句后为成语“吴牛喘月”,比喻见到曾备受其苦的同类事物,就条件反射,本能地产生恐惧心理。晋武帝看着满奋对琉璃窗产生错觉,十分好笑,然而满奋很敏捷,见武帝笑,即时悟到自己的错觉,巧妙解嘲,显得幽默有趣。

王世懋推测:“盖奋厌职事烦剧,故有此言。”但味此情景,似不至阴阳曲折,深言事君。此可看作纯粹一喜剧场景,表达了时人所推崇的机敏巧对,故入\CJKunderwave{言语}门。}

\lettrine{2.21} 诸葛靓\myidx{诸葛靓}在吴\footnote{诸葛靓(jìnɡ竟):见刘孝标注。据\CJKunderwave{三国志·吴书·孙亮传}、\CJKunderwave{三国志·魏书·诸葛诞传}载,魏司空诸葛诞于甘露二年(257)五月,起兵反抗司马氏专权,遣子靓入吴为质以求援助。靓后仕吴,为右将军、大司马。吴亡,隐居不出。},于朝堂大会\footnote{朝堂:国君与朝臣聚会议事的厅堂。}。{\fzxk\zihao{6}\textcolor{red}{\CJKunderwave{晋诸公赞}曰:“靓字仲思,琅邪人,司空诞少子也。雅正有才望。诞以寿阳叛,遣靓入质于吴,以靓为右将军、大司马。”}} 孙皓(晧)\myidx{孙皓}问\footnote{孙皓(249—284):字元宗,孙权孙。皓,史作“晧”。初封乌程侯,吴景帝孙休死,晧继帝位,在位十七年。吴亡,降晋,封归命侯。}:“卿字仲思,为何所思?”对曰:“在家思孝,事君思忠,朋友思信,如斯而已\footnote{斯:此。}。”

{\cangkai\zihao{5}【评】这也是一段君臣对话,在孙晧是随意戏乐,而诸葛靓之答却自有深意。孙晧是荒唐君主,“粗暴骄盈,多忌讳”,滥事杀伐,虽功臣近宠,如有忤逆也毫不留情。亡国之际他自认:“历位数年,政教凶悖……虐毒横流。”(见\CJKunderwave{三国志·孙晧传}裴松之注)诸葛靓是外来之士,虽得重用,为吴建过功业,但面对如此反复无常的残暴君王,仍不能不谨慎戒惧。此答既巧妙地回应君问,又很好地表白了自己,强调其忠诚。片言只语,表达了他深藏块垒的兢兢畏忌。}

\lettrine{2.22} 蔡洪\myidx{蔡洪}{\fzxk\zihao{6}\textcolor{red}{洪\CJKunderwave{集录}曰:“洪字叔开,吴郡人,有才辩。初仕吴朝,太康中,本州从事举秀才。”王隐\CJKunderwave{晋书}曰:“洪仕至松滋令。”}} 赴洛\footnote{蔡洪:见刘孝标注。晋惠帝元康初为松滋令,有才名,著\CJKunderwave{孤奋论}。洛:洛阳,西晋都城。},洛中人问曰:“幕府初开,群公辟命,求英奇于仄陋,采贤隽于岩穴\footnote{幕府:本为军队出征时,将帅所用的帐幕,因称将军府为幕府,这里指军政大吏办理公务的衙署。群公:指有开府资格的朝廷大吏,其开府即可招引人才,以为衙署的属官。辟命:征召、任命。仄陋:同“侧陋”,指出身卑微。岩穴:本为山洞,此指隐居之所。}。君吴楚之士,亡国之馀,有何异才而应斯举\footnote{“吴楚”句:吴楚泛指江浙、两湖一带,是东吴旧地,吴亡,所以此谓“亡国之馀”,以蔑称灭亡之国的遗民。斯举:此次选拔人才的举措。}?”蔡答曰:“夜光之珠,不必出于孟津之河\footnote{夜光之珠:即随珠,亦作“隋珠”,见刘孝标注。孟津:古黄河的渡口名,此指黄河。};{\fzxk\zihao{6}\textcolor{red}{旧说云:随侯出行,有蛇斩而中断者,侯连而续之,蛇遂得生而去,后衔明月珠以报其德,光明照夜同昼,因曰“随珠”。左思\CJKunderwave{蜀都赋}所谓“随侯鄙其夜光”也。}} 盈握之璧,不必采于昆仑之山\footnote{盈握:满把,形容其大。昆仑之山:昆仑山,古代传说该山产美玉。}。{\fzxk\zihao{6}\textcolor{red}{韩氏曰:“和氏之璧,盖出于井里之中。”}} 大禹生于东夷,文王生于西羌\footnote{东夷:古代对东方各族的称呼,此泛指东方偏远之地。西羌:羌族在西,此指西方偏远之地。}。{\fzxk\zihao{6}\textcolor{red}{案\CJKunderwave{孟子}曰:“舜生于诸冯,东夷人也。文王生于岐周,西戎人也。”则东夷是舜,非禹矣。}} 圣贤所出,何必常处\footnote{常处:固定的地方。}。昔武王伐纣,迁顽民于洛邑\footnote{顽民:殷商遗民中不顺从周王朝统治的人。\CJKunderwave{尚书·多士}:“成周既成,迁殷顽民。”},{\fzxk\zihao{6}\textcolor{red}{\CJKunderwave{尚书}曰:“成周既成,迁殷顽民,作\CJKunderwave{多士}。”孔安国注曰:“殷大夫心不则德义之经,故徙于王都,迩教诲也。”}} 得无诸君是其苗裔乎\footnote{得无:莫非,或许。苗裔:后代子孙。}?”{\fzxk\zihao{6}\textcolor{red}{案:华令思举秀才入洛,与王武子相酬对,皆与此言不异,无容二人同有此辞。疑\CJKunderwave{世说}穿凿也。}}

{\cangkai\zihao{5}【评】此条,刘孝标颇疑其穿凿,余嘉锡\CJKunderwave{笺疏}亦推断其为华令思事。\CJKunderwave{晋书}卷五十二载华令思对王武子问,确与此条大同小异。\CJKunderwave{世说}在撰结过程中,或容有甲乙错置之误,但不妨碍其准确记录当时精神。

蔡洪,吴郡人;华令思,广陵人,皆为故吴国才士。吴亡,江东才俊纷纷入洛求仕,而中原士人傲视江东亡国遗民,其心理的顽固障碍,可以王武子之说为典型——“危而不持,颠而不扶,至于君臣失位,国亡无主”(\CJKunderwave{晋书·华谭传}),这是冠带之士的耻辱,“亡国之馀”更何谈辅弼之才干。因而,这班北方中原士人对奔赴而来的南士,报以蔑视和嘲讽。面对持有狭隘自傲心态的战胜者之挑衅,南土才士采取了迎头痛击的态度。人才之产,本无定处,而自傲的你们也难保不曾是“亡国之馀”,将双方拉到了同等地位,句句在理,痛快淋漓,精神爽健,辞辩巧妙,堪称“俊辩”。不过,口舌之辩,虽逞一时之快,但其中却隐含了南、北士族的不和与斗争,现状如此,国家力量内耗,不久即西晋灭亡。中原狂士,自己也不免于“亡国之馀”的命运,悲哉!}

\lettrine{2.23} 诸名士共至洛水戏\footnote{洛水:即今河南洛河,源出陕西,东经洛阳等地,至巩县(今巩义市)洛口入黄河。戏:嬉戏,游乐。},{\fzxk\zihao{6}\textcolor{red}{\CJKunderwave{竹林七贤论}曰:“王济诸人尝至洛水解禊事。明日,或问济曰:‘昨游,有何语议?’济云云。”}} 还,乐令\myidx{乐广}{\fzxk\zihao{6}\textcolor{red}{广也。}} 问王夷甫\myidx{王衍}曰\footnote{乐令:乐广,曾作尚书令,乐广(?—304):字彦辅,南阳淯阳(今河南南阳东南)人。少孤贫,寒素为业,与物无竞。其清谈析理,与王衍并称,卫瓘以为有正始遗风。官至尚书令,八王乱中,以故忧卒。王夷甫:王衍(256—311)字夷甫,见刘孝标注。“以清虚通理称”,为当时清谈名家,“妙悟若神”,“妙善玄言,唯谈\CJKunderwave{老}、\CJKunderwave{庄}为事”。为政多谋略,不以经国为念,而善思自全之计,然终为石勒所害。(见\CJKunderwave{晋书}本传)}:“今日戏乐乎?”{\fzxk\zihao{6}\textcolor{red}{虞预\CJKunderwave{晋书}曰:“王衍字夷甫,琅邪临沂人,司徒戎从弟。父乂,平北将军。夷甫早知名,以清虚通理称。仕至太尉,为石勒所害。”}} 王曰:“裴仆射\myidx{裴頠}善谈名理,混混有雅致\footnote{裴仆射(yè业):即裴頠(267—300),见刘孝标注。博学多才识,“时人谓頠为言谈之林薮”。撰\CJKunderwave{崇有论}以推尊儒术,崇扬礼法,贬斥何晏、王衍等言“无”之蔽。名理:考核名与实之关系,循名责实,辨别、分析事物的是非、道理。混混:滚滚,\CJKunderwave{孟子}:“原泉混混,不舍昼夜。”此形容言辞滔滔不绝。雅致:高雅的情致。};{\fzxk\zihao{6}\textcolor{red}{\CJKunderwave{晋惠帝起居注}曰:“裴頠字逸民,河东闻喜人,司空秀之少子也。”\CJKunderwave{冀州记}曰:“頠弘济有清识,稽古,善言名理,履行高整,自少知名。历侍中、尚书左仆射。为赵王伦所害。”}} 张茂先\myidx{张华}论\CJKunderwave{史}、\CJKunderwave{汉},靡靡可听\footnote{张茂先:即张华,范阳方城(今河北固安西北)人。博学多才,贯通今古,以诗赋文章称名于世。为晋武帝筹设灭吴方略,一统天下。惠帝时官至司空,死于八王之乱。\CJKunderwave{史}、\CJKunderwave{汉}:指\CJKunderwave{史记}、\CJKunderwave{汉书}。靡靡:娓娓,动听的样子。};{\fzxk\zihao{6}\textcolor{red}{\CJKunderwave{晋阳秋}曰:“华博览洽闻,无不贯综。世祖尝问汉事,及建章千门万户,华画地成图,应对如流,张安世不能过也。”}} 我与王安丰\myidx{王戎}{\fzxk\zihao{6}\textcolor{red}{戎也。}} 说延陵、子房,亦超超玄著\footnote{王安丰:即王戎(234—305):魏晋时琅邪人,王祥族人,当时清谈名士,“竹林七贤”之一。入晋官至尚书令、司徒。延陵:春秋时吴王寿梦的少子季札,封于延陵(今江苏武进),称延陵季子,博学有贤能。子房:即西汉张良,助高祖刘邦定天下,封为留侯。超超玄著:议论超拔高妙、深入透彻。}。”{\fzxk\zihao{6}\textcolor{red}{\CJKunderwave{晋诸公赞}曰:“夷甫好尚谈称,为时人物所祭(宗)。”}}

{\cangkai\zihao{5}【评】依刘孝标注,本则谈论诸名士于洛水修禊时的情况。

修禊,应劭\CJKunderwave{风俗通义}卷八引\CJKunderwave{周礼}、\CJKunderwave{尚书},述其为古俗。\CJKunderwave{诗经·郑风}韩诗说“三月桃花水下之时,郑国之俗,三月上巳,于溱、洧两水上,执芄兰招魂续魄,祓除不祥也”。“修禊”本是一桩春日除恶祭,沐浴祈福的颇为热闹的俗间事,到了魏晋,这一节日成了名士们的宴游沙龙。王羲之著名的\CJKunderwave{兰亭集序}说的就是永和九年(353),四十馀雅士名流,高会兰亭,曲水流觞,赋诗唱和,堪称一时盛会,而序文中叙写的感受与思考,其情怀悲天悯人,恳恻深挚,让人更清晰地看到了上巳节日士大夫们的一番情致。

本则所记虽早于王羲之的兰亭集会,但两者情趣风味却是一致的。

“朝贤上巳禊洛”(\CJKunderwave{晋书·王戎传}),这一活动自然将朝贤名士集会到了一起。这时,思想的碰撞、学识的交流,又构成了知识精英洛滨修禊活动的生动内容。思索天人之际,体味当下生命意味是他们都共同关心着的主题,因此一旦相遇,即从不同角度阐发己见,辩难推究。不论是形而上的名理,还是史家记述过往的人生经验都可成为话题。每个人都有自己的体会与高见,表达出来,各具风采,或“混混有雅致”,或“靡靡可听”,或“超超玄著”,在思理碰撞中相互启迪。如此动人的风采,源于他们的自信、自负,其自信、自负又源于学识、才干,于是各具个性的气质风貌就通过他们自己的言语讲论生动地展现出来了。}

\lettrine{2.24} 王武子\myidx{王济}、{\fzxk\zihao{6}\textcolor{red}{\CJKunderwave{晋诸公赞}曰:“王济字武子,太原晋阳人,司徒浑第二子也。有隽才,能清言。起家中书郎,终太仆。”}} 孙子荆\myidx{孙楚}\footnote{王武子:见刘孝标注。\CJKunderwave{晋书}卷四十二本传,言其善\CJKunderwave{易}、\CJKunderwave{老}、\CJKunderwave{庄},长于清谈,当时与和峤、裴楷齐名。娶晋武帝常山公主为妻,官历侍中、太仆。为人尚武有勇,性情豪爽。孙子荆:亦当时豪爽之士,见刘孝标注。\CJKunderwave{晋书}卷五十六本传,言其才藻卓绝,爽迈不群,多所陵傲,缺乡曲之誉。年四十馀始仕。与王济相知甚深。}{\fzxk\zihao{6}\textcolor{red}{\CJKunderwave{文士传}曰:“孙楚字子荆,太原中都人也。”\CJKunderwave{晋阳秋}曰:“楚,骠骑将军资之孙,南阳太守宏之子。乡人王济,豪俊公子,为本州大中正。访问宏(楚)为乡里品状,济曰:‘此人非乡评所能名,吾自状之。’曰:‘天才英特,亮拔不群。’仕至冯翊太守。”}} 各言其土地人物之美。王云:“其地坦而平,其水淡而清,其人廉且贞\footnote{淡:言水质好,清而甘甜。贞:正,正直。}。”孙云:“其山㠑巍以嵯峨,其水㳌渫而扬波,其人磊砢而英多\footnote{㠑(zuì最)巍:山险峻的样子。嵯峨:形容山势高峻。㳌渫(yádié押蝶):浪波叠起的样子。磊砢:树木多节。比喻人有奇特的才能。}。”{\fzxk\zihao{6}\textcolor{red}{按\CJKunderwave{三秦记}、\CJKunderwave{语林}载,蜀人伊籍称吴土地人物,与此语同。}}

{\cangkai\zihao{5}【评】本则刘孝标注、王世懋评皆疑其有误。王曰:“注是也。吴、蜀当此语是本色。按王、孙同为太原人,不当风土之异如此。”

倘酌量风物之异,确如上述推论。然此则之妙,正在于山水风物描摹中所透出的言语之美。王济之言,于山水人物描摹中,渗透着一种清雅之美,颇有老、庄境界。表达了王济“有隽才,能清言”的气质修养。孙子荆言则巧于状物。述景物,用近乎拟人化的手法;说人物,用拟物手法,将山水人物合二为一而说得灵动可感。“㳌渫”,余嘉锡\CJKunderwave{笺疏}引\CJKunderwave{一切经音义}释为“谓冰冻相著也”。波浪叠起,似冰之相拥相著,亦如人之亲昵嬉戏,则波浪之状,流湍之态便注入了一段生命的灵气,使水的气韵风貌活灵活现。一句话,说得山魄水魂栩栩而来。“磊砢”,\CJKunderwave{说文}:“磊,众石貌。”“砢,磊砢也。”磊砢为双声连绵词,状石之众多,委积魁磊。又转而形容树木如磊石之多节。这里,孙楚以物状人,将峥嵘奇崛的精神气质和奇特英发的才思描绘如自然景象一般可睹可亲,富有内涵。孙楚的山水人物之说,也表达了其人“天才英特,亮拔不群”的才俊。

这里说山水人物的言语之妙,已如作诗一样,迁想妙得,隽永精美。}

\lettrine{2.25} 乐令\myidx{乐广}女适大将军成都王颖\myidx{司马颖}\footnote{适:女子出嫁。成都王颖:司马颖,\CJKunderwave{晋书}记其为晋武帝第十六子,而非第十九子。},{\fzxk\zihao{6}\textcolor{red}{虞预\CJKunderwave{晋书}曰:“乐广字彦辅,南阳人。清夷冲旷,加有理识。累迁侍中、河南尹。在朝廷用心虚淡,时人重其贞贵。代王戎为尚书令。”\CJKunderwave{八王故事}曰:“司马颖字叔度,世祖第十九子,封成都王、大将军。”}} 王兄长沙王\myidx{司马乂}执权于洛\footnote{长沙王:晋武帝第六子,封长沙王,拜抚军大将军,在京师洛阳执掌朝廷大权,与成都王颖兴兵相图,兵败被杀。},{\fzxk\zihao{6}\textcolor{red}{\CJKunderwave{晋百官名}曰:“司马乂字士度,封长沙王。”\CJKunderwave{八王故事}曰:“世祖弟(第)十七子。”}} 遂构兵相图\footnote{构兵:兴兵,交战。}。长沙王亲近小人,远外君子,凡在朝者,人怀危惧。乐令既处朝望,加有婚亲,群小谗于长沙\footnote{远外:远离,排斥。朝望:在朝廷享有威望。群小:众小人。}。长沙尝问乐令,乐令神色自若,徐答曰:“岂以五男易一女\footnote{易:换。}?”{\fzxk\zihao{6}\textcolor{red}{\CJKunderwave{晋阳秋}曰:“成都王之起兵,长沙王猜广,广曰:‘宁以一女而易五男?’犹疑之,遂以忧卒。”}} 由是释然,无复疑虑\footnote{释然:放心的样子。无复:不再。}。

{\cangkai\zihao{5}【评】李贽评此条:“弃一女保五男,盖古有此语,乐用之,非乐实有五男也。”李评是,\CJKunderwave{晋书}本传记其实有三男:凯、群、谟。

乐广以长于“约言析理”而为当时名流推重。本则既记其言语特色,也描摹了一个有修养、谙达宦海风云,敏感、成熟的政治人物。面对长沙王的猜虑,以他在朝的地位、声望及与成都王颖的特殊关系,正面辩白是徒劳的。这里切近人情的一语回应,不仅机敏,而且真实得令人再无可猜疑。刘辰翁评曰:“一语坦然,敬服敬服。”可叹的是,事实并非如本则所言“由是释然,无复疑虑”,刘孝标注与\CJKunderwave{晋书}并言长沙王“犹以为疑,广竟以忧卒”。乐广的存在,对构兵于成都王的长沙王说来,确实事关重大,其心理可以理解。而“广竟以忧卒”,也说明士处乱世,真要“宅心事外”,岂是易事!}

\lettrine{2.26} 陆机\myidx{陆机}诣王武子\myidx{王济}\footnote{陆机:见刘孝标注。参\CJKunderwave{晋书}本传,其为吴郡吴县华亭(今上海松江)人,当时著名的文学家。吴亡入晋后,累迁太子洗马、著作郎。曾任平原内史,故称“陆平原”。事成都王颖,颖兴兵攻掌权于洛阳的长沙王司马乂时,任陆机为后将军、河北大都督。机兵败遭谗,与弟陆云同为颖所杀。王武子:见本门24注。},{\fzxk\zihao{6}\textcolor{red}{\CJKunderwave{晋阳秋}曰:“机字士衡,吴郡人。祖逊,吴丞相。父抗,大司马。机与弟云并有俊才,司空张华见而说之,曰:‘平吴之利,在获二隽。’”\CJKunderwave{机别传}曰:“博学,善属文,非礼不动。入晋,仕著作郎,至平原内史。”}} 武子前置数斛羊酪\footnote{斛(hú湖):古量器名,十斗为一斛。酪:一种奶制食品。},指以示陆曰:“卿江东何以敌此\footnote{江东:指自今芜湖以下的长江南岸地区,即江南。敌:相当,对当。}?”陆云:“有千里莼羹,但未下盐豉耳\footnote{千里:湖名,在今江苏溧阳市东南,以产莼菜著名。莼羹:莼为水生植物,采其嫩茎、叶做羹汤为吴地风味名菜。盐豉(chǐ尺):即豆豉,煮熟发酵后,以盐制过的黄豆瓣,用为调味佐料。余嘉锡引陆游\CJKunderwave{剑南诗稿}卷二十七\CJKunderwave{戏咏山阴风物}自注云:“莼菜最宜盐豉,所谓‘未下盐豉’者,言下盐豉则非羊酪可敌,盖盛言莼菜之美尔。”}。”

{\cangkai\zihao{5}【评】王济“少有逸才”,“为一时秀彦”,同时也是京洛狂士,斗才斗富,目空一切,甚至对皇帝也倨傲不逊。陆机为江南俊才,出身之家“功名奕世,将相连华”,然而却是以亡国之馀的身份,拜见这位京洛狂士。有意趣的是,当此情景两人相会,唇齿之间,又各不相让。王济之问居高临下,语带轻蔑,陆机之答绵里藏针,不买这位狂士的账。余嘉锡引徐树丕\CJKunderwave{识小录}卷三云:“千里,湖名,其地莼菜最佳。陆机答未下盐豉,尚能敌酪,若下盐豉,酪不能敌矣。”刘辰翁亦谓:“最得占对之妙,言外谓下盐豉后,(其味美)尚未止此。第语深约,可以意得,难以俊赏耳。”两说皆能揭示陆机应对之语的内涵及其妙处。就对话来说,陆机以其俊才胜了王济,而两者又俱见性情,故\CJKunderwave{晋书}称“时人以为名对”。从“言语”角度说来,妙语如绘,活灵活现地表达了两个人的性情,也情趣盎然地展现出了陆机机智敏捷的才华,此即为\CJKunderwave{世说}记述人物之妙。}

\lettrine{2.27} 中朝有小儿,父病,行乞药\footnote{中朝:晋南渡以后,称建都于中原的西晋为“中朝”。中朝都洛阳,历四帝,五十四年。}。主人问病,曰:“患疟也。”主人曰:“尊侯明德君子,何以病疟\footnote{尊侯:晋人习语,在人子之前称其父为“尊君”、“尊公”、“尊侯”,犹尊大人。明德:美德。}?”{\fzxk\zihao{6}\textcolor{red}{俗传行疟鬼小,多不病巨人。故光武皇帝尝谓景丹曰:“尝闻壮士不病疟,大将军反病疟耶?”}} 答曰:“来病君子,所以为疟耳\footnote{为疟:疟谐“虐”音,即“为虐”。}!”

{\cangkai\zihao{5}【评】王世懋评本则:“转语佳甚。”

原本的说法是疟鬼弱小而为祟,只能祸害德行不彰、身体羸弱的小人,若依此逻辑,则令尊大人害此疾病,还是“明德君子”么?孩童临此窘迫,用谐音一转,妙语解困,表现的智趣,颇为可爱动人。

谐语解困、解颐,不但表现了临机智趣,而且也是一种民间语言艺术,从先秦至魏晋从未消歇,到南北朝民歌中则更见异彩。\CJKunderwave{世说}不因这一形式“俗”而不取,可见该书尚美、尚智的倾向。同时,这也是时代风尚,刘勰作\CJKunderwave{文心雕龙}也不因不够“大雅”而摒弃“谐隐”这一文体,对其中蕴涵的“辞浅会俗,皆悦笑也”的智慧与幽默给予肯定。}

\lettrine{2.28} 崔正熊\myidx{崔豹}诣都郡,都郡将姓陈\footnote{崔正熊:崔豹,字正熊,作\CJKunderwave{古今注}三卷传于今。都郡:大郡。都郡将:都郡的首长。余嘉锡谓:“都郡将者,以他郡太守兼都督本郡军事也。”},问正熊:“君去崔杼\myidx{崔杼}几世\footnote{去:距离。崔杼:春秋时齐国大夫,杀了自己的国君齐庄公。}?”答曰:“民去崔杼,如明府之去陈恒\myidx{陈恒}\footnote{明府:魏晋时,对郡首长太守、刺史的尊称,亦称“明府君”。陈恒:春秋末期的齐国大夫,杀了自己的国君齐简公。}。”{\fzxk\zihao{6}\textcolor{red}{\CJKunderwave{晋百官名}曰:“崔豹字正熊,燕国人。惠帝时,官至太傅(仆)丞。”}}

{\cangkai\zihao{5}【评】弑君,不仅罪在不赦,而且在封建道德中是永留恶名的耻辱。\CJKunderwave{左传·襄公二十五年}载,崔杼杀了齐庄公后,史官秉笔直书:“崔杼弑其君。”崔慌得杀了这史官,但史官的弟弟仍如此直书,崔又杀了其弟,另一弟弟又如此写,崔无可奈何,只好作罢。另一面,南史氏闻说齐的史官都因此而被杀了,义无反顾,冒死“执简以往”,准备继续直书下去。可见这是大原则大是非,而对弑君者说来,又是遗臭万年的肮脏事。这里都郡将和崔正熊开了一个大玩笑,调侃他是崔杼的后裔,崔针锋相对,马上回敬“明府”,同理推断,你姓陈,大约也应当是“陈恒”后裔了。“陈恒”即“田常”,他不仅杀了齐简公,而且是田氏篡夺姜氏齐国的关键人物。杀人国君,篡人社稷,陈氏比崔氏更有甚者。崔正熊的回答不仅机敏而且见出他的学养,因此颇显动人的意趣。

徐震堮\CJKunderwave{世说新语校笺}以为:“此条入\CJKunderwave{言语}不如入\CJKunderwave{排调}。”若按前述情形,则入\CJKunderwave{言语},或许是作者更看重对话中以机敏、学养为底蕴的人物言语之美,而不认为这里有多少可肯定的喜剧、调笑成分。其实它貌似玩笑,内容却严肃沉重,王世懋曰:“此问者自卖破绽”,确是一语中的。由此说来,本则的精彩处并不在都郡将的“幽默”,却恰在崔正熊针锋相对维护尊严的回答。}

\lettrine{2.29} 元帝\myidx{司马睿}始过江\footnote{元帝:晋元帝司马睿,东晋第一位皇帝,在位七年,庙号“中宗”。},{\fzxk\zihao{6}\textcolor{red}{朱凤\CJKunderwave{晋书}曰:“帝讳叡,字景文。祖伷,封琅邪王。父恭王瑾嗣。帝袭爵为琅邪王。少而明惠,因乱过江起义,遂即皇帝位。”\CJKunderwave{谥法}曰:“始建国都曰‘元’。”}} 谓顾骠骑\myidx{顾荣}曰:“寄人国土,心常怀惭\footnote{顾骠骑:顾荣,死后赠侍中、骠骑将军。怀惭:心怀惭愧之情。}。”荣跪对曰:“臣闻王者以天下为家,是以耿、亳无定处\footnote{耿:古邑名,又名邢,在今河南温县东。商代自祖乙到阳甲时都于此。亳(bó博):古邑名,在今河南商丘市,商汤时都城。},{\fzxk\zihao{6}\textcolor{red}{\CJKunderwave{帝王世纪}曰:“殷祖乙徙耿,为河所毁。”今河东皮氏耿乡是也。“盘庚五迁,复南居亳。”今景亳是也。}} 九鼎迁洛邑\footnote{九鼎:相传夏禹铸九鼎,为国之重器,王位的象征。商灭夏,迁九鼎于商邑;周灭商,又将其迁于洛邑(今河南洛阳)。}。{\fzxk\zihao{6}\textcolor{red}{\CJKunderwave{春秋传}曰:“武王克商,迁九鼎于洛邑。”今之偃师是也。}} 愿陛下勿以迁都为念。”

{\cangkai\zihao{5}【评】余嘉锡笺云:“顾荣卒于元帝未即帝位以前,不当称陛下。”司马睿此言为其“始过江”,尚未建立东晋朝廷时的心理,顾荣之对,也是江东世族对是否拥戴北来权势的态度回馈。

当晋怀帝时,司马睿作为镇东大将军,都督扬、江、湘、交、广五州诸军事,移镇建业(今南京市),虽然他成了晋王朝督统江南的最高军政长官,然而实际形如高级难民。江北八王混战,继之以胡骑攻逼,已是“中原萧条,白骨涂地”,世家大族四散奔逃,西晋王朝家当丧尽,名存实亡。北方琅邪王司马睿,在实际掌握王朝军政大权的北海王司马越的安排下,先期过江,准备江北失守后,退据江南。这时,江东世家大族未遭中原那样大规模、旷日持久的战乱,生活相对优裕,实力较为雄厚,开始时瞧不起这班北来奔命的“伧父”的。在这样的情境下,“始过江”的司马睿的这番感慨,具有真实性,道出了江北贵胄大族们流离江表的普遍心理,容纳了万千悲慨,极显真情动人。

然而,他剖白内心的对象是“江东首望”的顾荣,对话就别有一番意味了。这既是情词恳切的剖白,也是对这位江东大族头面人物的拉拢与探询。顾荣的回答同样妥帖得体。江东大族搞清楚了,要想维护自己的利益,必须有一个能代表自己的政权,而这位为北来大族所拥戴的江东之主所要组织的政权,正是能代表他们利益的新政权。所以,顾荣这话,并非谀词,恰是情词恳切的态度回馈。

这一对话选在\CJKunderwave{世说·言语}中,或许因为在当时情景里,两个人的话都有所为而发,但在话语中又都显得率真、诚恳,人情味十足,在这里不仅见出“言语”颇耐玩味,同时也描画了说话人的风神。

另外,参见\CJKunderwave{前言},敬胤对本则故事,曾辩之凿凿,力说其有失真实性,它恰证明这则记载作为传闻、故事的灵动,不失其情理之真,形象之真,显出\CJKunderwave{世说}这部笔记小说的精妙况味。}

\lettrine{2.30} 庾公\myidx{庾亮}造周伯仁\myidx{周顗}\footnote{庾公:即庾亮,(289—340)的敬称。他历仕东晋元、明、成三朝,作为外戚,曾执国政,显赫于朝。的卢:传说中的凶马之名,骑之不利主人。又\CJKunderwave{晋书}本传述其“美姿容,善谈论,性好\CJKunderwave{老}、\CJKunderwave{庄},风格峻整,动由礼节”。造:拜访。周伯仁:见刘孝标注。},{\fzxk\zihao{6}\textcolor{red}{虞预\CJKunderwave{晋书}曰:“周顗字伯仁,汝南安城人,扬州刺史浚长子也。”\CJKunderwave{晋阳秋}曰:“顗有风流才气,少知名,正体嶷然,侪辈不敢媟也。汝南贲泰渊,清操之士,尝叹曰:‘汝、颍固多贤士,自顷陵迟,雅道殆衰。今复见周伯仁,伯仁将祛旧风,清我邦族矣!’举寒素,累迁尚书仆射。为王敦所害。”}} 伯仁曰:“君何所欣悦而忽肥\footnote{欣悦:诸本作“欣说”。}?”庾曰:“君复何所忧惨而忽瘦\footnote{复:又。忧惨:忧伤。}?”伯仁曰:“吾无所忧,直是清虚日来,滓秽日去耳\footnote{直是:只是。清虚:清静虚无,少尘念。渣秽:污浊,指尘俗之念。}。”

{\cangkai\zihao{5}【评】周顗是世家子,“少有重名”。为人风格与庾亮有相近处,“正体嶷然”,“风格峻整”,正直敢为,睿智健谈。二人交往中,言语便颇显睿智才情。\CJKunderwave{晋书·周顗传}曾记过他们的对话:“庾亮尝谓顗曰:‘诸人咸以君方乐广。’顗曰:‘何乃刻画无盐,唐突西施也。’”本则的对话也是如此。本来“胖”、“瘦”这种话头很世俗,李贽就说本则“太无味”,刘辰翁也说“极鄙而隐”,但寻绎起来,还是可以体会出其中的味道。

一是言语往还中的机敏睿智。欣悦则肥,忧惨则瘦,此乃人之常情,所以伯仁以此问肥,庾亮以此问瘦,两人问对皆机敏有意趣。二是伯仁最后一答,不仅机敏还风神摇曳。至少在表面上,周顗、庾亮都不看重官位而看重作为,所以,伯仁之言并非故作高标,而是内心的一种崇尚——清虚淡泊,归之自然,以此来代替俗想,于是赘肉便日去而瘦。清虚淡泊,归之自然,又恰恰是当时士大夫所标榜的一种雅致,故尔此言此语便有了动人的意趣。}

\lettrine{2.31} 过江诸人,每至美日,辄相邀新亭,藉卉饮宴\footnote{过江诸人:西晋末年,中原失守,绝大部分世族豪强纷纷过长江,避地江南。后元帝司马睿建立东晋政权,这些世族集团的人物又在朝廷任职。这里的“诸人”,即指世族集团的人物。美日:风和日丽的日子。新亭:亭名,三国时吴建,旧址在今南京市西南的长江边。藉卉(huì会):坐在草地上。卉,草。}。{\fzxk\zihao{6}\textcolor{red}{\CJKunderwave{丹阳记}曰:“新亭,吴旧立,先基崩沦。隆安中,丹阳尹司马恢之徙创今地。”}} 周侯\myidx{周顗}{\fzxk\zihao{6}\textcolor{red}{顗也。}} 中坐而叹曰\footnote{周侯:即周顗,\CJKunderwave{晋书}本传说周顗“弱冠,袭父爵‘武成侯’”,故称。中坐:坐中。}:“风景不殊,正自有山河之异\footnote{不殊:没有不同。自有:只有。}!”皆相视流泪。唯王丞相\myidx{王导}{\fzxk\zihao{6}\textcolor{red}{导也。}} 愀然变色曰\footnote{王丞相:即王导,是辅助司马睿建立东晋政权的主要人物之一。政权建立,王导后为丞相。愀(qiǎo巧)然:脸色变得严肃的样子。}:“当共勠力王室,克复神州,何至作楚囚相对\footnote{戮(lù录)力:和力,并力。神州:古称中国为“赤县神州”,此指中原。楚囚:见刘孝标注,这里借指处境窘迫的人。}?”{\fzxk\zihao{6}\textcolor{red}{\CJKunderwave{春秋传}曰:“楚伐郑,诸侯救之。郑执郧公锺仪献晋,景公观军府,见而问之曰:‘南冠而絷者为谁?’有司对曰:‘楚囚也。’使脱之,问其族,对曰:‘伶人也。’‘能为乐乎?’曰:‘先父之职,敢有二事!’与之琴,操南音。范文子曰:‘楚囚,君子也。乐操土风,不忘旧也。君盍归之,以合晋楚之成。’”}}

{\cangkai\zihao{5}【评】余嘉锡引赵绍祖\CJKunderwave{通鉴注商}谓:“此大概言神州陆沉,非复一统之旧,故诸名士闻之伤心,相视流涕。”

坐中周顗的一席话,道出了诸名士共同的心理感受。王朝龟缩于半壁江山,诸公成了流离难民,今昔沧桑,丧家沦落之感是痛彻肺腑的。越是“美日”,就越令人伤心。周顗一向“正体嶷然”,敢作敢为,他的这番话尚带有发自内心的忧患和一种豪气,其浩叹,是山河变色令人悲不自胜的感觉。然坐中“相视流泪”者,倒真像“楚囚相对”,徒然悲怆,窘迫无计。

王导不愧是领袖群伦的人物,他的话并非故放大言。早在西晋时代,王氏家族就累世公卿,有大功于王室,而王导又少年知名,“识量清远”,参东海王越军事,与琅邪王司马睿“素相亲善”。“(王)导知天下已乱,遂倾心推奉(司马睿),潜有兴复之志”(\CJKunderwave{晋书·王导传})。在他参与谋划下,西晋崩溃前夜就与司马越有计划地经营江左。他最终拥戴司马睿,成功地建立了江东政权,脚踏实地,实践着他的“兴复之志”。这里的话,是他一贯作风的自画像,有谋略、能实干,胸襟高迈。因而,他的话不仅掷地有声,而且富有感召力。在国难当头时刻,一个爱国有为的政治家形象跃然纸上。

本则诸人诸语皆真切如绘,刘辰翁概括其中的意味:“俯仰情至。”}

\lettrine{2.32} 卫洗马\myidx{卫玠}初欲渡江,形神惨悴,语左右云\footnote{卫洗马:即卫玠,官拜太子洗马,故称。惨悴:忧伤憔悴的样子。左右:身边侍从人员。}:“见此茫茫,不觉百端交集。苟未免有情,亦复谁能遣此\footnote{茫茫:江水浩荡无边的样子。百端:指各种忧思愁绪。苟:如果。未免:不免。遣:排遣。}!”{\fzxk\zihao{6}\textcolor{red}{\CJKunderwave{晋诸公赞}曰:“卫玠字叔宝,河东安邑人。祖父瓘,太尉。父恒,黄门侍郎。”\CJKunderwave{玠别传}曰:“玠颖识通达,天韵标令。陈郡谢幼舆敬以亚父之礼。论者以为出王眉子、平子、武子之右,世咸谓‘诸王三子,不如卫家一儿’。娶乐广女,裴叔道曰:‘妻父有水清之姿,壻有璧润之望,所谓秦晋之匹也。’为太子洗马。永嘉四年,南至江夏,与兄别于梁里涧,语曰:‘在三之义,人之所重。今日忠臣致身之运,可不勉乎?’行至豫章,乃卒。”}}

{\cangkai\zihao{5}【评】卫玠被后人评为中兴名士第一。

引发卫玠感慨的是亡国之际的捐家园,渡江南奔,面对浩茫大江,百端交集,难堪愁绪。宗白华先生曾有过说明:魏晋时代人“精神上的真自由、真解放,才能把我们的胸襟像一朵花似的展开,接受宇宙和人生的全景,了解它的意义,体会它深沉的境地。近代哲学上所谓‘生命情调’、‘宇宙意识’遂在晋人这超脱的胸襟里萌芽起来。卫玠初欲过江,形容惨悴,语左右曰:‘见此茫茫,不觉百端交集,苟未免有情,亦复谁能遣此?’后来唐初陈子昂\CJKunderwave{登幽州台歌}:‘前不见古人,后不见来者。念天地之悠悠,独怆然而涕下!’不是从这里脱化出来的?而卫玠的一往情深,更令人心恸神伤,寄慨无穷”(\CJKunderwave{论〈世说新语〉和晋人的美})。卫玠将自我的心灵遭际,融进人生、宇宙这浩茫无涯的境界,于是他的愁绪感慨就超然于自我之外,有了更能动人的力量,也便能超越时间的界限,不断唤起读者的共鸣。王世懋读此就曾说过:“至今读之欲绝,况在当时德音面聆者耶?”刘应登评此,也强调它的超逸:“此匆匆出语耳,而微辞逸旨,超然风埃之表。江左诸公,叔宝真言语之科也。”}

\lettrine{2.33} 顾司空\myidx{顾和}未知名,诣王丞相\myidx{王导}\footnote{顾司空:顾和,死后追赠司空,故称。王丞相:即王导。}。丞相小极\footnote{小极:身体略感疲困不适。},对之疲睡。顾思所以叩会之\footnote{叩会:拜问交谈。},{\fzxk\zihao{6}\textcolor{red}{\CJKunderwave{顾和别传}曰:“和字君孝,陈(吴)郡人。祖(祖,当为曾祖。)容,吴荆州刺史。父(父,当为祖)相,晋临海太守。和总角知名,族人顾荣雅相器爱,曰:‘此吾家之骐骥也,必振衰族。’累迁尚书令。”}} 因谓同坐曰:“昔每闻元公\myidx{顾荣}{\fzxk\zihao{6}\textcolor{red}{顾荣。}} 道公协赞中宗\myidx{司马睿},保全江表\footnote{元公:即顾荣,顾荣(?—312):两晋之际江南士族领袖之一,与陆机、陆云同时入洛,时称“三俊”。南渡后,代表江南士族拥护和支持司马睿在江南开国,是为东晋。洛阳:西晋京师。顾荣死后谥“元”,顾和是其同族晚辈,故称其为“元公”。协赞:协助辅佐。中宗:晋元帝司马睿的庙号。江表:指江南。从中原看,江南地区在长江之外,故称。表:外。}。{\fzxk\zihao{6}\textcolor{red}{邓粲\CJKunderwave{晋纪}曰:“导与元帝有布衣之好。知中国将乱,劝帝度江,求为安东司马,政皆决之,号‘仲父’。晋中兴之功,导实居其首。”}} 体小不安,令人喘息\footnote{喘息:呼吸急促。此处表示焦虑、精神紧张的样子。}。”丞相因觉,谓顾曰:“此子珪璋特达,机警有锋\footnote{圭璋特达:圭、璋都是古代帝王、诸侯在典礼或朝会时所执的玉器。古以玉象征美德和聪明。此句本\CJKunderwave{礼记·聘仪}语,意谓美德、才能出众,不需人介绍,自然就会通达。}。”

{\cangkai\zihao{5}【评】尚未知名的顾和,想要得到名重天下的王导来提携奖掖,却又不巧,正赶上丞相疲困休息,于是顾和动脑筋,表现了他的“机警”。就“言语”说来,他组织的话确实精要完备,短短的一席话,说明了他的身世、对丞相的仰慕及关切。王导一向对“江东首望”的顾荣礼遇有加,这一望族后辈的这番话,足以引起王导的注意,顾和的“言语”确是“机警有锋”,见其聪明急智。这样,自然达到了目的,得到丞相的赞赏,“由是遂知名”(\CJKunderwave{晋书·顾和传})。但刘辰翁从另一个角度看到了本则的妙处:“\CJKunderwave{世说}长处,在写一时小小节次,如见可想。”这里更见精彩的是场景、细节的描绘,把“一时小小节次”,写得如在目前。}

\lettrine{2.34} 会稽贺生\myidx{贺循},体识清远,言行以礼\footnote{会(kuài快)稽:郡名,治所在山阴(今浙江绍兴)。贺生:贺循,字彦先,会稽山阴人。\CJKunderwave{晋书}卷三十八有传。出身\CJKunderwave{礼}学世家,博览群书,尤精\CJKunderwave{礼}学。经陆机举荐,入朝任太子舍人,京师八王之乱起,去职归乡。后作为江东世族代表人物,和顾荣等一起支持元帝司马睿建立东晋王朝。为元帝所倚重,仕太常,领太子太傅,卒赠司空。生:先生之省称,对有道德学养人的敬称。体识:胸襟、见识。}{\fzxk\zihao{6}\textcolor{red}{。贺循已见。}} 不徒东南之美,{\fzxk\zihao{6}\textcolor{red}{\CJKunderwave{尔雅}曰:“东南之美者,有会稽之竹箭焉。”}} 实为海内之秀\footnote{不徒:不只是。东南之美:见刘孝标注,此谓东南的杰出人才。秀:杰出的人才。}。

{\cangkai\zihao{5}【评】本则“不徒东南之美,实为海内之秀”句,见\CJKunderwave{晋书·顾和传},是王导赏评顾和的话。余嘉锡\CJKunderwave{笺疏}引李慈铭云:“按‘会稽贺生’上,疑有脱文。”余嘉锡谓:“此不知何人之言,\CJKunderwave{世说}自他书摘出,失其本末耳。”这一则确实与\CJKunderwave{言语}的常例不符,失去了\CJKunderwave{言语}描摹魏晋人物气质的特色,所以凌濛初看它:“甚似\CJKunderwave{赏誉}。”}

\lettrine{2.35} 刘琨\myidx{刘琨}虽隔阂寇戎,志存本朝\footnote{刘琨:见刘孝标注。出身士族,能诗文,有才略。西晋末,怀帝永嘉元年(307)出任并州刺史,加振威将军。愍帝初,拜大将军,都督并、冀、幽三州诸军事。他力拒刘聪、石勒,后因孤军无援,投段匹磾,希望与之联合抗敌,旋为磾所害。追赠侍中、太尉,谥愍。隔阂寇戎:被寇扰中原的匈奴、鲜卑军所阻隔,不能与晋王室联络。志存本朝:志在拯救晋室。本朝:晋王朝。},{\fzxk\zihao{6}\textcolor{red}{王隐\CJKunderwave{晋书}曰:“琨字越石,中山魏昌人。祖迈,有经国之才。父蕃,光禄大夫。琨少称隽朗,累迁司徒长史、尚书左右丞。迎大驾于长安,以有异勋,封广武侯。年三十五出为并州刺史,为叚日磾(段匹磾)所害。”}} 谓温峤\myidx{温峤}\footnote{温峤:见刘孝标注。当时追从姨夫刘琨,在并州为谋主,“琨所凭恃焉”(\CJKunderwave{晋书·温峤传})。建武元年(317)奉刘琨命出使江南,拥戴司马睿即帝位,建立东晋王朝。受司马睿重用,留为散骑常侍,后官至中书令,为东晋名臣。}曰:“班彪\myidx{班彪}识刘氏之复兴,马援\myidx{马援}知汉光\myidx{刘秀}之可辅\footnote{“班彪”、“马援”句:见刘孝标注。他们在两汉之际,王莽败政,天下大乱时,看出群雄中只有刘秀才是继统复兴汉室之主,便支持了刘秀。事见\CJKunderwave{后汉书}本传。汉光:即东汉光武帝刘秀。}。{\fzxk\zihao{6}\textcolor{red}{\CJKunderwave{汉书叙传}曰:“彪字叔皮,扶风人,客于天水。陇西隗嚣有窥觎之志,彪作\CJKunderwave{王命论}以讽之。”\CJKunderwave{东观汉记}曰:“马援字丈渊,茂陵人。从公孙述、隗嚣游。后见光武曰:‘天下反覆,盗名字者不可胜数,今见陛下寥廓大度,同符高祖,乃知帝王自有真也。’帝甚壮之。”}} 今晋阼虽衰,天命未改。吾欲立功于河北,使卿延誉于江南。子其行乎\footnote{阼:东阶,主人之位。天子登基称“践阼”,此指王朝的国运。延誉:播扬声誉,宣传功业。其:助词,表示命令、劝勉,犹“可”、“可要”。}?”温曰:“峤虽不敏,才非昔人,明公以桓、文之姿建匡立之功,岂敢辞命\footnote{明公:尊称有官职、地位的人。此指刘琨。桓、文:即春秋时齐桓公、晋文公。他们先后作为诸侯盟主,引领诸侯匡辅周室。姿:资质,才干。匡立之功:辅助朝廷,建立功业。}!”{\fzxk\zihao{6}\textcolor{red}{虞预\CJKunderwave{晋书}曰:“峤字太真,太原祁人。少标俊清彻,英颖显名,为司空刘琨左司马。是时二都倾覆,天下大乱,琨闻元皇受命中兴,忼慨幽、朔,志存本朝。使峤奉使,峤喟然对曰:‘峤虽乏管、张之才,而明公有桓、文之志,敢辞不敏,以违高旨?’以左长史奉使劝进,累迁骠骑大将军。”}}

{\cangkai\zihao{5}【评】“琨少负志气,有纵横之才”(\CJKunderwave{晋书·刘琨传}),曾与祖狄一同“闻鸡起舞”,壮怀激烈,有经世之志。后在怀、愍帝时,领并州刺史、大将军,都督并、冀、幽三州,可谓受命于危难之际,身当最前敌。他几乎是孤军奋战于北方,坚持十馀年,招抚流亡百姓,志存晋室,直至被害。这里与温峤之言,一如其人其诗,“气猛神王,意概不凡”,“忠义之气自然形见”,其睿智、胆识与胸襟志气都跃然纸上。温峤之对也诚恳、生动。他不只是刘琨同道,也是其知音,在天下危难之时绝不苟且,知难而进。相比之下,一个昂扬壮烈,一个沉实稳重,个性俱异,但字里行间都荡逸着令人感愤的同赴忧患的英雄气概。}

\lettrine{2.36} 温峤\myidx{温峤}初为刘琨\myidx{刘琨}使来过江。于时江左营建始尔,纲纪未举\footnote{温峤:当时追从姨夫刘琨,在并州为谋主,“琨所凭恃焉”(\CJKunderwave{晋书·温峤传})。建武元年(317)奉刘琨命出使江南,拥戴司马睿即帝位,建立东晋王朝。受司马睿重用,留为散骑常侍,后官至中书令,为东晋名臣。于时:当时,即晋元帝建武元年(317)。江左:江南。营建:指政权的经营创建。始尔:开始。尔,助词,无实义。纲纪:法度、秩序。举:建立。}。温新至,深有诸虑。既诣王丞相\myidx{王导},陈主上幽越,社稷焚灭,山陵夷毁之酷,有\CJKunderwave{黍离}之痛\footnote{王丞相:王导。陈:述说。主上:皇上,此指晋愍帝司马邺。建兴四年(316),匈奴刘曜围长安,城中绝粮,愍帝出降,刘曜掳愍帝至平阳,西晋亡。次年杀愍帝。幽越:囚禁、远迁。社稷:古代天子必立社(土神)稷(谷神)而祭祀,因以社稷作为国家标志,社稷之有无,表示国家之存亡。山陵:指帝王坟墓。夷毁,铲平毁坏。\CJKunderwave{黍离}:原为\CJKunderwave{诗经·王风}中的诗篇。据说该诗是周大夫行役,路过西周都城,见故都的宗庙、宫室都长起了禾黍,于是作歌,哀叹西周的覆亡。后“黍离”就成了痛悼亡国的典故。}。温忠慨深烈,言与泗俱,丞相亦与之对泣\footnote{泗:鼻涕。}。叙情既毕,便深自陈结,丞相亦厚相酬纳\footnote{陈结:倾谈结交。酬纳:酬答接纳。}。既出,欢然言曰:“江左自有管夷吾,此复何忧\footnote{管夷吾:见刘孝标注。其为春秋时的名相,辅佐齐桓公成就了霸业。这里以此喻王导。}?”{\fzxk\zihao{6}\textcolor{red}{\CJKunderwave{史记}曰:“管仲夷吾者,颍上人。相齐桓公,九合诸侯,一匡天下。”\CJKunderwave{语林}曰:“初,温奉使劝进,晋王大集宾客见之。温公始入,姿形甚陋,合座尽惊。既坐,陈说九服分崩,皇室㢮绝,晋王君臣莫不歔欷。及言天下不可以无主,闻者莫不踊跃,植发穿冠。王丞相深相付托。温公既见丞相,便游乐不住,曰:‘既见管仲,天下事无复忧。’”}}

{\cangkai\zihao{5}【评】在北方,晋怀帝被掳,匈奴首领刘聪置酒高会,“使帝著青衣行酒”;愍帝被掳,刘聪出猎“令帝行车骑将军,戎服执戟为导”,置酒大会,“使帝行酒洗爵。反而更衣,又使帝执盖”,一国之尊,被当作小丑一样尽情调笑、戏谑,国人已难堪这种奇耻大辱,加之以亡国之痛,还有戎寇横行,尸骨遍野,这些都是浴血奋战在北方的温峤所亲闻亲历的,其切肤之痛不难想见。温峤又与刘琨有着共同的宏愿——“建匡立之功”,因而,初过江,见到一切都在草创,没有头绪,自然忧心忡忡;见到王导,如同见到亲人,痛诉忧苦,这一切都描摹得真切感人,陈梦槐说:“全在描画出生韵,使我唏嘘酸痛。”描画这一过程、情景之后,出以对王导的赞叹,更显得真实。这种描摹的动人之处,不在于对王导的赞叹,而是温峤那种如释重负的神情。它精妙地表达出这位志士仁人,立宏愿复兴王朝的真诚。}

\lettrine{2.37} 王敦\myidx{王敦}兄含\myidx{王含}为光禄勋\footnote{王敦:字处仲,晋琅邪临沂(今属山东)人,王导堂兄。妻为晋武帝女襄城公主,拜驸马都尉。晋室东迁,与王导一起辅佐元帝,任要职,握重兵,镇守扬州、荆州等重镇。公元322年起兵谋反,入京都建康。王含:见刘孝标注。光禄勋:官名,九卿之一,领管光禄、大中、中散、谏议等大夫及羽林郎、五官、虎贲、左右等中郎将。}。{\fzxk\zihao{6}\textcolor{red}{\CJKunderwave{含别传}曰:“含字处弘,琅邪临沂人。累迁徐州刺史、光禄勋。与弟敦作逆,伏诛。”}} 敦既逆谋,屯据南州,含委职奔姑熟\footnote{逆谋:余嘉锡\CJKunderwave{校笺}谓当为“谋逆”误倒。南州:城名,东晋时建,军事重镇,为建康的西南门户,故址在今安徽当涂。委职:丢弃官职。}。{\fzxk\zihao{6}\textcolor{red}{邓粲\CJKunderwave{晋纪}曰:“初,王导协赞中兴,敦有方面之功。敦以刘隗为间己,举兵讨之,故含南奔武昌,朝廷始警备也。”}} 王丞相\myidx{王导}诣阙谢\footnote{王丞相:即王导。诣阕:到皇宫门口。谢:请罪。}。{\fzxk\zihao{6}\textcolor{red}{\CJKunderwave{中兴书}曰:“导从兄敦举兵讨刘隗,导率子弟二十馀人,旦旦到公车泥首谢罪。”}} 司徒、丞相、扬州官僚问讯,仓卒不知何辞\footnote{司徒、丞相、扬州:指当时王导所任官职。扬州:指扬州刺史。余嘉锡\CJKunderwave{校笺}谓当时王导为司空,任丞相为成帝咸康四年(339)的事情。官僚:王导所任诸官府中的僚属。仓卒:突然,匆忙。}。顾司空\myidx{顾和}时为扬州别驾\footnote{顾司空:顾和。别驾:刺史的佐吏。},授(援)翰曰\footnote{授翰:“授”,诸本作“援”,“援”字是。援翰,拿起笔。}:“王光禄\myidx{王含}远避流言,明公\myidx{王导}蒙尘路次,群下不宁,不审尊体起居何如\footnote{远避流言:此是对王含弃职奔姑孰,从王敦反的一种委婉讳饰的说法。蒙尘:蒙受风尘。也是对王导率族人日日到阙下泥首请罪的委婉说法。路次:路中。群下:指众属吏。不审:不清楚。尊体:犹言贵体。起居:指日常生活。}?”

{\cangkai\zihao{5}【评】王敦握重权、重兵,据守重镇,并且平定蜀中之乱有大功,一时威势逼主,晋元帝“畏而恶之”,宠用刘隗、刁协以资防范。王敦本雄豪“有问鼎之心”,此时受到刘隗、刁协钳制,便以“清君侧”、诛刘隗为名,兴兵反,其兄王含也弃职“叛奔于敦”(\CJKunderwave{晋书·王敦传})。对王氏家族而言,这谋逆之举是意味着灭顶之灾的,更有“刘隗劝帝悉诛王氏”(\CJKunderwave{晋书·王导传}),王氏家族之危可以想见。所以王导率族人日日到阙下泥首请罪。见周顗入宫,王导呼而语“伯仁,以百口累卿!”(\CJKunderwave{晋书·周顗传})其危如累卵之时的乞哀告怜,真惊心动魄。在如此危难之时,王导僚属出来探询、劝慰是情理之中的事。但情形却十分微妙,王导一面是时刻有被诛灭可能的“罪臣”,一面又是他们的现任上司,于公于私,探询、劝慰令僚属们着实左右为难。在这样的情景中,看出了顾和的聪明,一句话既慰问了王导,又不为好事者留下任何话柄。顾司空:顾和,死后追赠司空,故称。王丞相:即王导。王导未枉识顾和,顾和也未辜负王导。本则更加印证了顾和“机警有锋”的风流神采。}

\lettrine{2.38} 郗太尉\myidx{郗鉴}拜司空\footnote{郗太尉:郗鉴(269—339),字道徽,晋高平金乡(今属山东)人。东晋初官至司徒、进位太尉,位至朝廷三公,故称。明帝时,鉴都督扬州,牵制王敦;成帝时,平祖约、苏峻有功。永嘉:晋怀帝司马炽年号(307—312)。永嘉五年,匈奴族武装攻陷京师洛阳,怀帝被俘。史称永嘉之乱,不久西晋亡。司空:官名,三公之一,晋属一品官。},语同座曰:“平生意不在多,值世故纷纭,遂至台鼎\footnote{多:大。事故:世事。台鼎:古代以台鼎喻三公。台,三台星,上台、中台、下台;鼎有三足。晋以太尉、司徒、司空为三公,他们是执掌军政大权的最高官吏。}。朱博\myidx{朱博}翰音,实愧于怀\footnote{朱博:见刘孝标注。刘孝标注“音者,音飞而实不从也”句,诸本为“飞者,音飞而实不从也”。}。”{\fzxk\zihao{6}\textcolor{red}{\CJKunderwave{汉书}曰:“朱博字子元,杜陵人。为丞相,临拜,延登受策,有大声如钟鸣。上问扬雄,雄对曰:‘\CJKunderwave{洪范}所谓鼓妖者也。人君不聪,空名得进,则有无形之声。’博后坐事自杀。”故\CJKunderwave{序传}曰:“博之翰音,鼓妖先作。”\CJKunderwave{易·中孚}曰:“上九,翰音登于天,贞凶。”王弼\CJKunderwave{注}曰:“翰,高飞也。音者,音飞而实不从也。”}}

{\cangkai\zihao{5}【评】郗鉴“少孤贫”,然勤奋自厉,博览群籍,通世故,明事理,于纷乱之世,刚而不吐,柔亦不茹,从容儒雅,博得人厚爱与信赖,有大名于天下,是当时北来流民的领袖人物。当其“拜司空”位极人臣之时,又以朱博自警,益发见其谙练洞达之雅。

西汉末成、哀帝时的朱博,与郗鉴的经历有相似之处,都是由贫贱而登三公显位。所不同的是,朱博不能学,以狭气好交,为人廉俭,聪明果断,由佐吏“历位以登宰相”(\CJKunderwave{汉书·朱博传})。他只知驰骛进取,而不思道德事理,终于身败名裂。这印证了\CJKunderwave{易经}“翰音登于天”——追逐名声其实不副,结局必凶的古训。郗鉴以朱博故事来告诫自己,正深见其平生修养,绝无矫情文饰之态。其知进知退,从容豁达的一句话,解得其人浑实、生动,呼之欲出。刘辰翁赞曰:“解得精实。”}

\lettrine{2.39} 高座\myidx{高座}道人不作汉语,或问此意\footnote{此意:这里的原故。},简文\myidx{司马昱}曰\footnote{简文:东晋简文帝司马昱。指晋简文帝司马昱(320—372),穆帝年幼即位,昱任抚军大将军总理政务。后来大将军桓温专擅朝政,先废海西公,后立司马昱为帝,第二年崩。}:“以简应对之烦。”{\fzxk\zihao{6}\textcolor{red}{\CJKunderwave{高座别慱(传)}曰:“和尚胡名尸黎密,西域人。传云国王子,以国让弟,遂为沙门。永嘉中,始到此土,止于大市中。和尚天姿高朗,风韵遒迈。丞相王公一见奇之,以为吾之徒也。周仆射领选,抚其背而叹曰:‘若选得此贤,令人无恨。’俄而周侯遇害,和尚对其灵坐作胡咒数千言,音声高畅,既而挥涕收泪,其哀乐废兴皆此类。性高简,不学晋语。诸公与之言,皆因传译。然神领意得,顿在言前。”\CJKunderwave{塔寺记}曰:“尸密黎宋(冢)曰高坐,在石子冈。常行头陀,卒于梅冈,即葬焉。晋元帝于冢边立寺,因名‘高座’。”}}

{\cangkai\zihao{5}【评】简文帝的性情是清虚寡欲,多玄言而少俗累,凝尘满席也能坐处湛如,不以为意。另一方面,当时桓温骄横于天下,咄咄逼人,做皇帝的前后,简文都是“拱默守道而已”,所以他清谈可以无顾忌,而世俗应对则尚“简贵”,免得话多得祸。然而,他此刻的这番话又正合沙门“心观”旨趣。佛家以为,看得见摸得着的一切实相都是空的,因此对世界万物的实相要用“现观”的认识办法,也就是“心行言语断”(\CJKunderwave{中论·观法品}),用心去直接体会对象,与对象合而为一,无需通常的思维活动,也不要语言的中介,所谓“如哑受义”,这比语言的表达、传授更为清晰、深刻、真切。胡僧不作汉语正是佛家境界,胡语汉语有何关系?只要心行顿悟,证成正果就好;因此简文的话让人颇有玩味的馀地。}

\lettrine{2.40} 周仆射\myidx{周顗}雍容好仪形\footnote{周仆射:即周顗。雍容,仪态大方从容。仪形:仪容形貌。}。诣王公\myidx{王导},初下车,隐数人,王公含笑看之\footnote{王公:即王导。隐:依,让人搀扶。}。既坐,傲然啸咏\footnote{啸咏:亦作吟啸、长啸、歌啸等。啸,\CJKunderwave{说文}“吹声也”,段玉裁注“蹙口而出声也”。即吹口哨,其声长而清越。啸咏是魏晋人崇尚的风度、逸韵。}。王公曰:“卿欲希嵇\myidx{嵇康}、阮\myidx{阮籍}邪\footnote{希:仰慕。引申为仿效。阮:阮籍,嵇康(223—262):三国时谯郡铚 (今安徽亳县)人。“竹林七贤”之一。曾任中散大夫,故称嵇中散。当时著名思想家、文学家、清谈名家。因其主张越名教而任自然,抨击礼法之士,不与司马氏统治集团合作,盛年被杀。嵇:嵇康,晋文王:即司马昭(211—265),三国时河内温县(今属河南)人。懿次子,师同母弟。继兄师任魏之大将军,专擅朝政。灭蜀后,封晋公,加九赐,进位相国,已成篡魏开晋之势。后弑高贵乡公曹髦而立曹奂为帝,封晋王。死谥文,故称晋文王。阮籍(210—263):三国时陈留尉氏(今属河南)人。父瑀为建安七子之一,籍则为“竹林七贤”之一。曾官步兵校尉,故称“阮步兵”。当时著名的思想家及诗文名家,又是玄学清谈的代表人物。二人皆“竹林七贤”中大名士。}?”答曰:“何敢近舍明公,远希嵇、阮。”{\fzxk\zihao{6}\textcolor{red}{邓粲\CJKunderwave{晋纪}曰:“伯仁仪容弘伟,善于俯仰应答。精神足以荫映数人。深自持,能致人而未尝往焉。”}}

{\cangkai\zihao{5}【评】周顗颇有些名士风度,少年时即“神采秀彻,虽时辈亲狎,莫能亵也”(\CJKunderwave{晋书·周顗传}),自有一段风神气骨。作尚书左仆射,常醉不醒,人称“三日仆射”。他来到王导家,受到礼遇,可坐下却傲慢啸咏起来。当王导问他这副做派是不是仰慕人家大名士时,周顗真“善于俯仰应答”,一席话说得真诚妥帖。周顗不是阮、嵇那样的真名士,作为官场中人,他能真诚钦敬王导,心口如一,便很动人了,更何况应口回答,如此妥帖呢。可见,言语的精彩,还有赖于人本身的精彩。}

\lettrine{2.41} 庾公\myidx{庾亮}尝入佛图,见卧佛\footnote{庾公:庾亮(289—340)的敬称。他历仕东晋元、明、成三朝,作为外戚,曾执国政,显赫于朝。的卢:传说中的凶马之名,骑之不利主人。佛图:佛寺。卧佛:指侧身躺着的释迦牟尼像。见刘孝标注。},{\fzxk\zihao{6}\textcolor{red}{\CJKunderwave{涅盘经}云:“如来背痛,于双树间北首而卧。”故后之图绘者为此象。}} 曰:“此子疲于津梁\footnote{津梁:桥梁,此喻佛家济度众生。}。”于时以为名言。

{\cangkai\zihao{5}【评】东晋时期,佛教已盛,士大夫对佛教一般知识甚为熟悉,大家能够心领神会,于是才会有这样的“名言”。

佛教的神圣真理——“四圣谛”(苦谛、集谛、灭谛、道谛),给人生以“苦”的基本价值判断,佛家的全部热诚,就是要普度众生脱离人生此岸苦海,登彼岸佛国净土。而作为佛祖的释迦牟尼,正是肩负重任的佛家领袖,因而,以情理论,他应是最辛苦的,恰如船夫于渡口渡人,往返不息,疲于劳作。见此卧佛,庾亮以俗喻雅,于情于理皆有所关合,其智其趣,遂成幽默。蕴涵这样大有意味的智趣,刘辰翁赞说它“有味外味”。}

\lettrine{2.42} 挚瞻\myidx{挚瞻}曾作四郡太守、大将军户曹参军,复出作内史\footnote{户曹参军:官名,掌民户、祠祀、农桑等。内史:官名。晋袭汉制,郡县与封建并行,在王国中设内史,其职位、体制同于郡太守。}。{\fzxk\zihao{6}\textcolor{red}{\CJKunderwave{挚氏世本}曰:“瞻字景游,京兆长安人,太常虞兄子也。父育,凉州刺史。瞻少善属文,起家著作郎。中朝乱,依王敦为户曹参军,历安丰、新蔡、西阳内史。见敦以故坏裘赐老病外部都督,瞻谏曰:‘尊裘虽故,不宜与小吏。’敦曰:‘何为不可?’瞻时因醉,曰:‘若上服皆可用赐,貂蝉亦可赐下乎?’敦曰:‘非喻所引,如此不堪二千石。’瞻曰:‘瞻视去西阳,如脱屣耳!’敦反(怒),乃左迁随郡内史。”}} 年始二十九。尝别王敦\myidx{王敦},敦谓瞻曰:“卿年未三十,已为万石,亦太蚤\footnote{万石:汉制,丞相、太尉、御史大夫年俸号称万石,后以万石泛指高官。蚤:通“早”。}。”瞻曰:“方于将军,少为太蚤;比之甘罗\myidx{甘罗},已为太老\footnote{方:比。甘罗:见刘孝标注。}。”{\fzxk\zihao{6}\textcolor{red}{\CJKunderwave{挚氏世本}曰:“瞻高亮有气节,故以此合(答)敦。后知敦有异志,建兴四年与第五猗据荆州以拒敦,竟为所害。”\CJKunderwave{史记}曰:“甘罗,秦相茂之孙也。年十二,而秦相吕不韦欲使张唐相燕,唐不肯行,甘罗说而行之。又请车五乘以使赵,还报秦。秦封甘罗为上卿,赐以甘茂田宅。”}}

{\cangkai\zihao{5}【评】挚瞻,\CJKunderwave{晋书}无传,于\CJKunderwave{周访传}中提及“贼帅杜曾、挚瞻、胡混等”。余嘉锡\CJKunderwave{笺疏}引李慈铭云:“其冤甚矣。”余先生亦考证推论,认为:“挚瞻自以忤敦而死,而名为贼帅,何其谬耶!”

挚瞻与王敦甚为熟悉,在与王敦打交道的过程中,见出挚瞻的确“高亮有气节”。王敦蜂目豺声,是一个野心勃勃、残忍霸道的军阀。这点当为挚瞻所熟知。本则里,王敦之语带有欺凌、霸道意味。挚瞻不畏淫威,着实回敬了他一个软钉子,不卑不亢,机敏巧妙。王敦味此馀音,只能徒增恨恨。挚瞻终“以忤敦而死”,此亦终证见其气节。}

\lettrine{2.43} 梁国杨氏子九岁,甚聪惠\footnote{聪惠:惠通“慧”,聪明有智慧。}。孔君平\myidx{孔坦}{\fzxk\zihao{6}\textcolor{red}{王隐\CJKunderwave{晋书}曰:“孔坦字君平,会稽山阴人。善\CJKunderwave{春秋},有文辩。历太子舍人,累迁廷尉卿。”}} 诣其父,父不在,乃呼儿出。为设果,果有杨梅。孔指以示儿曰:“此是君家果。”儿应声答曰:“未闻孔雀是夫子家禽\footnote{夫子:对男子的尊称,此指孔君平。}。”

{\cangkai\zihao{5}【评】本则见出孔君平的幽默和九岁杨氏子的聪慧。孔因杨梅而故意逗孩子,说这是你们杨家的水果。谁知小儿机灵,应口以同样的联想回驳了孔君平。如此敏捷巧对让人拍案称奇。这样的智趣是可以引起人们普遍、经久的欣赏的。所以,同一事情,张冠李戴,版本不少,未知孰是。余嘉锡\CJKunderwave{笺疏}列:\CJKunderwave{太平御览}引\CJKunderwave{郭子}为杨修与孔融作此对答。\CJKunderwave{太平广记}引\CJKunderwave{启颜录}作杨修与孔君平对答。敦煌本\CJKunderwave{残类书}亦作:“杨德祖少时与孔融对食梅。融戏曰:‘此君家果。’祖曰:‘孔雀岂夫子家禽?’”余先生谓:“皆一事传闻异辞。”影响这样广泛,可见人们对智慧、幽默的崇尚和喜爱。}

\lettrine{2.44} 孔廷尉\myidx{孔坦}以裘与从弟沉(沈)\myidx{孔沉}\footnote{孔廷尉:即孔坦。从弟:堂弟。沉:\CJKunderwave{晋书}作“沈”。},{\fzxk\zihao{6}\textcolor{red}{\CJKunderwave{孔氏谱}曰:“沉字德度,会稽山阴人。祖父弈(奕),全椒令。父群,鸿胪卿。沉至琅邪王文学。”}} 沉辞不受。廷尉曰:“晏平仲\myidx{晏婴}之俭,祠其先人,豚肩不掩豆,犹狐裘数十年\footnote{祠其先人:祭祀他的祖先。此刘孝标注引\CJKunderwave{礼记}下“晏平仲记其先人”,“记”诸本作“祀”,“祀”字是。豚肩:豚,小猪;肩,猪肘。此为祭品。豆:古食器,形似高脚盘,多有盖。祭祀时常用。},{\fzxk\zihao{6}\textcolor{red}{刘向\CJKunderwave{别录}曰:“晏平仲名婴,东莱夷维人。事齐灵公、庄公,以节俭力行重于齐。”\CJKunderwave{礼记}曰:“晏平仲记(祀)其先人,豚肩不掩豆,君子以为俭也。”又曰:“晏子一狐裘三十年,晏子焉知礼?”\CJKunderwave{注}:“豚,俎实也。豆,径尺。言并豚之两肩不能掩豆,喻少也。”}} 卿复何辞此?”于是受而服之。

{\cangkai\zihao{5}【评】\CJKunderwave{晋书}本传称孔坦“少方直,有雅望,通\CJKunderwave{左氏传},解属文”。是个深通情理的人,在吴兴内史任上,属地饥荒,他运自家米以赈穷乏。本则说了他深重兄弟情谊,画出一个“有雅望”者的一贯性格。

裘皮大衣向来为贵重之物,因而孔沉不敢轻易接受。于是,孔坦举前贤为例,裘乃身份的表征,与奢侈无关,晏平仲以节俭重于时,却不辞狐裘。先贤如此,堂弟大可欣然接受此赠。这一做法,就生活经验说,他解除了孔沉顾虑,效法先贤理所当然,同时也表达着孔坦殷勤至诚的心意。这里凸显着动人的真诚,言语不多,却令孔坦的形象栩栩如生。}

\lettrine{2.45} 佛图澄\myidx{佛图澄}与诸石游\footnote{诸石:指羯人石勒、石虎等,公元319—352年建立后赵,先后都襄国(今河北邢台西南)、邺城(今河北临漳西南)。},{\fzxk\zihao{6}\textcolor{red}{\CJKunderwave{澄别传}曰:“道人佛图澄,不知何许人,出于敦煌,好佛道,出家为沙门。永嘉中至洛阳,值京师有难,潜遁草泽。闻石勒雄异好杀害,因勒大将军郭默(黑)略见勒,以麻油涂掌,占见吉凶数百里外,听浮图铃声,逆知祸福。勒甚敬信之。虎即位,亦师澄,号‘大和尚’。自知终日。开棺无尸,唯袈裟法服存焉。”}} 林公曰:“澄以石虎\myidx{石虎}为海鸥鸟\footnote{林公:支遁。字道林,东晋名僧。刘孝标注“海鸥鸟”句云语出\CJKunderwave{庄子},然事见今本\CJKunderwave{列子·黄帝篇},今本\CJKunderwave{庄子}无,余嘉锡\CJKunderwave{笺注}以为:“刘\CJKunderwave{注}所引,(\CJKunderwave{庄子})逸篇之文也。\CJKunderwave{列子}伪书,袭自\CJKunderwave{庄子}耳。”}。”{\fzxk\zihao{6}\textcolor{red}{\CJKunderwave{赵书}曰:“虎字季龙,勒从弟也。征伐每斩将搴旗。勒死,诛勒诸儿,袭位。”\CJKunderwave{庄子}曰:“海上之人好鸥者,每旦之海上从鸥游。鸥之至者数百而不止。其父曰:‘吾闻鸥鸟从汝游,取来玩之。’明日之海上,鸥舞而不下。”}}

{\cangkai\zihao{5}【评】佛图澄自始识石勒于葛陂(今河南新蔡县北)军营,至石虎死的前一年逝去,计三十馀年与诸石“游”。(见\CJKunderwave{晋书}本传及\CJKunderwave{高僧传}卷九)

佛图以慈悲为怀,而“诸石”尽豺虎之性。石勒将兵征战,动辄活埋降卒,有时多达“万馀”,其姐夫与之“戏言”,勒大怒,“叱力士折其胫而杀之”;石虎更有甚者,自少年就“性残忍”,征战中“至于降城陷垒,不复断别善恶,坑斩士女,鲜有遗类”。对他自己的儿子也不手软,处置其子石宣,“以铁环穿其颔而锁之”,令人“拔其发,抽其舌”,最后“断其手足,斫眼溃腹”于积柴之上纵火焚烧,“烟炎际天”,而他自己率数千人登台以观之;其子辈更逾于此,曾被立为太子的石邃,“荒淫酒色,娇恣无道”,“装饰宫人美淑者,斩首洗血,置于盘上,传共视之。又内诸比丘尼有姿色者,与其交亵而杀之,合牛羊肉煮而食之,亦赐左右,欲以识其味也”;石斌、石宣、石韬、石鉴等等,无不荒淫贪狠,其行径令人发指,有甚于桀纣豺虎(见\CJKunderwave{晋书·载记}石勒、石虎本传)。佛图澄“悯念苍生,欲以道化勒”(\CJKunderwave{高僧传}),与诸石游,并降服了石勒、石虎,都称他为“大和尚”。以慈悲为怀的佛图澄,面对豺虎之性的“诸石”,并与之“游”,支道林将此比作\CJKunderwave{庄子}的“海鸥鸟”事。徐震堮\CJKunderwave{世说新语校笺}解说:“刘辰翁曰:‘谓玩虎于掌中耳。’案此语未允。盖谓澄以无心应物,故物我两忘也。”

云其“忘”,实不能“忘”,优游“诸石”豺虎之中,略有疏忽则后果可知。佛图澄以高僧之姿,降服群豺,正见其智慧之高和“悯念苍生”之诚。“海鸥鸟”之比,不过是更加突出地渲染了当时人们对佛图澄飘逸优游超凡之境的歆羡、叹美而已。}

\lettrine{2.46} 谢仁祖\myidx{谢尚}年八岁,谢豫章\myidx{谢鲲}{\fzxk\zihao{6}\textcolor{red}{鲲子别见。}} 将送客,尔时语已神悟,自参上流\footnote{谢豫章:谢鲲,曾作豫章太守。刘孝标注“鲲子别见”,“子”字衍。将:携,谓携之送客。自:已经。参:参与、进入。上流:上等、上品。}。诸人咸共叹之曰:“年少一坐之颜回\myidx{颜回}\footnote{坐:同“座”。颜回:即颜渊,孔子门下最出色的弟子,以德行著称。}。”仁祖曰:“坐无尼父,焉别颜回\footnote{尼父:孔子。孔子字仲尼,其死,鲁哀公在诔文中称之“尼父”(见\CJKunderwave{礼记·檀弓上})。父:古代男子之美称。}?”{\fzxk\zihao{6}\textcolor{red}{\CJKunderwave{晋阳秋}曰:“谢尚字仁祖,陈郡人,鲲之子也。龆龀丧兄,哀恸过人。及遭父丧,温峤喭之,尚号叫极哀。既而收涕告诉,有异常童。峤奇之,由是知名,仕至镇西将军、豫州刺史。”}}

{\cangkai\zihao{5}【评】谢尚为早慧英才,\CJKunderwave{晋书}所谓“神悟夙成”,王导曾把他比作王戎,呼为“小安丰”。故事的确见其颖秀绝伦的辩悟,这一点,和昔日以颖秀辩悟负有盛名的王戎可以相提并论。“戎每与(阮)籍为竹林之游,戎尝后至。籍曰:‘俗物已复来败人意。’戎笑曰:‘卿辈意亦复易败耳!’”领悟妙捷,随口回答富于机辩诙谐;本则与王戎之辩如出一辙,童言无忌,但其自信与机辩的快利又颇有名士味道,因而显得精彩动人。凭这根基素质,年八岁“自参上流”,在当时风尚之下,前途无可限量。}

\lettrine{2.47} 陶公\myidx{陶侃}疾笃,都无献替之言,朝士以为恨\footnote{献替:“献可替否”的略语。对君主进献可行之人,除去不可行之人。为直言谏诤。恨:遗憾。刘孝标注“侃字士衡”,\CJKunderwave{晋书}本传作“士行”;又,“刘弘镇江南”,袁本作“刘弘镇沔南”,“沔南”是。}。{\fzxk\zihao{6}\textcolor{red}{\CJKunderwave{陶氏叙}曰:“侃字士衡,其先鄱阳人,后徙寻阳。侃少有远概,网(纲)维宇宙之志。察孝廉,入洛,司空张华见而谓曰:‘后来匡主宁民,君其人也!’刘弘镇江(沔)南,取为长史。谓侃曰:‘昔吾为羊太傅参佐,见语云:“君后当居身处。”今相观,亦复然矣。’累迁湘、广、荆三州刺史,加羽葆鼓吹,封长沙郡公、大将军,替(赞)拜不名,剑履上殿,进太尉,赠大司马,谥桓公。”按王隐\CJKunderwave{晋书}载侃临终表曰:“臣少长孤寒,始愿有限,过蒙先朝历世异恩。臣年垂八十,位极人臣,启手启足,当复何恨?但以馀寇未诛,山陵未复,所以愤慨兼怀,唯此而已。犹冀犬马之齿,尚可少延,欲为陛下北吞石虎,西诛李雄。势遂不振,良图永息。临书扼腕,涕泗横流。伏愿遴选代人,使必得良才,足以奉宣王猷,遵成志业。则虽死之日,犹生之年。”有表若此,非无献替。}} 仁祖\myidx{谢尚}闻之\footnote{仁祖:谢尚,见前篇。},曰:“时无竖刁,故不贻陶公话言\footnote{贻:遗留。话言:言论。}。”{\fzxk\zihao{6}\textcolor{red}{\CJKunderwave{吕氏春秋}曰:“管仲病,桓公问曰:‘子如不讳,谁代子相者?竖刁何如?’管仲曰:‘自宫以事君,非人情,必不可用。’后果乱齐。”}} 时贤以为德音\footnote{德音:善言。}。

{\cangkai\zihao{5}【评】\CJKunderwave{晋书·陶侃传}载,咸和七年六月侃疾笃,上表,此表前段如刘孝标引,后段即推赞王导、郗鉴、庾亮等几位当时名臣,谓“器用周时,即陛下之周(公)、召(公)也”。如刘孝标注,不可谓“都无献替之言”。然此则动人处在于仁祖之言。无“替”言者,因朝无竖刁之类乱亡之臣。与齐桓公比,晋有群贤当朝,陶公无管仲之忧,故不必诤谏。在东晋门阀政治中,平衡诸高门世族利益,稳定朝廷以图国家发展,是当时士人的共同心理要求。而仁祖这番话,既解陶公,又嘉时贤,于陶公、于时贤、于当朝皆大欢喜,可谓“德音”正合人心需求,于此可见仁祖言语之巧。}

\lettrine{2.48} 竺法深\myidx{竺潜}在简文\myidx{司马昱}坐,刘尹\myidx{刘惔}问\footnote{竺法深:竺潜,字法深,俗姓王,年十八出家,为晋代名僧。简文:即简文帝司马昱。据\CJKunderwave{高僧传},时简文为会稽王、丞相。刘尹:即刘惔,字真长,曾任丹阳尹,故称。谢安妻兄,尚明帝女庐陵公主。会稽王司马昱为相,与王濛并为其座上清谈之客。性简贵自重,与王羲之友善。卒年三十六。}:“道人何以游朱门\footnote{道人:六朝时称僧人为道人,僧人亦自谦称为“贫道”,意“谓我寡少此道”。朱门:红漆的大门,指达官显贵之家。}?”答曰:“君自见其朱门,贫道如游蓬户\footnote{蓬户:编蓬草做成的门,指贫苦寒门之家。}。”{\fzxk\zihao{6}\textcolor{red}{\CJKunderwave{高逸沙门传}曰:“法师居会稽,皇帝重其风德,遣使迎焉。法师暂出应命。司徒会稽王天性虚澹,与法师结殷勤之欢。师虽升履丹墀,出入朱邸,泯然旷达,不异蓬宇也。”}} 或云卞令\myidx{卞壸}\footnote{卞令:卞壸,徐震堮\CJKunderwave{世说新语校笺}以为简文执政时,卞壸已死四十馀年,故断非卞壸。}。{\fzxk\zihao{6}\textcolor{red}{别见。}}

{\cangkai\zihao{5}【评】简文、刘惔皆善清谈,惔“雅善言理”,惔为简文“谈客”,“蒙上宾礼”(\CJKunderwave{晋书·刘惔传})。名僧竺法深游简文处也是为了清谈。名家凑泊,于是有了这样的对白。刘惔因佛家尚“空”,而戏问道人何以如俗间风气,奔走权门。言下之意,还是有所执着。道人之答,也很精妙,正因一切皆空,所以“朱门”、“蓬户”无复差异。在彼是以俗见观察,在我是以佛道行事,各行其是,两不相碍,因而,“道人何以游朱门”之问,毫无意义。本则可见,言语智巧,不唯天资聪颖,也来源于学养。名僧博学道深,应声而辩,机锋劲健,令“谈客”之问难,顿时化为妄诞。}

\lettrine{2.49} 孙盛\myidx{孙盛}为庾公\myidx{庾亮}记室参军\footnote{记室参军:官名,掌表章、文书的幕僚。},{\fzxk\zihao{6}\textcolor{red}{\CJKunderwave{中兴书}曰:“盛字安国,太原中都人。博学强识,历著作郎、浏阳令。庾亮为荆州,以为征西主簿,累迁秘书监。”}} 从猎,将其二儿俱行,庾公不知,忽于猎场见齐庄\myidx{孙放}\footnote{齐庄:孙放,字齐庄。孙盛次子。},时年七八岁,庾谓曰:“君亦复来邪?”应声答曰:“所谓‘无小无大,从公于迈’\footnote{“无小无大”二句:出\CJKunderwave{诗经·鲁颂·泮水}。小、大,指官职大小。公:鲁僖公。于:无实义。迈:出行。此句原义谓,不论官职大小,都跟随鲁僖公出行。}。”

{\cangkai\zihao{5}【评】孙盛“博学,善言名理”,是当时的清谈名家,也是著名史家,所著\CJKunderwave{晋阳秋},时人赞为“良史”。这样一个有着文化、学问氛围的家学环境,自然对孩子在知识、文化、智力等方面的增进、开发大有裨益。所以,孙放才七八岁就能够出语不凡。他应声回答庾亮的问话,不仅见其机敏,也见出学养。他所称引的\CJKunderwave{诗经}作品,是颂美鲁僖公武功和威德的诗篇。诗中渲染鲁僖公能文能武,征服淮夷,威风凛凛,有才干,有美德,是人们效仿的楷模,大小官员,都乐于追随。庾亮当时为六州都督,领江、荆、豫三州刺史,进号征西将军,是东晋王朝顶梁柱般的人物,为朝野所瞩望。颂美鲁僖公的诗,正好可以移用来赞美庾亮。何况孙放又巧做意转,把原诗写官职的大小,转换成年龄的大小,使引诗应对具有独特的幽默。尽管孙放引诗的时候,未见得有多少周全之想,只是孩童机灵敏捷的巧应,但听者庾亮,却可因原诗而感受更多。所以,孙放的这一回答,着实精彩。}

\lettrine{2.50} 孙齐由\myidx{孙潜}、齐庄\myidx{孙放}二人,小时诣庾公\myidx{庾亮}\footnote{诣:拜访。}。公问齐由何字,答曰:“字齐由。”公曰:“欲何齐邪?”曰:“齐许由\footnote{齐:等同。许由:许、父:即许由、巢父,皆尧时的隐士。尧让君位给许由,由不受,又召其为九州长,由不欲闻,洗耳于颍滨。巢父饮牛,见其洗耳,问其故,而云:“污我犊口。”牵牛上流而饮。见皇甫谧\CJKunderwave{高士传}。}。”{\fzxk\zihao{6}\textcolor{red}{\CJKunderwave{晋百官名}曰:“孙潜字齐由,太原人。”\CJKunderwave{中兴书}曰:“潜,盛长子也。豫章太守殷仲堪下讨王国宝,潜时在郡,逼为谘议参军,固辞不就,遂以忧卒。”}} “齐庄何字?”答曰:“字齐庄。”公曰:“欲何齐?”曰:“齐庄周\footnote{庄周:即庄子。名周,战国时人,与老子同为道家学派的代表人物,崇尚天道自然,主张清静无为。\CJKunderwave{庄子}一书记载其主张。}。”公曰:“何不慕仲尼而慕庄周\footnote{仲尼:即孔子。}?”对曰:“圣人生知,故难企慕\footnote{圣人生知:\CJKunderwave{论语·季氏}:“生而知之者,上也;学而知之者,次也。”孙放此言是说,圣人不学而知,一般人只能学而知之,所以,圣人是很难企慕的。}。”庾公大喜小儿对。{\fzxk\zihao{6}\textcolor{red}{\CJKunderwave{孙放别传}曰:“放字齐庄,监君次子也。年八岁,太尉庾公召见之。放清秀,欲观试,乃授纸笔令书,放便自疏名字。公题后问之曰:‘为欲慕庄周邪?’放书答曰:‘意欲慕之。’公曰:‘何故不慕仲尼,而慕庄周?’放曰:‘仲尼生而知之,非希企所及;至于庄周,是其次者,故慕耳。’公谓宾客曰:‘王辅嗣应答恐不能胜之。’卒长沙王相。”}}

{\cangkai\zihao{5}【评】魏晋清谈之风的理论根据即是\CJKunderwave{易}、\CJKunderwave{老}、\CJKunderwave{庄},法自然,任天真,是时代思潮的风尚。风尚所及,连初识学行的儿童也濡染浸润,这里虽是庾亮与两小儿开的玩笑,但却充溢着清谈味道。庾亮“善谈论,性好\CJKunderwave{庄}、\CJKunderwave{老}”(\CJKunderwave{晋书·庾亮传});两小儿应对聪敏,大有“谈论”慧根,也崇尚\CJKunderwave{庄}、\CJKunderwave{老},性喜许由的任天真及庄子的任自然,正合玄学品格,因而庾亮“大喜小儿对”。}

\lettrine{2.51} 张玄之\myidx{张玄之}、顾敷\myidx{顾敷}是顾和\myidx{顾和}中外孙,皆少而聪惠,和并知之,而常谓顾胜\footnote{中外孙:孙子和外孙。儿子所生为中,女儿所生为外。聪惠:即聪慧,“惠”通慧。胜:优。}。亲重偏至,张颇不厌\footnote{亲重:亲近、爱重。偏:偏向,侧重一方。厌:满足,满意。}。{\fzxk\zihao{6}\textcolor{red}{敷别见。\CJKunderwave{续晋阳秋}曰:“张玄之字祖希,吴郡太守澄之孙也。少以学显,历吏部尚书,出为冠军将军、吴兴太守、会稽内史。谢玄同时之郡,论者以为南北之望。玄之名亚谢玄,时亦称‘南北二玄’。卒于郡。”}} 于时,张年九岁,顾年七岁。和与俱至寺中,见佛般泥洹像\footnote{般(bō波)泥洹(huán环):梵语音译,也译作“般涅槃”,简称“涅槃”,意译为“圆寂”,佛教所谓德备障尽,脱离一切烦恼,获得自由无碍境界。僧人之死也称“涅槃”、“圆寂”。},弟子有泣者,有不泣者。和以问二孙。玄谓:“彼亲故泣,彼不亲故不泣”\footnote{“彼亲”二句:袁本做“被亲故泣,不被亲故不泣”。}。敷曰:“不然。当由忘情故不泣,不能忘情故泣\footnote{此句刘孝标注“二学”,诸本作“三学”,佛家诫、定、慧为三学;一说佛家谓初果、二果、三果为三学人。}。”{\fzxk\zihao{6}\textcolor{red}{\CJKunderwave{大智度论}曰:“佛在阴庵罗双树间,入般涅槃,床北首,大地震动。诸二(三)学人佥然不乐,郁伊交涕。诸无学人,但念诸法,一切无常。”}}

{\cangkai\zihao{5}【评】顾和偏向孙子而外孙不满。寺院一问,其动因或是顾和有意让孙子展露才能来平复外孙的心理,但就两人的回答看,真是各有特色,难分高下。外孙的回答聪明并含着率真的童趣。虽是最崇敬的人死去,人们也会依亲情的感受程度而有不同的情感表达,这是人之常情。言下也借题表达了对外公偏向的抗议。顾敷的回答,故作高深。“圣人忘情,最下不及情”是清谈家们的玄学命题。忘情,是要修炼到极高的境界,才会对喜怒哀乐之事无动于衷。小儿在此情景能以这样的话头来应对,固然表达了聪明和才学,但年七岁讲如此玄深的话题,已无童趣。如果顾敷真的修炼到这样的境界,却也是顾和的悲哀,妄用了亲情苦心。综观两答,诚如李贽评价:“俱胜。俱有规讽。”}

\lettrine{2.52} 庾(康)法畅\myidx{法畅}造庾太尉\myidx{庾亮},握麈尾至佳\footnote{庾法畅:\CJKunderwave{高僧传}卷四作康法畅,所记与本则同。麈尾:\CJKunderwave{世说音释}:“鹿之大者曰麈,群鹿从之,视麈尾所传而往,故谈者挥焉。”其形制似羽扇,上圆下平,附以长毫毛。}。公曰:“此至佳,那得在?\footnote{在:留存。}”法畅曰:“廉者不求,贪者不与,故得在耳。”{\fzxk\zihao{6}\textcolor{red}{法畅,氏族所出未详。法畅箸\CJKunderwave{人物论},自叙其美云:“悟锐有神,才辞通辩。”}}

{\cangkai\zihao{5}【评】据\CJKunderwave{高僧传},这位康法畅也是健谈名僧,还著有\CJKunderwave{人物始义论}。他“常执麈尾行,每值名宾,辄清谈尽日”。麈尾,不仅标志着名士的雅致,也因是群鹿所瞻,清谈家挥动指授而谈,便具有领袖群伦的意味,因而它是清谈家所喜爱之仪饰。孙盛与殷浩两个大名家,谈辩不休,情急之中还“掷麈尾”,以至毫毛都落到了饭上。康法畅所执麈尾好而能一直留存,这在当时是有一点显眼的,所以庾亮要发疑问。名僧之答也别有意味。方外之人,不理俗家的一套风气,廉者自然不求,贪者欲求,超拔不理,恰也是一派名士风度。}

\lettrine{2.53} 庾稚\myidx{庾翼}恭为荆州{\fzxk\zihao{6}\textcolor{red}{,\CJKunderwave{庾翼别传}曰:“翼字稚恭,颍川鄢陵人也。少有大度,时论以经略许之。兄太尉亮薨,朝议推才,乃以翼都督七州,进征南(西)将军、荆州刺史。”}} 以毛扇上武帝,武帝疑是故物\footnote{庾稚恭:见刘孝标注。庾翼为庾亮弟。据\CJKunderwave{晋书}卷七十二,庾亮另一弟庾怿献白羽扇给成帝司马衍,其事与本则所记相同。考庾翼生于305年,值晋惠帝永兴二年,去武帝之卒已十馀年,献扇事不当为庾稚恭与武帝事,当如\CJKunderwave{晋书},为庾怿献白羽扇给成帝。故物:旧东西。}。{\fzxk\zihao{6}\textcolor{red}{傅咸\CJKunderwave{羽扇赋}序曰:“昔吴人直截鸟翼而摇之,风不减方、圆二扇,而功无加。然中国莫有生意者。灭吴之后,翕然贵之,无人不用。”按庾怿以白羽扇献武帝,帝嫌其非新,反之。不闻翼也。}} 侍中刘劭\myidx{刘劭}曰\footnote{侍中:官名。}:{\fzxk\zihao{6}\textcolor{red}{\CJKunderwave{文字志}曰:“劭字彦祖,彭城谏亭人。祖讷,司隶校尉。父松,成皋令。劭博识好学,多艺能,善草隶。初仕领军参军。太傅出东,劭谓京洛必危,乃单马奔扬州。历侍中、豫章太守。”}} “柏梁云构,工匠先居其下,管弦繁奏,钟、夔先听其音\footnote{柏梁:汉代台名,汉武帝时建造,故址在陕西长安城内。武帝尝置酒台上,诏群臣和诗,能七言者乃得上。云构:高耸入云的屋宇楼台。构,建构。钟夔:钟,钟子期,春秋时楚人,精通音乐;夔,舜时乐官。钟、夔合称,泛指古代精通音乐的人。}。{\fzxk\zihao{6}\textcolor{red}{钟,钟期也。夔,舜乐正。}} 稚恭上扇,以好不以新。”庾后闻之曰:“此人宜在帝左右。”

{\cangkai\zihao{5}【评】羽扇流行于当时,一者以实用,二来也成了艺术品,颇有欣赏价值。既流行,便会有些精品为人所珍爱。朝臣将自己的爱物奉献给皇帝,这在情理之中,如同楚人献曝,献的是一番心意。可皇帝挑剔,就非同小可。本则的意味,在于刘劭的解释。他巧譬疏解,入情入理,将奉献者的心意委婉引发出来,平息了皇帝的不快,免去了臣子的祸端。而这一切,又绝非是讨好献者,可见这位侍中的才能和品行,不止言语机智、学养广博,人品也淳厚动人,所谓“君子成人之美,不成人之恶”。这里的一席解说,使刘劭儒雅淳厚的君子形象,生动可感。}

\lettrine{2.54} 何骠骑\myidx{何充}亡后,{\fzxk\zihao{6}\textcolor{red}{何充别见。}} 征褚公\myidx{褚裒}入\footnote{何骠骑:何充,字次道,晋康帝时为骠骑将军。褚公:即褚裒。褚公:对褚裒的敬称。褚裒(póu 抔)(303—349),晋康帝皇后之父,朝廷议以“不臣之礼”,力辞执政,而赴外镇。官征北大将军。曾率军三万北伐,败后上疏自贬,忧慨发愤而卒。见\CJKunderwave{晋书·外戚传}。}。既至石头,王长史\myidx{王濛}、刘尹\myidx{刘惔}同诣褚\footnote{石头:城名。在建康西二里,地形险固,为军事要冲。王长史:王濛,字仲祖,为司徒左长史。刘尹:即刘惔,字真长,曾任丹阳尹,故称。谢安妻兄,尚明帝女庐陵公主。会稽王司马昱为相,与王濛并为其座上清谈之客。性简贵自重,与王羲之友善。卒年三十六。},褚曰:“真长,何以处我?”真长顾王曰:“此子能言。”因视王,王曰:“国自有周公\footnote{周公:西周成王之叔,名姬旦。他辅佐年幼的成王,代掌朝政,平定叛乱,建立制度,巩固了周朝政权。参刘孝标注,此指会稽王司马昱。刘孝标注“于是固辞归京”,“京”下脱一“口”字。当时褚裒镇京口。}。”{\fzxk\zihao{6}\textcolor{red}{\CJKunderwave{晋阳秋}曰:“充之卒,议者谓太后父裒宜秉朝政。裒自丹徒入朝,吏部尚书刘遐劝裒曰:‘会稽王令德,国之周公也,足下宜以大政付之。’裒长史王胡之亦劝归蕃,于是固辞归京。”}}

{\cangkai\zihao{5}【评】这一则故事,言语不见得精彩,但却颇见分量;褚公的判断、做法,则更令人钦服。

褚裒是康皇后的父亲,这时康皇帝的儿子穆帝二岁即皇帝位,朝中可谓六神无主,康皇后召褚公辅政,理所当然。然而,事情并不那么简单。何骠骑生前就坚决反对庾亮诸外戚掌政,皇帝也因舅氏干政而颇显尴尬。此番褚公到京,又是外戚问权。另一面,刘惔、王濛都是会稽王司马昱的老友,而会稽王是穆帝的本家。所以刘惔、王濛主动见褚公说项,意谓会稽王是辅政最佳人选,你外戚就不要再乱了司马家事。褚公是有器识的人,渊默有城府,早有“皮里阳秋”之评,听刘、王之言,也就固辞而还镇京口。在这样一些复杂的关系中,王长史的一句“国自有周公”,抵得上千军万马,了却了一场血腥的权力争斗。}

\lettrine{2.55} 桓公\myidx{桓温}北征\footnote{桓公北征:桓温曾有三次北征,刘盼遂\CJKunderwave{世说新语校笺}考订,此次当为太和四年(369)之征。时桓温已58岁。},经金城\footnote{金城:地名,在今江苏句容北。东晋成帝咸康元年割丹阳郡之江乘县设琅邪侨郡,治所在金城。},见前为琅邪时种柳,皆已十围\footnote{围:量词。两臂合抱的圆周长。},慨然曰:“木犹如此,人何以堪!”攀枝执条,泫然流泪\footnote{泫(xuàn眩)然:流泪的样子。}。{\fzxk\zihao{6}\textcolor{red}{\CJKunderwave{桓温别传}曰:“温字元子,谯国龙亢人,汉五更桓荣后也。父彝,有识鉴。温少有豪迈风气,为温峤所知。累迁琅邪内史,进征西大将军,镇西夏。时逆胡未诛,馀烬假息。温亲勒郡卒,建旗致讨,清荡伊、洛,展敬园陵。薨,谥宣武侯。”}}

{\cangkai\zihao{5}【评】东晋朝廷据半壁江山,始终风雨飘摇,豪雄之士对此懦弱王朝多隐有问鼎之心。桓温自幼即被目为“英物”,豪爽有风概,多略果行。十五岁为父仇,枕戈泣血,志在必报,后果手刃仇家三兄弟。其志其略,时人比之为孙仲谋、司马宣王之流亚。他青年起家,建功立勋,威震天下,面对如此王朝也志存问鼎,\CJKunderwave{晋书}说他:“以雄武专朝,窥觎非望。”可见,这豪雄高爽的桓温并不掩饰自己的勃勃野心。刘盼遂\CJKunderwave{世说新语校笺}考订:“海西公太和四年,温发姑孰伐燕。金城泣柳事,当在太和四年之行。由姑孰赴广陵,金城为所必经。攀枝流涕,当此时矣。”温于成帝咸康七年(341)做琅邪内史,至此已三十年,自己也由英发时节而届暮年,日月逾迈,志向未果,虽云北伐可以树威,继续进取,然而,毕竟人已垂垂向晚。本则正是以这样的内蕴显现着怆然悲慨。刘辰翁评曰:“写得沉至,正在后八字耳。若止于桓大口语,安得如此惨怆?”其实,在这里,“桓大口语”与后八字并重。魏晋任情酗酒,风流雅望大多未及五六十岁即凋谢,桓温虽“性简”、慎酒,可年望花甲也已自觉时不我待,望此见证岁月无情流逝的柳树,其心中的慨怨诉之于口语,正可说是流泻无馀;后八字的描写,因将自感英雄迟暮,不堪衰迈之情态写得真切,故而此情此景方显出惨怆沉至的情感分量,两者舍其一则无此表现深度。于是,它也便由个别而一般,在这一点上,牵起了后世人们的情怀。}

\lettrine{2.56} 简文\myidx{司马昱}作抚军时\footnote{简文:晋简文帝司马昱,晋简文:指晋简文帝司马昱(320—372),穆帝年幼即位,昱任抚军大将军总理政务。后来大将军桓温专擅朝政,先废海西公,后立司马昱为帝,第二年崩。抚军:将军的称号,即抚军大将军。},尝与桓宣武\myidx{桓温}俱入朝\footnote{桓宣武:桓温,谥宣武。},更相让在前\footnote{更相:互相。},宣武不得已而先之,因曰:“伯也执殳,为王前驱\footnote{“伯也”二句:\CJKunderwave{诗经·卫风·伯兮}句。伯,长兄;殳(shū书),兵器,长一丈二尺,有棱无刃。原诗为妻称颂丈夫之辞。}。”{\fzxk\zihao{6}\textcolor{red}{\CJKunderwave{卫诗}也。殳,长一丈二尺,无刃。}} 简文曰:“所谓‘无小无大,从公于迈’\footnote{“无小”二句:出\CJKunderwave{诗经·鲁颂·泮水}。小、大,指官职大小。}。”

{\cangkai\zihao{5}【评】时司马昱以会稽王进位抚军大将军、录尚书六条事,辅佐朝政;桓温为大司马掌控实权,而年略长于司马昱,曾被封为临贺郡公、南郡公。

本则的一番对答,尽显两人智趣。桓温虽雄心勃勃,立大功而收时望,咄咄逼人,但司马氏的皇权并不是轻易可以取代的,他仍得拿出臣子模样。所以谦恭其貌,一面“相让”,一面脱口而出,表示自己只是王家劲健奋勇的走卒。而俏皮的是,此情此景,在辞面上又正符合司马昱会稽王的称谓及自己略长于司马昱的臣子身份,一句“伯也执殳,为王前驱”,既富学问,而又妙趣无穷。司马昱应声之对,更见敏捷。迫于桓温的强势,不但自己的地位受其左右,就是整个王朝也要仰仗这位手握重权的“大司马”。所以司马昱更其谦恭,不但“让”,而且将桓温比于鲁僖公之领袖群伦,不论官职大小都乐于追随。此句在辞面上也正和了桓温“公”的称谓。脱口引用古\CJKunderwave{诗}成句,能达到如此境地,确如刘辰翁所评:是“两用各极其致”的“捷然天对”。}

\lettrine{2.57} 顾悦\myidx{顾悦}与简文\myidx{司马昱}同年\footnote{顾悦:见刘孝标注,\CJKunderwave{晋书}作顾悦之。},而发蚤白\footnote{蚤:通“早”。},{\fzxk\zihao{6}\textcolor{red}{\CJKunderwave{中兴书}曰:“悦字君叔,晋陵人。初为殷浩扬州别驾。浩卒,上疏理浩。或谏以浩为太宗所废,必不依许。悦固争之,浩果得申,物论称之。后至尚书左丞。”}} 简文曰:“卿何以先白?”对曰:“蒲柳之姿\footnote{蒲柳:柳树的一种,又名水杨、蒲杨,秋天早凋。},望秋而落\footnote{望:临近。};松柏之质,凌霜犹茂\footnote{犹:尚。}。”{\fzxk\zihao{6}\textcolor{red}{顾凯(恺)之为父传曰:“君以直道,陵迟于世。入见王,王发无二毛,而君已斑白。问君年,乃曰:‘卿何偏早白?’君曰:‘松柏之姿,凌霜犹茂;臣榆柳之质,望秋先零。受命之异也。’王称善久之。”}}

{\cangkai\zihao{5}【评】顾悦的回答,是一个精妙的譬喻。在词面上联类不俗,取譬皆为牵人遐想、深有意味的美景美物,两物相对比之中又含着谦逊儒雅;在词底却张扬着个性,即刘孝标注中所谓“受命之异”。味其辞气,顾悦并不因象征着衰老的早生华发去叹老嗟卑,而是从容妙对,神情自若。在这巧喻妙对中,应答者的内在个性、气质之美,溢于纸上。}

\lettrine{2.58} 桓公\myidx{桓温}入峡\footnote{桓公入峡:永和二年冬,桓温率军自江陵溯长江上行伐蜀,经三峡。入峡,进入三峡。},绝璧(壁)天悬\footnote{绝壁:陡峭的崖壁。天悬:耸入云天。},腾波迅急,{\fzxk\zihao{6}\textcolor{red}{\CJKunderwave{晋阳秋}曰:“温以永和二年,率所领七千馀人伐蜀,拜表辄行。”}} 乃叹曰:“既为忠臣,不得为孝子,如何\footnote{如何:等于说“奈何”,即“怎么办?”}?”{\fzxk\zihao{6}\textcolor{red}{\CJKunderwave{汉书}:“王阳为益州刺史,行部至邛僰九折坂,叹曰:‘奉先人遗体,奈何数乘此险!’以病去官。后王尊为刺史,至其坂,问吏曰:‘非三(王)阳所畏之道邪?’吏曰:‘是。’叱其驭曰:‘驱之!王阳为孝子,王尊为忠臣。’”}}

{\cangkai\zihao{5}【评】当壮盛之年的桓温(时年34岁),豪气慨然,乘盘踞蜀地的李势形微势弱,便“志在立勋于蜀”,上疏朝廷,不等复诏就率兵而行。面对三峡天险的“绝壁天悬,腾波迅急”,他的话豪爽可爱,显出“英物”之气。

“身体发肤,受之父母”,不能毁伤谓之“孝”;冲锋陷阵,不惜性命,为国立勋谓之“忠”。对此两难的选择,桓温借“王阳为孝子,王尊为忠臣”(见刘孝标注引\CJKunderwave{汉书})的故事慨兴浩叹。就在这似乎无可如何之惆怅、叹息中,突显了他绝不却步的个性。不难体会,他的话,重心是落在“既为忠臣”一端的。其意气豪情鼓动人心,既显出率真的性情,而感召将士奋勇相从的“大略”,也可在无迹之迹中求之了。}

\lettrine{2.59} 初,荧惑入太微\footnote{荧惑:火星的别名,因隐显不定令人迷惑,故名。古人视为灾星。太微:星宿名,三垣之一,位于北斗之南,由十颗星组成。古人认为是天庭,对应人间的朝廷。},寻废海西\myidx{司马奕}\footnote{寻:不久。海西:即海西公司马奕,晋哀帝之同母弟,继哀帝即位,后被桓温废为海西公,另立简文司马昱为帝。其因由见刘孝标注。};{\fzxk\zihao{6}\textcolor{red}{\CJKunderwave{晋阳秋}曰:“泰和六年闰十月,荧惑守太微端门;十一月,大司马桓温废帝为海西公。”\CJKunderwave{晋安帝纪}曰:“桓温于枋头奔败,知民望之去也,乃屠袁真于寿阳。既而谓郗超曰:‘足以雪枋头之耻耳。’超曰:‘未厌有识之情也。公六十之年,败于大举;不建高世之勋,未足以镇厌民望。’因说温以废立之事。时温夙有此谋,深纳超言,遂废海西。”}} 简文\myidx{司马昱}登祚\footnote{登祚:即皇帝位。},复入太微,帝恶之。{\fzxk\zihao{6}\textcolor{red}{徐广\CJKunderwave{晋纪}曰:“咸安元年十二月,荧惑逆行入太微,至二年七月,犹在焉。帝惩海西之事,心甚忧之。”}} 时郗超\myidx{郗超}为中书,在直\footnote{郗超:任桓温大司马,深得信任,立简文为帝后,迁中书侍郎,实代桓温监督朝廷而权重当时。在直:在宫中值班。}。{\fzxk\zihao{6}\textcolor{red}{\CJKunderwave{中兴书}曰:“超字景兴,高平人,司空愔之子也。少而卓荦不羁,有旷世之度。累迁中书郎、司徒左长史。”}} 引超入曰:“天命修短,故非所计。政当无复近日事否\footnote{政当:只应。“政”通“正”;当,表揣度的口气。近日事:指桓温废海西事。}?”超曰:“大司马\myidx{桓温}方将外固封疆\footnote{方将:正要。固:巩固,加强。封疆:疆界,引申为边防。},内镇社稷\footnote{镇:安抚。},必无若此之虑。臣为陛下以百口保之\footnote{百口:犹言一族人。保:担保。}。”帝因诵庾仲初\myidx{庾阐}诗\footnote{庾仲初:庾阐,字仲初,为散骑侍郎,领大著作,作\CJKunderwave{扬都赋},为世所重。诗:指庾仲初\CJKunderwave{从征诗}。}{\fzxk\zihao{6}\textcolor{red}{庾阐\CJKunderwave{从征诗}也。}} 曰:“志士痛朝危,忠臣哀主辱。”声甚悽厉。郗受假还东\footnote{受假:休假。东:指会稽,因在京城建康之东,故时人以东指会稽。},帝曰:“致意尊公\myidx{郗愔}\footnote{尊公:对别人父亲的敬称。此指郗超的父亲郗愔。愔忠于王室而超为桓温谋主,废立之事,即超与桓温始谋,见刘孝标注。},家国之事,遂至于此。由是身不能以道匡卫\footnote{是身:此身,犹言“我”。匡卫:匡正、护卫。},思患预防。愧叹之深,言何能喻\footnote{喻:表达。}!”因泣下流襟。{\fzxk\zihao{6}\textcolor{red}{\CJKunderwave{续晋阳秋}曰:“帝外厌疆(强)臣,忧愤不得志,在位二年而崩。”}}

{\cangkai\zihao{5}【评】简文帝的君位,处在外逼于北方异族的压力,内慑于权臣之强悍,朝夕不保,战战兢兢的尴尬境地。而他自己既无雄才亦乏胆识,谢安评价他是“惠帝之流”,虽未免过刻,但其政治才略平平却是实情。然而,他敏感、捷慧更具文人气质,颇有当时的名士风味。“荧惑入太微”使他时时想起海西公的遭遇,不寒而栗,与郗超语,可谓其心愁苦,几番言语,一意三叠。一则因郗超为桓温心腹,想通过郗超试探桓温实情。得到的回答只是桓温目前无暇虑及废立,郗超虽敢以“百口”担保,可简文并没有得到定心丸。二则诵诗,以性情侧讽郗超,试图唤起他为臣的良知。但是郗超在心底里早对司马王室失去信心,相信“天下之责将归于公(桓温)矣”(\CJKunderwave{晋书·郗超传}),作为桓温谋主,又怎会受此诗句的感动呢?三则想通过“致意”尽忠王室的“尊公”,来规讽这位敢谋王权废立的“不肖子”。然而,史称郗超与桓温之谋,对他的父亲一概保密,此时简文的良苦用心、哀哀之情又能起什么作用呢?本则故事,一个敏感、细腻,愁肠百结的尴尬君王的形象跃然纸上,显出了\CJKunderwave{世说}的精彩。简文帝一如魏晋名士,君王的独特身份,使他尴尬,然骨子里的性情,又使他的表达方式绝不是居高临下,深于谋算,而是风情摇曳,有着感人的人性。}

\lettrine{2.60} 简文\myidx{司马昱}在暗室中坐\footnote{简文:晋简文帝司马昱,晋简文:指晋简文帝司马昱(320—372),穆帝年幼即位,昱任抚军大将军总理政务。后来大将军桓温专擅朝政,先废海西公,后立司马昱为帝,第二年崩。},召宣武\myidx{桓温},宣武至\footnote{宣武:桓温,卒谥宣武。},问上何在\footnote{上:皇上,指简文帝。}。简文曰:“某在斯。”时人以为能\footnote{能:才能,此特指口才。}。{\fzxk\zihao{6}\textcolor{red}{\CJKunderwave{论语}曰:“师冕见,及阶,子曰:‘阶也。’及席,子曰:‘席也。’皆坐,子告之曰:‘某在斯,甚(某)在斯。’”注:“历告坐中人也。”}}

{\cangkai\zihao{5}【评】刘辰翁评本则说:“似讥不见也。”

简文故意导演这戏剧性的一幕,在他自己是寓有深意,而从旁观察则未免书生意气。\CJKunderwave{论语}是汉代以来的启蒙读物,可说是只要识字的人,没有不知\CJKunderwave{论语}的。孔子耐心引导盲人,遍告之,见其仁者之心。而简文用此,设局以讥桓温无视他司马氏皇权的存在及尊严,但是这对桓温说来,真同儿戏。“时人以为能”,是崇尚其聪明捷辩的口才,以为此情此景,幽默寄意,一语双关,实则可能弄巧成拙。无能的是作为在“上”的简文,而可爱的是作为才子的司马昱。就魏晋人物的才情风貌说来,本则动人处,是简文的童稚般的巧言,逞才使性而不计后果。}

\lettrine{2.61} 简文\myidx{司马昱}入华林园\footnote{华林园:宫苑名。西晋时洛阳有华林园,东晋在建康(今江苏南京市)仿洛阳名园修三国吴时旧宫苑,亦名华林园。},顾谓左右曰:“会心处不必在远\footnote{会心处:领悟、领会,心神交融之处。},翳然林水\footnote{翳然:林荫遮蔽的样子。},便自有濠、濮间想也\footnote{濠、濮:皆水名。濠,在安徽凤阳县东北。濮,古黄河济水的分流。见刘孝标注,因庄子的寓言,后人以“濠、濮”指高人隐士的逍遥垂钓之所。想:情怀。},{\fzxk\zihao{6}\textcolor{red}{濠、濮,二水名也。\CJKunderwave{庄子}曰:“庄子与惠子游濠梁水上。庄子曰:‘鲦鱼出游从容,是鱼乐也。’惠子曰:‘子非鱼,安知鱼之乐邪?’庄子曰:‘子非我,安知我之不知鱼之乐也?”庄周钓在濮水,楚王使二大夫造焉,愿以境内累庄子。庄子持竿不顾,曰:‘吾闻楚有神龟者,死已三千年矣,巾笥而藏于庙。此宁曳尾于涂中,宁留骨而贵乎?’二大夫曰:‘宁曳尾于涂中。’庄子曰:‘往矣!吾亦宁曳尾涂中。’”}} 不觉鸟兽禽鱼,自来亲人\footnote{不觉:袁本作“觉”。亲:亲近。}。”

{\cangkai\zihao{5}【评】作“濠、濮间想”,恐怕是简文发自内心的向往。内外重压,使得这位无力支撑下去的君王,想到庄子与惠子的濠上之乐。一声饱含复杂情感的喟叹,却恰好揭示了简文乃至魏晋人所具有的深刻的审美意识。只有挣脱了世俗功利,心境神思翳然入林水中,才能与丰富无限、动人情怀的山水林鸟作“会心”的共鸣,才能获得超然悠美的愉人感受。这是简文的深刻感悟,也是魏晋人山水审美的宣言,他“清言迳造”(刘辰翁语),一语道破。

就\CJKunderwave{世说·言语}来说,简文具有诗人气质的感受力,其脱口之语近乎诗意,自然优美,而更重要的恐怕还是此语引发时人共鸣,所以入选。}

\lettrine{2.62} 谢太傅\myidx{谢安}语王右军\myidx{王羲之}曰\footnote{谢太傅:谢安,谢奕(?—358):字无奕,谢安长兄,陈郡阳夏谢氏家族在东晋初期的代表人物之一。王右军:王羲之,见刘孝标注。}:“中年伤于哀乐\footnote{哀乐:悲哀和快乐,此偏指悲哀,伤感。},与亲友别,辄作数日恶\footnote{恶:指心境不好。}。”王曰:{\fzxk\zihao{6}\textcolor{red}{\CJKunderwave{文字志}曰:“王羲之字逸少,琅邪临沂人。父旷,淮南太守。羲之少朗拔,为叔父廙所赏。善草隶。累迁江州刺史,右军将军、会稽内史。”}} “年在桑榆\footnote{桑榆:落日馀晖所照桑树、榆树的顶端。转指日暮,用以比喻人的晚年。},自然至此,正赖丝竹陶写\footnote{丝竹:弦乐器和管乐器,借指音乐。陶写:写通“泻”,陶冶性情,宣泄忧闷。},恒恐儿辈觉,损欣乐之趣\footnote{欣乐:高兴,快乐。}。”

{\cangkai\zihao{5}【评】刘辰翁曰:“自家潦倒,忧及儿辈,真钟情语也。此少有喻者。”两位大名士,均为一代风流,引领雅望。作为政要,他们为王朝所倚重,各有建树,同时也饱尝了为政之艰难险厄;作为雅士,他们深通天命意味,从容游宴,诗酒管弦自娱,认真享受、品味当下生命,在这两方面他们都达到了时人仰望的境界。现在他们谈论老境感受。谢安之语,道着对人生、人情眷恋的深情,愈老愈敏感,愈老愈伤情,这种真情流泻,不由得不打动人心。王羲之语,以达人之观,体会人生,劝慰谢安。\CJKunderwave{易}曰:“日仄之离,不击缶而歌,则大耋之嗟,凶。”\CJKunderwave{易}告诫,垂老之人,若不顺其自然,作歌自娱,必将导致老暮穷衰的嗟叹,这将是凶兆。“赖丝竹陶写”以度桑榆晚景,正是王羲之深味\CJKunderwave{易}理奥旨,洞悟人生的妙语。但同时,他又细腻地想到,王、谢这鼎盛家族的富贵子弟,别因长辈的垂老的心态、娱逸作乐而产生错觉,年轻轻的就学着放纵,即\CJKunderwave{世说笺本}所谓:“常恐儿辈认我好之,遂亦仿效以为欣乐之具,为虑儿辈沉溺,致损我欣乐之趣。”这一念想,忧及儿辈,恐其不知世事艰难,放纵逸乐而毁了一生,如此,也就从根本上失去了长辈暮年“丝竹陶写”的欣乐之趣。右军之言,舐犊之情洋溢纸外,“真钟情语也”。

一番对话,真情真态种种,表达着魏晋人深情于人生的动人风采。}

\lettrine{2.63} 支道林\myidx{支遁}常养数匹马\footnote{支道林:见刘孝标注。为东晋名僧,善玄理,是当时佛学“般若学”的代表人物,多才艺,长于草隶。与王洽、刘惔、殷浩、许询、郗超、王羲之、谢安等名流游好。常:同“尝”,曾经。}。或言道人畜马不韵\footnote{道人:六朝称僧人为道人,僧人亦自称“贫道”。韵:风雅。}。支曰:“贫道重其神骏\footnote{神骏:骏逸有神采。刘孝标注中“沉思道术”,袁本作“沉思道行”。“行吟独畅”,袁本作“泠然独畅”。}。”{\fzxk\zihao{6}\textcolor{red}{\CJKunderwave{高逸沙门传}曰:“支遁字道林,河内林虑人。或曰陈留人,本姓关氏。少而任心独往,风期高亮,家世奉法。尝于馀抗(杭)山沉思道术,行吟独畅。年二十五,始释形入道。年五十三,终于洛阳。”}}

{\cangkai\zihao{5}【评】僧人在东晋舞台,本是雅人角色,优游方外,注解着魏晋风流。而“马,怒也,武也”(\CJKunderwave{说文解字}),自上古它就与粗武豪强、征战杀伐或达官显贵的高轩驷乘联系在一起。一句话,马为世俗之物。这一观念、心理,由来古远而且根深蒂固。沿此惯性,名僧养马,确给人们一些异样的心理感受,所以或言“不韵”。但对支道林说来,此乃地道的俗人之见。在支道林眼里,看到的,不是俗间那往来战场或引车就道的马,而是神飞骏放、自由奔逸的英物,是高蹈尘风之外,翩翩翱翔的心神寄托。刘辰翁评:“高视世外。”洵为的见。因修养气质而自有高韵的风雅神采,往往是难为一般人所理解的。这位高僧讲经时善掘精义而或遗章句,就为俗人所讥,只有谢安是其知音,评云:“此乃九方堙(皋)之相马也,略其玄黄,而取其骏逸也。”(\CJKunderwave{高僧传})其讲经风格与养马而赏会神骏一样,正见其人卓然独拔、善得天心的雅韵。}

\lettrine{2.64} 刘尹\myidx{刘惔}与桓宣武\myidx{桓温}共听讲\CJKunderwave{礼记}\footnote{\CJKunderwave{礼记}:又称小戴礼记,西汉博士戴圣编定,共四十九篇。为儒家经典之一,记载古代礼乐、仪节、教育思想等方面的内容。}。桓云:“时有入心处,便觉咫尺玄门\footnote{入心:会心,领悟。咫尺:八寸为咫,以咫尺喻距离极近。玄门:语出\CJKunderwave{老子}“玄之又玄,众妙之门”,后用以喻高深的境界。}。”刘曰:“此未关至极\footnote{关:关涉,到。至极:最高境界。},自是金华殿之语\footnote{自是:本是、只是。金华殿:西汉未央宫中殿名。见刘孝标注,此用“金华殿之语”指儒生讲经的老调常谈。}。”{\fzxk\zihao{6}\textcolor{red}{\CJKunderwave{汉书叙传}田(曰):“班伯少受诗于师丹。大将军王凤荐伯于成帝,宜劝学,召见宴昵,拜为中常侍。时上方向学,郑宽中、张禹朝夕入说\CJKunderwave{尚书}、\CJKunderwave{论语}于金华殿,诏伯受之。”}}

{\cangkai\zihao{5}【评】\CJKunderwave{礼记}是礼学论著,专门阐发典制仪礼之大义,涵盖着由形而下的礼制仪节到形而上之哲理大道,所以桓温听来会有所感悟而大发赞叹。但是,经生之论难免师守家法,循规蹈矩,泥于礼制仪节,是讲不出礼之妙谛精华的,所以刘惔听来有“金华殿语”味道。刘惔“每奇温才”,对桓温的政治才干有所认识,但两人气质风貌有很大差异。桓温志在经营天下而刘惔“性简贵”,“尤好\CJKunderwave{老}、\CJKunderwave{庄},任自然趣”,是风流才士。因而,桓温对节人之礼、整理群类的工具敏感欢喜,而刘惔不能满足于此,更希望听到对“礼”之精谛的发明。所以,刘惔以“高自标置”的自负与高傲,对经生的讲解予以尖刻的否定。两人的性情志趣,气质风貌,因志向需求不同而异其趣,这在对白中表现得清晰如画。}

\lettrine{2.65} 羊秉\myidx{羊秉}为抚军参军,少亡,有令誉\footnote{令誉:美好的声誉。}。夏侯孝若\myidx{夏侯湛}为之叙,极相赞悼\footnote{夏侯孝若:夏侯湛,字孝若。西晋有名的文士。叙:文体名,叙述、评价死者的生平。赞悼:赞美、哀悼。}。{\fzxk\zihao{6}\textcolor{red}{\CJKunderwave{羊秉叙}曰:“秉字长达,太山平阳人。汉南阳太守续曾孙。大父魏郡府君,即车骑掾元子也。府君夫人郑氏无子,乃养秉。龆龀而佳。小心敬慎。十岁而郑夫人薨,秉思容尽哀,俄而公府掾及夫人并卒,秉群从率礼相承,人不间其亲,雍雍如也。仕参抚军将军事,将奋千里之足,挥冲天之翼,惜乎春秋三十有二而卒。昔罕虎死,子产以为无与为善,自夫子之没,有子产之叹矣!亡后有子男,又不育,是何行善而祸繁也?岂非司马生之所惑欤?”}} 羊权\myidx{羊权}为黄门侍郎\footnote{黄门侍郎:官名,职任为侍从皇帝,传达诏命,掌门下事。},侍简文\myidx{司马昱}坐。帝问曰:“夏侯湛{\fzxk\zihao{6}\textcolor{red}{别见。}} 作\CJKunderwave{羊秉叙},绝可想\footnote{绝可想:绝,极;可想,可心、称意。}。是卿何物\footnote{何物:何人。物指人,晋人口语。},有后不?”{\fzxk\zihao{6}\textcolor{red}{\CJKunderwave{羊氏谱}曰:“权字道兴(舆),徐州刺史悦(忱)之子也。仕至尚书左丞。”}} 权潸然对曰\footnote{潸(shān山)然:泪流满面的样子。}:“亡伯令问夙彰\footnote{令问:即令闻,好声誉。夙:早。彰:显。},而无有继嗣\footnote{继嗣:后代。}。名播天听\footnote{天听:指皇上的听闻。},然胤绝圣世\footnote{胤(yìn印)绝:断绝后代。胤,后代。圣世:指当代。}。”帝嗟慨久之。

{\cangkai\zihao{5}【评】司马迁借伯夷事迹,深深指问所谓“天道无亲,常与善人”之说的真实性。论理,积善之家必有馀庆,积恶之家必有馀殃,而事实上并非尽皆如此,这一悖论的确给人以莫大的困惑。本则中,羊权的潸然应对,还偏重于对其伯父的痛惜,而简文则早已越出羊权话语所表达的具体感受,他的“嗟慨久之”,就不只是同情。这种“绝可想”的人物,竟落得“无有继嗣”的境地,它牵起简文心思的,怕还是印证了羊秉实情后,与\CJKunderwave{羊秉叙}的作者有着相同的感慨吧?眼前的事实和“司马生(迁)之所惑”(见刘孝标注引\CJKunderwave{羊秉叙})一起,撞击着简文心灵,引起了敏感而善究玄理的简文的更深的感想。本则含蓄地刻画了敏感而具有丰富内心世界的简文这一人物。这也恰是魏晋人的显著特点,在这里,简文不是君主而是一个生动的魏晋名士。}

\lettrine{2.66} 王长史\myidx{王濛}与刘真长\myidx{刘惔}别后相见\footnote{刘真长:刘惔,字真长,曾任丹阳尹,故称。谢安妻兄,尚明帝女庐陵公主。会稽王司马昱为相,与王濛并为其座上清谈之客。性简贵自重,与王羲之友善。卒年三十六。刘孝标注:“世为夫族”,纷欣阁本作“大”,“大”是。},{\fzxk\zihao{6}\textcolor{red}{\CJKunderwave{王长史别传}曰:“濛字仲祖,太原晋阳人。其先出自周室,经汉、魏,世为夫(大)族。祖父佐(佑),北军中侯。父讷,叶令。濛神气清韶,年十馀岁,放迈不群。弱冠检尚,风流雅正,外绝荣竞,内寡私欲。辟司徒掾、中书郎,以后父赠光禄大夫。”}} 王谓刘曰:“卿更长进。”答曰:“此若天之自高耳\footnote{若天之自高:就像天自然的高。语出\CJKunderwave{庄子·田子方}:“至人之于德也,不修而物不能离焉。若天之自高,地之自厚,日月之自明,夫何修焉!”}。”{\fzxk\zihao{6}\textcolor{red}{\CJKunderwave{语林}曰:“仲祖语真长曰:‘卿近大进。’刘曰:‘卿仰看邪?’王问何意,刘曰:‘不尔,何由测天之高也。’”}}

{\cangkai\zihao{5}【评】王濛、刘惔齐名友善,为当时清谈宗主,皆自视甚高。刘惔在此引\CJKunderwave{庄子}语以自我标榜,李慈铭批评曰:“人虽妄甚,无敢以天自比者。晋人狂诞,习为大言,所谓精理玄辞,大率摭袭佛、老浮文支语,眩惑愚蒙,盛自衿标,相为欺蔽。”(\CJKunderwave{世说新语汇校集注})“天之自高”,本无所谓“长进”,而蒙人夸赞“长进”,便以“天之自高”自我比况,目空一切,可见刘惔的回答确是“盛自矜标”。刘应登认为“皆戏语”,戏语如此,也见自负之狂。}

\lettrine{2.67} 刘尹\myidx{刘惔}云\footnote{刘尹:刘惔,字真长,曾任丹阳尹,故称。谢安妻兄,尚明帝女庐陵公主。会稽王司马昱为相,与王濛并为其座上清谈之客。性简贵自重,与王羲之友善。卒年三十六。}:“人想王荆产\myidx{王微}佳,此想长松下当有清风耳。\footnote{想:想象、推想。}”{\fzxk\zihao{6}\textcolor{red}{荆产,王微(徽)小字也。\CJKunderwave{王氏谱}:“微(徽)字幼仁,琅邪人。祖父乂,平北将军。父澄,荆州刺史。微(徽)历尚书郎、右军司马。”}}

{\cangkai\zihao{5}【评】刘惔“清远,有标奇”(\CJKunderwave{晋书}),这话也见出他“标奇”之论。琅邪王氏是负着盛名的家族,王微(\CJKunderwave{晋书}作“徽”)也出仕为官。按世俗的思维定式,名门贵胄之家,子弟定“佳”;但是却被刘惔一语道破,正如同人们习惯心理,以为长松之下当有清风一样,其实未必然。话语比喻清雅,道理发人深思,表达了刘惔这位名士的见识和放达个性。刘惔善作逆向思维,思想更为深刻。}

\lettrine{2.68} 王仲祖\myidx{王濛}闻蛮语不解\footnote{王仲祖:王濛。蛮:古时对南方少数民族的通称。},茫然曰\footnote{茫然:迷惑的样子。}:“若使介葛卢\myidx{介葛卢}来朝\footnote{介葛卢:见刘孝标注,春秋介国国君,传说能听懂兽语。},故当不昧此语\footnote{故当:肯定。不昧:懂得、明白。}。”{\fzxk\zihao{6}\textcolor{red}{\CJKunderwave{春秋传}曰:“介葛卢来朝鲁,闻牛鸣,曰:‘是生三牺,皆用之矣。其音云。’问之而信。”杜预注曰:“介,东夷国。葛卢,其君名也。”}}

{\cangkai\zihao{5}【评】古来列国分封的政治格局和小农经济的基础,使得商品交换、文化交流不能发达,这就决定了方言、方俗的顽强,客观上,它给人们的交流造成了很大的障碍。\CJKunderwave{孟子}就记有楚人学齐语的困难;有面对“南蛮鴂舌之人”的困惑。\CJKunderwave{左传}记载,同是北方的秦、魏,其语言也互不相通。汉代扬雄专门以当时的活语言为对象写了\CJKunderwave{方言}著作,以方便人们的言语沟通。可见,方言隔阂,由来已久。王濛山西晋阳人,南来江左吴地,吴方言与北方方言语音相去远甚,王濛听起来自然如入迷雾,如听鸟语,困惑不解。他对此感受,讲了一句类似幽默的俏皮话,比蛮夷吴人为鸟兽。这一面固然表现了这位大名士的敏慧,另一面,也见出其中的刻薄。史称王濛“有风流美誉,虚己应物,恕而后行”“王濛温润恬和”(\CJKunderwave{晋书·王濛传})。观本则语,见出作为累世贵族,其骨子里的高傲,面对非类,他并不“温润”也不“恬和”,更不“恕”,而是报以一种轻蔑的刻薄评价。一语表达了他骨子里的属性,也表达了魏晋名士的贵族属性。}

\lettrine{2.69} 刘真长\myidx{刘惔}为丹阳尹\footnote{刘真长:刘惔,字真长,曾任丹阳尹,故称。谢安妻兄,尚明帝女庐陵公主。会稽王司马昱为相,与王濛并为其座上清谈之客。性简贵自重,与王羲之友善。卒年三十六。丹阳在东晋是京师建康的门户,京都地区的行政长官称尹,丹阳尹相当于京畿地方长官。},许玄度\myidx{许询}出都就宿\footnote{出都:离京。刘孝标注中“司徒掾辟,大就,蚤卒”。袁本作“不就”,“不就”是。}。{\fzxk\zihao{6}\textcolor{red}{\CJKunderwave{续晋阳秋}曰:“许询字玄度,高阳人,魏中领军允玄孙。总角秀惠,众称祖(神)童,长而风情简素。司徒掾辟,大(不)就,蚤卒。”}} 床帷新丽\footnote{床帷:床铺帷幛。},饮食丰甘。许曰:“若保全此处,殊胜东山\footnote{东山:山名。在今浙江上虞市西南。谢安早年隐居于此,临安、金陵亦有东山,谢安也曾游憩,后因以东山代指隐居。}。”刘曰:“卿若知吉凶由人,吾安得不保此!”{\fzxk\zihao{6}\textcolor{red}{\CJKunderwave{春秋传}曰:“吉凶无门,唯人所召。”}} 王逸少\myidx{王羲之}在坐曰\footnote{王逸少:王羲之。按,\CJKunderwave{世说}诸本皆作“卿若知吉凶由人,安得不保此!”唯\CJKunderwave{晋书·王羲之传}记此作“卿若知吉凶由人,安得保此!”}:“今(令)巢、许遇稷、契,当无此言。”二人并有愧色\footnote{巢、许:即许由、巢父,皆尧时的隐士。稷、契(xiè谢):稷,周之始祖,名弃,发明农业,为尧之稷官。契,殷之始祖,舜之司徒,佐禹治水有功。后世以稷、契指称贤臣。}。

{\cangkai\zihao{5}【评】许询是隐士,\CJKunderwave{晋书}虽无传,然其名声却不小,常与谢安、王羲之、支遁等大名士往还。余嘉锡先生\CJKunderwave{世说新语笺疏}引\CJKunderwave{建康实录}八说他“幼冲灵,好泉石,清风朗月,举酒永怀”,皇家屡征不就,“策杖披裘,隐于永兴西山,凭树构堂,萧然自致”。可算是远俗的高雅隐士。然而本则所记,却是另一番景象。玄度艳羡刘惔的华屋美食,全然忘怀追求萧然自致的清高自由,表达了一副享乐庸人之趣的俗士面孔。刘惔亦如此,志在保全其既得的安乐窝。王羲之一席话,说破了两人思量“保全”的俗想,令其有愧色。若真如此,则一官一隐两名士皆虚伪得浊臭逼人。这是从“全人”的角度看名流。不过观史所录,并不如是。玄度舍宅为寺,家珍悉赠,倘艳羡享乐则不必舍宅散财,所以刘辰翁、王世懋对此记颇表怀疑。刘曰:“不谓真长、玄度有此谬谈。”王曰:“二君故复有此破绽邪?”或者此为小说家言,拿许询、刘惔这班大名士寓言述事,以鞭挞当时的假名士、假隐士之虚伪。}

\lettrine{2.70} 王右军\myidx{王羲之}与谢太傅\myidx{谢安}共登治(冶)城\footnote{王右军:王羲之。谢太傅:谢安,谢奕(?—358):字无奕,谢安长兄,陈郡阳夏谢氏家族在东晋初期的代表人物之一。冶城:故址在今江苏省朝天宫一带,三国时吴王孙权所筑。}。{\fzxk\zihao{6}\textcolor{red}{\CJKunderwave{扬州记}曰:“治(冶)城,吴时鼓铸之所。吴平,犹不废。王茂弘所治也。”}} 谢悠然远想,有高世之志。王谓谢曰:“夏禹勤王\footnote{夏禹勤王:禹是传说中著名的治水英雄,帮助虞舜治水,历时十三年,多次路过家门而不入。勤王:为王事尽力。},手足胼胝\footnote{胼胝(pián zhī骈枝):长老茧。};{\fzxk\zihao{6}\textcolor{red}{\CJKunderwave{帝王世纪}曰:“禹治洪水,手足胼胝。世传禹病偏枯,足不相过,今称‘禹步’是也。”}} 文王旰食\footnote{旰(ɡàn绀)食:天晚了才吃饭。旰,晚。},日不暇给\footnote{日不遐给(jǐ挤):时间不够用。}。{\fzxk\zihao{6}\textcolor{red}{\CJKunderwave{尚书}曰:“文王自朝至于日吴(昃),不遑暇食。”}} 今四郊多垒\footnote{四郊多垒:原指四外郊野都是营垒,战事紧逼。此指王朝不宁,到处是战火。},{\fzxk\zihao{6}\textcolor{red}{\CJKunderwave{礼记}曰:“四郊多垒,卿大夫之辱也。”}} 宜人人自效。而虚谈废务\footnote{虚谈:空谈,指不切实际的清谈。},浮文妨要\footnote{浮文:浮华不实的文辞。},恐非当今所宜。”谢答曰:“秦任商鞅,二世而亡,{\fzxk\zihao{6}\textcolor{red}{\CJKunderwave{战国策}曰:“卫鞅,卫诸庶孽子也,名鞅,姓公孙氏。少好刑名学,为秦孝公相,封于商。”}} 岂清言致患邪\footnote{清言:清谈。}?”

{\cangkai\zihao{5}【评】这里涉及对当时最盛行、最敏感的现象——玄言清谈的评介。谢安本人善玄理,能清言,出于对玄家清言之精神实质的真正了解,他用明白简洁的语言,回答了王羲之的批评。王氏所称“虚谈废务,浮文妨要”,如果针对当时士族贵要不务世事,不以国计为重的实际情况而言,是对的。但若把一国之政治危机,归罪于玄家之清言,则与历史实际不符。玄言仅是清言家之思想争鸣,是无能量把国家推向灭亡深渊的。事实正相反,魏晋清谈,恰是统治阶级的有识之士,针对汉代儒学之僵化,为王朝的统治、发展另觅一条新途径的理论争鸣。谢安针对王羲之的以史例证的批驳,也很有说服力。秦重法家,焚书坑儒,企图消灭百家争鸣,然而却由强而衰,二世而亡。“岂清言致患邪?”这一历史反思,很有说服力。他说明一个国家的兴亡,原因很复杂,不能把清谈与亡国画等号。在这个问题上,王、谢相比,谢的思考要更深入全面,更了解思想理论的运动与发展,更懂得思想理论建设对现实的意义。

刘应登:“右军之言真当时之药石。谢傅引秦喻晋,亦不类矣。”王世懋:“此在谢自为德音,然王是救时急务。”所评失于对清谈的全面省察。}

\lettrine{2.71} 谢太傅\myidx{谢安}寒雪日内集\footnote{谢太傅:谢安。内集:家庭内部集会。},与儿女讲论文义。俄而雪骤\footnote{骤:疾,急。},公欣然曰:“白雪纷纷何所似?”兄子胡儿\myidx{谢朗}曰:{\fzxk\zihao{6}\textcolor{red}{胡儿,谢朗小字也。\CJKunderwave{续晋阳秋}曰:“朗字长度,安次兄据之长子,安蚤知之。文义艳发,名亚于玄。仕至东阳太守。”}} “撒盐空中差可拟\footnote{差:略。拟:比。}。”兄女\myidx{谢道蕴}曰:“未若柳絮因风起。”公大笑乐,即公大兄无弈\myidx{谢奕}女\footnote{大兄无弈女:“弈”袁本作“奕”,“奕”是。谢奕,字无奕,谢安兄,谢奕(?—358):字无奕,谢安长兄,陈郡阳夏谢氏家族在东晋初期的代表人物之一。无奕女即谢道韫,名韬元,聪慧有才辩,人称有林下风。},左将军王凝之\myidx{王凝之}妻也。{\fzxk\zihao{6}\textcolor{red}{\CJKunderwave{王氏谱}曰:“凝之字叔平,右将军羲之第二子也。历江州刺史、左将军、会稽内史。”\CJKunderwave{晋安帝纪}曰:“凝之事五斗米道。孙恩之攻会稽,凝之谓民吏曰:‘不须备防,吾已清(请)大道,许遣鬼兵相助,贼自破矣。’既不设备,遂为恩所害。”\CJKunderwave{妇人集}曰:“谢夫人名道蕴,有文才,所箸(著)诗赋、诔、颂,传于世。”}}

{\cangkai\zihao{5}【评】陈郡谢家一门才子,家庭集会自然雅有风致。谢朗,\CJKunderwave{晋书}称其“善言玄理,文义艳发”,本则他对景摹状,亦见玄思才情。“骤”为米雪疾落景象,其状自然如空中撒盐,自天迅落,洁白一片,而不似鹅毛软雪,纷纷扬扬。面对此景,谢朗可谓善摹其形。道韫则不泥于眼前实景,迁想妙得,状雪之神韵,把飘扬飞舞之美,天涯无垠之感,注入了景相之中,使其深富意境,灵动有韵致。此可谓如九方皋相马,善得其神。\CJKunderwave{扪虱新话}评说:“谢氏二句,当各有谓,固未可优劣论也。”刘辰翁却说:“有女子风致,愈觉撒盐之俗。”陈梦槐亦云:“道韫答更娟美。”后世多因道韫之句而艳称其优。道韫句,的确更富诗意,更见妙想的神悟,更具诗人的天分。然就才思敏慧,捷于应对而言,此情此景,二人所拟一实一虚均见慧根。}

\lettrine{2.72} 王中郎令伏玄度、习凿齿\footnote{刘孝标注中“东海太守丞”,袁本“丞”作“承”,“承”是。}{\fzxk\zihao{6}\textcolor{red}{\CJKunderwave{王中郎传}曰:“坦之字文度,太原晋阳人。祖东海太守丞,清淡平远。父述,贞贵简正。坦之器度淳深,孝友天至,誉缉朝野,标的当时。累迁侍中、中书令,领北中郎将,徐兖二州刺史。”\CJKunderwave{中兴书}曰:“伏滔字玄度,平昌安丘人。小有才学,举秀才。大司马桓温参军,领大著作。掌国史,游击将军,卒。习凿齿,字彦威,襄阳人。少以文称,善尺牍。桓温在荆州,辟为从事,历治中、别驾,迁荣(衡)阳太守。”}} 论青、楚人物\footnote{论:评论。青、楚:青,指青州,在今山东境内,渤海沿岸及泰山一带。楚,指楚国旧地,今长江中下游一带。伏是青州人,习是楚人,故各自论赞自己家乡历史名人。}。{\fzxk\zihao{6}\textcolor{red}{滔\CJKunderwave{集}载其论,略曰:“滔以春秋时鲍叔、管仲、隰朋、召忽、轮扁、宁戚、麦丘人、逄丑父、晏婴、涓子;战国时公羊高、孟轲、邹衍、田单、荀卿、邹奭、莒大夫、田子方、檀子、鲁连、淳于髡、朌子、田光(文)、颜歜、黔子、於陵子仲、王叔(斗)、即墨大夫;前汉时伏征君、终军、东郭先生、叔孙通、万石君、东方朔、安期先生;后汉时大司徒、伏三老、江革、逄萌、禽庆、承幼(少)子、徐防、薛方、郑康成、周孟玉、刘祖荣、临孝存、侍其元矩、孙宾硕、刘仲谋、刘公山、王仪伯(伯仪)、郎宗、祢正平、刘成国;魏时管幼安、邴根矩、华子鱼、徐伟长、任昭先(光)、伏高阳。此皆青土有才德者也。凿齿以神农生于黔中,\CJKunderwave{邵南}咏其美化,\CJKunderwave{春秋}称其多才,\CJKunderwave{广汉}之风,不同\CJKunderwave{鸡鸣}之篇,子文、叔敖,羞与管、晏比德。接舆之歌\CJKunderwave{凤兮},渔父之咏\CJKunderwave{沧浪},汉阴丈人之折子贡,市南宜僚、屠羊说之不为利回,鲁仲连不及老莱夫妻,田光于屈原(田文不及屈原),邓禹、卓茂无敌于天下,管幼安不胜庞公,庞士元不推华子鱼,何、邓二尚书独步于魏朝,乐令无对于晋世。昔伏羲葬南郡,少昊葬长沙,舜葬零陵。比其人,则准的如此;论其士(土),则群圣之所葬;考其风,则诗人之所歌;寻其事,则未有赤眉、黄巾之贼。此何如青州邪?”滔与相往反,凿齿无以对也。}} 临成,以示韩康伯。康伯都无言,王曰:“何故不言?”韩曰:“无可无不可\footnote{无可无不可:语出\CJKunderwave{论语},原为孔子对仕或隐,认为只要义之所在,哪种做法都可以。后指怎么样都可以,对人或事无明确的可、否态度。}。”{\fzxk\zihao{6}\textcolor{red}{马融注\CJKunderwave{论语}曰:“唯义所在。”}}

{\cangkai\zihao{5}【评】品评人物作为汉魏以来的风尚,不仅名家品题月旦当代人物,而且扩展到对乡党地域历史人物的评论。在故事中,王坦之并就伏、凿所论向韩康伯讨教,期盼名士的高度评价。但面对期盼,王坦之失望了,然而却看到了韩康伯的妙对与风度。康伯“清和有思理”,人谓其能“澄世所不能澄,而裁世所不能裁”,精习\CJKunderwave{周易},善为折中。这里“无可无不可”的态度,表现了他的不澄、不裁。这种青、楚历史人物之论既无谈玄价值,又无关当代士风,康伯引孔子的话巧妙地予以否定,体现了含蓄、闲远的魏晋人物风度。}

\lettrine{2.73} 刘尹云\footnote{刘尹:即刘惔,字真长,曾任丹阳尹,故称。谢安妻兄,尚明帝女庐陵公主。会稽王司马昱为相,与王濛并为其座上清谈之客。性简贵自重,与王羲之友善。卒年三十六。}:“清风朗月,辄思玄度\footnote{玄度:许询。}。”{\fzxk\zihao{6}\textcolor{red}{\CJKunderwave{晋中兴士人书}曰:“许询能清言,于时士人,皆钦慕仰爱之。”}}

{\cangkai\zihao{5}【评】许询的人格特征,就是一种“清风朗月”般的人生境界,其才情文咏,一生形迹,尽赋予对清澄人生的况味与追求,因而标示了这一生命境界的范本。刘惔虽跻身名利场中,然其“少清远”、“好\CJKunderwave{老}、\CJKunderwave{庄},任自然趣”的性格、学养,使其对许询风范欣羡之,因而,本则的感喟,其情怀如诗如画,与清风朗月相媲美。亦如清雅诗句。}

\lettrine{2.74} 荀中郎在京口\footnote{京口:今江苏镇江,东晋为军事重镇。}{\fzxk\zihao{6}\textcolor{red}{,\CJKunderwave{晋阳秋}曰:“荀羡字今(令)则,颍川人,光禄大夫崧之子也。清和有识裁,少以主婿为驸马都尉。是时殷浩参谋百揆,引羡为援,频莅义兴、吴郡,超授北中郎将、徐州刺史,以蕃屏焉。”\CJKunderwave{中兴书}曰:“羡年二十八,出为徐、兖二州。中兴方伯之少,未有若羡者也。”}} 登北固望海云\footnote{北固:即北固山。}:{\fzxk\zihao{6}\textcolor{red}{\CJKunderwave{南徐州记}曰:“城西二(北)有别岭入江,三面临水,高数十丈,号曰北固。”}} “虽未睹三山,便自使人有陵云意\footnote{陵云意:飞升入云的感觉。陵,通“凌”,袁本作“凌”。}。若秦、汉之君\footnote{秦、汉之君:此指秦始皇、汉武帝。两人都幻想长生不死,求神仙,寻仙药,事见\CJKunderwave{史记}中的\CJKunderwave{秦始皇本纪}和\CJKunderwave{封禅书}。},必当褰裳濡足\footnote{褰(qiān牵)裳濡足:撩起下衣,涉水浸足。}。”{\fzxk\zihao{6}\textcolor{red}{\CJKunderwave{史记·封禅书}曰:“蓬莱、方丈、羸(瀛)洲,此三山,世传在海中,去人不远。尝有至者,言诸仙人不死药在焉。黄金白银为宫阙,草物禽兽尽白,望之如云。及至,反居水下。欲到,即风引船而去,终莫能至。秦始皇登会稽,并海上,冀遇三神山之奇药。汉武帝既封秦(泰)山,无风雨变至,方士更言蓬莱诸药可得,于是上欣然东至海,冀获蓬莱者。”}}

{\cangkai\zihao{5}【评】登北固山,纵目无碍,江天之间又缀以灏气烟霭,这种江天景色,自会让人有与自然冥合的感受。一睹此景,荀羡油然联想起传说中的海上三座仙山境像,其言语所描绘的审美心理,便使人超然,使江山含有灵气,而秦皇、汉武求长生不死故事的连缀引入,再使登临景色令人神思不已,富有极广阔、极灵妙的想象空间。这一番言语,不只表达了荀羡巧辞幽默,更表达了魏晋人对山水自然之美的审美感悟。}

\lettrine{2.75} 谢公云\footnote{谢公:谢安。}:“贤圣去人,其间亦迩\footnote{去:距离。迩:近。}。”子侄未之许\footnote{未之许:不赞同他的说法。},公叹曰:“若郗超开(闻)此语\footnote{郗超:见刘孝标注,少而卓荦不羁,有旷世之度,善谈玄,信佛理。开(开),袁本作“闻”,“闻”是。},必不至河汉\footnote{河汉:银河。见刘孝标注引\CJKunderwave{庄子},比喻谈话迂远,不切实际。刘孝标注中“肩吾开于连叔”,“开”袁本作“问”,“问”是。“坚梯其言”,“坚梯”袁本作“怪怖”,“怪怖”是。}。”{\fzxk\zihao{6}\textcolor{red}{\CJKunderwave{超别传}曰:“超精于理义,沙门支道林以为一时之俊。”\CJKunderwave{庄子}曰:“肩吾开(问)于连叔曰:‘吾闻言于接舆,大而无当,往而不反,坚梯(怪怖)其言,犹河汉而无极也。’”}}

{\cangkai\zihao{5}【评】孔子曾说:“仁远乎哉?我欲仁,斯仁至矣。”(\CJKunderwave{论语·述而})孟子亦有名训,“尧舜与人同耳”(\CJKunderwave{孟子·离娄下}),认为“人皆可以为尧舜”(\CJKunderwave{孟子·告子下}),贤圣在己,欲之则至,人人都有成贤作圣的可能性,贤圣并非超人。谢公善思辩,以古人圣贤未远的思路教育、激励子弟。子侄不赞同,引起了他的感叹,他们倘有郗超的精于理义、思辩之俊就好了。本则言语,活画出了谢公善以思辩启人心智的玄家风度,和他厚爱子弟的动人情致,也表现了他对卓荦之才的倾心推崇。}

\lettrine{2.76} 支公好鹤\footnote{支公:支道林,著名僧人,时称支公。好(hào浩):喜欢。},住剡东𡵙山\footnote{剡(shàn善):县名,晋属会稽郡,治所在今浙江嵊州市。𡵙(ánɡ昂):山名,在剡县境内。}。{\fzxk\zihao{6}\textcolor{red}{\CJKunderwave{支公书}曰:“山去会稽二百里。”}} 有人遗其双鹤\footnote{遗(wèi未):赠送。},少时翅长欲飞,支意惜之,乃铩其翮\footnote{铩(shā杀):伤残。翮(hé和):鸟羽的茎。}。鹤轩翥不复能飞\footnote{轩翥(zhù住):振翅。},乃反顾翅,垂头,视之如有懊丧意。林曰:“既有陵霄之姿\footnote{陵霄: 直升云霄。袁本作“凌霄”。姿: 资质, 才能。},何肯为人作耳目进说\footnote{进说:袁本作“近玩”。近玩,身边宠爱的玩物。}!”养令翮成,置使飞去\footnote{置:释放。}。

{\cangkai\zihao{5}【评】支公养鹤放鹤,同其养马而爱其神骏一样(参见本篇63),都是对翩然翱翔于自由之境的神往,从中可见支公精神,它也是魏晋精神风貌又一生动描画。刘梦槐评曰:“事属冲旷,意太怆伤。”该则与本篇63略为不同的是,养鹤放鹤中间,流淌着一段怆然感伤的意味,也是一种两难的选择。最后之举说明,还是任自然才能真正地体现真爱。在生命中,自由是最可宝贵的。}

\lettrine{2.77} 谢中郎经曲阿后湖\footnote{曲阿后湖:湖名,又名练湖,在今江苏丹阳城北。参见刘孝标注。},问左右:“此是何水?”{\fzxk\zihao{6}\textcolor{red}{\CJKunderwave{中兴书}曰:“谢万字万石,太傅安弟也。才气为后(高俊),蚤知名。历吏部、西中郎将、豫州刺史、散骑常待。”}} 答曰:“曲阿湖。”{\fzxk\zihao{6}\textcolor{red}{\CJKunderwave{太康地记}曰:“曲阿本名云染(阳),秦始皇以有王气,凿北阬山以败其势,截其直道,使其阿曲,故曰曲阿也。吴还为云阳,今复名曲阿。”}} 谢曰:“故当渊注渟箸(著)\footnote{故当:当然、自然是。渊注:大量汇聚注入。渟(tínɡ 亭)箸:停滞、积聚不流。箸,同“著”。},纳而不流。”

{\cangkai\zihao{5}【评】谢万是谢氏族中的虚浮之士,“才器隽秀,虽器量不及(谢)安,而善自衒曜”(\CJKunderwave{晋书·谢万传}),但小聪明还是有的。本则见出他善悟的聪明。见山之阿曲而使水渊渊汇聚,纳而不流,因有感悟。至于感悟出什么道理,\CJKunderwave{世说笺本}云:“此语不详其义,似借曲阿湖以喻为人不应曲己从人,阿谀取容,谓既名为曲阿,宜乎水之停著不流、藏垢纳污也。”张万起、刘尚慈\CJKunderwave{世说新语译注}谓:“此感慨的喻意是:学识上只有兼收并蓄才能渊博而深厚。”无论所悟何义,在这里它最突出的是,表现着时人将人生的品味融入山水审美的鲜明倾向。这里更醒目的是在山水观照中介入了“理”的思趣,于物我交流中感悟了人生哲理。在谢万言语中,展现了魏晋人山水审美心理的丰富性,从而在这一细微处,诠释着魏晋风度中的另一方面内涵。}

\lettrine{2.78} 晋武帝每饷山涛恒少\footnote{晋武帝:司马炎(236—290)。饷:馈赠。山涛:字巨源,西晋河内怀县(今河南武陟西南)人。好\CJKunderwave{老}、\CJKunderwave{庄},与阮籍、嵇康等交友,为竹林七贤之一。在魏为郎中、吏部郎;入晋为吏部尚书、司徒等。恒:常。},谢太傅{\fzxk\zihao{6}\textcolor{red}{安也。}} 以问子弟\footnote{谢太傅:谢安,谢奕(?—358):字无奕,谢安长兄,陈郡阳夏谢氏家族在东晋初期的代表人物之一。子弟:本家子侄晚辈。},车骑{\fzxk\zihao{6}\textcolor{red}{玄也。}} 答曰\footnote{车骑:此指谢玄,谢安侄,死后追赠车骑将军。}:“当由欲者不多\footnote{当:或许,表揣度。},而使与者忘少。”{\fzxk\zihao{6}\textcolor{red}{\CJKunderwave{谢车骑家传}曰:“玄字幼度,镇西弈(奕)弟(第)三子也。神理明俊,善微言。叔父太傅尝与子侄燕集,问:‘武帝任山公以三事,任以宫(官)人,至于赐予,不过斤合,当有旨不?’至(玄)答有辞致也。”}}

{\cangkai\zihao{5}【评】谢安对自家子弟的深情厚爱,表现在热切地期望子弟都能成为生于阶庭的芝兰玉树,成人成器。为此,作为长辈,他循循焉善诱人,以启发式的智慧感召,来教育子弟。本则既表现了谢安的善于启发,也表现了谢玄的聪明颖悟。}

晋武帝以政宽仁厚来补救司马氏谋取皇权过程中,闹得“天下名士少有全者”的人心之危,所以武帝一朝,常行宽惠恩赏。而山涛是景帝司马师以来一直为王朝所重用的重臣,及武帝“迁右仆射、加光禄大夫、侍中”执掌选举,后再拜“司徒”。这样的重臣要员,喜恩赏的晋武帝却每饷恒少。这凸显了山涛为官清廉的品质和智慧。山涛早孤、居贫,没有深固的根基背景,又处于权利争斗的中心,他一方面“中立于朝”,在纷繁的权利之争中,保持清醒头脑,另一方面,勤政寡求,一生“贞慎俭约”,清廉到连皇帝都看他“清俭无以供养”而于心不忍。真正是“欲者不多”了。因此,他能年七十九而善终,被谥曰“康”。一生形迹,于政有事功,于己有名德,可谓达到了为臣者的难能境界。山涛虽“欲者不多”,可晋武帝的“与者忘少”,绝非是其心中无数而健忘,是他以“厉以恭俭,敦以寡欲”的深刻用心,来力矫曹魏以来奢侈之风,从而稳固本朝统治的基本做法使然。这对赏赐的授受两者说来,能达到如此的神会,实属不易。

这样一段现代史的活教材,被谢安用活了,不必喋喋不休地诲尔谆谆,只一启发,颖悟的后生就感受到了。在谢玄虽表面是应声而答的言语敏捷,实则让人感到了应答背后的无尽馀味。

\lettrine{2.79} 谢胡儿语庾道季\footnote{谢胡儿:谢朗。}:{\fzxk\zihao{6}\textcolor{red}{道季,庾龢小字。徐广\CJKunderwave{晋纪}曰:“龢字道季,太尉亮子也。风情率悟,以文谈致称于时。历仕至丹阳尹,兼中领军。”}} “诸人暮当就卿谈\footnote{暮:袁本作“莫”,莫、暮古今字,晚上。当:将。谈:清谈论辩。},可坚城垒\footnote{坚城垒:本为加固城防,此指认真做好准备。}。”庾曰:“若文度来,我以偏师待之\footnote{文度:王坦之,字文度。偏师:军队的一部分,非主力。};康伯来,济河焚舟\footnote{康伯:韩伯,字康伯,清谈名家。济河焚舟:此指一拼到底,决不后退。}。”{\fzxk\zihao{6}\textcolor{red}{\CJKunderwave{春秋传}曰:“秦伯伐晋,济河焚舟。”杜预曰:“示必死。”}}

{\cangkai\zihao{5}【评】真正的论辩令人如亲临战场一般。陶珙曰:“此真可谓舌战矣。”(\CJKunderwave{世说新语汇校集注}引)一片临战紧张的景象。由此可见当时风气,清谈论辩为不容含糊的一桩严肃事项。论战之高下结果,是会影响到对一个人的价值品评的,所以充分估计到对手的实力、状况,然后有所准备。这是智慧、学养的较量,也是“标会”理论水准的拼搏。言语选此条,充分表现了当时清谈论辩家的精神状貌。}

\lettrine{2.80} 李弘度常叹不被遇\footnote{不被遇:没有得到机遇,受到当权者的赏识。遇,知遇。}。{\fzxk\zihao{6}\textcolor{red}{\CJKunderwave{中兴书}曰:“李充字弘度,江夏鄙(鄳)人也。祖康(秉)、父矩,皆有美名。充初辟丞相掾、记室参军,以贫,求剡县,迁大著作、中书郎。”}} 殷扬州{\fzxk\zihao{6}\textcolor{red}{殷浩别见。}} 知其家贫\footnote{殷扬州:指殷浩,任扬州刺史,褚公:对褚裒的敬称。褚裒(póu 抔)(303—349),晋康帝皇后之父,朝廷议以“不臣之礼”,力辞执政,而赴外镇。官征北大将军。曾率军三万北伐,败后上疏自贬,忧慨发愤而卒。见\CJKunderwave{晋书·外戚传}。},问:“君能屈志百里不\footnote{屈志百里:指做县令。屈志:降志以求,迁就。为客气话。百里:古代一县所辖约百里,因代指一县或县令。}?”李答曰:“北门之叹,久已上闻\footnote{\CJKunderwave{北门}:\CJKunderwave{诗经·邶风}诗篇,“出自北门,忧心殷殷。终窭且贫,莫知我艰。”诗序说此诗抒写仕宦不得志,不受重用。杨勇\CJKunderwave{世说新语校笺}引顾炎武语谓,\CJKunderwave{北门}本出\CJKunderwave{诗经·邶风},\CJKunderwave{世说}注屡以邶、鄘、卫三者通名卫诗者,则六朝之时,尚有古之遗制也。}。{\fzxk\zihao{6}\textcolor{red}{\CJKunderwave{卫诗·北门}刺仕不得志也。}} 穷猿奔林,岂暇择木\footnote{穷猿:穷途末路之猿。此喻处境窘迫之人。}!”遂授剡县\footnote{剡(shàn善):县名,今浙江嵊州。}。

{\cangkai\zihao{5}【评】才士不遇,本是悲剧,也是才士最不堪忍受的人生尴尬,而与不遇相连的,就是生活的窘迫。为摆脱尴尬、窘迫而求仕,是古才士的常情。本则李充除对此表达得真率之外,其措辞,亦确见才情。“穷猿奔林,岂暇择木?”比喻真切、生动,将其窘迫之境、怆然之哀,描摹得淋漓尽致。}

\lettrine{2.81} 王司州至吴兴印渚中看\footnote{吴兴:郡名,辖境相当于今浙江临安、馀杭、德清一带,治所在乌程。}。{\fzxk\zihao{6}\textcolor{red}{\CJKunderwave{王胡之别传}曰:“胡之字修龄,琅邪临沂人也。廙之子也。历吴兴太守,征侍中、丹阳尹、秘书监,并不就。拜使持节,都督司州诸军事、西中郎将、司州刺史。”\CJKunderwave{吴兴记}曰:“于潜县东七十里,有印渚,渚傍有白石山,峻壁四十丈。印渚盖众溪之下流也。印渚已上至县,悉石濑恶道,不可行船;印渚已下,水道无险,故行旅集焉。”}} 叹曰:“非唯使人情开涤\footnote{开涤:开阔,涤荡。},亦觉日月清朗。”

{\cangkai\zihao{5}【评】魏晋重情,不仅对人,也移情于山水自然。刘勰\CJKunderwave{文心雕龙}“登山则情满于山,观海则意溢于海”,是对时人情感体验的最好概括。本则是山水体验的一个生动记录。王修龄观览、察看吴兴印渚,为山水之美所动,他的直接体验就是,如此景色,令人心胸开阔,荡涤了一切尘俗杂念、污秽之想,使心怀如同山水一样,清爽秀彻,注入着一段富有自然之趣的生命力。于是,再看日月,也非同以往,所见更加滢澈、明亮。魏晋人陶冶山水,面对人生,个人的气质风貌便自然清新超拔。}

\lettrine{2.82} 谢万作豫州都督\footnote{谢万:谢万是谢氏族中的虚浮之士。豫州:州名,西晋时治所在今河南汝南辖今豫东、皖北地区,东晋于江南置侨郡,镇江西。都督:官名,掌一州或数州军事。谢万当时所领官职为“豫州刺史、领淮南太守、兼司豫冀并四州军事、假节”,执掌方面军政大权。},新拜\footnote{拜:授官。},当西之都邑\footnote{当西之都邑:将向西前往都督府所在的城邑。},相送累日,谢疲顿。于是高侍中往\footnote{于是:在这时。},{\fzxk\zihao{6}\textcolor{red}{\CJKunderwave{中兴书}曰:“高崧字茂琰,广陵人。父悝,光禄大夫。崧少好学,善史传。累迁吏部郎、侍中,以公累免官。”}} 径就谢坐,因问:“卿今仗节方州\footnote{仗节:手持符节。此指出任地方长官。方州:地方州郡。},当疆理西蕃\footnote{疆理:分界治理。疆,划分。西蕃:西部藩国。蕃通“藩”,藩国,此指大的行政区域。豫州镇江西,在都城建康西,故称。},何以为政\footnote{为政:处理政务。}?”谢粗道其意。高便为谢道形势\footnote{形势:指政要。},作数百语。谢遂起坐。高去后,谢追曰\footnote{追:追溯,回忆。}:“阿酃故粗有才具\footnote{粗有,略有。才具:才能。}。”{\fzxk\zihao{6}\textcolor{red}{阿酃,崧小字也。}} 谢因此得终坐。

{\cangkai\zihao{5}【评】高崧善史书,为当时智谋之士。简文辅政时桓温擅威,率军北伐,军次武昌,简文深为忧患,崧为简文谋划,陈说祸福利害以服桓温,去主之辱,立主之威,可谓智士能臣。而谢万对于政事、军谋可说是一个“绣花枕头”,遇到实务摆出一副啸咏高傲的姿态,虚浮无实才。崧在谢万领此要职时,来长篇大论地教导他一番。可以感受到他谈话的精彩,竟把这位本来“善自衒曜”,又累得疲惫不堪的贵公子说得来了精神,“起坐”而听。谢万一定受益不少,然而其自视甚高的虚浮秉性作祟,当人面只吝啬地对高崧稍加肯定——“粗有才具”。如此“言语”自画出了谢万的傲诞形象。}

\lettrine{2.83} 袁彦伯为谢安南司马\footnote{谢安南:谢奉,字弘道,东晋会稽山阴(今浙江绍兴)人。曾官安南将军、广州刺史、吏部尚书。司马:官名。魏晋时将军府及州郡设司马,掌管兵事。},{\fzxk\zihao{6}\textcolor{red}{安南,谢奉别见。}} 都下诸人送至濑乡\footnote{都下:京城。濑(lài赖)乡:古地名。在东晋京城建康附近。}。将别,既自凄惘\footnote{凄惘:伤感怅惘,若有所失。},叹曰:“江山辽落\footnote{辽落:空旷辽远。},居然有万里之势\footnote{居然:显然;确实。}。”{\fzxk\zihao{6}\textcolor{red}{\CJKunderwave{续晋阳秋}曰:“袁宏字彦伯,陈郡人,魏郎中令焕(涣)六世孙也。祖猷,侍中。父勗,临汝令。宏起家建威参军,安南司马记室。太傅谢安赏宏机捷辩速,自吏部郎出为东阳郡,乃祖之于冶亭,时贤皆集。安欲卒迫试之,执手将别,顾左右取一扇而赠之。宏应声答曰:‘辄当奉扬仁风,慰彼黎庶。’合坐叹其要捷。性亮直,故位不显。在郡卒。”}}

{\cangkai\zihao{5}【评】多情自古伤别离,屈原有“悲莫悲兮生别离”(\CJKunderwave{九歌})的句子,江淹说:“黯然销魂者,唯别而已矣。”(\CJKunderwave{别赋})这些句子都说出了别离的感受,而本则将别离的感受落实在江山辽远、万里之势的形象之中,就别有一番趣味。魏晋人将人情感受与山水自然相结合,便有了另外一番审美境界。就是发一感喟,也味同吟诗。此句曾引出评家的评点。刘辰翁云:“黯然销魂,直是注情语耳,未在能言。”其实,正表明了魏晋人另一特色的能言,不类善辩之巧言,而是如诗之意内言外,是更灵动、更加言少意多的“能言”。黄辉说:“别语,惟‘春草碧色,春水绿波,送君南浦,伤如之何’与此二语,千古作对。”这正说到本则的紧要处。刘梦怀说:“勿勿有此怀,是别离者。”}

\lettrine{2.84} 孙绰赋\CJKunderwave{遂初}\footnote{赋\CJKunderwave{遂初}:创作\CJKunderwave{遂初赋}。赋,写作。},筑室畎川\footnote{畎(quǎn犬)川:不详何地,或曰乃山野平川。},自言见止足之分\footnote{止足之分:知止知足的本分。\CJKunderwave{老子}四十四章“知足不辱,知止不殆,可以长久”。}。{\fzxk\zihao{6}\textcolor{red}{\CJKunderwave{中兴书}曰:“绰字兴公,太原中都人。少以文称,历太学博士、大著作、散骑常侍。”\CJKunderwave{遂初赋}叙曰:“余少慕老庄之道,仰其风流久矣。却感于陵贤妻之言,怅然悟之。乃经始东山,建五亩之宅,带长阜,倚茂林,孰与坐华幕、击钟鼓者同年而语其乐哉!”}} 斋前种一株松,恒自手壅治之。高世远时亦邻居\footnote{高世远:高柔,字世远。东晋乐安(今山东)人,多才博识,淡泊名利,曾为安固县令。刘孝标注中“高崇”,袁本作“高柔”,是。},{\fzxk\zihao{6}\textcolor{red}{世远,高崇(柔)字也。别见。}} 语孙曰:“松树子非不楚楚可怜\footnote{松树子:小松树。楚楚:纤弱的样子。可怜:可爱。},但永无栋梁用耳!”孙曰:“枫柳虽合抱\footnote{合抱:指树身有两臂合围那样粗。},亦何所施\footnote{施:使用。}?”

{\cangkai\zihao{5}【评】余嘉锡先生说:“兴公为孙子荆之孙。高柔之言,乃斥其祖之名以戏之。孙答语中当亦还斥高柔祖父之名,但不可考耳。”(\CJKunderwave{世说新语笺疏})观本则,当是两位怀才高逸之士的言语戏乐。孙绰的祖父,名楚,于是高柔以“楚楚”为说。晋人对家讳特别敏感,因而孙绰敏捷的反唇相讥,此讥如余先生说,当亦及高柔家讳,只是无从考究了。这里用谐音创造语言幽默,见出时人对语言积极修辞的自觉追求。\CJKunderwave{言语}选此,恐即是从善修辞这一角度来欣赏二人口才、智巧的。}

\lettrine{2.85} 桓征西治江陵城甚丽\footnote{桓征西:桓温,见本篇55刘孝标注。江陵:当时荆州治所,今属湖北。},{\fzxk\zihao{6}\textcolor{red}{盛弘之\CJKunderwave{荆州记}曰:“荆州城临汉江,临江王所治。王被征,出城北门而车轴折,父老泣曰:‘吾王去不还矣。’从此不开北门。”}} 会宾僚出江津望之\footnote{会:会集。宾僚:宾客僚属。出:到。江津:江边渡口。江,此指汉江。},云:“若能目此城者有赏\footnote{目:品题,评论。}。”顾长康时为客\footnote{顾长康:顾恺之,字长康,小字虎头。},在坐,因曰:“遥望层城\footnote{层城:高大宏伟的城。},丹楼如霞\footnote{丹楼:红色楼阁。}。”桓即赏以二婢。

{\cangkai\zihao{5}【评】顾恺之博学多才,时人谓其有三绝:才绝、画绝、痴绝。其绘画艺术,被谢安视为“有生民以来未之有也”(\CJKunderwave{晋书·顾恺之传})。这里顾恺之以画家特有的感悟,来体会桓温所治宏大壮丽的江陵城楼。作为画家,顾恺之深得“迁想妙得”(顾恺之\CJKunderwave{魏晋胜流画赞})之味。在观察体会对象时,画家移入想象,不但观对象之形,也体会对象之神,在顾恺之脑海里的江陵城楼,此时已是画中之景了。他不泥于具体城楼的形制,而是注入了气韵,所以能妙得其真——已经进入了艺术的真实境界。这是画家脑海里的画,又用诗人的语言描画出来,真是生动之极。此种妙语,就绝非一般的诗人或一般的画家所能“偶得”的了,于此可见顾恺之的“才绝”。}

凌濛初曰:“虎头每有画意,此遽正本。”李贽曰:“亦虎头画笔。”都画龙点睛一样,评点出了本则的妙处。

\lettrine{2.86} 王子敬语王孝伯曰\footnote{王子敬:王献之,(344—388),出于琅邪王氏家族。曾任谢安长史,官至中书令,故称王令或王大令。据\CJKunderwave{晋书·后妃传},尚简文帝女新安公主。少有令名,“风流一时之冠”。其书法已造神境,与父羲之并称“二王”。病笃:病重。王孝伯:王恭(?—398):孝武帝后兄,安帝舅父。与殷仲堪、桓玄等,二次兴兵清君侧,兵败被诛。会稽:郡治在今浙江绍兴市。}:“羊叔子自复佳耳\footnote{自复:确实。},然亦何与人事\footnote{人事:别人的事,“人”此指王子敬自己。},{\fzxk\zihao{6}\textcolor{red}{\CJKunderwave{晋诸公赞}曰:“羊祜字叔子,太山平阳人也。世长吏二千石,至祜九世,以清德称。为儿时,游汶滨,有行父止而观焉,叹息曰:‘处士大好相,善为之。未六十,当有重功于天下。即当贵,无相忘!’遂去,莫知所在。累迁都督荆州诸军事。自在南夏,吴人悦服,称曰羊公,莫敢名者。南州人闻公哀,号哭罢市。”}} 故不如铜雀台上妓\footnote{故:实在。铜雀台:楼台名。汉献帝建安十五年(210),曹操建于邺城(今河北临漳县)。台高十丈,有殿堂百二十间,楼顶置大铜雀,故名铜雀台。此台为当时曹氏集团的游宴之所。}。”{\fzxk\zihao{6}\textcolor{red}{魏武\CJKunderwave{遗令}曰:“以吾妾与妓人,皆着铜雀台上,施六尺床、穗帷,月朝十五日,辄使向帐作伎!”}}

{\cangkai\zihao{5}【评】余嘉锡谓:“子敬吉人辞寡,亦复有此放诞之言,有愧其父多矣。”王世懋曰:“羊公盛德,此语伤子敬之厚。”刘应登云:“此亦戏言,谓羊公清德自佳而已,不如铜雀妓娱人耳目。”“此正堕泪之言,人不能识耳。按此乃愤世不辨美恶,故作反语以示愤慨耳。”}

前贤诸评,是从传统道德的尺度来看待王子敬的,都有一定的道理。若从这种尺度衡量,子敬实不及乃父之德,有愧其父。据\CJKunderwave{晋书·王献之传}记载,他的放诞,骨子里是自私的,不念及他人的感受,是一种旁若无人的贵族的狂傲。虽寡言少语,好似“吉人之辞寡”,其实为人并不厚道。他宣言,崇奉刘惔这样的自负名士,也见其内心的狂诞不经。所以,他能漠视传统道德意义上的德才完备的楷模人物羊祜,并进一步说道德之士不如铜雀台之妓,更让人娱悦——确乎典型的离经叛道之论。或为子敬辩解:西晋有羊祜以才以德奠定征服东吴的基业,而渡江以来,所谓大才,德行皆不如羊祜,却为朝廷倚重,子敬的反语愤愤确为“堕泪之言”。倘换一个角度看,子敬此言似不深奥。他率性而发,在这看似狂诞不经的言语中,实则典型地表达了对个性的追求。魏晋才士在哲学上的命题——“越名教而任自然”广有影响,在为人的实践上,名士风流的一大特征,就是对传统道德教条的抗拒,而展演着自然之趣的生命追求。子敬之言虽颇有些极端,然而恰表达了那一时代的名士风尚,是“越名教而任自然”的最好注脚。这种人格表达,又正是后来腾涌起专重情感,妙发灵性,独抒怀抱的艺术风尚的先声。

\lettrine{2.87} 林公见东阳长山\footnote{林公:支道林,东晋名僧。刘孝标注“海鸥鸟”句云语出\CJKunderwave{庄子},然事见今本\CJKunderwave{列子·黄帝篇},今本\CJKunderwave{庄子}无,余嘉锡\CJKunderwave{笺注}以为:“刘\CJKunderwave{注}所引,(\CJKunderwave{庄子})逸篇之文也。\CJKunderwave{列子}伪书,袭自\CJKunderwave{庄子}耳。”东阳:郡名,治所在今浙江金华。长山:山名,在今东阳市内,山长三百馀里,县因山得名。},曰:“何其坦迤\footnote{何其:多么。坦迤:平坦而绵延不断。}。”{\fzxk\zihao{6}\textcolor{red}{\CJKunderwave{会稽土地志}曰:“山靡迤而长,县因山得名。”}}

{\cangkai\zihao{5}【评】这句喟叹,看去似乎平平,好像什么都没说。然而,出自林公之口,它表现了其人长期品味自然之美,胸中大有丘壑的审美修养。有此背景,一旦面对这种从所未见的壮观景色时,便蓦然唤起了独特的审美感受。在其惊叹于“何其坦迤”之时,已然说明了他对自然美的审美自觉和审美经验。这点也是支道林的动人之处。}

刘辰翁说:“如此四字,极似无谓,亦有可思。”

\lettrine{2.88} 顾长康从会稽还\footnote{顾长康:顾恺之。会稽:郡名,治所在今浙江绍兴。},人问山川之美,顾云:“千岩竞秀,万壑争流\footnote{万壑:众多的泉溪河流。},草木蒙笼其上\footnote{蒙笼:笼罩覆盖。},若云兴霞蔚\footnote{云兴霞蔚:云雾兴起,彩霞绚烂。}。”{\fzxk\zihao{6}\textcolor{red}{丘渊之\CJKunderwave{文章录}曰:“顾恺之字长康,晋陵人。父悦,尚书左丞。恺之,义熙初为散骑常侍。”}}

{\cangkai\zihao{5}【评】稍后于长康的名画家宗炳,在其\CJKunderwave{画山水序}中,谈到了画家会心于自然景色的感受,他强调“应目会心”,“应会感神”——眼有所见,心有所动,然后上升到画家自己独特的精神享受。王微在其\CJKunderwave{画叙}中说得更加明白浅切:“望秋云,神飞扬;临春风,思浩荡。”画家以情感物,才有“画之情”。那一时代的画家,共同感悟了把握自然景象的法门。作为画家,长康此言正是他领会自然美景的生动写照。会稽行程,一路走来,他神思飞扬,“应目会心”,于是便有了气象万千,灵动辉映的心中之景,如王世懋评点:此言正是“虎头画稿”。这一画稿,又正表达了画家感受景色的审美修养。就言语角度说,长康善用修辞,采取拟人、比喻,因而描摹生动,用文字点染出了他心中的画。正因为他自己很会感受大自然之画卷的生动,口述出来,如诗如画,也便让听者如闻如见,神往不已。}

史称顾长康“博学有才气”,本则便标示了魏晋才士之“博学有才气”的内涵和水准。

\lettrine{2.89} 简文崩\footnote{简文:东晋简文帝:晋简文:指晋简文帝司马昱(320—372),穆帝年幼即位,昱任抚军大将军总理政务。后来大将军桓温专擅朝政,先废海西公,后立司马昱为帝,第二年崩。据\CJKunderwave{晋书},简文卒于咸安二年(372)七月,年五十三,在位二年。崩:帝王之死曰崩。},孝武年十馀岁,立,至暝不临\footnote{暝:日暮。临(lìn吝):哭吊。依礼,亲人死要按时哭丧。}。{\fzxk\zihao{6}\textcolor{red}{宋明帝\CJKunderwave{文章志}曰:“孝武皇帝讳昌明,简文第三子也。初,简文观谶书曰:‘晋氏阼尽昌明。’及帝诞育,东方始明,故因生时以为讳,而相与忘告。简文问之,乃以讳对。简文流涕曰:‘不意我家昌明便出。’帝聪惠,推贤任才。年三十五崩。”}} 左右启:“依常应临\footnote{依常:按常礼。}。”帝曰:“哀至则哭,何常之有\footnote{至:极。何常之有:有什么常理。}?”

{\cangkai\zihao{5}【评】简文为多重压力威逼而仓促辞世,及其欲崩才立皇子,而皇子仅十岁。未经任何训练的孩提面临如此场面,自然不知所措。懵懵懂懂,一切听大人支配,一会儿哭临,一会儿接待盈门的吊客,礼仪规矩不胜烦琐,早把孩子弄得心烦了。尽管是天子,但毕竟是孩子,一句不耐烦的抗议,便显得率性任真。这里表达的是孩子的天性,且口角伶俐,言之有理,一派真实可爱情景,不失灵巧应对之妙语。在\CJKunderwave{世说}中,更表达了一种纯任自然的人性呼唤的意识,所以孩提之言亦得高列\CJKunderwave{言语}门中。}

\lettrine{2.90} 孝武将讲\CJKunderwave{孝经}\footnote{孝武:东晋孝武帝,见本篇89刘孝标注。\CJKunderwave{孝经}:孔子所述,讲究孝为德之本,明王以孝治天下之大义。},谢公兄弟与诸人松(私)庭讲习\footnote{谢公兄弟:谢安和弟弟谢石。松庭:袁本作“私庭”,是。私庭,自己家中。讲习:讲说研习。}。{\fzxk\zihao{6}\textcolor{red}{\CJKunderwave{续晋阳秋}曰:“宁康三年九月九日,帝讲\CJKunderwave{孝经},仆射谢安侍坐,吏部尚书陆纳、兼侍中卞耽读,黄门侍郎谢石、吏部袁宏兼执经,中书郎车胤、丹阳尹王温(混)摘句。”}} 车武子难苦问谢\footnote{车武子:车胤字武子,东晋南平(今湖北公安西南)人,以博学著名。历辅国将军、丹阳尹、吏部尚书等官职。难:难于。苦问:再三问、深问。},{\fzxk\zihao{6}\textcolor{red}{车胤,别见。}} 谓袁羊曰:“不问,则德音有遗\footnote{德音:善言,此指对\CJKunderwave{孝经}富有真知灼见的阐述。};多问,则重劳二谢\footnote{重劳:搅扰、麻烦。}。”{\fzxk\zihao{6}\textcolor{red}{袁羊,乔小字也。\CJKunderwave{袁氏家传}曰:“乔字彦升(叔),陈郡人。父瓌,光禄大夫。乔历尚书郎、江夏相。从桓温平蜀,封湘西伯、益州刺史。”}} 袁曰:“必无此嫌。”车曰:“何以知尔\footnote{尔:如此。}?”袁曰:“何尝见明镜疲于屡照,清流惮于惠风\footnote{惠风:和风。}?”

{\cangkai\zihao{5}【评】晋以“孝”治天下,所以\CJKunderwave{孝经}自晋以来,注家蜂起,而皇帝也亲自讲倡。既为朝廷所尚,皇帝也要讲\CJKunderwave{孝经},事关重大,于是谢安兄弟,这些朝廷大臣便在家里预先讲习。\CJKunderwave{孝经}关乎“至德要道”,“始于事亲,中于事君,终于立身”,文辞虽不多,然其间大义多可讲论生发,因而以好学博雅,“辩识义理”著名的车胤,自然会有心得、疑问;况且依刘孝标注引\CJKunderwave{续晋阳秋}的说法,车胤负责“摘句”提问,这就需要在预先讲习的过程中,与诸侍讲的朝臣讨论出一致的意见,不至在皇帝面前争议而闹得尴尬,所以预先必然要反复问难。而面对身为权要重臣的谢安兄弟,车胤确实有些为难,既想发挥心得,问难疑窦,又怕掌握不住分寸得罪了他们。袁羊的回话可谓生动,比喻精妙,说理透辟,就言语角度说是精彩的,可对于车胤说来,这又是一句大而化之的话,难解其忧。}

依余嘉锡先生的说法,袁羊并未参与宁康三年孝武帝的这次讲经,则\CJKunderwave{世说}采撷时闻,或与事实有出入,本则关注的只是言语的精妙而已。

\lettrine{2.91} 王乎(子)敬云\footnote{王子敬:王献之,(344—388),出于琅邪王氏家族。曾任谢安长史,官至中书令,故称王令或王大令。据\CJKunderwave{晋书·后妃传},尚简文帝女新安公主。少有令名,“风流一时之冠”。其书法已造神境,与父羲之并称“二王”。病笃:病重。此“王乎敬”,当作“王子敬”。}:“从山阴道上行\footnote{山阴:县名,晋时属会稽郡,治所在今浙江绍兴。},{\fzxk\zihao{6}\textcolor{red}{\CJKunderwave{会稽土地志}曰:“邑在山阴,故以名焉。”}} 山川自相映发\footnote{映发:辉映衬托。},使人应接不暇\footnote{应接不暇:美景众多,欣赏不过来。}。若秋冬之际,尤难为怀\footnote{难为怀:难以表达美丽景色给人带来的心情、感受。怀,心情、感受。}。”{\fzxk\zihao{6}\textcolor{red}{\CJKunderwave{会稽郡记}曰:“会稽境特多名山水。峰崿隆峻,吐纳云雾。松栝枫柏,擢干竦条。潭壑镜彻,清流写注。正(王)子敬见之,曰:‘山水之美,使人应接不暇。’”}}

{\cangkai\zihao{5}【评】凌濛初说:“合长康、子敬语一阅,便可卧游山阴道。”长康语(参见本篇88)是以画面式的描摹,展现了会稽山水的精神风貌,意在画面之中;\CJKunderwave{世说新语笺疏}引刘盼遂曰:“\CJKunderwave{戏鸿堂帖}载子敬\CJKunderwave{杂帖}云:‘镜湖澄澈,清流写(按,读为泻)注,山川之美,使人应接不暇。’较\CJKunderwave{世说}为详。”镜湖即今“鉴湖”,在绍兴,东汉会稽太守马臻纳山阴、会稽三十六源之水而成,此亦为山阴道上景,可与本则参读。子敬此一描摹,不仅有诗人想象“道上”诸多美景的连续画面,更在于他的感受,直抒胸臆,令人心向往之。然而他们共同神会着山阴道上的自然风韵,通过各异的描述,让人感受出山阴道上的美景妙意。而子敬语也道出了这位书法艺术家特有的审美情趣和审美水平。}

袁宏道说:“会稽诸山,遥望实佳,尖秀淡冶,亦自可人。昔王子敬语人,但云山阴道上,道上二字,可谓传神。”

\lettrine{2.92} 谢太傅问诸子侄\footnote{谢太傅:谢安。谢奕(?—358):字无奕,谢安长兄,陈郡阳夏谢氏家族在东晋初期的代表人物之一。}:“子弟亦何预人事\footnote{预:关涉、相干。},而正欲使其佳\footnote{正:必、一定。佳:好、出色。}?”诸人莫有言者,车骑答曰\footnote{车骑:指谢玄,谢安侄。}:{\fzxk\zihao{6}\textcolor{red}{谢玄。}} “譬如芝兰玉树\footnote{芝兰玉树:芝兰是香草,玉树为传说中的仙树,此用以比喻优秀子弟。},欲使其生于阶庭耳\footnote{阶庭:庭院。}。”

{\cangkai\zihao{5}【评】\CJKunderwave{孝经}把“行成于内,而立名于后世”,显身扬名,作为人生、家庭的最高境界。芝兰玉树生于阶庭,人才济济,满门焕乎光彩荣耀,这便成了古时对自家门庭的最高期待。谢安的启发,含着殷殷的期望,而谢玄的确颖悟慧捷,一下子感受到了长辈的良苦用心,脱口而答,不但口才英发,珠圆玉润,而且画龙点睛,直取主题。谢玄的对答,将谢家人物、谢家风貌都点活了。在东晋门阀政治中,高门士族,无不关心家族利益。为保障家族利益,又必须培养子侄后辈,使之成为国家栋梁之材。“芝兰玉树”之喻,正是这一理念的形象写照。}

\lettrine{2.93} 道壹道人好整饰音辞\footnote{道壹道人:东晋高僧,名德,吴人。俗姓陆,“少出家,贞正有学业”,后从高僧竺法汰受学,依师姓,亦称竺道壹。道人,晋时称僧人为道人。整饰:调整修饰。音辞:言辞。},{\fzxk\zihao{6}\textcolor{red}{王珣\CJKunderwave{游严陵濑诗}叙曰:“道壹姓竺氏。”\CJKunderwave{名德沙门题目}曰:“道壹文锋富赡。孙绰为之赞曰:‘驰骋游说,言固不虚,唯兹壹公,绰然有馀。譬若春圃,载芬载敷。条柯猗蔚,枝干扶疏。’”}} 从都下还东山\footnote{都下:京都。东山:山名,在今浙江上虞西南,当时名士如谢安等在此游处。},经吴中\footnote{吴中:吴郡地区。}。已而会雪下\footnote{已而:不久。会:正赶上。},未甚寒,诸道人问在道所经。壹公曰:“风霜固所不论\footnote{固:本来。},乃先集其惨澹\footnote{乃:竞。集:聚合。惨澹:萧索、凄清。};郊邑正自飘瞥\footnote{郊邑:乡间和城邑。正自:正在。飘瞥:急促飞过。},林岫便自皓然\footnote{林岫:林木峰峦。皓然:洁白而光亮的样子。}。”

{\cangkai\zihao{5}【评】依\CJKunderwave{高僧传},道壹师从竺法汰,博练经义,“思彻渊深”,又与汰公另一弟子,雅善\CJKunderwave{老}、\CJKunderwave{易}的昙一友善,“名德相继”。这说明他深受师门熏陶,修养与气质与一般名士不大一样。他一方面习惯于锻炼思理,另一方面,又与时风一致,崇尚文辞。这样,他在本则中表现的辞采就有些异样。不像前几则顾恺之、王献之那样,自然流畅地表达所见所感,而是刻意炼辞炼句,注重思理、文辞本身的优美。于是,刘辰翁说他:“问易答难,他人无此情也。”“小儿学语,体格未成,利锥书袋,面目可憎。”但他对自然雪景的感受,是真实而细腻的,体现了时人的审美趣味,显得可爱动人。他的文辞,工巧优美,可看作是稍后描摹山水作品“情必极貌以写物,辞必穷力而追新”(\CJKunderwave{文心雕龙·明诗})的早期消息。}

\lettrine{2.94} 张天锡为凉州刺史\footnote{张天锡:见刘孝标注。凉州:汉所置州部,辖境相当于今甘肃、宁夏、青海和内蒙古部分地区。},称制西隅\footnote{称制:行使皇帝的权力。制:皇帝的命令。西隅:西部角落,此指凉州。}。既为符(苻)坚所禽\footnote{既:不久。苻坚:字永固,氐人,前秦国君。禽:通“擒”。},用为侍中\footnote{侍中:官名,侍从皇帝,掌礼仪,护驾陪乘,备顾问。}。后于寿阳俱败\footnote{寿阳:县名,晋寿春,因晋武帝避祖母郑太后阿春讳,改称寿阳。今安徽寿县,谢石、谢玄淝水之战大败苻坚于此。},至都,{\fzxk\zihao{6}\textcolor{red}{张资\CJKunderwave{凉州记}曰:“天锡字纯嘏,安定乌氏人,张耳后也。曾祖轨,永嘉中为凉州刺史,值京师大乱,遂据凉土。天锡篡位,自立为凉州牧。符(苻)坚使将姚苌攻没凉州,天锡归长安,坚以为侍中、比部尚书、归义侯。从坚至寿阳。坚军败,遂南归,拜散骑常侍、西平公。”\CJKunderwave{中兴书}曰:“天锡后以贫拜庐江太守,薨赠侍中。”}} 为孝武所器,每入言论\footnote{言论:谈论、谈话。},无不竟日。颇有嫉己者,于坐问张:“北方何物可贵?”张曰:“桑椹甘香\footnote{桑椹:桑树的果实。椹,同“葚”。},鸱鸮革响\footnote{鸱鸮:猫头鹰。革响:变了声音。\CJKunderwave{世说笺本}:“北方有桑葚之甘香,飞鸮食之,故能变恶音为好音。”},{\fzxk\zihao{6}\textcolor{red}{\CJKunderwave{诗·鲁颂}曰:“翩彼飞鸮,集于泮林。食我桑椹,怀我好音。”}} 淳酪养性\footnote{淳酪:醇厚的乳酪。},人无嫉心。”{\fzxk\zihao{6}\textcolor{red}{\CJKunderwave{西河旧事}曰:“河西牛羊肥,酪过精好,但写酪置革上,都不解散也。”}}

{\cangkai\zihao{5}【评】史称“天锡少有文才,流誉远近”,是一贵胄才子。然而,他又是屡降之虏,先是丢了西北王,成了苻坚的“归义侯”,再于淝水一役归顺东晋,虽受到晋的恩遇,“以天锡为散骑常侍、左员外”,继又“拜金紫光禄大夫”,但毕竟是亡国之馀,“朝士以其国破身虏,多共毁之”。据\CJKunderwave{晋书},此“嫉已者”,又是权倾天下的司马道子。面对这样的情势,张天锡的回答既见“文才”,也表现了高傲难摧的个性。}

此用\CJKunderwave{诗经·鲁颂·泮水}意,“翩彼飞鸮,集于泮林。食我桑葚,怀我好音”。原诗祝颂鲁僖公能使淮夷来贡献方物。张天锡将自己比作淮夷朝鲁僖公,来颂美孝武帝。桑葚甜美,借比北方物产。猫头鹰叫声原难听,现在变得好听了。此借说“怀我好音”,意谓向晋臣服,而晋怀其德。以此来回答挑衅者不无蔑视的问话,江南鱼米之乡,北方朔漠荒凉,“北方有什么东西可贵?”不难见出,天锡之答,在委婉之中,藏着骨鲠之气,针对司马道子,这话便显得机巧而有个性。

\lettrine{2.95} 顾长康拜桓宣武墓\footnote{顾长康:顾恺之。桓宣武:桓温。},作诗云:“山崩溟海竭\footnote{溟海: 大海。},鱼鸟将何依。”{\fzxk\zihao{6}\textcolor{red}{宋明帝\CJKunderwave{文章志}曰:“恺之为桓温参军,甚被亲昵。”}} 人问之曰:“卿凭重桓乃尔\footnote{凭重: 倚重。乃尔: 竟然如此。},哭之状其可见乎?”顾曰:“鼻如广莫长风\footnote{广莫长风:即广莫风,指北风,语出\CJKunderwave{淮南子·天文训}。},眼如悬河决溜\footnote{悬河:瀑布。决溜:河堤溃决,水流奔泻。}。”{\fzxk\zihao{6}\textcolor{red}{\CJKunderwave{春秋考异邮}曰:“距不周风四十五日,广莫风至。广莫者,精大备也,盖北风也,一曰寒风。”}} 或曰:“声如震雷破山,泪如倾河注海。”

{\cangkai\zihao{5}【评】顾恺之为桓温引为参军,甚受赏识、亲重,温死,他作诗表达无所依恃的失落与哀痛,其情是认真的。余嘉锡先生评论,其父顾悦之曾持正与桓温斗争,桓温本又是一个心怀篡逆之志的枭雄之臣,而“恺之身为悦子,怀温入幕之遇,忘其问鼎之奸。感激伤恸,至于如此。此固可见温之能牢笼才俊,而当时士大夫之不识名义,亦已甚矣!恺之痴人,无足深责尔。”(\CJKunderwave{世说新语笺疏})“当时士大夫之不识名义”确有人在。若言本则,倒是可见恺之的重情,一旦为温所知遇,就不顾一切,甚至连传统道德的基本原则也忽略了,其“痴”如此。正因为如此,时人或有不平而为难他的人,故意要问他的哭状。他即顺口描述一番。和前面的诗句一样,所用文句,都是极度的夸张。寻绎恺之性情,他本来就有“衿伐过实”的特性,每自信、自夸,言过其实,曾引来少年的“戏弄”,因而,若在他人,本则的夸张句子已近狂怪失实,而在恺之,却绝不是他的“谐谑”之言,其极力渲染自己的感激伤恸,尽量描绘出真情与气势,情真之“痴”,可见一斑。}

\lettrine{2.96} 毛伯成既负其才气\footnote{负:自负。称:宣称。},常称:“宁为兰摧玉折\footnote{宁:宁可。兰摧玉折:比喻洁身自好却遭受迫害而死。兰、玉,喻高尚格。摧折,摧毁折断。},不作萧敷艾荣\footnote{萧敷艾荣:比喻丧失志节追求荣华富贵。萧、艾,均为恶草。敷、荣,开花。}。”{\fzxk\zihao{6}\textcolor{red}{\CJKunderwave{征西寮属名}曰:“毛玄字伯成,颍川人。仕至征西行军参军。”}}

{\cangkai\zihao{5}【评】这一宣称,表达着\CJKunderwave{离骚}品格。屈原守死善道,绝不苟同流俗的精神,在魏晋名士的人格中得到了深刻的反响。\CJKunderwave{离骚}:“人好恶其不同兮,惟此党人其独异。户服艾以盈要兮,谓幽兰其不可佩。”“何昔日之芳草兮,今直为此萧艾也。”屈原在其作品中,不断声明着自己佩玉、佩芳,象征着高尚其节操,芬芳其人格,而痛斥流俗,艾草盈腰,是非不分,黑白颠倒;痛惜昔日芳草人品,堕落变节为萧艾丑类。孔子说:“不得中道而与之,必也狂狷乎!狂者进取,狷者有所不为也。”(\CJKunderwave{论语·子路})不得与具有“中庸”理想人格者为伍,那就和沿正道进取,敢于舍得一身剐,特立独行,而绝不与苟同流俗者同调。屈原的狂狷品格在魏晋士人精神中发扬光大。伯成之言,感人的不是言语机巧,而是人品风范。}

\lettrine{2.97} 范宁作豫章\footnote{豫章:郡名,辖境为今江西省大部分地区,郡治在今南昌。此指豫章太守。},{\fzxk\zihao{6}\textcolor{red}{\CJKunderwave{中兴书}曰:“宁字武子,慎阳县人。博学通览。累迁中书郎、豫章太守。”}} 八日请佛\footnote{八日请佛:农历四月初八,为佛祖释迦牟尼诞辰,各寺院设法会,名香浸水浴佛,请佛像等,做一系列佛事,俗称“浴佛节”。},有板\footnote{板:札牍。此指写在木板上的礼佛、请佛文书。}。众僧疑或欲作答,有小沙弥在坐末曰\footnote{沙弥:梵文音译,指初戒出家的年轻和尚。}:“世尊嘿然\footnote{世尊:佛教徒对释迦牟尼的尊称。嘿:袁本作“默”,“嘿”同“默”。},则为许可。”众从其义\footnote{从:依从。义:通“议”,看法。}。

{\cangkai\zihao{5}【评】请佛像而附有疏板,是一种郑重礼敬,亦应答复,小和尚说话乖巧,拿世尊默然为由(佛像永远默然),免去了回复之劳。刘辰翁云:“代佛何呆,小沙弥故俊。”替佛祖写回话,是够呆傻的,小沙弥真有悟性,俊语解众僧之烦。}

此俊语,或为常典。\CJKunderwave{中本起经}:“(长者伯勒)整心白佛:‘唯愿世尊顾下薄食。’佛法默然已为许可。长者欣悦,接足而退,还家具膳。”\CJKunderwave{高僧传·杯度传}:“时湖沟有朱文殊者,少奉法。(杯)度多来其家,文殊谓度曰:‘弟子脱舍身没苦,愿见救;脱在好处,愿为法侣。’度不答。文殊喜曰:‘佛法默然,已为许矣。’”可见“世尊默然,已为许可”话头,佛门常用。而在本则情境之下,该句表意恰到好处,传情颇具幽默,不失为妙语。王世懋赞曰:“其义甚佳。”

\lettrine{2.98} 司马太傅斋中夜坐\footnote{斋:房舍。},{\fzxk\zihao{6}\textcolor{red}{\CJKunderwave{孝文(《晋书}作“文孝”,是)王传》曰:“王讳道子,简文皇帝第五子也。封会稽王、领司徒、扬州刺史,进太傅。为桓玄所害,赠丞相。”}} 于时天月明净,都无纤翳\footnote{都:完全。表程度的副词,多用在否定词前。纤:细微,一点点。翳:遮蔽。}。太傅叹以为佳。谢景重在坐\footnote{谢景重:谢重,字景重。曾作会稽王司马道子长史。刘孝标注中“摈贾树秋”,袁本作“续晋阳秋”,是。}{\fzxk\zihao{6}\textcolor{red}{,摈贾树秋(\CJKunderwave{续晋阳秋})曰:“谢重字景重,陈和(郡)人。哭(父)朗,东阳太守。重明秀有才会,终骠骑长史。”}} 答曰:“意谓乃不如微云点缀。”太傅因戏谢\footnote{戏:调侃。}曰:“卿居心不净,乃复强欲滓秽太清邪\footnote{乃复:竟然。复,助词,无实义。太清:天空。}?”

{\cangkai\zihao{5}【评】此则问对,描画出魏晋人悉心赏玩月色的场景,很珍贵。清空朗月,司马道子油然兴叹,此种赏会既有自然本身所给予的审美基础,也有时人审美趣味的因素。汉末以来,崇尚清流,人格以清为高,道子所叹,已含着人的崇尚与自然景色的往复回流的审美过程。谢景重之说,亦可谓妙境,依他的说法,夜月更富于动感和柔美。道子针对谢说之戏,虽为调侃,究其心理的深层次,更突出了“清”的人格崇尚,而且富有诗意与禅境,使得“清”的人格境界愈发引人,愈有魅力。}

\lettrine{2.99} 王中郎甚爱张天锡\footnote{王中郎:王坦之。},问之曰:“卿观过江诸人,经纬江左\footnote{经纬:治理。江左:江东,此指东晋王朝所统治的地区。},轨辙有何伟异\footnote{轨辙:本为车辙,此比喻遵循的法度。}?后来之彦\footnote{彦:有才学的士人。},复何如中原?”张曰:“研求幽邃\footnote{幽邃:幽深玄妙的道理,此指玄学之理。},自王、何以还\footnote{王、何:王弼、何晏,此二人于曹魏之时,倡导玄理,开清谈之风。};因时修制\footnote{因时:依据时势。修制:修定礼法制度。},荀、乐之风。”{\fzxk\zihao{6}\textcolor{red}{荀顗、荀勗修定法制,乐则未闻。}} 王曰:“卿知见有馀,何故为符(苻)坚所制?”{\fzxk\zihao{6}\textcolor{red}{张资\CJKunderwave{凉州记}曰:“天锡明鉴颖发,英声少著。”}} 答曰:“阳消阴息\footnote{阳消阴息:泛指客观规律的变化。语本\CJKunderwave{周易}的阴阳变化学说。消,消亡;息,生长。},故天步屯蹇\footnote{天步:国运,时运。屯、蹇:\CJKunderwave{周易}的两个卦名。二卦象与卦义皆为艰险困苦,后因称艰险困顿之事为“屯蹇”。},下(否)剥成象\footnote{下:当为“否”字之误。袁本“不”作“否”。否(pǐ痞)、剥是\CJKunderwave{周易}中的两卦。卦为阴盛碍阳,闭塞不通,剥落殆尽的凶咎之象。},岂足多讥!”

{\cangkai\zihao{5}【评】余嘉锡\CJKunderwave{世说新语笺疏}引程炎震曰:“坦之卒于宁康三年,天锡以淝水来降,不及见矣。此王中郎盖别是一人。”这也恰证明了\CJKunderwave{晋书}所记,天锡来降之后,朝士对之“多共毁之”是事实。本则的王中郎无论是哪一位,他对天锡具有如此见识但却国败身辱不能理解,虽然友善,问话却很重,如王辉说:“问语岸伟”,这的确反映了江左人士的普遍认识。无偏见尚且如此,若带偏见,便是“毁之”了。天锡有才,身居西凉而对魏晋、江左却能了如指掌,谈玄理、修制度他都关注,看得准确,一语中的;言及自身,他的自我辩护亦可谓聪慧巧智。天锡当国,主、客观因素都可谓之“屯蹇”、“否剥”。在他自身,耽于声色游晏,任人唯亲,荒于政治,有识之士苦谏而不悟;在客观,天灾人祸不断,他救之不及,只好眼睁睁将政权一损再损,直至落难江左,落到了形同“寓公”的田地。对此结局,他一概委之于“天步”,此真差可引司马迁评价项羽之论以说之:“乃引‘天亡我,非用兵之罪也’,岂不谬哉!”其“不觉悟而不自责,过矣”。天锡虽非项羽之比,然其自我回护之论,在本质上似为同调。}

人们对他嘲弄、“毁之”的根本原因,怕不只是结果,也有针对他败亡过程的因素。这里见出,天锡的修养和智巧,言语的水平,足可抵挡江左之士的非毁与问难。\CJKunderwave{世说}选此,突出的是名士飞扬的才气。

\lettrine{2.100} 谢景重女适王孝伯儿\footnote{谢景重:谢重,字景重。曾作会稽王司马道子长史。刘孝标注中“摈贾树秋”,袁本作“续晋阳秋”,是。适:嫁。王孝伯:王恭。},二门公甚相爱美\footnote{门公:家公,指父亲。爱美:亲敬。}。{\fzxk\zihao{6}\textcolor{red}{\CJKunderwave{谢女(氏)谱}曰:“重女月镜适王恭子愔之。”}} 谢为太傅长史\footnote{太傅:指司马道子。},被弹\footnote{弹:弹劾。}。王即取作长史\footnote{取:任用。},带晋陵郡\footnote{带晋陵郡:带,统辖;晋陵郡,东晋治所,在京口(镇江)。}。太傅已构嫌孝伯\footnote{构嫌:结怨。},不欲使其得谢,还取作谘议\footnote{还:再、又。谘议:官名,即谘议参军。六朝时各王府所置。},外示絷维\footnote{絷维:挽留。},而实以乖间之\footnote{乖间:隔离、分开。}。及孝伯败后,太傅绕东府城行散\footnote{行散:服食五石散后,需缓步行走,以散发药性,称为行散、行药。},{\fzxk\zihao{6}\textcolor{red}{\CJKunderwave{丹阳记}曰:“东府城西,有简文为会稽王时第,东则孝文王(‘孝文’,\CJKunderwave{晋书}作‘文孝’,是)道子府。道子领扬州,仍住先舍,故俗称‘东府’。”}} 僚属悉在南门,要望候拜,时谓谢曰:“王(阿)宁异谋\footnote{异谋:不轨的图谋。此指王恭举兵攻讨司马道子。},{\fzxk\zihao{6}\textcolor{red}{阿宁,王恭小字也。}} 云是卿为其计。”谢曾无惧色,敛笏对曰:“乐彦辅有言:‘岂以五男易一女\footnote{曾:竟。乐彦辅:乐广。}?’”太傅善其对\footnote{善:认为好的。},因举酒劝之曰:“故自佳,故自佳\footnote{故自:的确。自,语助词,无实义。}!”

{\cangkai\zihao{5}【评】对本则的背景,刘应登有一种理解:“谓谢已与道子有嫌,王亦与道子成隙,恐谢去职而还,为道子所害,故留之依己也。”司马道子非大器之才,好弄手腕,喜做小动作,而且位居权要,对王恭恨之入骨,为除这心腹之患,“日夜谋议”,所以对这甚相亲敬的两亲家,是不能容忍的。刘应登的理解确有道理。而司马道子留用谢重,又如本则所解析,实为“乖间”,不令两人凑到一起,分解对自己威胁的力量。这样看,在王恭讨司马道子之后,谢重面临的压力就不言而喻了。道子的问话,是藏着杀机的,直指谢为王恭谋主,谋主是要格杀勿论的——当然这是司马道子敲山震虎的虚张声势,他情知谢并不是王讨伐自己的谋主。但道子的心机、态度却是显而易见,且咄咄逼人的。面对如此困境,见出谢重的才能。他的从容,除了他心中无鬼,也见出他周旋于权力旋涡中的定力和智慧。所以他的一句恰到好处的回话,让道子赞叹。乐广当时的话,固然精彩(参见本篇25),而谢重用到这里,更显得是绝大的智巧。乐广当时情形与谢此时有着很多相似之处,乐广语稔用于此,正是天作巧合,十分精妙,同时也破解了道子猜疑。故道子有“故自佳,故自佳”之言。}

\lettrine{2.101} 桓玄义兴还后\footnote{桓玄:(369—404),袭父温之爵南郡公,故称。安帝时任江州刺史、都督荆州八郡诸军事,率军东下,篡晋自立,建国号楚。旋被刘裕击败,斩首京师。杨广(?—399):曾官淮南太守,南蛮校尉,后与弟佺期俱被桓玄攻杀。殷荆州:指殷仲堪。义兴:郡名。治所在阳羡,今江苏宜兴。据余嘉锡考,桓玄出为义兴太守,不得志,仅十许日即擅自去官。},见司马太傅\footnote{司马太傅:司马道子。},太傅已醉,坐上多客,问人云:“桓温来欲作贼\footnote{桓温:桓温曾有三次北征,刘盼遂\CJKunderwave{世说新语校笺}考订,此次当为太和四年(369)之征。时桓温已58岁。“来”上当脱一“晚”字,“晚来”,即晚年。\CJKunderwave{晋书·会稽王道子传}即作“桓温晚涂欲作贼”。作贼,谋逆造反。桓温雄豪,早有不臣之心,执掌权要,将废帝司马奕为海西公,立简文帝司马昱;晚年辅孝武帝司马曜,要求朝廷加九锡,欲逼司马氏禅位。事未果而桓温病死。道子所谓“欲作贼”即指此。},如何?”{\fzxk\zihao{6}\textcolor{red}{\CJKunderwave{晋安帝纪}曰:“温在姑熟(孰),讽朝廷求九锡。谢安使吏部郎袁宏具其草,以示仆射王彪之。彪之作色曰:‘大(丈)夫岂以此事语人邪?’安徐问其计。彪之曰:‘闻其疾已笃,且可缓其事。’安从之,故不行。”}} 桓玄伏不得起。谢景重时为长史\footnote{谢景重:谢重。},举板答\footnote{板:手板,即笏。}曰:“故宣武公黜昏暗,登圣明\footnote{黜昏暗,登圣明:指桓温废黜废帝,立简文帝事。},功超伊、霍\footnote{伊、霍:伊,伊尹,名挚,商汤的宰相。汤死,辅佐汤孙太甲,太甲荒淫无道,伊尹将其放逐于桐。霍,霍光,汉武帝时大将军。受遗诏辅佐昭帝,昭帝死,迎立昌邑王贺,贺淫乱,废之而立宣帝。}。纷纭之议,裁之圣鉴\footnote{裁:裁决,论定。圣鉴:圣明的识鉴。}。”太傅曰:“我知!我知!”即举酒云:“桓义兴,劝卿酒。”桓出谢过。{\fzxk\zihao{6}\textcolor{red}{檀道鸾论之曰:“道子可谓易于由言,谢重能解纷纭矣。”}}

{\cangkai\zihao{5}【评】余嘉锡先生\CJKunderwave{笺疏}议本则:“桓玄飞扬跋扈,包藏祸心,蜷伏爪牙,观衅而动,能早除之固善。然道子昏庸,见不及此。本无杀之之意,而乘醉肆詈,辱及所生,使之羞愤难堪。”“玄之伏不能起,不徒以道子直斥温名,加以大逆,使之无地自容而已,直恐其醉中暴怒,于座上收缚,或牵出就刑,故惧而流汗耳。”司马道子确实昏庸,既无深见识,又无驭服桓玄之术,逞酒诟骂,不留馀地。直指桓温名字,是重辱桓玄;又讲桓温欲谋逆造反,这罪名大得惊人,是任何臣子都惶恐战栗的。既无意于杀桓玄,又把事情闹到这步田地,足见道子心胸浅陋。这样一闹,不止桓玄,而是举座尴尬,所以,谢重出来打圆场。他的话很巧妙,如刘应登言:“谢乃举其废立之事言之,盖废海西立简文,道子乃简文第五子也。可谓善解纷矣。”景重的话说到了要害处,无桓温之废立,就没有道子之父简文做皇帝,而“黜昏暗,登圣明”,桓温也便化为伊、霍之论。这不但提醒了道子,也掩盖了桓温的野心,于桓玄、道子两方都顺理成章,十分体面,都可下得来台,言语圆转,足解一时之纠纷。}

景重有才而道子无术,在本则的描绘中,表现得清晰如画。

\CJKunderwave{晋书}记此事,当时虽解了一时之纷,而桓玄却对道子益发切齿深恨,可见道子的愚蠢。

\lettrine{2.102} 宣武移镇南州\footnote{宣武:桓温。南州:东晋时城名,又名姑孰,故址在今安徽当涂。地当长江重要渡口,为都城建康的门户。姑孰在都城西南,故称南州。},制街衢平直\footnote{街衢:泛指街道。衢,四通八达的大道。}。人谓王东亭曰:{\fzxk\zihao{6}\textcolor{red}{\CJKunderwave{王司徒传}曰:“王珣字元琳,丞相导之孙,领军洽之于也。少以清秀称。大司马桓温辟为主簿,从讨袁真,封交趾望海县东亭侯,累迁尚书左仆射,领选,进尚书令。”}} “丞相初营建康,无所因承\footnote{因承:借鉴,参照。},而制置纡曲\footnote{制置:修建布置。},方此为劣\footnote{方:比、相比。}。”{\fzxk\zihao{6}\textcolor{red}{\CJKunderwave{晋阳秋}曰:“苏峻既诛,大事克平之后,都邑残荒。温峤议徙都豫章,以即丰全。朝士及三吴豪杰,谓可迁都会稽,王导独谓:‘不宜迁都。建业,往之秣陵,古者既有帝王所治之表,又孙仲谋、刘玄德俱谓是王者之宅。今虽凋残,宜修劳、来、旋、定之道,镇静群情。且百堵皆作,何患不克复乎!’终至康宁,导之策也。”}} 东亭曰:“此丞相乃所以为巧。江左地促,不如中国\footnote{中国:指中原地区。};若使阡陌条畅\footnote{阡陌:田间小道,南北向为“阡”,东西向为“陌”。这里比喻道路的平直。},则一览而尽。故纡馀委曲\footnote{纡馀委曲:迂回曲折的样子。},若不可测。”

{\cangkai\zihao{5}【评】余嘉锡先生\CJKunderwave{笺疏}云:“\CJKunderwave{景定建康志}十六云:‘今台城在府城东北,而御街迤逦向南,属之朱雀门。’则其势诚纡回深远不可测矣。”王导因地制宜经营建康城,道路纡馀委曲,深不可测。人以道路当平直而非之。王珣的一席话,不仅很有道理地回护了其祖王导,更有意味的是,他所表达的审美情趣。道路平直是一种味道,而纡馀委曲也是一种美感。丞相之巧,不仅表现在能顺乎自然,造出一个功用完好的城市,更重要的是达到了一种审美的境界。含蓄无尽之美,是古典美学所崇尚的审美境界,人能把这种审美崇尚表达在城市的建造上,使其在功能俱备的前提下,又成为一个深富审美意味的艺术品,这不是大巧么?以此为巧,正可见魏晋人对美感的自觉追求与崇尚,它不仅止于文学、艺术、山水,而是全方位的自觉。}

\lettrine{2.103} 桓玄诣殷荆州\footnote{桓玄:(369—404),袭父温之爵南郡公,故称。安帝时任江州刺史、都督荆州八郡诸军事,率军东下,篡晋自立,建国号楚。旋被刘裕击败,斩首京师。杨广(?—399):曾官淮南太守,南蛮校尉,后与弟佺期俱被桓玄攻杀。殷荆州:指殷仲堪。殷荆州:殷仲堪,曾任荆州刺史。},殷在妾房昼眠,左右辞不之通\footnote{通:通报。}。桓后言及此事,殷云:“初不眠\footnote{初不:从来不,根本没有。},纵有此\footnote{纵:即使。},岂不以‘贤贤易色’也”\footnote{“贤贤”句:语出\CJKunderwave{论语·学而},意谓尊敬、看重贤人,轻视女色。易,调换。色,女色。}。{\fzxk\zihao{6}\textcolor{red}{孔安国注\CJKunderwave{论语}曰:“言以好色之心好贤人则善。”}}

{\cangkai\zihao{5}【评】凌濛初斥责殷仲堪:“饰语,厚颜。”}

从言语角度说,这里殷仲堪巧用了\CJKunderwave{论语}的现成话,合于情景,恰到好处,差可文过饰非。\CJKunderwave{论语·学而}:“子夏曰:‘贤贤易色;事父母,能竭其力;事君,能尽其身;与朋友交,言而有信。虽未学,吾必谓之学矣。”子夏说的这四者,都是难能可贵的人伦境界。孔子曾说过:“吾未见好德如好色者。”能将好色之心换为尊贤之心,这需要很深厚的修养,而能达到子夏说的四者,则必定是学有成就的贤人君子。殷仲堪情急之中顺便称引\CJKunderwave{论语}此言,桓玄虽听得懂的,但仅能表现殷的言语智巧,眼前的尴尬是无论如何也解不了的。

\lettrine{2.104} 桓玄问羊孚\footnote{桓玄:(369—404),袭父温之爵南郡公,故称。安帝时任江州刺史、都督荆州八郡诸军事,率军东下,篡晋自立,建国号楚。旋被刘裕击败,斩首京师。杨广(?—399):曾官淮南太守,南蛮校尉,后与弟佺期俱被桓玄攻杀。殷荆州:指殷仲堪。羊孚:见刘孝标注。羊后投桓玄,玄用为记室参军,为桓心腹。}:{\fzxk\zihao{6}\textcolor{red}{\CJKunderwave{羊氏谱}曰:“孚字子道,泰山人。祖楷,尚书郎。父绥,中书郎。孚历太学博士、州别驾、太尉参军。年四十六卒。”}} “何以共重吴声\footnote{吴声:吴地的民歌。\CJKunderwave{乐府诗集}卷四四:“盖自永嘉渡江之后,下及梁、陈,咸都建业,吴声歌曲起于是也。”今存三百四十多首,多为情歌。}?”羊曰:“当以其妖而浮\footnote{当:大概。}。”

{\cangkai\zihao{5}【评】吴声歌曲是当时的新声。这些江南的都市之歌,多为恋歌,表现了人们热烈而浪漫的情思,曲调繁富,抒情缠绵,辞采艳丽,极具水乡特色,是活泼泼流行于当时的流行歌曲,成为都市生活的重要部分。桓玄看到时人都珍重它,但何以会如此,却不能理解。羊孚的“妖而浮”之评,言虽不屑,但从艺术角度看却是评价中肯。以正统音乐做标准来衡量吴声歌曲,它的确显得妖冶而浮艳,具有清新真切的艺术新风,这对于固守传统的人来说,需要有一个认识的过程。但它的艺术趣味,有着不可阻遏的生命力、感召力,这又是现实中人所无法回避的。正如同战国时的魏文侯,听正统的先王雅乐惟恐睡着了,而听起郑、卫新声,则不知疲倦。梁惠王也和孟子说,只是爱听世俗之乐。这就是民歌的魅力,生活的魅力。吴声歌曲的现象,引起上流社会的关注、评价,正说明这一新声的影响力。}

羊孚的评价,若换一个角度看,他的确很敏锐地抓住了这一新声的特点——鲜活、艳丽。能有这样的评价,不仅说明了羊孚的艺术、审美修养,也说明了他对这些无法回避的流行歌曲的熟悉与钻研。

\lettrine{2.105} 谢混问羊孚\footnote{谢混:见刘孝标注。谢安孙,尚晋陵公主,后因党附刘毅,被太尉刘裕所诛。羊孚:见前篇。}:“何以器举瑚琏\footnote{瑚琏:古代祭祀时,用来盛黍稷的器皿。}?”{\fzxk\zihao{6}\textcolor{red}{\CJKunderwave{晋安帝纪}:“混字叔源,陈郡人,司空琰少子也。文学砥砺立名。累迁中书令、尚书左仆射。坐党刘毅伏诛。”\CJKunderwave{论语}:“子贡问曰:‘赐也何如?’子曰:‘汝器也。’曰:‘何器也?’曰:‘瑚琏也。’”郑玄\CJKunderwave{注}曰:“黍稷器。夏曰瑚,殷曰琏。”}} 羊曰:“故当以为接神之器\footnote{故当:当然是。}。”

{\cangkai\zihao{5}【评】谢混问器,实际是在问人。子贡是孔门的言语硕儒,“利口巧辞”,以其杰出的口才,游说齐、吴、越、晋而救鲁,结果是“子贡一出,存鲁,乱齐,破吴,强晋而霸越”(\CJKunderwave{史记·仲尼弟子列传}),他是孔门之中负有盛名的弟子,所谓“言语:宰我、子贡”。孔子为什么说他是瑚琏呢?瑚琏是宗庙祭祀用的尊贵之器,夏曰瑚、殷曰琏、周曰簠(方形)簋(圆形),用以盛黍稷。“国之大事,惟祀与戎”(\CJKunderwave{左传}),祀与戎都是安人民、存社稷的根本,因而,祭祀之重器就意味着国之重器。作为“接神之器”,承载着获得神灵辅佑的重大使命,其功用与地位不言而喻。孔子将子贡看作“瑚琏”,可见对他的赞赏与肯定的程度。本则提出“何以器举瑚琏”问题,实质是对类似子贡这样具有非凡才能的先贤的羡慕与追求,见出时人对才干的崇尚。而列在言语中,其实又隐括了言语之才,对成大器重要意义的理解。}

\lettrine{2.106} 桓玄既篡位,后御床微陷\footnote{御床:皇帝的座榻。},群臣失色。侍中殷仲文进\footnote{进:进言。}曰:{\fzxk\zihao{6}\textcolor{red}{\CJKunderwave{续晋阳秋}曰:“仲文字仲文,陈郡人。祖融,太常。父康,吴兴太守。闻玄平京邑,弃郡投焉。玄甚悦之,引为谘议参军。时王谧、见礼而不亲,卞范之被亲而少礼。其宠遇隆重,兼于王、卞矣。及玄篡位,以佐命亲贵,厚自封崇。舆马器服,穷极绮丽,后房妓妾数十,丝竹不绝音。性甚贪吝,多纳贿赂,家累千金,常若不足。玄既败,先投义军。累迁侍中、尚书,以罪伏诛。”}} “当由圣德渊重\footnote{渊重:深重。},厚地所以不能载。”时人善之。

{\cangkai\zihao{5}【评】殷仲文是桓玄的姊夫,玄不得志时不相友善,闻桓玄平京师,便弃郡投之。因有才藻,又是至亲,受到桓玄的宠遇。仲文本身是一个贪财谄谀之徒,毫无操守、原则,本则即活画出其谄谀嘴脸。王世贞云:“纵极澹辞,不能令人不呕哕。”王世懋曰:“群丑献谀,读之呕哕,哪得称佳?”这里虽显出仲文口才,然其丧心缺德,与裴楷解晋武帝探策得“一”而尽释群臣之危,有本质的不同(参见本篇19)。}

\lettrine{2.107} 桓玄既篡位,将改置直馆\footnote{直馆:值班、办公的官署。},问左右:“虎贲中郎省\footnote{虎贲中郎:官名,担负宫廷宿卫、侍从的职责,秩比六百石。置中郎将,秩比二千石。省:官署。},应在何处?”有人答曰:“无省。”当时绝迕旨\footnote{迕旨:违逆皇上的旨意。}。问:“何以知无?”答曰:“潘岳\CJKunderwave{秋兴赋叙}曰\footnote{潘岳:字安仁,西晋荥阳中牟(今河南)人。诗文名家,工诗赋,辞藻艳丽,善为哀诔之体。\CJKunderwave{秋兴赋}为其重要作品。}:‘余兼虎贲中郎将,寓直散骑之省\footnote{寓直:寄住在别的衙署办公。散骑:官名,散骑常侍的省称,为皇帝的谏官。}。’”{\fzxk\zihao{6}\textcolor{red}{岳别见。其(秋兴)赋叙曰:“晋十有四年,余年三十二,始见二毛,以太尉掾兼虎贲中郎将,寓直散骑之省。羞(高)阁连云,阳景罕曜。仆野人也,猥厕朝列,譬犹池鱼笼鸟,有江湖山薮之思。于是染翰操帋(纸),慨然而赋。于时秋至,故以\CJKunderwave{秋兴}命篇。”}} 玄咨嗟称善\footnote{咨嗟:赞叹。}。{\fzxk\zihao{6}\textcolor{red}{刘谦之\CJKunderwave{晋纪}曰:“玄欲复虎贲中郎将,宜(疑)应直与不,访之僚佐,咸莫能定。参军刘简之对曰:‘昔潘岳\CJKunderwave{秋兴赋序}云:“余兼虎贲中郎将,寓直于散骑之省。”以此言之,是应直也。’玄欢然从之。”此语微异,又答者未知姓名,故详载之。}}

{\cangkai\zihao{5}【评】本则可见魏晋时期对才情普遍赏识的风气。}

以桓玄雄豪跋扈的性格,若有回答生硬,当面忤旨者,其后果是可想而知的。这里桓玄的追问,已经透露出反感的凶信。但最后结果却是戏剧性的。回话人引潘岳赋叙,儒雅委婉、恰到好处地回答了桓玄的问题,桓的态度陡转直下,尽捐反感而“咨嗟称善”。刚刚篡位的桓玄,在最需要权威来巩固地位时,面对一个大胆的忤旨者能如此宽容,正是回话人当下的智巧、学养和文才打动了他。可见,就是桓玄这样的人,也崇尚才情。这一则记述,从一个独特的视角,生动地表现了魏晋风尚动人的一面。

\lettrine{2.108} 谢灵运好戴曲柄笠\footnote{谢灵运:见刘孝标注。南朝宋人,晋车骑将军谢玄孙,袭爵康乐公,又称谢康乐。其为人高才而性褊激,多愆礼度,与刘宋朝廷数不谐。乐游山水,擅长山水诗赋,为中国古代山水诗的创始人。后因被诬谋反,于广州遭受“弃市刑”,年四十九。曲柄笠:形似曲盖的斗笠。},{\fzxk\zihao{6}\textcolor{red}{丘渊之\CJKunderwave{新集录}曰:“灵运,陈郡阳夏人。祖玄,车骑将军。父涣(瑍),秘书郎。灵运历秘书监、侍中、临川内史。伏诛。”}} 孔隐士谓曰:“卿欲希心高远\footnote{希心:倾心,醉心。},何不能遗曲盖之貌\footnote{曲盖:古代高官出行的仪仗中所用的曲柄伞。它象征着权位。}?”{\fzxk\zihao{6}\textcolor{red}{\CJKunderwave{宋书}曰:“孔淳之字彦深,鲁国人。少以辞荣就约,征聘无所就。元嘉初,散骑郎征,不到,隐上虞山。”}} 谢答曰:“将不畏影者,未能忘怀\footnote{将不:莫非,莫不是。表揣测之词。}。”{\fzxk\zihao{6}\textcolor{red}{\CJKunderwave{庄子}云:“渔父谓孔子曰:‘人有畏影恶迹而去之走者,举足逾数而迹逾多,走逾疾而影不离,自以尚迟,疾走不休,绝力而死。不知处阴以休影,处静以息迹,愚亦甚矣!子修心守真,还以物与人,则无异矣。不修身而求之人,不亦外事者乎!’”}}

{\cangkai\zihao{5}【评】笠是野人高士的用物,高士是超绝尘想,视俗间荣华为敝屣的。然而自视清高的谢灵运戴的却是“曲柄笠”。虽为笠,可形制酷似象征着荣华富贵的“曲盖”,带着浓浓的希慕权势的印记,因而隐逸高士孔淳之便有了疑问。谢灵运之答,用\CJKunderwave{庄子}的典故,可谓绝妙。在自家,是法自然、贵天真,早已忘怀世俗,所以曲柄、富贵都无印象,只有那些守不住本真,心为世俗所系累的人,才对富贵荣华烙印深刻。孔隐士之论,正如同畏影子的人心里没能忘记影子,还没达到境界,未存本真,心有系累,因而才会看出“曲柄笠”的权势印记。如此一来,孔隐士的问难,反变成了自我嘲讽,而且如同那位蠢人,畏影恶迹而拼命奔跑,终因根本立场、方法出了毛病,硬是摆脱不掉,绝力而死,完全是一幕讽刺喜剧。可见,此一答,不仅将老庄玄理的深湛做了渲染,也使言语充满了幽默。谢灵运那摇曳着灵性,深识老庄玄理的风采,也便在这机智、幽默的回答之中荡逸而出了。}


%%% Local Variables:
%%% mode: latex
%%% TeX-engine: xetex
%%% TeX-master: "../Main"
%%% End:
