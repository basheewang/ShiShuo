%% -*- coding: utf-8 -*-
%% Time-stamp: <Chen Wang: 2025-12-09 23:00:58>

% ○ ◎ ‧ 「 」 『 』 々 ( ) “ ” ■ ^[一-龥]
% 【\([^】][^】][^】]+\)】 → {\\fzxk\\zihao{6}\\textcolor{red}{\1}}
% \(【评】.*\) → {\\cangkai\\zihao{5}\1}
% \(【题解】.*\) → {\\cangkai\\zihao{5}\1}
% 《\([^》]+\)》 → \\CJKunderwave{\1}
% ^\([0-9]+.[0-9]+\) → \\lettrine{\1}
% {\\fzxk\\zihao{6}\\textcolor{red}{[^o]*}}


\setlength{\parindent}{0pt}



\chapter{品藻第九}



{\cangkai\zihao{5}【题解】 品藻者,品评人物,定其高下也。品藻之风源于汉末清议,后在魏晋士人中间相沿成风,而偏重对人物之精神、才性的考量。魏晋士人受\CJKunderwave{老}、\CJKunderwave{庄}、\CJKunderwave{易}“三玄”影响,主张越名任心、委身自然,因此贯穿本篇的是对自由人格和自然境界的抒写。本门第17则所载明帝与谢鲲的对话,就极具典型意义地传达出了这一时代主题,也使其成为\CJKunderwave{世说}中的名篇。明帝问谢琨:“君自谓何如庾亮?”答曰:“端委庙堂,使百僚准则,臣不如亮;一丘一壑,自谓过之。”正是因为晋人内心希慕自然,才能对庙堂和权力看得相对平和,而非趋之若鹜。魏晋士人向外发现了自然,向内发现了自我的深情,自我的发现是魏晋时代的又一主题。本门第35则载,桓温少与殷浩齐名,常有竞心。桓问殷:“卿何如我?”殷曰:“我与我周旋久,宁作我。”晋人从过去被儒家礼教压抑、束缚中找回了久违的真我。他们嬉笑怒骂,皆成妙谛,展示了一个个丰富多彩的个性世界。}

{\cangkai\zihao{5}\CJKunderwave{世说新语}作为一部优秀的笔记小说,其简洁而传神的心理描写对后世小说当有一定开启作用,本门故事表现尤为突出。如第12则写到一代枭雄王敦在西朝时,见周辄以扇障面不停,显然是有一定心理障碍。后渡江,不再如此,王曰:“不知我进,伯仁退?”潜在意识中仍以周作为竞争对手。桓温亦有类似表现。38则载殷浩北伐失败被废,桓公语诸人曰:“少时与渊源共骑竹马,我弃去,己辄取之,故当出我下。”看似揭人短处不遗馀力,实则对殷浩仍有敬畏之心。两则小故事将两位政治家心灵世界的最潜在层面刻画得惟妙惟肖,极富人情意味。此外,第15则庾亮对王敦关于“何者居其右”的穷追不舍,第52则桓温欲说谢安、王坦之优劣,而“停欲言,中悔”的细节描写,都见出了常人心态。}

\lettrine{9.1} 汝南陈仲举\myidx{陈蕃}、颍川李元礼\myidx{李膺}\footnote{陈仲举:陈蕃,汝南平舆(今河南)人。(?—168):陈蕃字仲举,汝南平舆(今属河南)人。性方峻,不交非类。不畏强御而直言极谏,终为宦官所害。 李元礼(110—169):李膺字元礼,汉末颍川襄城(今属河南)人。在朝清议领袖之一,与杜密并称“李杜”。因反对宦官专政,被太学生称为“天下模楷”。后遭党锢之祸,死于狱中。},二人共论其功德\footnote{“二人共论”句:清人李慈铭考证,认为“二人”当为“士人”。},不能定先后。蔡伯喈\myidx{蔡伯喈}{\fzxk\zihao{6}\textcolor{red}{\CJKunderwave{绩(续)汉书}曰:“蔡伯喈,陈留圉人。通达有俊才,博学善属文,伎艺、术数无不精综。仕至左中郎将,为王允所诛。”}} 评之曰\footnote{蔡伯喈:蔡邕(132—192),字伯喈,东汉陈留圉(今河南杞县南)人。}:“陈仲举强于犯上\footnote{犯上:触犯君上。},李元礼严于摄下\footnote{摄下:威慑部下。};犯上难,摄下易。”{\fzxk\zihao{6}\textcolor{red}{张璠\CJKunderwave{汉纪}曰:“时人为之语曰:‘不畏强御陈仲举,天下摸(模)楷李元礼。’”}} 仲举遂在“三君”之下\footnote{三君:东汉末窦武、刘淑、陈蕃称“三君”。君,一世之宗。},{\fzxk\zihao{6}\textcolor{red}{谢沈\CJKunderwave{汉书}曰:“三君者,一时之所贵也。窦武、刘淑、陈蕃,少有高操,海内尊而称之,故得因以为目。”}} 元礼居“八俊”之上\footnote{八俊:东汉末李膺、荀昱、杜密、王畅、刘祐、魏朗、赵典、朱㝢称“八俊”。俊,人之英杰。}。{\fzxk\zihao{6}\textcolor{red}{薛莹\CJKunderwave{汉书}曰:“李膺、王畅、荀绲(昱)、朱㝢 、魏朗、刘佑(祐)、杜楷(密)、赵典为八俊。”\CJKunderwave{英雄记}曰:“先是张俭等相与作冠衣礼弹,弹中人相调,言:‘我弹中诚有八俊、八乂(及)、犹古之八元、八凯也。’”谢沈\CJKunderwave{书}曰:“八俊者,卓出之名也。”姚信\CJKunderwave{士纬}曰:“陈仲举胜气高烈,有王臣之节;李元礼忠壮正直,有社稷之能。海内论之未决,蔡伯喈抑一言以变之,疑论乃定也。”}}

{\cangkai\zihao{5}【评】东汉末,宦官当权,外戚专横。陈蕃为太傅,与大将军窦武谋诛宦官,反为所害。时有“不畏强御陈仲举”之语;李膺于朝纲不振之时,独持风裁,以声名自高,与太学生领袖郭泰等结交,时有“天下楷模李元礼”之评。蔡伯喈据此以为“犯上难,摄下易”,陈蕃高于李膺。蔡说未为的论。其实,犯上、摄下均需要一种以天下为己任的高度社会责任感和直面惨淡人生的勇气。犯上固然需要有“威武不能屈”的大丈夫气概;为士人表率亦须有当仁不让的儒者情怀做内在的支撑。君不见民众的一般心理吗——山不转水转,说不定谁求着谁,因此谁也不能得罪。在某种意义上说,李膺的做法更难以做到,因此二人之间不得强分高下。后党锢之祸起,李膺不求苟活,主动投狱,竟为正义事业而死,与陈蕃一样,死得其所。察其言、观其行可知,陈、李二人为中国传统知识分子的中坚,各以己之所长的方式为推动社会进步做出贡献,表现了儒者本色。}

\lettrine{9.2} 庞士元\myidx{庞统}至吴\footnote{庞士元:庞统字士元,号凤雏,汉末襄阳人。至吴:周瑜任南郡太守,庞统任功曹。周瑜死,庞统送丧至吴(治所在今苏州)。},吴人并友之。{\fzxk\zihao{6}\textcolor{red}{\CJKunderwave{蜀志}曰:“周瑜为岭(‘为岭’,袁本作‘领’,是)南郡,士元为功曹。瑜卒,士元送丧至吴,吴人多闻其名;及当还西,并会阖(阊)门,与士元言。”}} 见陆绩\myidx{陆绩}、{\fzxk\zihao{6}\textcolor{red}{\CJKunderwave{文士博(传)}曰:“绩字公纪。幼有隽朗才数,博学多通。庞士元年长于绩,共为交友。任(仕)至郁林太守。自知亡日,年三十二而卒。”}} 顾劭\myidx{顾劭}、全琮\myidx{全琮}\footnote{陆绩(187—219):字公纪,三国吴吴郡吴(今江苏苏州)人。顾劭:字孝则,吴相顾雍子。全琮(?—249):字子璜,三国吴吴郡钱塘(今浙江杭州)人。},{\fzxk\zihao{6}\textcolor{red}{环济\CJKunderwave{吴纪}曰:“琮字子黄(璜),吴郡钱塘人。有德行义槩,为大司马。”}} 而为之目曰\footnote{目:品评。}:“陆子所谓驽马有逸足之用\footnote{陆子:指陆绩。子,古代男子之美称。驽马:劣马。逸足:快足,捷足。},顾子所谓驽牛可以负重致远\footnote{驽牛:走不快的牛。负重致远:背负重物而达于远地。}。”或问:“如所目,陆为胜邪?”曰:“驽马虽精速,能致一人耳。驽牛一日行百里,所致岂一人哉?”吴人无以难\footnote{难:反驳,诘难。}。“全子好声名\footnote{好声名:看重名誉。\CJKunderwave{三国志·吴书·全琮传}载,琮父全柔,曾使宗运米数千斛至吴市出售,他全部用以赈济士大夫之贫者。},似汝南樊子昭\myidx{樊子昭}\footnote{樊子昭:东汉末汝南(在今河南)人。出身贫贱,以德行为许劭所奖拔。}。”{\fzxk\zihao{6}\textcolor{red}{蒋济\CJKunderwave{万机论}曰:“许子将褒贬不平,以拔樊子昭而抑许文休。刘晔难曰:‘子昭拔自贾竖,年至七十,退能守静,进不苟竞。’济答曰:‘子昭诚自幼至长,容貌完洁。然观其插齿牙,树颊颏,吐唇吻,自非文休之敌。’”}}

{\cangkai\zihao{5}【评】千里马虽追风绝响、奔逸绝尘,而百年不一遇,故世多驽马驽牛。驽马有逸足之用者,驽马中之佼佼者,虽不能一日千里,而于众马之中,固已出群矣;驽牛能负重致远,虽不如马之善走,然驮重负载动至千斤,亦非老弱不堪之牛。斯二者为马牛中之超群者,喻陆绩、顾劭性情一俊快、一厚重,各有其用,不能互相代替。世人不解,以为马胜牛。庞则反唇相讥,欲破世人观念中非此即彼之偏执,即荀子\CJKunderwave{劝学}篇所谓“骐骥一跃,不能十步;驽马十驾,功在不舍”意也。刘辰翁曰:“亦捷急变化语,即骏马所致亦如此耳。”斯言得之。庞士元之本意,亦非牛胜马。}

\lettrine{9.3} 顾劭\myidx{顾劭}尝与庞士元\myidx{庞统}宿语\footnote{顾劭:见前则。庞士元:庞统,见前则。宿语:夜里交谈。},问曰:“闻子名知人\footnote{知人:善于识别人。},吾与足下孰愈\footnote{孰:谁。愈:强,优胜。}?”曰:“陶冶世俗\footnote{陶冶:原指烧制陶器与冶炼金属。引申为化育、熏陶。世俗:社会风尚习俗。},与时浮沈\footnote{时:时势。浮沈:随波逐流。指随时而变。沈,同“沉”。},吾不如子;{\fzxk\zihao{6}\textcolor{red}{\CJKunderwave{吴志}曰:“劭好乐人伦,自州郡庶几及四方人事,往来相见,或讽议而去,或结友而别,风声流闻,远近称之。”}} 论王霸之馀策\footnote{王霸:王业与霸业。儒家谓施行仁政治理天下为王,凭借武力征服四方为霸。},览倚伏之要害\footnote{倚伏:语本\CJKunderwave{老子}:“祸兮福之所倚,福兮祸之所伏。”谓福祸相互依存转化。},吾似有一日之长\footnote{一日之长:谓在某方面略强些。}。”劭亦安其言\footnote{安:满意,认为妥当。}。{\fzxk\zihao{6}\textcolor{red}{\CJKunderwave{吴录}曰:“劭安其言,更亲之。”}}

{\cangkai\zihao{5}【评】顾劭结交广泛,聚合人望,为乡闾贤达,起家为豫章太守而风化大行;庞士元与诸葛亮并为刘备的智囊高参,屡出奇计,谈兵多中。此评虽用模糊语言造成一种各有短长高下的客观印象,但稍加分析可知有皮里阳秋之义。“陶冶世俗,与时沉浮”是一般干吏、勤吏所为,自是常才;“论王霸之馀策,览倚伏之要害”,则是运筹帷幄的廊庙器,不世出之大才。庞士元谦逊外表下透露出一股桀骜不驯之气,宜其有“凤雏”之号。刘辰翁评曰:“有怀其人。”凌濛初曰:“惜未见其止。”均对庞士元油然而生景仰之情。}

\lettrine{9.4} 诸葛瑾\myidx{诸葛瑾}、弟亮\myidx{诸葛亮}及从弟诞\myidx{诸葛诞}\footnote{诸葛瑾(174—241):字子瑜,三国琅邪阳都(今山东沂水南)人。诸葛亮之兄。亮:诸葛亮。从弟:族弟,堂弟。诞:诸葛诞(?一258),字公休。仕魏,见司马氏秉政,夏侯玄等被诛灭,于甘露二年(257)称臣于吴,据寿春反魏。},{\fzxk\zihao{6}\textcolor{red}{\CJKunderwave{吴书}曰:“瑾字子瑜,其先葛氏,琅邪诸县人,后徙阳都。阳都先有姓葛者,时人谓诸葛,因为氏。瑾少以至孝称,累迁豫州牧,六十八卒。”\CJKunderwave{魏志}曰:“诞字公休,为吏部郎。人有所属托,辄显其言而亟用之,后有得失当不,则公议其得失以为褒贬。自是群寮莫不慎其所举。累迁扬州刺史、镇东将军,以其谋逆伏诛。”}} 并有盛名,各在一国。于时以为蜀得其龙,吴得其虎,魏得其狗\footnote{狗:以狗比诸葛诞,有人以为是司马氏之党诋毁他;也有人以为狗泛指幼小动物,因诸葛诞在三兄弟中最幼,故称。}。诞在魏,与夏侯玄\myidx{夏侯玄}齐名\footnote{夏侯玄(209—254):字太初,三国魏人。曹爽辅政时,他以爽姑之子受重用。曹爽被诛,玄废黜。后与李丰等谋杀司马师,事败,同被诛。他是早期的玄学领袖人物。};瑾在吴,吴朝服其弘量\footnote{弘量:宏大的气度。比喻人有大才。}。{\fzxk\zihao{6}\textcolor{red}{\CJKunderwave{吴书}曰:“瑾避乱渡江,大皇帝取为长史,遣使蜀,但与弟亮公会相见,退无私面。而又有容貌思度,时人服其弘量。”}}

{\cangkai\zihao{5}【评】诸葛氏三兄弟各有龙、虎、狗之评,属于\CJKunderwave{世说}常用的比兴手法。龙、虎、狗之喻,盖当时拟人之佳评,虽略分高下层次,但却不可以今人之眼光强加褒贬,谓狗为劣评也。龙虎喻人中豪杰,狗仅下其一等,亦甚有功用。太公\CJKunderwave{六韬}以文、武、龙、虎、豹、犬为排列次序可知。又\CJKunderwave{史记·萧相国世家}载高祖分封功臣时云:“夫猎,追杀兽兔者狗也,而发踪指示兽处者人也。今诸君徒能走兽耳,功狗也。至如萧何,发踪指示,功人也。”刘邦将攻城略地的将帅比为“功狗”,绝非蔑称。不仅如此,更有人甘当走狗。如清代画家郑板桥服膺明代狂人徐渭(号青藤),治印一方,上刻字自称“青藤门下走狗”;无独有偶,齐白石老人亦有诗曰:“青藤雪个远凡胎,老否衰年别有才。我欲九原为走狗,三家门下转轮来。”寄托了对前辈画家徐渭、朱耷、吴昌硕等人的顶礼膜拜之情。西方文化中亦有以狗自喻者,如\CJKunderwave{天演论}的作者赫胥黎非常崇拜达尔文,他勇敢地捍卫达尔文的进化论学说,以甘当达尔文的“斗犬”而自豪。这样的例子不胜枚举。可见,狗是一种可亲、可敬的动物。全祖望\CJKunderwave{鲒𩸞亭集外编}曰:“予观东汉之末,东南淑气萃于诸葛一门。观其兄弟分居三国,世莫有以为猜者,非大英雄不能。”故于时龙、虎、狗为三兄弟,当无鄙视之意。}

\lettrine{9.5} 司马文王\myidx{司马昭}问武陔\myidx{武陔}\footnote{司马文王:司马昭。武陔:字元夏。}:“陈玄伯\myidx{陈泰}何如其父司空\myidx{陈群}\footnote{陈玄伯:陈泰,字玄伯。其父司空:指陈泰之父陈群,三国魏司空,汉末颍川许昌(今属河南)人。曾任太丘长,故云。其治政清明,百姓安业,以公正直名闻世。时人评云:“宁为刑罚所加,不为陈君所短。”党锢祸起,自请系狱。卒时远近赴吊,刊石立碑,谥文范。荀朗陵:荀淑曾任郎陵侯相,故云。}?”陔曰:“通雅博畅\footnote{通雅博畅:通达雅正,渊博畅洽。谓学问造诣深。},能以天下声教为己任者\footnote{声教:声威教化。},不如也;明练简至\footnote{明练简至:精明干练、处事简要。谓练达世务。立功立事:古人有三不朽之说,语本\CJKunderwave{左传·襄公二十四年}:“太上有立德,其次有立功,其次有立言。”},立功立事,过之。”{\fzxk\zihao{6}\textcolor{red}{\CJKunderwave{魏志}曰:“陔与泰善,故文王问之。”}}

{\cangkai\zihao{5}【评】陈群端委廊庙,其首倡九品官人法、议复肉刑、谏迎宫室诸举措,事关国家政令方针导向,宜武陔有“通雅博畅”之评;陈泰一生事业,则主要在行伍中攻城略地、为王前驱,于理论思考虽或缺,然立功立事,亦足不朽。父子相较,一以弘通雅正胜,一以干练简洁显。其实,盖棺定论,父子之间难分高下。陈群于庙堂上倡高标、立世范,至如“以天下声教为己任者”,固然是出于以天下为己任的博大情怀;而陈泰于高贵乡公被弑后,声讨贾充,矛头暗指司马氏,因此呕血而亡,不正以宝贵的生命践行了儒家道义吗?\CJKunderwave{三国志}裴松之注以为,陈氏一门寔、纪到群、泰四世,“其德渐渐小减”,并引时人语曰:“公惭卿,卿惭长”,实在是拘泥于抽象道德标准的冬烘之谈,儒家人格当在历史的发展流动中显示其变动不居的态势,每一时代均有一时代立德、立功、立言的具体特色、内涵,不得谓古人人格一定胜于今人也。民国初年,鲁迅关于“中国人失掉了自信力了吗?”的探讨和胡适“新不朽观”的提法,均见出了新时代的思辨高度。胡适曾言:“我们谈到古人的人格,往往想到岳飞、文天祥和晚明那些死在廷杖之下或天牢里的东林忠臣。我们何尝不想想这二三十年中为了各种革命慷慨杀身的无数志士!……我们试想想那些为排满革命而死的许多志士,那些为民十五六年的国民革命而死的无数青年,那些前两年中在上海在长城一带为抗日卫国而死的无数青年,那些为民十三以来的共产革命而死的无数青年——他们慷慨献身去经营的目标比起东林诸君子的目标来,其伟大真不可比例了。”(\CJKunderwave{写在孔子诞辰纪念之后})因此,对于曾国藩和孙中山这两位不同时代的代表性人物,胡适认为,在古典文学的成就上,在事故的磨炼上,在小心谨慎的行为上,孙中山比不上曾国藩;然而在见解的大胆,气象的雄伟,行为的勇敢上,理学名臣就远不如革命领袖孙中山了。胡适的见解,因具有进化论的历史观作理论基础,不能不说是相当透辟而启迪至深。}

\lettrine{9.6} 正始中\footnote{正始:三国魏齐王芳年号(240—248)。},人士比论\footnote{比论:比较评论。},以五荀方五陈\footnote{方:比。}:荀淑\myidx{荀淑}方陈寔\myidx{陈寔}\footnote{荀淑(83—149):汉末颍川颍阴(今河南许昌)人。好学而不为章句,见讥于俗儒。以德行及识人闻世。治事明理,人称“神君”。后弃官归隐。时贤李固、李贤等拜他为师。锺皓:汉末颍川长社(今河南长葛)人。隐居密山,敦\CJKunderwave{诗}、\CJKunderwave{书}而悦礼乐,教授门徒千馀人。与同郡陈寔、荀淑、韩韶称颍川四长。后为郡功曹,旋自劾去。公府征辟不赴。陈寔:汉末颍川许昌(今属河南)人。曾任太丘长,故云。其治政清明,百姓安业,以公正直名闻世。时人评云:“宁为刑罚所加,不为陈君所短。”党锢祸起,自请系狱。卒时远近赴吊,刊石立碑,谥文范。荀朗陵:荀淑曾任郎陵侯相,故云。},荀靖\myidx{荀靖}方陈谌\myidx{陈谌}\footnote{荀靖: 字叔慈, 荀淑第三子。陈谌:寔少子。},{\fzxk\zihao{6}\textcolor{red}{\CJKunderwave{逸士传}曰:“靖字叔慈,颍川人。有俊才,以孝箸名。兄弟八人,号‘八龙’。隐身修学,动止合礼。弟爽,亦有才学,显名当世。或问汝南许章(劭):‘爽与靖孰贤?’章(劭)曰:‘二人皆王(玉)也。慈明外朗,叔慈内润。’太尉辟不就。年五十终,时人惜之,号玄行先生。”}} 荀爽\myidx{荀爽}方陈纪\myidx{陈纪}\footnote{荀爽:字慈明,荀淑第六子,当世硕儒。陈纪:汉末颍川许昌(今属河南)人。曾任太丘长,故云。其治政清明,百姓安业,以公正直名闻世。时人评云:“宁为刑罚所加,不为陈君所短。”党锢祸起,自请系狱。卒时远近赴吊,刊石立碑,谥文范。荀朗陵:荀淑曾任郎陵侯相,故云。},荀或(彧)\myidx{荀彧}方陈群\myidx{陈群}\footnote{荀或:即荀彧,汉末颍川许昌(今属河南)人。曾任太丘长,故云。其治政清明,百姓安业,以公正直名闻世。时人评云:“宁为刑罚所加,不为陈君所短。”党锢祸起,自请系狱。卒时远近赴吊,刊石立碑,谥文范。荀朗陵:荀淑曾任郎陵侯相,故云。原刻误作“或”。陈群:汉末颍川许昌(今属河南)人。曾任太丘长,故云。其治政清明,百姓安业,以公正直名闻世。时人评云:“宁为刑罚所加,不为陈君所短。”党锢祸起,自请系狱。卒时远近赴吊,刊石立碑,谥文范。荀朗陵:荀淑曾任郎陵侯相,故云。},{\fzxk\zihao{6}\textcolor{red}{\CJKunderwave{典略}曰:“或(彧)字文若,颍川人。为汉侍中、守尚书令。或(彧)为人英伟,折节待士,坐不累席。其在台阁间,不以私欲挠意。年五十薨,谥曰敬侯,以其名德高,追赠太尉。”}} 荀顗\myidx{荀顗}方陈泰\myidx{陈泰}\footnote{荀顗:字景倩,荀彧子。}。{\fzxk\zihao{6}\textcolor{red}{\CJKunderwave{晋诸公赞}曰:“顗字景倩,或(彧)之子。蹈礼立德,思义温雅,加深识国体,累迁光禄大夫。晋受禅,封临淮公。典朝仪,刊正国式,为一代之制。转太尉,为台辅,德望清重,留心礼教。卒溢康公。”}} 又以八裴方八王:裴徽\myidx{裴徽}方王祥\myidx{王祥}\footnote{王祥(184—268):魏晋时琅邪临沂(今属山东,琅邪,一作琅玡、琅琊)人。以至孝闻世,传统“二十四孝”有其“卧冰求鱼”故事及图画。汉末避乱庐江,后为徐州别驾。入魏官至司空,晋拜太保。祥及弟览,为琅邪王氏发达之始祖,后来如敦、导等皆其子孙。},裴楷\myidx{裴楷}方王夷甫\myidx{王衍}\footnote{裴楷:曾官中书令,故云,又称“裴令”。善\CJKunderwave{老}、\CJKunderwave{易},当时著名清谈名家。 王夷甫:王衍。},裴康\myidx{裴康}方王绥\myidx{王绥}\footnote{裴康:字仲豫,裴徽子。王绥:字万子,王戎之子。},{\fzxk\zihao{6}\textcolor{red}{\CJKunderwave{晋百官名}曰:“康字仲豫,徽之子。”\CJKunderwave{晋诸公赞}曰:“康有弘量,历太子左率。”}} 裴绰\myidx{裴绰}方王澄\myidx{王澄}\footnote{裴绰:字季舒,官至黄门侍郎。王澄:(267—312)。出自琅邪王氏。兄衍为西晋士林清谈领袖,誉澄“阿平第一”。有士人“经澄所题者,衍不复有言,辙云‘已经平子矣’”。澄由是显名于世。澄官荆州刺史,日夜纵酒,不以军政为意。曾残杀巴蜀流民,激起民变。后因故为王敦所杀。胡毋彦国:即胡毋辅之,晋清谈名士。史称有知人之鉴。性嗜酒,任纵不拘小节。与王澄、王敦、庾敳俱为太尉王衍亲昵,号称“四友”。永嘉乱后,南渡卒于湘江刺史任上。},{\fzxk\zihao{6}\textcolor{red}{\CJKunderwave{王朝目录}曰:“绰字仲(季)舒,楷弟也。名亚于楷。历中书、黄门侍郎。”}} 裴瓒\myidx{裴瓒}方王敦\myidx{王敦}\footnote{裴瓒:字国宝,裴楷次子。王敦:字处仲,晋琅邪临沂(今属山东)人,王导堂兄。妻为晋武帝女襄城公主,拜驸马都尉。晋室东迁,与王导一起辅佐元帝,任要职,握重兵,镇守扬州、荆州等重镇。公元322 年起兵谋反,入京都建康。},{\fzxk\zihao{6}\textcolor{red}{\CJKunderwave{晋诸公赞}曰:“瓒字国宝,楷之子。才气爽隽,终中书郎。”}} 裴遐\myidx{裴遐}方王导\myidx{王导}\footnote{裴遐:裴叔道。王导:王丞相。},裴頠\myidx{裴頠}方王戎\myidx{王戎}\footnote{裴顗:博学多才识,“时人谓为言谈之林薮”。撰\CJKunderwave{崇有论}以推尊儒术,崇扬礼法,贬斥何晏、王衍等言“无”之蔽。王戎:魏晋时琅邪人,王祥族人,当时清谈名士,“竹林七贤”之一。入晋官至尚书令、司徒。},裴邈\myidx{裴邈}方王玄\myidx{王玄}\footnote{裴邈:,字景声,西晋河东闻喜(今属山西)人。裴从弟。王玄:(?—313?),西晋琅邪临沂(今属山东)人,字眉子。王衍子。}。

{\cangkai\zihao{5}【评】故事以事数标榜和两两比照的方式,将汉魏之际四大家族的英杰之士对照比附一番,实是汉末清议风气的自然承续。之所以选择荀、陈与裴、王,因为他们是汉末魏晋之际的望族,在一定程度上显出了门第观念的变化。家族前几代均以儒行立身,尔后守成者渐入玄学一流。从家族成员的社会身份、伦理道德观念的代代流变,可以看出时代思潮的嬗变。这样的比附,折射出魏初士人尚以声气相标榜的历史印痕。模糊视之,或有一定道理,若读者深信不疑,一一求其因果,则未免拘泥株守,太过较真了。}

\lettrine{9.7} 冀州刺史杨淮(准)\myidx{杨准}二子乔\myidx{杨乔}与髦\myidx{杨髦}\footnote{冀州:州名。晋代治所在房子(今河北)。刺史:州长官,掌州军政大权。杨准:“淮”应作“准(准)”,据沈校本改。乔:杨乔,字国彦。髦:杨髦,字士彦。},俱总角为成器\footnote{总角:指童年。成器:犹言成材。}。淮与裴頠\myidx{裴頠}、乐广\myidx{乐广}友善\footnote{裴頠:博学多才识,“时人谓頠为言谈之林薮”。撰\CJKunderwave{崇有论}以推尊儒术,崇扬礼法,贬斥何晏、王衍等言“无”之蔽。乐广(?—304):字彦辅,南阳淯阳(今河南南阳东南)人。少孤贫,寒素为业,与物无竞。其清谈析理,与王衍并称,卫瓘以为有正始遗风。官至尚书令,八王乱中,以故忧卒。},遣见之。頠性弘方\footnote{弘方:旷达正直。},爱乔之有高韵\footnote{高韵:高雅的气质。},谓淮(准)曰:“乔当及卿,髦小减也。”广性清淳\footnote{清淳:高洁淳朴。},爱髦之有神检\footnote{神检:精神操守。},谓淮(准)曰:“乔自及卿,然髦尤精出\footnote{精出:出类拔众。}。”淮(准)笑曰:“我二儿之优劣,乃裴、乐之优劣。”论者评之,以为乔虽高韵,而检不匝\footnote{检:节操,操守。匝:完善,完满。};乐言为得。然并为后出之隽。{\fzxk\zihao{6}\textcolor{red}{荀绰\CJKunderwave{冀州记}曰:“乔字国彦,爽朗有远意;髦字士彦,清平有贵识。并为后出之隽,为裴頠、乐广所重。”\CJKunderwave{晋诸公赞}曰:“乔似淮而疏,皆为二千石,髦为石勒所害。”}}

{\cangkai\zihao{5}【评】故事以一次富有生活情趣的人物品藻过程,展示晋人多元而高雅的精神世界。杨准遣二子拜见裴頠、乐广,希望两位名人予以品评、识鉴,见出父亲望子成龙的美好期待。而名士们的鉴定结论大相径庭,令人莫衷一是。根据西方表现论,任何一种主体对客体的观照,都不可能是纯粹镜像式的反映(即再现),而不可避免地带有鉴赏者主体印记(即表现)。主观性较强的人物评价活动更是如此。中国亦有以己度人的传统与此印证。裴、乐虽优游儒道间,而裴推重儒家之雅正高迈,乐更尚玄之素朴自然。裴頠以己性情之旷达正直为基准,认定长子杨乔当胜出;乐广以己气质之清净淳朴为标竿,则以次子髦为优。不论以何种标准衡量,杨准都为自己后继有人而自豪。}

\lettrine{9.8} 刘令言\myidx{刘纳}始入洛\footnote{刘令言:刘纳,字令言,西晋彭城(今江苏徐州)人。刘隗伯父。},{\fzxk\zihao{6}\textcolor{red}{\CJKunderwave{刘氏谱}曰:“纳字令言,彭城藂亭人。祖瑾,乐安长。父甝,魏洛阳令。纳历司隶校尉。”}} 见诸名士而叹曰:“王夷甫\myidx{王衍}太鲜明\footnote{王夷甫:王衍。},乐彦辅\myidx{乐广}我所敬\footnote{乐彦辅:乐广,(267—312)。出自琅邪王氏。兄衍为西晋士林清谈领袖,誉澄“阿平第一”。有士人“经澄所题者,衍不复有言,辙云‘已经平子矣’”。澄由是显名于世。澄官荆州刺史,日夜纵酒,不以军政为意。曾残杀巴蜀流民,激起民变。后因故为王敦所杀。胡毋彦国:即胡毋辅之,晋清谈名士。史称有知人之鉴。性嗜酒,任纵不拘小节。与王澄、王敦、庾敳俱为太尉王衍亲昵,号称“四友”。永嘉乱后,南渡卒于湘江刺史任上。},张茂先\myidx{张华}我所不解\footnote{张茂先:张华。},周弘武\myidx{周恢}巧于用短\footnote{周弘武:周恢,字弘武,西晋汝南(在今河南)人。与石崇、潘岳等党附贾谧,为“二十四友”之一。},{\fzxk\zihao{6}\textcolor{red}{王隐\CJKunderwave{晋书}曰:“周恢字弘武,汝南人。祖斐,永宁少府。父隆,州从事。恢仕至秦相, 秩中二千石。”}} 杜方\myidx{杜育}叔拙于用长\footnote{杜方叔:杜育,字方叔,西晋襄城(在今河南)人。}。”{\fzxk\zihao{6}\textcolor{red}{\CJKunderwave{晋诸公赞}曰:“杜育字方叔,襄城邓(定)陵人,杜袭孙也。育幼便岐嶷,号‘神童’。及长,美风姿,有才藻,时人号曰杜圣。累迁国子祭酒。洛阳将没,为贼所杀。”}}

{\cangkai\zihao{5}【评】\CJKunderwave{世说}人物品藻,有一个较为显著的特点,就是喜说人长而少言人短。即使是批评否定也极尽含蓄、委婉之能事。这固然是秉承了中国传统文化中的君子的温柔敦厚之德,可后人重读这些故事,发现这些历史人物大多光鲜亮洁、超俗绝尘,好似庄子笔下邈姑射山的神仙,总觉得难脱溢美之嫌。本则却脱出窠臼,是一个特例。刘令言评王衍太鲜明,褒中有贬,似谓王过刻露,少蕴藉;对乐广毫不掩饰尊敬之情;张华气象万千,难以情测,故付之阙如;周恢巧于用短,是称其所长;杜育拙于用长,是讥其所短。刘评快人快语,爱恶分明,别有一番滋味。}

\lettrine{9.9} 王夷甫\myidx{王衍}云\footnote{王夷甫:王衍。}:“闾丘冲\myidx{闾丘冲}{\fzxk\zihao{6}\textcolor{red}{荀绰\CJKunderwave{兖州记}曰:“冲字宾卿,高平人。家世二千石。冲清平有鉴识,学有文义。累迁太傅长史,虽不能立功盖世,然闻义不惑,当世莅事,务于平允。操持文案,必引经诰,饰以文采,未尝有滞。性尤通达,不矜不假。好音乐,侍婢在侧,不释弦管。出入乘四望车,居之甚夷,不以亏损恭素之行,淡然肆其心志。论者不以为侈,不以为僭。至于白首,而清名令望不渝于始。为光禄勋,京邑未溃,乘车出,为贼所害,时人皆痛惜之。”}} 优于满奋\myidx{满奋}、郝(郗)隆\myidx{郗隆}\footnote{闾丘冲:字宾卿,西晋高平(今山东巨野南)人。满奋:晋高平人。郝隆:当为郗隆。郗隆,字弘始,西晋高平人。郗鉴叔父。}。{\fzxk\zihao{6}\textcolor{red}{\CJKunderwave{晋诸公赞}曰:“隆字弘始,高平人。为人通亮清识,为吏部郎、扬州刺史。齐王冏起义,隆应檄稽留,为参军王邃所杀。”}} 此三人并是高才,冲最先达\footnote{先达:优秀显达。}。”{\fzxk\zihao{6}\textcolor{red}{\CJKunderwave{兖州记}曰:“于时高平人士偶盛,满奋、郝隆达在冲前,名位已显,而刘宝、王夷甫犹以冲之虚贵足先二人。”}}

{\cangkai\zihao{5}【评】闾丘冲心态淡泊,性情通达,清平有鉴识。这样的人生态度合于魏晋玄学的精神实质,也与大名士王衍的标榜的人格追求相默契,故三人之中王衍以冲为最先达。}

\lettrine{9.10} 王夷甫\myidx{王衍}以王东海\myidx{王承}比乐令\myidx{乐广}\footnote{王夷甫:王衍。王东海:王承,曾任东海太守。乐令:乐广,(?—304):字彦辅,南阳淯阳(今河南南阳东南)人。少孤贫,寒素为业,与物无竞。其清谈析理,与王衍并称,卫瓘以为有正始遗风。官至尚书令,八王乱中,以故忧卒。},{\fzxk\zihao{6}\textcolor{red}{\CJKunderwave{江左名士传}曰:“承言理辩物,但明其旨要,不为辞费,有识伏其约而能通。太尉王夷甫一世龙门,见而雅重之,以此(比)南阳乐广。”}} 故王中郎\myidx{王坦之}作碑云\footnote{王中郎:王坦之,王承之孙。}:“当时标榜\footnote{标榜:称扬,品评。},为乐广之俪\footnote{俪:匹偶。}。”

{\cangkai\zihao{5}【评】\CJKunderwave{晋书}承本传曰:“言理辩物,但明其指要而不饰文辞,有识者服其约而能通。”可见王承玄辩风格以通脱简约为宗。王衍以比乐广,倒也恰如其分地抓住了二人的相似之处。乐广击几案谈“旨不至”,以顿悟的思维方式,将深繁的玄理化为耐人寻味的“行为艺术”,又恰似禅宗的“不立文字,直指本心”。刘勰\CJKunderwave{文心雕龙·诔碑}云:“标序盛德,必见清风之华;昭纪鸿懿,必见峻伟之烈。此碑之制也。”碑铭之作,当以昭示亡者之盛德鸿业为目的。王坦之为祖父王承撰写碑文,称“当时标榜,为乐广之俪”,引述此意,可见王衍评语具有盖棺定论的意义。}

\lettrine{9.11} 庾中郎\myidx{庾敳}与王平子\myidx{王澄}雁行\footnote{庾中郎:庾敳字子嵩,庾峻子。曾作司马太傅从事中郎。王平子:王澄字平子,王衍弟。雁行:鸿雁在天空列阵齐飞。比喻二人并列齐一,不分高下。}。{\fzxk\zihao{6}\textcolor{red}{\CJKunderwave{晋阳秋}曰:“初,王澄有通朗称,而轻薄无行。兄夷甫有盛名,时人许以人伦鉴识。常为天下士目曰:‘阿平第一,子嵩第二,处仲第三。’敳以澄、敦莫己若也。及澄丧敦败,敳世誉如初。”}}

{\cangkai\zihao{5}【评】王衍尝为天下目曰:“阿平第一,子嵩第二,处仲第三。”拉扯提携自家兄弟近乎肉麻。但是,这是一个张扬自我的时代,人家就这样说了,你又能怎样?王澄轻薄放荡,为人处世无足多取,与庾敳之颓然渊放、王敦之立事立功,均非一路。虽然王澄仅得魏晋玄学的皮毛,但有早已“得道”的大名士哥哥王衍为舆论领袖,在圈子里提携帮衬、推波助澜,因此还是得到时论的认可。本则“庾中郎与王平子雁行”即说明公众已经习惯于将王、庾并提了,可见名人效应之强大攻势及其影响力。}

\lettrine{9.12} 王大将军\myidx{王敦}在西朝时\footnote{王大将军:王敦官至大将军。字处仲,晋琅邪临沂(今属山东)人,王导堂兄。妻为晋武帝女襄城公主,拜驸马都尉。晋室东迁,与王导一起辅佐元帝,任要职,握重兵,镇守扬州、荆州等重镇。公元322 年起兵谋反,入京都建康。西朝:指西晋。西晋建都洛阳,渡江后东晋建都建康,自建康而言,洛阳在西,故称。},见周侯\myidx{周顗}辄扇障面\footnote{周侯:指周顗。},不得住\footnote{不得住:不能停。}。{\fzxk\zihao{6}\textcolor{red}{敦性强梁,自少及长,季伦斩妓,曾无异色。若斯傲很,岂惮于周顗乎。此言不然也。}} 后度江左\footnote{度:通“渡”。江左:江东。},不能复尔\footnote{尔:如此。}。三叹曰:“不知我进伯仁退\footnote{三叹曰:据文义及诸本,“三”当作“王”。伯仁:周顗,字伯仁。}。”{\fzxk\zihao{6}\textcolor{red}{沈约\CJKunderwave{晋书}曰:“周顗,王敦素惮之,见辄面热,虽复腊月,亦扇面不休。其惮如此。”}}

{\cangkai\zihao{5}【评】石崇宴请,以斩妓的方式要挟宾客饮酒,王敦虽预其中却不为所动。就是这样一位性情强梁傲狠的“百炼钢”,到了周顗手里,则便成了“绕指柔”。扇面之说绝非空穴来风。可见周顗之风采,必有使王敦慑服处。无独有偶,\CJKunderwave{晋书}本传亦载东南秀士戴若思见周顗,终坐而出,不敢显其才辩,与此有异曲同工之妙。盖人的心理结构中总会有薄弱之处,周顗性情中的某种特质恰是王敦所缺乏的。俗话说,“卤水点豆腐,一物降一物”,王敦之扇面不休乃是一种心理障碍,当遇到刺激就会以不自觉的心理应激形式呈现出来,昭示了其性格缺欠。在长期的生活实践中,某些心理障碍会因事业的成功、自信心的增强而自行克服。渡江后,王敦“不能复尔”,说明其在理性层面建立起了对周顗的自信,“不知我进伯仁退”,则最终露出“本我”,貌似自负,实则不经意间流露出其早年的心理印痕。}

\lettrine{9.13} 会稽虞𩦎\myidx{虞𩦎}\footnote{会稽:郡名。东晋时治所在山阴(今浙江绍兴)。虞𩦎:字思行,东晋会稽馀姚(今属浙江)人。时论谓有宰辅之望。},元皇\myidx{司马睿}时与桓宣武\myidx{桓温}(城)\myidx{桓彝}同侠\footnote{元皇时与桓宣武同侠:元皇,指晋元帝司马睿。桓宣武,桓温。程炎震疑“桓宣武”为“桓宣城”之讹,指温父桓彝,因彝曾与𩦎俱为吏部郎。疑是。同侠,疑“侠”为“僚”之讹。},其人有才理胜望。{\fzxk\zihao{6}\textcolor{red}{\CJKunderwave{虞光禄传}曰:“𩦎字思行,会稽馀姚人,虞翻曾孙,右光禄潭兄子也。虽机干不及潭,而至行过之。历吏部郎、吴兴守,征为金紫光禄大夫,卒。”}} 王丞相\myidx{王导}尝谓𩦎曰\footnote{王丞相:指王导。}:“孔愉\myidx{孔愉}有公才而无公望\footnote{孔愉:字敬康,晋会稽山阴人。公:指三公。晋以太尉、司徒、司空为三公。},丁潭\myidx{丁潭}有公望而无公才\footnote{丁潭:字世康,东晋山阴(今浙江绍兴)人。},{\fzxk\zihao{6}\textcolor{red}{愉已见。\CJKunderwave{会稽后贤记}曰:“潭字世康,山阴人,吴司徒固(曾)孙也。沈婉有雅望,少与孔愉齐名。仕至光禄大夫。”\CJKunderwave{晋阳秋}曰:“孔敬康、丁世康、张伟康俱箸名,时谓‘会稽三康’。伟康名茂,尝梦得大象,以问万雅,雅曰:‘居(君)当为大郡而不善也。象,大兽也,取其音狩,故为大郡;然象以齿丧身。’后为吴郡,果为沈充所杀。”}} 兼之者其在卿乎?”𩦎未达而丧\footnote{达:显贵。}。{\fzxk\zihao{6}\textcolor{red}{\CJKunderwave{虞光禄传}曰:“𩦎未澄(登)台鼎,时论称屈。”}}

{\cangkai\zihao{5}【评】三公是清要之职,晋朝时多由高门士族之德隆望尊者担任,至于才能如何则并不重视。王导寄虞𩦎以宰辅之望,“望”包括由门第先天带来的位望和道德品质等后天声望。集才华与人望于一身的虞𩦎乃吴国虞翻之曾孙,门第并非显赫,王导此处之“望”乃重点指品行之声望,流露出其选材标准兼顾德、才多方面因素。}

\lettrine{9.14} 明帝\myidx{司马绍}问周伯仁\myidx{周顗}\footnote{明帝:东晋明帝司马绍。程炎震疑明帝为元帝之讹,疑是。周伯仁:周顗。}:“卿自谓何如郗鉴\myidx{郗鉴}\footnote{谓:认为。何如:和……相比如何。郗鉴:郗司空。}?”周曰:“鉴方臣,如有功夫\footnote{方:比。功夫:功力,修养。}。”复问郗,郗曰:“周顗比臣,有国士门风\footnote{国士:一国之中的杰出人才。门风:家风。}。”{\fzxk\zihao{6}\textcolor{red}{邓粲\CJKunderwave{晋纪}曰:“伯仁清正嶷然,以德望称之。”}}

{\cangkai\zihao{5}【评】郗鉴、周顗俱是经邦济世的国之重臣,郗鉴是北来流民帅的领袖,周顗则是过江中原士族的佼佼者,二者政治利益原有不同。俗话说“一山难容二虎”,旗鼓相当的战友往往会因争宠、争胜,而演化成刀兵相见的敌手,古往今来同僚中因互相掣肘、拆台而造成严重内耗的恶性事件史不绝书。但故事中,二人却在明帝面前互相推尊,显示了名士的雍容风度。特别是郗鉴,目周顗有国士之风,非有虚怀若谷的气度不能发此激赏。为了国家民族的共同利益,还有什么分歧和矛盾是不好解决的?刘辰翁曰:“两语各可观。”正是此意。故事从一个侧面折射出周顗在士人心目中的地位,也同时展示了晋人空明、澄澈的胸怀。}

\lettrine{9.15} 王大将军\myidx{王敦}下\footnote{王大将军:王敦,字处仲,晋琅邪临沂(今属山东)人,王导堂兄。妻为晋武帝女襄城公主,拜驸马都尉。晋室东迁,与王导一起辅佐元帝,任要职,握重兵,镇守扬州、荆州等重镇。公元322 年起兵谋反,入京都建康。下:指王敦自武昌到东晋都城建康。武昌在上游,从上游到下游称下。},庾公\myidx{庾亮}问\footnote{庾公:庾亮。}:“闻卿有四友,何者是?”答曰:“君家中郎\myidx{庾敳}\footnote{君家中郎:指庾敳。君家:您家。}、我家太尉\myidx{王衍}、阿平\myidx{王澄}\footnote{我家太尉、阿平:指王衍、王澄,分别博学多才识,“时人谓为言谈之林薮”。撰\CJKunderwave{崇有论}以推尊儒术,崇扬礼法,贬斥何晏、王衍等言“无”之蔽、\CJKunderwave{德行}23注。}、胡毋彦国\myidx{胡毋辅之}\footnote{胡毋彦国:胡毋辅之,字彦国。}。{\fzxk\zihao{6}\textcolor{red}{\CJKunderwave{八王故事}曰:“胡毋辅之少有雅俗鉴识,与王澄、庾敳、王敦、王夷甫为四友。”今故答也。}} 阿平故当最劣\footnote{故当:当然是。}。”庾曰:“似未肯劣。”庾又问:“何者居其右\footnote{右:前,上。古代尚右,以右为上为尊。}?”王曰:“自有人。”又问:“何者是?”王曰:“噫!其自有公论\footnote{其:可能;或许。表示估计、推测而语气较委婉。}。”左右蹑公\footnote{蹑公,指踩庾亮的脚。},公乃止。{\fzxk\zihao{6}\textcolor{red}{敦自谓右者在己也。}}

{\cangkai\zihao{5}【评】故事以极富个性化和生活气息的人物对白,展示了一代枭雄王敦丰富的内心世界。庾亮问王敦,谁为“四友”中之优者,敦不作正面回答,而以“自有人”搪塞。不料庾亮不解其意,仍刨根问底。王敦欲说还休,一感叹词“噫”,意味深长地传达出不被理解的苦恼。“其自有公论”,实将难题又抛给了庾亮。结尾处,有“左右蹑公,公乃止”一细节,将庾亮智者千虑而难免一时木讷的情态淋漓尽致地描画出来。庾亮并未以第一人视敦,故仍穷追不舍。王敦不仅有武功,更是得陇望蜀,想做名士班头。但他也有矜持之心,内心真实羞于表露,希望庾亮猜测其意,等待士林的“公论”。故事以生动细节刻画出一代枭雄的内心世界,人物形象活灵活现,其指画确是神来之笔。}

\lettrine{9.16} 人问丞相\myidx{王导}\footnote{丞相:指王导。}:“周侯\myidx{周顗}何如和峤\myidx{和峤}\footnote{周侯:周顗。和峤:字长舆,汝南西平人。}?”答曰:“长舆嵯蘖\footnote{嵯蘖:山高峻峭拔。引申指人出众超群。}。”{\fzxk\zihao{6}\textcolor{red}{虞预\CJKunderwave{晋书}曰:“峤厚自封植,嶷然不群。”}}

{\cangkai\zihao{5}【评】人问王导,周顗与和峤相比如何,王导未作正面回答,而是避其词锋,让他自己去体会。“嵯蘖”,山势高峻峭拔貌,优点缺点尽在其中。古人有“仁者乐山”之喻,又有“高山仰止”之说,其实是“比德”观念的表现形式之一。前者取山之虚静仁厚的特点,后者含高耸难及之情。唐代柳宗元\CJKunderwave{始得西山宴游记}中以“特立”状西山,实则抒写己之人格。可见名士以山之某一特点摹写人之襟抱,成为传统。故事中,王导评和峤“嵯蘖”,正显出和峤之高自砥砺的气质特点。而周顗涵容大度的气魄,也因此烘托而出。这样的回答,符合丞相王导的工作方法和性格特点,正所谓“不著一字,尽得风流”。故刘辰翁有“得体”之评。}

\lettrine{9.17} 明帝\myidx{司马绍}问谢鲲\myidx{谢鲲}\footnote{明帝:晋明帝司马绍。谢鲲:字幼舆,陈郡阳夏人。}:“君自谓何如庾亮\myidx{庾亮}\footnote{庾亮:庾公。}?”答曰:“端委庙堂\footnote{端委庙堂:穿上朝服在朝廷执政。端委,端正宽舒的朝服。此用为动词,指穿上朝服。庙堂,朝廷。},使百僚准则\footnote{使百僚准则:使百官学习效仿。准则,典范、表率。这里用动词。},臣不如亮;一丘一壑\footnote{一丘一壑:丘壑为隐士栖隐之处,此谓放情山水,隐居不仕。},自谓过之。”{\fzxk\zihao{6}\textcolor{red}{\CJKunderwave{晋阳秋}曰:“鲲随王敦下,入朝,见太子于东宫,语及夕。太子从容问鲲曰:‘论者以君方庾亮,自谓孰愈?’对曰:‘宗庙之美,百官之富,臣不如亮;纵意丘壑,自谓过之。’”邓粲\CJKunderwave{晋纪}曰:“鲲与王澄之徒,慕竹林诸人,散首披发,裸祖(袒)箕踞,谓之‘八达’。故邻家之女,折其两齿,世为谣曰:‘任达不已,幼舆折齿。’鲲有胜情远概,为朝廷之望,故时以庾亮方焉。”}}

{\cangkai\zihao{5}【评】谢鲲此评,有矜持自高之意。庾亮以国舅之尊,把持朝政,实际上起到端委庙堂的作用。谢鲲更希慕风流,服膺玄道。晋人标榜得自然之旨,实则对自然的理解,歧义纷呈。“自然”,有大化流行的自然,有外化为山水的自然,有质性放任之自然,有落笔为诗文的风格自然等等。诸义分则得其一偏,合则获其全貌。谢氏子弟风神峻爽,风流标举,易于与秀丽的山水风月发生某种精神共鸣,比其他家族更宜于承当起开创山水诗章的角色。谢鲲自信在纵意丘壑方面优于他人,把山水当作自己纵情肆意之具,他好文学,有文集,却未流传下来。后其子小安丰谢尚有诗传世,庶几可以弥补这个遗憾。}

\lettrine{9.18} 王丞相\myidx{王导}二弟不过江\footnote{王丞相:指王导。不过江:谓永嘉之乱时留在中原,没有渡江南下。},曰颖\myidx{王颖}、曰敞\myidx{王敞}\footnote{曰颖曰敞:一个叫王颖,一个叫王敞。}。时论以颖比邓伯道\myidx{邓攸}\footnote{邓伯道:邓攸,字伯道。},敞比温仲武\myidx{温峤}\footnote{温仲武:据诸本,“仲”为“忠”之讹。温忠武,指温峤,卒谥忠武,当时追从姨夫刘琨,在并州为谋主,“琨所凭恃焉”(\CJKunderwave{晋书·温峤传})。建武元年(317)奉刘琨命出使江南,拥戴司马睿即帝位,建立东晋王朝。受司马睿重用,留为散骑常侍,后官至中书令,为东晋名臣。},议郎、祭酒者也\footnote{议郎、祭酒者也:颖位至议郎,敞至丞相祭酒。}。{\fzxk\zihao{6}\textcolor{red}{\CJKunderwave{王氏谱}曰:“颖字茂英,位至议郎,年二十卒。敞字茂平,丞相祭酒,不就,袭爵堂邑公,年二十有二而卒。”}}

{\cangkai\zihao{5}【评】\CJKunderwave{晋书}王导传所载与此相反,以颖比温峤,敞比邓攸。颖、敞分别于二十、二十二岁而卒,恐风流不显,功业未建,有何德何能比于二贤?当是时人以王导执政居高位,风闻揣度,为之虚声造势邪?}

\lettrine{9.19} 明帝\myidx{司马绍}问周侯\myidx{周顗}\footnote{明帝:晋明帝司马绍。据史,疑为元帝之讹。周侯:周顗。}:“论者以卿比郗鉴\myidx{郗鉴}\footnote{郗鉴:郗司空。},云何?”周曰:“陛下不须牵顗比\footnote{不须:不应该。牵:引,牵拉。}。”{\fzxk\zihao{6}\textcolor{red}{案:顗死弥年,明帝乃即位。\CJKunderwave{世说}此言妄矣。}}

{\cangkai\zihao{5}【评】故事当发生在明帝为太子时,“明帝”云者,后人追称。后面第22则亦当作如是观。此则与本门第十四则,盖为一事而记载不同。周顗回答,颇费人思索。“不须牵顗比”,是赞扬语——郗鉴是何等人物,根本用不着把我拉进来就能自显其峥嵘;抑或是不屑语——你郗鉴是什么货色,有何资格跟我相比?细忖度之,当为前者。以周顗之雍容气度,当乐于成人之美、逢人说项,必不为仗气使性、苛刻细屑之言。}

\lettrine{9.20} 王丞相\myidx{王导}云\footnote{王丞相:王导。}:“顷下论,以我比安期\myidx{王承}、千里\myidx{阮瞻}\footnote{顷下:时下,近来。安期:王承,字安期。千里:阮瞻,字千里。},亦推此二人\footnote{亦推此二人:此句似有脱漏。\CJKunderwave{太平御览}卷四四七引\CJKunderwave{郭子},作“我亦不推此二人”。};唯共推太尉\myidx{王衍}\footnote{太尉:王衍,博学多才识,“时人谓顗为言谈之林薮”。撰\CJKunderwave{崇有论}以推尊儒术,崇扬礼法,贬斥何晏、王衍等言“无”之蔽。},此君特秀\footnote{秀:出类拔萃。}。”{\fzxk\zihao{6}\textcolor{red}{\CJKunderwave{晋诸公赞}曰:“夷甫性矜峻,少为同志所推。”}}

{\cangkai\zihao{5}【评】王导提议时论推尊王衍,不是一时的情急之言,而是出于多重的考虑。首先,导位极人臣,是东晋公认的政治领袖,现在士林间又似有推其为精神领袖的舆论,这不能不引起其警觉。“木秀于林,风必摧之”,睿智的领袖人物决不会把一切荣誉包揽在自己身上。他一定通晓“分誉”之理,说到底,“分誉”就是“分谤”。其次,从为家族营造“三窟”的长远利益看,也必须将一部分影响力转移到王衍身上。王导、王敦、王衍,政治、军事、文化,可以说基本上将国家的上层建筑全部承包了。这种说法并非凭空设想,王衍本人就曾有为琅邪王氏家族谋三窟设计。再次,不管王衍是真名士、假名士,至少从外在的风度、言论看,曾是一代士人班头。王导也不得不对其礼让几分。}

\lettrine{9.21} 宋袆\myidx{宋袆}曾为王大将军\myidx{王敦}妾\footnote{宋袆:晋艺妓。美容貌,原位石崇婢绿珠弟子,善吹笛。先后属晋明帝、阮孚、王敦、谢尚等。王大将军:王敦,字处仲,晋琅邪临沂(今属山东)人,王导堂兄。妻为晋武帝女襄城公主,拜驸马都尉。晋室东迁,与王导一起辅佐元帝,任要职,握重兵,镇守扬州、荆州等重镇。公元322 年起兵谋反,入京都建康。},后属谢镇西\myidx{谢尚}\footnote{谢镇西:谢尚,曾任镇西将军。}。镇西问袆:“我何如王?”答曰:“王比使君\footnote{使君:对刺史的尊称。谢尚曾任江州刺史,故称。},田舍、贵人耳\footnote{田舍、贵人:乡下人与富豪。}。”镇西妖冶故也\footnote{妖冶:艳丽。}。{\fzxk\zihao{6}\textcolor{red}{未详宋袆。}}

{\cangkai\zihao{5}【评】谢尚是一个潇洒不羁、风神楚楚的名士,时人评其风格是“清易令达”、“率易挺达”,总之离不开一个“达”字,似承袭了乃父的任达之风。他常穿一条绣有花纹的套裤,算是奇装异服吧,用今天的话说是有点儿“酷”;又精通多种乐器,善跳八哥舞,凡是公子哥们的高雅玩意儿无不在行,是一个风流绰约、多才多情、风度翩翩、招人爱怜的贵公子艺术家。在一个年轻艺妓眼里,王敦无论相貌和风情,都不能望谢尚项背。故宋袆以田舍翁和贵人相类比。虽不能完全排除宋袆有某种取悦新主的心理成分在,但衡以事实,当大致不差。}

\lettrine{9.22} 明帝\myidx{司马绍}问周伯仁\myidx{周顗}\footnote{明帝:晋明帝司马绍。周伯仁:周顗。}:“卿自谓何如庾元规\myidx{庾亮}\footnote{庾元规:庾亮。}?”对曰:“萧条方外\footnote{萧条方外:指退隐山林,过隐居生活。萧条,闲逸。方外,世俗之外。},亮不如臣;从容廊庙\footnote{从容廊庙:在朝廷从政。从容:安处,优游。廊庙:殿下屋和太庙,古代君臣议政之处。借指朝廷。},臣不如亮。”{\fzxk\zihao{6}\textcolor{red}{案:诸书皆以谢鲲比亮,不闻周顗。}}

{\cangkai\zihao{5}【评】余嘉锡、朱铸禹等前贤俱以此与明帝问谢鲲语同,周顗非萧条方外者,当是传闻之误也。余意或乃周顗自以为身在庙堂、心期江海,比庾亮多些超脱,亦未可知。}

\lettrine{9.23} 王丞相\myidx{王导}辟王蓝田\myidx{王述}为掾\footnote{王丞相:王导。辟:征召,招聘。王蓝田:王述。掾:属官。},庾公\myidx{庾亮}问丞相\footnote{庾公:庾亮。}:“蓝田何似\footnote{何似:怎么样。}?”王曰:“真独简贵\footnote{真独简贵:率真孤傲,简约高贵。},不减父祖\footnote{减:比……差。};旷然澹处,故当不如尔\footnote{旷然澹处:心胸开朗,淡泊名利。故当:或许,可能。}。”{\fzxk\zihao{6}\textcolor{red}{王述狷隘故也。}}

{\cangkai\zihao{5}【评】作为属下,王述曾在公众面前使王导难堪,但王导仍能以宽广的胸怀包容其率真的性格,客观全面地做出评价,实属难得。王述是未经雕琢的璞玉,不谙事故的赤子,优点缺点都一览无馀地展现在世人面前。这样的人格,因符合庄子所憧憬的“真人”理想而为晋人喜爱。“旷然澹处故当不如”,当指王述胸襟气度及对待名利方面有些微的瑕疵,不过也正因此而展现其真挚的个性。正如简文评述云:“才既不长,于荣利又不淡,直以真率少许,便足对人多多许。”}

\lettrine{9.24} 卞望之\myidx{卞壸}云\footnote{卞望之:卞壸字望之。}:“郗公\myidx{郗鉴}体中有三反\footnote{郗公:郗鉴。体中:犹言胸中、心中。三反:三件矛盾的事。}:方于事上\footnote{方:端方正直。事上:侍奉上级。},好下佞己\footnote{佞:谄媚。},一反;治身清贞\footnote{治身:犹修身。清贞:犹廉洁清正。},大修计校\footnote{修:讲求,讲究。计校:算计。指为个人得失考虑。},二反;自好读书,憎人学问,三反。”{\fzxk\zihao{6}\textcolor{red}{案:太尉刘宝(寔)论王肃方于事上,好下佞己;性嗜荣贵,不求苟合;治身不秽,尤惜财物。王、郗志性傥亦同乎?}}

{\cangkai\zihao{5}【评】\CJKunderwave{世说}所记人物大多光鲜亮洁、人雅事雅,整天谈玄论道,一个“俗”字很难与他们挨边,但事实上,纯真无瑕的人是不存在的。故事的可贵之处,就在于刻画了一位有血有肉又分担了人类弱点的士族精英。方于事上、治身清贞与好读书,都是中国知识分子的传统美德;好下佞己、大修计校与憎人学问等,则是人类通病。善与恶,崇高与卑劣,天使与魔鬼,如影随形地隐藏在每个人的心中,接受教育、提升修养则是一个逐渐扶阳祛蔽的过程。正是在这个意义上,刘辰翁评以简单的“人人同”三个字,当是对人性深刻的洞察。英国保罗·约翰逊\CJKunderwave{知识分子}一书,选取了西方思想家和作家十馀人,如卢梭、雪莱、易卜生、托尔斯泰等,通过对这些人私生活种种可耻、可恶、可笑、可悲的方面的揭示,抖落掉他们头上的光环,深化了我们对知识分子甚至是圣贤的思考。}

\lettrine{9.25} 世论温太真\myidx{温峤},是过江第二流之高者\footnote{温太真:温峤,当时追从姨夫刘琨,在并州为谋主,“琨所凭恃焉”(\CJKunderwave{晋书·温峤传})。建武元年(317)奉刘琨命出使江南,拥戴司马睿即帝位,建立东晋王朝。受司马睿重用,留为散骑常侍,后官至中书令,为东晋名臣。第二流:第二等。多指人的门第、品德或才能、声望。}。时名辈共说人物,第一将尽之间,温常失色。{\fzxk\zihao{6}\textcolor{red}{\CJKunderwave{温氏谱序}曰:“晋大夫郗(郤)志封于温,子孙因氏,居太原祁县,为郡箸姓。”}}

{\cangkai\zihao{5}【评】故事逗露了中国士大夫的好名心理。正如道教追求长生不老和佛教追求死后的极乐世界,中国士人更重视死后的青史留名,不求肉体或灵魂转世,而是以留名的方式,与历史同在。追求不朽——正是中国人独特的“宗教”。俗语说“人过留名,雁过留声”,也正是这个意思。追求第一流就是要做最好,追求一流,可以说是人类进步的原动力,是人类的普遍心理。晋室播迁之际,温峤作为一个想要在动荡时代有所作为的弄潮儿,关注社会声誉、积极建功立业,高出那些无所作为、消极等待、空谈玄理的士人远甚。刘琨、祖逖、温峤等一大批能文能武的士人,为这个尚柔守文的时代注入了一丝清刚之气。故事的另一价值是抓住了人物表情瞬间动态的细节描写,写出了温峤紧张的心理活动,堪称神来之笔。在门阀社会中,“一流”、“二流”之评,如今之等级职称,给人以无形的压力。英雄如温峤而不免俗态,悲乎!}

\lettrine{9.26} 王丞相\myidx{王导}云\footnote{王丞相:王导。}:“见谢仁祖\myidx{谢尚}\footnote{谢仁祖:谢尚。},恒令人得上\footnote{得上:意为精神振奋,使人向上。}。”与何次道\myidx{何充}语\footnote{何次道:何充,晋康帝时为骠骑将军。},唯举手指地曰:“正自尔馨\footnote{正自:正是。尔馨:这样。可能王导欣赏何充为政之才干,而贬抑其清谈析理之平庸。}。”{\fzxk\zihao{6}\textcolor{red}{前篇及诸书皆云王公重何充,谓必代己相。而此章以手指地,意如轻诋。或清言析理,何不逮谢故邪?}}

{\cangkai\zihao{5}【评】王世懋曰:“此方言,意云:也只如此,故非誉之也。”\CJKunderwave{赏誉}门几篇都说王导看重何充,有选定丞相接班人之意,而此则以手指地,有轻视意。前贤以为在清谈析理方面,何不及谢。王导在此,非谈政治,专指精神文化而言,故扬谢抑何,也在情理之中。}

\lettrine{9.27} 何次道\myidx{何充}为宰相\footnote{何次道:何充,晋康帝时为骠骑将军。},人有讥其信任不得其人\footnote{不得其人:谓没有用上恰当的人。}。{\fzxk\zihao{6}\textcolor{red}{\CJKunderwave{晋阳秋}曰:“充所昵庸杂,以此损名。”}} 阮思旷\myidx{阮裕}慨然曰\footnote{阮思旷:阮裕。}:“次道自不至此。但布衣超居宰相之位\footnote{布衣:指代庶民百姓或未仕宦者。超:越,超越。},可恨唯此一条而已。”{\fzxk\zihao{6}\textcolor{red}{\CJKunderwave{语林}曰:“阮光禄闻何次道为宰相,叹曰:‘我当何处生活?’此则阮未许何为鼎辅。二说便相符也。”}}

{\cangkai\zihao{5}【评】庾冰、庾翼相继而逝,何充官侍中、录尚书事,辅佐晋穆帝,为一朝宰相。\CJKunderwave{晋书}何充传载“(充居宰相)以社稷为己任,凡所选用,皆以功臣为先,不以私恩树亲戚,谈者以此重之。然所昵庸杂,信任不得其人”。何充获讥于世,并非无根之谈,有史明文为证。但从另外一个角度看,一个刚正不徇私情的宰相,不可避免地会触及某些人的既得利益,从而引起众口哓哓的谣诼和谩骂;另外,为政者的自身弱点,也会成为以私德之亏掩其公德大节的“罪证”。古往今来,此风不熄。隐士阮裕隔岸观火,一针见血地指出“布衣超居宰相之位”,为何充唯一的遗憾。何充并非庶民出身,此言其升迁太快,羽翼未丰,难以高翔,政治手腕和影响力还不足以强大到抗衡士林舆论的程度。}

\lettrine{9.28} 王右军\myidx{王羲之}少时\footnote{王右军:王羲之。},丞相\myidx{王导}云\footnote{丞相:王导。}:“逸少何缘复减万安\myidx{刘绥}邪\footnote{逸少:王羲之,字逸少。何缘:哪里,岂。复减:不如。万安:刘绥,字万安。}!”{\fzxk\zihao{6}\textcolor{red}{刘绥,已见。}}

{\cangkai\zihao{5}【评】王导评语,为王羲之抱不平。刘绥、庾琮有“灼然玉举”之誉,又有“千人亦见,百人亦见”之评。大概王羲之少时,舆论以刘绥高于王羲之,站在大家长的角度看,王导对羲之寄予了无限厚望,虽为丞相,也有家族私情,对侄子的任何不利之辞都分外警觉,从而流露出不满之情。}

\lettrine{9.29} 郗司空\myidx{郗鉴}家有伧奴\footnote{郗司空:郗鉴。伧奴:原籍北方的奴仆。伧,六朝时南方人对北方人或南渡北人的鄙称。},知及文章,事事有意\footnote{事事:处处。有意:有情致,有意味。}。王右军\myidx{王羲之}向刘尹\myidx{刘惔}称之\footnote{王右军:王羲之。刘尹:刘惔。},刘问:“何如方回\myidx{郗愔}\footnote{方回:郗愔字方回,郗鉴长子,郗超父。}?”{\fzxk\zihao{6}\textcolor{red}{\CJKunderwave{郗愔别传}曰:“愔字方回,高平金乡人,太宰鉴长子也。渊端(靖)纯素,无执无竞,简昵交游。历会稽内史、侍中、司徒。”}} 王曰:“此正小人有意向耳\footnote{正:只,仅。 小人:指仆役。 意向:心思,志向。},何得便比方回?”刘曰:“若不如方回,故是常奴耳\footnote{故:仍然。}。”

{\cangkai\zihao{5}【评】故事可见刘惔性情中刁钻、刻薄的一面,其行为非仅高门士族之矜持所能解释。首先,在门阀社会中,以奴比主,已大不恭敬,意谓郗愔亦仅奴仆中之非常者,此奴尚不如郗愔,则只不过寻常之奴耳;其次,与王羲之善于发现、称道奴仆一技之长的积极态度相反,刘惔持非常苛厉的评判标准,“入眼平生未曾有”,眼光高得恐怕连王羲之都会为之咋舌。以这样的胸襟待人待事,只能熄灭而不是点燃创造性的火花。凌濛初评曰:“辄问方回,薄态可拘。”王世懋亦云:“刘尹大是轻薄人。”可见刘惔已经惹得古人今人的一致反感了。}

\lettrine{9.30}时人道阮思旷\myidx{阮裕}\footnote{道:评论,评说。阮思旷:阮裕。},骨气不及右军\myidx{王羲之}\footnote{骨气:风骨气度。右军:王羲之。},简秀不如真长\myidx{刘惔}\footnote{简秀:简约俊秀。真长:刘惔。},韶润不如仲祖\myidx{王濛}\footnote{韶润:美好温润。仲祖:王濛。},思致不如渊源\myidx{殷浩}\footnote{思致:思想意趣。渊源:殷浩。},而兼有诸人之美。{\fzxk\zihao{6}\textcolor{red}{\CJKunderwave{中兴书}曰:“裕以人不须广学,正应以礼让为先。故终日颓然无所修综,而物自宗之。”}}

{\cangkai\zihao{5}【评】\CJKunderwave{礼记·学记}曰:“五声弗得不和”、“五色弗得不章”、“五官弗得不治”,故五色相合而成文,五音克谐而成咏,五味调和而成馔;一色、一音、一味,美则美矣,却失之单调,少有回味。右军、真长、渊源,皆一时翘楚,各有胜处;阮思旷虽不如各家之长,而能兼擅众美,“豪华落尽见真淳”,最终达致涵容淡泊一路,若非有陶冶锤炼之功,不能臻此妙境。刘辰翁曰:“如此更高”,推赏此渊博之美,更在诸名士上。}

\lettrine{9.31} 简文\myidx{司马昱}云\footnote{简文:晋简文帝司马昱。}:“何平叔\myidx{何晏}巧累于理\footnote{何平叔:何晏。巧累于理:谓何晏处世取巧,殊损于其所谈论的玄理。累,牵累。},嵇叔夜\myidx{嵇康}隽伤其道\footnote{嵇叔夜:嵇康,魏晋时琅邪人,王祥族人,当时清谈名士,“竹林七贤”之一。入晋官至尚书令、司徒。隽伤其道:指嵇康俊逸不群,伤害了其所持的“越名教而任自然”之道。}。”{\fzxk\zihao{6}\textcolor{red}{理本真率,巧则乖其致;道唯虚澹,隽则违其宗。所以二子不免也。}}

{\cangkai\zihao{5}【评】何晏持“贵无论”,认为宇宙以无为本,与王弼、夏侯玄倡导玄学,开魏晋风气;然党附曹爽,以党争故,为司马懿所杀。嵇康为继何、王之后的清谈领袖,标举“越名教而任自然”,后因多次触忤小人,为当权者所忌,横尸司马氏的屠刀之下。二人都没有将清言玄理化作生活中的保身之道,性情之机巧与才气之隽逸,是分别导致其个人命运悲剧的直接原因。何晏以无为本,却偏偏卷进政治斗争的漩涡;嵇康倡导自然,可就是难以做到心平气和。他们虽都死于非命,但何晏死于自作聪明;嵇康则是以崇高对抗邪恶,“宁为玉碎,不为瓦全”,性质并不相同。}

\lettrine{9.32} 时人共论晋武帝\myidx{司马炎}出齐王\myidx{司马攸}之与立惠帝\myidx{司马衷}\footnote{晋武帝:司马炎,司马昭长子,西晋开国君主。齐王:司马攸(248—283),字大猷,司马昭次子,武帝司马炎胞弟。惠帝:晋惠帝司马衷(259—306),字正度,司马炎子。},其失孰多?{\fzxk\zihao{6}\textcolor{red}{\CJKunderwave{晋阳秋}曰:“齐王攸,字大猷,文帝弟(第)二子。孝敬忠肃,清和平允,亲贤下士,仁惠好施。能属文,善尺牍。初,荀勖、冯紞为武帝亲幸,攸恶勖之佞。勖惧攸或嗣立,必诛己,且攸甚得众心,朝贤景附。会帝有疾,攸及皇太子入问讯,朝士皆属目于攸,而不在太子。至是,勖从容曰:‘陛下万年后,太子不得立也。’帝曰:‘何故?’勖曰:‘百寮内外,皆归心于齐王,太子安得立乎?陛卜(下)试诏齐王归国,必举朝谓之不可。若然,则臣言征矣。’侍中冯紞又曰:‘陛下必欲建诸侯,成五等,宜从亲始。亲莫若齐王。’帝从之。于是下诏,使攸之国。攸闻勖、紞间己,忧忿不知所为。入辞出,欧(呕)血薨。帝哭之恸,冯紞侍曰:‘齐王名过其实,而天下归之。今自薨殒,陛下何哀之甚!’帝乃止。刘毅闻之,故终身称疾焉。”}} 多谓立惠帝为重。桓温\myidx{桓温}曰\footnote{桓温:桓温曾有三次北征,刘盼遂\CJKunderwave{世说新语校笺}考订,此次当为太和四年(369)之征。时桓温已58岁。}:“不然,使子继父业\footnote{子继父业:此处指司马攸以嗣子身份继承父亲司马师的事业。},弟承家祀\footnote{弟承家祀:此处指司马攸以弟弟的身份承续家族的香火。},有何不可?”{\fzxk\zihao{6}\textcolor{red}{武帝兆祸乱,覆神州,在斯而已。舆隶且知其若此,况宣武之弘隽乎!此言非也。}}

{\cangkai\zihao{5}【评】古代君主继承有“立嫡以长不以贤”的礼制,其着眼点在于避免任何可引起争端的因素存在,维护政权的超稳定结构。关于惠帝司马衷的立嗣问题,则又与此有异。\CJKunderwave{晋书·惠帝本纪}载:“帝又尝在华林园,闻蛤蟆声,谓左右曰:‘此鸣者为官乎,私乎?’”又“及天下荒乱,百姓饿死,帝曰:‘何不食肉糜?’”可见衷是不折不扣的弱智。而太子母杨皇后及荀勖等一班弄臣,以古训阻止废太子立齐王,当然只是一个堂皇的借口,其真实想法完全出于一己身家私利,怕齐王称帝后于己不利,司马氏的国运则被完全抛之脑后。尔后,惠帝在位,愚昧昏暗,政事出于后党贾氏,最终酿成“八王之乱”,西晋政权遂滑入每况愈下的轨道,也就毫不足奇了。桓温提出傻瓜儿子可以“子继父业”,其着眼点表面上是维持晋室帝祚,而其内里真实心思,在于立君以贤,则政治清明而皇权巩固;而若立昏君,则政治混乱,致使奸雄有浑水摸鱼的机会。桓温一代枭雄,其觊觎非常的勃勃野心,在此自然流露了出来。}

\lettrine{9.33} 人问殷渊源\myidx{殷浩}\footnote{殷渊源:殷浩。}:“当世王公,以卿比裴叔道\myidx{裴遐}\footnote{裴叔道,裴遐。},云何?”殷曰:“故当以识通暗处\footnote{故当:自然,当然。识:才识。暗处:玄理中的隐晦精微之处。裴遐、殷浩,并是玄理高手,故浩有此言。}。”{\fzxk\zihao{6}\textcolor{red}{遐与浩并能清言。}}

{\cangkai\zihao{5}【评】一说,“识”为才识,“通”为通晓,“暗处”指精微隐讳之玄理。裴遐、殷浩,并能清言,且是玄理高手,故浩有此自诩之语;一说,“识通”是内典之八识六通,浩以自喻。“暗处”是谓内蕴不明显处,指裴遐。其意似谓当世不知我内蕴聪明智慧,而以裴相比拟。故刘辰翁曰:“似谓裴暗。”“识通暗处”分指殷、裴,语意晦涩难通。二说相较,以前者为佳。}

\lettrine{9.34} 抚军\myidx{司马昱}问殷浩\myidx{殷浩}\footnote{抚军:简文帝司马昱,曾任抚军大将军。殷浩:殷渊源。}:“卿定何如裴逸民\myidx{裴頠}\footnote{定:究竟。 裴逸民:裴頠,字逸民,博学多才识,“时人谓頠为言谈之林薮”。撰\CJKunderwave{崇有论}以推尊儒术,崇扬礼法,贬斥何晏、王衍等言“无”之蔽。}?”良久答曰:“故当胜耳\footnote{故当:可能,或许。}。”

{\cangkai\zihao{5}【评】殷浩善玄言,为风流谈论者所宗,裴頠亦折衷儒玄,发“崇有”之论,故有“言谈之林薮”的雅号,二人各有胜处。若相较水平高下,因缺少具体量化的参照指标很难做出权衡评判。“良久”一词,见出殷浩激烈的内心冲突,若奉行谦谦君子之风,以自愧弗如作答,则有违本意。“故当胜耳”一语,既有以己之“崇无”胜裴之“崇有”,涉及理论旨趣之异;同时又于自负中透出自信,有晋人个性张扬的时代特色。}

\lettrine{9.35} 桓公\myidx{桓温}少与殷侯\myidx{殷浩}齐名\footnote{桓公: 桓温。殷侯: 殷浩。},常有竞心\footnote{竞心: 争胜之心。}。桓问殷:“卿何如我?”殷云:“我与我周旋久\footnote{我与我周旋久:诸本同,但\CJKunderwave{晋书}浩传作“我与君周旋久”,可备一说。周旋:交往、应酬。},宁作我\footnote{宁:宁可,宁愿。}。”

{\cangkai\zihao{5}【评】晋人向外发现了自然,向内发现了自我。他们发现了自然属人的深情,也从过去被儒家礼教压抑、束缚下找回了久违的真我。他们嬉笑怒骂,皆成妙谛,展示了一个个丰富多彩的个性世界;涂抹勾画,都具真我,开启了文学、艺术的自觉时代。即如殷浩,面对桓温提出的难题,不卑不亢,出之以“我与我周旋久,宁作我”,表现了高度的玄言智慧和追求独特自我的时代风尚。其深层意思还是以为己胜。王世懋评曰:“妙于自夸”,从语言的机智性角度着眼,有一定道理。}

\lettrine{9.36} 抚军\myidx{司马昱}问孙兴公\myidx{孙绰}\footnote{抚军:晋简文帝司马昱,曾作抚军将军。孙兴公:孙绰字兴公。}:“刘真长\myidx{刘惔}何如\footnote{刘真长:刘惔。}?”曰:“清蔚简令\footnote{清蔚简令:清淳有文采,简约美好。}。”“王仲祖\myidx{王濛}何如\footnote{王仲祖:王濛字仲祖。}?”曰:“温润恬和\footnote{温润恬和:温和柔顺,恬静平和。}。”{\fzxk\zihao{6}\textcolor{red}{徐广\CJKunderwave{晋纪}曰:“凡称风流者,皆举王、刘为宗焉。”}} “桓温\myidx{桓温}何如\footnote{桓温:桓温曾有三次北征,刘盼遂\CJKunderwave{世说新语校笺}考订,此次当为太和四年(369)之征。时桓温已58岁。}?”曰:“高爽迈出\footnote{高爽迈出:高爽豪迈,超群出众。}。”“谢仁祖\myidx{谢尚}何如\footnote{谢仁祖:谢尚。}?”曰:“清易令达\footnote{清易令达:清明平易,美好通达。}。”“阮思旷\myidx{阮裕}何如\footnote{阮思旷:阮裕。}?”曰:“弘润通长\footnote{弘润通长:宽广平和,淹通渊博。}。”“袁羊\myidx{袁乔}何如\footnote{袁羊:袁乔。}?”曰:“洮洮清便\footnote{洮洮清便:滔滔畅达,清雅简易。}。”“殷洪远\myidx{殷融}何如\footnote{殷洪远:殷融,字洪远。殷浩叔。}?”曰:“远有致思\footnote{远有致思:旷远深邃,颇有情趣。}。”“卿自谓何如?”曰:“下官才能所经,悉不如诸贤;至于斟酌时宜\footnote{斟酌:考虑衡量。 时宜:时势所宜,指当世政务。},笼罩当世\footnote{笼罩:洞察把握。},亦多所不及。然以不才\footnote{不才:自谦的说法。},时复托怀玄胜\footnote{托怀:寄托情怀。玄胜:指玄理。},远咏老、庄\footnote{远:高远。老、庄:\CJKunderwave{老子}、\CJKunderwave{庄子}。},萧条高寄\footnote{萧条:闲逸超脱。高寄:寄托高远,超脱世俗。},不与时务经怀,自谓此心无所与让也\footnote{让:谦让。}。”

{\cangkai\zihao{5}【评】孙绰早年有肥遁之志,居于会稽,游放山水,作\CJKunderwave{遂初赋}以致其意。王羲之“兰亭之游”,孙绰亦预其事。故事中绰自评“托怀玄胜,远咏老庄,萧条高寄,不与时务经怀”,即是对这一段经历的诗意概括,并自诩为高出众人的特出之处。后来孙绰并没有固守住“遂初”之意,出山做了大大小小不少的官职。桓温将移都洛阳,朝廷无一人敢谏,独孙绰上表陈情反对。\CJKunderwave{晋书}本传史臣评绰“献直论辞,都不慴元子,有匪躬之节,岂徒文雅而已哉”!可见孙绰绝非逍遥方外的隐士,关键时刻,还是有无法割舍的火热情怀。晋人标榜清高,以脱俗自许,实际上俗网难脱。滚滚红尘,怎能说忘就忘?真正超越名利“绛云在霄,舒卷自如”似陶渊明者,曲高和寡。晋人往往是身在庙堂,心存江海;或人在天涯,而情系魏阙。庙堂之志与山川之情,是晋人心灵深处永远无法释怀的“累”!一方面传统文化中的兼济情怀扎根太深,另一方面老庄玄远超脱之旨又极具诱惑力,只能依游两间,呈现给后人以矛盾的面孔!}

\lettrine{9.37} 桓大司马\myidx{桓温}下都\footnote{桓大司马:桓温。下都:到京都建康。建康位于长江下游,顺江而下至建康,故曰下都。},问真长\myidx{刘惔}曰\footnote{真长:刘惔。}:“闻会稽王\myidx{司马昱}语奇进\footnote{会稽王:晋简文帝司马昱曾封会稽王。奇进:大有进步。},尔邪\footnote{尔:如此。}?”{\fzxk\zihao{6}\textcolor{red}{\CJKunderwave{桓温别传}曰:“兴宁九(元)年,以温克复旧京,肃静华夏,进都督中外诸军事、侍中、大司马,加黄钺,使入参朝政。”}} 刘曰:“极进,然故是第二流中人耳\footnote{故:毕竟。}。”桓曰:“第一流复是谁?”刘曰:“正是我辈耳!”

{\cangkai\zihao{5}【评】王戎丧子后曾说过:“圣人忘情,最下不及情;情之所钟,正在我辈。”与刘真长此云第一流“正是我辈”,有异曲同工之妙,都表现了士人的自我意识和个性的张扬。在天地大宇宙之中,晋人有一个独特自足的小宇宙,那就是士人的心灵世界。有人说“中国诗很少用‘我’字,除非他自己在诗中起一定作用,因此他的情感里呈现出一种很难达到的非个人性质”(A·C格雷厄姆\CJKunderwave{中国诗的翻译})。岂止中国诗歌,中国人在一切领域里,都不太凸显自我。不过,\CJKunderwave{世说}中所记晋人好像有点例外,“我”、“是我辈”、“宁作我”,以及下则的“出我下”,诸多俯拾即是的“我”字,其核心都是张扬个性,使得晋人的总体性格显得有点另类。在他们的世界中,每个人都是天造地设的精英翘楚,有着独一无二的个性气质,谢鲲、周顗、殷浩、刘真长、桓温,无不将其才情气质挥洒自如。翻开黄卷青史,他们的音容笑貌依稀浮现,活跃纸上。其中,刘真长以近乎刻薄的自负、自傲,同样给人留下了鲜明印象。}

\lettrine{9.38} 殷侯\myidx{殷浩}既废\footnote{殷侯:殷浩。既废:永和六年(350年)殷浩以中军将军督师北伐,征许洛,大败而归,为桓温所劾,废为庶人。},桓公\myidx{桓温}语诸人曰\footnote{桓公:桓温。}:“少时与渊源共骑竹马\footnote{渊源:殷浩,字渊源。竹马:孩提游戏,以竹竿为马跨而骑之。},我弃去,已辄取之\footnote{已:犹了或以后。},故当出我下\footnote{故当:自然,当然。}。”{\fzxk\zihao{6}\textcolor{red}{\CJKunderwave{续晋阳秋}曰:“简文辅政,引殷浩为扬州,欲以抗桓;桓素轻浩,未之惮也。”}}

{\cangkai\zihao{5}【评】桓温此番话当出于殷浩北伐失败被贬为庶人之后。表面看来,殷浩此时已是虎落平阳,桓温还不依不饶、痛打落水狗,是不是乘人之危,缺少几分君子气度呢?实际上,这是桓温内心对殷浩仍存一丝忌惮的外部折射。殷浩少与桓温齐名,可能在某些方面超过桓温,这就给早年桓温不甘落后的心灵蒙上一层阴影,甚至化为终生攀比争竞的原动力,在殷浩一蹶不振以后,仍难以平复旧日痛苦的记忆。刘辰翁评曰:“此语能长人价格”,有理。桓温主观上贬损殷浩,揭其短处,却在客观上抬高其声价。故事从一个侧面展现了一代枭雄的常人心态,读来生活气息很浓。}

\lettrine{9.39} 人问抚军\myidx{司马昱}\footnote{抚军:简文帝司马昱,先为抚军大将军。}:“殷浩\myidx{殷浩}谈竟何如\footnote{殷浩:字渊源,善言玄理。谈:指清谈。竟:究竟。}?”答曰:“不能胜人,差可献酬群心\footnote{差:尚可,大体上能。献酬:原谓主人向宾客敬酒,此指令大家尽兴、欢畅。}。”

{\cangkai\zihao{5}【评】人的自我主观感觉和他人的印象之间并不能总是完全符合,本门第三十三、三十四则有殷浩对自己的评价,自视甚高,与简文帝的降格评价恰恰形成反差。不仅如此,就是不同的他者评价,其视角也不会完全一致。王濛、谢尚、刘真长、简文诸人,其实都带着各自不同的期待视野对殷浩做出评价。正因为如此,故刘勰有“音实难知,知实难逢,逢其知音,千载其一乎”之叹!(\CJKunderwave{文心雕龙·知音})}

\lettrine{9.40} 简文\myidx{司马昱}云\footnote{简文:简文帝司马昱。}:“谢安南\myidx{谢奉}清令不如其弟\myidx{谢聘}\footnote{谢安南:谢奉,曾任安南将军。清令:清纯美好。其弟:谢奉之弟谢聘。},{\fzxk\zihao{6}\textcolor{red}{安南,谢奉也。已见。\CJKunderwave{谢氏谱}曰:“奉弟聘,字弘远,历侍中、廷尉卿。”}} 学义不及孔巖(严)\myidx{孔巖}\footnote{学义:学问义理。孔巖:\CJKunderwave{晋书}本传作“孔严”。字彭祖,晋会稽山阴(今浙江绍兴)人。},{\fzxk\zihao{6}\textcolor{red}{\CJKunderwave{中兴书}曰:“巖(嚴)字彭祖,会稽山阴人。父伦,黄门侍郎。巖(严)有才学,历丹阳尹、尚书、西阳侯,在朝多所匡正。为吴兴太守,大得民和。后卒于家。”}} 居然自胜。”{\fzxk\zihao{6}\textcolor{red}{言奉任天真也。}}

{\cangkai\zihao{5}【评】谢安因谢奉对待贬官的态度令人赞叹,曾目之为“奇士”。其才华不必面面俱到,仅旷达任真一点,便足以成为世人赏誉的资本。于中可见晋人品评的标准。}

\lettrine{9.41} 未废海西公\myidx{司马奕}时\footnote{海西公:即晋废帝司马奕(342—386),字延龄。桓温败绩,欲内树威权,乃讽太后废帝为东海王,再降为海西县公,史称废帝。},王元琳\myidx{王珣}问桓元子\myidx{桓温}\footnote{王元琳:王珣。桓元子:桓温,字元子。}:“箕子、比干迹异心同\footnote{箕子:商纣王叔父,纣无道,箕子谏不从,佯狂为奴。周武王克商,封箕子于朝鲜。比干:纣王叔父,纣淫乱,比干进谏,纣怒而剖其心。},不审明公孰是孰非\footnote{不审:不知。明公:对有名位者的尊称。犹言“阁下”。孰是孰非:赞同谁不赞同谁。孰,谁。是、非,此处用为动词。认为是,认为非。}?”曰:“仁称不异,宁为管仲\footnote{仁称不异,作为仁人,称呼没有不同。宁为管仲:宁愿做管仲那样的仁人。管仲,名夷吾,春秋时齐国人。相齐桓公,九合诸侯,一匡天下,使齐桓公成为春秋五霸之一。\CJKunderwave{论语·宪问}:“子路曰:‘桓公杀公子纠,召忽死之,管仲不死,曰:未仁乎?’子曰:‘桓公九合诸侯,一匡天下,不以兵车,管仲之力也。如其仁!如其仁!’”}。”{\fzxk\zihao{6}\textcolor{red}{\CJKunderwave{论语}曰:“微子去之,箕子为之奴,比干谏而死。子曰:‘殷有三仁焉。’子路曰:‘桓公杀公子纠,召忽死之,管仲不死。曰未仁乎?’子曰:‘桓公九合诸侯,一匡天下,不以兵车,管仲之力。如其仁!如其仁!’”}}

{\cangkai\zihao{5}【评】箕子佯狂,比干谏死,孔子称其为殷之仁人。王珣问桓温对二位仁人的看法,桓温答以宁为管仲之仁。史载管仲事齐公子纠,及公子小白立为桓公,公子纠死,管仲遂为桓公所用,助其建立不世之功。儒家倡导的仁有小有大,境界不同。\CJKunderwave{论语·宪问}中有孔子对子路的回答:“桓公九合诸侯,一匡天下,不以兵车,管仲之功,如其仁,如其仁!”管仲虽细行有缺而大节不亏,因而为孔子赞叹。“大行不顾细谨,大礼不辞小让”,儒家追求造福苍生、泽被万世的大仁的高远境界。王珣、桓温问答皆用\CJKunderwave{论语},是一组很有水平的问答。而桓温对或佯狂或谏死的箕子、比干不置一词,已曲折透露出废立之心。以对话写活人物,言虽简约而内涵复杂,言外传达出无形的内心微妙,确是高妙笔法。}

\lettrine{9.42} 刘丹阳\myidx{刘惔}、王长史\myidx{王濛}在瓦官寺集\footnote{刘丹阳:刘惔,曾任丹阳尹,故称。王长史:王濛。瓦官寺:东晋佛寺名。故址在今南京附近。集:会集,聚会。},桓护军\myidx{桓伊}亦在坐\footnote{桓护军:桓伊,曾任护军将军。},{\fzxk\zihao{6}\textcolor{red}{桓伊已见。}} 共商略西朝及江左人物\footnote{商略:品评,评论。西朝:指西晋。江左:江东,此指东晋。}。或问:“杜弘治\myidx{杜乂}何如卫虎\myidx{卫玠}\footnote{杜弘治:杜乂字弘治,杜预孙。美姿容,有盛名。卫虎:卫玠,小字虎,美姿容,好言玄理。}?”桓答曰:“弘治肤清\footnote{肤清:外表清丽。肤,指外在仪容。},卫虎弈弈神令\footnote{奕奕:神采焕发。}。”王、刘善其言。{\fzxk\zihao{6}\textcolor{red}{虎,卫玠小字。\CJKunderwave{玠别传}曰:“永和中,刘真长、谢仁祖共商略中朝人。或问:‘杜弘治可方卫洗马不?’谢曰:‘安得比!其间可容数人。’”\CJKunderwave{江左名士传}曰:“刘真长曰:‘吾请评之。弘治肤清,叔宝神清。’论者谓为知言。”}}

{\cangkai\zihao{5}【评】桓伊此评甚为高妙。“肤清”、“神令”,即晋人所谓之“形清”、“神清”。晋人虽重外在形貌之清秀,而更重由形入神,传神写照,由外在入手传达出人物的内在情怀,这是人物品藻的最高境界。神清高出形清几多许,则卫玠高于杜乂多多许。恰如谢仁祖所说,“其间可容数人”。桓伊言辞不置褒贬而实有皮里阳秋之义。抛开故事中人物不论,仅从字面上之“形清”、“神清”出发,参之以今日人物审美时尚,亦能体味二者之差距。今日之人物审美,如各种夺人眼球的模特大赛和选美活动,风靡一时的人造美女、美男,及对演艺界明星的疯狂崇拜,实际的情况是,选手除了漂亮的脸蛋和身材,内里大多空空荡荡;所谓的巨星、歌后,亦不乏人渣、垃圾。这就足见世俗对美的欣赏和追求多流于表面,大概对应于“形清”的层面。话虽如此,杜玠绝非腹内草莽的绣花枕头,而是风标一时的风流名士。“肤清”、“神令”,只是二人之间的相较而言,岂是今日的帅哥们所能望其项背的?“肤清”与“神令”,自是不同的人物审美境界,何去何从,值得深省。}

\lettrine{9.43} 刘尹\myidx{刘惔}抚王长史\myidx{王濛}背曰\footnote{刘尹:刘惔。王长史:王濛。}:“阿奴比丞相\myidx{王导}\footnote{阿奴:相当于第二人称代词,是长者对幼者的亲昵之称,有时也用于同辈。丞相:指王导。},但有都长\footnote{但:还是有,确实有。都长:体貌闲雅。都:美,善。}。”{\fzxk\zihao{6}\textcolor{red}{阿奴,濛小字也。都,美也。\CJKunderwave{司马相如传}曰:“闲雅甚都。”\CJKunderwave{语林}曰:“刘真长与丞相不相得,每曰:‘阿奴比丞相,条达清长。’”}}

{\cangkai\zihao{5}【评】王濛美姿容,少居贫,帽破,入市买帽,妪悦其美而赠以新帽。此云王濛体貌闲雅,胜过王导,不为无据。刘应登曰:“刘与丞相不相得,故为优濛之言,谓皆胜之也。”可为参照。}

\lettrine{9.44} 刘尹\myidx{刘惔}、王长史\myidx{王濛}同坐\footnote{刘尹:刘惔。王长史:王濛。},长史酒酣起舞。刘尹曰:“阿奴今日不复减向子期\myidx{向秀}\footnote{向子期:向秀字子期,晋河内怀人。}。”{\fzxk\zihao{6}\textcolor{red}{类秀之任率也。}}

{\cangkai\zihao{5}【评】向秀为竹林名士,与嵇康锻铁于树下,又与吕安灌园于山阳。刘尹评王濛“不复减向子期”,当是称其襟怀洒落,类秀之任率也。以王濛比向秀,当是出于对前辈“林下名士”的崇敬与钦羡。锻铁、灌园,本是艰辛劳动,而由这些风神飘逸的名士为之,就有了一种非同寻常的诗意,成为人们效仿的样本。这些名士,虽其形体、足迹早已烟消云散,而其气质风度还常留后人的心中,时时作为比照的楷模。这又是一种不朽的方式——于传统的立德、立功、立言之外,开创了一种名士气质的不朽,像后来的陶渊明、苏轼。他可能萧条方外,毫无功绩德业可言,或者主要的并不以事功被彰显,却以其卓尔不群的精神,为人们指出了一种潇洒的生活方式,对当世或后代,永远具有精神烛照的意义。}

\lettrine{9.45} 桓公\myidx{桓温}问孔西阳\myidx{孔嚴}\footnote{桓公:桓温。孔西阳:孔嚴(严),封西阳侯。}:“安石\myidx{谢安}何如仲文\myidx{殷仲文}\footnote{安石:谢安,字安石。 仲文:殷仲文。殷仲文妻乃桓玄之姐,仲文是桓温女婿。有才藻,美容貌。}?”{\fzxk\zihao{6}\textcolor{red}{西阳,即孔巖(嚴)也。}} 孔思未对,反问公曰:“何如?”答曰:“安石居然不可陵践\footnote{居然:显然。陵践:侵凌欺侮。},其处故胜也\footnote{其处:他的自处之道。故:确实。胜:胜过别人。}。”

{\cangkai\zihao{5}【评】殷仲文为桓温女婿,其人有才藻,美容貌。桓温有此佳婿,自然是看之恒若不足,欲拟之于谢安。热血昏头、一时忘乎所以之情可以想见。所幸随后恢复了清醒,以年辈不伦、资望不及、比拟不经等原因,打消此意。“安石居然不可以陵践”一语,可见谢安身上自有一种不怒自威的气质,连政治对手桓温也不得不心生敬意,而不敢妄加比较。}

\lettrine{9.46} 谢公\myidx{谢安}与时贤共赏说\footnote{谢公:谢安。赏说:谈论评说。},遏\myidx{谢玄}、胡儿\myidx{谢朗}并在坐\footnote{遏:谢玄,小字遏。胡儿:谢朗,小字胡儿。},公问李弘度\myidx{李充}\footnote{李弘度:李充。}曰:“卿家平阳\myidx{李重},何如乐令\myidx{乐广}\footnote{平阳:李重,字茂曾,西晋江夏钟武(今河南信阳东南)人,弘度从父,晋惠帝时官至平阳太守,故称。乐令:乐广,(267—312)。出自琅邪王氏。兄衍为西晋士林清谈领袖,誉澄“阿平第一”。有士人“经澄所题者,衍不复有言,辙云‘已经平子矣’”。澄由是显名于世。澄官荆州刺史,日夜纵酒,不以军政为意。曾残杀巴蜀流民,激起民变。后因故为王敦所杀。胡毋彦国:即胡毋辅之,晋清谈名士。史称有知人之鉴。性嗜酒,任纵不拘小节。与王澄、王敦、庾敳俱为太尉王衍亲昵,号称“四友”。永嘉乱后,南渡卒于湘江刺史任上。}?”{\fzxk\zihao{6}\textcolor{red}{\CJKunderwave{晋诸公赞}曰:“李重字茂曾,江夏钟武人。少以清尚见称,历吏部郎、平阳太守。”}} 于是李潸然流涕曰\footnote{潸然:流泪的样子。}:“赵王\myidx{司马伦}篡逆\footnote{赵王篡逆:赵王司马伦,曾官中书令,故云,又称“裴令”。善\CJKunderwave{老}、\CJKunderwave{易},当时著名清谈名家。},乐令亲授玺绶\footnote{玺绶:指帝王所用之印。绶,印钮上所系的丝带。};{\fzxk\zihao{6}\textcolor{red}{\CJKunderwave{晋阳秋}曰:“赵王伦篡位,乐广与满奋、崔随进玺绶。”}} 亡伯\myidx{李重}雅正\footnote{亡伯:指李重。雅正:正派方直。},耻处乱朝,遂至仰药\footnote{仰药:服毒自杀。},恐难以相比。此自显于事实,非私亲之言。”{\fzxk\zihao{6}\textcolor{red}{\CJKunderwave{晋诸公赞}曰:“赵王为相国,取重为左司马。重以伦将篡,辞疾不就。敦喻之,重不复自治,至于笃甚;扶曳受拜,数日卒。时人惜之。赠散骑常侍。”}} 谢公语胡儿曰:“有识者果不异人意。”

{\cangkai\zihao{5}【评】六朝士人国家观念、忠君意识淡薄,遂造成只知有家、不知有国的局面。贤如乐令者,亦不能免受此讥。但乐令之授玺绶与阮籍之写\CJKunderwave{劝进文}一样,是身处乱朝的一种自我保全之举,有着难以明言的内心痛苦,并非以获新宠、助纣为虐为荣,与夫望风使舵、毫无人格操守的势利之徒有天渊之别。李重耻处乱朝,遂致仰药,以历史的眼光看,固是可圈可点的忠贞之举,但对乐广也不必太过苛责。如果是“赤条条来去无牵挂”的单身汉,阮籍也好、乐令也好,决不会如王衍之流因贪生怕死而弃置人生大义。无奈,父母、妻儿的鲜活生命会因自己的一时之勇,而遭受无辜的株连。千百年来的封建统治者真是狠毒,用连坐、族诛的办法逼迫无数不怕死的英雄就范,被迫低下高傲的头颅!即如为保全部下生命而无奈降胡的李陵,原图日后报效故国,不料“汉恩自浅胡自深”,李陵保全了部下却无法再保全家人的性命,从此遂断归汉之念,栖身朔漠,徒留英雄无奈的叹息!}

\lettrine{9.47} 王修龄\myidx{王胡之}问王长史\myidx{王濛}\footnote{王修龄:王胡之。王长史:王濛。}:“我家临川\footnote{临川:王羲之字逸少,曾为临川太守,故称。王羲之和王胡之都出自琅邪王氏家族,是堂兄弟。},何如卿家宛陵\footnote{宛陵:王述字怀祖,王承子,曾为宛陵令。王述和王濛都是太原晋阳王氏家族,是同族叔侄。}?”长史未答,修龄曰:“临川誉贵\footnote{誉贵:声誉显贵。}。”长史曰:“宛陵未为不贵。”{\fzxk\zihao{6}\textcolor{red}{\CJKunderwave{中兴书}曰:“羲之自会稽王友改授临川太守。王述从骠骑功曹出为宛陵令。述之为宛陵,多修为家之具,初有劳苦之声。丞相王导使人谓之曰:‘名父之子,(不患无禄),屈临小县,甚不宜尔!’述答曰:‘足自当止。’时人未知达也。后屡临州郡,无所造作,世始叹服之。”}}

{\cangkai\zihao{5}【评】凌濛初曰:“直是自相夸胜。”有理。王胡之与王濛品藻名家秀出子弟,本是高雅的时尚行为,却因王胡之的一句“临川誉贵”,改变了正常的评点轨道,演成了针尖对麦芒的门第夸耀之举。琅邪王氏堪称簪缨世家,太原王氏亦门第高华,世家子弟优越感已化为内心根深蒂固的“本我”,旁人切不可触惹,因为它总会不自觉地在各种场合流露出来,故事即是明显的一例。}

\lettrine{9.48} 刘尹\myidx{刘惔}至王长史\myidx{王濛}许清言\footnote{刘尹:刘惔。王长史:王濛。许:住处。清言:清谈玄理。},时苟子\myidx{王修}年十三\footnote{苟子:王修,小字苟子,王濛子。},倚床边听。既去,问父曰:“刘尹语何如尊\footnote{尊:敬称对方,相当于“您”,此指父亲。}?”长史曰:“韶音令辞不如我\footnote{韶音令辞:生动的语言,优美的辞令。韶音:优美的音调。令辞:美好的言辞。},往辄破的胜我\footnote{破的:射中靶心。比喻谈论能切中要点。}。”{\fzxk\zihao{6}\textcolor{red}{\CJKunderwave{刘惔别传}曰:“惔有隽才,其谈咏虚胜,理会所归,王濛略同,而叙致过之。”其词当也。}}

{\cangkai\zihao{5}【评】刘惔与王濛齐名友善,堪称知音,濛每云“刘君知我,胜我自知”。想必王濛对刘惔相知当亦不浅。故事中王濛之评刘及自评,平实中含着谦抑,当较为可信。“韶音令辞”指音调和言辞的美好,属于言谈的外在形式技巧层面。\CJKunderwave{晋书·王濛传}云:“谢安亦常称美濛云:‘王长史语甚不多,可谓有令音。’”可为佐证;“往辄破的”,指言谈能切中要害,属言谈的内容本质层面。显然“韶音令辞”要比“往辄破的”逊了一筹。评语中可见王濛对刘惔的推尊,故刘辰翁评曰:“韶音令辞亦属矜持”,实非的论。}

\lettrine{9.49} 谢万\myidx{谢万}寿春败后\footnote{谢万:谢安弟。寿春败后:晋穆帝升平三年(359年),谢万受命率军北征,结果在寿春大败而归,万被贬黜。寿春,县名,晋属淮南郡,治所即今安徽寿县。},简文\myidx{司马昱}问郗超\myidx{郗超}\footnote{简文:简文帝司马昱,时任抚军大将军。郗超:任桓温大司马,深得信任,立简文为帝后,迁中书侍郎,实代桓温监督朝廷而权重当时。}:“万自可败,那得乃尔失卒情\footnote{那得乃尔失卒情:诸本“卒”前增一“士”字,是。意谓为什么竟那样失掉士卒之心。那得,为什么。乃,竟然。尔,这样、那样。}?”超曰:“伊以率任之性,欲区别智勇\footnote{伊:他。率任:轻率任性。}。”{\fzxk\zihao{6}\textcolor{red}{\CJKunderwave{中兴书}曰:“万之为豫州,氏(氐)、羌暴掠司、豫,鲜卑屯结并、冀。万既受方任,自率众入颍,以援洛阳。万矜豪傲物,失士众之和。北中郎郗昙以疾还彭城,万以为贼盛致退,便回还南,遂自溃乱,狼狈单归。太宗责之,废为庶人。”}}

{\cangkai\zihao{5}【评】朝廷任命谢万为豫州刺史,后出师北伐,非人尽其材的任命,而是出于抑制桓温一族势力的考虑。谢万又偏无领兵布阵的才能,王羲之对此早有预言,谢安亦深以为忧。主帅虽弱,但如能兼听从善、上下同心,也不是毫无胜算。史载谢万“既受任北征,矜豪傲物,尝以啸咏自高,未尝抚众”,完全是一副名士派头。后“召集诸将,都无所说,直以如意指四坐云:‘诸将皆劲卒。’”(\CJKunderwave{晋书·谢万传})在魏晋时贵族重文轻武,所以呼将为卒,是极不礼貌的蔑视性称呼,必然挑起众人极大的不满而导致离心离德。名士和武人之间有着不尽相同的话语符号系统,谢万之败,不仅因缺乏军事才干,同时也败在名士风度的滥用,缺乏自知和知人之明!刘辰翁曰:“人人有区别,正坐失士卒情处,可以为戒。”此言得之。}

\lettrine{9.50}刘尹\myidx{刘惔}谓谢仁祖\myidx{谢尚}曰\footnote{刘尹:刘惔。谢仁祖:谢尚。}:“自吾有四友\footnote{四友:四个相知的朋友。\CJKunderwave{尚书大传}云孔子自述有“四友”,即颜回、子贡、子张、子路。},门人加亲\footnote{门人加亲:门生弟子更加亲近。此处刘尹以颜回比谢尚。}。”谓许玄度\myidx{许询}曰\footnote{许玄度:许询。}:“自吾有由\footnote{由:仲由,即子路。},恶言不及于耳\footnote{恶言不及于耳:坏话传不到我的耳朵里了。此以仲由比许询。}。”二人皆受而不恨\footnote{受而不恨:接受而无不满之意。}。{\fzxk\zihao{6}\textcolor{red}{\CJKunderwave{尚书大传}曰:“孔子曰:‘文王有四友。自吾得回也,门人加亲,是非胥附邪?自吾得赐也,远方之士至,是非奔走邪?自吾得师也,前有辉,后有光,是非先后邪?自吾得由也,恶言不入于耳,是非御悔(侮)邪?’”}}

{\cangkai\zihao{5}【评】刘惔活用\CJKunderwave{尚书大传}之语,其文曰:“孔子曰:‘文王有四友,自吾得回也,门人加亲,是非胥附耶?……自吾得由也,恶言不入于耳,是非御侮耶?’”刘惔之言,自比于孔子,以谢尚、许询比附颜回、子路,极度自负以至忘乎所以之情态跃然纸上。但骄狂之中,又带几分张扬自我的天真,所以也有可爱的一面,以此谢、许二人不恨。}

\lettrine{9.51} 世目殷中军\myidx{殷浩}“思纬淹通\footnote{殷中军:殷浩。思纬:思理,思路。淹通:精深广博。}”,比羊叔子\myidx{羊祜}\footnote{羊叔子:羊祜,字叔子。}。{\fzxk\zihao{6}\textcolor{red}{羊祜德高一世,才经夷险;渊源蒸烛之曜,岂喻日月之明也。}}

{\cangkai\zihao{5}【评】阮籍尝登广武城楼,有“时无英雄,使竖子成名”之叹。东晋末年,国势日衰,大树飘零。桓温虎视晋鼎,朝中文武罕有其匹,难与抗衡。殷浩就是在这样的特定历史条件下被推到政治斗争的风口浪尖,并领导北伐。正如桓温所言,朝廷用违其才。羊祜既能攻城略地,制定战略国策,为国家统一做贡献;同时又是一位多情的士子,是晋初璨若星辰的人杰中之佼佼者,其清言析理,或不如殷浩名气,但综合衡量二人之实际能量与历史贡献,以殷浩比拟羊祜,可谓不伦。}

\lettrine{9.52} 有人问谢安石\myidx{谢安}、王坦之\myidx{王坦之}优劣于桓公\myidx{桓温}\footnote{谢安石:谢安,字安石。王坦之:王承之孙。桓公:桓温。}。桓公停欲言\footnote{停:正。汉魏六朝常用语。},中悔曰:“卿喜传人语,不能复语卿。”

{\cangkai\zihao{5}【评】简文帝时,谢安为侍中,王坦之为左卫将军,是简文所倚靠的佐命重臣。简文死后,谢、王又尽忠匡辅晋孝武帝,成为桓温篡晋道路上的重要障碍。常言道“树大招风”,谢、王二人因位高权重而成为朝廷上下关注、品评的对象。在此权力斗争的微妙时刻,一招不慎,都会造成严重后果。所以桓温欲言又止,大概其所欲言必有臧否谢、王之辞,问者又是一个爱传播小道消息的“长舌妇”,故就此打住。故事运用白描手法刻画了桓温豪放不羁性格中细腻、谨慎的一面,寥寥几笔而人物性格逼真传神。诚如陈梦槐所评“有情有景”。}

\lettrine{9.53} 王中郎\myidx{王坦之}尝问刘长沙\myidx{刘奭}曰\footnote{王中郎:王坦之。刘长沙:刘奭,晋彭城(今江苏徐州)人。曾任长沙相。}:“我何如苟子\myidx{王修}\footnote{苟子:王修字敬仁,小字苟子。}?”{\fzxk\zihao{6}\textcolor{red}{\CJKunderwave{大司马官属名}曰:“刘奭字文时,彭城人。”\CJKunderwave{刘氏谱}曰:“奭祖昶,彭城内史。父济,临海令。奭历车骑咨议、长沙相、散骑常侍。”}} 刘答曰:“卿才乃当不胜苟子\footnote{乃当:自然是。},然会名处多\footnote{会名处:领悟名理能融会贯通的地方。}。”王笑曰:“痴。”

{\cangkai\zihao{5}【评】王坦之声誉超出王修远甚,故对比较胜券在握。不料刘长沙并不送顺水人情,倒真的一本正经地指出二人优劣。刘之率真惹得王坦之发笑,以其为痴人。但痴人之呆,却正是刘奭不媚权贵的可爱之处。}

\lettrine{9.54} 支道林\myidx{支遁}问孙兴公\myidx{孙绰}\footnote{支道林:支遁字道林,东晋名僧。孙兴公:孙绰字兴公。}:“君何如许掾\myidx{许询}\footnote{许掾:许询字玄度,被征为司徒左掾,不就。}?”孙曰:“高情远致\footnote{高情远致:高尚的情操,超逸的旨趣。此指隐逸之情。},弟子蚤已服膺\footnote{弟子:学生。此用于自称,表示谦恭。 蚤:通“早”。服膺:心悦诚服。};一吟一咏\footnote{一吟一咏:指吟诗作赋。},许将北面\footnote{北面:指向人称臣或居于人下。引申为折服于人。}。”

{\cangkai\zihao{5}【评】孙绰、许询俱隐居东山,有终老此间之意。孙绰后来没能坚持住“遂初”之志,许询倒是终其一生守住了固穷之节。相较于孙绰的言行乖谬,许询确实堪称“高情远致”。再看文学成就,二人都是著名的玄言诗人。简文帝称许询“玄度五言诗,可谓妙绝时人”(\CJKunderwave{文学})、“襟情之咏,偏是许之所长”(\CJKunderwave{赏誉})。孙绰固然为一时之选,然许询并非不善吟咏者。不过,许询诗文集亡佚,或许是经不起历史考验而遭淘汰。而孙绰作品,则入\CJKunderwave{昭明文选}。故当时人士已有“或爱孙才藻,而无取于许”的说法,可参阅本门第61则。二人文学才情,终显高下。孙绰以文学自诩,也是实事求是之言。看准别人,认清自己,而非信口雌黄,当时士人的确可爱可敬。}

\lettrine{9.55} 王右军\myidx{王羲之}问许玄度\myidx{许询}\footnote{王右军:王羲之。许玄度:许询。}:“卿自言何如安石\myidx{谢安}\footnote{安石:谢安。}?”许未答,王因曰:“安石故相与雄,阿万\myidx{谢万}当裂眼争邪\footnote{故:确实,当然。阿万:谢万,谢安弟。裂眼:睁大眼睛。}!”{\fzxk\zihao{6}\textcolor{red}{\CJKunderwave{中兴书}曰:“万器量不乃(及)安石,虽居藩任,安在私门之时,名称居万上也。”}}

{\cangkai\zihao{5}【评】王羲之与许询、谢安、谢万乃相与友善的一时名流,经常在轻松随意的气氛中选取人物进行品藻。王羲之先问许询自比于谢安如何?随即觉得相比不伦,故转而以谢万当怒目与谢安强争高下自答。“攀安提万”一语及此处的“裂眼争”都形象地见出了兄弟二人的差距。王羲之似以此暗示谢安卓绝特出,许询诸流难以望其项背,故朱铸禹\CJKunderwave{汇校集注}评曰:“言外之意足下尚不及万,况安石乎?”}

\lettrine{9.56} 刘尹\myidx{刘惔}云\footnote{刘尹:刘惔。}:“人言江虨\myidx{江虨}田舍\footnote{江虨:字思玄,东晋陈留(今河南开封东北)人。江统子。为晋中兴大臣,累官至尚书左仆射。田舍:乡下人。借以讥人土气,见识浅陋。},江乃自田宅屯\footnote{乃自:确实。田宅屯:耕种宅田。田,名词动化。}。”{\fzxk\zihao{6}\textcolor{red}{谓能多出有也。}}

{\cangkai\zihao{5}【评】儒家传统观念里有所谓的大人之事、小人之事之别,其界限泾渭分明、不得逾越。孟子更大讲“劳心者治人,劳力者治于人”,即在统治阶级内部,六朝士人受门阀世族观念濡染,更将士庶差别发挥到淋漓尽致的程度。贵族阶级凭借天生的血统,过着锦衣玉食的生活,“四体不勤,五谷不分”,根本无法体会劳动稼穑的艰辛。江虨以士族的身份亲自耕种宅田,是极能吸引士人眼球的行为,难怪被视为异类,士林间有“田舍翁”、“乡巴佬”之讥。南朝到溉祖上曾亲自担粪自给,到溉后来做了大官,仍不免被人讥为尚有馀臭,可与此印证。}

\lettrine{9.57} 谢公\myidx{谢安}云\footnote{谢公:谢安。}:“金谷中苏绍\myidx{苏绍}最胜\footnote{金谷:又称金谷涧,地名。在今河南洛阳东北。西晋武帝太康中,石崇在此筑园,世称金谷园。惠帝元康六年(296),石崇邀集当世名流苏绍等三十人在金谷游宴赋诗。苏绍:字世嗣,西晋始平武功(今属陕西)人。}。”绍是石崇\myidx{石崇}姉(姊)夫\footnote{石崇(249—300):字季伦,西晋渤海南皮(今属河北)人。任侠无行,在荆州劫掠客商而成巨富。于河阳金谷置别馆,每与贵戚羊琇、王恺等夸富竞侈,极尽奢靡。与潘岳等谄事贾后、贾谧。为赵王司马伦收斩。},苏则\myidx{苏则}孙\footnote{苏则:三国魏时历官金城太守。},愉\myidx{苏愉}子也\footnote{愉:苏愉,字休豫。仕晋为光禄大夫。}。{\fzxk\zihao{6}\textcolor{red}{石崇\CJKunderwave{金谷诗叙}曰:“余以元康六年,从太仆卿出为使,持节监青、徐诸军事、征虏将军。有别庐在河南县界金谷涧中,或高或下,有清泉茂林,众果、竹柏、药草之属,莫不毕备。又有水碓、鱼池、土窟,其为娱目欢心之物备矣。时征西大将军祭酒王诩当还长安,余与众贤共送往涧中,昼夜游宴,屡迁其坐,或登高临下,或列坐水滨。时琴瑟笙筑,合载车中,道路并作。及住,令与鼓吹递奏。遂各赋诗,以叙中怀,或不能者,罚酒三斗。感性命之不永,惧凋落之无期,故具列时人官号、姓名、年纪,又写诗箸后。后之好事者,其览之哉!凡三十人,吴王师、议郎、关中侯,始平武公(功)苏绍,字世嗣,年五十,为首。”\CJKunderwave{魏书}曰:“苏则字文师,扶风武功人。刚直疾恶,常慕汲黯之为人。仕至侍中、河东相。”\CJKunderwave{晋百官名}曰:“愉字休豫,则次子。”山涛\CJKunderwave{启事}曰:“愉忠义有智意。”位至光禄大夫。}}

{\cangkai\zihao{5}【评】故事可见风流名相谢安对金谷宴游的缅怀、神往之情。此情结与王羲之相似。羲之\CJKunderwave{兰亭集序},明显见石崇\CJKunderwave{金谷诗序}影迹,时人以之比美石崇,羲之闻而喜,事载\CJKunderwave{晋书}本传。于此可见金谷宴游在近代士人中的影响。}

\lettrine{9.58} 刘尹\myidx{刘惔}目庾中郎\myidx{庾敳}\footnote{刘尹:刘惔。庾中郎:庾敳。}:“虽言不愔愔似道\footnote{愔愔:幽深貌。道:学说。特指老、庄之道。},突兀差可以拟道\footnote{突兀:特出。差:差不多。拟:类似,比拟。}。”{\fzxk\zihao{6}\textcolor{red}{\CJKunderwave{名士传}曰:“敳颓然渊放,莫有动其听者。”}}

{\cangkai\zihao{5}【评】\CJKunderwave{老子}云“大巧若拙,大变若讷”,庾敳虽颓然渊放,酣醉无为,给人以木讷、糊涂的印象,实则对自身处境有着非常清醒的洞察,对玄理更能深刻体察。正如\CJKunderwave{世说}\CJKunderwave{文学}门所载,“庾子嵩读\CJKunderwave{庄子},开卷一尺许便放去,曰:‘了不异人意。’”可以印证刘尹之言不虚。庾敳的颓然渊放是对现实政治无能为力而采取的一种无奈自保行为。}

\lettrine{9.59} 孙承公\myidx{孙统}云\footnote{孙承公:孙统,字承公,东晋太原中都(今山西平遥西)人。孙绰兄。}:“谢公\myidx{谢安}清于无弈(奕)\footnote{谢公:谢安。清:高洁,清逸。无奕:谢奕,字无奕,谢安兄。},{\fzxk\zihao{6}\textcolor{red}{\CJKunderwave{中兴书}曰:“孙纯(统)字承公,太原人。善属文,时人谓其有祖楚风。仕至馀姚令。”}} 润于林道\myidx{陈逵}\footnote{润:文雅有风采。林道:陈逵,字林道,东晋颍川许昌(今属河南)人。}。”{\fzxk\zihao{6}\textcolor{red}{\CJKunderwave{陈逵别传}曰:“逵字林道,颍川许昌人。祖淮(准),太尉。父畛,光禄大夫。逵少有干,以清敏立名。袭封广陵公、黄门郎、西中郎将,领梁、淮南二郡太守。”}}

{\cangkai\zihao{5}【评】清,清纯、清逸;润,文雅有风采。清逸、文雅颇能概括出谢安的性情气度,后人有诗云:“隋炀不幸为天子,安石可怜作相公。若使二人穷到老,一为名士一文雄。”可见谢安的文采风流不仅独擅一代,而是成为历代文人名士倾慕的高标。孙统以二贤陪衬作比,而刘辰翁则弃二贤于不顾而直评谢安本人,其评价之高,更在孙统之上,评曰:“谁知二贤,只见谢公清润耳。”令人仰思谢公风标。}

\lettrine{9.60} 或问林公\myidx{支遁}\footnote{林公:支道林。}:“司州\myidx{王胡之}何如二谢\myidx{谢安}\myidx{谢万}\footnote{司州:王胡之,曾任西中郎将、司州刺史。二谢:指谢安、谢万兄弟。}?”林公曰:“故当攀安提万\footnote{故当:当然是。攀安提万:意谓王胡之的才能不及谢安,强于谢万。攀,攀缘。 提,提携。}。”{\fzxk\zihao{6}\textcolor{red}{\CJKunderwave{王胡之别传}曰:“胡之好谈讲,善属文辞,为当世所重。”}}

{\cangkai\zihao{5}【评】林公以出家人的冷眼旁观,更容易对红尘中人有清醒的洞察。“攀安提万”一语,简洁、精当地概括出了王胡之与二谢兄弟的高下优劣,意谓王胡之须用力攀登方能达到谢安的高度,攀登中又可提携等而下之的谢万!刘辰翁评曰:“语强,然有思。”此言为是,林公之言确实耐人寻味。王胡之长谢安十五六岁,他在公元349年死时,谢安还是年届而立的青年。但士林间自有公论,英雄豪杰也左右不得,即便心有不甘也只能徒唤奈何了!后万败而安石成就伟业,早在林公料中。高僧之高,何减名士!}

\lettrine{9.61} 孙兴公\myidx{孙绰}、许玄度\myidx{许询}皆一时名流\footnote{孙兴公:孙绰。许玄度:许询。一时:当时,当代。}。或重许高情\footnote{高情:高逸的情致。},则鄙孙秽行\footnote{秽行:污浊、恶劣的行为。};或爱孙才藻\footnote{才藻:才思文采。},而无取于许。{\fzxk\zihao{6}\textcolor{red}{\CJKunderwave{宋明帝文章志}曰:“绰博涉经史,长于属文,与许询俱有负俗之谈。询卒不降志,而绰婴纶世务焉。”\CJKunderwave{续晋阳秋}曰:“绰虽有文才,而诞纵多秽行,时人鄙之。”}}

{\cangkai\zihao{5}【评】故事可与本门第五十四则参看。孙绰之所谓“秽行”,正史不载,或许指其中年以后出仕的经历,违背了早年的高尚之志。国人思维观念中,二元对立、非此即彼的两极思维模式处处可见,看人只看下半截的积习,是此模式的典型反映。“妓女从良,则半生烟花不论;贞妇失节,则一世声名尽非。”古人的一副对联,形象地描绘了这一传统陋习。孙绰虽违初衷,可以理解,但时人冠之以“秽行”,则见出隐士本位主义的狭隘心态,谢安中年出山即蒙“小草”之讥,亦属此类。桓温欲移都洛阳,以便控制朝廷,“朝廷畏温,不敢为异,而北土萧条,人情疑惧,虽并知不可,莫敢先谏”。独孙绰上疏提出反对意见,表现了威武不能屈的士人气节,何秽之有?对于贵族偏见,岂可当真!}

\lettrine{9.62} 郗嘉宾\myidx{郗超}道谢公\myidx{谢安}\footnote{郗嘉宾:郗超,字嘉宾,任桓温大司马,深得信任,立简文为帝后,迁中书侍郎,实代桓温监督朝廷而权重当时。道:评论。谢公:谢安。}:“造膝虽不深彻\footnote{造膝:犹促膝。引申为谈论、议论。深彻:深刻透彻。},而缠绵纶至\footnote{缠绵:周详细密。纶至:指思想极有条理。}。”又曰\footnote{又曰:言时人又有此说。}:“右军\myidx{王羲之}诣嘉宾\footnote{右军:王羲之。诣:造诣,指思理深刻。按,“右军诣嘉宾”句,徐震堮\CJKunderwave{校笺}疑“诣”下“嘉宾”二字衍,疑是。}。”嘉宾闻之云:“不得称诣,政得谓之朋耳\footnote{政:通“正”,只。朋:同等,齐同。}。”谢公以嘉宾言为得\footnote{得:正确。}。{\fzxk\zihao{6}\textcolor{red}{凡彻、诣者,盖深核之名也。谢不彻,王亦不诣。谢、王于理,相与为朋俦也。}}

{\cangkai\zihao{5}【评】王羲之、谢安是东晋中后期清谈领袖,郗超言谈当不在王、谢之下,\CJKunderwave{晋书}本传载超“交游士林,每存胜拔,善谈论,义理精微”,\CJKunderwave{弘明集}存其\CJKunderwave{奉法要}一文。郗超论王、谢,心平气和地道其优劣,既不“捧杀”,亦不“棒杀”,连被评者谢安亦深许之。故事客观上展现了高层名士间良好的品评风气,虽所行之道不同,而能超越家族罅隙、人际纠结、政见纷争,评者、被评者均能以从容不迫的心态,对待学术交锋,千载之下,令人缅怀。对今日学术界“酷评”、“谀评”之风,是不是有一定的警示作用呢?}

\lettrine{9.63} 庾道季\myidx{庾龢}云\footnote{庾道季:庾龢,字道季。}:“思理伦和\footnote{思理:思辨能力。伦和:有序,有条理。},吾愧康伯\myidx{韩伯}\footnote{愧:有愧于。康伯:韩伯。};志力强正\footnote{志力:意志。强正:坚强正直。},吾愧文度\myidx{王坦之}\footnote{文度:王坦之,王承之孙。}。自此以还\footnote{以还:以下。},吾皆百之\footnote{百之:百倍于他们。}。”{\fzxk\zihao{6}\textcolor{red}{庾钦(龢)已见。}}

{\cangkai\zihao{5}【评】庾道季自评思维的逻辑性方面不如韩康伯,意志力品格方面逊于王坦之,其馀则强人百倍,自谦的口气中洋溢着自负,实以第一流人物自命。而这份自负若离了高华的门第作支撑,恐怕就不会显得那么潇洒!}

\lettrine{9.64} 王僧恩\myidx{王祎之}轻林公\myidx{支遁}\footnote{王僧恩:王祎之,字文劭,小字僧恩,东晋太原晋阳(今山西太原)人。林公:支道林。},蓝田\myidx{王述}曰\footnote{蓝田:王述,王祎之父。}:“勿学汝兄\myidx{王坦之}\footnote{汝兄:指王坦之。},汝兄自不如伊。”{\fzxk\zihao{6}\textcolor{red}{僧恩,王祎之小字也。\CJKunderwave{王氏世家}曰:“祎之字文劭,述次(少)子。少知名,尚寻阳公主。仕至中书郎,未三十而卒,坦之悼念,与桓温称之。赠散骑常侍。”}}

{\cangkai\zihao{5}【评】王坦之弱冠有重名,弟弟王祎之可能对兄长有点崇拜心理,唯兄马首是瞻。坦之尚刑名之学,曾著\CJKunderwave{废庄论}以非时俗放荡,故难与尚玄虚的支道林相得。(坦之与支道林不睦,\CJKunderwave{轻诋}门载二人互相诘难之辞)祎之也人云亦云地跟着效仿,轻视支道林的恶言恣意出口,故老爹王述有此训诫之言。王述此评,对儿子毫无袒护偏爱之心,而对被轻视的对象支道林给予客观、公正的评价,这样的家庭教育无疑是健康的,对孩子的成长是恰逢其时的良药、警钟,值得今天的家长和教育界人士借鉴。}

\lettrine{9.65} 简文\myidx{司马昱}问孙兴公\myidx{孙绰}\footnote{简文:简文帝司马昱。孙兴公:孙绰。}:“袁羊\myidx{袁乔}何似\footnote{袁羊:袁乔字彦叔,小字羊。 何似:怎样。}?”答曰:“不知者不负其才\footnote{负:违弃。引申为舍弃。},知之者无取其体\footnote{体:指德行。}。”{\fzxk\zihao{6}\textcolor{red}{言其有才而无德也。}}

{\cangkai\zihao{5}【评】孙兴公评袁乔语颇巧妙,似褒实贬,形象地传达出其有才无德的人格轮廓。\CJKunderwave{晋书}本传载“乔博学有文才,注\CJKunderwave{论语}及\CJKunderwave{诗},并诸文笔皆行于世”。又通晓用兵之策,桓温伐蜀,乔屡建奇谋,是文武全才。孙兴公称其人品无足取,不知何本?莫非因其尝为桓温司马,与桓温过从甚密而属桓氏党人?孙绰不买桓温的账,触忤过桓温,袁乔当然更不在话下了。若依此评价一个人的品德,就未免失之主观而有失公允了。}

\lettrine{9.66} 蔡叔子(子叔)\myidx{蔡系}云\footnote{蔡叔子:当作“蔡子叔”。蔡系字子叔,晋司徒蔡谟子,官至抚军长史。}:“韩康伯\myidx{韩伯}虽无骨干\footnote{韩康伯:韩伯字康伯。},然亦肤立\footnote{然亦肤立:外表能自树立。意谓韩康伯外观形象尚挺立,并非臃肿得不像样子。肤,指外观形象。}。”

{\cangkai\zihao{5}【评】故事语涉双关,耐人寻味。韩康伯肥胖无骨,人有“肉鸭”之讥,但并非臃肿得不成样子,凭其外表亦有可观,此为表层意;进一步理解,其人虽有肉无骨,然亦足以自立于世,精神气度特出故也。\CJKunderwave{晋书}本传载“识者谓伯可谓澄世所不能澄,而裁世所不能裁者矣,与夫容己顺众者,岂得同时而共称哉”!康伯刚正有器局,表现了士大夫的无形傲骨。“肤立”者,玄家所谓无骨之骨也,自是高评。}

\lettrine{9.67} 郗嘉宾\myidx{郗超}问谢太傅\myidx{谢安}曰\footnote{郗嘉宾:郗超。谢太傅:谢安。}:“林公\myidx{支遁}谈何如嵇公\myidx{嵇康}\footnote{林公:支道林。嵇公:嵇康,三国时谯郡铚(今安徽亳县)人。“竹林七贤”之一。曾任中散大夫,故称嵇中散。当时著名思想家、文学家、清谈名家。因其主张越名教而任自然,抨击礼法之士,不与司马氏统治集团合作,盛年被杀。}?”谢云:“嵇公勤箸脚\footnote{勤箸脚:不断举足,努力奋进。勤,努力。箸脚,落脚。},裁可得去耳\footnote{裁:通“才”。这二句意谓,若谈玄论辩嵇康不是支道林对手。}。”{\fzxk\zihao{6}\textcolor{red}{\CJKunderwave{支遁传}曰:“遁神悟机发,风期所得,自然超迈也。”}} 又问:“殷\myidx{殷浩}何如支\footnote{殷:指殷浩。}?”谢曰:“正尔有超拔\footnote{正尔:恰好。超拔:指高超特出的风采。},支乃过殷;然亹亹论辩\footnote{亹亹:形容议论滔滔不绝。},恐殷欲制支。”

{\cangkai\zihao{5}【评】本门稍后几条又载谢安答王子敬语,评支遁不如庾亮,又答王孝伯,谓支不如王濛。前贤释此,均谓嵇康言谈须努力向前,才赶得上支遁,又谓亹亹论辩,殷浩恐胜过支遁,而嵇康逊于庾亮、王濛、刘惔及殷浩。余意以为,如此抑扬,不符合谢安的一贯风格,且与其“先辈初不臧贬七贤”(见本门第七十一则)语相悖缪,当是传闻之讹。此另有一解:王导等名流,均对嵇康礼敬有加,聪睿如谢安当不至于轻薄嵇康。故“嵇公勤箸脚”,于嵇康褒而非贬,意谓支公须于嵇康面前不停努力著脚,才有可能接近赶上。如此理解,则前后文义通顺。谢安于支公玄谈,评价一般,正见其理论自负。}

\lettrine{9.68} 庾道季\myidx{庾龢}云\footnote{庾道季:庾龢字道季。}:“廉颇、蔺相如虽千载上死人\footnote{廉颇、蔺相如:战国时赵国良将、贤相。},懔懔恒有生气\footnote{懔懔:同凛凛,严正而令人敬畏的样子。};{\fzxk\zihao{6}\textcolor{red}{\CJKunderwave{史记}曰:“廉颇者,赵良将也,以勇气闻诸侯。蔺相如者,赵人也。赵惠文王时,得楚和氏璧,秦昭王请以十五城易之。赵遣相如送璧,秦受之,无还城意。相如请璧示其瑕,因持璧却立倚柱,怒发上冲冠,曰:‘王欲急臣,臣头今与璧俱碎。’秦王谢之。后秦王使赵王鼓瑟,相如请秦王击筑。赵以相如功大,拜上卿,位在廉颇上。”}} 曹蜍\myidx{曹蜍}、{\fzxk\zihao{6}\textcolor{red}{蜍,曹茂之小字也。\CJKunderwave{曹氏谱}曰:“茂之字永世,彭城人也。祖韶,镇东将军司马。父曼,少府卿。茂之仕至尚书郎。”}} 李志\myidx{李志}{\fzxk\zihao{6}\textcolor{red}{\CJKunderwave{晋百官名}曰:“志字温祖,江夏钟武人。”\CJKunderwave{李氏谱}曰:“志祖重,散骑常侍。父慕,纯阳令。志仕至员外常侍、南康相。”}} 虽见在\footnote{曹蜍:曹茂之,字永世,小字蜍,晋彭城(今江苏徐州)人。李志:字温祖,晋江夏钟武(今河南信阳东南)人。见在:现在还活着。},厌厌如九泉下人\footnote{厌厌:萎靡不振的样子。}。人皆如此,便可结绳而治\footnote{结绳而治:上古结绳而治以不同形状和数量的绳结标识不同的事。此处比喻最原始最简单的治理方法。},但恐狐狸狢猯啖尽\footnote{狢:同“貉”,一种似狸的野兽。猯:猪獾。啖:吃。}。”{\fzxk\zihao{6}\textcolor{red}{言人皆如曹、李质鲁淳慤,则天下无奸民,可结绳致治。然才智无闻,功迹俱灭,身尽于狐狸,无擅世之名也。}}

{\cangkai\zihao{5}【评】司马迁\CJKunderwave{报任少卿书}有言:“人固有一死,或重于泰山,或轻于鸿毛,用之所趋异也。”庾道季此处将赵国良将、贤相廉颇、蔺相如与晋人曹蜍、李志两相比较,情感爱憎倾向极其鲜明,表达了立事、立功求不朽的价值取向,“重于泰山”正是庾道季等众多志士仁人前仆后继追求的超拔人生境界。曹、李之流尸位素餐,才智无闻,实是对人生毫无设计,只是混活等死。这类人数量虽多,却不足成为社会的中坚力量,有愧于士大夫的称号。故王世懋评曰:“道季辞严亦殊有生气。”道季为庾亮子,高门显第、锦衣玉食的生活并没有使其养成贵游子弟游手好闲、虚浮放诞的恶习,而是高自砥砺,在思想言谈(本门第六十三则有“自此以还,吾皆百之”之语)、立功等方面均有不凡之举。}

\lettrine{9.69} 卫君长\myidx{卫永}是萧祖周\myidx{萧轮}妇兄\footnote{卫君长:卫永字君长。 萧祖周:萧轮字祖周。},谢公\myidx{谢安}问孙僧奴\myidx{孙腾}\footnote{谢公:谢安。孙僧奴:孙腾,字伯海,小字僧奴。孙统子。}{\fzxk\zihao{6}\textcolor{red}{:僧奴,孙腾小字也。\CJKunderwave{晋百官名}曰:“腾字伯海,太原人。”\CJKunderwave{中兴书}曰:“腾,紞(统)子也,博学,历中庶子、廷尉。”}} “君家道卫君长云何\footnote{君家:犹言君,用于尊称对方。相当于您。云何:怎样。}?”孙曰:“云是世业人\footnote{世业人:经心世务的人。即办实事、干事业的人。}。”谢曰:“殊不尔\footnote{殊:颇,很。不尔:不是如此。},卫自是理义人\footnote{理义人:讲求玄学义理的人。}。”于时以比殷洪远\myidx{殷融}\footnote{殷洪远:殷融。}。

{\cangkai\zihao{5}【评】看来卫永确是一位备受争议的人物,孙绰曾非议其“此子神情都不关山水,而能作文”,但庾亮为之开释;今孙绰侄孙腾又称其“世业人”,即经心世务的实干型人物。魏晋士人以脱略世务、游心太玄为高,故卫永之遭受清高人士的白眼也在情理之中。幸好有谢安以与众不同的眼光,视其为我辈玄学中人,有名人的掩护,得以顺利过关。看来,要透过现象看本质,深入品评人物,殊非易事。众说纷纭,各执一词,永远也争执不清。阅人无数的士林领袖往往语出惊人,有一语定乾坤之功。}

\lettrine{9.70} 王子敬\myidx{王献之}问谢公\myidx{谢安}\footnote{王子敬:王献之。 谢公:谢安。}:“林公\myidx{支遁}何如庾公\myidx{庾亮}\footnote{林公:支道林。 庾公:庾亮。}?”谢殊不受\footnote{殊不受:很不愿意接受。此谓不愿接受王献之所问,发表意见。},答曰:“先辈初无论,庾公自足没林公\footnote{自:原本,本来。足:够得上。 没:盖过。}。”{\fzxk\zihao{6}\textcolor{red}{\CJKunderwave{殷羡言行}曰:“时有人称庾太尉理者,羡曰:‘此公好举素本槌人。’”}}

{\cangkai\zihao{5}【评】支道林与庾亮是玄谈界两个重量级人物,一位是游心方外的高僧大德、王公的座上宾,一位是经纬庙廊的晋室重臣兼清谈领袖。因先前并无人商较二人的先例,而自己位高言重,一出言极有可能成为士人口耳相传的定论,故谢安内心极为审慎,“谢殊不受”一语传达出极不情愿之意。推躲不过,也只能根据自己的品评视角据实而答。其倾心钟情处乃在庾亮,正是与其出处大原则相似的同道中人。}

\lettrine{9.71} 谢遏\myidx{谢玄}诸人共道“竹林”优劣\footnote{谢遏:谢玄,谢安侄。竹林:指竹林七贤。},谢公\myidx{谢安}云:“先辈初不臧贬‘七贤\footnote{初不:从来不,一点不。臧贬:褒贬,谓评论优劣。}’。”{\fzxk\zihao{6}\textcolor{red}{\CJKunderwave{魏氏春秋}曰:“山涛通简有德,秀、咸、戎、伶,朗达有隽才。于时之谈,以阮为首,王戎次之,山、向之徒,皆其伦也。”若如盛言,则非无臧贬。此言谬也。}}

{\cangkai\zihao{5}【评】竹林七贤,虽当时齐名,然知人论世,其人之优劣可得而言;且先前亦有评论先例,故谢安之“先辈初不臧贬七贤”,“亦非公论”(王世懋语)。谢安何以对七贤闭口不谈?当是林下风度已成为魏晋士人追求自由精神的象征符号,虽其间引类不齐,良莠杂陈,人生选择多元化,名士们多能和谐共处,展示了可圈可点的不羁风流。谢安将其视为心摹手追的高标,不容他人妄置雌黄也。又,子侄辈妄论先贤,谢安心有不满,又不便明言,以搁置不谈的方式实行不言之教。这也是谢安作为教育家,一贯重视言传身教的教育风格的体现。}

\lettrine{9.72} 有人以王中郎\myidx{王坦之}比车骑\myidx{谢玄}\footnote{王中郎:王坦之,王承之孙。车骑:谢玄。},车骑闻之曰:“伊窟窟成就\footnote{窟窟:通“矻矻”,勤奋的样子。}。”{\fzxk\zihao{6}\textcolor{red}{\CJKunderwave{续晋阳秋}曰:“坦之雅贵有识量,风格峻整。”}}

{\cangkai\zihao{5}【评】王坦之嫉时俗放荡,以勤勉任事的实干家形象著称于世,谢玄称其“窟窟成就”可谓恰当。谢玄亦有经国才略,平生使才虽屐履之间亦得其任,是从政的天生好材料,淝水之战中立下金石之功,更奠定其一代政治家、军事家的基础和实力。王坦之、谢玄一文韬一武略,一庙堂一疆场,同为晋室安危所系,故时人以其相比。谢玄虽称赏坦之,但味其言外,可能也有夫子自道之意。}

\lettrine{9.73} 谢太傅\myidx{谢安}谓王孝伯\myidx{王恭}\footnote{谢太傅:谢安。王孝伯:王恭字孝伯,王濛孙。}:“刘尹myidx{\}亦奇自知\footnote{刘尹:刘惔。 奇:极,非常。},然不言胜长史\myidx{王濛}\footnote{长史:指王濛。}。”

{\cangkai\zihao{5}【评】刘惔为人矜持太厉,常以风流第一人自许,甚而自比于孔子。一方面天真纯粹得可爱,另一方面又骄狂得令人生厌。他与王濛均善玄言清谈,二人齐名友好,王濛常言“刘君知我,胜我自知”,对刘颇有崇拜之意。此处刘不言胜王濛,可见狂人也有惺惺相惜、珍视友情的清醒一面。}

\lettrine{9.74} 王黄门兄弟三人\myidx{王徽之}\myidx{王操之}\myidx{王献之}俱诣谢公\myidx{谢公}\footnote{王黄门:王徽之,字子猷,仕至黄门侍郎,故称。兄弟三人:指王徽之、王操之、王献之兄弟三人。谢公:谢安。},子猷、子重多说俗事\footnote{子猷:王徽之。子重:王操之,字子重。王羲之第六子。历仕豫章太守。},{\fzxk\zihao{6}\textcolor{red}{\CJKunderwave{王氏谱}曰:“操之字子重,羲之弟(第)六子,历秘书监、侍中、尚书、豫章太守。”}} 子敬寒温而已\footnote{子敬:王献之,字子敬,王羲之第七子。寒温:犹言寒暄。}。既出,坐客问谢公:“向三贤孰愈\footnote{向:刚才。愈:强,优胜。}?”谢公曰:“小者最胜\footnote{小者:此指王献之,他在弟兄中最小。}。”客曰:“何以知之?”谢公曰:“吉人之辞寡,躁人之辞多\footnote{“吉人”两句:语出\CJKunderwave{易传·系辞}。吉人:善人,贤人。躁人:浮躁的人。}。推此知之。”

{\cangkai\zihao{5}【评】崇尚简约,是六朝人物品藻的重要审美标准,其与清通是一而二、二而一的关系。\CJKunderwave{易传·系辞}云:“乾以易知,坤以简能。易则易知,简则易从。易知则有亲,易从则有功。有亲则可久,有功则可大。可久则贤人之德,可大则贤人之业。易简而天下之理得矣。”由博返约,直窥\CJKunderwave{易}理。\CJKunderwave{周易}为“三玄”之一,对中古士人思想观念必然产生深远的影响。谢安评价子敬最胜,就是秉持尚简的审美观念。反之,出语喋喋不休,涉及柴米油盐人间烟火事,则被认为是俗气,少了几分清简的修养。一斑窥豹,晋人心态可闻可见。}

\lettrine{9.75} 谢公\myidx{谢安}问王子敬\myidx{王献之}\footnote{谢公:谢安。王子敬:王献之。}:“君书何如君家尊?”答曰:“固当不同\footnote{固当:当然。}。”公曰:“外人论殊不尔。”王曰:“外人那得知!”{\fzxk\zihao{6}\textcolor{red}{宋明帝\CJKunderwave{文章志}曰:“献之善隶书,变右军法为今体,字画秀媚,妙绝时伦,与父俱得名。其章草疏弱,殊不及父。或讯献之,云:‘羲之书胜不?’莫能判。有问羲之云:‘世论卿书不逮献之。’答曰:‘殊不尔也。’它日见献之,问:‘尊君书何如?’献之不答。又问,‘论者云,君固当不如。’献之笑而答曰:‘人那得知之也。’”}}

{\cangkai\zihao{5}【评】王羲之、献之父子俱妙善书法,\CJKunderwave{晋书}本传称羲之“尤善隶书,为古今之冠,论者称其笔势,以为飘若浮云,矫若惊龙”;谓献之“工草隶,善丹青”。世人或论羲之书法不及献之,或谓献之书法逊于其父,求诸父子二人,亦各不相让。谢安书法曾师右军,亦能入流,故推尊右军,而轻子敬。“君书何如君家尊?”虽是平常之问,实则语含成见,意激子敬自叹弗如。子敬不说自优,又不说自劣,只说“固当不同”,意谓各有特点。谢安以为答案不惬于心,又以“外人论殊不尔”质问。子敬答曰:“外人那得知”,即外行人哪里能了断自家事,指斥谢安为书道外行。子不让父,一竞高低,见出献之的自负。其个性之张扬,生动有趣。又,故事对话看似平淡无奇,实则语含机锋,互不相让,体现了晋人言谈尚简的美学追求。}

\lettrine{9.76} 王孝伯\myidx{王恭}问谢太傅\myidx{谢安}\footnote{王孝伯:王恭,字孝伯。谢太傅:谢安。}:“林公\myidx{支遁}何如长史\myidx{王濛}\footnote{林公:支道林。长史:王濛。}?”太傅曰:“长史韶兴\footnote{韶兴:美好的兴致。}。”问:“何如刘尹\myidx{刘惔}\footnote{刘尹:刘惔。}?”谢曰:“噫,刘尹秀\footnote{秀:秀拔杰出。}。”王曰:“若如公言,并不如此二人邪?”谢云:“身意正尔也\footnote{身:我。}。”

{\cangkai\zihao{5}【评】故事中,王恭之问话直来直去,使人无法避其锋芒;谢安的答语则委婉巧妙,有回味馀地。虽三言两语而二人之性格生动展现:一直率,一持重,令人如面。}

\lettrine{9.77} 人有问太傅\myidx{谢安}\footnote{太傅:谢安。}:“子敬\myidx{王献之}可是先辈谁比\footnote{子敬:王献之。}?”谢曰:“阿敬近撮王\myidx{王濛}、刘\myidx{刘惔}之标\footnote{阿敬:对王献之的昵称。撮:撮取。王、刘:指王濛、刘惔。标:标格,风度。}。”{\fzxk\zihao{6}\textcolor{red}{\CJKunderwave{续晋阳秋}曰:“献之文义并非所长,而能撮其胜会;故擅名一时,为风流之冠也。”}}

{\cangkai\zihao{5}【评】王献之书法蔚为大家、独步百代,又因是琅邪王氏的芝兰玉树,故深得谢安的赏识,比之于王濛、刘惔。王濛通达近于任诞,刘惔矜持近于矫情,王献之更将名士习气演绎得淋漓尽致,夺人眼球,丝毫不亚于王、刘诸君。“近撮王、刘之标”云云,实是将其与王、刘视为一流人物。这大概是名士间以声气相标榜的门面语,只是“大体须有,定体则无”,不必细究可也。}

\lettrine{9.78} 谢公\myidx{谢安}语孝伯\myidx{王恭}\footnote{谢公:谢安。孝伯:王恭。王濛孙。}:“君祖比刘尹\myidx{刘惔},故为得逮\footnote{刘尹:刘惔。故:确实。逮:追及,赶上。}?”孝伯云:“刘尹非不能逮,直不逮\footnote{直:只是。}。”{\fzxk\zihao{6}\textcolor{red}{言濛质而惔文也。}}

{\cangkai\zihao{5}【评】士林中公认刘惔比王濛更为丰姿特出,名士的名头也更加响亮一些。谢安的问话中含有一种不容置疑的倾向色彩,颇能代表社会舆论的定评。名士之孙王恭不甘心祖父居人下风,为其争颜面,不屑之意表露无遗。王世懋曰:“孝伯自私其祖,未为公论,毕竟刘胜王。”其实王世懋本不必过分认真,对祖先的尊崇乃是人之常情,谁都知道有情感倾向在里面,这与理性的考量迥异。王、刘之高下优劣,经历了朝代的变迁、岁月的更迭,评判的价值标准与昔日可能不可同日而语,谁又能说得清楚呢?王质而刘文,时下浮华虚假流行,天真质实之人当更为可敬可爱。}

\lettrine{9.79} 袁彦伯\myidx{袁宏}为吏部郎\footnote{袁彦伯:袁宏,字彦伯。吏部郎:掌官吏选拔的官。},子敬\myidx{王献之}与郗嘉宾\myidx{郗超}书曰\footnote{子敬:王献之。郗嘉宾:郗超。}:“彦伯已入\footnote{已入:谓已进吏部为郎。},殊足顿兴往之气\footnote{殊足:特别能够。顿:摧挫。兴往之气:一往无前的锐气,豪迈之气。}。故知捶挞自难为人\footnote{捶挞自难为人:谓身为吏部郎,一旦犯过,要受鞭挞之辱,使人难堪。捶挞:指杖刑。自:确实。自东汉至魏晋以后,郎官尚不免杖责。},冀小却,当复差耳\footnote{小却:稍后。当复差:谓杖罚或可减免。差:减。王献之写信给郗超,请他关照袁宏。}。”

{\cangkai\zihao{5}【评】魏晋时期重视吏部曹郎的人选,其职位高于诸曹郎。但历史的发展有时候会给后人留下颇耐人寻味的足迹,即如东汉至魏晋,郎官有错要承受杖责之苦,这在今天人权被充分受尊重的社会,真是不可想象。\CJKunderwave{晋书·王濛传}载“为司徒左西属。濛以此职有谴则应受杖,固辞”。可见,吏部郎位虽高而须承担皮开肉绽的风险,做这样的官成本太高,故一般高门名士视为畏途。王献之为袁宏担心,考虑到袁宏若受此耻辱可能难以做人,故写信给表兄弟郗超,表达为袁宏担忧的心情。\CJKunderwave{晋书}本传载宏“性强正亮直,虽被(桓)温礼遇,至于辩论,每不阿屈,故荣任不至”。“兴往”云云,不为虚言,以袁宏豪迈认真的品性,杖责之辱,恐是难以逃脱了!献之之书,见其爱才之心,情真意挚,切合实际,也着实令人感动。}

\lettrine{9.80} 王子猷\myidx{王徽之}、子敬\myidx{王献之}兄弟共赏\CJKunderwave{高士传}人及赞\footnote{王子猷:王徽之。子敬:王献之。\CJKunderwave{高士传}:嵇康著\CJKunderwave{圣贤高士传}的简称。人及赞:指\CJKunderwave{高士传}中所记载之人物以及赞语。},子敬赏“井丹\myidx{井丹}高洁\footnote{井丹高洁:\CJKunderwave{圣贤高士传}中\CJKunderwave{井丹传}后的赞语。井丹,字大春,东汉扶风郿(今属陕西)人。}。”子猷云:“未若‘长卿\myidx{司马相如}慢世\footnote{长卿慢世:\CJKunderwave{圣贤高士传}中\CJKunderwave{司马相如传}后的赞语。司马相如,字长卿,西汉成都(今属四川)人。仕宦不慕高爵,终于家。慢世,轻蔑世俗之事。}。’”{\fzxk\zihao{6}\textcolor{red}{\CJKunderwave{嵇康高士传}曰:“丹字大春,扶风郿人。博学高论,京师为之语曰:‘\CJKunderwave{五经}纷纶井大春,未尝书刺谒一人。’北宫五王更请,莫能致。新阳侯阴就使人要之,不得已而行。侯设麦饭、葱菜,以观其意。丹推子(却)曰:‘以君侯能供美膳,故来相过,何谓如此!’乃出盛馔。侯起,左右进辇,丹笑曰:‘闻桀、纣驾人车,此所谓人车者邪?’侯即未(去)辇。越骑梁松贵震朝廷,请交丹,丹不肯见。后丹得时疾,松自将医视之,病愈。久之,松失大男磊,丹一往吊之。时宾客满廷,丹裘褐不完,入门,坐者皆悚望其颜色。丹四向长揖,前与松语。客主礼毕后,长揖经坐,莫得与语。不肯为吏,径出,后遂隐遁。”其赞曰:“井丹高洁,不慕荣贵;抗节五王,不交非类。显讥辇车,左右失气,披褐长揖,义陵群萃。”“司马相如者,蜀郡成都人,子(字)长卿。初为郎,事景帝。梁孝王来朝,从游说士邹阳等,相如说之,因病免游梁。后过临邛,富人卓王孙女文君新寡,好音,相如以琴心挑之,文君奔之,俱归成都。后启(居)贫,至临邛买酒舍,文君当垆,相如箸犊鼻裈,涤器市中。为人口吃,善属文。仕官(宦)不慕高爵,常托来(疾)不与公卿大事。终于家。”其赞曰:“长卿慢世,越礼自放。犊鼻居市,不耻其状。托疾避官,蔑此卿相。乃赋\CJKunderwave{大人},超然莫尚。”}}

{\cangkai\zihao{5}【评】子猷、子敬兄弟,一赏“长卿慢世”,一慕“井丹高洁”,这与魏晋玄学精神浸润下,士人对隐逸人格推崇追求有关。但是,长卿、井丹之慢世、高洁,有特定的时代背景及身世原因,有其自然的特出之处。子猷兄弟不问过程、方式及其环境,而只取其结果,则仅得皮毛而遗漏了真精神,如邯郸学步,最后造就了一个四不像的怪胎。陈梦槐评曰:“俱有胜气”,指涉模糊,对象不明。用以指井丹、长卿则可,指二王兄弟则不类矣。}

\lettrine{9.81} 有人问袁侍中\myidx{袁恪之}\footnote{袁侍中:袁恪之,字元祖,东晋阳夏(今河南太康)人。安帝义熙初为侍中。},{\fzxk\zihao{6}\textcolor{red}{\CJKunderwave{袁氏谱}曰:“恪之字元祖,陈郡阳夏人。祖王孙,司徒从事中郎。父纶,临汝令。恪之仕黄门侍郎。义熙初,为侍中。”}} 曰:“殷仲堪\myidx{殷仲堪}何如韩康伯\myidx{韩伯}\footnote{殷仲堪:(?—399):善清谈,当时与韩康伯齐名。韩康伯:韩伯,字康伯。}?”答曰:“理义所得,优劣乃复未辨\footnote{理义所得:谓玄理造诣。乃复:竟然。};然门庭萧寂\footnote{门庭萧寂:门前庭院萧索寂静。形容无客人往来。},居然有名士风流,殷不及韩。”故殷作诔云\footnote{诔:叙述死者生平品德以示哀悼之文。此指哀悼韩伯的诔文。}:“荆门昼掩\footnote{荆门:用荆条编的门户,犹言柴门。},闲庭晏然\footnote{晏然:安然平静貌。}。”

{\cangkai\zihao{5}【评】儒家思想重慎独修身,故“独处不愧屋漏,出门如见大宾”;道家则尚简约清心、虚室生白。二家理想相反相成而形成一股合力,作用于中国人的日用起居、出处行藏。殷仲堪作诔文评韩康伯“荆门昼掩,闲庭晏然”,其评价标准当是融合了儒、道二家人格境界,表现了魏晋玄学士人对人伦生活常态理想的追求。“读书仍隐居,弹琴复长啸”,是中国人对“诗意栖居”(西方“存在主义”哲学术语)人生境界的中国化理解和阐释。韩康伯“门庭萧寂,居然有名士风流”,正是此期士人企踵的高标,故袁侍中以为殷不及韩。}

\lettrine{9.82} 王子敬\myidx{王献之}问谢公\myidx{谢安}\footnote{王子敬:王献之。谢公:谢安。}:“嘉宾\myidx{郗超}何如道季\myidx{庾龢}\footnote{嘉宾:郗超。道季:庾龢。}?”答曰:“道季诚复钞撮清悟\footnote{诚复:确实,的确。钞撮:谓汇集众说。清悟:敏捷颖悟。},嘉宾故自上\footnote{故自:确实。肯定语气较强。上:杰出。言其自然超拔。}。”{\fzxk\zihao{6}\textcolor{red}{谓超拔也。}}

{\cangkai\zihao{5}【评】庾龢之谈名理能够汇集众家之长,得其清悟,曾自言,仅在思维的逻辑性和意志力方面逊于韩康伯与王文度,自此以还百倍于人。但摭采群言,不如戛戛独造。故谢安以为自然超拔,郗超当更上一层。谢安与郗超,一维护王室利益,一为桓温伺窥晋鼎出谋划策,二人为政敌。但谢安评郗,并不因人废言,正见其健康心态。}

\lettrine{9.83} 王珣\myidx{王珣}疾,临困\footnote{临困:到病情严重时。},问王武冈\myidx{王谧}曰\footnote{王武冈:王谧,王导孙,王劭子。袭爵武冈侯。}:{\fzxk\zihao{6}\textcolor{red}{\CJKunderwave{中兴书}曰:“谧字雅远,丞相导孙,车骑劭子。有才器,袭爵武冈侯,位至司徒。”}} “世论以我家领军\myidx{王洽}比谁\footnote{我家领军:指王洽,王导子,王珣父。}?”武冈曰:“世以比王北中郎\myidx{王坦之}\footnote{王北中郎:指王坦之,王承之孙。}。”东亭\myidx{王珣}转卧向壁\footnote{东亭:即王珣,珣封东亭侯,故称。},叹曰:“人固不可以无年\footnote{固:的确。无年:寿不长(王洽卒年三十六)。王珣叹其父短寿,名德只能与王坦之相比。}!”{\fzxk\zihao{6}\textcolor{red}{领军,王洽,珣之父也。年二(三)十六卒。珣意以其父名德过坦之而无年,故致此论。}}

{\cangkai\zihao{5}【评】一个人病笃弥留之际的交代,一定是其终生魂牵梦绕、割舍不下的情感郁结。王珣临终时的一问一叹,毫不涉及个人名利,因挂系乃父的士林评价,而更耐人寻味。其言语间流露出的情绪由期待转为哀怨不平,特别是“转卧向壁”一细节,都生动传神地展示出古代士人受家族功业观念、父祖崇拜意识困扰影响的历史特点。无数士人怒发冲冠,甚至是拼死一搏,正是为了捍卫家族的荣誉。家族利益,是中古士人永远走不出的怪圈。}

\lettrine{9.84} 王孝伯\myidx{王恭}道谢公\myidx{谢安}“浓至”\footnote{王孝伯:王恭。谢公:谢安。浓至:谓人性情厚重深沉。}。又曰:“长史\myidx{王濛}虚\footnote{长史:王濛。虚:清虚。},刘尹\myidx{刘惔}秀\footnote{刘尹:刘惔。秀:秀拔杰出。},谢公\myidx{谢公}融\footnote{融:融通。}。”{\fzxk\zihao{6}\textcolor{red}{谓条畅也。}}

{\cangkai\zihao{5}【评】王恭评王、刘,虽各有特出之处,实则意在表彰谢安之兼擅众美。孙绰曾言王濛“温润恬和”、“性和畅”(\CJKunderwave{晋书·王濛传}),即此“虚”之义也。谢安又曾对王恭发“刘尹秀”之辞,此处之言或为转述。“泰山不辞土壤,故能成其大;江河不择细流,故能就其深”,谢安则融汇诸人之长。平素的思想锻炼,终于造就了肚里能撑船又宽宏大量的一代风流名相。此之谓圆融无碍而通于“浓至”之境。}

\lettrine{9.85} 王孝伯\myidx{王恭}问谢公\footnote{王孝伯:王恭。谢公:谢安。}:“林公\myidx{支遁}何如右军\myidx{王羲之}\footnote{林公:支道林。右军:王羲之。}?”谢曰:“右军胜林公。林公在司州\myidx{王胡之}前\footnote{司州:王胡之,官至司州刺史。},亦贵彻\footnote{贵彻:尊贵而通达。}。”{\fzxk\zihao{6}\textcolor{red}{不言若羲之,而言胜胡之。}}

{\cangkai\zihao{5}【评】刘会孟曰:“本书\CJKunderwave{文学}篇中多美林公,而\CJKunderwave{品藻}篇恒抑之,何也?”所问启人深思。但此则谓林公“贵彻”,评价不低。在右军后处司州前,也是切合实际之言,并非一味贬损。言外之意,谢安本人精于玄理,而暗寓超越林公之意。其理论自负可见一斑。}

\lettrine{9.86} 桓玄\myidx{桓玄}为太傅\footnote{桓玄:大司马桓温子,晋安帝时为太傅,掌朝权。太傅:官名。三公之一。},大会,朝臣毕集,坐裁竟\footnote{裁:通“才”,刚刚。},问王桢之\myidx{王桢之}曰\footnote{王桢之:字公幹,小字思道。王徽之子。}:“我何如卿弟(第)七叔\myidx{王献之}\footnote{卿第七叔:此指王献之。}?”{\fzxk\zihao{6}\textcolor{red}{\CJKunderwave{王氏谱}曰:“桢之字公幹,琅邪人,徽之子。历侍中、大司马长史。”弟(第)七叔,献之也。}} 于时宾客为之咽气\footnote{咽气:屏气。言气氛紧张。}。王徐徐答曰:“亡叔是一时之标\footnote{标:楷模。},公是千载之英\footnote{英:英杰。}。”一坐欢然。

{\cangkai\zihao{5}【评】桓玄篡晋之心路人皆知,安帝元兴元年(402年),桓玄兴兵东下,东晋政权尽掌控于桓玄之手。本则所记当在此年攻入建康之时。桓玄一朝权在握,便把令来行,正直士大夫惧怕惨遭杀戮而噤若寒蝉。桓玄在书法艺术史上有一定地位,绝非附庸风雅、沽名钓誉的禄蠹之士,其企慕高标也是情在理中。庾肩吾\CJKunderwave{书品}云:“桓玄、敬道,品在中上。论曰:‘季琰(王珉字)、桓玄,筋力俱骏。’”李嗣真\CJKunderwave{后书品}中品云:“桓公如惊蛇入草,铦锋出匣。”可见,桓玄是书艺里手,他酷爱书法,毕生敬仰二王,常以王献之自比,故有此问。王桢之以其机智言辞逃过了一场劫难,喻玄为“千载之英”,盖过了献之的“一时之标”,但就其内容分析,对桓玄这一野心家未免有曲媚阿顺之嫌。类似之事,如桓玄问谢道韫以谢安出处问题,道韫答曰:“亡叔……以无用为心,显隐为优劣,始末正当动静之异耳。”(见\CJKunderwave{排调}门第26则刘注引\CJKunderwave{妇人集})桓玄篡晋,对王谢家族虎视眈眈,但道韫为女中人杰,回答不卑不亢,远胜于王家子弟。故凌濛初评此曰:“真是怕他。”道出了王桢之的真实心理,令人读之心伤。}

\lettrine{9.87} 桓玄\myidx{桓玄}问刘太常\myidx{刘瑾}曰\footnote{刘太常:刘瑾,字仲璋,东晋南阳(今属河南)人。官至太常卿。}:“我何如谢太傅\myidx{谢安}\footnote{谢太傅:谢安。}?”{\fzxk\zihao{6}\textcolor{red}{\CJKunderwave{刘瑾集叙}曰:“瑾字仲璋,南阳人。祖遐,父畅。畅娶王羲之女,生瑾。瑾有才力,历尚书、太常卿。”}} 刘答曰:“公高,太傅深。”又曰:“何如贤舅子敬\myidx{王献之}\footnote{贤舅:尊称别人之舅。子敬:王献之。刘瑾母为羲之女,故瑾称子敬为舅。}?”答曰:“楂梨橘柚,各有其美\footnote{“楂梨”两句:谓各种水果滋味不同,比喻人物各有美好之处。\CJKunderwave{庄子·天运}:“故譬三皇五帝之礼义法度,其犹楂梨橘柚邪?其味相反而皆可于口。”}。”{\fzxk\zihao{6}\textcolor{red}{\CJKunderwave{庄子}曰:“楂梨橘柚,其味相反,皆可于口也。”}}

{\cangkai\zihao{5}【评】桓玄灭杨佺期、殷仲堪及司马元显后,掌握朝廷军政大权,又自封为太尉,有禅晋之意。他在篡位前后,对王谢家族是虎视眈眈的。谢安和王献之虽然早已谢世,并不对其构成威胁,但因二人为王谢家族的精英,而被视为放矢之的。桓玄此问,有指鹿为马、投石问路之意,借以显示其权威。刘瑾回答,虽从容机智,貌似公允无偏,凌濛初评曰“最好答法”,实则已被桓玄的淫威慑服。故事折射出六朝知识分子在更代之际较为常见的自保心理——汉末以降,篡弑频仍,士夫之心,如履霜临渊,被政治强盗拿捏得元气大伤、斯文扫地,与夫嵇康“宁为玉碎,不为瓦全”的高尚情怀不可同日而语。}

\lettrine{9.88} 旧以桓谦\myidx{桓谦}比殷仲文\myidx{殷仲文}\footnote{旧:原来。桓谦:字敬祖。桓冲子,桓玄从兄。殷仲文:桓玄姐夫。}。{\fzxk\zihao{6}\textcolor{red}{\CJKunderwave{中兴书}曰:“谦字敬祖,冲弟(第)三子。尚书仆射、中军将军。”\CJKunderwave{晋安帝纪}曰:“仲文有器貌才思。”}} 桓玄时\footnote{桓玄时:晋安帝时,桓玄为太傅,当权掌朝政。},仲文入,桓于庭中望见之\footnote{庭:厅堂。},谓同坐曰:“我家中军,那得及此也\footnote{我家中军:指桓谦。那得:怎么能够。}!”

{\cangkai\zihao{5}【评】殷仲文是属于典型的有气韵而无气概、有灵气而无灵魂,“才子加流氓”型的文人。自命才高的谢灵运曾推重曰“若殷仲文读书半袁豹,则文才不减班固”,足见其天赋之高。但此人乃是政治投机家。史载桓玄篡晋入宫,其床忽陷,群下失色,殷仲文曰:“将由圣德深厚,地不能载”(\CJKunderwave{晋书}本传),以机智乖巧的谀媚之辞,替桓玄摆脱了尴尬。殷仲文不吝惜自己的名誉与尊严,虽为士林所不齿,但往往因善于揣摩、迎合主上的意图而受宠一时。桓玄赞“我家中军,那得及此也”。史载仲文“以佐命亲贵,厚自封崇,舆马器服,穷极绮丽,后房伎妾数十,丝竹不绝音”(\CJKunderwave{晋书}本传)。但他最后,因叛而亡,实是为桓玄作陪葬。一个毫无骨气的文人,其悲剧命运自也是顺理成章的了。}





%%% Local Variables:
%%% mode: latex
%%% TeX-engine: xetex
%%% TeX-master: "../Main"
%%% End:
