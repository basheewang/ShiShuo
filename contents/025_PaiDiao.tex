%% -*- coding: utf-8 -*-
%% Time-stamp: <Chen Wang: 2025-12-08 20:13:18>

% ○ ◎ ‧ 「 」 『 』 々 ( ) “ ” ■ ^[一-龥]
% 【\([^】][^】][^】]+\)】 → {\\fzxk\\zihao{6}\\textcolor{red}{\1}}
% \(【评】.*\) → {\\cangkai\\zihao{5}\1}
% \(【题解】.*\) → {\\cangkai\\zihao{5}\1}
% 《\([^》]+\)》 → \\CJKunderwave{\1}
% ^\([0-9]+.[0-9]+\) → \\lettrine{\1}
% {\\fzxk\\zihao{6}\\textcolor{red}{[^o]*}}

\setlength{\parindent}{0pt}


\chapter{排调第二十五}




{\cangkai\zihao{5}【题解】 排者,俳也,滑稽也。古代俳优戏说,大多充满机智的笑声。调者,调侃戏谑也,在嘲讽中见其人生智慧。“排调”合称,则是嘲笑戏谑的意思。本门故事共65则,分量不轻,情节多带戏剧性成分,对话斗嘴,尤见精神,活用经典而点铁成金,巧于修辞而机锋四起,寓情于理而才华横溢,在幽默风趣的笑声中,凸显出魏晋士人那机敏睿智的喜剧性格。人生并不永远都是刀光血影,也不是整天都在愤世嫉俗。当时名士同时善于调整自己的生活方式,遇悲则啼,该喜则笑,悲喜剧的人生并存于一身,不仅有狂傲不拘、放荡失检的一面,同时也有笑口常开、可喜可爱的另一面,合而观之,才见魏晋士人自然人生之全面。\CJKunderwave{排调}门诸多故事,犹如大摆文学擂台,如张万起、刘尚慈所评:“看似在嘲戏,实则在斗智慧、斗才学、斗捷悟、斗思辨、斗哲理,读来兴味盎然。”其引人入胜的艺术魅力,实在令人击赏称绝。}

\lettrine{25.1} 诸葛瑾\myidx{诸葛瑾}为豫州\footnote{诸葛瑾:字子瑜,三国时琅邪阳都人。诸葛亮兄。仕吴官至大将军、豫州牧。参前\CJKunderwave{品藻}第4则注。为豫州:任豫州牧。豫州治所谯(今安徽亳州)。},遣别驾到台\footnote{别驾:州府佐官。到台:赴朝请示汇报。台,朝廷台省。},{\fzxk\zihao{6}\textcolor{red}{瑾,已见。}} 语云:“小儿知谈\footnote{知谈:健谈,善议论。},卿可与语。”连往诣恪\myidx{诸葛恪}\footnote{连往诣恪:接连拜访诸葛恪。恪字元逊,瑾长子。仕吴至大将军,封阳都侯。后被孙峻所害。},{\fzxk\zihao{6}\textcolor{red}{\CJKunderwave{江表传}曰:“恪字元逊,瑾长子也。少有才名,发藻岐嶷,辩论应机,莫与为对。孙权见而奇之,谓瑾曰:‘蓝田生玉,真不虚也。’仕吴至太傅。为孙峻所害。”}} 恪不与相见。后于张辅吴\myidx{张昭}坐中相遇\footnote{张辅吴:指张昭,字子布,仕吴为辅吴将军,故称。},{\fzxk\zihao{6}\textcolor{red}{环济\CJKunderwave{吴纪}曰:“张昭,字子布。忠正有才义,仕吴,为辅吴将军。”}}别驾唤恪:“咄咄郎君\footnote{咄咄:象声口语,犹今之“喂喂”。郎君:汉官二千石以上者荫子为郎,后来因称师长或长官之子为郎君,犹言今之公子。}!”恪因嘲之,曰:“豫州乱矣,何咄咄之有?”答曰:“君明臣贤,未闻其乱。”恪曰:“昔唐尧在上\footnote{唐尧:传说中的上古帝王,儒家以为圣人。},四凶在下\footnote{四凶:指传说中尧、舜时代的四大恶人共工、驩兜、三苗、鲧。一说四凶为浑敦、穷奇、梼杌、饕餮。}。”答曰:“非唯四凶,亦有丹朱\footnote{丹朱:尧的不孝子,故尧传位于舜。}。”于是一坐大笑。

{\cangkai\zihao{5}【评】故事中的诸葛恪,并非等闲之辈,而是三国东吴的著名人物。其辩论才捷,连吴主孙权也很赞赏。因其才华出众,加以贵游子弟,所以对父亲的部下态度傲慢,拒而不见,偶然相遇,又出言不逊,首先挑起口舌之“战”。但是,别驾官职虽低,态度却不亢不卑,不仅维护了做人的尊严,而且以其治人之道,反治其人之身,用武侠小说的话说,是借力打力。二人皆用典故作生动的修辞譬喻:“四凶在下”,是挑衅;“亦有丹朱”,是反击。一来一往,并非真有恶意,但却令人忍俊不禁,几乎成了学问与机智的展示。}

\lettrine{25.2} 晋文帝\myidx{司马昭}与二陈\myidx{陈骞}\myidx{陈泰}共车\footnote{晋文帝:指司马昭。昭生前为晋王,死后谥号文王。其子炎篡位建晋,即追尊为文帝。二陈:指陈骞和陈泰。骞字休渊,官至大司马。泰字玄伯,官至侍中、左仆射。},过唤锺会\myidx{锺会}同载\footnote{过唤:路过呼唤。锺会:字士季。太傅锺繇子。参前注。},即驶车委去\footnote{驶:疾驶。委去:丢下。}。比出\footnote{比:等到。},已远。既至,因嘲之曰:“与人期行\footnote{期:约定。},何以迟迟?望卿遥遥不至\footnote{遥遥:长远。按:会父繇,遥、繇同音。}。”会答曰:“矫然懿实,何必同群\footnote{矫然懿实,何必同群:矫然高举而美好充实,又何必与你们同群呢?}!”帝复问会:“皋繇何如人\footnote{皋繇:传说中舜臣,掌刑狱司法。}?”答曰:“上不及尧、舜\footnote{不及:不如。},下不逮周、孔\footnote{逮:及。不逮,比不上。},亦一时之懿士\footnote{懿士:美德之士。}。”{\fzxk\zihao{6}\textcolor{red}{二陈,骞与泰也。会父名繇,故以“遥遥”戏之。骞父矫,宣帝讳懿,泰父群,祖父寔,故以此酬之。}}

{\cangkai\zihao{5}【评】这是一场语言艺术的游戏。游戏双方的社会地位并不平等:“晋文帝”一方是君,另一方的会是臣。古时君臣之间,是主子与奴仆的关系,是指挥与服从的关系。但君臣双方一旦进入了游戏圈中,君凌下而臣犯上,则又另当别论,其游戏规则又似乎是平等的。君主主动与臣下平等游戏,在提倡“以孝治国”的司马昭身上,直犯臣下家讳,不拘礼教而用以取乐,不过是为自己的生活增添一点色彩和乐趣。锺会颇富才情,在游戏中寸步不让,两军相对,擒贼擒王。锺会针锋相对地直犯君主的家讳。昭父名懿,“矫然懿实”、“何必同群”、“一时懿士”,再次犯上,同时顺便把二陈家讳捎带上。其巧用修辞,实矫然不群而可称懿士。司马昭当时容忍了锺会言语的尖酸刻薄,这只有在魏晋社会中可以见到,宋明以后则不可能出现。}

\lettrine{25.3} 锺毓\myidx{锺毓}为黄门郎\footnote{锺毓:字稚叔。官至廷尉、青州刺史。繇子,会兄。黄门郎:官名,管侍从皇帝,传达诏命。},有讥(机)警\footnote{讥警:王先谦刻本作“机警”,是。机警,机敏警觉。},在景王\myidx{司马师}坐燕饮\footnote{景王:指司马师。懿子,昭兄。晋建,追尊景王或景帝。坐:通“座”。燕:通“宴”。}。时陈群\myidx{陈群}子玄伯\myidx{陈泰}、武周\myidx{武周}子元夏\myidx{武陔}同在坐\footnote{陈群:字长文,魏司空。玄伯:陈泰字玄伯,群子,官至尚书右仆射。武周:字伯南,沛国竹邑(今安徽宿州)人。魏光禄大夫。元夏:武陔字。官至左仆射。},{\fzxk\zihao{6}\textcolor{red}{\CJKunderwave{魏志}曰:“武周字伯南,沛国竹邑人。仕至光禄大夫。”}} 共嘲毓。景王曰:“皋繇何如人\footnote{皋繇:参前则注。按:毓父名繇,犯其名讳。}?”对曰:“古之懿士\footnote{懿士:美德之士。按:师父名懿,犯其名讳。}。”顾谓玄伯、元夏,曰:“君子周而不比,群而不党\footnote{君子周而不比,群而不党:意谓君子团结而不结党营私。语出\CJKunderwave{论语·为政}:“君子周而不比,小人比而不周。”又见\CJKunderwave{论语·卫灵公}:“君子……群而不党。”按:“周”、“群”之字,故犯武周、陈群名讳。}。”{\fzxk\zihao{6}\textcolor{red}{孔安国注\CJKunderwave{论语}曰:“忠信为周,阿党为比。党,助也。君子虽众,不相私助。”}}

{\cangkai\zihao{5}【评】此与前则,是为兄弟篇。景王司马师为文王司马昭之兄,锺毓为锺会之兄。兄弟主角换人,但游戏规则未变。这可能原是一事而传闻有异,也可能是此类玩笑之事,当时较为普遍,故广泛传播而成为美谈。但二者比较而言,锺毓之言,虽不如锺会尖刻而咄咄逼人,但横扫陈泰、武陔之时,引用\CJKunderwave{论语}之典,却又在典雅思理中见其批评嘲讽之意,辞义贴切而意味悠长隽永。毓不同于会的个性于此可见。}

\lettrine{25.4} 嵇\myidx{嵇康}、阮\myidx{阮籍}、山\myidx{山涛}、刘\myidx{刘伶}在竹林酣饮\footnote{嵇、阮、山、刘:指竹林七贤中的嵇康、阮籍、山涛、刘伶。参前注。酣饮:畅饮。},王戎\myidx{王戎}后往\footnote{王戎:字濬冲,也是竹林七贤之一。晋时官至司徒,封安丰侯。},步兵曰\footnote{步兵:指阮籍,曾任步兵校尉。}:“俗物已复来败人意\footnote{俗物:俗人。已复:又来。意:意兴。}!”{\fzxk\zihao{6}\textcolor{red}{\CJKunderwave{魏氏春秋}曰:“时谓王戎未能超俗也。”}} 王笑曰:“卿辈意亦复可败邪\footnote{卿辈:你们这些人。}?”

{\cangkai\zihao{5}【评】在竹林七贤中,王戎年纪最轻,小阮籍二十四岁。但其对话神态,并没有面对前辈而自降一等的感觉。前面锺氏兄弟引经据典的机智,在王戎则没有那么典雅,而是巧妙动用逻辑推理,借人之力,反攻过去。既然嵇、阮辈称人俗物,则是自视超越世俗之高人。脱俗高人之意兴,不与俗世相干,岂是俗人可以败坏?既然我王戎可以败坏你们的意兴,则嵇、阮和我王戎是站在同一平台之上,我为俗物,君为何人?不过是同饮之俗人而已。王戎善清言,明玄理,颇有哲学修养,也自有高明之处。}

\lettrine{25.5} 晋武帝\myidx{司马炎}问孙晧\myidx{孙晧}\footnote{晋武帝:即司马炎,字安世。昭长子。西晋开国之主。孙晧:三国吴主,太康元年(280)降晋。},{\fzxk\zihao{6}\textcolor{red}{\CJKunderwave{吴录}曰:“晧字元宗,一名彭祖,大皇帝孙也。景帝崩,晧嗣位,为晋所灭,封归命侯。”}} “闻南人好作\CJKunderwave{尔汝歌}\footnote{南人:指当时东吴之人。\CJKunderwave{尔汝歌}:魏晋时南方民歌。},颇能为不\footnote{颇:疑问副词。不:否。}?”晧正饮酒,因举觞劝帝而言曰\footnote{举觞:举杯。劝:劝酒。}:“昔与汝为邻,今与汝为臣。上汝一杯酒,令汝寿万春\footnote{令汝寿万春:祝汝万岁。万春,万年。}。”帝悔之。

{\cangkai\zihao{5}【评】司马炎与孙晧,原为北、南敌国二主。转眼之间,一为胜利者,得意扬扬;一为阶下囚,威风尽丧。酒宴之上,炎以龙兴之主,君临亡国奴上,犹如猫玩老鼠,一切尽在掌握之中,故戏而不杀。命晧作\CJKunderwave{尔汝歌},即为此戏。“颇能为不?”是疑问句,盖不相信眼前这个孙晧,昔日有剥人皮、剜人眼之酷行,今日能有即席作歌之才,故借此侮之。但孙晧却因“戏”而暗中反击。“昔与汝为邻,今与汝为臣”,是现实的概括;“上汝一杯酒,令汝万寿春”,祝酒嘏辞,即使贵为帝王,又有谁真能万岁呢?以今日之我,提醒胜利者作明日之思考,暗寓得意一世,又岂是太平万年之理。“尔”、“汝”,第二指代人称,原是古代尊长对卑幼晚辈的称呼,平辈所用,则或示亲狎,或示轻贱。但依\CJKunderwave{尔汝歌}之体制,又需“尔”、“汝”之言以足其歌。司马炎本想借故辱之,但晧却当场作歌,冲口而成,一句一“汝”,令晋主成“汝”之指代,这样晋主反而被嘲讽,想不到侮人反受侮。晧本昏君,昏君也有酒醒之时,但却悔之已晚,呜呼哀哉!}

\lettrine{25.6} 孙子荆\myidx{孙楚}年少时欲隐\footnote{孙子荆:孙楚,字子荆,西晋太原中都人。善著文,官至冯翊太守。参前\CJKunderwave{言语}第24则注。隐:归隐。},语王武子“\myidx{王济}当枕石漱流\footnote{枕石漱流:意喻山林隐居生活,头枕山石而口漱泉流。}”,误曰“漱石枕流”。王曰:“流可枕,石可漱乎?”孙曰:“所以枕流,欲洗其耳\footnote{洗耳:传说尧让天下于许由,由因其功利而洗耳除俗。};{\fzxk\zihao{6}\textcolor{red}{\CJKunderwave{逸士传}曰:“许由为尧所让,其友巢父责之,由乃过清泠水洗耳拭目。曰:‘向闻贪言,负吾之友。’”}} 所以漱石,欲砺其齿\footnote{砺:砥砺,磨炼。}。”

{\cangkai\zihao{5}【评】孙楚名士,颇擅言谈。以“枕石漱流”形容归隐山林生活,乃是直叙,合于修辞手段,石可枕而流可漱,二个动宾词语搭配得当,成一妥帖的四言词组。却因一时口误为“漱石枕流”,原是可以理解之事。但楚所遇者王济,于晋初亦称才子,其机敏不亚于楚,因而迅速抉其语病追问,意欲逼其认错。但故事之妙,在于楚之将错就错,借汉语语序颠倒,来改变语意境界:枕流洗耳,用事典雅,志趣高洁而无俗世之心;漱石砺齿,有助消化吸收,则可接纳虚空万物。其超凡脱俗之心,合于“欲隐”之旨。纵然王济再高傲,于此能不佩服其词锋机智吗?王世懋评曰:“误语乃佳,遂为口实,此王子敬画蝇也。”所评贴切。不过,“漱石枕流”之言,并非孙楚首创,据康僧会\CJKunderwave{法镜经序}云:“或有隐处山泽,漱石枕流,专心涤垢,神与道俱。”康僧会是三国时高僧,吴赤乌十年(247)孙权为之建寺塔。楚本才学之士,可能熟知康僧会之言,但不说破而“点铁成金”,终成千古文坛佳话。}

\lettrine{25.7} 头责秦子羽云\footnote{头责秦子羽:见于刘注\CJKunderwave{张敏集},是一篇讽刺性极强的俳谐文。秦子羽,不详,疑是文学想象的虚构人物。此则描写了秦子羽的头责备秦子羽安处陋巷,不思求取功名。}:{\fzxk\zihao{6}\textcolor{red}{子羽,未详。}} 子曾不如太原温颙\myidx{温颙}\footnote{曾:竟然。温颙:字长仁,太原人。史上与任恺、庾纯、张华、和峤友善,与贾充一党不和。},颍川荀寓\myidx{荀寓},{\fzxk\zihao{6}\textcolor{red}{温颙,已见。\CJKunderwave{荀氏谱}曰:“寓字景伯,祖式(彧),大尉。父保(俣),御史中丞。”\CJKunderwave{世语}曰:“寓少与裴楷、王戎、杜默俱有名,仕晋至尚书。”}} 范阳张华\myidx{张华}\footnote{张华:字茂先。官至司空。八王乱中遇害。},士卿刘许\myidx{刘许},{\fzxk\zihao{6}\textcolor{red}{\CJKunderwave{晋百官名}曰:“刘许,字文生,涿鹿郡人。父放,魏骠骑将军。许,惠帝时为宗正卿。”按:许与张华同范阳人,故曰士卿,互其辞也。宗正卿或曰士卿。}} 义阳邹湛\myidx{邹湛},河南郑诩\myidx{郑诩}。{\fzxk\zihao{6}\textcolor{red}{\CJKunderwave{晋诸公赞}曰:“湛字润甫,新野人。以文义达,仕至侍中。诩字思渊,荥阳开封人,为卫尉卿。祖泰,扬州刺史。父褒(袤),司空。”}} 此数子者,或謇吃无宫商\footnote{謇吃:说话口吃。宫商:原指宫、商、角、徵、羽五音,比喻语言表达的流畅。},或尫陋希言语\footnote{尫(wānɡ汪):突胸丑陋。},或淹伊多姿态\footnote{淹伊多姿态:矫揉作态以媚俗。},或讙哗少智谞\footnote{讙哗:喧哗吵闹。智谞:才智,智谋。},或口如含胶饴\footnote{胶饴:蜜糖。},或头如巾齑杵\footnote{巾齑杵:包着头巾的捣齑杵。},{\fzxk\zihao{6}\textcolor{red}{\CJKunderwave{文士传}曰:“华为人少威仪,多姿态。”推意此语,则此六句,还以目上六人。而“口如含胶饴”,则指邹湛,湛辩丽英博,而有比(此)称,未详。}} 而犹以文采可观,意思详序\footnote{意思:思想意图。详序:表达周详有序。},攀龙附凤\footnote{攀龙附凤:依附权贵。},并登天府\footnote{天府:指朝廷。}。{\fzxk\zihao{6}\textcolor{red}{\CJKunderwave{张敏集}载\CJKunderwave{头责子羽}文曰:“余友有秦生者,虽有姊夫之尊,少而狎焉。同时好昵,有太原温长仁颙,颍川荀景伯寓,范阳张茂先华,士卿刘文生许,南阳邹润甫湛,河南郑思渊诩。数年之中,继踵登朝,而此贤身处陋巷,屡沽而无善价,亢志自若,终不衰堕,为之慨然。又怪诸贤既已在位,曾无\CJKunderwave{伐木}嘤鸣之声,甚违王、贡弹冠之义。故因秦生容貌之盛,为头责之文以戏之,并以嘲六子焉。虽以谐谑,实有兴也。”其文曰:“维泰始元年,头责子羽曰:‘吾托子为头,万有馀日矣。大块禀我以精,造我以形。我为子植发肤,置鼻耳,安眉须,插牙齿。眸子摛光,双颧隆起。每至出入之间,遨游市里,行者辟易,坐者竦跽。或称君侯,或言将军,捧手倾侧,伫立崎岖。如此者,故我形之足伟也。子冠冕不戴,金银不佩,钗以当笄,帢以代帼。旨味弗尝,食粟茹莱,隈摧园间,粪壤汙黑。岁莫年过,曾不自悔。子厌我于形容,我贱子乎意态。若此者乎,必子行己之累也。子遇我如雠,我视子如仇。居常不乐,两者俱忧,何其鄙哉!子欲为人宝也,则当如皋陶、后稷、巫咸、伊陟,保乂王家,永见封殖。子欲为名高也,则当如许由、子臧、卞随、务光,洗耳逃禄,千岁㳅(流)芳。子欲为游说也,则当如陈轸、蒯通、陆生、邓公,转祸为福,令辞从容。子欲为进趣也,则当如贾生之求试,终军之请使,砥砺锋颖,以干王事。子欲为恬淡也,则当如老聃之守一,庄周之自逸,廓然离欲,志陵云日。子欲为隐遁也,则当如荣期之带索,渔父之瀺灂,栖迟神丘,垂饵巨壑。此一介之所以显身成名者也。今子上不希道德,中不交儒墨,块然穷贱,守此愚惑。察子之情,观子之志,退不为于处士,进无望于三事,而徒玩日劳形,习为常人之所喜,不亦过乎!’于是子羽愀然深念而对曰:‘凡所教敕,谨闻命矣。以受性拘系,不闲礼义。设以天幸,为子所寄,今欲使吾为忠也,即当如伍胥、屈平。欲使吾为信也,则当杀身以成名。欲使吾为介节邪,则当赴水火以全贞。此曰(四)者,人之所忌,故吾不敢造意。’头曰:‘子所谓天州(刑)地网,刚德之尤。不登山抱木,则褰裳赴㳅(流)。吾欲告尔以养性,诲尔以优游,而以虮虱同情,不听我谋,悲哉!俱寓人体,而独为子头。且拟人其伦,喻子侪偶:子不如大(太)原温颙,颍川荀寓,范阳张华,士卿刘许,南阳邹湛,河南郑诩。此数子者,或謇吃无宫商,或尫陋希言语,或淹伊多姿态,或讙哗少智谞,或口如含胶饴,或头如巾齑杵。而犹文采可观,意思详序,攀龙附凤,并登天府。夫舐痔得车,沉渊得珠,岂若夫子,徒令唇舌腐烂,手足沾濡哉!居有事之世,而耻为权图,譬犹凿地抱瓮,难以求富。嗟乎子羽,何异槛中之熊,深阱之虎,石间饥蟹,窦中之鼠。事力虽勤,见功甚苦,宜其拳局煎蹙,至老无所希也。支离其形,犹能不困,非命也夫?岂与夫子同处也!’”}}\footnote{张敏,西晋太原中都(今山西平遥西)人。官平南参军、济北长史、领秘书监、益州刺史。}

{\cangkai\zihao{5}【评】头责子羽,寓言之体,俳谐之文,以羽头责羽身,数落身之无能,实则讽刺时政,寄寓精深,给人以启迪。看来作者张敏,是熟谙当日政坛腐败之人。謇吃丑陋或哗众取宠之徒,通过无耻手段,攀龙附凤而直登天府。反之,贤明耿直之士的遭遇和命运,则可想而知。其指画对象六人,具体当否,姑且勿论,其性质不过举隅而已,实则其讽刺的匕首,是投向了官场腐败和社会黑暗。}

\lettrine{25.8} 王浑\myidx{王浑}与妇锺氏\myidx{锺琰}共坐\footnote{王浑:字玄冲。昶子。太原晋阳人。晋初名臣,官至司徒。参前\CJKunderwave{贤媛}第12则注。锺氏:锺琰,太傅繇孙女。参前\CJKunderwave{贤媛}第12则注。},见武子\myidx{王济}从庭过\footnote{武子:王济字武子,浑子。参前\CJKunderwave{言语}第24则注。},浑欣然谓妇曰:“生儿如此,足慰人意\footnote{足慰人意:意谓心满意足。意,心意。}。”妇笑曰:“若使新妇得配参军\myidx{王沦}\footnote{参军:王沦,字太冲,浑弟。曾任大将军参军,故称。},生儿故可不啻如此\footnote{故:本来。不啻:不仅,不只。}。”{\fzxk\zihao{6}\textcolor{red}{\CJKunderwave{王氏家谱}曰:“伦(沦)字太冲,司空穆侯中子,司徒浑弟也。醇粹简远,贵\CJKunderwave{老}、\CJKunderwave{庄}之学,用心淡如也。为\CJKunderwave{老子例略}、\CJKunderwave{周纪}。年二十馀,举孝廉,不行。历大将军参军。二十五卒。大将军为之驩㳅(流)涕。”}}

{\cangkai\zihao{5}【评】魏晋士人的家庭生活,夫妻之间,也有充满玩笑逗趣的一面,颇有生活气息。锺琰之子王济,已能趋庭而过,而非襁褓之儿,说明琰结婚生子有年。其称“新妇”,盖非今天的新娘,而是魏晋已婚妇女的自称。作为年轻妇女,琰之可爱,在于胸怀坦荡,怎么想就怎么说,毫无虚情矫饰。当着丈夫的面,公开欣羡小叔子,致讥后世,如王世懋评曰:“此岂妇人所宜言!宁不启疑,恐贤媛不宜有此。”这是以明代士人的道德眼光来看问题。实际上,王浑夫妇之间,照样其乐融融,并不因妻子的小小玩笑而生气猜疑。这就是魏晋时代贵族妇女生活较为自由解放的结晶。锺琰名登\CJKunderwave{晋书·列女传},号称贤媛,并不奇怪。}

\lettrine{25.9} 荀鸣鹤\myidx{荀隐}、陆士龙\myidx{陆云}二人未相识\footnote{荀鸣鹤:荀隐字鸣鹤。西晋颍川人。官至司徒掾。参注。陆士龙:陆云字士龙,吴郡华亭人。祖逊,三国时吴丞相。官清河内史。西晋著名文学家,与兄机并称二陆。},俱会张茂先\myidx{张华}坐\footnote{张茂先:张华字茂先。建平吴策。官至司空。华赏识二陆,以为平吴之利,在获此二俊。}。张令共语,以其并有大才,可勿作常语。陆举手曰:“云间陆士龙\footnote{云间:华亭古称。}。”荀答曰:“日下荀鸣鹤\footnote{日下:指京师。}。”陆曰:“既开青云睹白雉\footnote{白雉:白色野鸡。},何不张尔弓,布尔矢?”荀答曰:“本谓云龙骙骙\footnote{骙骙 :马强壮貌。},定是山鹿野麋\footnote{麋:麋鹿,又名“四不像”。定:竟然,却。},兽弱弩强,是以发迟。”张乃抚掌大笑\footnote{抚掌:拍手。}。{\fzxk\zihao{6}\textcolor{red}{\CJKunderwave{晋百官名}曰:“荀隐字鸣鹤,颍川人。”\CJKunderwave{荀氏家传}曰:“隐祖昕,乐安太守。父岳,中书郎。隐与陆云在张华坐语,互相反覆,陆连受屈。隐辞皆美丽,张公称善。云世有此书,寻之未得。历太子舍人、廷尉平,蚤卒。”}}

{\cangkai\zihao{5}【评】故事中的张华、荀隐、陆云,不仅是政治家,更是西晋的一代才子、著名文学家。张华作为主人,因荀、陆二人“并有大才”,故劝令相见时“勿作常语”——也就是要求摆脱平时世俗的应酬话,而各显其语言艺术天才。陆云南方之士,荀隐中原士族,均为南北精英。“云间陆士龙”,云中之龙,自由翱翔,不仅与陆云名字相称,而且显示了一往无前的恢宏气魄。“日下荀鸣鹤”,“荀”字从“日”,日下鸣鹤,与荀隐名字贴切。“鸣鹤”之称,又来自\CJKunderwave{诗经·鹤鸣}之诗,有“鹤鸣于九皋,声闻于天”之句。在中国古代,鸣鹤已成为高雅贤人君子的文化象征。一开场,“云龙”与“鸣鹤”各显神通。但后来以白雉嘲讽荀氏非鹤,以山鹿野麋嘲弄陆氏非龙,则在修辞暗藏玄机,此一玩笑已颇见南北士族相互轻诋的潜意识。故张华站在中间立场,折衷于南北之间,“抚掌大笑”,在赏识的笑声中力图消弭南北之士的对抗情绪,表现了团结的诚意。}

\lettrine{25.10} 陆太尉\myidx{陆玩}诣王丞相\myidx{王导}\footnote{陆太尉:指陆玩(注谓陆琬,同音之讹),字士瑶。官至司空,卒赠太尉。诣:拜访,到……去。王丞相:王导。},{\fzxk\zihao{6}\textcolor{red}{陆琬(玩),已见。}}王公食以酪\footnote{王公:王导。酪:奶酪。}。陆还,遂病。明日与王笺云:“昨食酪小过\footnote{小过:稍微过分,稍多一点。},通夜委顿\footnote{委顿:精神困顿。}。民虽吴人\footnote{民:对长官用以自称,表示谦卑。},几为伧鬼\footnote{伧鬼:南人称北方人为伧。北人喜食酪,王导北人,故以此嘲之。}。”

{\cangkai\zihao{5}【评】陆玩是吴郡陆机的从弟,在江东士族中颇有影响。司马南渡之后,王导辅政,为了争取江东士民的广泛支持和拥护,有必要采取“统战”手段笼络江东士族及其代表。琅邪王家,东晋贵族首屈一指,由王导来请陆玩“食酪”,就是一种友好的表示。但是陆玩其人,偏不识相。大概二陆兄弟惨死之初,他对中原士族态度是半信半疑,不敢不信,也不敢全信。王导请他吃饭,存心修好;陆玩食酪过量,肚子不消化而难过,是个人身体素质问题,并非故意作弄他。但他却说自己“几为伧鬼”——险些做了北方鬼。在玩世不恭的调侃中,隐约可见南北士人对立的潜意识。这对国家的团结产生了一定的影响,所以王导要花大力气来加以弥合。}

\lettrine{25.11} 元帝\myidx{司马睿}皇子生\footnote{元帝:指司马睿。西晋末为安东将军,镇建康。愍帝死,王导等拥立,是东晋开国之君。},普赐群臣。殷洪乔\myidx{殷羡}谢曰\footnote{殷洪乔:殷羡,字洪乔。参前\CJKunderwave{任诞}第31则注。}{\fzxk\zihao{6}\textcolor{red}{殷羡,已见。}}:“皇子诞育,普天同庆。臣无勋焉\footnote{勋:功勋,功劳。},而猥颁厚赉\footnote{猥:谦词,犹如“辱”,表示委屈对方。赉:赏赐。}。”中宗笑曰\footnote{中宗:司马睿死后的庙号。}:“此事岂可使卿有勋邪!”

{\cangkai\zihao{5}【评】皇子诞生,普天同庆,群臣上表称贺,古时正常之事。但细读殷辞,“臣无勋焉”,确实措辞不当。生子原是夫妻男女私事,无容第三者插足其间,“勋”当何来?但是,作为一国君主,当众以此开玩笑取乐,古来少见。这说明魏晋皇室与世家豪族共执国政,在特定的阶段,必须君臣团结,挽狂澜于既倒。这时,君臣关系较为融洽,为朝廷的政治生活增添了一些人性的欢乐色彩。}

\lettrine{25.12} 诸葛令\myidx{诸葛恢}、王丞相\myidx{王导}共争姓族先后\footnote{诸葛令:指诸葛恢,字道明,琅邪人。南渡后官至尚书令,故称。王丞相:王导。},王曰:“何不言葛、王\footnote{葛:琅邪诸葛原为葛氏。},而云王、葛?”令曰:“譬言驴马,不言马驴,驴宁胜马邪\footnote{宁:岂,难道。}?”{\fzxk\zihao{6}\textcolor{red}{诸葛恢,已见。}}

{\cangkai\zihao{5}【评】南渡之初,司马朝廷为在江南开基立国,就必须争取南北士族各家各姓的广泛支持。但是,魏晋实行九品中正制,门阀意识早已根深蒂固,即使贤如王导,其潜意识深处的门第偏见,也不免在不经意的玩笑中,自然地流露了出来。琅邪王家,东晋初拥立有功的有王导、王敦,掌握了朝廷的文武大权。因此,作为渡江后的第一贵族之家,已被社会承认,民间甚至有“王与马,共天下”的传说。故社会舆论序次排列是“王葛”,确是事实。王导言外有得意骄矜之色。但琅邪诸葛,也非等闲,其所反驳,巧用语序修辞,譬喻生动风趣,令人捧腹,“驴马”之比,一击成功,至少暂时压制了琅邪王导的气焰,促使执政者清醒自己的头脑。}

\lettrine{25.13} 刘真长\myidx{刘惔}始见王丞相\myidx{王导}\footnote{刘真长:刘惔字真长,沛国人,官至丹阳尹。东晋玄理清谈名士。始见:初次见到。王丞相:王导。},时盛暑之月,丞相以腹熨弹棋局\footnote{熨:贴。弹棋局:弹棋盘,中间隆起,平滑,多以玉石为质。},曰:“何乃渹\footnote{渹(qìnɡ庆,一读chènɡ秤):凉,冷。吴方言。}!”{\fzxk\zihao{6}\textcolor{red}{吴人以冷为渹。}} 刘既出,人问见王公云何\footnote{云何:感觉如何。},刘曰:“未见他异,唯闻作吴语耳。”{\fzxk\zihao{6}\textcolor{red}{\CJKunderwave{语林}曰:“真长云:‘丞相何奇,止能作吴语及细唾也。’”}}

{\cangkai\zihao{5}【评】王导、刘惔都是中原士族,二者北人,见面时该用京洛“国语”才是,为什么王导反而作吴语呢?就其谈话来说,王导是故意矫饰,而刘惔的批评才是出于自然。但是否做作者就坏,而自然者则好呢?却也未必,要看具体的语境要求。}

{\cangkai\zihao{5}刘惔是东晋一代清谈名家,中原士族之翘楚。他恃才傲物,江东士庶,自然不在眼下。他拒学吴语,可以想象。但作为朝廷执政的丞相,王导却强迫自己,超越中原士族的眼界,考虑的更多是国家的安定团结问题。建国江东,南、北士庶必须团结。因此,他学吴语,另有用心。如近人陈寅恪所评:“王导、刘惔本北人,而又皆士族,导何故用吴语接之?盖东晋之初,基业未固,导欲笼络江东人心,作吴语者,亦其开济政策之一端也。”(余嘉锡\CJKunderwave{笺疏}引)其“做作”出于形势的需要。而刘惔的讥讽,则说明他并不明白王导的良苦用心,反而成了政治上的糊涂人了。}

\lettrine{25.14} 王公\myidx{王导}与朝士共饮酒\footnote{王公:王导。},举琉璃碗谓伯仁\myidx{周顗}曰\footnote{琉璃:宝石的一种。伯仁:周顗字伯仁。官至尚书左仆射。参前\CJKunderwave{言语}第30则注。}:“此碗腹殊空\footnote{殊:甚,非常。},谓之宝器何邪?”{\fzxk\zihao{6}\textcolor{red}{以戏周之无能。}} 答曰:“此碗英英\footnote{英英:清明精美貌。},诚为清彻,所以为宝耳。”

{\cangkai\zihao{5}【评】东晋建国之初,百废待兴。时王导辅政,深望众士同心努力,以图恢复之计。周顗为人,虽深达危乱,颇获海内盛名,但终日濡首酒池,作左仆射时,略无醒日,时人讥为“三日仆射”。作为上司,王导对他不满,也在料中。故有“腹空”之谑,讥其无能为也。但周顗辩辞,与王导一样善用修辞譬喻,以碗之“英英”清明,喻己之清彻可宝,待之以时日之用,机会一到,则将大显宝器之价值。周顗不为上司而低下自己高傲的头颅,宝器之宝,在于心上有我——对自己充满了自信。}

\lettrine{25.15} 谢幼舆\myidx{谢鲲}谓周侯\myidx{周顗}曰\footnote{谢幼舆:谢鲲字幼舆。参前\CJKunderwave{文学}第20则注。周侯:指周顗,袭父浚爵为武城侯,故称。}:“卿类社树\footnote{社树:土地庙边的树,作为神社标志。},远望之,峨峨拂青天\footnote{峨峨:高峻貌。拂:披拂。};就而视之\footnote{就:靠近。},其根则群狐所托\footnote{托:寄托。},下聚溷而已\footnote{溷:粪便秽物。}。”{\fzxk\zihao{6}\textcolor{red}{谓顗好媟渎故。}} 答曰:“枝条拂青天,不以为高;群狐乱其下,不以为浊。聚溷之秽,卿之所保\footnote{保:保存,拥有。},何足自称\footnote{自称:自赞自夸。}!”

{\cangkai\zihao{5}【评】巧用修辞,无论是明喻暗喻,魏晋士人用来,大多得心应手,不仅见其语言艺术,而且同时窥其智慧和雅量。谢鲲为一代放诞名士,但顗之放诞,水平当不在鲲之下。谢鲲以为社树峨峨而空有其表,故以群狐聚溷讥顗。顗之辩辞,仍以社树为喻,但反其道而行之。“聚溷之秽,卿之所保,何足自称”,反戈一击,令鲲自受其辱。二人虽是玩笑戏语,但其语言艺术,却见聪明机智。}

\lettrine{25.16} 王长豫\myidx{王悦}幼便和令\footnote{王长豫:王悦字长豫,导长子。官至中书侍郎。早卒。和令:温和美好。},丞相\myidx{王导}爱恣甚笃\footnote{丞相:指王导。爱恣:溺爱骄纵。笃:深厚。}。每共围棋,丞相欲举行\footnote{举行:悔棋重走。},长豫按指不听\footnote{不听:不让。},丞相笑曰:“讵得尔\footnote{讵得尔:岂能这样。讵,岂。},相与似有瓜葛\footnote{相与:相互,彼此。瓜葛:以瓜、葛藤蔓牵连,喻其亲戚血缘关系。}。”{\fzxk\zihao{6}\textcolor{red}{蔡邕曰:瓜葛,疏亲也。}}

{\cangkai\zihao{5}【评】王导是政治家,政治家也是人,也有血缘关系和七情六欲。在诸子中,王悦“事亲色养”,因而其父“爱恣甚笃”;但对王恬,则因其性傲诞而不拘礼法,故“见恬便有怒色”。导见二子,一喜一怒,态度形成鲜明对照。故事中写父子弈棋,老父不遵守游戏规则而欲悔棋,悦按父之指而“不听”,细节相当生动,心理描绘细致,把丞相之家的日常生活乐趣,透过小动作形象地展现。“讵得尔,相与似有瓜葛”,父子玩笑,语言诙谐风趣,令人忍俊不禁。}

\lettrine{25.17} 明帝\myidx{司马绍}问周伯仁\myidx{周顗}\footnote{明帝:司马绍,字道畿,元帝长子,太宁元年(323)至三年(325)在位,年二十七崩,庙号肃祖。周伯仁:周顗字伯仁。}:“真长\myidx{刘惔}何如人\footnote{真长:刘惔字真长。参前第13则注。何如:怎样。}?”答曰:“故是千斤犗特\footnote{犗 (jiè介)特:阉公牛。特,公牛。}。”王公\myidx{王导}笑其言\footnote{王公:王导。}。伯仁曰:“不如卷角牸\footnote{卷角牸(zì字):弯卷头角的母牛。牸,母牛。},有盘辟之好\footnote{盘辟:盘旋进退。好:妙。}。”{\fzxk\zihao{6}\textcolor{red}{以戏王也。}}

{\cangkai\zihao{5}【评】魏晋君臣之间的朝廷政治生活,似乎不太“严肃”。唐宋明清之后封建专制大大加强,则君臣关系立刻变得“严肃”刻板了起来。相较之下,魏晋君臣之间,似还具一点人情味。皇帝随意问人,臣子率尔作答。以“千斤犗特”比喻刘惔,是说刘惔如骟牛力大无穷、性格驯服,可堪重任。王导嘲笑其言不雅,周顗又转掉话锋,把王导比作卷角老母牛,只是盘旋而不知前进。故余嘉锡评曰:“导在当时虽为元老宿望,而有不了事之称,故伯仁以此戏之。”周顗言笑戏谑,锋芒不避君主丞相,正见其性格自然率真的可爱。}

\lettrine{25.18} 王丞相\myidx{王导}枕周伯仁\myidx{周顗}膝\footnote{王丞相:王导。枕:名词动化,以……当枕。周伯仁:周顗。},指其腹曰:“卿此中何所有\footnote{此中:指腹中。参前第14则讥其腹中空空如也。}?”答曰:“此中空洞无物,然容卿辈数百人\footnote{卿辈:尔辈之人。}。”

{\cangkai\zihao{5}【评】在\CJKunderwave{排调}门中,王导与周顗,犹今相声艺术中的一对搭档,对话常是一正一反,针锋相对,把人引入发噱可笑的境地。但论其用心,却是善意的批评,而非相互敌意的诋毁。周顗常醉作达,故王导讥其腹中空空;但顗自我信心十足,等待时机而发挥更大的作用。“此中空洞无物,然容卿辈数百人!”气魄何等恢宏。在关键时刻,顗忠于王事,骂贼而死,正见其无用之大用。}

\lettrine{25.18} 干宝\myidx{干宝}向刘真长\myidx{刘惔}{\fzxk\zihao{6}\textcolor{red}{\CJKunderwave{中兴书}曰:“宝字令升,新蔡人。祖正(统),吴奋武将军。父莹,丹阳丞。宝少以博学才器箸称。历散骑常侍。”}} 叙其\CJKunderwave{搜神记}\footnote{干宝:东晋初,元帝置史官,干宝以著作郎领修国史,著\CJKunderwave{晋纪}二十卷,有良史之称。其志怪小说笔记\CJKunderwave{搜神记}三十卷,多写鬼神故事。原书佚,后人辑佚有二十卷。},{\fzxk\zihao{6}\textcolor{red}{\CJKunderwave{孔氏志怪}曰:“宝父有嬖人,宝母至妒,葬宝父时,因推箸藏中。经十年而母丧,开墓,其婢伏棺上,就视犹暖,渐有气息,舆还家,终日而苏。说宝父常致饮食,与之接寝,恩情如生。家中吉凶辄语之,校之悉验。平复数年后方卒。宝因作\CJKunderwave{搜神记},中云‘有所感起’是也。”}} 刘曰:“卿可谓鬼之董狐\footnote{董狐:春秋时晋国史官。注引\CJKunderwave{春秋传}事,见\CJKunderwave{左传}宣公二年。因孔子称赞,董狐已成为秉笔直书“古之良史”的代表人物。}。”{\fzxk\zihao{6}\textcolor{red}{\CJKunderwave{春秋传}曰:“赵穿攻晋灵公于桃园,赵宣子未出境而复。太史书‘赵盾弑其君’。宣子曰:‘不然。’对曰:‘子为正卿,亡不越境,反不讨贼,非子而谁?’孔子曰:‘董狐,古之良史也,书法不隐。赵盾,古之贤大夫也,为法受恶。’”}}

{\cangkai\zihao{5}【评】干宝及其\CJKunderwave{搜神记},经一代名士刘惔推扬,身价倍增,传之不朽。今天传世的魏晋小说中,志人以\CJKunderwave{世说新语}为魁,志怪则以\CJKunderwave{搜神记}称杰。董狐是春秋时以秉笔直书著名的正直史官,干宝则是东晋史官,其\CJKunderwave{晋纪}有“良史”之称,而其\CJKunderwave{搜神记}更获“鬼之董狐”之誉。干宝其人,史称其“性好阴阳术数”,鬼神之事,传闻甚广而信以为真,于是他“博采异同,遂混虚实”,成此\CJKunderwave{搜神记}三十卷。其所著述,如其自序所称,是为了“明神道之不诬也”。他是用“实录”的方法来作小说的,在艺术之虚与实观念矛盾中,他特别强调传闻故事的真实性。他笔下的鬼神,无不与人一样,有血有肉,有灵魂有感情,正是魏晋士人热爱生命的一种特殊的浪漫表现。}

\lettrine{25.19} 许文思\myidx{许璪}往顾和\myidx{顾和}许\footnote{许文思:刘注名琛,生履未详。据徐震堮\CJKunderwave{校笺}:“案许琛前未见,\CJKunderwave{晋书}亦无传,唯\CJKunderwave{雅量}一六许侍中下注:‘许璪字思文。’疑即其人,‘琛’或是‘璪’之误。”许璪为义兴阳羡人,曾与顾和友善,并为王导赏识。顾和:字君孝。吴郡人。顾荣族子。官到尚书令。卒赠司空。参前\CJKunderwave{言语}第33则注。许:处所,住处。},顾先在帐中眠,许至,便径就床角枕共语\footnote{角枕:饰以兽角的枕头。}。{\fzxk\zihao{6}\textcolor{red}{许琛,已见。}}既而唤顾共行,顾乃命左右取机枕上新衣\footnote{取机枕上:袁本作“杭上”,朱铸禹\CJKunderwave{汇校集注}引王先谦曰:“按‘杭’与‘桁’同声字。桁,衣架也。古乐府\CJKunderwave{东门行}:‘还视桁上无悬衣’是也。此本作‘枕’,涉上文角枕字误。”},易己体上所箸。许笑曰:“卿乃复有行来衣乎王\footnote{行来:犹往来。王:无义,疑衍。}?”

{\cangkai\zihao{5}【评】顾和出身于江南四大家族之一的吴郡顾家,两晋之交,顾和是继顾荣之后,把顾氏家族发扬光大的又一代表人物。大概因门第之故,做事未免矫饰自高,而不肯同于一般。出门必换新衣,近似今天模特儿的服装表演,做作而欠真率自然,故致许氏之讥,亦在情理之中。}

\lettrine{25.21} 康僧渊\myidx{康僧渊}目深而鼻高\footnote{康僧渊:西域高僧,晋成帝时南渡。与王导、庾亮等名士交游。参前\CJKunderwave{文学}第47则注。},王丞相\myidx{王导}每调之\footnote{王丞相:王导。调:调侃,嘲笑。},僧渊曰:“鼻者,面之山;{\fzxk\zihao{6}\textcolor{red}{\CJKunderwave{管辂别传}曰:“鼻者,天中之山。”\CJKunderwave{相书}曰:“鼻之所在为天中,鼻有山象,故曰山。”}} 目者,面之渊\footnote{渊:水潭,深池。}。山不高则不灵,渊不深则不清。”

{\cangkai\zihao{5}【评】王导与康僧渊的关系非同一般,“每调之”就是一个证明。他嘲笑和尚,“调”前加“每”,就说明是经常性的,一见面就拿他寻开心,如果不是比较熟悉的朋友,是不可能这样做的。其所谓“调”,是善意的玩笑,含有亲近的意思。身为一人之下、万人之上的丞相,为什么要在百忙中抽空与和尚搞亲近套热乎呢?一是出自对于“胡”族高僧的尊重;一是忙中偷闲,从高僧身上学习佛学义理,给自己增添一点精神食粮;一是康僧渊在东晋上流贵族社会中有一定影响,他又曾在豫章立寺建私人学校,据\CJKunderwave{高僧传}载,该校生徒甚众,“名僧胜达,响附成群,常以\CJKunderwave{持心梵天经}空理幽远,故偏加讲说。尚学之徒,往还填巷”。通过康僧渊,可以扩大朝廷的思想影响。王导到底是政治家,目光深远。至于康僧渊,其自我解嘲之语,幽默风趣,又颇见学问修养。他虽“胡”僧,却熟悉中国学问中的三玄之理。“鼻者面之山”,即来自\CJKunderwave{易经}。\CJKunderwave{说卦传}有“艮为山”之说。八卦中的艮卦,象征物为山。而鼻在人的脸上,部位突出如山,故艮卦又有鼻象,和尚所言,完全合乎\CJKunderwave{易}理。为在东土宣扬佛学,他先学中国学问,以便使佛学中国化,更易为中土信徒所接受。看来,康僧渊亦是用心良苦,长远打算。}

\lettrine{25.22} 何次道\myidx{何充}往瓦官寺礼拜甚勤\footnote{何次道:何充字次道,东晋庐江人。穆帝时拜相辅政。参前\CJKunderwave{政事}第17则注。瓦棺寺:佛寺名,在建康西南隅。寺有瓦官阁。礼拜:礼敬参拜。},{\fzxk\zihao{6}\textcolor{red}{充崇释氏,甚加敬也。}} 阮思旷\myidx{阮裕}语之曰\footnote{阮思旷:阮裕字思旷,陈留尉氏人。诏征金紫光禄大夫,不就。参前\CJKunderwave{德行}第32则注。}:“卿志大宇宙,{\fzxk\zihao{6}\textcolor{red}{\CJKunderwave{尸子}曰:“天地四方曰宇,往古来今曰宙。”}} 勇迈终古\footnote{迈:超越。}。”{\fzxk\zihao{6}\textcolor{red}{终古,往古也。\CJKunderwave{楚辞}曰:“吾不能忍此终古也。”}} 何曰:“卿今日何故忽见推\footnote{见推:加以推扬。推:赞许。}?”阮曰:“我图数千户郡,尚不能得;卿乃图作佛\footnote{乃:竟然。},不亦大乎!”{\fzxk\zihao{6}\textcolor{red}{思旷,裕也。}}

{\cangkai\zihao{5}【评】史称何充“性好释典,崇修佛寺,供给沙门以百数,糜费巨亿而不吝”,故致时人佞佛之讥。其勤于礼拜佛寺,应属迷信性质。阮裕之言,先扬后抑,语有转折,更有深致。其所讥含蓄蕴藉,令人深思。何充作为宰辅重臣,不问苍生问鬼神,其于国政,无所改革,亦在料中。何充祈求成佛之心颇切。佛者,觉者,智者,佛陀也,成佛即摆脱六道轮回之苦而获永生。永生之佛,超迈终古,何其伟哉,岂是佞佛者可用金钱买来的!佛在人心中,不在钱眼里,愚哉何充!至于阮裕求一郡守之难,却是实话实说。他为人淡泊功名,而有肥遁之志,但为生计,也曾任临海、东阳二郡太守。他坦白地说:“虽屡辞王命,非敢为高也。吾少无宦情,兼拙于人间,既不能躬耕自活,必有所资,故曲躬二郡,岂以骋能,私计故耳。”倾吐心曲,真率可爱。何、阮二人相较,思想境界高低自判。}

\lettrine{25.23} 庾征西\myidx{庾翼}大举征胡\footnote{庾征西:指庾翼。在北伐时,进征西将军,领南蛮校尉,故称。参前\CJKunderwave{言语}第53则注。胡:指后赵石氏政权。},既成行,止镇襄阳\footnote{襄阳:县名,即今之湖北襄樊。}。{\fzxk\zihao{6}\textcolor{red}{\CJKunderwave{晋阳秋}曰:“翼率众入沔,将谋伐狄。既至襄阳,狄尚强,未可决战。会康帝崩,兄冰薨,留长子方之守襄阳,自驰还夏口。”}} 殷豫章\myidx{殷羡}与书\footnote{殷豫章:指殷羡,时任豫章太守。},送一折角如意以调之\footnote{折角如意:缺一角的如意。调:嘲弄。}。{\fzxk\zihao{6}\textcolor{red}{豫章,殷羡。}} 庾答书曰:“得所致,虽是败物\footnote{败物:破损之物。},犹欲理而用之\footnote{理:修理,治理。}。”

{\cangkai\zihao{5}【评】故事中的主角是庾翼,而殷羡则是作为对照而存在的陪衬人物。翼兄亮卒,代其都督江荆司雍梁益六州军事、荆州刺史,进征西将军。庾翼是个重实干而不徒空言的政治家,平素雅有大志,意图恢复,以灭胡平蜀为己任。故其言论慷慨,形于辞色。殷羡曾守长沙郡,是荆州下属,其为人,骄奢无赖,在郡贪残,在荆州地区一二十郡中最为凶恶而民愤颇大。他曾托亲朋走“后门”,作为上司,庾翼坚决纠治而不假贷。现在,翼率军四万北伐受阻,殷羡则赠折角如意加以讥讽,这不是从国家利益出发的善意批评,而是另一特殊形式的打击报复,讥其事不如意,壮志受挫。其态度是幸灾乐祸。对于殷羡的调侃嘲讽,庾翼并不退缩,其答辞委婉蕴藉,有君子之风,但态度坚决,言外之意,殷虽是缺德“败物”,但经修治,仍冀物尽其用。二者相形,君子小人显明呈现。}

\lettrine{25.24} 桓大司马\myidx{桓温}乘雪欲猎\footnote{桓大司马:指桓温。猎:打猎。},先过王\myidx{王濛}、刘\myidx{刘惔}诸人许\footnote{过:探望。王刘:王指王濛,刘指刘惔。皆为一时清谈玄家。许:住处。}。真长见其装束单急\footnote{单急:紧身轻便戎装。},问:“老贼欲持此何作\footnote{老贼:戏谑之称,犹言老家伙,老东西。何作:干什么。}?”桓曰:“我若不为此\footnote{为此:指穿作战服。},卿辈亦那得坐谈\footnote{坐谈:坐而论道,指清谈。}?”{\fzxk\zihao{6}\textcolor{red}{\CJKunderwave{语林}曰:“宣武征还,刘尹数十里迎之,桓都不语,直云:‘垂长衣,谈清言,竟是谁功?’刘答曰:‘晋德灵长,功岂在尔!’”二人说小异,故详载之。}}

{\cangkai\zihao{5}【评】桓温本人颇精玄理,也曾热心清谈。但论其究竟,却是一个雄心与野心并存的政治家。因而,他更看重实干。这与前则庾翼颇有异同。庾翼很看重桓温的政治才干,在其年轻时,曾向成帝热情推荐,曰:“桓温有英雄才略,愿陛下勿以常人遇之,常婿畜之,宜委之以方郡之任,必有弘济艰难之勋。”(见\CJKunderwave{晋书}翼传)故事中刘惔一声“老贼”,随意之中,又带几分亲昵,几分调侃,说明了桓温与王(濛)、刘(惔),关系非同一般,是清谈圈中的知友。但当王、刘辈讥其“装束单急”之时,桓温反唇相讥,加以调侃,如果没有人穿军装去打仗,你们这些自命高雅的人,能够安稳坐在这里高谈阔论吗?所言境界更高一层。}

\lettrine{25.25} 褚季野\myidx{褚裒}问孙盛\myidx{孙盛}\footnote{褚季野:褚裒,字季野。以外戚出为江、兖二州刺史。卒赠太傅。参前\CJKunderwave{德行}第34则。孙盛:字安国。著\CJKunderwave{魏氏春秋}、\CJKunderwave{晋阳秋}等史书。参前\CJKunderwave{言语}第49则注。}:“卿国史何当成\footnote{何当成:何日写成。}?”孙云:“久应竟\footnote{竟:完功,终了。},在公无暇,故至今日。”褚曰:“古人述而不作\footnote{述而不作:\CJKunderwave{论语·述而}有“述而不作,信而好古”之言,意谓继承前人而无须创立。},何必在蚕室中\footnote{在蚕室中:汉司马迁因李陵事件下蚕室受腐刑,因而发愤著\CJKunderwave{史记}以垂之不朽。}!”{\fzxk\zihao{6}\textcolor{red}{\CJKunderwave{汉书}曰:“李陵降匈奴,武帝甚怒。太史令司马迁盛明陵之忠,帝以迁为陵游说,下迁腐刑。乃述唐虞以来,至于获麟,为\CJKunderwave{史记}。”迁\CJKunderwave{与任安书}曰:“李陵既生降,仆又茸(佴)之以蚕室。”苏林注曰:“腐刑者,作密室蓄火,时如蚕室。”旧时平阴有蚕室狱。}}

{\cangkai\zihao{5}【评】孙盛作为桓温下属,著\CJKunderwave{晋阳秋}直叙温枋头之败,温怒,谓盛子曰:“枋头诚为失利,何至乃如尊君所说!若此史遂行,自是关君门户事。”以孙盛全家性命前途相胁,迫其删改。诸子号泣,乞盛为家族百口计。但盛大怒,断然置权臣的生命威胁于不顾,无愧于一代良史之誉。而褚裒自小具简贵冲默之风,外无臧否而内有褒贬,故世有“皮里阳秋”之称。其作风与玄道为近,以无用为用,故讥孙盛著书劳累且有风险,劝其述而不作而无须创立。季野调盛,并无恶意,但对史学发展,却是个馊主意。依盛刚强之性,必拒而不纳。幸哉,\CJKunderwave{晋阳秋}!}

\lettrine{25.26} 谢公\myidx{谢安}在东山\footnote{谢公:谢安。东山:山名,在今浙江上虞市境内。},朝命屡降而不动\footnote{朝命:朝廷征聘的诏命。不动:不应命。}。后出为桓宣武\myidx{桓温}司马\footnote{桓宣武:指桓温,卒谥宣武,故称。},将发新亭\footnote{将发新亭:将从新亭出发。新亭,故址在今南京附近长江边上。},朝士咸出瞻送\footnote{瞻送:送行。瞻,表仰观敬意,在此仅取其虚意表敬。}。高灵\myidx{高灵}时为中丞\footnote{高灵:高崧小字𨟯,“灵”当作“𨟯”。广陵人,官至侍中。中丞:御史台长官。},亦往相祖\footnote{祖:饯行,后引申为送别。祖,祭名,祭路神以祈行旅平安。}。先时,多少饮酒\footnote{多少:稍微。},因倚如醉,戏曰:“卿屡违朝旨,高卧东山,诸人每相与言:‘安石不肯出,将如苍生何\footnote{如苍生何:怎样对待百姓呢?}!’今亦苍生将如卿何?”谢笑而不答。{\fzxk\zihao{6}\textcolor{red}{高灵,已见。\CJKunderwave{妇人集}载桓玄问王疑(凝)之妻谢氏曰:“太傅东山二十馀年,遂复不终,其埋(理)云何?”谢答曰:“亡叔太傅先正,以无用为心,显隐为优劣,始末正当动静之异耳。”}}

{\cangkai\zihao{5}【评】魏晋名士,以隐逸为高,介然超俗,养气浩然,藏声江海之上,卷迹俗尘之中。年轻谢安,原是志在高卧,寓居会稽,与王羲之、许询、支遁友善,“出则渔弋山水,入则言咏属文”,无仕宦意。但自弟万北伐败归被废,陈郡谢氏家族复兴,陷于危机之中。谢安出山,与谢氏家族的整体利益密切相关,谢氏家族以此为转折,终于从挫折走向了繁荣的顶峰。高灵之讥,王世懋评谓其“似醉不醉,语绝妙”。发点酒疯,略带讥讽,言语微妙,极富语言艺术魅力。但魏晋乃门阀社会,没有家族门阀的利益,人生就失去了表演的舞台。因此,为家族利益计,谢安不为所动,“笑而不答”,同样不失名士风度。}

\lettrine{25.27} 初,谢安\myidx{谢安}在东山居,布衣时\footnote{居布衣时:出仕之前,谢安长期隐居于会稽东山。布衣,平民。},兄弟已有富贵者\footnote{兄弟已有富贵者:指堂兄尚,长兄奕和弟万等,相继为节镇将军,专任一方之长。},集翕家门,倾动人物\footnote{倾动人物:令社会人士倾心敬服。}。刘夫人戏谓安曰\footnote{刘夫人:安妻为刘耽女,惔妹,出于名门。}:“大丈夫不当如此乎?”谢乃捉鼻曰\footnote{捉鼻:捏鼻。}:“但恐不免耳。”

{\cangkai\zihao{5}【评】魏晋是门阀社会,基本上是世家望族与皇室共享特权。一姓高门豪族,常是一荣俱荣,一败则满门皆输。因此,只要谢家兄弟富贵,谢安自然生活无忧,可以满足其隐居生活的夙愿。“集翕家门,倾动人物”,写富贵之态,言简意赅,具体形象。刘夫人之“戏”,在调侃丈夫时,偶然透露出几分欣羡的意思。谢安“捉鼻”,\CJKunderwave{晋书}作“掩鼻”,如闻臭气,恶而掩之,正见其鄙夷不屑之意。据传安少有鼻疾,语音重浊,故捏鼻使气息调畅,成为习惯。“但恐不免耳!”正是明白富贵无常的深谋远虑,也可说是一种忧患意识。果然,在兄尚、奕俱死之后,弟万败废,打击接踵而至。为了陈郡阳夏谢氏家族的复兴,谢安不得不违背志愿,东山再起,终于成就了东晋一代的风流名相。违心之举,个人不幸;但对家族和国家,却是大幸。}

\lettrine{25.28} 支道林\myidx{支遁}因人就深公\myidx{竺法深}买印(𡵙)山\footnote{支道林:东晋高僧支遁,字道林。本姓关氏。河内人。或称支氏、支公、林公、林道人或林法师。参前\CJKunderwave{言语}第63则注。深公:即东晋僧人竺法深,法名道潜。永嘉南渡,隐居剡县𡵙山(今浙江嵊州市东)。参前\CJKunderwave{德行}第30则注。印山:深公居𡵙山,故“印”当为“𡵙”之形误。},深公答曰:“未闻巢、由买山而隐\footnote{巢由:指巢父和许由。传说中尧时隐士。由居箕山之下。相传尧欲让位与巢、由,二人鄙之。}。”{\fzxk\zihao{6}\textcolor{red}{\CJKunderwave{逸士传}曰:“巢父者,尧时隐人,山居,不营世利。年老,以树为巢而寝其上,故号巢父。”\CJKunderwave{高逸沙门传}曰:“遁得深公之言,惭恧而已。”}}

{\cangkai\zihao{5}【评】支遁是东晋一代高僧,多与王羲之、谢安、许询等名士交游。深公则是另一种人,其隐居𡵙山,浩然高栖而道徽高扇,终于山中而人称其德行。其“未闻巢、由买山而隐”之言,味其言外,意讥支公方外之人而游于方内,虽富足而不免于俗也。}

\lettrine{25.29} 王\myidx{王濛}、刘\myidx{刘惔}每不重蔡公\myidx{蔡谟}\footnote{王刘:指王濛和刘惔。东晋清谈名士。每:常。重:尊重,重视。蔡公:指蔡谟,字道明,官至司徒。参前\CJKunderwave{方正}第40则注。}。二人尝诣蔡语\footnote{诣:拜访,到……去。},良久,乃问蔡曰:“公自言何如夷甫\myidx{王衍}\footnote{夷甫:王衍字夷甫,官太尉。是西晋的清谈领袖。}?”答曰:“身不如夷甫\footnote{身:第一人称代词,我。}。”王、刘相目而笑曰\footnote{相目:相视。目,作动词用,看。}:“公何处不如?”答曰:“夷甫无君辈客。”

{\cangkai\zihao{5}【评】王濛、刘惔是东晋一代的清谈名士,自视甚高而傲倪前贤,正是其轻佻儇薄的意识作祟。二人“相目而笑”,细节生动传神,调笑嘲讽之意,自眼神流出。但蔡谟先是自认“身不如夷甫”,以谦卑之态引王、刘入其彀中,一折。待王、刘得意不备之时,“夷甫无君辈客”之言冲口而出,反唇相讥,一击而中。如刘辰翁所评,“不深不浅”之间,足令轻薄者茫然自失。}

\lettrine{25.30} 张吴兴\myidx{张玄之}年八岁\footnote{张吴兴:张玄之,字希祖。与谢玄齐名称“南北二玄”。曾官吴兴太守,故称。参前\CJKunderwave{言语}第51则注。},亏齿\footnote{亏齿:掉牙。亏,缺。},{\fzxk\zihao{6}\textcolor{red}{玄之,已见。}} 先达知其不常\footnote{先达:先辈贤达之人。不常:不同凡常。},故戏之曰:“君口中何为开狗窦\footnote{狗窦:狗洞。}?”张应声答曰:“正使君辈从此中出入!”

{\cangkai\zihao{5}【评】儿童应对之智慧,冲口而出,令人惊讶佩服。看来,八岁的张玄之是早熟的神童。先达因其缺牙而故意以“狗洞”嘲之,言外之意是狗嘴吐不出象牙,讥其日后难有出息;而张玄之则是初生牛犊,无所畏缩,针锋相对而犹如宿构,他引用春秋时晏子出使楚国故事,谓使狗国者从狗门入,反讥先达尽是狗国中人。辱人者反而受辱。此虽善意的玩笑,但看得出八岁儿童所受教育颇佳,熟听故事而成其智慧,亦是一奇。于此可见,早期儿童教育及其智慧开发,大有可为。}

\lettrine{25.31} 郝隆\myidx{郝隆}七月七日出日中仰卧\footnote{郝隆:东晋人,长期任桓温幕府僚佐。七月七日:古代风俗,于此日晒衣服、经书,以避免虫蠹。},人问其故,答曰:“我晒书\footnote{晒书:意指腹中熟读之书。}。”{\fzxk\zihao{6}\textcolor{red}{\CJKunderwave{征西寮属名}曰:“隆字佐治,汲郡人。仕吴(‘吴’字衍)至征西参军。”}}

{\cangkai\zihao{5}【评】郝隆其人,虽然官卑职微,但为人幽默诙谐而颇见学问。古时习俗,七月七日晒衣服及书籍以防蠹。但后来风气渐变,如前\CJKunderwave{任诞}第10则谓“七月七日,北阮盛晒衣,皆纱罗锦绮”。这就使晒物防蠹科学之举,化为贵族之家以富贵骄人的服装展览会。针对这种异化现象,郝隆反其道而行之。绫罗绸缎,华丽服装,寒士所缺;但满腹经纶而饱读诗书,则是贵游子弟所无。坦腹“晒书”,行为滑稽,言语可笑,但却充满了自信而见其人格精神。}

\lettrine{25.32} 谢公\myidx{谢}始有东山之志\footnote{谢公:指谢安。东山之志:谢安曾隐居会稽东山二十馀年。},后严命屡臻\footnote{严命:严厉诏令。臻:到达。},势不获已\footnote{不获已:不得已。},始就桓公\myidx{桓温}司马\footnote{桓公:指桓温。时任征西将军、荆州刺史。}。于时人有饷桓公药草\footnote{饷:赠。},中有远志\footnote{远志:中药名。}。公取以问谢:“此药又名小草,何一物而有二称?”{\fzxk\zihao{6}\textcolor{red}{\CJKunderwave{本草}曰:“远志一名棘宛,其叶名小草。”}} 谢未即答。时郝隆\myidx{郝隆}在坐\footnote{郝隆:参前则注。},应声答曰:“此甚易解。处则为远志,出则为小草\footnote{处则为远志,出则为小草:远志,中药名,其根埋土中为处,名远志;其叶生地上为出,名小草。}。”谢甚有愧色。桓公目谢而笑曰:“郝参军此过(通)乃不恶\footnote{此过:犹言这回。\CJKunderwave{太平御览}卷九八九引作“此通”,义亦顺畅。通,阐述,解释。},亦极有会\footnote{会:会心,意兴。}。”

{\cangkai\zihao{5}【评】这是一个“三人转”的游戏,三人言语态度各异旨趣。谢安之隐与仕,也即处与出,自有其苦衷。其素愿在隐,高卧东山二十馀年,岂是作假之人的矫饰!其终不免于仕,关键在陈郡谢氏家族利益。郝隆不同,他大概出于庶族寒门,故对士族名士不稍宽容,其言虽戏,其态度却是咄咄逼人而逞其智辩。余嘉锡\CJKunderwave{笺疏}评云:“远志与小草,虽一物而有根与叶之不同。叶名小草,根不可名小草也。郝隆之答,谓出与处异名,亦是分根与叶言之。根埋土中为处,叶生地上为出。既协物情,又因此以讥谢公,语意双关,故为妙对。”郝隆以任人采撷的“小草”,影射谢安,旨在嘲讽他改变了自己高隐山林的“远志”,成为热衷功名的人物。至于桓温,能礼聘安入幕,倾动朝野,因而“大喜,深礼重之”(\CJKunderwave{通鉴}卷一○一)。故其言与郝隆的嘲讽不同,略带几分得意之色。但又因其少小贫寒,门第并非一流士族,故在郝、谢二人之间,其理解与同情砝码,似向郝氏倾斜,这是潜意识的作用,而非出自理性思考的作秀。}

\lettrine{25.33} 庾爰客\myidx{庾爰之}诣孙监\myidx{孙盛}\footnote{庾爰客:庾爰之,小字爰客。征西将军庾翼子。后代父任荆州刺史,为桓温所废。参前\CJKunderwave{识鉴}第19则注。孙监:孙盛,字安国,曾任秘书监,故称。参前\CJKunderwave{言语}第49则注。},值行\footnote{值行:恰逢外出。},见齐庄\myidx{孙放}在外\footnote{齐庄:孙放字齐庄,盛次子。参前\CJKunderwave{言语}第50则注。},尚幼,而有神意\footnote{神意:风神意趣。}。庾试之曰:“孙安国何在?”即答曰:“庾稚恭\myidx{庾翼}家\footnote{庾稚恭:庾翼,字稚恭。爰客父。}。”庾大笑曰:“诸孙大盛\footnote{诸孙大盛:故犯放父盛名讳以戏。},有儿如此!”又答曰:“未若诸庾之翼翼\footnote{诸庾翼翼:故犯庾爰客父翼名讳以报复。翼翼:繁盛貌。}。”还语人曰:“我故胜\footnote{故:确实,的确。},得重唤奴父名\footnote{奴:对人鄙称。}。”{\fzxk\zihao{6}\textcolor{red}{\CJKunderwave{孙放别传}曰:“放兄弟并秀异,与庾翼子爰客同为学生。爰客少有佳称,因谈笑嘲放曰:‘诸孙于今为盛。’盛,监君讳也。放即答曰:‘未若诸庾之翼翼。’放应机制胜,时人仰焉。司马景王、陈、锺诸贤相酬,无以逾也。”}}

{\cangkai\zihao{5}【评】支撑东晋政权的著名门阀,主要有王、谢、庾、桓四大家族,几乎与司马皇朝相始终。颍川鄢陵庾氏家族,自庾亮、冰、翼诸人,高踞要津,身兼文武,威势显赫。故其子弟,如爰客辈,傲视寒士,犯人家讳,故意挑起年轻人之间的口舌之辩,虽然并非出于敌对立场的诋毁攻击,但却是高门士族潜意识的表现。面对贵游子弟甚为嚣张的气焰,孙放虽为幼童,却也绝不畏缩退让,而是有问自有答,一来必有往,针锋相对,谑语生趣,洋溢了人生的智慧,真是令人叫绝。魏晋士庶对立,又在儿辈身上,见其影迹。孙家儿郎,又多一神童也!}

\lettrine{25.34} 范玄平\myidx{范汪}在简文\myidx{司马昱}坐\footnote{范玄平:范汪字玄平。曾任桓温长史,平蜀后,自请还京,出为东阳太守。在郡大兴学校,从容讲肆。简文:简文帝司马昱,即位前为会稽王、抚军大将军、录尚书事。席:坐席。},谈欲屈\footnote{谈:玄理清谈。屈:挫折。},引王长史\myidx{王濛}曰\footnote{王长史:指王濛,清谈名家。}:“卿助我。”{\fzxk\zihao{6}\textcolor{red}{\CJKunderwave{范汪别传}曰:“汪字玄平,颍阳人。左将军略(晷)之孙。少有不常之志,通敏多识,博涉经籍,致誉于时。历吏部尚书,徐、兖二州刺史。”}} 王曰:“此非拔山力所能助。”{\fzxk\zihao{6}\textcolor{red}{\CJKunderwave{史记}曰:“项羽为汉兵所围,夜起歌曰:‘力拔山兮气盖世,时不利兮骓不逝。’”}}

{\cangkai\zihao{5}【评】清谈之聚,虽简文帝贵为帝王亦有此乐,于此见时风众尚之一斑。魏晋清谈,并非仅是“虚谈废务,浮文妨要”,而是一种思想交锋和理论训练。而理论的自由论争,与两军打仗不同,依靠的不是力大无穷的盖世武功,而是平素积累的追求真理的理论修养,以理服人,是其规律准则。范汪虽是当时著名的教育家,博学多通,善谈名理。但老马也有失蹄时。面对“谈欲屈”的被动局面,倔犟老头,却顾脸面,急时呼助,见其困窘之态,形象颇为生动传神。而王濛“此非拔山力所能助”,虽为调侃,却也实事求是地道出了理论争锋的游戏规则。}

\lettrine{25.35} 郝隆\myidx{郝隆}为桓公\myidx{桓温}南蛮参军\footnote{郝隆:见前注。桓公:指桓温。南蛮参军:幕府官名,即南蛮校尉府的僚佐。}。三月三日会\footnote{三月三日会:古时风俗,三月三日上巳节祓禊,人们在水边赏玩,祈福驱邪,饮酒赋诗取乐。会:聚会。},作诗,不能者罚酒三斗。隆初以不能受罚,既饮,揽笔便作一句云:“娵隅跃清池\footnote{娵隅(jū yú居于):古代西南少数民族方言称鱼为娵隅。}。”桓问:“娵隅是何物\footnote{何物:什么东西。}?”答曰:“蛮名鱼为娵隅。”桓公曰:“作诗何以作蛮语?”隆曰:“千里投公,始得蛮府参军,那得不作蛮语也!”

{\cangkai\zihao{5}【评】郝隆其人,\CJKunderwave{排调}门中共有3则,见其机敏智慧,是个幽默风趣的喜剧人物。“娵隅跃清池”,以西南少数民族方言入诗,这是古代的白话诗,打破了古代贵族的雅文学的传统。桓温所问,从维护传统文学尚雅的立场出发。而郝隆所答,则是语带双关,具有一箭双雕的穿透力。南蛮参军作蛮语,自是顺理成章。言外谓府主辜负了自己那坦腹晒书的学问文章,其牢骚之言,令人发噱捧腹。}

\lettrine{25.36} 袁羊\myidx{袁乔}尝诣刘恢\myidx{刘惔}\footnote{袁羊:袁乔字彦叔,小字羊,陈郡人。官至益州刺史,爵湘西伯。参前\CJKunderwave{言语}第90则注。刘恢:余嘉锡\CJKunderwave{笺疏}引程炎震曰:“恢当作惔,各本皆误。”按:“恢”与“惔”形近而讹。刘惔,尚明帝女庐陵公主。},恢(惔)在内眠未起\footnote{内:内室,卧室。},袁因作诗调之曰:“角枕粲文茵\footnote{角枕:兽角装饰的枕头。粲:鲜明貌。文茵:华丽褥垫。},锦衾烂长筵\footnote{锦衾:锦缎做的被子。烂:光亮貌。长筵:长席。筵,竹席。}。”{\fzxk\zihao{6}\textcolor{red}{\CJKunderwave{唐诗}。曰:晋献公好攻战,国人多丧,其诗曰:“角枕粲兮,锦衾烂兮,予美亡此,谁与独旦?”袁故嘲之。}} 刘尚晋明帝\myidx{司马绍}女\footnote{尚:娶公主为妻称尚。晋明帝女:指庐陵公主。晋明帝,司马绍。},{\fzxk\zihao{6}\textcolor{red}{\CJKunderwave{晋阳秋}曰:“恢尚庐陵长公主,名南弟。”}} 主见诗不平,曰:“袁羊,古之遗狂\footnote{古之遗狂:古代遗留的放荡狂人。}!”

{\cangkai\zihao{5}【评】袁羊之诗,反古诗意而用之,嘲讽驸马贪恋女色,日晚不起,对不起等待多时的朋友。其用心够刻薄的。魏晋名士,连朋友的男女隐私都可作为戏谑调笑的作料,甚至是高贵的公主驸马也不顾,其放荡不拘之性,出于自然,乃真狂也。但也因魏晋社会在男女生活方面,具有相对的开放意识,成为袁羊狂人存在的社会基础。一旦失却社会基础,如置放于明清时代,则袁羊的头颅危乎殆哉!}

\lettrine{25.37} 殷洪远\myidx{殷融}答孙兴公\myidx{孙绰}诗云\footnote{殷洪远:殷融字洪远,陈郡人。官至吏部尚书。与兄子浩俱为当时清言名家。参前\CJKunderwave{文学}第74则注。孙兴公:孙绰字兴公,太原中都人。当时著名文学家。官至散骑常侍。参前\CJKunderwave{言语}第84则注。}:“聊复放一曲\footnote{聊复:姑且。放:引吭高歌。}。”刘真长\myidx{刘惔}笑其语拙\footnote{刘真长:刘惔。见前注。拙:拙劣。},问曰:“君欲云那放\footnote{那:怎么。}?”殷曰:“㯓腊亦放\footnote{㯓 腊:同于“榻腊”,象声词,鼓声。㯓腊鼓以手揩之,其声㯓腊,故云。},何必其鎗铃邪\footnote{鎗 铃:象声词,钟铃之声。}?”{\fzxk\zihao{6}\textcolor{red}{殷融已见。}}

{\cangkai\zihao{5}【评】魏晋名士,大概受到民间乐府诗的影响,经历了从雅到俗的转化。殷融诗“聊复放一曲”,刘惔笑其“语拙”,其实是讥其诗直用口语的平民化倾向。但殷融并不买账,他坚持自己的主张和实践。“㯓腊”与“鎗铃”,都是象声的联绵词。钟铃金石,制作昂贵,多为贵族庙堂之音;而㯓腊鼓之类,则是民间常用的打击乐器,制作虽“土”,但可用以节制音乐,兼有指挥的作用。如余嘉锡\CJKunderwave{笺疏}所评:“此云‘㯓腊亦放,何必鎗铃’者,谓己诗虽不工,亦足以达意,何必雕章绘句,然后为诗?犹之鼓虽无当于五声,亦足以应节,何必金石铿锵,然后为乐也?”从中可体会到魏晋文学雅与俗的矛盾及其发展。}

\lettrine{25.38} 桓公\myidx{桓温}既废海西\myidx{司马奕}\footnote{桓公:指桓温。海西:指司马奕。太和元年(366)即位,太和六年(371)被废为海西公。},立简文\myidx{司马昱}\footnote{简文:简文帝司马昱,在位二年,忧崩。},{\fzxk\zihao{6}\textcolor{red}{\CJKunderwave{晋阳秋}曰:“海西公讳奕,字延龄。成帝子也。兴宁中即位。少同阉人之疾,使宫人与左右淫通生子。大司马温目(自)广陵还姑孰,过京都,以皇太后令废帝为海西公。”}}侍中谢公\myidx{谢安}见桓公拜\footnote{谢公:指时任侍中的谢安。},桓惊笑曰:“安石,卿何事至尔\footnote{安石:谢安之字。何事:为什么。尔:这样。}?”谢曰:“未有君拜于前,臣立于后。”

{\cangkai\zihao{5}【评】桓温与谢安,是当日东晋政坛两颗最为耀眼的明星。论能力二人旗鼓相当,但论势力及野心,则桓温权势熏天,势压朝廷,废立自专,谢哪能相比!桓早有觊觎帝位的勃勃野心,关键在等待时机。谢安从维护司马朝廷和其他世家望族的共同利益出发,顽强地与桓氏集团做斗争,在时机不利的情况下,则以“排调”的戏谑之言,暗中加以讽刺抨击。“未有君拜于前,臣立于后”,谓君主被废而拜于前,作为臣下则怎敢不拜?言外之旨,婉讽桓温势压君主,气盖群臣,有自立为帝的篡位阴谋。戏谑之言,令人发噱,但在笑声中,却饱含了忧国忧民的热泪。}

\lettrine{25.39} 郗重熙\myidx{郗昙}与谢公\myidx{谢安}书\footnote{郗重熙:郗昙,字重熙。鉴子。官至北中郎将,徐、兖二州刺史。谢公:指谢安。},道:“王敬仁\myidx{王修}閒(闻)一年少怀问鼎\footnote{王敬仁:王修字敬仁。濛子。参前\CJKunderwave{文学}第38则注。閒:袁本作“闻”,是。年少:年轻人。问鼎:古以鼎为国之重宝,代表社稷。问鼎之轻重,暗示意在夺取天子之位。},{\fzxk\zihao{6}\textcolor{red}{郗云(昙)、王修已见。\CJKunderwave{史记}曰:“楚庄王观兵于周郊,周定王使王孙满迎劳楚王。王问鼎大小轻重,对曰:‘在德不在鼎。’庄孙(王)曰:‘子无阻九鼎,楚国折钩之喙,足以为九鼎也。’”}} 不知桓公德衰\footnote{桓公:春秋五霸之一的齐桓公。},为复后王(生)可畏\footnote{为复:还是。后王:袁本作“后生”,是。刘注亦注作“后生”。}?”{\fzxk\zihao{6}\textcolor{red}{\CJKunderwave{春秋传}曰:“齐桓公伐楚,责苞茅之不贡。”\CJKunderwave{论语}曰:“后生可畏,正知来者之不如今。”孔安国曰:“后生,少年。”}}

{\cangkai\zihao{5}【评】此则当与前则并读而味其言外之旨。郗愔与昙兄弟,太宰鉴子。愔子超虽党于桓温,愔实不知,而始终“忠于王室”。昙为愔弟,其政治立场,同于乃兄。但处桓温势力方炽之时,难以明言。故借他人(王修)之口,以传闻中“年少怀问鼎”的流言,来对桓温的政治野心进行不点名的批判。桓公者,原指春秋五霸之一的齐桓公,这里则语带双关,同时暗喻桓温之失德,其勃勃野心,难以实现。言外讽示谢安要警惕和预防,早做准备,以防国家之不测。言简意深,见其防患于未然的忧患意识。}

\lettrine{25.40} 张苍梧\myidx{张镇}是张凭\myidx{张凭}之祖\footnote{张苍梧:张镇曾官苍梧太守,故称。张凭:字长宗,东晋吴郡人。善清言,有“理窟”之誉。见前\CJKunderwave{文学}第53则及注。},尝语凭父曰:“我不如汝。”凭父未解所以,苍梧曰:“汝有佳儿。”{\fzxk\zihao{6}\textcolor{red}{\CJKunderwave{张苍梧碑}曰:“君讳镇,字义远。吴国吴人。忠恕宽明,简正贞粹。太安中,除苍梧太守。讨王含有功,封兴道县侯。”}}凭时年数岁,敛手曰\footnote{敛手:拱手示敬。}:“阿翁\footnote{阿翁:魏晋口语,称祖父。},讵宜以子戏父\footnote{讵宜:岂可,怎么可以。讵,岂,怎么。}!”

{\cangkai\zihao{5}【评】吴郡张氏家族,是东吴江南地区的四大家族之一。从吴至于东晋,其文化承传相当深厚。张凭之善清言而称“理窟”,与其士族家学的继承和发展有关。故事所反映的张氏家庭生活,并非如传统三\CJKunderwave{礼}所规定的那样呆板教条,而是在两晋的新思潮下,另有生动风趣的面目:老父可调侃儿子,孙子可反驳“阿翁”,带有某种家庭“民主”色彩,语调诙谐幽默,老小平等对话,在欢乐嬉笑中见其生活情趣及文化内涵。}

\lettrine{25.41} 习凿齿\myidx{习凿齿}、孙兴公\myidx{孙绰}未相识\footnote{习凿齿:字彦威,襄阳人。善文史,著\CJKunderwave{汉晋春秋}。官荥阳太守,因忤桓温,降为参军。参前\CJKunderwave{言语}第72则注。孙兴公:孙绰字兴公,太原人。官至散骑常侍,著名文学家。参前\CJKunderwave{言语}第84则注。},同在桓公\myidx{桓温}坐\footnote{桓公:桓温。坐:座席。}。桓语孙:“可与习参军共语。”孙云:“‘蠢尔蛮荆’,敢与大邦为雠\footnote{“蠢尔蛮荆”二句:语出\CJKunderwave{诗经·小雅·采芑},描写周宣王南征楚国之事。古代楚可称荆,蛮则是对南方少数民族的鄙称。}!”习云:“薄伐猃狁,至于太原\footnote{“薄伐猃狁”二句:语出\CJKunderwave{诗经·小雅·六月},描写周宣王北伐猃狁获得胜利之事。}。”{\fzxk\zihao{6}\textcolor{red}{\CJKunderwave{小雅}诗也。\CJKunderwave{毛诗注}曰:“蠢,动也。荆蛮,荆之蛮也。猃狁,北夷也。”习凿齿襄阳人,孙兴公太原人,故因诗以相戏也。}}

{\cangkai\zihao{5}【评】不相识之人,甫见面则相嘲,其中必然有故。习凿齿与孙绰,皆为东晋一代文史名家。一个出于襄阳,一个太原。南北二士,俱是满腹诗书,皆引\CJKunderwave{诗经·小雅}之句以相嘲讽,可说是旗鼓相当,虽是戏言,实亦戏中有戏。结合两晋南北朝士人的对立情绪来加以考察,则为潜意识中士人积习所致,是传统偏见在作怪。南北习、孙二士的口舌之争,追究其始,是中原之士挑逗在先,“蠢尔蛮荆”,以愚蠢蛮族讥南人;而“大邦”云云,则是高自风标。尔后南士不甘示弱,习凿齿肆其机敏博辩,称引\CJKunderwave{诗经·小雅·六月}之句“薄伐猃狁,至于太原”,以北狄贬孙,针锋相对,反攻过去,属对妥帖。相互之间,如此调笑,有伤感情,实是无形内耗,不利安定团结。但自西晋以来,积重难返,属于潜意识的驱使,实也无奈他何了。}

\lettrine{25.42} 桓豹奴\myidx{桓嗣}是王丹阳\myidx{王混}外生\footnote{桓豹奴:桓嗣小字豹奴。冲子,温侄。王丹阳:指王混,恬子。},形似其舅,桓甚讳之。{\fzxk\zihao{6}\textcolor{red}{豹奴,桓嗣小字。\CJKunderwave{中兴书}曰:“嗣字恭祖,车骑将军冲子也。少有清誉。仕至江州刺史。”\CJKunderwave{王氏谱}曰:“混字奉正,中将军恬子。仕至丹阳尹。”}} 宣武\myidx{桓温}云\footnote{宣武:指桓温。}:“不恒相似\footnote{恒:经常,永久。},时似耳\footnote{时似:有时相像。}。恒似是形,时似是神。”桓逾不说\footnote{逾:更加。说:同“悦”。}。

{\cangkai\zihao{5}【评】王导六子:悦、恬、洽、协、劭、荟。悦虽为长兄而无子,以弟恬子琨为嗣。故琨袭导爵,官至丹阳尹,见载于\CJKunderwave{晋书}导传。据\CJKunderwave{世说},则“琨”当作“混”,形近而讹。王丹阳(混)是琅邪王导的嫡长孙,其门第高贵可知。作为王混外甥的桓嗣,原应感到骄傲才是。但嗣羞与母舅相似,却是为何?一来可能作为东晋四大家族之冠的琅邪王家,自王导死后,势力日蹙;而桓氏门第稍次,地位并非高贵,却势力日炽,在桓温之时,达到权势的顶点。桓、王二家,门第与权力的矛盾斗争,错综复杂。桓嗣处于家族鼎盛时期,盛气凌舅,如王献之兄弟之傲倪郗愔一样,并非特例。而且,琅邪王家子孙,并非人人优秀,王混可能就是一个“君子之泽,三世而斩”的俗物。故王世懋评曰:“观此知王混不为风流所与。”但是作为长辈,桓温调侃侄子,“恒似是形,时似是神”,外貌似娘舅,形出自然,无法更改;神情之“时似”,性情嗜好相似,才是真缺点。嘲弄的口吻,说明了桓温也不以嗣为风流人物。}

\lettrine{25.43} 王子猷\myidx{王徽之}诣谢万\myidx{谢万}\footnote{王子猷:王徽之,羲之第五子。谢万:字万石。安弟,工言论,善属文,早有时誉。参前\CJKunderwave{言语}第77则注。},林公\myidx{支遁}先在坐\footnote{林公:指东晋高僧支道林,亦称支公。},瞻瞩甚高\footnote{瞻瞩甚高:指顾盼之间的高朗神态。}。王曰:“若林公须发并全,神情当复胜此不\footnote{当复:将。}?”谢曰:“唇齿相须\footnote{相须:相依。},不可以偏亡。{\fzxk\zihao{6}\textcolor{red}{\CJKunderwave{春秋传}曰:“唇亡齿寒。”}} 须发何关于神明\footnote{神明:精神风貌。}?”林公意甚恶,曰:“士(七)尺之躯\footnote{士尺之躯:“士尺”之义不明,据袁本当作“七尺”,是。七尺之躯,古时男子汉的一般高度。\CJKunderwave{容止}第18则“庾子嵩长不满七尺”,则是身材矮小,不合男人标准高度。},今日委君二贤\footnote{委:交给,托付。}。”

{\cangkai\zihao{5}【评】林公时誉,在神不在形。据\CJKunderwave{容止}第31则,王濛生病,林公探视,守门人报称“一异人在门”,刘注谓“林公之形,信当丑异”。余嘉锡\CJKunderwave{笺疏}据此议论云:“疑道林有齞唇历齿之病。谢万恶其神情高傲,故言正复有发无关神明;但唇亡齿寒,为不可缺耳。其言谑而近虐,宜林之怫然不悦也。”虽属臆测之辞,但可备一说。其实何止谢万,王徽之因僧人秃顶无发而作为发噱谈笑之资,亦同样是恶作剧。唇齿之病,自然所出。拿别人忌讳的生理缺陷来开玩笑,正见王谢家族贵游子弟的无礼与傲慢。}

\lettrine{25.44} 郗司空\myidx{郗愔}拜北府\footnote{郗司空:郗愔,字方回,高平金乡人。太宰鉴长子。卒赠官司空。参前\CJKunderwave{品藻}第29则注。拜:拜官。北府:东晋都建康,军府设于广陵(今江苏扬州),称北府,掌朝廷重兵。后治所移镇京口。},{\fzxk\zihao{6}\textcolor{red}{\CJKunderwave{南徐州记}曰:“旧徐州都督以东为称。晋氏南迁,徐州刺史王舒加北中郎将。‘北府’之号,自此起也。”}} 王黄门\myidx{王徽之}诣郗门拜云\footnote{王黄门:王徽之字子猷,曾官黄门侍郎,故称。郗门:郗家。拜:拜贺。}:“应变将略,非其所长\footnote{应变将略,非其所长:战略应变,不是他的擅长。}。”骤咏之不已\footnote{骤咏:反复吟咏。}。郗仓\myidx{郗融}谓嘉宾\myidx{郗超}曰\footnote{郗仓:即郗融,字景山,小字仓。嘉宾:郗超字景兴,又字嘉宾。愔长子。参前\CJKunderwave{言语}第59则注。}:“公今日拜\footnote{公:仓对其父愔的敬称。},子猷言语殊不逊\footnote{殊:非常,很。不逊:不客气,不恭顺。},深不可容\footnote{深不可容:实在不可原谅。}。”{\fzxk\zihao{6}\textcolor{red}{仓,郗融小字也。\CJKunderwave{郗氏谱}曰:“融字景山,愔弟(第)二子。辟琅邪王文学,不拜,而蚤终。”}} 嘉宾曰:“此是陈寿作诸葛评\footnote{陈寿作诸葛评:陈寿\CJKunderwave{三国志·蜀书·诸葛亮传}评曰:“可谓识治之良才,管萧之亚匹矣。然连年动众,未能成功,盖应变将略,非其所长欤!”陈寿,字承祚。\CJKunderwave{三国志}的作者。},{\fzxk\zihao{6}\textcolor{red}{\CJKunderwave{蜀志}陈寿评曰:“亮连年动众而无成功,盖应变将略,非其所长也。”王隐\CJKunderwave{晋书}曰:“寿字承祚,巴西安汉人。好学,善箸述。仕至中庶子。初,寿父为马谡参军,诸葛亮诛谡,髠其父头,亮子瞻又轻寿。撰\CJKunderwave{蜀志},以爱憎为评也。”}} 人以汝家比武侯\footnote{汝家:你父。武侯:诸葛亮,三国蜀相,卒谥忠武侯。},复何所言\footnote{复何所言:还有什么话说!}!”

{\cangkai\zihao{5}【评】对于高平郗氏家族而言,鉴卒之后,势力衰替。因此,这次郗愔拜北府,作为掌控朝廷军府实权的封疆大吏,确是复兴家族门第的大事一件。故其拜官之贺,必然隆重。王徽之母是愔姐,愔是其亲娘舅,因此,外甥前来祝贺,也是势出自然。一般来说,贺辞应是大吉大利之言,这是世俗礼仪需要。但是,徽之放浪不拘礼法,只说自己想说的话,而不问吉利与否。他借陈寿评诸葛亮的话,“应变将略,非其所长”,来讥评自己的亲娘舅,态度并非友善,嘲讽调笑之意明显。当着众位来宾,“骤咏之不已”,使升任统帅之任的亲娘舅下不了台。于此见琅邪王家贵游子弟的潜意识深处的傲慢与偏见,时时作祟,即使亲娘舅也不能免。但是,徽之不知人情世故的调侃,却也是歪打正着。史称愔迷恋天师道,性好聚敛,而暗于事机,岂能是优秀将帅之才?徽之所嘲,却也有几分真实的可爱。知父莫若子。郗超的自我解嘲,也是对乃父的一种认识。超曾背父代其作笺给桓温,谓“己非将帅才,不堪军旅”,虽出于政治机变之需,却也是实话实说。徽之与郗超,二人立场视角有异,但其言论,却是英雄所见略同。}

\lettrine{25.45} 王子猷\myidx{王徽之}诣谢公\myidx{谢安}\footnote{王子猷:王徽之字子猷。参前注。谢公:谢安。},谢曰:“云何七言诗\footnote{云何:什么是。七言诗:此指七言古诗,而非唐后之七言近体诗。明徐师曾\CJKunderwave{文体明辨序说}引徐祯卿云:“七言沿起,咸曰\CJKunderwave{柏梁}。然宁戚叩牛,已肇\CJKunderwave{南山}之篇矣。”据此,七言诗之始,说或不同,有称出汉武帝时\CJKunderwave{柏梁台}诗,有谓出于先秦时代\CJKunderwave{诗经}、\CJKunderwave{楚辞}诸说。}?”{\fzxk\zihao{6}\textcolor{red}{\CJKunderwave{东方朔传}曰:“汉武帝在柏梁台上,使群臣作七言诗。”七言诗自此始也。}} 子猷承问,答曰:“昂昂若千里之驹,泛泛若水中之凫\footnote{“昂昂若千里之驹”二句:\CJKunderwave{楚辞·卜居}:“宁昂昂若千里之驹乎?将泛泛若水中之凫乎?与波上下,偷以全吾躯乎?”王徽之改\CJKunderwave{卜居}句以成此七言二句。昂昂,形容气宇轩昂振奋。驹,少壮之骏马。泛泛,随波逐流貌。凫,野鸭。}。”{\fzxk\zihao{6}\textcolor{red}{出\CJKunderwave{离骚}。}}

{\cangkai\zihao{5}【评】魏晋名士,大多熟读\CJKunderwave{楚辞}。如前\CJKunderwave{任诞}第53则王恭所言,“痛饮酒,熟读\CJKunderwave{离骚},便可称名士”。因此,王徽之熟练地化用\CJKunderwave{楚辞·卜居}之句,改为二句七言句式之诗,实际并非真正七言诗。其用心所在,如张万起、刘尚慈所说:“意思是‘宁愿昂扬如千里驹呢?还是做泛游水中的野鸭,随波起伏,以苟且偷生呢?’王子猷巧妙地用\CJKunderwave{卜居}的诗句回答谢安,以千里驹与水中凫对举来影射谢公出处之不同态势。”故事的时代背景,当在谢安隐居东山的四十岁以前。高隐山林则志气昂扬,自由奔驰;一旦出仕,则如水中之凫,随波逐流而苟且偷生。二者生命价值不一样。徽之戏谑之言,寓其人生认识,并为谢安之或出或处,暗中出谋划策。大概因谢安与王羲之友善,父执名士,故子猷排调之时,尚存善意,这在子猷身上,似不多见。}

\lettrine{25.46} 王文度\myidx{王坦之}、范荣期\myidx{范启}俱为简文\myidx{司马昱}所要\footnote{王文度:王坦之字文度。太原晋阳人。官至侍中、中书令、领北中郎将、徐兖二州刺史。参前\CJKunderwave{言语}第72则注。范荣期:范启字荣期,护军长史坚子,终于黄门侍郎。俱:一起,共同。简文:指简文帝司马昱。要:通“邀”。},范年大而位小\footnote{位小:职位低。按:下句“位大”,则谓职位高。},王年小而位大。将前,更相推在前\footnote{更相推:相互推让。},既移久\footnote{移久:很久。},王遂在范后\footnote{遂:于是。}。王因谓曰:“簸之扬之,糠秕在前。”范曰:“洮之汰之\footnote{洮汰:淘汰。洮,通“淘”,洗也。},沙砾在后。”{\fzxk\zihao{6}\textcolor{red}{王坦之、范启,已见上。一说是孙绰、习凿齿言。}}

{\cangkai\zihao{5}【评】读此故事,可有二解。一是就事论事。王坦之和范启,都是当日名士。他们先是打破官本位的意识,相互谦让,这是正面的言行,但却是表面的文章;接着的对话,引经据典,炫耀学问,句对虽然生动贴切,但说话尖酸刻薄,相互奚落调侃的同时,表现出名士相轻的陋习,这虽是负面的小动作,但却是从潜意识深处流露出来的实质认识。一正一反,一虚一实,生动地传达了故事的复杂内涵和诸多信息。}

{\cangkai\zihao{5}另一解正相反。故事前半部分与后半部分发展逻辑自相矛盾,似乎是自我否定。史称王坦之有风格,“尤非时俗放荡,不敦儒教”,意在调和儒玄孔老之间。为人朴实稳重,言不及私,惟忧国家之事,是个谦谦君子,怎会在谦让之后突然主动挑衅,侮人为秕糠呢?这不符合王坦之的人生哲学。而作为范启,其反击当然有足够的理由。但他也不是放荡不拘之士,而是“以才义显于当世”的名士,与清谈之士庾和、韩伯、袁宏友善。他虽有矜饰之病,但还不至于轻薄。古人对于这一难解的矛盾,有自己的解释,如刘辰翁评曰:“二语易位乃可。”此翁眼光如炬,令人信服。王、二人对话颠倒一下,王在后而自谓沙砾,范在前而自称糠秕,于是故事自然发展,前后逻辑顺理成章。}

{\cangkai\zihao{5}二解孰是?留待读者自辨。}

\lettrine{25.47} 刘遵祖\myidx{刘爰之}少为殷中军\myidx{殷浩}所知\footnote{刘遵祖:刘爰之字遵祖。殷中军:殷浩曾任中军将军,故称。称:推扬,推荐。},称之于庾公\myidx{庾亮}\footnote{庾公:庾亮。},庾公甚忻\footnote{忻:同“欣”。欢喜,高兴。按:袁本“忻”下有“然”字,亦通。},便取为佐。既见,坐之独榻上\footnote{独榻:只坐一人的席位,以示尊重。},与语。刘尔日殊不称\footnote{殊不称:很不称意。},庾小失望,遂名之为“羊公鹤”。昔羊叔子\myidx{羊祜}有鹤善舞\footnote{羊叔子:羊祜字叔子。西晋开国元勋之一。参前\CJKunderwave{言语}第86则注。 善舞鹤:据\CJKunderwave{舆地纪胜}卷六四云:“晋羊祜镇荆州,江陵泽中多有鹤,常取之教舞以娱宾客。”},尝向客称之,客试使驱来,氃氋而不肯舞\footnote{氃氋(tónɡ ménɡ童蒙):联绵词,羽毛松散委顿的样子。},故称比之。{\fzxk\zihao{6}\textcolor{red}{徐广\CJKunderwave{晋纪}曰:“刘爰之字遵祖,沛郡人。少有才学,能言理。历中书郎、宣城太守。”}}

{\cangkai\zihao{5}【评】殷浩一代清谈名士,经其推荐,即入胜流。此庾亮所以“独榻”以待宾也。但并非名人所荐,个个名流。盛名之下,亦有名不符实之时。庾亮面谈,一经深入,即是实际检验。刘爰之经不起反复推敲,在于肚中非具真才实学,故庾氏以“羊公鹤”加以比拟。但教训又不仅在刘,更在推荐者的眼光器识,是深入了解,还是浮光掠影的扫描?为国荐才,责任重大,能不慎乎!}

\lettrine{25.48} 魏长齐\myidx{魏顗}雅有体量\footnote{魏长齐:魏顗字长齐。据余嘉锡\CJKunderwave{笺疏}引程炎震曰:“\CJKunderwave{金楼子·立言篇}作‘魏长高’。”余氏据此以为“长齐当作长高,草书相近之误耳”。说可参考。体量:气量,体识。},而才学非所经\footnote{才学:才情学问。经:擅长。}。初宦当出\footnote{初宦:初次做官。当出:将要出任。},虞存\myidx{虞存}嘲之曰\footnote{虞存:字道长,会稽山阴人。官至吏部郎。参\CJKunderwave{政事}第17则注。}:“与卿约法三章:谈者死\footnote{谈:清谈。},文笔者刑\footnote{文笔:作动词用。有韵为文,无韵为笔,泛指诗文写作。},商略抵罪\footnote{商略:商议,议论,引申为品评和鉴赏人物。}。”魏怡然而笑\footnote{怡然:安乐喜悦貌。},无忤于色\footnote{忤:抵触。}。{\fzxk\zihao{6}\textcolor{red}{\CJKunderwave{魏氏谱}曰:“顗字长齐,会稽人。祖胤,处士。父说,大鸿胪卿。顗仕至山阴令。”\CJKunderwave{汉书}曰:“沛公入咸阳,召诸父老曰:‘天下苦秦可(苛)法久矣,今与父老约法三章耳:杀人者死,伤人及盗抵罪。’”应劭注曰:“抵,至也,但至于罪。”}}

{\cangkai\zihao{5}【评】据\CJKunderwave{赏誉}第85则:“会稽孔沈、魏顗、虞球、虞存、谢奉并是四族之俊,于时之杰。”可见在会稽地方,魏顗与虞存并举齐名,都是一时人杰。但在魏顗初任官时,虞存却借机对魏顗大加嘲讽一番。清谈玄理、吟诗作赋、品鉴人物,是魏晋士人心目中极其高雅之事。不过虞存认为魏顗“才学非所经”,此三项本领是其所短,因戏改汉高祖旧约法三章为新约法三章,嘲弄魏顗缺乏儒雅风流的资质。此虽玩笑戏言,却也是名士相轻的陋习作怪。而魏顗面对嘲讽,却是“怡然而笑”,无须回答,便见雅量与胸怀。故梁元帝萧绎在\CJKunderwave{金楼子·立言篇上}评云:“更觉长高(魏顗)之为高,虞存之为愚也。”轻人者反而自轻自慢,能无慎哉!}

\lettrine{25.49} 郗嘉宾\myidx{郗超}书与袁虎\myidx{袁宏}\footnote{郗嘉宾:郗超字景兴,一字嘉宾。桓温谋士。参前\CJKunderwave{言语}第59则注。袁虎:袁宏字彦伯,小字虎。曾为桓温大司马记室参军,后为东阳太守。参前\CJKunderwave{言语}第83则注。},道戴安道\myidx{戴逵}、谢居士\myidx{谢敷}云\footnote{戴安道:戴逵字安道。参前\CJKunderwave{雅量}第34则注。谢居士:谢敷字庆绪,会稽人。崇信释氏,隐居不仕,故称居士。}:“恒任之风\footnote{恒:永恒,持久。任:责任。},当有所弘耳\footnote{当:应该。弘:同“宏”,弘扬。}。”以袁无恒,故以此激之。{\fzxk\zihao{6}\textcolor{red}{袁、戴、谢,并已见。}}

{\cangkai\zihao{5}【评】郗超与袁宏,俱为一代英杰,二人皆为桓温幕府重要僚属,关系密切。比较而言,郗超为登堂入室之宾,作为谋主更获桓温宠信。他写信给袁宏,排调之言,出于善意。因宏缺乏恒心,则难成大事。故超道戴逵、谢敷等高士风节加以砥砺。“恒任之风,当有所弘耳”,一语双关,既讲一个普通的人生常理,又通过袁宏名讳之释义,鼓励他发扬“恒任”传统精神,从而克服自己那“无恒”的缺点。郗超意在为桓温收罗和培养人才。}

\lettrine{25.50} 范启\myidx{范启}与郗嘉宾\myidx{郗超}书曰\footnote{范启:字荣期,护军长史坚子,终于黄门侍郎。郗嘉宾:即郗超,参前则注。}:“子敬\myidx{王献之}举体无饶纵\footnote{子敬:王献之,字子敬。羲之少子。参前\CJKunderwave{德行}第39则注。举体:全身。饶纵:丰腴。},掇皮无馀润\footnote{掇皮:去皮。馀润:膏泽。}。”郗答曰:“举体无馀润,何如举体非真者\footnote{何如:比……怎么样。}?”范性矜假多烦\footnote{矜假:矜饰虚假。},故嘲之。

{\cangkai\zihao{5}【评】魏晋名士,以品评人物为风流雅事。但范启评王献之,则有名士相轻之弊,时献之“风流为一时之冠”,批评风流领袖的不是,在当日是一种炒作以抬高自我身份的手段。其调弄并非幽默,而是诋毁。因其心非善良,故招来郗超的不满。尽管郗超与王献之政治主张对立,但他对范启“矜假”而有违自然的虚伪做作更为讨厌。献之“举体无饶纵”,身瘦干巴,缺乏润泽,当然不是优点;但与范启的“举体非真”——全身没有一点真正属于自己的东西相比,谁的缺陷更大呢?史称范启继承父亲坚的经学,则其重礼教之矜饰,也是可以理解。郗超的反批判,正见魏晋名士心灵深处更推崇的是自然的真实。}

\lettrine{25.51} 二郗\myidx{郗愔}\myidx{郗昙}奉道\footnote{二郗:指郗愔与昙兄弟。},二何\myidx{何充}\myidx{何准}奉佛\footnote{二何:指何充与准兄弟。二郗、二何生平仕履,均参前注。},皆以财贿\footnote{财贿:指为奉佛、道二教而耗费大量财货。}。谢中郎\myidx{谢万}云\footnote{谢中郎:指谢万,曾任抚军从事中郎,故称。}:“二郗谄于道\footnote{谄:谄媚,奉承。},二何佞于佛\footnote{佞:巧言媚人。}。”{\fzxk\zihao{6}\textcolor{red}{\CJKunderwave{中兴书}曰:“郗愔及弟昙奉天师道。”\CJKunderwave{晋阳秋}曰:“何充性好佛道,崇修佛寺,供给沙门以百数。久在扬州,征役吏民,功赏万计,是以为遐迩所讥。充弟准亦精勤,读佛经、营治寺庙而已。”}}

{\cangkai\zihao{5}【评】由于士族门第之高贵,二郗及二何兄弟,皆身居高位。以谄道佞佛的愚蠢迷信之人占据朝廷之要津,国家怎能兴旺发达?谢万之调弄嘲讽,入情入理。但谢万也是清醒于一时。他本人以王谢家族贵游子弟身份,傲慢轻狂,眼中何曾有人?王羲之致信教戒而不醒,乃兄规劝而不听,故有北伐败归之废。不仅误己,更误国家。悲哉!}

{\cangkai\zihao{5}另:余嘉锡引\CJKunderwave{高僧传}佛图澄曰:“事佛在于清静无欲,慈矜为心。檀越虽仪奉大法,而贪吝不已,游猎无度,积聚不穷,主受现世之罪,何福报之可希耶?”如二郗二何之谄道佞佛,敛财营寺,“非惟达识之所讥,亦古德高僧所不许也。”菩萨至善,在于一心之虔诚,而不在礼佛财宝之多寡。此一胜解,启人至深。}

\lettrine{25.52} 王文度\myidx{王坦之}在西州\footnote{王文度:王坦之字文度。太原晋阳人。官至侍中、中书令、领北中郎将、徐兖二州刺史。西州:东晋扬州刺史治所,原设在台城西,故称西州。},与林法师\myidx{支遁}讲\footnote{林法师:支遁字道林,东晋高僧。讲:清谈玄理。},韩\myidx{韩伯}、孙\myidx{孙绰}诸人并在坐\footnote{韩、孙:指韩伯与孙绰诸人。},林公理每欲小屈\footnote{屈:亏,挫折。}。孙兴公曰:“法师今日如箸弊絮在荆棘中\footnote{弊絮:破旧丝棉絮。},触地挂阂\footnote{触地挂阂:处处挂碍。}。”

{\cangkai\zihao{5}【评】魏晋清谈盛况,具体可见。支遁高僧,为弘扬佛学而先谈玄理,是佛学中国化的前奏曲。支遁理论修养颇深,其讲\CJKunderwave{庄子·逍遥游},新义迭出,自成一家之言。王坦之诸人,能在清谈辩论中令其“理每欲小屈”,亦见诸名士“环攻”支公之用心。难倒名家,则其高明更在名家之上。故孙绰幸灾乐祸,谓支公如著破絮行走在荆棘中,处处有碍,不得自由发挥。清谈小胜之乐,浮现于名家脸上。我国理论思辨的发展,就是在一次次的清谈论辩中,见造微之功,从而获得了一步步的提高与发展。}

\lettrine{25.53} 范荣期\myidx{范启}见郗超\myidx{郗超}俗情不淡\footnote{范荣期:范启字荣期。郗超:小字嘉宾。桓温谋主。俗情不淡:世俗之情不轻。},戏之曰:“夷、齐、巢、许一诣垂名\footnote{夷、齐、巢、许:指古代四大高隐名士伯夷、叔齐、巢父、许由。一诣垂名:一下子名垂青史。},必劳神苦形\footnote{必:何必。据袁本,“必”上有“何”字。},支策据梧邪\footnote{支策据梧:拄杖凭几。典出\CJKunderwave{庄子·齐物论}:“师旷之枝策也,惠施之据梧也。”师旷为春秋时晋平公乐师,精音律鼓琴。其支策拄杖,见精神劳顿之态。惠施为战国时名家代表人物,善辩名理,其倚靠梧几而瞑,更见耗费心血疲惫之状。}?”郗未答,韩康伯曰:“何不使游刃皆虚\footnote{游刃皆虚:意谓遵循自然规律,则无不自由自在而任我所行。语出\CJKunderwave{庄子·养生主}庖丁解牛的寓言故事。}?”{\fzxk\zihao{6}\textcolor{red}{\CJKunderwave{庄子}曰:“昭文之鼓琴,师旷之支策,惠子之据梧,三子之智几矣,皆其盛也,故载之末年。”“庖丁为文惠君解牛,三年之后,未尝见全牛也。用刀十九年矣,所解千牛,而刀刃若新发于硎。文惠君问之,庖丁曰:‘彼节者有间,而刀刃无厚。以无厚入有间,恢恢乎其于游刃必有馀地。’”}}

{\cangkai\zihao{5}【评】范启不仅通儒经,而且与清谈之士庾和、韩伯、袁宏友善,是一个儒、玄双修的人物。谈玄之人,高倡\CJKunderwave{庄}、\CJKunderwave{老}、\CJKunderwave{周易},原该淡泊功名而眼无俗物。史称郗超“善谈论,理精微”,为支遁所知赏,称其“一时之俊”,亦是一代清谈名士。所不同者,他是个清谈而不忘政治的厉害角色。故以“俗情不淡”见讥于范启。“何必劳神苦形支策据梧”,以玄家熟知之\CJKunderwave{庄}相谏,言语生动而蕴藉。韩康伯以“游刃皆虚”代答,则又更进一层,重新回到玄家的立场。以虚入实,则游刃有馀。这大概和佛教\CJKunderwave{般若波罗蜜多心经}“色即是空,空即是色”的意思相似,色空虚实之间,关键在自己的不沾不滞而明道见性。但实际是政治大于理论,郗超郁郁而终,何能“游刃有馀”而超越功名?世俗物累害人不浅!}

\lettrine{25.54} 简文\myidx{司马昱}在殿上行\footnote{简文:指简文帝司马昱,桓温扶立为帝,在位两年崩。},右军\myidx{王羲之}与孙兴公\myidx{孙绰}在后\footnote{右军:指王羲之,曾官右军将军,故称。孙兴公:孙绰字兴公。参前\CJKunderwave{言语}第84则注。}。右军指简文语孙曰:“此啖名客\footnote{啖(dàn但)名客:贪求名声之人。}。”简文顾曰:“天下自有利齿儿\footnote{利齿儿:伶牙俐齿、能言善辩之人。}。”后王光禄\myidx{王蕴}作会稽\footnote{王光禄:王蕴,字叔仁,小字阿兴。濛子。作吴兴、晋陵二郡时颇有德政。官至尚书左仆射、镇军将军、会稽内史。卒赠左光禄大夫,故称。\CJKunderwave{晋书·外戚}有传。},谢车骑\myidx{谢玄}出曲阿祖之\footnote{谢车骑:谢玄,参前\CJKunderwave{言语}第78则注。 曲阿:县名,在今江苏丹阳。祖:祖道饯行以送别。},{\fzxk\zihao{6}\textcolor{red}{王蕴、谢玄,已见。}} 王孝伯\myidx{王恭}罢秘书丞\footnote{王孝伯:王恭字孝伯,蕴子。官前将军、都督兖青冀幽并徐州晋陵诸军事、兖青二州刺史。后起兵清君侧,兵败被杀。 罢秘书丞:王恭罢秘书丞,旋即升任中书郎,未拜,丁父忧。},在坐,谢言及此事,因视孝伯曰:“王丞齿似不钝。”王曰:“不钝,颇亦验。”

{\cangkai\zihao{5}【评】这则故事颇难解读。故事前半部分余氏\CJKunderwave{笺疏}引殷芸\CJKunderwave{小说},疑“啖名客”是“啖石客”之讹,右军所指对象是孙绰而非简文。因为简文贵为帝王,右军并非狂诞之徒,“安敢如此轻相戏侮”?此解缺乏版本及其他可靠证据,只可备一说参考而已。多数研究者仍是就事论事,右军嘲讽简文是“啖名客”,简文回头反唇相讥王为“利齿儿”。其时简文虽未即位,但在穆帝永和年间,头衔很多,封琅邪王而不去会稽王号,侍中、抚军大将军、司徒、丞相、录尚书事,相王之尊,头上光环炫人眼目,有名过其实之嫌,故右军有“啖名客”之调。从年辈看,右军大简文一辈,长者戏言,并非“狂诞”。而简文虽贵为相王,但个人喜清言而善玄理,常与清谈之士聚会,故君臣之间,无所隔阂,与右军争口舌,正见其颇富古代“民主”意识之色彩。下半部分则是另一相似故事。谢玄以往事相比拟,讥王恭“齿似不钝”——即“利齿儿”的委婉说法。王恭“不钝,颇亦验”之答,谓己齿虽利,但见实效,并非徒逞口舌之辩,这是从政治实际出发。朱铸禹\CJKunderwave{汇校集注}称:“验,谓坐言语罢官也。”按:王恭年轻时仕途一帆风顺,其“罢秘书丞”,旋迁中书郎,并非罢官失意之言。朱解误。}

\lettrine{25.55} 谢遏\myidx{谢遏}夏月尝仰卧\footnote{谢遏:指谢玄,遏是其小名。安侄。淝水大战中的名将,卒赠车骑将军。夏月:夏天,夏季。仰卧:仰天卧眠。},谢公\myidx{谢安}清晨卒来\footnote{谢公:谢安。卒来:突然来到。卒,通“猝”。},不暇箸衣\footnote{不暇箸衣:来不及穿衣服。},跣出屋外\footnote{跣:赤脚。},方蹑履问讯\footnote{蹑履:穿鞋示敬。按,在非正式场合,魏晋士人平常著屐,以示闲适自由。在见长辈的正式场合,则“蹑履”示敬。}。公曰:“汝可谓前倨而后恭\footnote{前倨后恭:成语,先傲慢而后谦恭。}。”{\fzxk\zihao{6}\textcolor{red}{\CJKunderwave{战国策}曰:“苏秦说惠王而不见用,黑貂之裘弊,黄金百斤尽,大困而归。父母不与言,妻不为下机,嫂不为炊。后为从长,行过洛阳,车骑辎重甚众,秦之昆弟妻嫂,侧目不敢视。秦笑谓其嫂曰:‘何先倨而后恭?’嫂谢曰:‘见季子位高而金多。’秦叹曰:‘一人之身,富贵则亲戚畏惧,贫贱则轻易之,而况于他人哉!’”}}

{\cangkai\zihao{5}【评】魏晋士族,大多是兄弟大排行,过的是大家族的生活。谢玄父奕,在兄弟中是长兄,但与二弟据俱早卒,因此,老三谢安,就是当然的谢氏家长。他对侄儿玄,不仅视如己出,宠爱有加,而且严格教育,盼其成材。此所谓愿芝兰玉树生于庭中也。玄对谢安,敬如严父。夏天暑热之时,赤膊光脚仰天卧睡,是纳凉时的一种自由闲适之态。“谢公清晨卒来”,打破侄儿清梦,玄“跣出屋外”,出于意外,并非其过。后按礼仪,“蹑履问讯”,亡羊补牢,以示尊敬。于此可见,衣冠可以改变人的形象。安“前倨而后恭”之戏,言辞生动准确,幽默风趣,在玩笑调弄中,逗露出宠爱儿辈的善意,更增添了士人家庭生活之乐趣。}

\lettrine{25.56} 顾长康\myidx{顾恺之}作殷荆州\myidx{殷仲堪}佐\footnote{顾长康:顾恺之字长康,晋陵(今江苏无锡)人。东晋著名画家。参前\CJKunderwave{言语}第88则注。殷荆州:指殷仲堪,时任荆州刺史。佐:府佐,僚属。},请假还东\footnote{还东:顾恺之家乡晋陵在长江下游。从荆州顺游东下,故称还东。}。尔时例不给布颿\footnote{布颿:原指船帆,此泛指帆船。颿通“帆”。},顾苦求之\footnote{苦求:苦苦要求,竭力争取。},乃得。发至破冢\footnote{破冢:地名,长江的一个小洲,在今湖北江陵县东。},遭风大败\footnote{败:败坏,破坏。}。{\fzxk\zihao{6}\textcolor{red}{周祇\CJKunderwave{降(隆)安记}曰:“破冢,洲名,在华容县。”}} 作牋与殷云\footnote{牋 :今作“笺”。刘勰云:“牋者,表也,识表其情也。”东汉后郡将向府主汇报称奏牋。实际是一种上行奏事的书信。}:“地名破冢,真破冢而出,行人安稳,布颿无恙。”

{\cangkai\zihao{5}【评】史称顾恺之博学多才,是一个文艺天才,素有“才绝、画绝、痴绝”的“三绝”之誉。谢安誉美其画,“以为有苍生以来未之有也”。其为人好谐谑,人多爱狎之。但常“矜伐过实,少年因相称誉以为戏弄”。诙谐、滑稽、幽默,与其文艺天才,配合得天衣无缝。其上殷仲堪笺,“地名破冢,真破冢而出”,借实际地名,描述自己如从败坟古墓中夺路而出,死里逃生。虽极惊险,但却语带诙谐,态度从容。“行人安稳,布颿无恙”,则又故意颠倒语序,错相搭配,以求既不违背事实,又见语言修辞之艺术效果。如余氏\CJKunderwave{笺疏}所评:“盖本当云:‘布帆安稳,行人无恙。’因帆已破败,不可言安稳,故易其语以见意。此乃以文滑稽耳。”当他回荆州交差时,布颿虽败而入库,岂非“无恙”?自己终于归来,岂非“安稳”!大风大浪中仍然激发其文学天才,风险中仍然笑声爽朗,真是不可多得的天才!}

\lettrine{25.57} 苻朗\myidx{苻朗}初过江\footnote{苻朗:字元达。前秦苻坚从兄之子。刘注谓“从兄”,误。官青州刺史,后降晋,用为散骑侍郎。为人恃才傲物,后被谗杀。初过江:指苻朗降晋后初至京师建康。},{\fzxk\zihao{6}\textcolor{red}{裴景仁\CJKunderwave{秦书}曰:“朗字元达,苻坚从兄。性宕放,神气爽悟。坚常曰:‘吾家千里驹也。’坚为慕容冲所围,朗降谢玄,用为员外散骑侍郎。吏部郎王忱与兄国宝命驾诣之。沙门法太问朗曰:‘见王吏部兄弟未?’朗曰:‘非一狗面人心,又一人面狗心者是邪?’忱丑而才,国宝美而很故也。朗常与朝士宴,时贤并用唾壶,朗欲夸之,使小儿跪而开口,唾而含出。又善识味,会稽王道子为设精馔,讫,问:‘关中之食,孰若于此?’朗曰:‘皆好,唯盐味小生。’即问宰夫,如其言。或人杀鸡以食之,朗曰:‘此鸡栖恒半露。’问之,亦验。又食鹅炙,知白黑之处。咸试而记之,无毫厘之差。箸\CJKunderwave{苻子}数十篇,盖老庄之㳅(流)也。朗矜高忤物,不容于世,后众谗而杀之。”}} 王咨议\myidx{王肃之}大好事\footnote{王咨议:王肃之,羲之第四子。仕履见注。},问中国人物及风土所生\footnote{中国:此指中原地区。},终无极已\footnote{终无极已:没完没了。},{\fzxk\zihao{6}\textcolor{red}{\CJKunderwave{王氏谱}曰:“肃之字幼恭,右将军羲之弟(第)四子。历中书郎、骠骑咨议。”}} 朗大患之\footnote{大患:非常讨厌。}。次复问奴婢贵贱,朗云:“谨厚有识中者\footnote{谨厚:恭谨朴实。有识中:有识见。魏晋六朝时,谓得其当者为“中”,如“理中”、“事中”等。},乃至十万;无意为奴婢问者\footnote{无意:无识见。为奴婢问:只问奴婢之事。},止数千耳。”

{\cangkai\zihao{5}【评】苻朗原为前秦苻坚家族中坚,官青州刺史。太元九年(384),谢玄伐秦,取河南,取青州,朗降于玄。据此,故事当发生在太元九年以后若干年中。苻朗是前秦贵族中的佼佼者,史称其“动怀远操,不屑时荣”。谈虚语玄,手不释卷。但又恃才傲物,常自夸诞,虽渡江降晋而本性依然。于是南北贵族的相互调侃戏言中,又潜藏了一番唇枪舌剑。苻朗风流超迈,志凌万物,但作为降官,身份改变,又不得不与江东士人应酬。王肃之虽然出身琅邪王氏家族,但本人颇俗。问奴婢价,即是一例。故朗患之,一语双关,讽刺肃之如“无意为奴婢问者”,是个心中无知无识的俗物,其身价“止数千耳”,何劳动问?肃之以此自讨没趣。但后来苻朗也因其“忤物侮人”的狂傲,为王国宝谗杀,从而付出了生命的代价。临刑,绝命赋诗:“旷此百年期,远同嵇叔子。命也归自天,委化任冥纪。”死前啸咏自若,自我调侃,真通\CJKunderwave{庄}、\CJKunderwave{老}之高人也。}

\lettrine{25.58} 东府客馆是版屋\footnote{东府:东晋扬州刺史府第原在台城西,称西府。自会稽王司马道子兼领扬州之时,其府第在州东,故时人号为东府。版屋:即板屋,用木板盖的房屋。}。谢景重\myidx{谢重}诣太傅\myidx{司马道子}\footnote{谢景重:谢重字景重,朗子,陈郡阳夏人。其人秀有才具,终骠骑长史。参\CJKunderwave{言语}第98则注。太傅:指会稽王司马道子。简文帝子,时任太傅,故称。},时宾客满中\footnote{满中:一屋充满。},初不交言\footnote{初不:全然不。},直仰视云\footnote{直:只。}:“王乃复西戎其屋\footnote{乃复:竟然。西戎其屋:把房间装饰成西戎的木板屋。语出\CJKunderwave{诗经·秦风·小戎}:“在其版屋,乱我心曲。”版屋,以所居之木板屋喻西戎。\CJKunderwave{汉书·地理志}称:“天水郡陇西,山多林木,民以板为室屋。故\CJKunderwave{秦诗}曰:‘在其版屋。’”}。”{\fzxk\zihao{6}\textcolor{red}{\CJKunderwave{秦诗}叙曰:“襄公备其兵甲,以讨西戎。妇人闵其君子,故作。”诗曰:“在其版屋,乱我心曲。”毛公注曰:“西戎之版屋也。”}}

{\cangkai\zihao{5}【评】自谢安后,陈郡阳夏谢氏家族迅速跻升于一流门阀士族,与琅邪王氏并称王谢家族。因此,谢重的狂傲,有其出身与时代背景,是贵游子弟门阀意识作祟。满屋之人,谢重两眼望天,而不与交一言,连寒暄一下都懒。他引\CJKunderwave{诗经·秦风·小戎}“在其版屋,乱我心曲”之句,一指版屋内众宾皆为俗物,不值一言。更深一层,则直逼府主司马道子这一执掌国政的人物。前\CJKunderwave{言语}第101则载,重为桓玄而冲撞道子。现又以“乱我心曲”云云,暗喻道子将乱天下。狂傲之中,同时又见超前的忧患意识。}

\lettrine{25.59} 顾长康\myidx{顾恺之}啖甘蔗\footnote{顾长康:顾恺之字长康。参前\CJKunderwave{言语}第88则注。啖:吃。},先食尾。人问所以,云:“渐至佳境。”

{\cangkai\zihao{5}【评】“渐至佳境”,成语又作“渐入佳境”。恺之语虽为一般的生活经验,但体悟颇深,同样适合于生活的方方面面。人们知道,甘蔗的根部最甜,糖分最高,而尾部则糖分渐减,吃到最后,甚至有点咸味。一般人食蔗有二法:一是先吃根部,由下往上,把最甜的吃掉后,常舍弃尾部不食;一是如顾恺之,由尾到根,从上往下,越吃越甜,不仅一点也不浪费,而且有引人入胜的渐至佳境之感。“渐入佳境”或“渐至佳境”已化为成语,比喻生活越来越甜,境况渐好或兴会愈浓。恺之的天才悟性,启人至多。}

\lettrine{25.60} 孝武\myidx{司马曜}属王珣\myidx{王珣}求女婿\footnote{孝武:东晋孝武帝司马曜,简文帝子,在位二十四年。属:通“嘱”,托付。王珣:字元琳,导孙。爵东亭侯。官至尚书令。参前\CJKunderwave{言语}第102则注。},曰:“王敦\myidx{王敦}、桓温\myidx{桓温},磊砢之不(流)\footnote{王敦、桓温:王敦、桓温皆为驸马,敦尚武帝女襄城公主,温尚晋明帝女南康长公主。磊砢之不:袁本“不”作“流”,是。磊砢之流,俊伟卓异之人。磊砢,原指树大多节,后用以喻人之奇才异节。},既不可复得,且小如意\footnote{小:稍。},亦好豫人家事\footnote{好豫人家事:指王敦和桓温二人手握兵权,野心勃勃,觊觎帝位。豫,通“预”,干预。家事,皇家之事,指帝位大权。},酷非所须\footnote{酷:极。}。正如真长\myidx{刘惔}、子敬\myidx{王献之}比\footnote{真长:指刘惔。子敬:指王献之。刘、王二人皆为驸马,刘尚明帝女庐陵公主,王尚简文帝女新安公主。 如……比:像……一样。},最佳。”珣举谢混\myidx{谢混}\footnote{谢混:字叔源,小字益寿,安孙。官至中领军、尚书仆射。其文才一代之英。尚孝武帝女晋陵公主。}。后袁山松\myidx{袁山松}欲拟谢婚\footnote{袁山松:与谢混同为陈郡阳夏(今河南太康)人。参前\CJKunderwave{任诞}第43则注。},{\fzxk\zihao{6}\textcolor{red}{\CJKunderwave{续晋阳秋}曰:“山松,陈郡人。祖乔,益州刺史。父方平,义兴太守。山松历秘书监、吴国内史。孙恩作乱,见害。初,帝为晋陵公主访婿于王珣,珣举谢混,云:‘人才不及真长,不减子敬。’帝曰:‘如此便已足矣。’”}} 王曰:“卿莫近禁脔\footnote{禁脔:喻皇家禁物,外人不得染指。禁,宫禁。脔,肉块。语出\CJKunderwave{晋书·谢混传}:“元帝始镇建业,公私窘罄,每得一㹠,以为珍膳,项上一脔尤美,辄以荐帝,群下未尝敢食。于时呼为禁脔。故珣因以为戏。”}?”

{\cangkai\zihao{5}【评】孝武帝为女求婿,煞费苦心。如刘辰翁评曰:“谋婿至矣。”皇帝也是人,爱女心切,体现了人性的真实。但皇家求婿,重在政治品格,也就是政治上的可靠,而非关心儿女的男女感情。孝武提到的王敦、桓温“好豫人家事”,对皇权构成直接威胁,政治上不可靠,人再英俊能干,也不在考虑之列。至于王献之与新安公主,婚后感情不佳,但政治上对皇家有利,因此作为选婿的典范提出。其择婿标准是政治第一,少问情感。这就为公主与驸马的婚姻生活,常是带来了感情的悲剧,顾此失彼,能无悔乎?}

\lettrine{25.61} 桓南郡\myidx{桓玄}与殷荆州\myidx{殷仲堪}语次\footnote{桓南郡:指桓玄,温子,袭爵南郡公,故称。殷荆州:殷仲堪,孝武时拔任荆州刺史,故称。语次:谈话之时。},因共作了语\footnote{了语:意思终了的话。}。顾恺之\myidx{顾恺之}曰:“火烧平原无遗燎。”桓曰:“白布缠棺竖旒旐\footnote{旒旐(liú zhào 流兆):灵前旗幡。旐,丧葬的招魂幡。旒,旗上飘带一类装饰。}。”殷曰:“投鱼深渊放飞鸟。”次复作危语\footnote{危语:以危险之事做隐语。}。桓曰:“矛头浙(淅)米剑头炊\footnote{矛头浙(淅)米剑头炊:在枪矛小凹槽中淘米,以利剑支锅做饭。喻极危险。}。”殷曰:“百岁老翁攀枯枝。”顾曰:“井上辘轳卧婴儿。”殷有一参军在坐\footnote{参军:官名,军府重要僚佐。},云:“盲人骑瞎马,夜半临深池。”殷曰:“咄咄逼人。”仲堪眇目故也。{\fzxk\zihao{6}\textcolor{red}{\CJKunderwave{中兴书}曰:“仲堪父尝疾患经时,仲堪衣不解带数年。自分剂汤药,误以药手拭泪,遂眇一目。”}}

{\cangkai\zihao{5}【评】这是东晋名士的一场语言游戏,既展示了形象思维的文学水平,更充分表现了语言修辞的譬喻艺术。前三句作“了语”,是精美的联句,“燎”、“旐”、“鸟”押韵,同属上声十七篠。三句隐语,都有终了、了结之义。至于后半四句危语,则其形象隐喻,一句比一句惊险。“矛头淅米剑头炊”,余嘉锡评曰:“此不过言于战场中造饭,死生呼吸,所以为危也。”战场之事,尚是生死未卜,而“百岁老人攀枯枝”,“井上辘轳卧婴儿”,则是必死无疑。其险更甚于战场。至于参军之句,尤其令人震骇,其语对仗工整,同时又语带双关。盲人瞎马,影射殷仲堪眇目之病。殷是其府主上司,但参军却故意讽刺,话语相当刻薄。故殷有“咄咄逼人”之叹。但也仅此而已,并未加以打击报复,这说明魏晋士人思想较为开明。如置于明清之时,则后果可想而知。}

\lettrine{25.62} 桓玄\myidx{桓玄}出射\footnote{出射:外出靶场射箭。},有一刘参军与周参军朋赌\footnote{朋赌:分组比赛。一朋,即一组。},垂成,唯少一破\footnote{破:破的,射中箭靶。}。刘谓周曰:“卿此起不破\footnote{起:发。},我当挞卿\footnote{挞:鞭笞。}。”周曰:“何至受卿挞?”刘曰:“伯禽之贵\footnote{伯禽:周公长子。周初封鲁公。},尚不免挞,而况于卿?”{\fzxk\zihao{6}\textcolor{red}{\CJKunderwave{尚书大传}曰:“伯禽与康叔见周公,三见而三笞。康叔有骇色,谓伯禽曰:‘有商子者,贤人也,与子见之。’乃见商子而问焉。商子曰:‘南山之阳有大(木)焉,名乔。二三子往观之。’见乔,实高高然而上,反以告商子。商子曰:‘乔者,父道也。南山之阴有木焉,名曰梓。二三子复往观焉。’见梓,实晋晋然而俯。反以告商子。商子曰:‘梓者,子道也。’二三子明日见周公,入门而趋,登堂而跪。周公拂其首,劳而食之,曰:‘尔安见君子乎!’”\CJKunderwave{礼记}曰:“成王有罪,周公则挞伯禽。”亦其义也。}} 周殊无仵色\footnote{仵:与“忤”通,抵触。}。桓语庾伯鸾\myidx{庾鸿}曰\footnote{庾伯鸾:庾鸿,亮玄孙。}:{\fzxk\zihao{6}\textcolor{red}{\CJKunderwave{晋东宫百官名}曰:“庾鸿字伯鸾,颍川人。”\CJKunderwave{庾氏谱}曰:“鸿祖义(羲),吴国内史。父棤(楷),左卫将军。鸿仕至辅国内史。”}} “刘参军宜停读书,周参军且勤学问。”

{\cangkai\zihao{5}【评】桓玄其人,终因篡位被杀。但魏晋六朝之士,不以成败论英雄。桓氏温、玄父子,仍是文人津津乐道的风流人物。在\CJKunderwave{排调}门中,桓玄故事共六则,成为重要角色。史称桓玄“风神疏朗,博综艺术,善属文”,故常自负才地,讥嘲人物。“刘参军宜停读书,周参军且勤学问”,其批评取俯视轻蔑的眼光,不仅因其府主的地位,更在于其才情学问高高凌越于二位参军之上,这是一种内在精神批评,而非仅以权势压人的政治断语。如余嘉锡所评:“刘滥引故事,比拟不伦,以\CJKunderwave{书传}资其利口,故曰宜停读书。周被骂而无忤色,盖不知伯禽为何人,故曰‘且勤学问’。”}

\lettrine{25.63} 桓南郡\myidx{桓玄}与道曜\myidx{道曜}讲\CJKunderwave{老子}\footnote{桓南郡:桓玄,袭爵南郡公,故称。\CJKunderwave{老子}:即老子\CJKunderwave{道德经},魏晋清谈玄理的三玄之一。},王侍中\myidx{王桢之}为主簿\footnote{王侍中:指王桢之,曾官侍中,故称。当时在桓玄府衙任主簿。主簿是当时中央或地方府衙的重要属官。},在坐。桓曰:“王主簿,可顾名思义\footnote{顾名思义:王桢之字公傒,小字思道。\CJKunderwave{老子}论道,王桢之字思道。思道者,思名字则知\CJKunderwave{老子}之道也,故曰“顾名思义”。}。”王未答,且大笑。桓曰:“王思道能作大家儿笑\footnote{大家儿:世家望族的贵游子弟。}。”{\fzxk\zihao{6}\textcolor{red}{道曜,未详。思道,王祯(桢)之小字也。\CJKunderwave{老子}明道,祯之字思道,故曰“顾名思义”。}}

{\cangkai\zihao{5}【评】东晋政权支柱中的王、谢、庾、桓四大家族,论其门阀身份,桓氏当兵出身,品流在下。但自桓温父子拥兵掌权之后,权势气焰陡升。因此,其他的高门士族常不自觉地联合反对桓氏集团。桓玄以此常想方设法摧挫王、谢等士人望族,以抬高自家的品望。他任太尉时拥兵自重,大会朝臣,曾当面问王桢之:“我何如君亡叔(按:指王献之)?”桢之曰:“亡叔一时之标,公是千载之英。”琅邪王家子弟,马屁没有少拍。在此故事中,他又以王桢之小字加以调谑。“王未答且大笑”,可能是感到了桓玄的话不太友好,故大笑不予置答以免祸。但玄还是不依不饶,更追一层,讥之为“大家儿笑”——贵游子弟只知轻薄无礼的傻笑,欠缺的是实际的才干。这是否有点痛打落水狗的味道?}

\lettrine{25.64} 祖广\myidx{祖广}行恒缩头\footnote{恒:常,总是。},诣桓南郡\myidx{桓玄}\footnote{桓南郡:桓玄。},始下车,桓曰:“天甚晴朗,祖参军如从屋漏中来\footnote{屋漏:漏雨的破屋。此喻遭雨淋。}。”{\fzxk\zihao{6}\textcolor{red}{\CJKunderwave{祖氏谱}曰:“广字渊度,范阳人。父台之,光禄大夫。广仕至护军长史。”}}

{\cangkai\zihao{5}【评】桓玄自恃才地,拿人生理缺陷开玩笑,其言行心理,已日渐逼近了傲诞放达的贵游子弟的意识。但高门士族,却不认账。桓温欲与太原王氏联婚,即被鄙视为兵家子而遭拒绝,更何况是桓玄。}

\lettrine{25.65} 桓玄\myidx{桓玄}素轻桓崖\myidx{桓崖}\footnote{桓崖:桓修字承祖,小字崖。冲子、温侄。与桓玄是堂兄弟。尚简文帝女武昌公主。在晋官至中护军。桓玄篡位,进抚军大将军,安成王。},崖在京下有好桃\footnote{京下:京师。好桃:优良品种的桃子。},玄连就求之,遂不得佳者。{\fzxk\zihao{6}\textcolor{red}{崖,桓修小字。\CJKunderwave{续晋阳秋}曰:“修少为玄所侮,于言端常嗤鄙之。”}} 玄与殷仲文\myidx{殷仲文}书\footnote{殷仲文:陈郡人。桓玄引为咨议参军。玄篡位,成为心腹佐命大臣。玄败后投义军,后被诛。},以为嗤笑\footnote{嗤笑:嘲笑。},曰:“德之休明\footnote{休明:美好盛明。},肃慎贡其楛矢\footnote{肃慎:我国古代东北少数民族。楛矢:用楛木作杆的箭。};如其不尔,篱壁间物\footnote{篱壁间物:泛指庭院间的普通之物。},亦不可得也。”{\fzxk\zihao{6}\textcolor{red}{\CJKunderwave{国语}曰:“仲尼在陈,有隼集陈侯之庭而死,楛矢贯之,石砮尺有咫。问于仲尼,对曰:‘隼之来远矣,此肃慎之矢也。昔武王克商,通道于九夷百蛮,使各以方贿贡,于是肃慎氏贡楛矢。古者分异姓之职,使不忘服也,故分陈以肃慎之贡。若求之故府,其可得。’使求,得之金椟如初。”}}

{\cangkai\zihao{5}【评】桓玄其人,自负才地,以雄豪自处,一旦得意,眼中何曾有物?他不仅势摧王、谢华丽家族,而且在桓氏家族内部,也瞧不起桓修诸人,这就叫内战内行。可见桓氏家族内部,也是矛盾重重。如桓冲就不同于温,他是“忠言嘉谋,每尽心力”,与谢安一起,忠心辅助朝廷。冲子桓修,与野心勃勃的桓玄有矛盾,并不奇怪。他不把京下好桃赠玄,并非惜桃,而是因玄从小就欺侮之、嗤鄙之。故后面\CJKunderwave{仇隙}门云:“桓玄将篡,桓修欲因玄在修母许袭之。”被修母所止。故事中桓玄“德之休明,肃慎贡其楛矢”之言,称引\CJKunderwave{国语·鲁语上}的故事,隐约透露出篡逆之心。为了实现政治目标,他以宗亲血缘关系为纽带,把家族作为篡逆的核心力量。因此,他篡位后进修抚军大将军、安成王。一顶高帽子,终于把桓玄送上了断头台,悲乎哀哉!}




%%% Local Variables:
%%% mode: latex
%%% TeX-engine: xetex
%%% TeX-master: "../Main"
%%% End:
