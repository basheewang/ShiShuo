%% -*- coding: utf-8 -*-
%% Time-stamp: <Chen Wang: 2025-12-09 20:05:17>

% ○ ◎ ‧ 「 」 『 』 々 ( ) “ ” ■ ^[一-龥]
% 【\([^】][^】][^】]+\)】 → {\\fzxk\\zihao{6}\\textcolor{red}{\1}}
% \(【评】.*\) → {\\cangkai\\zihao{5}\1}
% \(【题解】.*\) → {\\cangkai\\zihao{5}\1}
% 《\([^》]+\)》 → \\CJKunderwave{\1}
% ^\([0-9]+.[0-9]+\) → \\lettrine{\1}
% {\\fzxk\\zihao{6}\\textcolor{red}{[^o]*}}

\setlength{\parindent}{0pt}


\chapter{黜免第二十八}




{\cangkai\zihao{5}【题解】 黜免者,贬斥放废和免职罢官也。这原是历朝历代都有的寻常事。一般来说,因时代变化及具体要求不同来安排官吏的进退升降,属正常之事,不在本门叙述的范围之内。\CJKunderwave{黜免}门所见故事,大多是非正常的罢官放废故事。所谓非正常,有的是因对重大事故负有直接责任,如第3则殷浩因北征失败而废为庶人;有的则是意外之事触动了当权者某根敏感的神经,如第2则捕三峡猿子者,第4则食烝薤不助人而笑者,都被桓温“敕令免官”;有的是政治上猜忌报复,如第6则桓温罢免邓遐;第7则桓温欲杀武陵王晞父子而不仅是罢官放逐,因而激发了司马王室与桓温集团政治上的对抗。总之,非正常的内里,潜伏了深刻的社会矛盾和政治斗争。被罢官者,有的是罪有应得,如殷仲文;有的则属冤屈,或是有错误而罪不至此;但可怕的是政治斗争中的残酷打击和无情报复。这类历史教训,在本门故事中有形象的展现,各类人物的艺术描绘,上至帝王权贵,下至行伍小吏的心理刻画及其精神状态,无不栩栩如生而跃然纸上。}

\lettrine{28.1} 诸葛厷\myidx{诸葛厷}在西朝\footnote{诸葛厷:字茂远。琅邪人。官司空主簿。参前\CJKunderwave{文学}第13则注。西朝:西晋。},少有清誉,为王夷甫\myidx{王衍}所重\footnote{王夷甫:王衍,字夷甫。},时论亦以拟王\footnote{拟王:与王衍相比拟。}。后为继母族党所谗\footnote{族党:家族亲属。},诬之为狂逆。将远徙\footnote{徙:流放。},友人王夷甫之徒诣槛与别\footnote{槛:囚车。袁本“槛”下有“车”字。}。厷问:“朝廷何以徙我?”王曰:“言卿狂逆。”厷曰:“逆则应杀\footnote{逆:叛逆。},狂何所徙!”{\fzxk\zihao{6}\textcolor{red}{厷,已见。}}

{\cangkai\zihao{5}【评】司马夺人江山,弑其君主,故羞言“忠”而提倡“以孝治国”。以“孝”杀人,早被于嵇康,现又降临诸葛厷的头上。厷为继母族党所谗,看来也因与继母关系紧张,故被诬以“狂逆”——也即“不孝”大罪。厷言反驳,颇有道理。逆者叛乱,则应置以死刑重典,岂容远徙生还?而狂者进取,孔子早有明言,又何罪之有?罗织罪状,处置失当,朝廷之羞。王衍诣槛与别,则明知其冤屈而无可奈何,这就证明一旦戴上“不孝”大帽,则连朝廷重臣王衍也救不了。呜呼,礼教杀人,一至于此,悲哉!}

\lettrine{28.2} 桓公\myidx{桓温}入蜀\footnote{桓公:桓温。蜀:即今四川之地。},至三峡中\footnote{三峡:从四川奉节至湖北宜昌之间的长江水道,两岸是悬崖峭壁,所称三峡,古今说法多有不同,今指瞿塘峡、巫峡、西陵峡。},部伍中有得猿子者\footnote{部伍:部曲行伍,指军队。},{\fzxk\zihao{6}\textcolor{red}{\CJKunderwave{荆州记}四:“峡长七百里,两岸连山,略无绝处,重岩叠障,隐天蔽日。常有高猿长啸,属引清远,渔者歌曰:‘巴东三峡巫峡长,猿鸣一声泪沾裳。’”}} 其母缘岸哀号,行百馀里不去,遂跳上船,至便即绝。破视其腹中,肠皆寸寸断。公闻之,怒命黜其人\footnote{黜:斥退,罢免。}。

{\cangkai\zihao{5}【评】桓温西征入蜀,事在永和二年(346)。成语“肝肠寸断”或“柔肠寸断”,出典于此。军队手握武器,原来是用来杀人的。但所杀何人,为何杀人,为谁杀人?却是大有文章的,这里有正义与否的性质问题。除恶扬善,是战士的职责。因此,作为军人,不仅要擅长杀人,更要有爱心,保护国家与人民。一味嗜杀的残忍之人,不可能是优秀的军人。桓温虽是奸雄,但在其聚集力量的成长壮大期间,却是颇通人性,为峡猿肝肠寸断而黜免士卒。刘辰翁评曰:“此怒亦何可少!”爱猿者更会爱惜战士的生命。这就成了桓温团结三军的一种号召。桓温初期征战,攻无不克,战无不胜,与此或有关系。桓之老谋深算,目光远大,而非寻常官僚,在此透露了消息。}

\lettrine{28.3} 殷中军\myidx{殷浩}被废\footnote{殷中军:殷浩,曾任中军将军,故称。被废:指废为庶人。},在信安\footnote{信安:县名,晋属东阳郡,今浙江衢江区。},终日恒书空作字。扬州吏民寻义逐之\footnote{寻义:追念情义。逐:追随。},窃视\footnote{窃视:偷偷观察。},唯作“咄咄怪事”四字而已\footnote{咄咄怪事:令人惊叹之事。咄咄,感叹之声。}。{\fzxk\zihao{6}\textcolor{red}{\CJKunderwave{晋阳秋}曰:“初,浩以中军将军镇寿阳。羌姚襄上书归降,后有罪,浩阴图诛之。会关中有变,符徤(苻健)死。浩伪率军而行,云修复山陵,襄前驱,恐,遂反。军至山桑,闻襄将至,弃辎重驰保谯。襄至,据山桑,焚其舟实,至寿阳,略流民而还。浩士卒多叛。征西温乃上表黜浩,抚军大将军奏免浩,除名为民。浩驰还谢罪,既而迁于东阳信安县。”}}

{\cangkai\zihao{5}【评】永和九年殷浩北伐失败,十年(354年)被废为庶人。殷浩被废,一方面是他作为三军统帅,指挥失误,可以说是咎由自取。另一方面是责任事故中又埋伏了很深的政治阴谋。政敌桓温的上奏攻讦,背后有复杂的政治背景。桓温镇荆州,掌控长江中上游的雄师,势力日炽,威胁司马皇室。为了牵制桓温,朝廷起用殷浩为扬州刺史,建武将军,参综朝权,作为朝廷心膂,桓温颇相疑贰。浩败而温伐蜀大胜,一胜一败,形成强烈对比。故温之斥浩,表面合于情理而无违公心;但实际却是翦除朝廷支柱,以为日后的夺权专政作铺垫。桓温野心,路人皆知,但惮其权势,朝廷一忍再忍。殷浩之书空作字——即用手指在空中虚写“咄咄怪事”四字,不仅为个人出处感慨,更是为桓温得意、国家前途堪忧而发。}

\lettrine{28.4} 桓公\myidx{桓温}坐有参军椅烝薤\footnote{桓公:桓温。 参军:军府属官。椅:当作“掎”,用筷子挟取食物的动作。烝薤:即\CJKunderwave{齐民要术}的“薤白蒸”,用秫米、葱、薤(\xpinyin*{藠}头,一种蔬菜)、油、豆豉蒸调黏合的食品。},不时解\footnote{不时解:一时分不开。},共食者又不助,而椅终不放,举坐皆笑。桓公曰:“同盘尚不相助\footnote{同盘: 同盘共餐。},况复危难乎\footnote{况复:何况。}?”敕令免官。

{\cangkai\zihao{5}【评】这故事与第二则为姐妹篇,应连读咀嚼,方知其味。刘辰翁评曰:“二怒皆可观。” 这同样是处于上升时期桓温争取人心的做法。同事共食不相助而笑之,虽是小事一桩,但桓温作为政治家,却因小见大,小事不相助,况复危难大事乎?一个团体,一支军队,因小事闹不团结,临危难而岂能一心对敌?不团结就没有战斗力,就不能成就大事。为了将来的大业,桓温拿其开刀。因开玩笑而丢官,表面过于严厉,因为罪不至此。但桓温是个集雄心和野心于一身的枭雄,为了自己的未来事业,不惜牺牲局部。严酷之中是另有远虑。}

\lettrine{28.5} 殷中军\myidx{殷浩}废后\footnote{殷中军:殷浩。参前注。},恨简文\myidx{司马昱}曰\footnote{简文:简文帝司马昱。时穆帝年幼,司马昱以会稽王、抚军将军、录尚书事辅政。}:“上人箸百尺楼上\footnote{上人箸百尺楼上:把人推上了百尺高楼。},儋梯将去\footnote{儋:通“担”,担当,肩扛。将去:撤掉。}。”{\fzxk\zihao{6}\textcolor{red}{\CJKunderwave{续晋阳秋}曰:“告(浩)虽废黜,夷神委命,雅咏不辍,虽家人不见其有流放之戚。外生韩伯始随至徙所,周年还都,浩素爱之,送至水侧,乃咏曹颜远诗曰:‘富贵他人合,贫贱亲戚离。’因泣下。”其悲见于外者,唯此一事而已。则书空去梯之言,未必皆实也。}}

{\cangkai\zihao{5}【评】此则与第三则为姐妹篇,应一起体味。刘注以为书空、去梯之言,未必皆实。这是不了解殷浩。殷浩也是人,也有七情六欲,喜怒哀乐。浩隐居十载,因简文劝解,说是“足下去就即是时之废兴”,以家国复兴之责相委,于是浩始出山任事。为什么简文一定要浩出山?史称“简文以浩有盛名,朝野推伏,故引为心膂,以抗于(桓)温”。殷浩是作为一张重要的政治牌打出来的。但殷不久败废,温素忌浩,闻其败,于是落井下石,必欲置之死地而后快,这是残酷的政治党争使然。但当时朝中主政者为简文,废浩之举,出于温言;而定其罪罚者,却是简文。当然,这是简文迫于桓温,非其本怀。但如余嘉锡\CJKunderwave{笺疏}所言:“明明抚军(简文)之所奏请,不得谓非太宗之所废也。”用也简文,废也简文,人心险恶,政治家唯利是图,此殷浩之所怨也。}

\lettrine{28.6} 邓竟陵\myidx{邓遐}免官后赴山陵\footnote{邓竟陵:邓遐字应玄。曾官竟陵太守,故称。山陵,原指皇帝陵墓,此特指参加简文帝高平陵的葬礼,时咸安二年(372)。},过见大司马桓公\myidx{桓温}\footnote{大司马桓公:桓温。时大司马桓温专擅朝政。}。公问之曰:“卿何以更瘦?”{\fzxk\zihao{6}\textcolor{red}{\CJKunderwave{大司马寮属名}曰:“邓遐字应玄,陈郡人。平南将军岳之子。勇力绝人,气盖当世,时人方之樊哙。为桓温参军,数从温征伐,历竟陵太守。枋头之役,温既怀耻忿,且惮遐,因免遐官。病卒。”}} 邓曰:“有愧于叔达\myidx{孟敏},不能不恨于破甑\footnote{“有愧于叔达”二句:出典参刘注。以“破甑”喻丢官。意谓自己不能像孟敏失甑那样豁达,对于丢官不能不有遗憾。}!”{\fzxk\zihao{6}\textcolor{red}{\CJKunderwave{郭林宗别传}曰:“钜鹿孟敏,字叔达。敦朴质直。客居太原,杂处凡俗,未有所名。尝至市买甑,荷儋堕地坏之,径去不顾。适遇林宗,见而异之。因问曰:‘坏甑可惜,何以不顾?’客曰:‘甑既已破,视之何益?’林宗赏其介决,因以知其德性,谓必为美士,劝令读书。游学十年,遂知名。三府并辟,不就,东夏以为美贤。”}}

{\cangkai\zihao{5}【评】故事发生在简文帝驾崩的咸安二年(372)。桓温一代枭雄,前期颇通人性,身边团聚一批英才为己所用,故常战无不胜。后期则负其才力,久怀异志,专擅朝政而失却众心,故胜败无常。苻坚得知温废海西公而立简文后,谓群臣曰:“温前败灞上,后败枋头,十五年间,再倾国师。……不能思愆免退,以谢百姓,方废君以自悦,将如四海何!”(见\CJKunderwave{晋书·载记·苻坚传上})温前曾责殷浩北伐之败,贬为庶人。但自己连连大败,却不仅毫无自责之心,而是诿罪僚属,废主立威。其中,邓遐就成为其政治斗争的牺牲品。遐原为温之参军,勇力绝人,气盖当世,随温征战,功勋卓著,号为名将。但温于枋头大败后,既怀耻忿,且忌惮遐之勇果,因免遐官。罢官废主等一连串大动作,是其实现政治野心的重要步骤,明眼人一看便知。故邓遐“不能不恨于破甑”的答辞,语极真切,义形于色而话外有话,桓温听后,能无愧乎?}

\lettrine{28.7} 桓宣武\myidx{桓温}既废太宰父\myidx{司马晞}子\myidx{司马综}\footnote{桓宣武:桓温卒谥宣武,故称。太宰父子: 指司马晞及子综。晞字道叔,与简文帝司马昱同为元帝子。晞排行四,昱最小。晞任太宰,辅助朝政,故称。},仍上表曰\footnote{仍:仍然,继而。}:“应割近情\footnote{近情:近亲之情。},以存远计\footnote{远计:长远大计。}。若除太宰父子,可无后忧。”简文\myidx{司马昱}手答表曰\footnote{手答:亲手批复。}:“所不忍言,况过于言。”宣武又重表,辞转苦切。简文更答曰:“若晋室灵长\footnote{灵长:绵长久远。},明公便宜奉行此诏\footnote{明公:对有地位之人的尊称。此诏:\CJKunderwave{晋书·简文帝纪}作“前诏”,疑是。};如大运去矣\footnote{大运:皇室命运。},请避贤路\footnote{请避贤路:避位让贤,请求下台。}。”桓公读诏,手战流汗,于此乃止。太宰父子远徙新安\footnote{徙:流放。新安:县名,晋时属东阳郡。今浙江衢江区。}。{\fzxk\zihao{6}\textcolor{red}{\CJKunderwave{司马晞传}曰:“晞字道升,元帝弟(第)四子。初封武陵王,拜太宰。少不好学,尚武凶恣。时太宗辅政,晞以宗长不得执权,常怀愤慨,欲因桓温入朝,杀之。太宗即位,新蔡王晃首辞,引与晖(晞)及子综谋逆。有司奏晞等斩刑,诏原之,徙新安。晞未败四五年中,喜为挽歌,自摇大铃,使左右习和之。又燕会,倡妓作新安人歌舞离别之辞,其声甚悲,后果徙新安。”}}

{\cangkai\zihao{5}【评】凌濛初评桓温读诏,谓“不得不流汗”。实际上,当时以简文为代表的司马皇室,处在生死存亡的关键时刻,更是“不得不流汗”。其手答“若大运去矣,请避贤路”,实在是对司马皇室命运的最后一搏,哀告之声,不绝于耳。桓温在其废立树威之后,继而要杀害司马晞父子,这是为什么?史称晞“无学术而有武干,为桓温所忌”,一旦皇族之人把握了“枪杆子”,桓温的篡立野心还有戏唱吗?司马皇族与桓氏集团,争权夺利,已是势不两立。据\CJKunderwave{晋书·元四王传},晞少子忠敬王司马遵十二岁时,桓伊过门拜访,遵曰:“门何为通桓氏?” 左右曰:“伊与桓温疏宗,相见无嫌。”遵曰:“我闻人姓木边,便欲杀之,况诸桓乎!”于此可见,桓温好豫司马家事,实是为其篡立作必要的铺垫,司马皇室与之势不两立,有心报仇,无力除奸,简文虽贵为皇帝,仅是空名而已,也只能徒唤奈何!}

\lettrine{28.8} 桓玄\myidx{桓玄}败后\footnote{桓玄:温少子,袭父爵。安帝元兴二年(403)篡晋建楚,自立为帝。次年被刘裕击灭。},殷仲文\myidx{殷仲文}还为大司马\myidx{桓温}咨议\footnote{殷仲文:妻为桓玄姊。闻玄平京师,投之,成为伪楚佐命亲贵。玄败,又奉二后归朝廷。参前\CJKunderwave{言语}第106则注。大司马咨议:官名,即大司马咨议参军,公府或军府的重要属官。},意似二三\footnote{意似二三:三心二意,犹疑不定。},非复往日。大司马府听(厅)前有一老槐\footnote{听:通“厅”。},甚扶疏\footnote{扶疏:枝叶纷披茂盛。}。殷因月朔与众在听(厅)\footnote{月朔:每月初一。府衙依例聚会议事。},视槐良久,叹曰:“槐树婆娑\footnote{婆娑:原为舞姿,引申为披散弛纵而缺乏精神的样子。},无复生意!”{\fzxk\zihao{6}\textcolor{red}{\CJKunderwave{晋安帝纪}曰:“桓玄败,殷仲文归京师,高祖以其卫从二后,且以大信宜令,引为镇军长史。自以名辈先达,位遇至重,而后来谢混之徒,皆畴昔之所附也,今比肩同列,常怏然自失。后果徙信安。”}}

{\cangkai\zihao{5}【评】殷仲文是晋末著名诗人,义熙中为“华绮之冠”(\CJKunderwave{诗品}下),但同时又是无耻文人之典型。朝秦暮楚,反复小人,而唯利是图,最后落得身败名裂,正是历史的惩罚。桓玄年少失势,仲文与之疏远;玄兴兵入京师,建伪楚,他即投怀送抱,拍马溜须,无不尽其极,因而成为新主的“佐命亲贵”,贪吝纳贿,厚自封崇。玄败,又投机归刘裕。裕出于自己的长远政治大计,暂留用之。但小人嗜利,永无止境。咨议之职,岂是亲贵?“槐树婆娑,无复生意!”所叹物是人非,怏怏失志,如此心态,岂能有好下场?}

\lettrine{28.9} 殷仲文\myidx{殷仲文}既素有名望,自谓必当阿衡朝政\footnote{阿衡朝政:主持朝政。},忽作东阳太守\footnote{东阳:郡名,治所在长山(今浙江金华)。},意甚不平。{\fzxk\zihao{6}\textcolor{red}{\CJKunderwave{晋安帝纪}曰:“仲丈(文)后为东阳,愈愤怨,乃桓胤谋反,遂伏诛。仲文尝照镜不见头,俄而难及。”}} 及之郡,至富阳\footnote{富阳:今属浙江,原名富春。},慨然叹曰:“看此山川形势,当复出一孙伯符\footnote{孙伯符:三国时孙策,字伯符。时天下大乱,策东征西讨,占据江东,为后来吴国建立基础。}。”{\fzxk\zihao{6}\textcolor{red}{孙策,富春人。故及此而叹。}}

{\cangkai\zihao{5}【评】此与第8则为姐妹篇,当并读合观。仲文之叹,“当复出一孙伯符”,正说明他反叛朝廷的意志已决。但仲文之人,岂可与孙策相比拟?加以形势不同,其野心很快败露而伏诛,终成历史之笑柄。}




%%% Local Variables:
%%% mode: latex
%%% TeX-engine: xetex
%%% TeX-master: "../Main"
%%% End:
