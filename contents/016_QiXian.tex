%% -*- coding: utf-8 -*-
%% Time-stamp: <Chen Wang: 2025-12-06 19:04:06>

% ○ ◎ ‧ 「 」 『 』 々 ( ) “ ” ■ ^[一-龥]
% 【\([^】][^】][^】]+\)】 → {\\fzxk\\zihao{6}\\textcolor{red}{\1}}
% \(【评】.*\) → {\\cangkai\\zihao{5}\1}
% \(【题解】.*\) → {\\cangkai\\zihao{5}\1}
% 《\([^》]+\)》 → \\CJKunderwave{\1}
% ^\([0-9]+.[0-9]+\) → \\lettrine{\1}
% {\\fzxk\\zihao{6}\\textcolor{red}{[^o]*}}

\setlength{\parindent}{0pt}


\chapter{企羡第十六}


{\cangkai\zihao{5}【题解】 企羡,企望和羡慕的意思。\CJKunderwave{汉书·高帝纪}早有“日夜企而望归”之言。企,许慎\CJKunderwave{说文}解释是“举踵也”,也就是踮起脚跟以便开阔视野,看得更远。后来“企羡”合为一词,就引申为举踵仰慕的意思。本门六则故事,大多是魏晋士人站在主观思考立场上表现了对于理想人物道德才华的企羡仰慕和尊敬之情。所企羡的对象,成为自己人生的榜样,甚至是心目中追求的偶像。这与今天年轻追星族的明星偶像崇拜,多少有些相似,都是一种主观热情的自然宣泄;所不同的是,今天的明星崇拜对象,属娱乐圈;而魏晋人的偶像,则大多是道德文章方面的名士,重在德望和才华。这就在一定程度上形象地反映出当时的社会风尚、审美情趣和思想境界之所在。}

\lettrine{16.1} 王丞相\myidx{王导}拜司空\footnote{王丞相:指王导。司空:官名,朝廷三公之一,参议国事。},桓廷尉\myidx{桓彝}作两髻\footnote{桓廷尉:指桓彝,字茂伦,死后赠廷尉,故称。桓彝为当时名士,以识鉴、品评人物知名于世。髻:发髻。},葛裙策杖\footnote{葛裙:葛布质料的下裳。策杖:挟着手杖。},路边窥之,叹曰:“人言阿龙超\footnote{阿龙:王导小名赤龙。小名前加“阿”,是魏晋人的习惯,以示亲切、随意。超:高超,卓越。},阿龙故自超\footnote{故自:原来如此,引申为确实。}。”{\fzxk\zihao{6}\textcolor{red}{阿龙,丞相小字。}} 不觉至台门\footnote{台门:中央台城之门。朝廷禁省称为台,禁城为台城,台城之门为台门。}。

{\cangkai\zihao{5}【评】故事发生于太兴四年(321)七月,王导任司空之时。桓彝是两晋之交的名士,史称其“性通朗,早获盛名。有人伦识鉴,拔才取士,或出于无闻,或得之孩抱,时人方之许(劭)、郭(泰)”。他自己是脱俗高人。当王导拜司空时,他的头上梳了两个发髻,身穿葛布衣裙,其着装即落尽豪华而又不同凡俗。可见他也是一个高自标榜而又很有个性的人物。但通过他那发自内心的叹美之言:“阿龙故自超”,以及在路边观看,因神往而不知不觉跟随仪仗队走到台门的行动,其言其行,皆出于随心所欲的下意识,这正是一种由衷的欣赏与赞美。王导是当时的大名士,其神采风度,已成为人们竞相谈论和模仿的偶像,也是桓氏心仪的典型。当时士林学界产生对王导的崇拜心理,对王导的辅政及其号召力量,是很有好处的;同时也对草创时期的东晋王朝,产生了一股无形的向心力。}

\lettrine{16.2} 王丞相\myidx{王导}过江\footnote{过江:指永嘉之乱后,渡过长江,避乱江南。},自说昔在洛水边\footnote{洛水:水名,流经西晋京师洛阳。},数与裴成公\myidx{裴顗}、阮千里\myidx{阮瞻}诸贤共谈道\footnote{裴成公:指裴顗,死后谥号成,故称。阮千里:阮瞻,字千里,阮咸之子。裴、阮二人皆为西晋名士,裴著\CJKunderwave{崇有论},阮有“将毋同”之说,皆轰动一时。谈道:谈玄论道。}。羊曼\myidx{羊曼}曰\footnote{羊曼:字延祖。与温峤、庾亮等为东晋中兴名臣。参\CJKunderwave{雅量}20注。}:“人久以此许卿\footnote{许:称许,称赞。},何须复尔\footnote{何须复尔:何必重复呢?}?”王曰:“亦不言我须此\footnote{亦:并。须此:需要名声。},但欲尔时不可得耳\footnote{但:只是。尔时:那时。}。”{\fzxk\zihao{6}\textcolor{red}{“欲”一作“叹”。}}

{\cangkai\zihao{5}【评】此事发生于永嘉之乱后的南渡避难之时。当时社稷丘墟,国家倾覆,局势混乱,东晋朝廷正待组建之际,能否在江南顺利开国以图中兴,尚待努力。因此,王导经常回忆西京和平时代与裴、阮诸名士谈玄论道的开心时日,这一美好的回忆,并非自我标榜的炒作宣传,而是追念昔日之游不可再得的今昔兴亡之叹。“但欲尔时不可得耳”,正是琅邪王家的贵族名士,在经历了人世沧桑之后,走向成熟而发出的深沉人生慨叹。}

\lettrine{16.3} 王右军\myidx{王羲之}得人以\CJKunderwave{兰亭集序}方\CJKunderwave{金谷诗序}\footnote{王右军:即王羲之。\CJKunderwave{兰亭集序}:晋穆帝永和九年(353)三月三日(祓禊日),王羲之与谢安等四十一人宴集山阴之兰亭,在赏玩山水时赋诗咏怀,后汇集成册,由羲之作序冠其首。序记山水之美及聚会的欢乐之情,抒发了时不我待,人生无常的感慨。其文笔清新疏朗而直抒胸臆,颇富艺术感染力。\CJKunderwave{兰亭序}书法艺术,“飘若浮云,矫若惊龙”,成为中国古代法帖艺术之冠。方:比拟,媲美。\CJKunderwave{金谷诗序}:作者石崇。记晋惠帝元康六年(296),石崇、苏绍等三十人于河南金谷涧石崇别业为征西将军祭酒王诩送行而燕集赋诗事,诗汇成册,石崇作序。序文参见\CJKunderwave{品藻}57则刘孝标注称引。},又以己敌石崇\myidx{石崇}\footnote{敌:匹敌。石崇:字季伦,善诗文。},甚有欣色。{\fzxk\zihao{6}\textcolor{red}{王羲之\CJKunderwave{临河叙}曰:“永和九年,岁在癸丑,莫春之初,会于会稽山阴之兰亭,修禊事也。群贤毕至,少长咸集。此地有崇山峻岭,茂林修竹;又有清流激湍,映带左右,引以为流觞曲水,列坐其次。是日也,天朗气清,惠风和畅,娱目骋怀,信可乐也。虽无丝竹管弦之盛,一觞一咏,亦足以畅叙幽情矣。故列序时人,录其所述。右将军司马太原孙承(宋本原缺‘承’字,诸本增‘丞’字,王利器校以‘丞’为‘承’之讹,今据以校增)。公等二十六人,赋诗如左,前馀姚令、会稽谢胜等十五人,不能赋诗,罚酒各三斗。”}}

{\cangkai\zihao{5}【评】唐代诗人杜牧有“大抵南朝皆旷达,可怜东晋最风流”的诗句。东晋一代名士的风流旷达,并非只是注重外在仪表之美,他们更看中的是内在人格之独立,精神之自由,感情之率真自然,情趣之高雅幽远。在东晋名士的文章风流中,琅邪王羲之首屈一指,其\CJKunderwave{兰亭集序}不仅在书法艺术方面夺冠,即在文学史上,其艺术成就同样赫赫有名,影响颇大。\CJKunderwave{兰亭集序},又名\CJKunderwave{临河序},因为羲之叙修祓诗时,原无标题,其题目是后人所加,因此所题不一,名异实同。王羲之眼界极高,一般的文人学士,他是轻易看不上眼的。但是,只要表现出真正的风流才性,又很快成为他叹美的对象。比如孙绰曾推荐支遁和他认识,他是“殊自轻之”,不屑一顾。但当他听支遁谈\CJKunderwave{庄子·逍遥游}后,见其才藻新奇,花烂映发,于是“披襟解带,流连不能已”。事见\CJKunderwave{文学}第36则。王羲之的文章风流,实是在学习和借鉴中形成的,而非凭空而降。关于\CJKunderwave{兰亭集序}与石崇\CJKunderwave{金谷诗序}的关系,因在文学史上,只见王羲之,不知有石崇,二人声名悬殊,于是刘辰翁批评云:“敌石崇,亦何等语?”认为比拟不伦不类。这一说法,不合事实。只要具体比照二序的文字,自然明白,\CJKunderwave{兰亭集序}确曾多方借鉴\CJKunderwave{金谷诗序},颇受艺术启迪。故杨慎明白指出:“\CJKunderwave{金谷序}实\CJKunderwave{兰亭序}之所祖也。”今人的批评,因政治或道德方面的原因,讥贬石崇,连带其文章也一并抹煞;但古人并不因人废言,王羲之欣赏石崇才华及其\CJKunderwave{金谷诗序},以之作为自己学习借鉴的榜样,然后加以创造和开拓,才有今天所见\CJKunderwave{兰亭集序}的艺术光彩。}

\lettrine{16.4} 王司州\myidx{王胡之}先为庾公\myidx{庾亮}记室参军\footnote{王司州:指王胡之,曾任司州刺史,故称。庾公:指庾亮,时任征西将军,开府武昌。记室参军:官名,诸王、三公或将军节镇的僚属,掌表章文书等。},后取殷浩\myidx{殷浩}为长史\footnote{殷浩:字渊源。长史:诸王、三公及将军督府属下的重要辅助官。};始到,庾公欲遣王使下都\footnote{下都:顺游出差京师建康。},王自启求住\footnote{启:启禀,一般用于下对上。住:停留。},曰:“下官希见盛德\footnote{下官:下级对上级的谦词。希:同“稀”,少。盛德:很有德望之人,此特指殷浩。},渊源始至,犹贪与少日周旋\footnote{少日:几天。周旋:接近,亲近。}。”

{\cangkai\zihao{5}【评】王胡之出身琅邪世家,王廙之子。喜谈谐,善属文,为世所重,并非一般之士。但他心仪殷浩,为了与殷浩“少日周旋”,甚至把上司的差遣任务推辞掉,一个生动的细节描写就突出了殷浩在人们心目中的偶像地位。王誉殷为“盛德”之人,史称“浩识度清远,弱冠有美名,尤善玄言,与叔父融俱好\CJKunderwave{老}、\CJKunderwave{易}……由是为风流谈论者所宗。或问浩曰:‘将莅官而梦棺,将得财而梦粪,何也?’浩曰:‘官本腐臭,故将得官而梦尸。钱本粪土,故将得钱而梦秽。’时人以为名言。”(\CJKunderwave{晋书}本传)其入仕前,并非浪得虚名,其称“盛德”,成为士人崇拜对象,并非偶然。}

\lettrine{16.5} 郗嘉宾\myidx{郗超}得人以己比苻坚\myidx{苻坚}\footnote{郗嘉宾:郗超,字嘉宾,一字景兴。任桓温大司马,深得信任,立简文为帝后,迁中书侍郎,实代桓温监督朝廷而权重当时得人:得知有人。比:比拟。苻坚:十六国中前秦皇帝,曾统一北方,有吞并东晋以一统天下之志。在淝水之战后败亡。},大喜。

{\cangkai\zihao{5}【评】东晋中期,郗超是个光芒炫目的政治明星,属于桓温之党,连政敌谢安也畏惮三分。其为人“善谈论义理精微”,精明卓识,运筹帷幄,莫测高深。桓温心图不轨,“欲立王霸之基,超为之谋”。其勃勃野心,于此可见一斑。人或以之比拟苻坚。王世懋讥为“无谓”,以为苻坚是前秦皇帝,东晋劲敌,比之敌酋,有何光彩?这是从政治上看问题,郗超并非从此着眼。如果抛开一时政局,从历史角度看问题,则苻坚曾是一个具有雄才大略而功业显赫的优秀人物,他曾统一了中国北方,并且志在统一整个中国。在郗超眼里,苻坚是英雄,是偶像,是崇拜的对象。以之相比,正可见自己的个性追求和理想心胸,其“大喜”过望,当在料中。}

\lettrine{16.6} 孟昶\myidx{孟昶}未达时\footnote{孟昶:字彦达,东晋平昌(今山东安丘南)人,曾官丹阳尹,后卢循攻石头,他饮鸩而死。达:发达,得志。},家在京口\footnote{京口:地名,即今江苏镇江市。}。{\fzxk\zihao{6}\textcolor{red}{\CJKunderwave{晋安帝纪}曰:“昶字彦达,平昌人。父馥,中护军。昶矜严有志局,少为王恭所知。豫义旗之勋,迁丹阳尹。卢循下,昶虑事不济,仰药而死。”}} 尝见王恭\myidx{王恭}乘高舆\footnote{王恭:字孝伯。时任青、兖二州刺史,节镇京口。京口是东晋京师建康门户,形势重要。舆:肩舆,一种人抬之轿。},被鹤氅裘\footnote{被:通“披”。鹤氅裘:鹤的羽毛制作的外套。},于时微雪\footnote{于时:当时。},昶于篱间窥之\footnote{篱间:竹篱笆的隙缝之间。据徐震堮\CJKunderwave{校笺}引环济\CJKunderwave{吴记},谓吴天纪二年(278),“自宫门至朱雀桥,夹路作府舍。又开大道……夹道皆筑高墙瓦覆,或作竹藩”。则六朝时重要道路旁边有修建竹篱笆的习俗。窥:视。},叹曰:“此真神仙中人!”

{\cangkai\zihao{5}【评】孟昶路边的一声惊叹,代表了当时士人对俊男帅哥神采风度的一种审美评价。但后人常脱离历史,转换视角,简单地以后世的政治观念来代替当时的审美评价,因而结论大相径庭。清李慈铭痛斥王恭“凭藉戚畹,早据高资,学术全无,骄淫自恣”,以为其人装腔作态,虽“枭首灭族,未抵厥罪”。这实是以成败论英雄,不足为凭。李氏同时又讥诋“孟昶寒人,奴颜乞相,惊其绚丽,望若天人,鄙识琐谈,何足称述?而当时叹为名士,后世载其风流,六代陵迟,职由于此”(见余嘉锡\CJKunderwave{笺疏}称引)。似乎孟昶的一声惊呼,就有了亡国的罪责。这也是言过其实。实则王恭美姿仪,当时誉播人口,其乘高舆、披鹤裘,是魏晋贵族生活之一斑,正见其兴味风神之所在,当时自然而然,并非矫饰;至于孟昶之叹美,出自内心真情流露,是潜意识的言行,更无炒作之嫌。故事真实反映了魏晋士人的生活风尚。今之视昔,应有理解的态度。一旦有了历史的眼光,才能真正地与古人交流与对话,从而形成批判与借鉴。如若不然,如今天的国际服装节或模特大奖赛,不都要成了亡国的罪人了吗?其理何在?}







%%% Local Variables:
%%% mode: latex
%%% TeX-engine: xetex
%%% TeX-master: "../Main"
%%% End:
