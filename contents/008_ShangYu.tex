%% -*- coding: utf-8 -*-
%% Time-stamp: <Chen Wang: 2025-12-10 21:24:01>

% ○ ◎ ‧ 「 」 『 』 々 ( ) “ ” ■ ^[一-龥]
% 【\([^】][^】][^】]+\)】 → {\\fzxk\\zihao{6}\\textcolor{red}{\1}}
% \(【评】.*\) → {\\cangkai\\zihao{5}\1}
% \(【题解】.*\) → {\\cangkai\\zihao{5}\1}
% 《\([^》]+\)》 → \\CJKunderwave{\1}
% ^\([0-9]+.[0-9]+\) → \\lettrine{\1}
% {\\fzxk\\zihao{6}\\textcolor{red}{[^o]*}}


\setlength{\parindent}{0pt}

\chapter{赏誉第八}


{\cangkai\zihao{5}【题解】赏誉,欣赏赞誉之谓,其对象是人,也即品评人物、加以揄扬。本门共156则,比较集中地表现了魏晋人物品题的审美标准和眼光趣味。玄学,作为一种哲学体系和思想潮流,它决定着魏晋士大夫观察和解释宇宙万物的原则、思辨方式和审美观念。体现在人物鉴赏上,就是重玄远超脱的境界,注重人物的精神气度和内在品质。}

{\cangkai\zihao{5}魏晋士人赏誉成风,品题形式多样。具体而言,有今人对逝者的赏誉,如第79则载,桓温行经王敦墓边过,望之云:“可儿!可儿!”传达出对一代枭雄的由衷赞叹,同时也是“夺他人之酒杯,浇自己之块垒”的托古抒怀;书中更多的是同代人之间的互相赏誉,如第99则载朝野以殷浩拟管、葛,以其出处,卜江左兴亡。这是一则典型的对当世名士的疯狂崇拜行为,在今天还具有一定的启示意义;受六朝家族制度影响,家族成员间的“戏台里喝彩”的方式亦所在多是,如王大将军称其儿云:“其神候似欲可”,望子成龙的心情溢于言表。}

{\cangkai\zihao{5}魏晋人重外在形貌、气质、风度、神情,表现出有别于汉儒绝重道德伦理的时代特点。所以,像王衍这样善于包装的大名士,也就能在士林中享有至高的声誉。王戎称“太尉神姿高彻,如瑶林琼树,自然是风尘外物”之语,可见其极有市场。但魏晋士人毕竟受老、庄思想濡染较深,能够由形入神,体会到人物更精微、更细腻的心灵世界,所以那些形残神全的大名士为人欣赏也就不足为奇了。第30则载庾子躬有废疾,甚知名,家在城西,号曰“城西公府”。奇人奇闻为士林画廊添上绝妙的一笔。}

{\cangkai\zihao{5}本门人物赏誉多用传统“意象思维”方法,言近旨远、言约意丰,给人无穷回味。如第2则记述,世目李元礼“谡谡如劲松下风”,第4则载公孙度目邴原“所谓云中白鹤,非燕雀之网所能罗也”,譬喻意象优美,表征了\CJKunderwave{世说}的语言魅力。\CJKunderwave{世说}记人记事多用白描手法,三言两语而勾画传神,极富生活气息。如第39则通过蔡谟之口,回忆陆机兄弟当年生活场景:“陆机兄弟住参佐廨中,三间瓦屋,士龙住东头,士衡住西头,士龙为人文弱可爱,士衡长七尺馀,声作钟声,言多慷慨。”读来二陆音容笑貌,宛在目前。}

\lettrine{8.1} 陈仲举\myidx{陈蕃}常叹曰\footnote{陈仲举:陈蕃字仲举,(?—168),汝南平舆(今河南)人。汉灵帝时,官至太傅。谋除宦官,被杀。性方峻,不交非类。不畏强御而直言极谏,终为宦官所害。}:“若周子居\myidx{周乘}者\footnote{周子居:周乘,汉末汝南安城(今河南正阳东北)人。\CJKunderwave{世说·赏誉}第1 则刘注引\CJKunderwave{汝南先贤传}云:“天姿聪朗,高峙岳立,非陈仲举、黄叔度之俦不交也。……为太(泰)山太守,甚有惠政。},真治国之器。{\fzxk\zihao{6}\textcolor{red}{\CJKunderwave{汝南先贤传}曰:“周乘字子居,汝南安城人。天资聪明,高峙岳立,非陈仲举、黄叔度之俦则不交也。仲举常叹曰:‘周子居者,真治国之器也。’为太山太守,甚有惠政。”}} 譬诸宝剑,则世之干将\footnote{干将:古宝剑名。相传为春秋时吴人干将与其妻莫邪所铸,有二剑,阳曰“干将”,阴曰“莫邪”。}。”{\fzxk\zihao{6}\textcolor{red}{\CJKunderwave{吴越春秋}曰:“吴王阖闾请干将作剑。干将者,吴人。其妻曰莫邪。干将采五山之精,六金之英,候天地,司阴阳,百神临视,而金铁之精未流。夫妻乃剪发及爪而投之炉中,金铁乃濡,遂成二剑。阳曰‘干将’,而作龟文;阴曰‘莫邪’,而作漫理。干将匿其阳,出其阴以献阖闾。阖闾甚宝重之。”}}

{\cangkai\zihao{5}【评】周子居之事迹,史乏详载,刘孝标注引\CJKunderwave{汝南先贤传}云:“为太山太守,甚有惠政。”极略。然可以通过观其交友、察其时誉,而勾勒其人的大致轮廓。诗云“嘤其鸣矣,寻其友声”(\CJKunderwave{诗经·小雅·伐木}),君子同声相应,同气相求。周子居与汝南陈蕃、黄宪为乡党,与“汪汪如万顷之陂”、气度深厚的黄宪交好,被“言为士则,行为世范”的陈蕃赞为治国之器,誉为干将宝剑。干将为吴钩之首,陈蕃实以无双国士、第一流人物目之。陈蕃诸人不交非类,却视周子居视为士人典范,可见子居其人品位非凡。故事还约略折射出汉末人物品藻重治世干才的时代特点。}

\lettrine{8.2} 世目李元礼\myidx{李膺}\footnote{目:品题;品评。李元礼:李膺,东汉颍川(今河南)人。汉末名臣,为世人所宗仰。在朝清议领袖之一,与杜密并称“李杜”。因反对宦官专政,被太学生称为“天下模楷”。后遭党锢之祸,死于狱中。}:“谡谡如劲松下风\footnote{谡谡:象声词。同“肃肃”,形容风之清冽强劲。}。”{\fzxk\zihao{6}\textcolor{red}{\CJKunderwave{李氏家传}曰:“膺岳峙渊清,峻貌贵重,华夏称曰:‘颍川李府君,頵頵如玉山。汝南陈仲举,轩轩如千里马。南阳朱公叔,飂飂如行松柏之下。’”}}

{\cangkai\zihao{5}【评】每一时代都有应运而生“为天地立心,为生民立命”的精神导师,李膺、郭泰等人就是把握汉末清议脉搏、领导时势潮流的领袖人物。李膺在桓帝时任司隶校尉,结交郭泰,反对宦官专政,后遭党锢之祸。太学生称“天下楷模李元礼”,一经其品题者,则称“登龙门”,乃士林间的“无冕之王”。膺与窦武、陈蕃谋诛宦官,败后不避其难,主动就狱,刑戮加颈而不变色,为追求正义、真理而死,死得其所。“谡谡如劲松下风”一语,继承了儒家文化“比德”于物的传统,以意象批评的方式,形象地传达出李膺刚劲方峻、高自砥砺的品格、气节。}

\lettrine{8.3} 谢子微myidx{\}见许子将\myidx{许劭}兄\myidx{许靖}弟曰\footnote{谢子微:谢甄,字子微,东汉末汝南召陵(今河南郾城东)人。与陈留边并善谈论,有盛名。当时名士郭泰称他“英才有馀”。许子将:许劭(150—195),字子将,东汉末汝南平舆(今属河南)人。能品评鉴识人才,曾经当面品评曹操为“治世之能臣,乱世之奸雄”。他与从兄许靖俱负高名,一同评论乡党人物,月更其品题,汝南效之成俗,称为“月旦评”。}:“平舆之渊\footnote{平舆:县名。东汉时为汝南郡治,今属河南,此指许虔、许劭家乡。},有二龙焉。”见许子政\myidx{许虔}弱冠之时\footnote{许子政:许虔,字子政。许劭之兄。为人雅正,知名当时。弱冠:指男子二十岁左右。},叹曰:“若许子政者,有干国之器\footnote{干国之器:治国的才能。干,辅佐。器,才干。}。正色忠謇\footnote{正色:脸色庄重。忠謇:忠直。},则陈仲举\myidx{陈蕃}之匹\footnote{陈仲举:陈蕃。匹:匹敌,比配。};{\fzxk\zihao{6}\textcolor{red}{\CJKunderwave{汝南先贤传}曰:“谢甄字子微,汝南邵陵人。明识人伦,虽郭林宗不及甄之鉴也。见许子将兄弟弱冠时,则曰:‘平舆之渊有二龙。’仕为豫章从事。许虔字子政,平舆人。体尚高洁,雅正宽亮。谢子微见虔兄弟,叹曰:‘若许子政者,干国之器也。’虔弟劭,声未发时,时人以谓不如虔,虔恒抚髀称劭,自以为不及也。释褐,为郡功曹,黜奸废恶,一郡肃然。年三十五卒。”\CJKunderwave{海内先贤传}曰:“许劭字子将,虔弟也。山峙渊停,行应规表。邵陵谢子微,高才远识,见劭十岁时,叹曰:‘此乃希世之伟人也。’初,劭拔樊子昭于市肆,出虞承贤于客舍,召李叔才于无闻,擢郭子瑜于小吏。广陵徐孟本来临汝南,闻劭高名,召功曹。时表(袁)绍以公族为濮阳长,弃官还。副车从骑将入郡界,乃叹曰:‘许子将秉持清格,岂可以吾舆服见之邪?’遂单马而归。辟公府掾,敦辟皆不就。避地江南,卒于豫章也。”}} 伐恶退不(肖)\footnote{伐恶:打击恶人。退不:据袁本,“不”下增一“肖”字。退不肖,即贬斥不良小人。或谓“不”当读“鄙”,退不,即斥退鄙陋小人,可备一说。},有范孟博\myidx{范滂}之风\footnote{范孟博:范滂(137—169),字孟博,举孝廉,为清诏使,力图澄清吏治,每至州境,不法官吏望风而逃。}。”{\fzxk\zihao{6}\textcolor{red}{张璠\CJKunderwave{汉纪}曰:“范滂字孟博,汝南伊(细)阳人。为功曹,辟公府掾。升车揽辔,有澄清天下之志。百城闻滂高名,皆解印绶去。为党事见诛。”}}

{\cangkai\zihao{5}【评】\CJKunderwave{续谈助}引\CJKunderwave{许劭列传}曰:“劭幼时,谢子微便云:‘此贤当持汝南管钥。’”可与此相互印证。劭与从兄靖俱负高名,一同评论乡党人物,月更其品题,汝南效之成俗,称为“月旦评”。劭与郭泰齐名,天下言拔士者,咸称许、郭。曹操微末之时,劭评其“治世之能臣,乱世之奸雄”,操竟大悦而去。可见其金口一开,士人便奉若神明。后诸葛恪、葛洪等人,出于僵化正统立场,对汝南月旦多所批评,以为“汉末俗弊,朋党分部。许子将之徒,以口舌取戒,争讼论议,门宗成仇”(葛洪\CJKunderwave{抱朴子})。对此该如何看待呢?许、郭诸人,通过社会舆论改善现实风气进而影响政治,对于腐败政治有激浊扬清之效,客观上又有奖掖人才之功;虽不免有臧否任意以快其恩怨之弊,但在中国历史发展进程中,代表着一种要求社会舆论自由、改善腐败政治的正义呼声,其在历史上的积极影响远大于其负面效应,可谓功不可没。}

\lettrine{8.4} 公孙度\myidx{公孙度}目邴原\myidx{邴原}\footnote{公孙度:子升济,一作叔济,东汉襄平(今辽宁辽阳北)人。自立为辽东侯、平州牧。曹操表之为威武将军,封永宁乡侯。目:品评。邴原(?—211):字根矩,东汉朱虚(今山东临朐东)人。少与管宁以操尚齐名。汉末黄巾起义,他避地辽东,士人百姓从者甚众。后回中原,归曹操。}:“所谓云中白鹤,非燕雀之网所能罗也\footnote{“云中白鹤”两句:邴原在辽东,公孙度很厚待他。后来他自坐捕鱼大船离辽东。过了几天公孙度才发觉,手下吏员建议追赶,公孙度说了这两句话。罗:张网捕捉。}。”{\fzxk\zihao{6}\textcolor{red}{\CJKunderwave{魏书}曰:“度字叔济,襄平人。累迁冀州刺史、辽东太守。”\CJKunderwave{邴原别传}曰:“原字根矩,东管(北海)朱虚人。少孤,数岁时,过书舍而泣。师问曰:‘童子何泣也?’原曰:‘凡得学者,有亲也。一则愿其不孤,二则羡其得学,中心感伤,故泣耳。’师恻然曰:‘苟欲学,不须资也。’于是就业。长则博览洽闻,金玉其行。知世将乱,避世辽东,公孙度厚礼之。中国既宁,欲还乡里,为度禁绝。原密自治严,谓部落曰:‘移北近郡。’以观其意。皆曰:‘乐移。’原旧有捕鱼大船,请村落皆令熟醉,因夜去之。数日,度乃觉。吏欲追之,度曰:‘邴君所谓云中白鹤,非鹑鷃之网所能罗也。’魏王辟祭酒,累迁五官中郎长史。”}}

{\cangkai\zihao{5}【评】邴原少与管宁俱以操尚称。因黄巾方盛,避地辽东,为公孙度优待。后欲还乡,公孙度以为人才宜得不宜失,故百般设防阻禁。邴原归思难收,设计将部署灌醉,坐捕鱼大船离开辽东。公孙度目邴原“所谓云中白鹤,非燕雀之网所能罗也”,是在察觉邴原离后有感而发。“云中白鹤”,意象优美,引人高远联想。鹤鸣九皋,杳然不群,非燕雀之网所能网罗。言外之意,邴原志存高远,辽东边邑非其终老之乡。公孙度气量不凡,成人之美,有知人、自知之明。邴原后归曹操,一展身手。}

\lettrine{8.5} 锺士季\myidx{锺会}目王安丰\myidx{王戎}\footnote{锺士季:锺会,锺毓、锺会:魏锺繇二子,颍川长社人。毓,字稚叔,官至廷尉、青州刺史,督徐州、荆州军事,死后追赠车骑将军,谥惠侯。会,字士季,官至司徒。受命伐蜀,蜀破,欲率军谋反,内部先乱,为乱军所杀。魏以谋反论其罪。令誉:美好的声誉注。目:品评。王安丰:王戎。}:“阿戎了了解人意\footnote{阿戎:王戎。了了:聪明懂事。}。”{\fzxk\zihao{6}\textcolor{red}{王隐\CJKunderwave{晋书}曰:“戎少清明晓悟。”}} 谓裴公\myidx{裴楷}之谈\footnote{裴公:裴楷,裴令公:即裴楷,曾官中书令,故云,又称“裴令”。善\CJKunderwave{老}、\CJKunderwave{易},当时著名清谈名家。二国租钱:指从梁、赵二国税收所获钱财。},经日不竭。{\fzxk\zihao{6}\textcolor{red}{裴楷,已见。}} 吏部郎阙\footnote{阙:通“缺”。},文帝\myidx{司马昭}问其人于锺会\footnote{文帝:司马昭。},会曰:“裴楷清通\footnote{清通:清明通达。},王戎简要\footnote{简要:简洁切要。},皆其选也\footnote{选:指人选。}。”于是用裴。{\fzxk\zihao{6}\textcolor{red}{案:诸书皆云锺会荐裴楷、王戎于晋文王,文王辟以为掾,不闻为吏部郎。}}

{\cangkai\zihao{5}【评】锺会为太傅锺繇之子,少敏惠夙成。蒋济谓“观其眸子,足以知人”,可见生就了一双慧眼。会博学名理,倾心名士,尝论“易”无互体,又博综当时才性辩论之要,著\CJKunderwave{四本论}。弱冠与王弼并知名,为一时谈宗。其论裴楷清通、王戎简要,可谓要言不烦,有知人识鉴。汉末章句之学渐趋烦琐、僵化,成为桎梏思想的枷锁;魏晋玄学崇尚清通简要,正是对汉代学风的反拨,故裴、王之清通、简要,与生俱来地带着魏晋时代的胎记。又锺会品评之词简至四言,片言居要,深得玄言清旨。锺会虽倡才性合同,然其为司马氏构陷名士,又据蜀造反,苍黄反覆,恰与其所持理论相背离,实为士林小人。}

\lettrine{8.6} 王濬冲\myidx{王戎}、裴叔则\myidx{裴楷}二人总角诣锺士季\myidx{锺会}\footnote{王濬冲:王戎。裴叔则:裴楷,裴令公:即裴楷,曾官中书令,故云,又称“裴令”。善\CJKunderwave{老}、\CJKunderwave{易},当时著名清谈名家。二国租钱:指从梁、赵二国税收所获钱财。总角:童年。诣:拜访。锺士季:锺会。}。须臾去,后客问锺曰:“向二童何如\footnote{向:刚才,先前。}?”锺曰:“裴楷清通,王戎简要。后二十年,此二贤当为吏部尚书,冀尔时天下无滞才\footnote{滞才:淹留遗落的人才。}。”{\fzxk\zihao{6}\textcolor{red}{\CJKunderwave{晋阳秋}曰:“戎为儿童,锺会异之。”}}

{\cangkai\zihao{5}【评】故事从正面记述了锺会的一次人物赏誉的场面,客观上展示了人性的多面性和复杂性。锺会以裴、王二十年后为吏部尚书相期许,且向往彼时天下无遗贤的美好图景,可以看成是对太平盛事的憧憬。锺会一方面是甘为人梯、愿为伯乐的睿智长者,另一方面又是背叛人格尊严而向最高统治者靠拢、自甘堕落的士林败类。锺会对裴、王的美好期待并未变成现实,据王隐\CJKunderwave{晋书}记载王戎领吏部尚书时的情况曰:“自戎居选,未尝进一寒素,退一虚名,理一冤枉,杀一疽嫉。随其沉浮,门调户选。”根本就是一个政治滑头,实在大失人望。魏晋清谈尚清通简要,与汉儒考据之琐屑、拘泥相对立。然物极必反,一些人格操守本不强固的知识分子,举起清通简要的旗帜而放纵自己,越过了良心底线,堕落为玄学末流。王戎之苟媚取容,即为此类。清通、简要为时代利器,于汉代学术、思想有摧枯拉朽之功,但因所用之人有异,或促进时代思想的发展,或走向无所作为的虚无主义,或变成彻头彻尾的大奸大滑。不可划一以视,读者自当明辨。}

\lettrine{8.7} 谚曰:“后来领袖有裴秀\myidx{裴秀}\footnote{后来:晚辈。领袖:衣领和衣袖,为衣服的提契部位。因借喻能提契他人或为人表率的人物。裴秀(224—271):字季彦,西晋河东闻喜(今属山西)人。三国魏尚书令裴潜子。少好学能文,有才名,仕魏累迁至尚书仆射。司马炎即位(西晋武帝),他拜尚书令,官至司空。著\CJKunderwave{禹贡地域图}十八篇,为中国地图绘制学奠基之作。}。”{\fzxk\zihao{6}\textcolor{red}{虞预\CJKunderwave{晋书}曰:“秀字季彦,河东闻喜人。父潜,魏太常。秀有风操,八岁能箸文。叔父徽有声名。秀年十馀岁,有宾客诣徽,出则过秀。时人为之语曰:‘后进领袖有裴秀。’大将军辟为掾。父终,推财与兄。年二十五,迁黄门侍郎。晋受禅,封钜鹿公,后累迁左光禄、司空。四十八薨,谥元公,配食宗庙。”}}

{\cangkai\zihao{5}【评】裴秀天才颖发,八岁能属文。叔父徽有盛名,秀十馀岁时,宾客诣徽,出则过秀。其为世人所重如此。说起来似有些天方夜谭,一个乳臭未干的小孩子,有何德何能让士人们刮目相看!盖魏晋时代思想、学术相对自由多元化,神童容易涌现;又晋人等级、礼教观念相对淡薄,“无贵无贱,无长无少,道之所存,师之所存”(韩愈\CJKunderwave{师说})的意识深入人心,才能发生这样的“咄咄怪事”。裴秀后著\CJKunderwave{禹贡地域图}十八篇,为中国地图学的奠基之作,果然不负众望。“后进领袖有裴秀”,乃当时谚语,正说明裴秀声名广远,获得当时士林的承认。赞语运用了叶韵手法。汉末以后,有时通过“风谣”谚语的形式,来表达对某人的舆论鉴定,这一方式的特殊之处在于,它以押韵的语言来揭示人物特点,因生动形象而能够在社会上广泛流传。如本句中“袖”、“秀”之叶韵。}

\lettrine{8.8} 裴令公\myidx{裴楷}目夏侯太初\myidx{夏侯玄}\footnote{裴令公:裴楷,裴令公:即裴楷,曾官中书令,故云,又称“裴令”。善\CJKunderwave{老}、\CJKunderwave{易},当时著名清谈名家。二国租钱:指从梁、赵二国税收所获钱财。夏侯太初:夏侯玄,(209—254):字太初,三国魏人。曹爽辅政时,他以爽姑之子受重用。曹爽被诛,玄废黜。后与李丰等谋杀司马师,事败,同被诛。他是早期的玄学领袖人物。}:“肃肃如入廊庙中\footnote{肃肃:恭敬貌。廊庙:朝堂。},不修敬而人自敬\footnote{修敬:讲求恭敬。}。”{\fzxk\zihao{6}\textcolor{red}{\CJKunderwave{礼记}曰:“周丰谓鲁哀公曰:‘宗庙社稷之中,末(未)施敬而民自敬。’”}} 一曰:“如入宗庙,琅琅但见礼乐器\footnote{琅琅:形容玉石的光彩。礼乐器:礼器和乐器。宗庙中祭祀行礼奏乐所用。}。见锺士季\myidx{锺会}\footnote{锺士季:锺会,锺毓、锺会:魏锺繇二子,颍川长社人。毓,字稚叔,官至廷尉、青州刺史,督徐州、荆州军事,死后追赠车骑将军,谥惠侯。会,字士季,官至司徒。受命伐蜀,蜀破,欲率军谋反,内部先乱,为乱军所杀。魏以谋反论其罪。令誉:美好的声誉注。},如观武库,但睹矛戟。见傅兰硕\myidx{傅嘏}\footnote{傅兰硕:傅嘏。},汪廧靡所不有\footnote{汪廧:“廧”,为“翔”之讹文。水势浩大的样子。靡:无。}。见山巨源\myidx{山涛}\footnote{山巨源:山涛,车骑:此指谢玄,谢安侄,死后追赠车骑将军。},如登山临下,幽然深远。”{\fzxk\zihao{6}\textcolor{red}{玄、会、嘏、涛,并已见上。}}

{\cangkai\zihao{5}【评】此前二则裴楷、王戎为锺会所赏誉,此则锺会又成为裴楷品目的对象。于此可见魏晋名士间乐于相互品藻的风气,既以人伦识鉴自任,又以被人赏誉为荣。故事记载裴楷集中品目玄、会、嘏、涛四人,人才荟萃,气象万千,如浏览人物画廊。故事运用意象批评的方式,又称比兴之体,以生动妥帖的修辞比喻画活了人物形象,如夏侯玄之庄严肃穆,有圣人气象;锺会锋芒太露,令人心寒;傅嘏汪洋博大,黄金与泥土混杂;山涛幽然深远,风度涵容。故事用语虽含蓄混沌,而细加品味,尚可以决其人之高下,这就见出品评者精准传神的语言艺术功力。\CJKunderwave{晋书}载楷有知人识鉴,非虚言也。}

\lettrine{8.9} 羊公\myidx{羊祜}还洛\footnote{羊公:羊祜。},郭弈\myidx{郭弈}为野王令\footnote{郭弈:字太业,一作泰业,西晋太原阳曲(今山西太原)人。“弈”,\CJKunderwave{晋书}本传作“奕”。少有重名,山涛称其高简有雅量。初为野王令。官至尚书。野王:县名。晋属河内郡,在今河南沁阳。},{\fzxk\zihao{6}\textcolor{red}{\CJKunderwave{晋诸公赞}曰:“弈字泰业,太原阳曲人。累世旧族。弈有才望,历雍州刺史、尚书。”}} 羊至界\footnote{界:指野王县境。},遣人要之\footnote{要:遮留、拦截。},郭便自往。既见,叹曰:“羊叔子何必减郭太业\footnote{何必:为什么一定。减:逊色,不如。}!”复往羊许,小悉还\footnote{小悉:少顷;不多久。},又叹曰:“羊叔子去人远矣\footnote{去人远:此谓羊祜人品远胜一般人。去,距离。}!”羊既去,郭送之弥日\footnote{弥日:整日。},一举数百里,遂以出境免官\footnote{出境:越出境界。免官:古制,地方官不得无端越出自己所辖境界。}。复叹曰:“羊叔子何必减颜子\myidx{颜回}\footnote{颜子:颜回,孔子弟子。}!”

{\cangkai\zihao{5}【评】故事“以追光摄影之笔,写通天尽人之怀”;一唱三叹、淋漓尽致地展示出郭弈的交友过程。羊、郭之交,可分为三个阶段:郭弈一见羊祜便已倾心,视若同俦,有相见恨晚之叹;二见已仰之弥高,以为超出常人;三见之后,便视为孔门颜回,竟因送别出境而免官。郭弈交友超出了一般功利层面,是出于精神气象、风度气质的契合,表现了高雅名士一往情深的心灵世界。郭弈对羊祜有高山仰止般的赏誉,甚而为此丢官,非出于术士相面式的神秘预测,亦超越了功名、利益互求,完全是自然而然的应感之会,表征了晋人纯净澄明的精神世界,具有极大的感染力。送友“一举数百里”,完全是情动于中而行于外,是对常礼的超越。其纵情越礼之举是否合适,为此丢官是否值得,尚可讨论。但由此而呈现出来的情对礼(理)的突破,在“但伤知音稀”的浇离俗世中,无疑是有深刻启示意义的,以至陈梦槐大发“如此流连叹赏,令我常怀古人”之叹。}

\lettrine{8.10} 王戎\myidx{王戎}目山巨源\myidx{山涛}:“如璞玉浑金\footnote{璞玉浑金:未经雕琢的玉,未曾冶炼的金。比喻人的质性纯美。},人皆钦其宝\footnote{钦:看重。宝:珍贵。},莫知名其器\footnote{名:估量。器:才识度量。}。”{\fzxk\zihao{6}\textcolor{red}{顾恺之\CJKunderwave{画赞}曰:“涛无所标名,淳深渊默,人莫见其际,而嚣然亦入道。故见者莫能称谓,而服其伟量。”}}

{\cangkai\zihao{5}【评】王戎、山涛同为魏晋名士,共预竹林之游,相知当必甚深。“如璞玉浑金,人皆钦其宝,而莫知名其器”,意谓人皆知山涛如璞玉浑金,质性纯美,而其价值到底几何,则人多茫然,有待识宝的慧眼。言外之意,能够真正衡量山涛价值的人并不多。这就与本门第八则裴楷之评“见山巨源,如登山临下,幽然深远”相互照应。竹林七贤之中,山涛既不同于嵇康的桀骜难驯,故缺少了“宁为玉碎,不为瓦全”的悲壮;也不同于阮籍的酣饮自藏,少了几分哲人痛苦的沉思;当然更不同于王戎的苟媚取容、自降身价的堕落。竹林七贤中,每个人都是独特而自足的个体存在,山涛身上似聚合了名士洒落不羁的风度,和常人惟求自保的平庸,故能允执其中、左右俱宜。他既为天下名士叹赏,亦被司马氏政权恩遇。山涛的人生轨迹,反映了知识分子在政治高压时代的“第三条道路”(与嵇康之不合作、锺会之甘当走狗相较)。}

\lettrine{8.11} 羊长和\myidx{羊忱}父繇\myidx{羊繇}与太傅祜\myidx{羊祜}同堂相善\footnote{羊长和:羊忱,见\CJKunderwave{方正}19注。繇:羊繇,忱之父。太傅祜:太傅羊祜。同堂:同一祖父。},仕至车骑掾\footnote{车骑掾:车骑将军的属官。},蚤卒\footnote{蚤:通“早”。}。长和兄弟五人,幼孤\footnote{孤:年幼丧父。}。{\fzxk\zihao{6}\textcolor{red}{\CJKunderwave{羊氏谱}曰:“繇字堪甫,太山人。祖续,汉太尉,不拜。父祕,京兆太守。繇历车骑掾,娶乐国祯女,生五子:秉、给、式、亮、忱也。”}} 祜来哭,见长和哀容举止宛苦(若)成人\footnote{宛苦(若)成人:“苦”当是“若”之形讹。},乃叹曰:“从兄不亡矣\footnote{从兄:堂兄。}!”

{\cangkai\zihao{5}【评】羊繇五子,羊忱最优,繇死,羊忱哀容举止宛若成人。这一方面是长期良好家庭教育的体现,一方面又是“发乎情止于礼仪”,会通情、礼的结果。临时抱佛脚是装不出“无声”的“真悲”的!羊祜因此感动,叹曰:“从兄不亡矣!”意谓羊忱虽最小而懔懔有父风,能振兴门庭。羊祜之言背后,传达出了中国人的家族传承意识,是儒家“兴灭国,继绝世”观念的一脉延伸。后羊忱果历官扬州刺史、侍中等高官,远远超过父亲羊繇的功业。在八王之乱中,忱能审时度势,拒绝赵王伦的封职,有识鉴之明。}

\lettrine{8.12} 山公\myidx{山涛}举阮咸\myidx{阮咸}为吏部郎\footnote{山公:山涛,车骑:此指谢玄,谢安侄,死后追赠车骑将军。阮咸(234—305):字仲容,西晋陈留尉氏(今属河南)人。阮籍兄子。为“竹林七贤”之一,与阮籍并称“大小阮”。吏部郎:主管官吏选拔的官。},目曰:“清真寡欲\footnote{清真:犹纯真。寡欲:少私欲,淡于外物。},万物不能移也\footnote{移:改变。}。”{\fzxk\zihao{6}\textcolor{red}{\CJKunderwave{名士传}曰:“咸字仲容,陈留人,籍兄子也。任达不拘,当世皆怪其所为。及与之处,少嗜欲,哀乐至到,过绝于人,然后皆忘其向议。为散骑(侍)郎,山涛举为吏部,武帝不用。太原郭弈见之心醉,不觉叹服。解音,好酒以卒。”山涛\CJKunderwave{启事}曰:“吏部郎史曜出,处缺当选。涛荐咸曰:‘真素寡欲,深识清浊,万物不能移也。若在官人之职,必妙绝于时。’诏用陆亮。\CJKunderwave{晋阳秋}曰:咸行己多违礼度,涛举以为吏部郎,世祖不许。”\CJKunderwave{竹林七贤论}曰:“山涛之举阮咸,固知上不能用,盖惜旷世之隽,莫识其意故耳。夫以咸之所犯,方外之意;称其清真寡欲,则迹外之意自见耳。”}}

{\cangkai\zihao{5}【评】阮咸预竹林之游,任性放达、纵情越礼堪居七贤之首。其追鲜卑婢、与群猪共饮故事,以怪诞的方式传达了对儒家礼法名教教条化、桎梏化的激烈批判。由于大大超出常人的接受程度,致世俗舆论讥,而为礼法名教所禁锢。故山涛主吏部,举咸为吏部郎,三举而晋武帝不用,颇能说明问题。只有山涛等少数同是出身竹林的名士们,能把握阮咸的内心世界,目之以“清真寡欲,万物不能移也”。清真寡欲,是老、庄道家哲学的妙谛,魏晋士人以此相尚,是对自然宇宙和自然性情的回归。咸之怪异举止,虽能为少数开风气的先行者理解,但毕竟“曲高和寡”,难得世俗认可。这种标新立异、无所不用其极的“耍酷”行为,难以得到封建伦理道德最高代表——皇帝的认可,也在料中。}

\lettrine{8.13} 王戎\myidx{王戎}目阮文业\myidx{阮武}:“清伦有鉴识\footnote{清伦:高雅豁达,人品清高。鉴识:洞察事物的能力。},汉元以来未有此人\footnote{汉元:汉代建元。犹言汉初。}。”{\fzxk\zihao{6}\textcolor{red}{杜笃\CJKunderwave{新书}曰:“阮武字文业,陈留尉氏人。父谌,侍中。武阔达博通,渊雅之士。”\CJKunderwave{陈留志}曰:“武,魏末河清(清河)太守。族子籍,年总角,未知名。武见而伟之,以为胜己。知人多此类。箸书十八篇,谓之\CJKunderwave{阮子}。终于家。”郭泰友人宋子俊称泰:“自汉元以来,未有林宗之匹。”}}

{\cangkai\zihao{5}【评】余嘉锡先生以为,此评乃汉代宋子俊称郭林宗之言,而王戎取以称阮武,故陷入自相矛盾之境地。郭林宗为人伦领袖,高名盖世。信如王戎所言,则阮武为一时之选,林宗无足道;诚如宋子俊所称,则郭林宗为汉以来第一人,阮武无所处。二者必有一谬。其实,此恐名士之间互相标榜之言,大体须有,而不必太过拘泥求真。若一定把名士的品评之辞,当作不可改易的金科玉律,像数学公式一样,步步推算现实生活中的人和事,就未免有株守之嫌了。但以王戎之人伦识鉴,亦必非捕风捉影之谈,还是大体切实的。魏、晋二代,诸阮家风师心任性,名士辈出,气象万千,云蒸霞蔚。山公、王戎屡有嘉赞之词。王戎目阮武“汉元以来未有此人”,殆出于一时会心的激赏,主观情绪化色彩不能排除。}

\lettrine{8.14} 武元夏\myidx{武陔}目裴\myidx{裴楷}、王\myidx{王戎}曰\footnote{武元夏:武陔,字元夏,西晋初沛国竹邑(今安徽宿县北)人。年少知名,有知人之鉴。裴、王:裴楷,王戎。俱见前。}:“戎尚约,楷清通。”{\fzxk\zihao{6}\textcolor{red}{虞预\CJKunderwave{晋书}曰:“武陔字元夏,沛国竹邑人。父周,魏光禄大夫。陔及二弟歆(韶)、茂皆总角见称,并有器望。乡人诸父,未能觉其多少。时同郡刘公荣名知人,尝造周,周见其三子。公荣曰:‘君三子皆国士,元夏器量最优,有辅佐之风,力仕宦,可为亚公。叔夏、季夏不减常伯、纳言也。’陔至左仆射。”}}

{\cangkai\zihao{5}【评】刘公荣于武陔有国士之目,观陔赏誉王戎、裴楷之言,与锺会英雄所见略同。公荣之评非虚言也。魏晋玄学清谈,崇尚清通简约,裴、王之精神气度与时代特点有契合之处,故屡为时人激赏,以致凌濛初有“清通简要,何以叠见”之微辞。细加揣摩,清通、简要之间亦有区别。“清”乃魏晋审美之核心范畴,清真、清通、清远、清平、清明等等,不一而足。由中国传统文化中清、浊二气之别,到汉末清流之清议、魏晋清谈,以及外化为六朝诗文中之山水清音,其间当有某种必然联系。“通”则通脱、通达,故无往而不通。能清通则该繁则繁,该简则简,随其时宜,变化无方,故清通之中实含简约。众人评价裴、王之清通、简约,看似并列处之,实则有高下之别。}

\lettrine{8.15} 庾子嵩\myidx{庾敳}目和峤\myidx{和峤}\footnote{庾子嵩:庾敳。和峤:和峤(?—292):魏晋时汝南西平(今属河南)人。官至中书令。为政清简得民,有风格,善礼法,朝野许其能正风俗人伦。家财富而性至吝,人称有“钱癖”。大丧:指父母之丧。据\CJKunderwave{晋书}戎传,时戎遭母丧,而峤遭父丧。}:“森森如千丈松\footnote{森森:树木高耸貌。},虽磊砢有节目\footnote{磊砢:树木多节貌。节目:树木枝干将接、纹理纠结不顺之处。},施之大厦,有栋梁之用\footnote{栋梁:房屋的正梁。用以比喻能为国家担当重任的人才。}。”{\fzxk\zihao{6}\textcolor{red}{\CJKunderwave{晋诸公赞}曰:“峤常慕其舅夏侯玄为人,故于朝士中峨然不群,时类传(惮)其风节。”}}

{\cangkai\zihao{5}【评】诸贤以为故事中和峤当为温峤,可备一说。庾敳品目和峤之言,运用意象批评,以中国文化传统中常见的松树为喻,言其有栋梁之材。千丈松,伟岸高大,与魏晋人物惯用之“玉树”、“春月柳”等阴柔意象迥异,见出阳刚雄壮之美的追求。合而观之,可见魏晋人物品评,阳刚、阴柔等美学风貌共存,异彩纷呈,美不胜收。和峤一生刚直不阿,在立储问题上坚持己见,对晋武帝寸步不让;又疾恶如仇,不与佞人荀勖同车,表现了洁身自好和毫不迁就的叫真精神。和峤这样的千丈松多多益善。}

\lettrine{8.16} 王戎\myidx{王戎}云\footnote{王戎:嵇康(223—262):三国时谯郡铚 (今安徽亳县)人。“竹林七贤”之一。曾任中散大夫,故称嵇中散。当时著名思想家、文学家、清谈名家。因其主张越名教而任自然,抨击礼法之士,不与司马氏统治集团合作,盛年被杀。}:“太尉\myidx{王衍}神姿高彻\footnote{太尉:王衍,王夷甫:王衍(256—311)字夷甫,见刘孝标注。“以清虚通理称”,为当时清谈名家,“妙悟若神”,“妙善玄言,唯谈\CJKunderwave{老}、\CJKunderwave{庄}为事”。为政多谋略,不以经国为念,而善思自全之计,然终为石勒所害。(见\CJKunderwave{晋书}本传)注。神姿:风度姿态。高彻:高迈爽朗。},如瑶林琼树\footnote{瑶林琼树:传说神仙世界的美好洁净的玉树。},自然是风尘外物\footnote{自然:天然。风尘外物:尘世以外的人。物,人。}。”{\fzxk\zihao{6}\textcolor{red}{\CJKunderwave{名士传}曰:“夷甫天形奇特,明秀若神。”\CJKunderwave{八王故事}曰:“石勒见夷甫,谓长史孔苌曰:‘吾行天下多矣,未宦(尝)见如此人,当可活不?’苌曰:‘彼晋三公,不为我用。’勒曰:‘虽然,要不可加以锋刃也。’夜使推墙杀之。”}}

{\cangkai\zihao{5}【评】王戎品目王衍,俨然视之为不食五谷、吸风饮露的神仙。“瑶林琼树”之义,当从容貌非凡、神情高远两个层面看待。魏晋名士注重姿容,王衍之美丰姿闲雅,玉柄麈尾与手同色,山涛至有“何物老妪,生宁馨儿”之叹。其容貌为人间罕有,故有“瑶林琼树”之评,此第一义;又王衍尚清谈,祖述老庄,谈空说有,其谈锋玄旨,远离人间烟火,超越现实层面,直抵逍遥胜境。此“瑶林琼树”、“风尘外物”之第二义。又晋武帝曾问王戎,王衍当世谁比?王戎答曰:“未见其比,当从古人中求之。”戎为衍从兄,合上观之,难脱“戏台里喝彩”之嫌。王衍一张漂亮的脸蛋和能言善辩的利口,足以倾倒时人,而其人格道德和社会建树,最终不齿于人。魏晋风度虽兼容并蓄,但也难免鱼目混珠、泥金俱下,令人扑朔迷离,其审美标准耐人寻味!}

\lettrine{8.17} 王汝南\myidx{王湛}既除所生服\footnote{王汝南:王湛(249—295),字处冲,西晋太原晋阳(今山西太原)人。王昶子,兄弟宗族皆以为痴。曾官汝南内史,故称。除所生服:守父母丧期满,除去孝服。所生:生养自己的父母。这里似指其父。},遂停墓所。兄子济\myidx{王济}每来拜墓\footnote{兄子济:王济,亦当时豪爽之士,\CJKunderwave{晋书}卷五十六本传,言其才藻卓绝,爽迈不群,多所陵傲,缺乡曲之誉。年四十馀始仕。与王济相知甚深。注。王湛兄王浑,浑子王济。},略不过叔\footnote{略不:几乎完全。 过:探望,问候。},叔亦不候。济脱时过\footnote{脱时:偶或,偶尔。},止寒温而已。后聊试问近事,答对甚有音辞\footnote{音辞:言辞。},出济意外,济极惋愕\footnote{惋愕:惊讶。},仍与语,转造精微\footnote{造:至。精微:精深微妙。}。济先略无子侄之敬,既闻其言,不觉懔然\footnote{懔然:严敬的样子。},心形俱肃。遂留共语,弥日累夜\footnote{弥日累夜:连日连夜。}。济虽隽爽\footnote{隽爽:俊迈豪爽。},自视缺然\footnote{缺然:感到不足的样子。},乃喟然叹曰:“家有名士,三十年而不知!”济去,叔送至门。济从骑有一马,绝难乘,少能骑者。济聊问叔:“好骑乘不\footnote{好:喜欢。不:同“否”。}?”曰:“亦好尔。”济又使骑难乘马。叔姿形既妙,回策如萦\footnote{回策如萦:谓挥旋马鞭,萦绕自如。策,马鞭。萦,缠绕。},名骑无以过之。济益叹其难测,非复一事。{\fzxk\zihao{6}\textcolor{red}{邓粲\CJKunderwave{晋纪}曰:“王湛字处冲,太原人。隐德,人莫之知,虽兄弟宗族亦以为痴,唯父昶异焉。昶丧,居墓次。兄子济往省湛,见床头有\CJKunderwave{周易},谓湛曰:‘叔父用此何为?颇曾看不?’湛笑曰:‘体中佳时,脱复看耳。今日当与汝言。’因共谈\CJKunderwave{易},剖析入微,妙言奇趣,济所未闻,叹不能测。济性好马,而所乘马骏驶,意甚爱之。湛曰:‘此虽小驶,然力薄不堪苦。近见督邮马,当胜此,但养不至耳。’济取督邮马,谷食十数日,与湛试之。长(湛)未尝乘马,卒然便驰骋,步骤不异于济,而马不相胜。湛曰:‘今直行车路,何以别马胜不,唯当就蚁封耳。’于是就蚁封盘马,果倒踣。其隽识天才乃尔。”}} 既还,浑\myidx{王浑}门(问)济\footnote{浑门济:诸本“门”作“问”,是。}:“何以暂行累日?”济曰:“始得一叔。”浑问其故,济具叹述如此。浑曰:“何如我?”济曰:“济以上人。”武帝\myidx{司马炎}每见济,辄以湛调之\footnote{调:调侃;嘲弄。},曰:“卿家痴叔死未?”济常无以答。既而得叔后,武帝又问如前,济曰:“臣叔不痴。”称其实美。帝曰:“谁比?”济曰:“山涛\myidx{山涛}以下\footnote{山涛:车骑:此指谢玄,谢安侄,死后追赠车骑将军。},魏舒\myidx{魏舒}以上\footnote{魏舒(209—290):字阳元,西晋任城樊(今山东济宁附近)人。善于射箭,众人莫敌。}。”{\fzxk\zihao{6}\textcolor{red}{\CJKunderwave{晋阳秋}曰:“济有人伦鉴识,其雅俗是非,少所优调。见湛,叹服其德宇。时人谓湛:‘上方山涛不足,下比魏舒有馀。’湛闻之曰:‘欲以我处季孟之间乎?’”王隐\CJKunderwave{晋书}曰:“魏舒字阳元,任城人。幼孤,为外氏宁家所养。宁氏起宅,相者曰:‘当出贵甥。’外祖母意以盛氏甥小而惠,谓应相也。舒曰:‘当为外氏成此宅相。’少名潺纯(迟钝),叔父衡使守水碓,每言:‘舒堪八百户长,我愿毕矣。’舒不以介意。身长八尺二寸,不修常人近事。少工射,箸韦衣,入山泽,每猎大获。为后将军锺毓长史。毓与参佐射戏,舒常为坐画筹。后值朋人少,以舒充数。于是发无不中,加博(举)措闲雅,殆尽其妙。毓叹谢之曰:‘吾之不足尽卿,如此射矣!’转相国参军。晋王每朝罢,目送之曰:‘魏舒堂堂,人之领袖。’累迁侍中、司徒。”}} 于是显名,年二十八始宦。

{\cangkai\zihao{5}【评】两晋之间,有几个著名的“痴”人。如故事所记之王湛及其王述、晋简文帝等。这几个人或少言寡语、拱默无为,或举止违常,以此获“痴”讥,实如璞玉浑金、大智若愚,不为人知赏而已。王湛非惟乖违于世,阖门之内亦无知者,可见“知音其难”,令人叹惋。湛有隐德,冲素自守、土木形骸,不交当世。若非有炉火纯青的定力和超凡脱俗的人生心态,不能达此大巧若拙的境界,庄子\CJKunderwave{逍遥游}中无名的圣人,无功的神人和无己的至人,假如真有其人,不过此耳。凌濛初曰:“岂有如此名士,三十年不知者,不信,不信。”盖湛得\CJKunderwave{周易} “潜龙勿用”之理,故能知止自足,遁世无闷。湛侄王济知耻能改,为叔逢人说项,亦深得名士旨趣。}

\lettrine{8.18} 裴仆射\myidx{裴頠}\footnote{裴仆射:裴頠,(267—300),见刘孝标注。博学多才识,“时人谓頠为言谈之林薮”。撰\CJKunderwave{崇有论}以推尊儒术,崇扬礼法,贬斥何晏、王衍等言“无”之蔽。},时人谓为“言谈之林薮\footnote{言谈之林薮:比喻善于言谈。林薮,山林水泽聚集之处,比喻事物聚集的地方。}”。{\fzxk\zihao{6}\textcolor{red}{\CJKunderwave{惠帝起居注}曰:“頠理甚渊博,赡于论难。”}}

{\cangkai\zihao{5}【评】裴頠家学有自然科学传统,父秀为地理学家,著\CJKunderwave{禹贡地域图}十八篇;頠则通博多闻,兼擅医术。魏晋玄学主调乃希慕自然,以无为本。一般人的本能是迷信权威和从众随俗。但在玄学的时代大潮冲击下,裴頠却违俗而以儒术为宗,深患何、王之口谈浮虚,著\CJKunderwave{崇有论},以纠虚无之偏。发表之后,士人群起围攻,王衍亲自上阵与其直接展开辩论,竟未能屈之。时人谓頠为“言谈之林薮”,裴頠在众口一词的时代坚持独立思考,不人云亦云,抛开其理论的是非曲直不谈,裴頠至少是一位特立独行的名士,精神可敬可佩。}

\lettrine{8.19} 张华\myidx{张华}见褚陶\myidx{褚陶}\footnote{张华:范阳方城(今河北固安西北)人。博学多才,贯通今古,以诗赋文章称名于世。为晋武帝筹设灭吴方略,一统天下。惠帝时官至司空,死于八王之乱。褚陶:字季雅,西晋吴郡钱塘(今浙江杭州)人。聪慧早成,善属文。不乐仕进。},语陆平原\myidx{陆机}曰\footnote{陆平原:陆机,参\CJKunderwave{晋书}本传,其为吴郡吴县华亭(今上海松江)人,当时著名的文学家。吴亡入晋后,累迁太子洗马、著作郎。曾任平原内史,故称“陆平原”。事成都王颖,颖兴兵攻掌权于洛阳的长沙王司马乂时,任陆机为后将军、河北大都督。机兵败遭谗,与弟陆云同为颖所杀。陆机仕晋为平原内史。}:“君兄弟龙跃云津\footnote{云津:犹言云间、云中。“龙跃云津”比喻英才崛起,暗切二陆家乡华亭(今上海松江),古称云间。},顾彦先myidx{\}凤鸣朝阳\footnote{顾彦先:顾荣(?—312):两晋之际江南士族领袖之一,与陆机、陆云同时入洛,时称“三俊”。南渡后,代表江南士族拥护和支持司马睿在江南开国,是为东晋。洛阳:西晋京师。凤鸣朝阳:比喻贤才遇时而起。语出\CJKunderwave{诗·大雅·卷阿}:“凤皇鸣矣,于彼高冈。梧桐生矣,于彼朝阳。”},谓东南之宝已尽;不意复见褚生。”陆曰:“公未睹不鸣不跃者耳!”{\fzxk\zihao{6}\textcolor{red}{\CJKunderwave{褚氏家传}曰:“陶字季雅,吴郡钱塘人,褚先生后也。陶聪惠绝伦,年十三,作\CJKunderwave{鸥鸟}、\CJKunderwave{水碓}二赋,宛陵严仲弼见而奇之,曰:‘褚先生复出矣。’弱不好弄,清淡(谈)闲默,以坟典自娱。语所亲曰:‘圣贤备在黄卷中,舍此何求?’州郡辟,不就。吴归命,世(祖)补台郎、建忠校尉。司空张华与陶书曰:‘二陆龙跃于江、汉,彦先凤鸣于朝阳,自此以来,常恐南金已尽,而复得之于吾子。故知延州之德不孤,渊岱之宝不匮。’仕至中尉。”}}

{\cangkai\zihao{5}【评】平吴后,全国统一,中原士族的傲慢与偏见发展到极端,歧视南方士人,称之为亡国之馀。这就加剧了南北士人的矛盾和对抗,不利于国家发展。二陆兄弟与顾荣乃是吴平入洛的南士精英,时人谓之三俊。张华是深谋远虑的政治家,太康文坛的老将,以善奖掖人物著称。史称“穷贱候门之士有一介之善者,便咨嗟称咏,为之延誉”。(\CJKunderwave{晋书}本传)华于诸人有龙、凤之评,既出于其爱惜人才的心理,亦出于抚慰南士心灵创伤的“统一战线”的政治策略。龙、凤乃中华民族文化传统中至大至美之图腾,张华将其赠予陆、顾诸人,可见其宽广博大的胸怀。褚陶仅以文学名世,张华高扬其一偏之才,给予同陆、顾诸人相提并论的评价。虽未免夸饰,然实令人感其至诚,对招揽东南之士必将起到积极作用。张华品评词句又有艺术,“龙跃云津”喻英才崛起,暗切二陆家乡华亭(云间),语义双关。}

\lettrine{8.20} 有问秀才\myidx{蔡洪}\footnote{秀才:指蔡洪。字叔开,吴郡人。晋惠帝元康初为松滋令,有才名,著\CJKunderwave{孤奋论}。}:“吴旧姓何如\footnote{吴:吴郡。旧姓:旧族,历史悠久的名门望族。}?”答曰:“吴府君\myidx{吴展}\footnote{吴府君:指吴展。展字士季,三国吴人,官吴郡太守。},圣王之老成\footnote{老成:指年高有德者。},明时之隽乂\footnote{“隽乂”:才智高明俊秀出众的人。};朱永长\myidx{朱诞}\footnote{朱永长:朱诞,字永长,三国吴吴郡(治所在今苏州)人。举贤良,累迁至议郎。},理物之至德\footnote{理物:从政治民。至德:有高尚道德的人。},清选之高望;严仲弼\myidx{严隐}\footnote{严仲弼:严隐,字仲弼,三国吴吴郡人。},九皋之鸣鹤\footnote{九皋之鸣鹤:\CJKunderwave{诗经·小雅·鹤鸣}:“鹤鸣于九皋,声闻于野。”九皋,深远的水潭淤地。鹤鸣九皋比喻人的声名高远。},空谷之白驹;顾彦先\myidx{顾荣}\footnote{顾彦先(?—312):顾荣字彦先,吴郡人。两晋之际江南士族领袖之一,与陆机、陆云同时入洛,时称“三俊”。南渡后,代表江南士族拥护和支持司马睿在江南开国,是为东晋。洛阳:西晋京师。},八音之琴瑟\footnote{八音:古代称钟、磬、琴瑟等八种乐器。琴瑟:八音中之弦乐。声音悠扬华美。},五色之龙章\footnote{五色:青、黄、赤、白、黑为五色。泛指各种色彩。龙章:龙形图纹,用于帝王、诸侯礼服,或仪卫军旗等。比喻文采光明显耀。};张威伯\myidx{张鴫}\footnote{张威伯:张鴫,字威伯,西晋初吴郡人。禀性坚正,志趣高洁。},岁寒之茂松,幽夜之逸光;陆士龙\myidx{陆云}\footnote{陆士龙:据袁本,“士龙”前增“士衡”二字,是。 陆士衡、士龙,即陆机、云兄弟。},鸿鹄之裴回\footnote{鸿鹄:大雁,即天鹅。 裴回:同“徘徊”。},悬鼓之待槌。{\fzxk\zihao{6}\textcolor{red}{秀才,蔡洪也。集载洪与刺史周俊书曰:“一日侍坐,言及吴士,询干(于)刍荛,遂见下问。造次承颜,载辞不举,敕令条列名状,退辄思之。今称疏所知:吴展字士季,下邳人。忠足矫非,清足厉俗,信可结神,才堪干世。仕吴为广州刺史、吴郡太守。吴平,还下邳,闭门自守,不交宾客。诚圣王之老成,明时之隽乂也。朱诞字永长,吴郡人。体履清和,黄中通理。吴朝举贤良,累迁议郎。今归在家。诚理物之至德,清选之高望也。严隐字仲弼,吴郡人。禀气清纯,思度渊伟。吴朝举贤良,宛陵令。吴平,去职。九皋之鸣鹤,空谷之白驹也。张鴫字威伯,吴郡人。禀性坚明,志行清朗,居磨涅之中,无淄磷之损。岁寒之松柏,幽夜之逸光也。”\CJKunderwave{陆云别传}曰:“云字士龙,吴大司马抗之弟(第)五子,机同母之弟也。儒雅有俊才,容貌瑰伟,口敏能谈,博闻强记。善箸述,六岁便能赋诗,时人以为项托、扬乌之畴也。年十八,刺史周俊命为主簿,俊常叹曰:‘陆士龙,当今之颜渊也。’累迁太子舍人、清河内史。为成都王所害。”}} 凡此诸君,以洪笔为鉏耒\footnote{洪笔:大笔。鉏:同“锄”。耒:木制翻土农具。},以纸札为良田,以玄默为稼穑\footnote{玄默:沉静寡言。稼穑:播种和收获。泛指农事劳动。},以义理为丰年,以谈论为英华,以忠恕为珍宝,箸文章为锦绣,蕴五色为缯帛\footnote{缯帛:丝绸。},坐谦虚为席荐\footnote{席荐:席子,坐垫。},张义让为帷幕,行仁义为室宇,修道德为广宅。”{\fzxk\zihao{6}\textcolor{red}{案:蔡所论士十六人,无陆机兄弟。又无“凡此诸君”以下,疑益之。}}

{\cangkai\zihao{5}【评】故事与\CJKunderwave{言语}门第二十三则“蔡洪赴洛”似同出一源。蔡洪出身吴郡,地域纽带使其对吴国旧姓天然地抱有浓厚感情,故一一道来,如数家珍。江南山川秀丽,人杰地灵,加以蔡洪踵色增华,出之以优美意象,足令人发神往之思。平吴之后,国家统一,中原士族之优越感空前膨胀。一批持狭隘本位立场的中原世族,视江南世族为亡国之馀,取歧视态度,造成严重的能量内耗,削弱了国家实力。吴地才士,在压抑中求生存。蔡洪极尽能事夸耀吴国旧姓,是对地域文化的自觉护卫,同时使人想起了关于南北文化差异较有名的一桩公案:北地才士王尔烈至南方,南人以“江南千山千水千夫子”相炫耀,王便以“塞北一地一天一圣人”反击,便足以说明这种隔阂的源远流长。即便在文化充分融合的今天,南北人之文化隔阂尚不能说已完全消除。在普通民众口中,“北方佬”、“小南蛮”之类口头禅,还是具有一定情感、文化色彩的象征符号。}

\lettrine{8.21} 人问王夷甫\myidx{王衍}\footnote{王夷甫:王衍(256—311)字夷甫,见刘孝标注。“以清虚通理称”,为当时清谈名家,“妙悟若神”,“妙善玄言,唯谈\CJKunderwave{老}、\CJKunderwave{庄}为事”。为政多谋略,不以经国为念,而善思自全之计,然终为石勒所害。(见\CJKunderwave{晋书}本传)注。}:“山巨源\myidx{山涛}义理何如\footnote{山巨源:山涛,车骑:此指谢玄,谢安侄,死后追赠车骑将军。}?是谁辈?”王曰:“此人初不肯以谈自居\footnote{初不:完全不,从不。谈:谈玄,清谈名理。},然不读\CJKunderwave{老}、\CJKunderwave{庄}\footnote{老、庄:\CJKunderwave{老子}、\CJKunderwave{庄子}。魏晋玄学家崇尚\CJKunderwave{老子}、\CJKunderwave{庄子}和\CJKunderwave{周易},总称“三玄”,成为玄学清谈的主要题目。},时闻其咏\footnote{时:时常,常常。咏:讽诵。},往往与其旨合。”{\fzxk\zihao{6}\textcolor{red}{顾恺之\CJKunderwave{画赞}曰:“涛有而不恃,皆此类也。”}}

{\cangkai\zihao{5}【评】评者王衍与被评者山涛恰形成鲜明对照。山涛为竹林中人,虽性好老庄,然并不以清谈自高,可谓老子所谓“为而不恃”;王衍以名士自任,口中雌黄,其举止却最终违背道家主旨。山涛领会的是玄学的灵魂,作为儒家思想的有益补充,化为滋润心灵的营养,故无往而不达;王衍则照搬照抄,空得玄学之形式外壳,又缺乏儒者方正刚毅的人格底蕴,最后落得身败名裂的下场。}

\lettrine{8.22} 洛中雅雅有三嘏\footnote{洛中:指洛阳。雅雅:文雅之士众多貌。}:刘粹\myidx{刘粹}字纯嘏\footnote{刘粹:字纯嘏,西晋沛国相(今安徽濉溪西北)人。},宏\myidx{刘宏}字终嘏\footnote{宏:刘宏,字终嘏。刘粹弟。},漠\myidx{刘漠}字冲嘏\footnote{漠:刘漠,字冲嘏。},是亲兄弟,王安丰\myidx{王戎}甥\footnote{王安丰:王戎。},并是王安丰女婿。宏,真长\myidx{刘惔}祖也\footnote{真长:刘惔,字真长,曾任丹阳尹,故称。谢安妻兄,尚明帝女庐陵公主。会稽王司马昱为相,与王濛并为其座上清谈之客。性简贵自重,与王羲之友善。卒年三十六。}。{\fzxk\zihao{6}\textcolor{red}{\CJKunderwave{晋诸公赞}曰:“椊(粹),沛国人,历侍中、南中郎将。宏历秘书监、光禄大夫。”\CJKunderwave{晋后略}曰:“漠少以清识为名,与王夷甫友善,并好以人伦为意。故世人许以才智之名。自相国右长史出为襄(湘)州刺史,以贵简称。”案:\CJKunderwave{刘氏谱},刘邠妻武周女,生粹、宏、漠,非王氏甥。}} 洛中铮铮冯惠卿\myidx{冯荪}\footnote{铮铮:形容人名声响亮。冯惠卿:冯荪,字惠卿,西晋长乐(今河南安阳东)人。},名荪,是播\myidx{冯播}子\footnote{播:冯播,字友声。}。{\fzxk\zihao{6}\textcolor{red}{\CJKunderwave{晋后略}曰:“播字友声,长乐人,位至大宗正。生荪。”\CJKunderwave{八王故事}曰:“荪少以才悟,识当世之宜,蚤历清职,仕至侍中。为长沙王所害。”}} 荪与邢乔\myidx{邢乔}俱司徒李胤\myidx{李胤}外孙\footnote{李胤:字宣伯,西晋辽东襄平(今辽宁辽阳)人。官至司徒。},及胤子顺\myidx{李顺}并知名\footnote{顺:李顺,字真长,一说字曼长。}。时称“冯才清,李才明,纯粹邢\footnote{纯粹:谓人品质纯净。}。”{\fzxk\zihao{6}\textcolor{red}{\CJKunderwave{晋诸公赞}曰:“乔字曾伯,河间人。有才学,仕至司隶校尉。慎字曼长,仕至太仆卿。”}}

{\cangkai\zihao{5}【评】故事运用了人物品评中“事数标榜” 和“音节相谐”两种常用方法。事数标榜是中古时期常见的文化现象,也是人物品藻的重要方式之一,由东汉末年清议名士开创。如三君、四友、敦煌五龙、卞氏六龙、八俊等等,都是概括某一些人或某一类人的群体特征。事数标榜的称谓方式,对后世有较大影响,如文学史所谓的文章四友、江湖四灵、苏门四学士等,声气相求、类聚群分,当然其内涵也由原来的人物品评转为流派概括。雅雅,就是儒雅家风烙刻在刘氏兄弟三人身上的外在标志。另外雅、嘏,铮、卿,清、明、邢等均音节琅琅,声调相谐而添彩增色。}

\lettrine{8.23} 卫伯玉\myidx{卫瓘}为尚书令\footnote{卫伯玉:卫瓘,字伯玉,西晋初河东安邑(今山西运城东北)人。卫筁之祖父。尚书令:官名。尚书省长官,负责政令。},见乐广\myidx{乐广}与中朝名士谈议\footnote{乐广:字彦辅,南阳人。乐广(?—304):字彦辅,南阳淯阳(今河南南阳东南)人。少孤贫,寒素为业,与物无竞。其清谈析理,与王衍并称,卫瓘以为有正始遗风。官至尚书令,八王乱中,以故忧卒注。中朝:晋代南渡以后,称西晋为中朝。},奇之,曰:“自昔诸人没已来,常恐微言将绝\footnote{微言:精深微妙的言辞。此指玄学清谈。},今乃复闻斯言于君矣!”命子弟造之\footnote{造:拜访。},曰:“此人,人之水镜也\footnote{水镜:比喻人的思想或性格如静水、如明镜一般清明。},见之若披云雾睹青天\footnote{披:分开。}。”{\fzxk\zihao{6}\textcolor{red}{\CJKunderwave{晋阳秋}曰:“尚书令卫瓘见广曰:‘昔何平叔诸人没,常谓清言尽矣。今复闻之于君。’”王隐\CJKunderwave{晋书}曰:“卫瓘有名理,及与何晏、邓飏等数共谈讲,见广,奇曰:‘每见此人则莹然,犹廓云雾而睹青天也。’”}}

{\cangkai\zihao{5}【评】卫瓘“常恐微言将绝,今乃复闻斯言”云云,王敦品目卫玠亦有类此之评,以拟“孔子没而微言绝”之意。乐广在西晋是一个独特的存在,他以智者超凡的悟性将儒、玄二道融为一体,化作独立不倚、左右俱宜的人生智慧。他对王澄、胡毋辅之等玄学末流的任性放达、裸体之游,付之一笑;“名教内自有乐地”一语,足见其在众人愦愦之际,保持了我之昭昭。他并不固守儒家思想一端,而是赋予其玄学的时代内涵。\CJKunderwave{晋书}本传载其“所在为政,无当时功誉,然每去职,遗爱为人所思”。可见受道家无为思想影响较深,并不重风光一时的面子工程。乐广思想中还有可贵的唯物主义和辩证法倾向,如对卫玠阐释梦的含义为“想”,为客人解释“杯弓蛇影”现象的成因,在“闹妖怪”的官舍居住,均体现了知识分子的独立思考和理性判断。乐广见识广博,故能释疑析理,境界圆通,涵容无滞,周流不居,是一个不可多得的思想家和学问家,故卫瓘有“水镜”美称。}

\lettrine{8.24} 王太尉\myidx{王衍}曰\footnote{王太尉:王衍,(256—311)字夷甫,见刘孝标注。“以清虚通理称”,为当时清谈名家,“妙悟若神”,“妙善玄言,唯谈\CJKunderwave{老}、\CJKunderwave{庄}为事”。为政多谋略,不以经国为念,而善思自全之计,然终为石勒所害。(见\CJKunderwave{晋书}本传)注。}:“见裴令公\myidx{裴楷}精明朗然\footnote{裴令公:裴楷,裴令公:即裴楷,曾官中书令,故云,又称“裴令”。善\CJKunderwave{老}、\CJKunderwave{易},当时著名清谈名家。二国租钱:指从梁、赵二国税收所获钱财。精明:精细明察。朗然:高洁开朗。},笼盖人上\footnote{笼盖:高出……之上。},非凡识也。若死而可作\footnote{作:起,起来。“死而可作”,语出\CJKunderwave{礼记·檀弓下}:“死而如可作也,吾谁与归?”},当与之同归。”或云王戎\myidx{王戎}语\footnote{王戎:嵇康(223—262):三国时谯郡铚 (今安徽亳县)人。“竹林七贤”之一。曾任中散大夫,故称嵇中散。当时著名思想家、文学家、清谈名家。因其主张越名教而任自然,抨击礼法之士,不与司马氏统治集团合作,盛年被杀。}。{\fzxk\zihao{6}\textcolor{red}{\CJKunderwave{礼记}曰:“赵文子与叔誉观于九原,文子曰:‘死者如可作也,吾谁与归?’”郑玄曰:“作,起也。”}}

{\cangkai\zihao{5}【评】程炎震通过考证官职,以此为王衍语;朱铸禹以为王夷甫不喜裴楷,当是王戎语。余意倾向程说。裴楷清通,得玄谈之神旨,故无往而不达。\CJKunderwave{晋书}载其“风神高迈,容仪俊爽,博涉群书,特精理义”,可与“精明朗然,笼盖人上”相印证。故王衍有“与之同归”之妙赏。}

\lettrine{8.25} 王夷甫\myidx{王衍}自叹\footnote{王夷甫:王衍。}:“我与乐令\myidx{乐广}谈\footnote{乐令:乐广。},未尝不觉我言为烦\footnote{烦:繁杂。}。”{\fzxk\zihao{6}\textcolor{red}{\CJKunderwave{晋阳秋}曰:“乐广善以约言厌人心,其所不知,默如也。太尉王夷甫、光禄大夫裴叔则能清言,常曰:‘与乐君言,觉其简至,吾等皆烦也。’”}}

{\cangkai\zihao{5}【评】晋人尚简约,故高坐道人不作汉语,简文评曰:“以简应对之烦”,不觉其短,而服其高,可见时代风气。乐广辞约旨达,是化繁为简的高手。如卫玠总角时,与其讨论梦的成因,乐广仅示一“想”字,卫玠琢磨不得,遂以成病;再如与客人探讨“旨不至”的命题,直以麈尾柄触几案,问客:“至不?”客曰:“至。”因又举麈尾,曰:“若至者,那得去?”客亦服其致。简约风格已深入晋人玄谈,后来禅宗顿悟的思维方式与之类似。于此可见影响。}

\lettrine{8.26} 郭子玄\myidx{郭象}有隽才\footnote{郭子玄:郭象。隽才:卓越的才智。},能言老、庄,庾敳\myidx{庾敳}尝称之\footnote{庾敳:字子嵩。},每曰:“郭子玄何必减庾子嵩\footnote{何必:为什么一定。减:不如,比……差。}!”{\fzxk\zihao{6}\textcolor{red}{\CJKunderwave{名士传}曰:“郭象字子玄,(自)黄门郎为太傅主簿,任事用势,倾动一府。敳谓象曰:‘卿自是当世大才,我畴昔之意都已尽矣!’其伏理推心,皆此类也。”}}

{\cangkai\zihao{5}【评】庾敳于道家玄旨有着天然超常的领悟力,尝读\CJKunderwave{老}、\CJKunderwave{庄},曰:“正与人意暗同。”郭象亦是治“庄”的专家,曾注\CJKunderwave{庄子},清代郭庆藩\CJKunderwave{庄子集释}就保留了郭注,见其影响之深。其注以玄解庄,带有时代色彩。庾敳每言:“郭子玄何必减庾子嵩!”其伏理推心、坦诚相待之意,恰和魏晋名士间互相褒扬、赏誉的良好传统相凑泊,而绝无后世文人相轻乃至百般诋毁的恶习。庾敳因与郭象有心灵上的共鸣,故能逢人说“象”,大相推介。但正常的学术交流,难在缺乏纯净的容身空间,有时不免受到学者政治态度的影响。当郭象做了太傅司马越的主簿,庾敳已很难与郭象平等相待,而是被迫为自己套上层层保护铠甲。敳对象曰:“卿自是当世大才,我畴昔之意都已尽矣。”(\CJKunderwave{晋书}敳本传)政治分野使人际温情受到如此“异化”,不能不说是莫大的悲哀!}

\lettrine{8.27} 王平子\myidx{王澄}目太尉\myidx{王衍}\footnote{王平子:王澄,乐广(?—304):字彦辅,南阳淯阳(今河南南阳东南)人。少孤贫,寒素为业,与物无竞。其清谈析理,与王衍并称,卫瓘以为有正始遗风。官至尚书令,八王乱中,以故忧卒注。王衍弟。目:品评。太尉:王衍。}:“阿兄形似道\footnote{道:此指有道之人。},而神锋太隽\footnote{神锋:神采锋芒。隽:特出。}。”太尉答曰:“诚不如卿落落穆穆\footnote{落落穆穆:疏淡平和。}。”{\fzxk\zihao{6}\textcolor{red}{王隐\CJKunderwave{晋书}曰:“澄通朗好人伦,情无所系。”}}

{\cangkai\zihao{5}【评】王衍、王澄二人,虽为同出兄弟,而风度气质迥然不同。“神锋太俊”、“落落穆穆”寥寥数字,能大体概括二人性格特点。王世懋曰:“兄弟间品题略尽”,甚当,可知兄弟间相知甚深。王衍以名士自矜,刻意矫饰,顾影自怜。“口未尝言钱”一例,看似脱俗,实大大损其自然;王澄则率性而为,师心自任。将镇荆州,为封疆大吏,责任重大,送者倾朝,澄则上树捉鸟,神气萧然,旁若无人,将庄重严肃的饯行气氛解构于无形,遂入任诞一路。王澄不能为王衍之刻意,王衍不能为澄之任心。}

\lettrine{8.28} 太傅府\myidx{司马越}有三才\footnote{太傅:东海王司马越。}:刘庆孙\myidx{刘舆}长才\footnote{刘庆孙:刘舆。长才:高才,多才。},{\fzxk\zihao{6}\textcolor{red}{\CJKunderwave{晋阳秋}曰:“太傅将召刘舆,或曰:‘舆,犹腻也,近将汙人。’太傅疑而御之。舆乃密视天下兵簿,诸屯戍及仓库处所,人谷多少,牛马器械,水陆地形,皆默识之。是时军国多事,每会议事,自潘滔以下皆不知所对,舆便屈指筹计,所发兵仗处所、粮廪运转,事无凝滞。于是太傅遂委仗之。”}} 潘阳仲\myidx{潘滔}大才\footnote{潘阳仲:潘滔,(?—311),字阳仲,西晋荥阳(在今河南)人。潘岳之侄。大才:超群出众之才。},裴景声\myidx{裴邈}清才\footnote{裴景声:裴邈,字景声,西晋河东闻喜(今属山西)人。裴頠从弟。清才:清俊之才。}。{\fzxk\zihao{6}\textcolor{red}{\CJKunderwave{八王故事}曰:“刘舆才长综覈,潘滔以博学为名,裴邈强立方正。皆为东海王所昵,俱显一府。故时人称曰:‘舆长才,滔大才,邈清才也。’”}}

{\cangkai\zihao{5}【评】刘舆乃刘琨兄,为东海王司马越左长史,是专门寻人之过而行构陷之实的小人。品行卑下与弟琨恍若天渊,但治事之才令人啧啧称奇。史称“宾客满筵,文案盈机,远近书记日有数千,终日不倦,或以夜继之,皆人人欢畅,莫不悦附”。因此获“长才”之誉,从惟才是举的角度看,未为不可。潘滔在王敦小时,预测敦“蜂目已露,但豺声未振”,有一定人伦识鉴,目之“大才”,亦不算牵强。对裴邈“清才”之评价,因合于时代精神,故属于至高的评价。观三才之中,长才为一偏之才,大才超出众类,清才最受推崇,一字之中能看出皮里阳秋之义。}

\lettrine{8.29} 林下诸贤\myidx{竹林七贤}\footnote{林下诸贤:魏晋间山涛、阮籍、嵇康、向秀、刘伶、阮咸、王戎七名士,常共游宴于竹林之下,人称“竹林七贤”。},各有隽才子\footnote{隽才:有卓越才智的人。}:籍\myidx{阮籍}子浑\myidx{阮浑}\footnote{籍子浑:阮籍的儿子阮浑,字长成。},器量弘旷\footnote{器量:器局度量。弘旷:宏大旷达。};{\fzxk\zihao{6}\textcolor{red}{\CJKunderwave{世语}曰:“浑字长成,清虚寡欲,位至太子中庶子。”}} 康\myidx{嵇康}子绍\myidx{嵇绍}\footnote{康子绍:嵇康的儿子嵇绍,字延祖。二十八岁出仕,性刚烈,敢直谏,忠于晋室,八王乱时,随惠帝与成都王司马颖战,身翼惠帝,被箭而血染帝衣。晋元帝时谥“忠穆”,载\CJKunderwave{晋书·忠义传}。},清远雅正;{\fzxk\zihao{6}\textcolor{red}{已见。}} 涛\myidx{山涛}子简\myidx{山简}\footnote{涛子简:山涛的儿子山简,字季伦。},疏通高素\footnote{疏通:疏放通达。 高素:高雅朴素。};{\fzxk\zihao{6}\textcolor{red}{虞预\CJKunderwave{晋书}曰:“简字秀伦,平雅有父风,与嵇绍、刘漠等齐名,迁尚书,出为征南将军。”}} 咸\myidx{阮咸}子瞻\myidx{阮瞻}\footnote{咸子瞻:阮咸的儿子阮瞻,字千里。},虚夷有远志\footnote{虚夷:恬淡寡欲。},瞻弟孚\myidx{阮孚}\footnote{瞻弟孚:阮瞻的弟弟阮孚,阮咸次子,晋元帝世为安东参军,历侍中、吏部尚书、丹阳尹、广州刺史等。},爽朗多所遗\footnote{爽朗:直爽开朗。 多所遗:指不拘小节、不矜细行的性格。};{\fzxk\zihao{6}\textcolor{red}{\CJKunderwave{名士传}曰:“瞻字千里,夷任而少嗜欲,不修名行,自得于怀,读书不甚研求而识其要。仕至太子舍人,年三十卒。”\CJKunderwave{中兴书}曰:“孚风韵疏诞,少有门风。初为安东参军,蓬发饮酒,不以王务婴心。”}} 秀\myidx{向秀}子纯\myidx{向纯}、悌\myidx{向悌}\footnote{秀子纯、悌:向秀的儿子向纯、向悌。纯,字长悌。悌,字叔逊。},并令淑有清流\footnote{令淑:美好善良。 有清流:清高而有时望。};{\fzxk\zihao{6}\textcolor{red}{\CJKunderwave{竹林七贤论}曰:“纯字长悌,位至侍中。悌字叔逊,位至御史中丞。”\CJKunderwave{晋诸公赞}曰:“洛阳败,纯、悌出奔,为贼所害。”}} 戎\myidx{王戎}子万子\myidx{王绥}\footnote{戎子万子:王戎的儿子王万子,名绥,字万子。},有大成之风\footnote{大成:成大器。},苗而不秀\footnote{苗而不秀:\CJKunderwave{礼记·子罕}:“苗而不秀者有矣夫!”是孔子哀叹弟子颜渊早死的话,后用来比喻人才能尚未发挥而早逝。};{\fzxk\zihao{6}\textcolor{red}{\CJKunderwave{晋诸公赞}曰:“王绥字万子,辟太尉掾,下(不)就,年十九卒。”\CJKunderwave{晋书}曰:“戎子万,有美号而太肥,戎令食糠,而肥愈甚也。”}} 唯伶\myidx{刘伶}子无闻\footnote{伶子:刘伶的儿子。无闻:没有名声。}。凡此诸子,唯瞻为冠\footnote{冠:最优。},绍\myidx{嵇绍}、简\myidx{山简}亦见重当世。

{\cangkai\zihao{5}【评】故事专论竹林诸贤子弟,为我们从遗传学、教育学等角度审视晋人家庭,提供了典型的素材。从遗传学的角度看,阮浑、嵇绍、山简、阮瞻阮孚兄弟、向纯向悌兄弟、万子诸人,均遗传了父辈造化所钟的超凡资质。或成为立事、立功而名垂青史的政治家和晋室忠臣,如嵇绍、山简;或成为各具风姿神情的玄学名士,如万子、阮氏兄弟。其中阮瞻深合时代精神,最为清通。从教育学的角度看,阮氏老一辈纵情任心,无意仕宦,其子弟也多以名士风度,而非政治功业显;山涛仕途顺利,位至公卿,其子山简以政声致誉;嵇康虽桀骜不驯,然亦有谆谆\CJKunderwave{家诫},嵇绍竟成为晋室忠臣。诸俊才之子的成功,家庭教育当占其中分量较重的比例。此间惟刘伶子湮没无闻,当与其酗酒无度,与优生学原则相悖有莫大关系。晋人好酒,自是通累,刘伶为晋人中酗酒之尤甚者,后世称其为酒仙,遂成为酒店招牌人物。}

\lettrine{8.30} 庾子躬\myidx{庾琮}有废疾\footnote{庾子躬:庾琮,字子躬。废疾:残疾。},甚知名,家在城西,号曰“城西公府”。{\fzxk\zihao{6}\textcolor{red}{虞预\CJKunderwave{晋书}曰:“琮字子躬,颍川人,太常峻弟(第)二子,仕至太尉掾。”}}

{\cangkai\zihao{5}【评】刘孝标注以为庾琮为庾峻第二子,\CJKunderwave{晋书}峻本传载其二子:珉、敳,不见有琮,诸贤亦未辨。庾琮为太尉掾,因有废疾而家居。“城西公府”之号,盖谓其家地处偏僻,而名士魅力光芒四射,希心企慕者趋之若鹜,宾客络绎不绝、车马填巷,有如公府。秦大臣李斯位极人臣,其子李由省亲回家,借机贿赂者亦车马填巷,李斯有物禁大胜之叹。庾琮与李斯获誉当时,看似相同,而其本质南辕北辙。李斯位高权重,官吏们因缘射利而自甘谄媚;这就与废疾居家、无职无权的庾琮“太公钓鱼,愿者上钩”的交友方式异趣。魏晋名士追求自由适意的精神生活,许多人并不把案牍之事太过看重,宁愿参加一些与升官发财无涉的无功利性精神文化沙龙。魏晋风度以此胜出一筹。}

\lettrine{8.31} 王夷甫\myidx{王衍}语乐令\myidx{乐广}\footnote{王夷甫:王衍。乐令:乐广。}:“名士无多人,故当容平子\myidx{王澄}知\footnote{故当:自然,当然。容:允许。平子:王澄,王衍弟,乐广(?—304):字彦辅,南阳淯阳(今河南南阳东南)人。少孤贫,寒素为业,与物无竞。其清谈析理,与王衍并称,卫瓘以为有正始遗风。官至尚书令,八王乱中,以故忧卒注。知:知道。}。”{\fzxk\zihao{6}\textcolor{red}{\CJKunderwave{王澄别传}曰:“澄风韵迈达,志气不群。从兄戎、兄夷甫名冠当年,四海人士一为澄所题目,则二兄不复措意,云:‘已经平子。’其见重如此,是以名闻益盛。天下知与不知,莫不倾注。澄后事迹不逮,朝野失望。及旧游识见者,犹曰:‘当今名士也。’”}}

{\cangkai\zihao{5}【评】王衍虽祖述老、庄,以无为本,而他于道家玄理,只不过是其取名世、装点门面的终南捷径而已,其高谈阔论的口实,根本经不起现实行动的检验。王衍在清读领袖乐广面前,大力提携弟弟王澄,正透露出其胸中无法释怀的家族门第私计。王衍深知名士间舆论力量,可以化腐朽为神奇,故时时在乐广之类的大牌名士面前,为王澄作免费广告宣传。虽然手足之情可怀可感,但却难以掩其为门户计的自私本质。}

\lettrine{8.32} 王太尉\myidx{王衍}云\footnote{王太尉:王衍。}:“郭子玄\myidx{郭象}语议如悬河写水\footnote{郭子玄:郭象。语议:指谈论玄学。悬河:瀑布。写:通“泻”。倾泻。},注而不竭。”{\fzxk\zihao{6}\textcolor{red}{\CJKunderwave{名士传}曰:“子玄有隽才,能言庄、老。”}}

{\cangkai\zihao{5}【评】魏晋玄学的发展流变中,郭象是一位集大成的人物。其玄学思想,整合了嵇康、阮籍的以道批儒,裴頠、孙盛的以儒攻道,继承并发展了王弼“贵无”、裴頠“崇有”诸论,适逢其时地提出自己“独化”的理论主张,并反映在\CJKunderwave{庄子注}中。所谓“独化”,指现象界一切事物是独自、孤立、无所凭依地生成变化,即“外不资于道,内不由于己,掘然自得而独化也”(\CJKunderwave{庄子·大宗师注})。郭象的理论,折衷于名教与自然,是对当时各种玄学主张的总结和调和,是玄学发展愈加精致化的标志。王衍评其谈论“如悬河泻水,注而不竭”,可见其理论自成系统,具有一种生生不息的创造力。}

\lettrine{8.33} 司马太傅\myidx{司马越}府多名士\footnote{司马太傅:东海王司马越。},一时隽异\footnote{一时:指当世,当时。隽异:指卓越特出的人才。}。庾文康\myidx{庾亮}云\footnote{庾文康:庾亮,庾亮(289—340)的敬称。他历仕东晋元、明、成三朝,作为外戚,曾执国政,显赫于朝。的卢:传说中的凶马之名,骑之不利主人。}:“见子嵩\myidx{庾敳}在其中\footnote{子嵩:庾敳,此时为太傅从事中郎。},常自神王\footnote{神王:精神振奋。王,通“旺”。}。”{\fzxk\zihao{6}\textcolor{red}{\CJKunderwave{晋阳秋}曰:“敳为太傅从事中郎。”}}

{\cangkai\zihao{5}【评】\CJKunderwave{晋书}本传载敳为陈留相时,未尝以事婴心,“从容酣畅,寄通而已。处众人中,居然独立”。与“常自神王”之意正合。敳借助道家的慧眼,看透了世间政治的无常纷争,自能元气内充,悠游其间。因能保全心灵之自足境界,而常于喧嚣中获取一份诗意快感,虽处官府之中,如游山川丘壑。与夫为蝇头微利而蜂聚蚁争之徒相比,自有天渊之别!}

\lettrine{8.34} 太傅东海王\myidx{司马越}镇许昌\footnote{太傅东海王:司马越。许昌:县名。在今河南。},以王安期\myidx{王承}为记室参军\footnote{王安期:王承。 记室参军:官名。诸王、三公、大将军等所设属官,掌表章文书等。},雅相知重\footnote{雅:素常。知重:赏识器重。}。敕世子毗\myidx{司马毗}曰\footnote{敕:告诫,诫饬。世子:帝王或诸侯正妻所生的长子。毗:司马毗,东海王越子。}:“夫学之所益者浅\footnote{益:受益。},体之所安者深\footnote{体:体验履践。安:感到满意、合适。}。闲习礼度\footnote{闲习:熟悉。反复演习。},不如式瞻仪形\footnote{式瞻:瞻。式,发语词。仪形:仪容形貌。};讽味遗言\footnote{讽味:讽咏玩味。遗言:死者留下来的话。},不如亲承音旨\footnote{亲承:亲自聆听。承,闻。音旨:同“辞旨”。言辞旨趣。}。王参军人伦之表\footnote{王参军:王承。 人伦之表:为人的表率。人伦,指有名望、有身份的人。},汝其师之\footnote{其:助词。表示祈使、期望。 师:师从;师法。}。”或曰:“王\myidx{王承}、赵\myidx{赵穆}、邓\myidx{邓攸}三参军人伦之表\footnote{王、赵、邓三参军:王承、赵穆、邓攸三位参军。},汝其师之。”谓安期、邓伯道、赵穆也。{\fzxk\zihao{6}\textcolor{red}{\CJKunderwave{赵吴郡行状}曰:“穆字季子,汲郡人。真淑平粹,才识清通,历尚书郎、太傅参军。代(后)太傅越与穆及王承、阮瞻、邓攸书曰:‘礼,八岁出就外傅,十年曰幼学,明可以渐先王之教也。然学之所受者浅,体之所安者深。是以闲习礼度,不如式瞻轨仪;讽味遗言,不如亲承辞旨。小儿毗既无令淑之资,未闻道德之风,欲屈诸君时以闲豫,周旋燕诲也。’穆历晋明帝师、冠军将军、吴郡太守,封南乡侯。”}} 袁宏\myidx{袁宏}作\CJKunderwave{名士传}\footnote{\CJKunderwave{名士传}:书名,袁宏撰。},直云王参军\footnote{直:通“特”,只,只是。}。或云:“赵家先犹有此本。”

{\cangkai\zihao{5}【评】两晋民间私学兴盛,成为官学的有益补充。司马越为世子延师课读,并对老师王承极尽赞美之词。事实证明,司马越于王承并非溢美。太尉王衍以王承比南阳乐广,而渡江名臣王导、卫玠、周顗、庾亮之徒皆出其下。请到王承这样的老师,可见司马越眼光犀利,有此贤父,是世子的造化。王承是一位有着儒者仁爱情怀的教育大师。\CJKunderwave{晋书}本传载,承为东海太守时,差役捉到一个因从师受学不觉日暮而犯夜的年轻学子,承曰:“鞭挞宁越以立威名,非政化之本。”于是命令下吏送还其回家。(宁越为春秋时期苦学成才的著名人物,后为周威王师。)故事说明,王承重视教育,深晓教育的精神实质,故不拘法律条文为下层寒士提供庇护。另外,东海王越训诫世子的一番话,也说明他对教育有独到的思考。其核心思想,就是反对生吞活剥地死记硬背,注重亲身体验、联系实际。这就必然为老师创造性地施教,提供宽松的氛围和良好的环境。}

\lettrine{8.35} 庾太尉\myidx{庾亮}少为王眉子\myidx{王玄}所知\footnote{庾太尉:庾亮,庾亮(289—340)的敬称。他历仕东晋元、明、成三朝,作为外戚,曾执国政,显赫于朝。的卢:传说中的凶马之名,骑之不利主人。注。王眉子:王玄,(?—313?),西晋琅邪临沂(今属山东)人,字眉子。王衍子。},庾过江,叹王曰:“庇其宇下\footnote{宇下:屋檐底下。比喻受到庇护。},使人忘寒暑。”{\fzxk\zihao{6}\textcolor{red}{\CJKunderwave{晋诸公赞}曰:“玄少希慕简旷。”\CJKunderwave{八王故事}曰:“玄为陈留太守,或劝玄过江投琅邪王。玄曰:‘王处仲得志于彼,家叔犹不免害,岂能容我?’谓其器宇不容于敦也。”}}

{\cangkai\zihao{5}【评】王玄与卫玠齐名,亦沾溉名士家风。庾亮感叹玄知遇之恩,情深义重。“使人忘寒暑”一句有二义,一为忘记了寒来暑往之季节更替,可见其玄谈有吸引力,使人不觉时间流逝;二为忘却了冷暖,意谓王玄有人格感召力,使人感到如坐春风般的舒适惬意。二义虽歧而可相互补充,读者在阅读过程中有更多的思考联想空间,从而凸现了魏晋名士言谈简约风格所带来的魅力。台湾著名诗人、学者余光中先生在\CJKunderwave{朋友四型}一文中,将朋友概括为四种:高级而有趣,高级而无趣,低级而有趣,低级而无趣。做一个可能不太恰当的比较,汉儒有点像高级而无趣的朋友,大概是古人所谓的诤友,甚至是畏友;魏晋名士则多为高级而有趣的朋友,使人敬而不畏,亲而不狎。王玄当属于高级而有趣的类型,故庾亮有“忘寒暑”之譬。}

\lettrine{8.36} 谢幼舆\myidx{谢鲲}曰\footnote{谢幼舆:谢鲲,谢豫章:谢鲲,曾作豫章太守。刘孝标注“鲲子别见”,“子”字衍。将:携,谓携之送客。自:已经。参:参与、进入。上流:上等、上品注。}:“友人王眉子\myidx{王玄}清通简畅\footnote{眉子:王玄,(?—313?),西晋琅邪临沂(今属山东)人,字眉子。王衍子。清通简畅:清明通达,简约疏放。},嵇延祖\myidx{嵇绍}弘雅劭长\footnote{嵇延祖:嵇绍,字延祖。二十八岁出仕,性刚烈,敢直谏,忠于晋室,八王乱时,随惠帝与成都王司马颖战,身翼惠帝,被箭而血染帝衣。晋元帝时谥“忠穆”,载\CJKunderwave{晋书·忠义传}。弘雅劭长:宽宏端正,美好高尚。},董仲道\myidx{董养}卓荦有致度\footnote{董仲道:董养,字仲道,西晋陈留浚仪(今河南开封西北)人。卓荦:卓越出众,不同流俗。致度:气度。}。”{\fzxk\zihao{6}\textcolor{red}{王隐\CJKunderwave{晋书}曰:“董养字仲道。太始初到洛,下(不)干禄求荣。永嘉中,洛城东北角步广里中地陷,中有二鹅,苍者飞去,白者不能飞。问之博识者,不能知。养闻,叹曰:‘昔周时所盟会狄泉,此地也。卒有二鹅,苍者胡象,后胡当入洛;白者不能飞,此国讳也。’”谢鲲\CJKunderwave{元化论序}曰:“陈留董仲道,于元康中见惠帝废杨悼后,升太学堂叹曰:‘建此堂也,将何为乎?每见国家赦书,谋反逆皆赦,孙杀王父母、子杀父母不赦,以为王法所不容也。奈何公卿处议,文饰礼典,以至此乎!天人之理既灭,大乱斯起。’顾谓谢鲲、阮孚曰:‘\CJKunderwave{易}称知几其神乎,君等可深藏矣。’乃与妻荷儋入蜀,莫知其所终。”}}

{\cangkai\zihao{5}【评】王玄有名士家风,谢鲲评其“清通简畅”,正寄托了晋人的审美好尚。嵇绍公忠体国、勠力晋室,谢鲲评绍“弘雅邵长”,符合汉儒贤良方正的风度。董养见大乱将作,乃与妻荷担入蜀,为“知机其神”的明哲君子。玄、绍、养三人,一道、一儒、一隐,各成佳胜。\CJKunderwave{语}曰:“益者三友”,谢鲲可谓“三径俱开”的益者。}

\lettrine{8.37} 王公\myidx{王导}目太尉\myidx{太尉}\footnote{王公:王导。太尉:王衍,王夷甫:王衍(256—311)字夷甫,见刘孝标注。“以清虚通理称”,为当时清谈名家,“妙悟若神”,“妙善玄言,唯谈\CJKunderwave{老}、\CJKunderwave{庄}为事”。为政多谋略,不以经国为念,而善思自全之计,然终为石勒所害。(见\CJKunderwave{晋书}本传)注。}:“岩岩清峙\footnote{岩岩:高耸貌。 清峙:挺拔的山峰。},壁立千仞\footnote{壁立:像峭壁一样笔直耸立。仞:古代长度单位,八尺(一说七尺)为一仞。}。”{\fzxk\zihao{6}\textcolor{red}{顾恺之\CJKunderwave{夷甫画赞}曰:“夷甫天形瑰特,识者以为岩岩秀峙,壁立千仞。”}}

{\cangkai\zihao{5}【评】魏晋人重姿容体貌、气质风神,就此而论,王衍有着罕见其匹的先天优势,其人好像是竞秀千岩中矗立不群的顶峰,又如毫无垒块赘石的千仞峭壁。王衍若活在今天,凭借其迷人的外表,不凡的举止和优雅的谈吐,定会成为少男少女疯狂崇拜的明星偶像。常人多数不能识破其假象,必为其蒙蔽一时。这也从一个侧面说明了无论是精英文化还是大众文化,都难免其肤浅的一面,精英文化中一般名士有蚁附权威的心理,大众文化更是经常掀起疯狂的偶像崇拜狂潮,二者都与健全、理性的文化心态相去甚远,应该不断修正自身。正像今人评价大汉奸胡兰成,胡文有气韵而无气节,正像他做人,有灵气而无灵魂。这话完全适用于一千七百多年前的王衍。}

\lettrine{8.38} 庾太尉\myidx{庾亮}在洛下\footnote{庾太尉:庾亮,庾亮(289—340)的敬称。他历仕东晋元、明、成三朝,作为外戚,曾执国政,显赫于朝。的卢:传说中的凶马之名,骑之不利主人。注。洛下:洛阳。},问讯中郎\myidx{庾敳}\footnote{问讯:问候。指礼节性问安。 中郎:指庾敳。敳曾作司马太傅从事中郎。},{\fzxk\zihao{6}\textcolor{red}{庾敳。}} 中郎留之云:“诸人当来。”寻温元甫\myidx{温几}、{\fzxk\zihao{6}\textcolor{red}{\CJKunderwave{晋诸公赞}曰:“温几字元甫,太原人。才性清婉,历司徒右长史、湘州刺史,卒官。”}} 刘王乔\myidx{刘畴}、{\fzxk\zihao{6}\textcolor{red}{曹嘉之\CJKunderwave{晋纪}曰:“刘畴字王乔,彭城人。父讷,司隶校尉。畴善谈名理,曾避乱坞壁,有胡数百欲害之,畴无惧色,援笳而吹之,为\CJKunderwave{出塞}、\CJKunderwave{入塞}之声,以动其游客之思。于是群胡皆泣而去之。位至司徒左长史。”}} 裴叔则\myidx{裴楷}俱至\footnote{温元甫:温几,字元甫,西晋太原(今属山西)人。刘王乔:刘畴,字王乔,西晋彭城(今江苏徐州)人。刘讷子。裴叔则:裴楷,裴令公:即裴楷,曾官中书令,故云,又称“裴令”。善\CJKunderwave{老}、\CJKunderwave{易},当时著名清谈名家。二国租钱:指从梁、赵二国税收所获钱财。},酬酢终日\footnote{酬酢:主宾互相敬酒,主敬客曰酬,客敬主曰酢。引申为宾朋间谈论应对。}。庾公犹忆刘、裴之才隽\footnote{才隽:卓越的才华。},元甫之清中\footnote{清中:心地清白。中,内心。}。{\fzxk\zihao{6}\textcolor{red}{“中”一作“平”。}}

{\cangkai\zihao{5}【评】庾亮为敳堂侄,虽早年从父过江,犹忆洛下时事。故事追记,是一次曾在庾敳家举行的文化沙龙。聚会的主客有庾敳、温几、刘畴、裴楷诸人,庾敳为一时士人领袖,名士来聚其家,有如辐辏。大概庾氏子弟特别具有亲和力,易形成以其为核心的名士集团,本门庾琮废疾在家而蔚成“城西公府”条可与此印证。故事表现的场面,恰合宋儒所言“活泼泼”的生活教育,名士间酬酢的风雅,在庾亮幼小的心灵深处留下了深刻的烙印,故过江之后,犹忆诸人之“清中”、“才俊”。}

\lettrine{8.39} 蔡司徒\myidx{蔡谟}在洛\footnote{蔡司徒:蔡谟,(281—356):字道明,东晋陈留考城(今河南民权东北)人。},见陆机\myidx{陆机}兄弟\myidx{陆云}住参佐廨中\footnote{陆机兄弟:陆机、陆云。参佐:僚属。廨:官署。},三间瓦屋,士龙住东头,士衡住西头。士龙为人文弱可爱\footnote{文弱:文雅柔弱。},士衡长七尺馀,声作钟声\footnote{钟声:\CJKunderwave{晋书·陆机传}称:“机身长七尺,其声如雷。”},言多慷慨\footnote{慷慨:意气风发,情绪激昂。}。{\fzxk\zihao{6}\textcolor{red}{\CJKunderwave{文士传}曰:“云性弘静,怡怡然为士友所宗。机清厉有风格,为乡党所惮。”}}

{\cangkai\zihao{5}【评】陆机、陆云惨死时,蔡谟年仅十九岁,可见故事为追忆之词。二陆兄弟为天才颖迈的南士精英,吴平入洛,怀负着超拔的家族理想走向西晋的政治舞台。故事所述细节,如住东屋、西屋,言语姿态等,富于生活情趣,读来栩栩如生;兄弟之音容笑貌,亦宛在目前,使人顿生沧海桑田、梓泽丘墟之感。陆云文弱可爱,息事宁人;陆机言多慷慨,锋芒毕露。然而他们的率真与才情,并没有为其带来好运,反而加速其成为复杂政治的牺牲品。王世懋评曰:“二陆即被祸,犹为名贤忆慕如此,盖以得见为幸也。”可见二陆兄弟在士人心目中的地位。}

\lettrine{8.40} 王长史\myidx{王濛}是庾子躬\myidx{庾琮}外孙\footnote{王长史:王濛。庾子躬:庾琮。},{\fzxk\zihao{6}\textcolor{red}{\CJKunderwave{王氏谱}曰:“濛父讷,娶颍川庾琮之女,字三寿也。”}} 丞相\myidx{王导}目子躬云\footnote{丞相:王导。}:“入理泓然\footnote{入理:指钻研玄理,深入玄理之中。泓然:幽深宽广。},我已上人\footnote{已上:以上。}。”{\fzxk\zihao{6}\textcolor{red}{子躬,子嵩兄也。}}

{\cangkai\zihao{5}【评】魏晋清谈的主要内容是辨名析理的形上探讨,前期以老、庄、易为“三玄”,后期则有佛教义理的比附参与。除姿容相貌、气度风神之外,领悟事理的能力,成为人物赏誉又一重要考察标准。庾琮有废疾,虽然外貌欠佳,但形残神全,通过钻研玄理为士林所认可。王导称其“入理泓然,我已上人”,评价相当高。}

\lettrine{8.41} 庾太尉\myidx{庾亮}目庾中郎\myidx{庾敳}\footnote{庾太尉:庾亮。庾中郎:庾敳。}:“家从谈谈之许\footnote{家从:我家从父(堂叔),指庾敳。谈谈:通“沈沈”、“潭潭”。指思想言论深邃。}。”{\fzxk\zihao{6}\textcolor{red}{\CJKunderwave{名士传}曰:“敳不为辨析之谈,而举其旨要,太尉王夷甫雅重之也。”一作“家从谈之祖”,“从”一作“诵”,“许”一作“辞”。}}

{\cangkai\zihao{5}【评】“谈谈”,深沉貌。深沉,故沉郁内敛。史载庾敳值天下多故、祸变屡起,常静默无为、袖手旁观。这是一种将通天尽人的生命智慧,化为不得已的自我保全之术,深沉中潜藏着几许苍凉和无奈。\CJKunderwave{晋书}本传载其与郭象关系的微妙变化,以及智对刘庆孙的刻毒构陷,都可以见出庾敳的良苦用心。没有人生来喜欢装疯卖傻,使自己理想和才情“匏瓜徒悬”;人们期待着海晏河清时代的到来。但对于大多数人而言,在历史的长河中,这样的境界大多是一种奢望,少数“假高衢而骋力”者是幸运儿,不能将其看成是政治的常态。庾敳的“谈谈之许”实在是迫不得已的外在掩饰!}

\lettrine{8.42} 庾公\myidx{庾亮}目中郎\myidx{庾敳}\footnote{庾公:庾亮。中郎:庾敳。}:“神气融散\footnote{融散:恬淡豁达。},差如得上\footnote{差如:颇为。得上:能够超拔向上。}。”{\fzxk\zihao{6}\textcolor{red}{\CJKunderwave{晋阳秋}曰:“敳颓然渊放,莫有动其听者。”}}

{\cangkai\zihao{5}【评】庾亮目从父敳“神气融散,差如得上”。类似于本门第三十三则所评之“常自神王”之意。从常态的审美标准而言,庾敳算得上畸形。论身材相貌,“长不满七尺,而腰带十围”,十围约今天的五尺,称得上极胖,徒能成为世俗取笑的对象。然而却因“神气融散”、“雅有远韵”(\CJKunderwave{晋书}本传),为士林所重。敳兄弟琮有废疾,更加不堪,却有“城西公府”美誉。合而观之可见,晋人赏誉虽重外貌之美,但却尤其赏识内在的精神气度之美,从而呈现出一种健全、宽容的心态。}

\lettrine{8.43} 刘琨\myidx{刘琨}称祖车骑\myidx{祖逖}为朗诣\footnote{刘琨:字越石。当时追从姨夫刘琨,在并州为谋主,“琨所凭恃焉”(\CJKunderwave{晋书·温峤传})。建武元年(317)奉刘琨命出使江南,拥戴司马睿即帝位,建立东晋王朝。受司马睿重用,留为散骑常侍,后官至中书令,为东晋名臣。注。祖车骑:祖逖(266—321),字士稚,东晋范阳遒县(今河北涞水)人。出身幽冀望族。青年时与刘琨同为司州主簿,俱以雄豪著称。中夜闻鸡起舞,常以恢复中原为己任。朗诣:开朗通达。诣,通“逸”。},曰:“少为王敦\myidx{王敦}所叹\footnote{王敦:王敦:字处仲,晋琅邪临沂(今属山东)人,王导堂兄。妻为晋武帝女襄城公主,拜驸马都尉。晋室东迁,与王导一起辅佐元帝,任要职,握重兵,镇守扬州、荆州等重镇。公元322 年起兵谋反,入京都建康。王含:见刘孝标注。光禄勋:官名,九卿之一,领管光禄、大中、中散、谏议等大夫及羽林郎、五官、虎贲、左右等中郎将注。叹:赞美。}。”{\fzxk\zihao{6}\textcolor{red}{虞预\CJKunderwave{书}曰:“祖逖字士稚,范阳遒人。豁荡不修仪检,轻财好施。”\CJKunderwave{晋阳秋}曰:“逖与司空刘琨俱以雄豪箸名。年二十四,与琨同辟司州主簿,情好绸缪,共被而寝。中夜闻鸡鸣,俱起,曰:‘此非恶声也。’每语世事,或中宵起坐,相谓曰:‘若四海鼎沸,豪杰共起,吾与足下相避中原耳。’为汝南太守,值京师倾覆,率流民数百家南度,行达泗口,安东板为徐州刺史。逖既有豪才,常慷慨以中原为己任。乃说中宗雪复神州之计,拜为豫州刺史,使自招募。逖遂率部曲百馀家,北度江,誓曰:‘祖逖若不清中原而复济此者,有如大江!’攻城略地,招怀义士。屡摧石虎,虎不敢复窥河南。石勒为逖母墓置守吏。刘琨与亲旧书曰:‘吾枕戈待旦,志枭逆虏,常恐祖生先吾箸鞭耳!’会其病卒,先有妖星见豫州分。逖曰:‘此必为我也,天未灭寇故耳。’赠车骑将军。”}}

{\cangkai\zihao{5}【评】祖逖闻鸡起舞、击楫中流,以刻石立功的方式实现对国家的贡献。他的故事流传广远,为中华大地上的热血儿女铭记在心,每当事关民族兴亡的易代之际,总能激发起国人同仇敌忾、保家卫国的爱国热情。南宋爱国诗人陆游诗曰:“刘琨死后无奇士,独所荒鸡泪满衣”(\CJKunderwave{夜归偶怀故人独孤景略})、“功名在子何殊我,惟恨无人快着鞭”(\CJKunderwave{书事}),就是引用了刘琨、祖逖这一对慷慨义士的典故,呼唤南宋抗金志士的出现。逖与刘琨俱为豪杰,为两晋士风带来一股清刚之气。王敦一代枭雄,与逖同年(均为266年出生),因年龄相同、气质相近,故易从其身上找到某种精神上的共鸣。可惜人生价值取向不同,东晋初年,王敦觊觎京师,祖逖则志复中原,二人是道不同而不相为谋。}

\lettrine{8.44} 时人目庾中郎\myidx{庾敳}\footnote{庾中郎:庾敳。}:“善于托大\footnote{托大:托身于玄默之大道。谓襟怀恢廓,超脱世事。“托”即超脱,不为世事所牵。},长于自藏\footnote{自藏:韬晦自隐,不露锋芒。}。”{\fzxk\zihao{6}\textcolor{red}{\CJKunderwave{名士传}曰:“敳虽居职任,未尝以事自婴,从容博畅,寄通而已。是时天下多故,机事屡起,有为者拔奇吐异,而祸福继之。敳常默然,故忧喜不至也。”}}

{\cangkai\zihao{5}【评】庾敳见世事混乱,有为者虽拔奇吐异而福祸相继,故采取托大自藏的办法以名哲自保。托大者,寄心博大,不拘细节;自藏者,以拱默自保也。但这是一个不讲常规的时代,玄学的障眼法,躲得过奸佞的利口,却躲不过胡虏的利刃,与王衍同为石勒所杀。纷乱的政治与无常的官场,消磨了多少有为之士的英雄豪气和无羁才情,造成了巨大的人才毁灭,只落得或红巾揾泪,或沉冥自藏,幸好有一个玄学天地聊以发挥过剩的能量。庾敳之死,说明个体对于自己的生死无能为力,这是时代和社会的悲哀,后人虽惋惜其智慧、才情,也只能徒唤奈何!}

\lettrine{8.45} 王平子\myidx{王澄}迈世有隽才\footnote{王平子:王澄,乐广(?—304):字彦辅,南阳淯阳(今河南南阳东南)人。少孤贫,寒素为业,与物无竞。其清谈析理,与王衍并称,卫瓘以为有正始遗风。官至尚书令,八王乱中,以故忧卒注。迈世:超出世俗。隽才:卓越的才智。},少所推服\footnote{推服:推重佩服。}。每闻卫玠\myidx{卫玠}言\footnote{卫玠:即卫筁,官拜太子洗马,故称。惨悴:忧伤憔悴的样子。左右:身边侍从人员注。},辄叹息绝倒\footnote{辄:就,总是。绝倒:极为佩服倾倒。}。{\fzxk\zihao{6}\textcolor{red}{\CJKunderwave{玠别传}曰:“玠少有名理,善通庄、老。琅邪王平子高气不群,迈世独傲。每闻玠之语议,至于理会之间、要妙之际,辄绝倒于坐;前后三闻,为之三倒。时人遂曰:‘卫君谈道,平子三倒。’”}}

{\cangkai\zihao{5}【评】王澄长卫玠十七岁,论辈分则为忘年,而每为之倾心折节,“卫玠谈道,平子绝倒”,成为魏晋名士的一段风流佳话。晋人尚达,不以齿序、地位相矜,在真理面前持平等的态度——“无贵无贱,无长无少”、“道之所存,师之所存也”。这样的美德对于学术、理论的繁荣进步,功莫大焉。然世易时移,此风日衰。唐柳宗元\CJKunderwave{答韦中立论师道书}云:“由魏晋氏以下,人益不事师。今之事不闻有师,有辄哗笑之,以为狂人。”概括了南朝以后的士风。“卫玠谈道,平子绝倒”的美谈,在后世很长一段时间内,恐怕将成为绝响了。}

\lettrine{8.46} 王大将军\myidx{王敦}与元皇\myidx{司马睿}表云\footnote{王大将军:王敦,王敦:字处仲,晋琅邪临沂(今属山东)人,王导堂兄。妻为晋武帝女襄城公主,拜驸马都尉。晋室东迁,与王导一起辅佐元帝,任要职,握重兵,镇守扬州、荆州等重镇。公元322 年起兵谋反,入京都建康。王含:见刘孝标注。光禄勋:官名,九卿之一,领管光禄、大中、中散、谏议等大夫及羽林郎、五官、虎贲、左右等中郎将注。元皇:东晋元帝司马睿,东晋第一位皇帝,在位七年,庙号“中宗”。}:“舒\myidx{王舒}风概简正\footnote{舒:王舒,(266?—333):字处明,王导从弟,时为荆州刺史。风概:风度气概。简正:简约正直。},作雅人\footnote{作雅人:据袁本,“作”上增一“允”字。雅人,高尚之士。},自多于邃\myidx{王邃}\footnote{自:原来,本来。多:胜过。邃:王邃,字处重。王舒弟。},{\fzxk\zihao{6}\textcolor{red}{王舒,已见。\CJKunderwave{王邃别传}曰:“邃字处重,琅邪人,舒弟也。意局刚清,以政事称。累迁中领军,尚书左仆射。”舒、邃,并敦从弟。}} 最是臣少所知拔\footnote{最是:尤其是,特别是。知拔:赏识奖拔。}。中间夷甫\myidx{王衍}、澄\myidx{王澄}见语\footnote{夷甫、澄:王衍,字夷甫,王夷甫:王衍(256—311)字夷甫,见刘孝标注。“以清虚通理称”,为当时清谈名家,“妙悟若神”,“妙善玄言,唯谈\CJKunderwave{老}、\CJKunderwave{庄}为事”。为政多谋略,不以经国为念,而善思自全之计,然终为石勒所害。(见\CJKunderwave{晋书}本传)注。王澄,王衍弟,乐广(?—304):字彦辅,南阳淯阳(今河南南阳东南)人。少孤贫,寒素为业,与物无竞。其清谈析理,与王衍并称,卫瓘以为有正始遗风。官至尚书令,八王乱中,以故忧卒注。见语:对我说。}:‘卿知处明\myidx{王舒}、茂弘\myidx{王导}\footnote{处明:王舒。茂弘:王导。},茂弘已有令名,真副卿清论\footnote{副:符合。清论:高明的议论。指对人的品评。};处明亲疏无知之者。吾常以卿言为意\footnote{以卿言为意:把你的话当回事。指重视你的话。},绝未有得,恐已悔之。’臣慨然曰:‘君以此试。’顷来始乃有称之者\footnote{顷来:不久以来,近来。},言常人正自患知之使过\footnote{言:以为,认为。正自:只,只是。患:忧虑,担心。知之使过:了解的就让说过头。},不知使负实\footnote{负实:违背事实。}。”{\fzxk\zihao{6}\textcolor{red}{“使”一作“便”。}}

{\cangkai\zihao{5}【评】“言常人正自患知之使过,不知使负实”二句,较难解,意谓一般人对于人才的任用,只是担心知遇超过其(人才)的实际,而未考虑不知遇就会辜负了他的才干。王敦对元帝这番话,则反其意而行。可见其人物品评从人的未来发展着眼,宁可高估一些,而不致才非所用,淹没了一个人的真实水平。这就说明王敦并非只知攻城略地的武人,还有识人之鉴及对人才的宽厚态度。凌濛初曰:“古今同患”,实际上指出人性中嫉妒心理比比皆是。一代枭雄对精英人才的注重,正与其勃勃野心相契合。}

\lettrine{8.47} 周侯\myidx{周顗}于荆州败绩还\footnote{周侯:周顗。于荆州败绩还:在荆州大败而回。周顗荆州败绩事,见刘孝标注。},未得用。王丞相\myidx{王导}与人书曰\footnote{王丞相:王导。}:“雅流弘器\footnote{雅流:高雅之流。流,辈。弘器:宏大之器。喻有大才之人。},何可得遗!”{\fzxk\zihao{6}\textcolor{red}{邓粲\CJKunderwave{晋纪}曰:“顗为荆州,始至,而建平民傅密等叛,逆蜀贼,顗狼狈失据,陶侃求(救)之,得免。顗至武昌,投王敦,敦更选侃代顗,顗还建康,未即得用也。”}}

{\cangkai\zihao{5}【评】元帝初镇江左,以周顗为荆州刺史。荆州有江、汉之险,元帝有社稷之托。建平流人迎蜀贼相攻,顗狼狈败绩,甚失朝野之望,故一时未得任用。王导总揆百僚,深知周顗气度宽宏、处惊不乱,是堂堂廊庙之器,而非节镇军国之才。若端委朝廷,足以雍容镇物;分符封疆,正是用其所短。可见人伦识鉴,式瞻仪形,仅能得其外在风度;闲习音旨,方能得其神髓。朝廷诸臣,地位声名相若者,可能时时心存提防、嫉妒。但王导并不落井下石,而是多方为之延誉,怎能不令孤独无助的周顗心存感激,并在日后王导失势、命悬一线时挺身营救呢?}

\lettrine{8.48} 时人欲题目高坐\myidx{高坐}而未能\footnote{题目:品评,品题。高坐:高坐道人,晋高僧帛尸黎密多罗。},桓廷尉\myidx{桓彝}以问周侯\myidx{周顗}\footnote{桓廷尉:桓彝,曾官散骑常侍,故云。周侯:周顗。},周侯曰:“可谓卓朗\footnote{卓朗:高超开朗。}。”桓公\myidx{桓温}曰:“精神渊箸\footnote{桓公:桓温。渊箸(著):深沉彰明。}。”{\fzxk\zihao{6}\textcolor{red}{\CJKunderwave{高坐传}曰:“庾亮、周顗、桓彝,一代名士,一见和尚,披衿致契。曾为和尚作目,久之未得。有云:‘尸利(黎)密可称卓朗。’于是桓始咨嗟,以为标之极。但宣武尝云:‘少见和尚,称其精神渊箸,当年出伦。’其为名士所叹如此。”}}

{\cangkai\zihao{5}【评】王导、庾亮、周顗、谢鲲、桓彝,皆一代名士,见高坐道人,终日累叹,引为侪辈,可见魏晋玄学在东晋时期与佛教的密切关系,以及道人超凡的人格魅力。道人为西域某国王子,让王位于弟,继而悟心天启,遁入沙门,其经历有似佛祖释迦牟尼。道人对权位与荣华的超迈态度,与魏晋名士希慕的“逍遥”、“齐物”的人生追求,有着本质的默契,其让国后又远涉东土的传奇经历,更是名士们心仪,而决不肯付诸实践的高标。其为王公卿相激赏甚至迷恋,正在情理之中。名士们对道人的人生经历与佛教义理既感亲切,又觉新奇,他们以宽广的胸怀、友好的态度,接纳了异质宗教,证明东晋虽偏安一角,就文化的包容性而言,仍是一个极有活力的存在。对高坐道人这样的“外来和尚”进行品目,是一件严肃的事情。梁释慧皎撰\CJKunderwave{高僧传}载,并未表明是周侯所云,桓公品藻亦非同时,而是以回忆方式道出。晋人赏誉讲究瞻形得神,由“卓朗”到“精神渊著”,恰是由形入神的路数,由外在气度到内在精神,处处透露出不凡。}

\lettrine{8.49} 王大将军\myidx{王敦}称其儿\myidx{王应}云\footnote{王大将军:王敦。其儿:指王应。王应,本王含子,王敦无子,养为嗣子,(266?—333):字处明,王导从弟,时为荆州刺史。}:“其神候似欲可\footnote{神侯:精神面貌。欲可:还行,还可以。可:称人心、使人满意均曰“可”。如桓温目王敦之“可人”。}。”{\fzxk\zihao{6}\textcolor{red}{王应也。}}

{\cangkai\zihao{5}【评】王敦无子而以兄含子应为嗣子,敦对养子应满心喜爱,且寄予厚望。从字面上看,王敦品评之言较为低调,“欲可”意为尚可,无足多论,但此语出自舐犊情深的父亲之口,意味大不寻常,恐怕掩饰不住眼角眉梢所传达出的欣喜之情,与谢安之“小儿辈大破贼”有异曲同工之妙。\CJKunderwave{世说}记事记言简约传神,寥寥数字,可引发无限联想。读者如能以中国传统文学观念“兴味”的态度,或以西方接受理论所主张的“填空”、“对话”等阅读方式,进行创造性的还原,自能体会到无穷的艺术魅力,感悟不尽的阅读意趣。}

\lettrine{8.50} 卞令\myidx{卞壸}目叔向\myidx{羊舌肸}\footnote{卞令:卞壸,徐震堮\CJKunderwave{世说新语校笺}以为简文执政时,卞壸已死四十馀年,故断非卞壸。叔向:刘孝标注为羊舌肸,字叔向,春秋时晋大夫,似误,疑卞壸有叔名向。}:“朗朗如百间屋\footnote{朗朗:开朗明亮。}。”{\fzxk\zihao{6}\textcolor{red}{\CJKunderwave{春秋左氏传}曰:“叔向,羊肸也,晋大夫。”}}

{\cangkai\zihao{5}【评】\CJKunderwave{世说}之赏誉、品藻两门,止于魏晋两朝。凡品题人者,多亲见其人。卞壸目其叔向朗朗如百间屋,盖言其气度恢宏,神情开朗,胸襟坦白,就像上百间屋子的宏大建筑。}

\lettrine{8.51} 王敦\myidx{王敦}为大将军\footnote{王敦:王敦:字处仲,晋琅邪临沂(今属山东)人,王导堂兄。妻为晋武帝女襄城公主,拜驸马都尉。晋室东迁,与王导一起辅佐元帝,任要职,握重兵,镇守扬州、荆州等重镇。公元322 年起兵谋反,入京都建康。王含:见刘孝标注。光禄勋:官名,九卿之一,领管光禄、大中、中散、谏议等大夫及羽林郎、五官、虎贲、左右等中郎将注。},镇豫章\footnote{豫章:郡名,辖境相当今江西省,治所在南昌。},卫玠\myidx{卫玠}避乱\footnote{卫玠:即卫筁,官拜太子洗马,故称。惨悴:忧伤憔悴的样子。左右:身边侍从人员注。避乱:指避西晋末年的战乱。},从洛投敦,相见欣然,谈话弥日\footnote{弥日:竟日,整天。}。于时谢鲲\myidx{谢鲲}为长史\footnote{谢鲲:谢豫章:谢鲲,曾作豫章太守。刘孝标注“鲲子别见”,“子”字衍。将:携,谓携之送客。自:已经。参:参与、进入。上流:上等、上品注。},敦谓鲲曰:“不意永嘉之中\footnote{永嘉:西晋怀帝年号(307—313)。},复闻正始之音\footnote{正始之音:正始,三国魏齐王芳年号(240—249)。当时以何晏、王弼为代表的士大夫崇尚玄学清谈,后称当时的言论风尚为“正始之音”。}。阿平\myidx{王澄}若在\footnote{阿平:王澄,字平子,王衍弟。},当复绝倒\footnote{当:将,会。绝倒:因佩服而倾倒。}。”{\fzxk\zihao{6}\textcolor{red}{\CJKunderwave{玠别传}曰:“玠至武昌见王敦,敦与之谈论,弥日信宿。敦顾谓僚属曰:‘昔王辅嗣吐金声于中朝,此子今复玉振于江表,微言之绪,绝而复续。不悟永嘉之中,复闻正始之音。阿平若在,当复绝倒矣。’”}}

{\cangkai\zihao{5}【评】王敦訏谟军机,卫玠避乱江左,竟一见倾心,谈话弥日。他们对理论探讨的热情,超越了战火硝烟与门第稻粱等功利性层面,完全是纯粹的学术争鸣。中华民族虽偶或命悬一线,而其文化终数千载不绝,正是靠了千百代士人的传承与发扬之功。而精神之火的绵延对民族向心力与国家认同又会产生巨大的影响。王敦虽戎马倥偬,仍关心玄学建设,与大名士卫玠谈论弥日,足见其理论功底非等闲之辈可比;同时也验证了一个时代现象,即魏晋士人玄谈的生活化,或换言之,士人生活的玄学化。朱铸禹先生释“金声”、“玉振”曰:“金声言乐之将始,先击玠钟,以宣其声,声宣也。玉振,言乐之将终,其声靡杀。玉声锵然清越而作之,所以美成也。”(\CJKunderwave{世说新语汇校集注})拟之于何晏、卫玠,若合符契。}

\lettrine{8.52} 王平子\myidx{王澄}与人书\footnote{王平子:王澄,乐广(?—304):字彦辅,南阳淯阳(今河南南阳东南)人。少孤贫,寒素为业,与物无竞。其清谈析理,与王衍并称,卫瓘以为有正始遗风。官至尚书令,八王乱中,以故忧卒注。},称其儿\myidx{王微}“风气日上\footnote{其儿:王微,一作王徽。风气:风度气质。},足散人怀\footnote{足散人怀:足以使人开怀。意思是使人散心高兴。}”。{\fzxk\zihao{6}\textcolor{red}{\CJKunderwave{永嘉流人名}曰:“澄弟(第)四子微(徽)。”\CJKunderwave{澄别传}曰:“微(徽)迈上有父风。”}}

{\cangkai\zihao{5}【评】“早相题目”是魏晋人物品评的特点之一。人伦识鉴之水平高下,主要体现在对早慧人才的发现和预见是否准确。如卫瓘对卫玠、羊祜对王衍的品目就很准确。本门王敦对子王应、王澄对子王微(徽)的赏誉则属于心理预期。“望子成龙”一语,正是对王澄内在心态的最好说明。李慈铭则从否定角度论及当时人物品评中的恶劣倾向:“晋、宋、六朝膏粱门第,父誉其子,兄夸其弟,以为声价;其为子弟者,则务鄙薄父兄,以示通率……于是未离乳臭,已得华资;甫识一丁,即为名士;沦胥及溺,凶国害家。平子本是妄人,荆产岂为佳子,所谓风气日上者,淫荡之风,痴顽之气耳。”从六朝门阀制度的腐朽性角度立论,有一定道理,但将所有早慧人才都一竿子打倒,未免株连无辜。}

\lettrine{8.53} 胡毋彦国\myidx{胡毋辅之}吐佳言如屑\footnote{胡毋彦国:胡毋辅之字彦国。嗜酒放达,不拘小节,与王澄、王敦、庾敳为王衍四友,官至湘州刺史。乐广(?—304):字彦辅,南阳淯阳(今河南南阳东南)人。少孤贫,寒素为业,与物无竞。其清谈析理,与王衍并称,卫瓘以为有正始遗风。官至尚书令,八王乱中,以故忧卒注。屑:细末。},后进领袖\footnote{后进:晚辈;后辈。}。{\fzxk\zihao{6}\textcolor{red}{言谈之流,靡靡如解木出屑也。}}

{\cangkai\zihao{5}【评】王澄、胡毋辅之等人为王衍四友,受嵇、阮“越名教而任自然”风气影响,纵酒裸裎,以任放为达,其实只得“自然”之细枝末节,未通大道之本,并无太多进步意义,未免使人兴东施效颦之叹。此举在当时,当属于少数标新立异者的“先锋”、“实验”行为,恐不为入流名士买账;若在今天,也只能成为新闻媒体捕捉的笑料,入八卦报刊一列。于此可以推测,王澄赏其“吐佳言如屑,后进领袖”,恐不免出于名士小集团中人互相提携,以获世誉的狭隘功利目的。}

\lettrine{8.54} 王丞相\myidx{王导}云\footnote{王丞相:王导。}:“刁玄亮\myidx{刁协}之察察\footnote{刁玄亮:刁协,字玄亮,东晋勃海铙安(今河北盐山南)人,晋元帝亲信近臣。协久在内朝,谙练旧事,中兴制度,多为协所建,于朝廷制度多所谋划。性刚悍,好媚上抑下。后为王敦所杀。察察:明察的样子。形容人做事精明。},戴若思\myidx{戴渊}之岩岩\footnote{戴若思:戴渊,字若思,东晋广陵(今江苏淮阴东南)人。岩岩:高峻的样子。比喻态度严峻。},{\fzxk\zihao{6}\textcolor{red}{虞预书曰:“戴俨字若思,广陵人。才义辩济,有风标锋颖。累迁征西将军,为王敦所害。赠左光禄大夫,仪同三司。”}} 卞望之\myidx{卞壸}之峰距\footnote{卞望之:卞壸,字望之,徐震堮\CJKunderwave{世说新语校笺}以为简文执政时,卞壸已死四十馀年,故断非卞壸。峰距:比喻为人严正有锋芒。}。”{\fzxk\zihao{6}\textcolor{red}{\CJKunderwave{卞壸别传}曰:“壸字望之,济阴冤句人。父粹,太常卿。壸少以贵正见称,累迁御史中丞,权门屏迹。转领军、尚书令。苏峻作乱,率众拒战,父子二(三)人,俱死王难。”邓粲\CJKunderwave{晋纪}曰:“初,咸和中,贵游子弟能谈嘲者,慕王平子、谢幼舆等为达。壸厉色于朝曰:‘悖礼伤教,罪莫斯甚,中朝倾覆,实由于此。’欲奏治之,王导、庾亮不从,乃止。其后皆折节为名士。”\CJKunderwave{语林}曰:“孔坦为侍中,密启成帝,不宜拜曹夫人。丞相闻之,曰:‘王茂弘驽痾耳,若卞望之之岩岩,刁玄亮之察察,戴若思之峰距,当敢尔不?’”此言殊有由绪,故聊载之耳。}}

{\cangkai\zihao{5}【评】王导之言,似有未尽之意。\CJKunderwave{晋书}卞壸传载此事甚详,可助理解。东晋初,成帝临幸王导府第,尝拜导妇曹氏。侍中孔坦密表不宜拜。导闻之曰:“王茂弘驽痾耳,若卞望之之严严,刁玄亮之察察,戴若思之峰距,当敢尔邪!”读后令人豁然开朗。卞、刁、戴三人,卞、刁性情刚悍,与物多忤,不畏强御,连王导也敢于弹劾,有法家峻厉之风;戴若思为人亦有不拘常规的侠义之气。察察、岩岩、峰距,同义互文,三人都与王导之优游宽和的作风不同,因刚直太过而拒人千里之外。王导此语乃讥讽孔坦欺软怕硬。}

\lettrine{8.55} 大将军\myidx{王敦}语右军\myidx{王羲之}\footnote{大将军:王敦。右军:王羲之,官右军将军。}:“汝是我佳子弟,{\fzxk\zihao{6}\textcolor{red}{案:\CJKunderwave{王氏谱},羲之是敦从父兄子。}} 当不减阮主簿\myidx{阮裕}\footnote{阮主簿:阮裕,即阮裕,曾以金紫大夫征,故称。\CJKunderwave{世说}作者刘义庆为避宋武帝刘裕名讳,从不称阮裕之名。剡(shàn 善):古县名,在今浙江嵊州。}。”{\fzxk\zihao{6}\textcolor{red}{\CJKunderwave{中兴书}曰:“阮裕少有德行,王敦闻其名,召为主簿。知敦有不臣之心,纵酒昏酣,不综其事。”}}

{\cangkai\zihao{5}【评】刘劭\CJKunderwave{人物志}以为“夫人才不同,成有早晚,有早智而速成者,有晚智而晚成者……夫幼智之人,才智精达,然其在童髦,皆有端绪,故文本辞繁,辩始给口。”世间确有天才颖迈的夙慧之人。王羲之十馀岁时,为周顗所异,先割牛心炙与之,待以殊礼。王敦、王导更以传承家风的使命相期许。事实证明,王羲之没有辜负父辈的厚望,他不仅对现实政治有独到的思考,且创立了卓绝千古的书法艺术。他与谢安为领袖的“兰亭之游”,在玄学发展史上是具有里程碑性质的大事。就此而言,王羲之不仅超越了阮裕等第一代过江名士,且其声名亦远非王氏家族“佳子弟”所能拘囿。}

\lettrine{8.56} 世目周侯\myidx{周顗}“嶷如断山\footnote{周侯:周顗,曾官散骑常侍,故云。嶷:高峻。断山:切断的山崖。比喻周顗性格清高刚正。}”。{\fzxk\zihao{6}\textcolor{red}{\CJKunderwave{晋阳秋}曰:“顗正情嶷然,虽一时侪类,皆无敢媟近。”}}

{\cangkai\zihao{5}【评】\CJKunderwave{晋书}顗传载顗“少有重名,神采秀彻,虽时辈亲狎,莫能媟也”。法国女权主义思想家西蒙娜·德·波伏瓦在\CJKunderwave{第二性}一书中,曾经说过这样一句话:“女人不是天生的,而是被造就的。”我们可以套用过来说:名士不是天生的,而是被造就的。周顗少有重名,代表社会对他的认同和期许,“时辈亲狎,莫能媟也”并不符合小孩子的天性,乃是他本人自觉认同并故意长期强化这种角色意识,进而内化成为外在的习惯。“嶷如断山”之评,见出世人对其“海内盛名”的承认,这正是其长期自觉向社会文化靠拢而获得的最好报偿。不仅是周顗,其实诸多名士,都经历过这样一个自觉或不自觉的社会化过程。甚至可以扩大一步说,每一种社会身份都是长期社会化的产物。}

\lettrine{8.57} 王丞相\myidx{王导}招祖约\myidx{祖约}夜语\footnote{王丞相: 王导。招: 请来。},至晓不眠。明旦有客,公头鬓未理\footnote{头鬓:头发和鬓毛。},亦小倦\footnote{小倦:略感疲倦。},客曰:“公昨如是,似失眠。”公曰:“昨与士少语\footnote{士少:祖约字士少。},遂使人忘疲。”

{\cangkai\zihao{5}【评】祖约朝廷叛臣,王导未能预先体察,竟与之夜语不眠。遂使后代评家对王导之识鉴力提出质疑。王世懋曰:“祖约叛臣何足尔,清谈真不足贵。”凌濛初亦有“丞相每与作逆者倾注。”用今天的话来讲,就是批评王导敌我不分,阵线不明。该怎样看待这些质疑呢?王导作为一位有人格感召力的政治家,非常注重统一战线工作,他以有容乃大的胸襟团结南北士人形成合力、勠力王室,功不可没。对于北来的流民领袖人物,如祖逖、祖约兄弟以及郗鉴辈,更是努力争取和团结,以便在江淮地区,组织一支抗战御侮、捍卫京师的军事力量。因此,与祖约清言以示好,正是王导争取流民帅支持的措施,不可轻视其政治意义。而且,王导是人而不是神,其识人偶有走眼也在所难免,不必苛责,因为人的本性可以伪装,更何况人的思想又是变动不居的,祖约叛乱,是后来之事,而且也有朝廷执政处理不当而为形势所激的因素。以王导之贤,也不可能料事如神。}

\lettrine{8.58} 王大将军\myidx{王敦}与丞相\myidx{王导}书\footnote{王大将军:王敦。丞相:王导。},称杨朗\myidx{杨朗}曰\footnote{杨朗:字世彦,东晋弘农华阴(今属陕西)人,有器识,为王敦、谢安所赏识,历南郡太守,官至雍州刺史。}:“世彦识器理致\footnote{识器理致:识鉴能力、思想情趣。},才隐明断\footnote{才隐:才学深邃。明断:明于判断。}。既为国器\footnote{国器:治国之才。},且是杨侯淮\myidx{杨准}之子\footnote{且是杨侯淮之子:“淮”当为“准”之误,\CJKunderwave{魏志·陈思王传}注引\CJKunderwave{世语}及\CJKunderwave{冀州记}并作“准”。杨准,字始立。杨修孙,杨朗父。},{\fzxk\zihao{6}\textcolor{red}{\CJKunderwave{世语}曰:“淮字始立,弘农华阴人。曾祖彪,祖修,有名前世。父嚣,典军校尉。淮,元康末为冀州刺史。”荀绰\CJKunderwave{冀州记}曰:“淮见王纲不振,遂纵酒,不以官事规意,消摇卒岁而已。成都王知淮不治,犹以其名士,惜而不遣,召为军咨议祭酒。府散停家,关东诸侯欲以淮补三事,以示怀贤尚德之事,未施行而卒,时年二十有七矣。”}} 位望绝为陵迟\footnote{位望:地位名望。陵迟:衰落不振。引申为淹滞。 “位望绝为陵迟”之“绝”,袁本作“殊”。},卿亦足与之处\footnote{足:值得。}。”

{\cangkai\zihao{5}【评】\CJKunderwave{识鉴}门第13则载杨朗苦谏王敦事。杨朗为王敦赏识,有知人、料事识鉴,此则王敦给王导写推荐信,时间当在前。王敦认为杨朗才能与位望不符,屈居下僚,造成人才浪费,故向从弟王导说人情。王敦此举,符合其一贯的爱才之心,也昭示了窃国大盗另有其天真的一面。}

\lettrine{8.59} 何次道\myidx{何充}往丞相\myidx{王导}许\footnote{何次道:何充,字次道,晋康帝时为骠骑将军。丞相:王导。许:处所。},丞相以麈尾指坐\footnote{麈尾:魏晋时一种用具,兼有拂尘和凉扇的功用。清谈家手执麈尾以指授而增饰其容仪。},呼何共坐曰:“来,来,此是君坐\footnote{此是君坐:这是您的座位。意谓何充当宰相之位。}。”{\fzxk\zihao{6}\textcolor{red}{何充,已见。}}

{\cangkai\zihao{5}【评】王导为何充姨父,充少为与王导所赏,对其才能器局有深刻的了解,故虽为晚辈,而导雅重其人。导呼充共坐,看出二人情款非比寻常。且云“此是君坐”,语义双关,意谓何充后当居宰相之位。“来,来”二字,殷切、急迫之情呼之欲出,描写极其传神,可见出\CJKunderwave{世说}言约意丰的语言艺术。}

\lettrine{8.60} 丞相\myidx{王导}治扬州廨舍\footnote{丞相:王导。治:修建。 扬州廨舍:指扬州刺史官署。},按行而言曰\footnote{按行:巡行查看。}:“我正为次道\myidx{何充}治此尔\footnote{正:仅,只。次道:何充字次道。尔:罢了。}!”何少为王公所重,故屡发此叹。{\fzxk\zihao{6}\textcolor{red}{\CJKunderwave{晋阳秋}曰:“充,导妻姉(姊)之子,明穆皇后之妹夫也。思韵淹济,有文义才情,导深器之,由是少有美誉,遂历显位。导有副贰己使继相意,故屡显此指于上下。”}}

{\cangkai\zihao{5}【评】东晋时京城建康属扬州,此州乃富足之区,地位极为重要。扬州刺史一职往往为宰相兼领,王导即以丞相而领扬州刺史。其后庾冰、何充、蔡谟、桓温、谢安诸人皆兼任扬州刺史。故事中王导言“正为次道治此尔”,传达出以何充为接班人的坚定意愿。何充为王导外甥,又为明帝连襟,王导欲以后事相委,是否有任人唯亲之嫌呢?非也。其实非惟王导,庾亮亦曾郑重向成帝推荐何充。何充亦不负重托,居宰相期间,强力有器局,以社稷为己任,成为继王、庾之后优秀的政治家。王导屡屡赏拔,可谓得人。王世懋曰:“殊得佛祖传钵心事。”亦是此意。}

\lettrine{8.61} 王丞相\myidx{王导}拜司徒\footnote{王丞相:王导。拜:拜官,授官。司徒:官名。东汉以太尉、司徒、司空为三公。晋时司徒官位相当丞相。},而叹曰:“刘王乔\myidx{刘畴}若过江\footnote{刘王乔:刘畴,字王乔。},我不独拜公\footnote{独:单独。公:指三公之位。}。”{\fzxk\zihao{6}\textcolor{red}{曹嘉之\CJKunderwave{晋纪}曰:“畴有重名,永嘉中为阎鼎所害。司徒蔡谟每叹曰:‘若浦(使)刘王乔得南渡,司徒公之美选也。’”}}

{\cangkai\zihao{5}【评】东晋初,王导拜司徒而思刘王乔,刘孝标注亦引蔡谟之叹,其为名流推重如此,刘王乔当有政治家的素质和风范。东汉以来官制,以太尉、司徒、司空为三公,名为领袖朝廷,但大多虚名尊美,少有实权。如\CJKunderwave{晋纪总论}六臣注曰:“皆萧然自放,机尔无为,名称标著、上议以正朝廷,则蒙虚谈之名。”只是为了尊崇世族中门望特高的人士而已。三公或八公,须加上录尚书事的称号,才可能接触实际权力。刘王乔乃汉高祖少弟楚元王刘交后裔,是老牌贵族,本人少有美誉,有良好的风度。这双重因素,使其获得了王导、蔡谟之流的赏誉。}

\lettrine{8.62} 王蓝田\myidx{王述}为人晚成\footnote{王蓝田:王述,字怀祖,太原晋阳人。袭爵蓝田县侯。晚成:成就较晚。},时人乃谓之痴。{\fzxk\zihao{6}\textcolor{red}{\CJKunderwave{晋阳秋}曰:“述体道清粹,简贵静正,怡然自足,不交非类。虽群英纷纷,俊乂交驰,述独蔑然,曾不莫(慕)羡。由是名誉久蕴。”}} 王丞相\myidx{王导}以其东海\myidx{王承}子\footnote{以其东海子:王述之父王承,曾任东海太守,故称。},辟为掾\footnote{辟:征召,招聘。掾:属官。}。常集聚,王公每发言,众人竞赞之;述于末坐曰:“主非尧、舜\footnote{主:主人,指王导。},何得事事皆是!”丞相甚相叹赏。{\fzxk\zihao{6}\textcolor{red}{言非圣人,不能无过,意讥赞述之徒。}}

{\cangkai\zihao{5}【评】王述是太原王氏子弟中形象极其鲜明的一个,其个性可用两个字概括:真率。\CJKunderwave{晋书}本传载其:“每受职,不为虚让,其有所辞,必于不受。”这是一种纯净得容不下一丝杂滓的真率,庶几只可以用“不失其赤子之心”一语以写其一斑。这样的性格,大概也只有虚怀若谷的王导才能包容。凌濛初曰:“一语令千古佞谀羞死。”可谓入木三分。为了某种利益或原因,阿谀逢迎上司,似乎已成为中国的官场习俗,王导作为政治领袖,耳中颂声充溢;但是王述的大胆批评,让他在昏昏然的自我陶醉中清醒过来。人非圣贤,不可能一句顶一万句,过而能改,岂非大幸?若人人惟求自保,则主上充耳颂词,又怎能从善如流、改过自新呢?从这个角度看,王述真率几近于憨痴,而憨痴处,正见其可爱,并因此成为故事主角。但若细加品味,王导的形象也很可爱,在大庭广众之下,容忍、接受下属的讽刺批评。我们常常讲,领导要从善如流、闻过则喜,可作为至高的君子品格,在现实生活中,几人见过?王导之雅量,同样令人叹赏。}

\lettrine{8.63} 世目杨朗\myidx{杨朗}“沈审经断”\footnote{目:品评。杨朗:东晋弘农华阴(今属陕西)人,有器识,为王敦、谢安所赏识,历南郡太守,官至雍州刺史。沉审:深沉明察。经断:有决断。},蔡司徒\myidx{蔡谟}云\footnote{蔡司徒:蔡谟,(281—356):字道明,东晋陈留考城(今河南民权东北)人。}:“若使中朝不乱\footnote{中朝:晋南渡后称渡江前的西晋为中朝。},杨氏作公方未已\footnote{公:指三公。魏晋时,以太尉、司徒、司空为三公。方:正。}。”谢公\myidx{谢安}云:“朗是大才\footnote{大才:才能极高的人。}。”{\fzxk\zihao{6}\textcolor{red}{\CJKunderwave{八王故事}曰:“杨淮(准)有六子,曰乔、髦、朗、琳、俊、伸,皆得美名,论者以谓悉有台辅之望。文康庾公每追叹曰:‘中朝不乱,诸杨作公未已也!’”}}

{\cangkai\zihao{5}【评】王敦、庾亮、蔡谟、谢安这些政界巨擘,交相称赞杨朗的才华,并为其生不逢时深深叹惋,可见朗之声誉绝非浪得。晋室南迁,中原板荡,命运多舛的士子,在国家的浩劫中,承受着个体的不幸与苦难,但士人群体并非铁板一块,其内部也发生了剧烈的分化:有识之士把握奋发有为的大好时机,加上老天眷顾,成就了一番功业;有人在硝烟战火中陨灭了理想和激情,因意志消沉而无所作为;还有一种人属于命运不济的类型,纵有理想和激情,在命运的大潮中奋力搏击,但难敌运道,终被无情岁月雨打风吹去,刘王乔、杨朗的时代悲剧就属于最后一类。}

\lettrine{8.64} 刘万安\myidx{刘绥}\footnote{刘万安:刘绥,字万安,晋高平(今山东巨野南)人。},即道真\myidx{刘宝}从子\footnote{道真:刘宝,字道真,少贫贱,“常渔草泽,善歌啸,闻名莫不留连”(见\CJKunderwave{世说·任诞})。为司马骏赏拔,后成为士人领袖而与王衍齐名,一经其品题,身价陡增。故陆机入洛之初,张华以为其所宜拜访者,“刘道真是其一”。可见当时刘宝在士林中的声望。从子:侄儿。伯父叔父为从父,故称侄为从子。},庾公\myidx{庾琮}{\fzxk\zihao{6}\textcolor{red}{琮字子躬。}} 所谓“灼然玉举\footnote{庾公:庾琮。灼然:明彻出众貌。一说,灼然为魏晋九品中正察举科目之名。玉举:美好的人选。玉,喻美好。}”。又云:“千人亦见,百人亦见。”{\fzxk\zihao{6}\textcolor{red}{\CJKunderwave{刘氏谱}曰:“绥字万安,高平人。祖奥,太祝令。父赋(斌),箸作郎。绥历骠骑长史。”}}

{\cangkai\zihao{5}【评】“灼然”为汉末人物品评之科目,晋世于“九品中正”中,称上层贵族二品门第有此目。\CJKunderwave{晋书·温峤传}载“举秀才灼然”。汉之陈寔、晋之邓攸等亦举灼然,魏晋时除皇族外无一品,故灼然二品为最高,可见此科难以跻入。故庾琮道万安“千人亦见,百人亦见”。意谓千百人中可谓特出。}

\lettrine{8.65} 庾公\myidx{庾亮}为护军\footnote{庾公:庾亮,庾亮(289—340)的敬称。他历仕东晋元、明、成三朝,作为外戚,曾执国政,显赫于朝。的卢:传说中的凶马之名,骑之不利主人。注。护军:护军将军。},属桓廷尉\myidx{桓彝}觅一佳吏\footnote{属:嘱托。桓廷尉:桓彝,曾官散骑常侍,故云。},乃经年\footnote{乃:竟。}。桓后遇见徐宁\myidx{徐宁}而知之\footnote{徐宁:字安期,东晋东海郯(今山东郯城北)人。知:知遇,欣赏。},遂致于庾公\footnote{致:转达意旨。},曰:“人所应有,其不必有;人所应无,己不必无\footnote{“人所应有”四句:谓徐宁才识高超,不同世俗。},真海岱清士\footnote{海岱:指东海郡与泰山一带地区,指\CJKunderwave{尚书·禹贡}所述青州、徐州。清士:高洁之士。}。”{\fzxk\zihao{6}\textcolor{red}{\CJKunderwave{徐江州本事}曰:“徐宁字安期,东海剡(郯)人。通朗有德素,少知名。初为舆县令。谯国桓彝有人伦鉴识,尝去职无事,至广陵寻亲旧,遇风,停浦中累日,在船忧邑,上岸消摇,见一空宇,有似廨舍。彝访之,云:‘舆县廨也,令姓徐名宁。’彝既独行,思逢悟赏,聊造之。宁清惠博涉,相遇怡然。遂停宿,因留数夕,与宁结交而别。至都,谓庾亮曰:‘吾为卿得一佳吏部郎。’亮问所在,彝即叙之。累迁吏部郎、左将军、江州刺史。”}}

{\cangkai\zihao{5}【评】此则义解,前贤有二说。李慈铭根据本门第84则“王长史道江道群“人可应有,乃不必有;人所应无,己必无”推断,本则“己不必无”中之“不”字为衍文。若“人所应无,己不必无”,则庸下人矣,安得谓之清士。余嘉锡则以为,徐宁与江灌(道群)之为人不必相同,则品目之言,亦当有异。“人所应无”者,谓衡之礼法不当有者也。晋之名士因不为礼法所拘,礼所应无者而竟有之者多矣。如王澄、谢鲲之徒所为皆是也。徐宁见用于庾亮,疑亦不羁之流,故称“己不必无”。余谓二说似均有理,而李说更优。“不必有”、“必无”,脱略两端,体中清通,更切近玄家体无之旨,故其下紧接“海岱清士”。若持“不必无”乃为崇有,已落第二义矣。}

\lettrine{8.66} 桓茂伦\myidx{桓彝}云\footnote{桓茂伦:桓彝,曾官散骑常侍,故云。}:“褚季野\myidx{褚裒}皮里阳秋\footnote{褚季野:褚裒,褚公:对褚裒的敬称。褚裒(póu 抔)(303—349),晋康帝皇后之父,朝廷议以“不臣之礼”,力辞执政,而赴外镇。官征北大将军。曾率军三万北伐,败后上疏自贬,忧慨发愤而卒。见\CJKunderwave{晋书·外戚传}。皮里阳秋:谓口头不加评论,内心却有所褒贬。皮里,指腹中。阳秋,即春秋,晋人避简文宣郑太后阿春讳,以“阳”代“春”。孔子作\CJKunderwave{春秋},暗含褒贬之义。}。”谓其裁中也\footnote{裁中:指心中有裁断、分析。}。{\fzxk\zihao{6}\textcolor{red}{\CJKunderwave{晋阳秋}曰:“裒简穆有器识,故为彝所目也。”}}

{\cangkai\zihao{5}【评】谢安亦有此评:“褚季野虽不言,而四时之气亦备。”\CJKunderwave{易}曰:“吉人之辞寡。”褚季野之不臧否人物,并非出于明哲保身的庸人心理,而是有从容、沉着的人生智慧作内在的指南。这样的立身处世,不会因立意太狭,而冷落一批可以团结的同道,更不会口中雌黄挑起不必要的内耗。汉末党人持论高远,标置太厉,后世有党同伐异之讥;魏晋名士自矜门第,严分士庶,遂入顾影自怜窠臼。褚季野是康献皇后父,位为国丈,既谦冲自守,又敢于担当。去世时“远近嗟悼,吏士哀慕之”,有良好的社会声誉。褚季野等一大批士人重建设、戒虚浮的人生态度,对于魏晋士风是有益的补充。}

\lettrine{8.67} 何次道\myidx{何充}尝送东人\footnote{何次道:何充,字次道,晋康帝时为骠骑将军。东人:指东边来的人。东指会稽、吴郡一带。东晋偏安江左,侨姓高门多在会稽一带广治田宅产业,常在此流连享乐。},瞻望,见贾宁\myidx{贾宁}在后轮中\footnote{贾宁:字建宁,东晋长乐人。后轮:指随从在后车辆。},曰:“此人不死,终为诸侯上客\footnote{诸侯:此指地方大吏。上客:尊贵的客人。}。”{\fzxk\zihao{6}\textcolor{red}{\CJKunderwave{晋阳秋}曰:“宁字建宁,长乐人,贾氏孽子也。初,自结于王应、诸葛瑶。应败,浮游吴、会,吴人咸侮辱之。闻京师乱,驰出,投苏峻,峻甚昵之,以为谋主。及峻闻义军起,自姑孰屯于石头,是宁之计。峻败,先降,仕至新安太守。”}}

{\cangkai\zihao{5}【评】据刘孝标注引\CJKunderwave{晋阳秋}可知,贾宁乃一苍黄反覆、毫无士操的小人。初参与王应、诸葛瑶叛逆,事败后,浪迹江浙间,为人不齿。晋成帝咸和二年(327),历阳内史苏峻起兵,他投奔苏峻,为其出谋划策,深得信任。后见苏峻败死,归降朝廷,官至新安太守。王世懋评曰:“贼何足道,尝是缘丞相保存意耳。”此或许是何充意味深长的反语。观古今历史可知,“高明”的当政者有非常微妙的心理,即并非全然排斥小人,而是希望小人与贤臣并存,自己权衡轻重短长,以求政局的大体平衡。贤臣正直可能无趣,因冷若冰霜而拒人千里;小人奸邪却不乏取乐之术,亦或许具备一定的实际能力,因善于揣度上意,而经常代言主上之心声。关键在于驾驭得法。另外,对主上而言,若少了弄臣的开心取乐与言听计从,充斥着诤臣板正的脸孔和道德训诫,将是多么憋闷无聊!因此,古来的当政者思考的不是如何远小人,而是如何利用小人!}

\lettrine{8.68} 杜弘治\myidx{杜乂}墓崩\footnote{杜弘治:杜乂,字弘治。杜预孙。墓崩:指祖坟崩塌。},哀容不称\footnote{称:适合。}。庾公\myidx{庾亮}顾谓诸客曰\footnote{庾公:庾亮。顾:回头看。}:“弘治至羸\footnote{羸:瘦弱。},不可以致哀\footnote{致哀:过分哀痛。}。”{\fzxk\zihao{6}\textcolor{red}{\CJKunderwave{晋阳秋}曰:“杜乂字弘治,京兆人。祖预,父锡,有誉前朝。乂少有令名,仕丹阳丞,早卒。成帝纳乂女为后。”}} 又曰:“弘治哭不可哀。”

{\cangkai\zihao{5}【评】司马氏以孝治天下,士人居丧尽礼以致鸡骨支床、杖而后起者所在多是。又有所谓生孝、死孝之别,走上反人性的误区。是居丧孝子的情不能已?还是统治者利用儒家名教加以鼓励诱导?抑或是二者兼而有之?杜乂因墓崩而哀容不称,不免引起爱才又重容止的庾亮的担忧。并有“弘治至羸”之叹。史载乂“肤如凝脂,眼如点漆”,桓彝将其与卫玠并称“卫玠神清,杜乂形清”。合而观之,可见杜乂体质弱不禁风,属柔弱之美,与被“看杀”的卫玠为同一类型。}

\lettrine{8.69} 世称庾文康\myidx{庾亮}为丰年玉\footnote{庾文康:庾亮,谥文康。丰年玉:庆祝丰收之年的玉器。},稚恭\myidx{庾翼}为荒年谷\footnote{稚恭:庾翼,字稚恭,庾亮弟。荒年谷:饥荒之年的谷。}。庾家论云:“是文康称恭为荒年谷\footnote{恭:稚恭之省,即庾翼。},庾长仁\myidx{庾统}为丰年玉\footnote{庾长仁:庾统,字长仁,小字赤玉。庾亮从子。}。”{\fzxk\zihao{6}\textcolor{red}{谓亮有廊庙之器,翼有匡世之才,各有用也。}}

{\cangkai\zihao{5}【评】刘辰翁评曰:“好语有味。”其“味”何在?味在因“丰年玉”、“荒年谷”之比喻意象生动、耐人咀嚼。丰年玉指庆祝丰收之年的玉器,喻太平之世的廊庙之器;荒年谷指饥荒之年的粮食,喻时事艰难中的匡济人才。“丰年器”踵色增华,有锦上添花之效;“荒年谷”,切于实用,有雪中送炭之功。二者难分伯仲。个人质性有高下,学习有深浅,其才不必兼擅,亦不必求全。为政者如能善用一偏,则人尽其才,众星拱之,必能实现天下归心的局面。}

\lettrine{8.70} 世目杜弘治\myidx{杜乂}标鲜\footnote{杜弘治:杜乂,字弘治。标鲜:风采华美,仪表光鲜。},季野\myidx{褚裒}穆少\footnote{季野:褚裒,褚公:对褚裒的敬称。褚裒(póu 抔)(303—349),晋康帝皇后之父,朝廷议以“不臣之礼”,力辞执政,而赴外镇。官征北大将军。曾率军三万北伐,败后上疏自贬,忧慨发愤而卒。见\CJKunderwave{晋书·外戚传}。穆少:宁静淡泊,肃穆少言语。}。{\fzxk\zihao{6}\textcolor{red}{\CJKunderwave{江左名士传}曰:“乂清标令上也。”}}

{\cangkai\zihao{5}【评】杜乂、褚裒俱有盛名于江左,又皆是外戚,故二人并称。“标鲜”言乂风度俊美出众,与桓彝称赏之“形清”义正合;“穆少”言裒为人宁静淡泊,颇符彝“皮里阳秋”之评。}

\lettrine{8.71} 有人目杜弘治\myidx{杜乂}\footnote{杜弘治:杜乂。}:“标鲜清令\footnote{标鲜清令:风度俊美,纯洁佳妙。},盛德之风\footnote{盛德:大德。},可乐咏也\footnote{乐咏:和着音乐歌颂。}。”{\fzxk\zihao{6}\textcolor{red}{\CJKunderwave{语林}曰:“有人目杜弘治标鲜甚清令,初若熙怡,容无韵非;盛德之风,可乐咏也。”}}

{\cangkai\zihao{5}【评】杜乂身上兼备儒家人格理想的道德美和魏晋人物赏誉之风度美、形质美。融会内外之美的士人,在中国历史上如凤毛麟角,昙花一现,难得其全。有晋一朝,潘岳、王衍虽美轮美奂,而大节有亏;嵇康、卫玠、杜乂等造化所钟的宁馨儿,因难得一见,故为有爱美本性的世人激赏。魏晋品藻所表征的审美理想标准,与今天的崇尚形体美之风看似暗合,但更注重文化底蕴,和精神内涵外化的风度、气质之美,与今日单纯追求“酷”、“骨感”等标新立异、以求炫目之效的审美怪圈大相径庭。杜乂属于余光中先生所概括的高雅而有趣的一类,“可乐咏也”,即为精神风度之美对于世人的熏染。}

\lettrine{8.72} 庾公\myidx{庾亮}云\footnote{庾公:庾亮。}:“逸少\myidx{王羲之}国举\footnote{逸少:王羲之,字逸少。国举:一国所举。意为全国推崇的人。}。”故庾倪\myidx{庾倩}为碑文云\footnote{庾倪:庾倩,字少彦,小字倪,庾冰子。}:“拔萃国举\footnote{拔萃国举:才能出众,国人所仰。}。”{\fzxk\zihao{6}\textcolor{red}{倪,庾倩小字也。徐广\CJKunderwave{晋纪}曰:“倩字少彦,司空冰子,皇后兄也。有才具,仕至太宰长史。桓温以其宗强,使下邳(新蔡)王晃诬与谋反而诛之。”}}

{\cangkai\zihao{5}【评】“国举”者,国士之目也。王羲之小时候,伯父王敦便称赞他是王氏“佳子弟”,后又成为郗鉴的“坦腹东床”。庾亮临死前曾向朝廷举荐其下属王羲之,使其升任为江州刺史,有选定接班人的意味。虽然王羲之有着琅邪王氏这样高华的门第出身,但他更膺情于浙东会稽的佳山秀水。四十五岁以后,他一直居于此地,与谢安、孙绰、许询诸人优游往还,以至终老。他虽有负于父辈的期待,却在诗歌、书法与清谈等领域蔚然而成一代名士领袖,实现了王氏历史上的超越。历史上少了一位高级官吏无关大体,而多了一位卓绝千古的书法家和玄学领袖,岂非浙东山水之幸,中华文化之幸?}

\lettrine{8.73} 庾稚恭\myidx{庾翼}与桓温\myidx{桓温}书\footnote{庾稚恭:庾翼。桓温:桓公北征:桓温曾有三次北征,刘盼遂\CJKunderwave{世说新语校笺}考订,此次当为太和四年(369)之征。时桓温已58岁。},称:“刘道生\myidx{刘恢}日夕在事\footnote{刘道生:刘恢,字道生,东晋沛国(治所在今安徽濉溪西北)人。日夕在事:终日居官任事。},大小殊快\footnote{大小:指职务、地位高者与低者。快:畅快。},义怀通乐既佳\footnote{义怀:道义之怀。通乐:通达乐观。},且足作友,正实良器\footnote{正:确实。 良器:喻出众之才具。},推此与君同济艰不者也\footnote{推:推荐。艰不:艰难困苦。不,读否,闭塞不通,命运不好。}。”{\fzxk\zihao{6}\textcolor{red}{\CJKunderwave{宋明帝文章志}曰:“刘恢字道生,沛国人。识局明济,有文武才。王濛每称其思理淹通,蕃屏之高选。为车骑司马,年三十六卒,赠前将军。”}}

{\cangkai\zihao{5}【评】这是一封情辞恳切、举贤为公的推荐信,推荐的对象刘恢工作勤勉、办事高效、开朗通达,有良好的人际关系。这样德才兼备的人才,在魏晋崇尚玄虚无为的官场中,当属罕见,在今天亦殊为难得,故庾翼有“同济艰不”之厚望。余嘉锡先生考证后认为,刘恢即刘惔刘真长。但惔“居官无官官之事,处事无事事之心”,又高自标置、白眼向人,轻视兵家桓温,与庾翼描述之“日夕在事”、“且足作友”诸语不符,故知绝非一人。}

\lettrine{8.74} 王蓝田\myidx{王述}拜扬州\footnote{王蓝田:王述。拜扬州:受任扬州刺史。},主簿请讳\footnote{主簿:负责文书、印鉴的属官。请讳:请教家讳。晋人重家讳,长官就任,僚属必先请讳,以防无意中触犯。},教云\footnote{教:上对下的告谕。}:“亡祖\myidx{王湛}、先君\myidx{王承}\footnote{亡祖:指王湛。先君:对人称已故的父亲。此指王承。},名播海内,远近所知,内讳不出于外\footnote{内讳:指应该避讳的已故女性尊长的本名。}。{\fzxk\zihao{6}\textcolor{red}{\CJKunderwave{礼记}曰:“妇人之讳不出门。”}} 馀无所讳。”

{\cangkai\zihao{5}【评】“避讳”是我国古代特有的文化现象,其俗起于周,历史垂二千馀年。而避讳学则成为史学一辅助学科。司马氏高倡以孝治国,故晋人重家讳。桓玄闻“温酒”而流涕呜咽,陆机怒斥卢志,以其直呼父祖之名。新官上任,僚属请讳,以防他时无意中触犯。故事中王述回答主簿“亡祖、先君,名播海内,远近所知”,与陆机“我父、祖名播海内,宁有不知?”语出一辙。言辞间流露出来的,无非是浓厚的家族自豪感和门第优越意识。陆机有\CJKunderwave{祖德}、潘岳有\CJKunderwave{家风}、陶潜有\CJKunderwave{命子}等诗、赋,堪为此证。李慈铭云:“此条是六朝人矜其门第之常语耳,所谓专以冢中枯骨骄人者也,临川列之\CJKunderwave{赏誉},谬矣。”言合常理。赏誉当来自他人,不应自卖自夸,但对真率的王述而论,则为胶柱鼓瑟之言。“馀无所讳”云云,述把避讳之尊降低到最低的程度,不能不说是开明之举,与陆游\CJKunderwave{老学庵笔记}中那个“不许百姓点灯”的郡守田登,可谓有天壤之别。王述之举实是对别人的一种尊重。}

\lettrine{8.75} 萧中郎\myidx{萧轮}\footnote{萧中郎:萧轮,字祖周,东晋青州乐安(今山东博兴一带)人。},孙丞(承)公\myidx{孙统}妇父\footnote{孙承公:孙统,字承公,东晋太原中都(今山西平遥西)人。孙楚孙,孙绰兄。 妇父:妻子的父亲,即岳父。};刘尹\myidx{刘惔}在抚军坐\footnote{刘尹:刘惔,字真长,曾任丹阳尹,故称。谢安妻兄,尚明帝女庐陵公主。会稽王司马昱为相,与王濛并为其座上清谈之客。性简贵自重,与王羲之友善。卒年三十六。抚军:东晋简文帝司马昱,时为抚军将军,晋简文:指晋简文帝司马昱(320—372),穆帝年幼即位,昱任抚军大将军总理政务。后来大将军桓温专擅朝政,先废海西公,后立司马昱为帝,第二年崩。},时拟为太常\footnote{太常:官名。九卿之一,掌宗庙礼仪。}。刘尹云:“萧祖周不知便可作三公不\footnote{三公:魏晋以太尉、司徒、司空为三公。}?自此以还\footnote{以还:以下。},无所不堪\footnote{堪:胜任。}。”{\fzxk\zihao{6}\textcolor{red}{\CJKunderwave{晋百官名}曰:“萧轮字祖周,乐安人。”刘谦之\CJKunderwave{晋纪}曰:“轮有才学,善\CJKunderwave{三礼},历常侍、国子博士。”}}

{\cangkai\zihao{5}【评】萧祖周精\CJKunderwave{三礼},简文拟以祖周为太常。太常为九卿之一,掌宗庙礼仪。东晋时期,九卿职权被尚书省侵夺,空有其名,失去存在价值。三公为位望极高的朝廷重臣,刘惔推荐祖周做三公,属于破格提拔,不合常例。从\CJKunderwave{世说}诸例可见,晋朝管理选拔,权在望族,重名人推荐,选拔制度形同虚设。虽不乏发自公心、慧眼识英的伯乐,王、庾、桓、谢诸公均有识珠玉于瓦砾、拔英雄于下僚的佳话,但毕竟例属偶然,因缺少有效选拔机制,不免沙金俱下,鱼龙混杂,造就了一批无所作为的尸位素餐者。萧祖周为人如何,史罕记载,不敢妄评。但透过刘惔荐贤之美誉背后,却让我们看到了六朝人事制度的危机。而由推荐察举走向科举选拔,虽为历史之必然,但却步履维艰,康庄大道望之而不及。悲乎!}

\lettrine{8.76} 谢太傅\myidx{谢安}未冠\footnote{谢太傅:谢安。未冠:古代男子二十岁成年行冠礼,未冠即未成年。},始出西\footnote{出西:往西边去。谢安少时寓居会稽,自会稽入都城建康,故称往西。},诣王长史\myidx{王濛}\footnote{王长史:王濛。},清言良久\footnote{清言:清谈、谈论玄学。}。去后,苟子\myidx{王修}问曰\footnote{苟子:王修,王濛之子,小字苟子。}:{\fzxk\zihao{6}\textcolor{red}{王濛、子修并已见。}} “向客何如尊\footnote{向:刚才。何如:比……怎么样。 尊:称父亲。}?”长史曰:“向客亹亹\footnote{亹亹:侃侃而谈、言语不绝的样子。},为来逼人\footnote{为:用在动词前,无实义。逼人:气势凌驾别人。}。”

{\cangkai\zihao{5}【评】王濛、王修父子一问一答,真实可信,宛如目前。少年谢安卓荦不凡的形象,就由风流名士王濛口中赋予光鲜活力而丰满起来。孔子曰:“后生可畏”,唐李白亦由此发挥出“宣父犹能畏后生,丈夫未可轻年少”(\CJKunderwave{上李邕})之诗句。少年谢安,黄吻未退,其人生角色处于易塑期,社会阅历尚浅,甫一出西,就使前辈王濛有“为来逼人”之感,可见其不同于醉生梦死的贵游子弟,亦非仅恃门第欺人,而是对人生目标早有了高远的设计。故或出或处,或行或藏,均能游刃有馀,毫不局促。惟其如此,才能为日后成就功业打下坚实基础。}

\lettrine{8.77} 王右军\myidx{王羲之}语刘尹\myidx{刘惔}\footnote{王右军:王羲之。刘尹:刘惔,字真长,曾任丹阳尹,故称。谢安妻兄,尚明帝女庐陵公主。会稽王司马昱为相,与王濛并为其座上清谈之客。性简贵自重,与王羲之友善。卒年三十六。}:“故当共推安石\myidx{谢安}\footnote{故当:当然、肯定。安石:谢安,字安石。}。”刘尹曰:“若安石东山志立\footnote{东山志立:确立东山隐居的志向。谢安曾在会稽上虞县隐居多年。后以东山志喻隐居的志向。},当与天下共推之。”{\fzxk\zihao{6}\textcolor{red}{\CJKunderwave{续晋阳秋}曰:“初,安家于会稽上虞县,优游山林,六七年间,征召不至。虽弹奏相属,继以禁锢,而晏然不屑也。”}}

{\cangkai\zihao{5}【评】魏晋士大夫以隐为高,故谢安出山而获讥于世。此中可窥见当时世风。谢安四十岁以前,处会稽东山,与王羲之、刘惔、孙绰、许询诸人渔弋山水,言咏属文,在士人中有崇高威望。王羲之亦服其雅量,有推举其做领袖之意。刘惔以为谢安如确立隐居东山之志,就与天下共同推举他。言外之意,料谢安将出仕,则不足为高。颇有“非我族类,其心必异”的狭隘名士本位意识,实际上是对名士精神的片面理解,与其一贯的以名士头衔自矜的处世态度同出一源。持此类见解的士人不在少数,本书\CJKunderwave{排调}门载谢安出仕为桓公司马后,时人就以“远志”、“小草”之喻,以见出处异称,声价不同。魏晋士人隐于官、悠游两间者多矣,何以谢安独受此累?俗语道“人怕出名猪怕壮”,公众人物一言一行有其影响力,名高谤至在所难免。}

\lettrine{8.78} 谢公\myidx{谢安}称蓝田\myidx{王述}\footnote{谢公:谢安。称:称誉。蓝田:王述,袭封蓝田县侯。},掇皮皆真\footnote{掇皮皆真:极言王述真率爽直,不虚伪,无矫情的特点。掇皮,去掉皮。}。{\fzxk\zihao{6}\textcolor{red}{徐广\CJKunderwave{晋纪}曰:“述贞审,真意不显。”}}

{\cangkai\zihao{5}【评】“掇皮皆真”者,用晋世口语,仅四字而妙趣全出,言性情真率、表里如一。魏晋士人祖述自然,故崇尚性情之“真”。实际上,士人对真的理解不尽相同,有人因过分注重感官本能之真而堕入欲望的泥潭,有人追求性情风度之真,但因缺少理性的驾驭走进了顾影自怜的歧途。要之,皆背离了自然之真的本义。须知,离开了善与美的辅翼而片面求真,即是放纵人欲本能,人兽何异?有何风度可言?王述“座斥王导”、“不求虚让”等风流轶事,正是针对人性之伪而言,将真的本质演绎得淋漓尽致,其至性自然的背后,积淀着士大夫深沉的理性思考,故可敬可爱。}

\lettrine{8.79} 桓温\myidx{桓温}行经王敦\myidx{王敦}墓边过\footnote{桓温:桓公北征:桓温曾有三次北征,刘盼遂\CJKunderwave{世说新语校笺}考订,此次当为太和四年(369)之征。时桓温已58岁。注。王敦:王敦:字处仲,晋琅邪临沂(今属山东)人,王导堂兄。妻为晋武帝女襄城公主,拜驸马都尉。晋室东迁,与王导一起辅佐元帝,任要职,握重兵,镇守扬州、荆州等重镇。公元322 年起兵谋反,入京都建康。王含:见刘孝标注。光禄勋:官名,九卿之一,领管光禄、大中、中散、谏议等大夫及羽林郎、五官、虎贲、左右等中郎将注。},望之云:“可儿\footnote{可儿:犹可人。即可爱的人,称人心意的人。}!可儿!”{\fzxk\zihao{6}\textcolor{red}{孙绰\CJKunderwave{与庾亮笺}:“王敦可人之目,数十年间也。”}}

{\cangkai\zihao{5}【评】晋人“可儿”即“可人”,意为可爱之人。王敦、桓温二人同出豪门,均为晋室女婿,又都威权震主,有非常之志。桓温赞王敦为可人,既出于对前辈的敬仰礼赞之情,更是惺惺相惜的自然反映,与王敦酒后以如意敲击唾壶而歌魏武之“老骥伏枥,志在千里”,乃出于同一鹄的。这些历史上屈指可数的一代枭雄,千古以来默默承受着“包藏逆谋”的骂名(李慈铭语),偶有一二知己同好在坟前献一瓣馨香,以慰孤寂的灵魂。“可儿”之叹,内蕴知音难觅之慨。“王侯将相宁有种乎?”刘邦、李渊、赵匡胤、朱元璋,与王莽、曹操、王敦、桓温,前者成了开国明君,后者却要背负千古骂名!历史命运的公与不公,贤达又能奈何?明代王世懋评曰:“英雄相识,故不以成败论”,见解倒是高明。}

\lettrine{8.80} 殷中军\myidx{殷浩}道王右军\myidx{王羲之}云\footnote{殷中军:殷浩,(?—356):见刘孝标注。浩善谈玄,负盛名,简文执政时惧桓温势盛,引浩为建武将军、扬州刺史,以对抗桓温。后因北征许洛败绩,为桓温所弹,废为庶人。注。道:称道。王右军:王羲之。}:“逸少清贵人\footnote{逸少:王羲之,字逸少。清贵:清高尊贵。},吾于之甚至\footnote{于之:待他。至:诚恳。},一时无所后\footnote{一时:当时。 无所后:意为事事把他摆在前面,从无慢待失礼。}。”{\fzxk\zihao{6}\textcolor{red}{\CJKunderwave{文章志}曰:“羲之高爽有风气,不类常流也。”}}

{\cangkai\zihao{5}【评】“清贵”,清高尊贵之意,\CJKunderwave{晋书·王羲之传}载:“征西将军庾亮请为参军,累迁长史。亮临薨,上疏称羲之清贵有鉴裁。”与殷浩之言可互证。王羲之出身簪缨世家的琅邪王氏,又得魏晋玄学尚清通自然的真谛,蔚成一代清谈与清游领袖,故人有“清贵”之目。王羲之由各种因素叠加而形成的独特身份,庶几只有陈郡谢安堪可相与匹偶。}

\lettrine{8.81} 王仲祖\myidx{王濛}称殷渊源\myidx{殷浩}\footnote{王仲祖:王濛。殷渊源:殷浩,(?—356):见刘孝标注。浩善谈玄,负盛名,简文执政时惧桓温势盛,引浩为建武将军、扬州刺史,以对抗桓温。后因北征许洛败绩,为桓温所弹,废为庶人。}:“非以长胜人,处长亦胜人\footnote{处长:对待自己的长处。指殷浩不傲物凌人。处,对待。}。”{\fzxk\zihao{6}\textcolor{red}{\CJKunderwave{晋阳秋}曰:“浩善以通和接物也。”}}

{\cangkai\zihao{5}【评】现代著名报人、学者曹聚仁先生在其回忆录\CJKunderwave{文坛五十年}中有一段深刻的议论:“人这种有血有肉的动物,总是有缺点的;一成为文人,便不足观,也可以说,他们的光明面太闪眼了,他们的黑暗面更是阴森;所以诗人住在历史上,几乎等于神仙,要是住在我们的楼上,便是一个疯子。”魏晋士人并不以道德君子命世,故其言谈举止表现出更多的个性“原生态”,恰如曹聚仁所说的“诗人”。天公造物,各有长短。而人性弱点之一端,就在于“以己之长,度人所短”。故“文人相轻,自古而然”,“神仙”也可能变成“疯子”。王濛评价殷浩善于看待自己的长处,谓其能通和接物,谦逊礼让,不恃己长。这种态度与虚伪名教迥异其趣,在崇尚任真纵放的魏晋玄风中也别树一帜,令人想其风采。}

\lettrine{8.82} 王司州\myidx{王胡之}与殷中军\myidx{殷浩}语\footnote{王司州:王胡之。殷中军: 殷浩。},叹云:“己之府奥\footnote{府奥:胸中所有。},蚤已倾写而见\footnote{蚤:同“早”。倾写:倾泻。写,同“泻”。};殷陈势浩汗\footnote{陈势:即“阵势”。指论战的阵容情势。浩汗:同“浩瀚”。},众源未可得测\footnote{众源:许多来源。}。”{\fzxk\zihao{6}\textcolor{red}{徐广\CJKunderwave{晋纪}曰:“浩清言妙辩玄致,当时名流皆为其美誉。”}}

{\cangkai\zihao{5}【评】王胡之评价自己与殷浩的辩论,有二义。第一,“己之府奥,早已倾写而见”,意谓如山间百转千回蕴蓄势能的溪流,化作不择地而出的悬河泻水,令人有一览无馀的瞬间震撼;“殷陈势浩汗,众源未可得测”,如地上滚滚流淌的大江大河,一路上广聚千支万杈的溪流,故能浩瀚汪洋、不辨牛马,盖与“春江潮水连海平”的境界相埒。两个对比意象,可看出王胡之性情直率近于急迫,殷浩从容不迫近乎雍容。第二,两对比意象还给人这样的暗示:王胡之言谈因受学识限制而有捉襟见肘的局促感,如同形形色色的专卖店,精则精矣,商品未免单一;而殷浩因涉猎广博而气度涵容,如特大型的超市,琳琅满目,令人难测其深广。扬雄云:“言,心声也;书,心画也。”通过言谈方式、内容,可以看出其人之气质性情,正所谓“言如其人”。王胡之评殷浩句中,用“浩”、“源”二字,暗合殷浩字渊源之事,语涉双关,用意贴切,运思巧妙。}

\lettrine{8.83} 王长史\myidx{王濛}谓林公\myidx{支遁}\footnote{王长史:王濛。林公:支道林,为东晋名僧,善玄理,是当时佛学“般若学”的代表人物,多才艺,长于草隶。与王洽、刘惔、殷浩、许询、郗超、王羲之、谢安等名流游好。常:同“尝”,曾经。}:“真长可谓金玉满堂\footnote{真长:刘惔,字真长,曾任丹阳尹,故称。谢安妻兄,尚明帝女庐陵公主。会稽王司马昱为相,与王濛并为其座上清谈之客。性简贵自重,与王羲之友善。卒年三十六。金玉满堂:语出\CJKunderwave{老子·九章}:“金玉满堂,莫之能守”。此比喻刘惔才学富实。}。”林公曰:“金玉满堂,复何为简选\footnote{简选:挑选。}?”王曰:“非为简选,直致言处自寡耳\footnote{直:通“特”,只。致言:发出言辞。 自寡:自然少了。}。”{\fzxk\zihao{6}\textcolor{red}{谓吉人之辞寡,非择言而出也。}}

{\cangkai\zihao{5}【评】“金玉满堂”这里用来描述人的先天才、气与后天学、习综合而成的丰富内心世界,属于“内语言”;发为言辞(或为书面语,或为口头语),则为“外语言”。由内语言到外语言,则经历了意、象、言的艰难思维过程。故\CJKunderwave{文心雕龙·神思}篇云:“神用象通,情变所孕。物以貌求,心以理应。”支道林问王濛:“既然金玉满堂即可随意发挥,何以刘之言语矜慎,似有所选择检点而出?”林公所问,看似简单,实则已经涉及魏晋玄谈论题的“言意之辨”。王之回答,避其锋芒,将问题引到“吉人之辞寡”另一风马牛不相及的路径上去。今天的思维学、心理学,已将内语言和外语言涉及的心理机制讲得很清楚了。清代画家郑板桥谈“胸中之竹”与“画上之竹”之区别的著名论述,则更从艺术加工的角度给人以深刻启悟。王世懋评曰:“观此知林公未简于辞。”当是望文生义之言。}

\lettrine{8.84} 王长史\myidx{王濛}道江道群\myidx{江灌}\footnote{王长史:王濛。道:称道。江道群:江灌(?—375),字道群,东晋陈留(今河南开封东北)人。}:“人可应有,乃不必有,人可应无,己必无\footnote{“人可应有”四句:谓徐宁才识高超,不同世俗。}。”{\fzxk\zihao{6}\textcolor{red}{\CJKunderwave{中兴书}曰:“江权(灌)字道群,陈留人,仆射虨从弟也。有才器,与从兄逌名相亚。仕尚书中护军。”}}

{\cangkai\zihao{5}【评】司马昱为抚军,引江道群为从事中郎。后迁御史中丞,转吴兴太守。他为人方正。“人可应有”者,如门第观念、功名意识等世俗观念,江灌不一定有;“人可应无”者,如阿谀权贵、骄矜之心这些人性的惯常弱点,江灌一定没有。史载,江灌性方正,视权贵蔑如也,为大司马桓温所恶,见其人格之一斑。王濛评价江灌应有、应无之言,乃一事两说,从正反两个角度叹赏其方正的品格。}

\lettrine{8.85} 会稽孔沉(沈)\myidx{孔沉}、魏顗\myidx{魏顗}、虞球\myidx{虞球}、虞存\myidx{虞存}、谢奉\myidx{谢奉},并是四族之隽\footnote{会稽:郡名。治所在今浙江绍兴。孔沉(沈)。魏顗:字长齐,东晋会稽人。参看\CJKunderwave{排调}48。虞球:字和琳,东晋会稽馀姚(今属浙江)人。虞存:字道长,见\CJKunderwave{政事}17注。谢奉:字弘道。隽:优秀出众的人。},于时之桀。{\fzxk\zihao{6}\textcolor{red}{沉、存、顗、奉,并别见。\CJKunderwave{虞氏谱}曰:“球字和琳,会稽馀姚人。祖授,吴广州刺史。父基,右军司马。球仕至黄门侍郎。”}} 孙兴公目之曰\footnote{孙兴公:孙绰。目:品评。}:“沉为孔家金\footnote{金:比喻珍贵。下句“玉”同。},顗为魏家玉,虞为长、琳宗\footnote{长、琳:道长、和琳,即虞存、虞球。宗:尊崇;景仰。},谢为弘道伏\footnote{弘道:即谢奉。伏:通“服”。}。”{\fzxk\zihao{6}\textcolor{red}{长、琳,即存及球字也。弘道,谢奉字也。言虞氏宗长、琳之才,谢氏伏弘道之美也。}}

{\cangkai\zihao{5}【评】“金”、“玉”为物之高贵稀有者,“宗”、“伏”状景仰推崇之情。孙绰尝居会稽十馀年,对会稽著姓及此邦贤达有至深之了解,故品评孔、魏、虞、谢四姓佳子弟如数家珍。故事还从一个侧面展示了晋人的家庭传承意识,江山维新,门第代兴,总要涌现出优秀的子弟才能光扬门第。因为姓氏家族的利益在魏晋时代是至关重要的大事。}

\lettrine{8.86} 王仲祖\myidx{王濛}、刘真长\myidx{刘惔}造殷中军\myidx{殷浩}谈\footnote{王仲祖:王濛。刘真长:刘惔。造:拜访。殷中军:殷浩。谈:指清谈。},谈竟俱载去。刘谓王曰:“渊源真可\footnote{渊源:殷浩字。可:表示赞许,犹言“行”、“好”。}。”王曰:“卿故堕其云雾中\footnote{故:真的。云雾:比喻使人迷惑之物。}。”{\fzxk\zihao{6}\textcolor{red}{\CJKunderwave{中兴书}曰:“浩能言理,谈论精微,长于\CJKunderwave{老}、\CJKunderwave{易},故风流者皆宗归之。”}}

{\cangkai\zihao{5}【评】故事可见识人之难。常言道,“智者千虑,必有一失”。刘惔素以人伦识鉴为世人所称。本书\CJKunderwave{识鉴}门第18夫的评则载王濛、谢尚、刘惔俱造殷浩所,王、谢为浩表象迷惑,感叹“渊源不起,当如苍生何?”独刘惔一针见血地指出殷浩矫情取誉的真实用意。此则恰好相反,刘惔为殷浩的“陈势浩汗”慑服,失去了理性的辨别力,唯发出啧啧称叹而已;王濛则不为所动,而是醍醐灌顶地泼下一盆凉水使刘惔猛醒。这就好比剧终人散,而观众的反应各不相同,有的还沉浸在故事情节中不能自拔,甚至与剧中人同哭同乐(符合斯坦尼斯拉夫斯基的“体验”理论);有的则一直冷眼旁观,时时不失理性的判断(类似于布莱希特的“间离效果”说)。两次造访后的识鉴经历说明了,人伦识鉴绝非一蹴而就的易事,不经过察其言、观其行的多次往还,极有可能为表象蒙蔽,大名士刘惔这次就险些看走了眼。}

\lettrine{8.87} 刘尹\myidx{刘惔}每称王长史\myidx{王濛}云\footnote{刘尹:刘惔。王长史:王濛。}:“性至通而自然有节\footnote{通:通达,豁达。节:节制。\CJKunderwave{晋书·王濛传}:“与沛国刘惔齐名友善, 常称濛性至通而自然有节,濛每云:‘刘君知我,胜我自知。’”}。”{\fzxk\zihao{6}\textcolor{red}{\CJKunderwave{濛别传}曰:“濛之交物,虚己纳善,恕而后行,希见其喜愠之色。凡与一面,莫不敬而爱之。然少孤,事诸母甚谨,笃义穆亲,不修小洁,以清贫见称。”}}

{\cangkai\zihao{5}【评】刘惔目王濛“性至通”,当是指王濛通达任诞的个性。魏晋重父讳,王濛自矜美貌,揽镜自照,称父字曰:“王文开生如此儿邪!”又入市买帽,受老妪馈赠。达则达矣,不过掉了贵族的价,适足成为士人茶馀饭后的舆论谈资。然以宽容的眼光来看,王濛言行是展示生命之美的自我张扬,有似于西方当代社会的雅皮士风度,虽吸引公众眼球,却也无伤大体。又言“自然有节”,当指王濛“晚节始克己励行,有风流美誉,虚己应物,恕而后行,莫不敬爱焉”(\CJKunderwave{晋书}濛本传)。通而有节,正是魏晋风度的主流。}

\lettrine{8.88} 王右军\myidx{王羲之}道谢万石\myidx{谢万}“在林泽中,为自遒上\footnote{王右军:王羲之。谢万石:谢万,字万石。林泽:山林水泽。指隐逸之所。为自:算得上,称得上。遒上:挺拔高迈。}”,叹林公\myidx{支遁}“器朗神隽\footnote{林公:支道林,为东晋名僧,善玄理,是当时佛学“般若学”的代表人物,多才艺,长于草隶。与王洽、刘惔、殷浩、许询、郗超、王羲之、谢安等名流游好。常:同“尝”,曾经。注。器朗:胸怀宽广开朗。神俊:风神秀出。}”,{\fzxk\zihao{6}\textcolor{red}{\CJKunderwave{支遁别传}曰:“遁任心独往,风期高亮。”}} 道祖士少\myidx{祖约}“风领毛骨,恐没世不复见如此人\footnote{祖士少:祖约。风领毛骨:指骨相气派不凡。没世:终身。}”,道刘真长\myidx{刘惔}“标云柯而不扶疏\footnote{刘真长:刘惔字真长。标云柯:指树枝高耸云端。比喻身登显位。扶疏:枝叶分披的样子。不扶疏,比喻在高位而自抑降,闲静自守。}”。{\fzxk\zihao{6}\textcolor{red}{\CJKunderwave{刘尹别传}曰:“惔既令望,姻娅帝室,故屡居达官。然性不偶俗,心淡荣利,虽身登显列,而每挹降,闲静自守而已。”}}

{\cangkai\zihao{5}【评】王羲之品目四人,各为一类代表。谢万有隐逸之风,尝著\CJKunderwave{八贤论},以处者为优,出者为劣;支道林为高僧大德,周顗曾有“卓朗”之目,桓温评“精神渊著”,王评与前贤暗合;祖约是驰骋沙场的将军,有将军之风;刘惔为富贵优游的玄士。王羲之将隐、僧、武、玄各色人物并列齐观,可见其人伦识鉴眼光及标准是较为宽容的。}

\lettrine{8.89} 简文\myidx{司马昱}目庾赤玉\myidx{庾统}\footnote{简文:晋简文帝司马昱,晋简文:指晋简文帝司马昱(320—372),穆帝年幼即位,昱任抚军大将军总理政务。后来大将军桓温专擅朝政,先废海西公,后立司马昱为帝,第二年崩。目:品评。庾赤玉:庾统,小字赤玉。}“省率治除\footnote{省率:爽直坦率,不拘小节。治除:指治身修养,纯洁高尚。}”,谢仁祖\myidx{谢尚}云\footnote{谢仁祖:谢尚,谢豫章:谢鲲,曾作豫章太守。刘孝标注“鲲子别见”,“子”字衍。将:携,谓携之送客。自:已经。参:参与、进入。上流:上等、上品注。}:“庾赤玉胸中无宿物\footnote{宿物:隔夜之物,喻芥蒂,指中心的嫌隙不快。}。”{\fzxk\zihao{6}\textcolor{red}{赤玉,庾统小字。\CJKunderwave{中兴书}曰:“统字长仁,颍川人,卫将军择(怿)子也。少有令名,仕至寻阳太守。”}}

{\cangkai\zihao{5}【评】现代心理学将人的气质类型分为胆汁质、多血质、黏液质、抑郁质等四种。庾赤玉之“省率治除”、“胸中无宿物”,非纯然出于后天道德品格之涵养,更多的是先天气质作用的产物。临川将故事收入\CJKunderwave{赏誉}门,有叹赏其道德人格的意味,实际上是一种误解。中国文化传统中,颇有将伦理道德与气质性情纠结缠绕在一起的特点,“君子坦荡荡,小人长戚戚”,就陷入了非此即彼的对立思维模式,值得反思。“君子”、“小人”是道德评价,“坦荡”、“戚戚”则属于心理学的范畴。君子未必坦荡,小人何以总是戚戚?刁协、卞壸诸人,在名士眼里属于小人一流,其身上何尝没有君子的刚峻之风呢?王导是君子吧,可他因猜疑而心怀愤愤,默许王敦杀害国之柱石周顗;谢安该算是君子吧,可他出山,乃出于谢万败后而重振陈郡谢氏的门户私计。王、谢二人也难免有戚戚之虞。常言道“江山易改,禀性难移”,气质性情乃从娘胎里带来,任何人无力回天,故当善于发现各种性情之独特魅力,不必一定赋予其道德意义。有些心理现象,从伦理学的角度讲不通,而从心理学的角度看,就可能豁然开朗了。}

\lettrine{8.90} 殷中军\myidx{殷浩}道韩太常\myidx{韩伯}曰\footnote{殷中军:殷浩。韩太常:韩伯,时任豫章太守,故称。曾为王弼\CJKunderwave{周易注}补注\CJKunderwave{易传}之系辞、说卦、杂卦等,是当时著名玄学名家。}:“康伯少自标置\footnote{康伯:韩伯的字。标置:标榜:自负。},居然是出群器\footnote{居然:显然。出群器;超越众人的人才。}。及其发言遣辞\footnote{遣辞:用词。},往往有情致\footnote{情致:情趣。}。”{\fzxk\zihao{6}\textcolor{red}{\CJKunderwave{续晋阳秋}曰:“康伯清和有思理,幼为舅殷浩所称。”}}

{\cangkai\zihao{5}【评】晋人赏誉因了主客体双方的不同,大体有如下区别:一、生人对逝者的赏誉,如桓温目王敦为“可人”,这种情况为异代知己,虽为思古,实为自赏,可慰生命行程中之寂寥。此方式在传统文化中有很大影响,后代如陈子昂登幽州台而思燕丹、荆轲;苏东坡临赤壁而叹“雄姿英发”的周郎;二、朋侪之间的赏誉,如庾子嵩目和峤“森森如千丈松”,出于心灵交感、声气相求,招朋引类以丰富生命的内涵;三、家族或亲友间长对少的赏誉。魏晋重门第,此类数量不少。如王敦目王羲之为佳子弟,王衍对王澄的提携,以及本则殷浩对外甥韩康伯的激赏,其对家族(或亲族)利益的期待是不言自明的。这类赏誉,情感因素是很强烈的。虽少了几许谦冲礼让之风,却也符合魏晋玄风的自然真率。}

\lettrine{8.91} 简文\myidx{司马昱}道王怀祖\myidx{王述}\footnote{简文:晋简文帝司马昱。王怀祖:王述。}:“才既不长,于荣利又不淡\footnote{荣利:功名利禄。淡:淡泊。},直以真率少许\footnote{直:只。真率:自然坦率。},便足对人多多许\footnote{多多许:当时口语,谓众多。}。”{\fzxk\zihao{6}\textcolor{red}{\CJKunderwave{晋阳秋}曰:“述少贫约,箪瓢陋巷,不求闻达。由是为有识所重。”}}

{\cangkai\zihao{5}【评】简文评王述四句话,极具概括力,如漫画笔法之删繁就简,寥寥几笔线条,将其人的性情声吻烘托略尽。王世懋评此曰:“道尽蓝田,简文妙于言乃尔。”简文、蓝田是两晋有名的“痴人”,或许“痴人”之间相对更多了一点灵犀,容易欣赏对方的神采。王述确实不见有什么大作为,又有为人检举的所谓受贿污点,然而如简文所说,正是真率这一点,便足以抵得上世人的诸多优点,也使其在中古人物画廊上永远风采依然。谢安称赞蓝田“掇皮皆真”,正与此意同。魏晋尚清通真率,王述一生所为,形象而生动地展示了这一时代主题。}

\lettrine{8.92} 林公\myidx{支遁}谓王右军\myidx{王羲之}\footnote{林公:支道林。王右军:王羲之。}:“长史\myidx{王濛}作数百语\footnote{长史:王濛。},无非德音\footnote{德音:善言。此指明哲有卓识的言谈。},如恨不苦\footnote{如:转折连词。只是。恨:遗憾。 苦:用为使动,使人苦,指陷人于困境。}。”{\fzxk\zihao{6}\textcolor{red}{苦,谓穷人以辞。}} 王曰:“长史自不欲苦物\footnote{物:指人。}。”

{\cangkai\zihao{5}【评】\CJKunderwave{诗经·邶风·谷风}云:“德音莫违,及尔同死。”李陵\CJKunderwave{答苏武书}亦曰:“时因北风,复惠德音。”德音,善言也,是对他人言辞的敬称。王濛言谈不以疾言厉色屈人,不以使人辞穷自高,充耳皆美善之言。在支道林眼里,这是一种遗憾、不足,在王羲之看来,正显王濛的高致。支道林看中的是辩论的技巧和结果;王羲之则更重视谈论中呈现出的雍容气度。支道林乃方外之人,却仗才使气,一定要在论辩中将对方挑个人仰马翻才大呼过瘾,未免执于名相、为目标所累;王羲之风流名士,反而超越胜负结果而逍遥无待。孙绰商略诸人风流,以为王濛“温润恬和”,颇有儒家“温柔敦厚”的人格气象,就是“德音”的最好注脚。}

\lettrine{8.93} 殷中军\myidx{殷浩}与人书\footnote{殷中军:殷浩。},道谢万\myidx{谢万}“文理转遒\footnote{文理转遒:指文辞义理愈发刚劲有力。},成殊不易\footnote{成:通“诚”。实在,诚然。}”。{\fzxk\zihao{6}\textcolor{red}{\CJKunderwave{中兴书}曰:“万才器隽秀,善自炫曜,故致有时誉;兼善属文,能谈论,时人称之。”}}

{\cangkai\zihao{5}【评】殷浩评价谢万“文理转遒,成殊不易”,可见其本人亦是深谙写作甘苦,才能发此在行、中肯之言。一个人的创作风格,与才气学习(刘勰语)或才胆识力(叶燮语)等因素,有莫大之关联,一旦形成,则相对稳定,很难移易。“文理转遒”,当建立在生活阅历与思虑精进的基础上才有可能。杜甫诗称“庾信文章老更成,凌云健笔意纵横”。庾信成为南北朝文学的集大成者,是其由南入北生活经历所致。杜甫自己“晚节渐于诗律细”,亦与其颠沛流离的生活遭遇和不懈的艺术追求有关。谢万虽不是什么文学大家,但其文理的转变,也符合艺术规律的变化。}

\lettrine{8.94} 王长史\myidx{王濛}云\footnote{王长史:王濛。}:“江思悛\myidx{江惇}思怀所通\footnote{江思悛:江惇,字思悛,东晋陈留(今河南开封东北)人。江虨弟。笃学博览,儒道兼综。尊崇礼法,著\CJKunderwave{通道崇检论}。思怀:思虑。 通:通晓。},不翅儒域\footnote{不翅:通“不啻”,意不仅,不止。儒域:儒学领域。}。”{\fzxk\zihao{6}\textcolor{red}{徐广\CJKunderwave{晋纪}曰:“江惇字思悛,陈留人,仆射虨弟也。性笃学,手不释书,博览坟典,儒道兼综。征聘无所就,年四十九而卒。”}}

{\cangkai\zihao{5}【评】江惇笃学博览、儒玄双修,故王濛云:“不翅儒域。”江惇虽濡染时风,但总体上还是以儒者面目出现。他特别提倡儒家礼教,以为君子立行,应依礼而动,“若乃放达不羁,以肆纵为贵者,非但动违礼法,亦道之所弃也”。 著\CJKunderwave{通道崇检论},在玄风扇炽的东晋社会,江惇是一个异类。作为玄谈名家,王濛赞美江惇儒学,亦可见其宽容的学术胸怀。}

\lettrine{8.95} 许玄度\myidx{许询}送母始出都\footnote{许玄度:许询,字玄度。 出都:赴京都,到京都。},人问刘尹\myidx{刘惔}\footnote{刘尹:刘惔。}:“玄度定称所闻不\footnote{定:究竟,到底。称:适合,相副。所闻:听到的。不:同“否”。}?”刘曰:“才情过于所闻\footnote{才情:才华,才思。是魏晋品藻人物的重点之一。}。”{\fzxk\zihao{6}\textcolor{red}{\CJKunderwave{许氏谱}曰:“玄度母,华轶女也。”案询集,询出都迎姉(姊),于路赋诗。\CJKunderwave{续晋阳秋}亦然。而此言送母,疑缪(谬)矣。}}

{\cangkai\zihao{5}【评】故事似触及了人物赏誉中的“名人效应”。“名人效应”古今皆有,可谓见怪不怪。当今为传媒时代,当红人物为广告代言,出场费动辄成百上千万,甚至有人开出千万元以下免谈的身价。另一方面,大众又心甘情愿地把大把大把的钞票掏出来为少数人聚敛。今日“名人效应”之本质,不戳自破。魏晋时期“名人效应”表现在,品目者若名高位重,可服众人之口,造成一种社会风气,与今日之惟“财”是举的商业行为殊途。大概许询已然有了一定的社会声誉,而大家拿捏不准,是否名副其实?故请更大的名士刘惔作最后的甄别。刘惔评许询“才情过于所闻”,给予了极高的评价,相信许询在短期内定会声名鹊起。古代之“名人效应”,是名士间倾心的赞叹,总体上属于高雅的精神活动;今日之“名人效应”,多出于传媒的热炒,则相当于一只无形的巨手操纵、分割社会舆论、财富,是变了味的炒作行为。}

\lettrine{8.96} 阮光禄\myidx{阮裕}云\footnote{阮光禄:阮裕,即阮裕,曾以金紫大夫征,故称。\CJKunderwave{世说}作者刘义庆为避宋武帝刘裕名讳,从不称阮裕之名。剡(shàn 善):古县名,在今浙江嵊州。}:“王家有三年少:右军\myidx{王羲之}、安期\myidx{王应}、长豫\myidx{王悦}\footnote{右军:王羲之。安期:王应,(266?—333):字处明,王导从弟,时为荆州刺史。长豫:王悦。三人是王家优秀子弟。}。”{\fzxk\zihao{6}\textcolor{red}{阮裕、王悦、安期王应,并已见。}}

{\cangkai\zihao{5}【评】字安期者,有二人。一为琅邪王应,一为太原王承。此处之安期当为王应。因为阮光禄品目之其馀二人右军、长豫,为琅邪王氏子弟,而王承为太原王氏,且年辈高于三人。右军令名早已特出,不劳赘述;王应为王含子,继嗣王敦,敦尝称其“神候似欲可”,受王敦牵连而惨死;王悦乃王导长子,事亲色养,性至孝,然中岁而夭。唯右军以五十五岁寿终。丰子恺先生有小诗打趣“车厢世界”,其中有几句大意是:有的先上后下,有的后上先下。“人生天地间,忽如远行客。”寿夭福祸,你不得不承认有那么一股运道因素在起作用!另外,故事运用了“事数标榜”的品目方式。}

\lettrine{8.97} 谢公\myidx{谢安}道豫章\myidx{谢鲲}\footnote{谢公:谢安。豫章:谢鲲,曾为豫章太守,谢豫章:谢鲲,曾作豫章太守。刘孝标注“鲲子别见”,“子”字衍。将:携,谓携之送客。自:已经。参:参与、进入。上流:上等、上品注。}:“若遇七贤\footnote{七贤:指竹林七贤,即阮籍、嵇康、山涛、向秀、阮咸、王戎、刘伶。他们相与友善,常宴集竹林之下。},必自把臂入林\footnote{必自:一定。 把臂入林:拉着胳膊。}。”{\fzxk\zihao{6}\textcolor{red}{\CJKunderwave{江左名士传}曰:“鲲通简有识,不修威仪,好(\CJKunderwave{老}、\CJKunderwave{易}),迹逸而心整,形浊而言清,居身若秽,动不累高。邻家有女,尝往挑之,女方织,以梭投折其两齿。既归,傲然长啸,曰:‘犹不废我啸歌。’其不事形骸如此。”}}

{\cangkai\zihao{5}【评】谢鲲乃谢安伯父,安谓鲲“若遇七贤,必自把臂入林”,当出于对伯父内心世界的深刻体察。谢鲲曾将自己和庾亮比较,以为“端委庙堂,使百僚准则,鲲不如亮。一丘一壑,自谓过之”。其一生钟情处,正在山川丘壑之自然。自我的舒张和对自然的挚爱,是林下诸贤的集体品格。在这一点上,谢鲲与他们有着精神交感。“把臂入林”,状貌生动,点出了名士间心灵撞击后激发处的巨大情感磁力,能够超越身份、地位、相貌甚至是时间等阻碍性因素,亦可热情、诚挚的手紧紧相握,将矜持与忸怩扫于无形。正基于此,俞伯牙、锺子期才能在高山流水间寻觅知己,贺知章一见李白唤为谪仙,留下“金龟换酒”的感人诗章!}

\lettrine{8.98} 土(王)长史\myidx{王濛}叹林公\myidx{支遁}\footnote{王长史:王濛。林公:支道林。}:“寻微之功\footnote{寻微:指在玄学上探寻精微深奥的义理。},不减辅嗣\footnote{辅嗣:王弼,字辅嗣。}。”{\fzxk\zihao{6}\textcolor{red}{\CJKunderwave{支遁别传}曰:“遁神心警悟,清识玄远。尝至京师,王仲祖称其造微之功,不异王弼。”}}

{\cangkai\zihao{5}【评】王弼、支遁,一玄一佛,对于探求事物精微之理,都做出了贡献。王弼这位玄学理论的奠基人,通过对有无、动静和言意关系的逻辑论证和抽象概括,构造出了一套颇有系统的玄学本体论。王弼注重义理辨析的思想方法,有批判两汉经学烦琐学风的积极意义,对提高中华民族的抽象思维能力有一定的促进作用。从思维的缜密和精细化程度而言,确有造微之功。余嘉锡先生评价其“排击汉儒,自标新学”,恰如其分。支遁理趣符老庄,风神类谈客,乐与名士往还,以中国原有思想资料与佛学义理相比拟配合,目的使人了解信从佛教,同时亦适应清谈内容。其释\CJKunderwave{逍遥游},于向、郭之外揭示新理,以为逍遥乃指“至人之心”,只有无待的至人(圣人)才能逍遥,境界显得高远,于当日一般玄士之逍遥义外又别开生面。汤用彤先生以为“实写清谈家心胸,曲尽其妙”(\CJKunderwave{汉魏两晋南北朝佛教史})。支遁乃著名的宗教活动家,又能以玄学思想创造性地阐发了佛教义理,郗超称其“实数百年来,绍明大法,令真理不绝,一人而已”。}

\lettrine{8.99} 殷渊源\myidx{殷浩}在墓所几十年\footnote{殷渊源:殷浩字渊源。},于时朝野以拟管\myidx{管仲}、葛\myidx{诸葛亮}\footnote{朝野:朝廷内外。拟:比作。管、葛:管仲、诸葛亮。}。起不起\footnote{起:出仕。},以卜江左兴亡\footnote{卜:占卜。此指预测、估量。江左:长江下游以东地区。此指东晋王朝。}。{\fzxk\zihao{6}\textcolor{red}{\CJKunderwave{续晋阳秋}曰:“时穆帝幼冲,母后临朝,简文亲贤民望,任登宰辅。桓温有平蜀、洛之勋,擅强西陕。帝自料文弱,无以抗之。陈郡殷浩素有盛名,时论比之管、葛,故征浩为扬州。温知意在抗己,甚忿焉。”}}

{\cangkai\zihao{5}【评】\CJKunderwave{识鉴}门有云:“王、谢相谓曰:‘渊源不起,当如苍生何?’”\CJKunderwave{晋书}浩本传又载简文答浩书:“足下去就,即是时之废兴。”可与此条互证。盖其时穆帝幼冲,简文辅政,慑于桓温难驭,故起用名声极大的殷浩以抗衡桓温。殷浩好玄言清谈,时拟管、葛,时论以为殷浩的出仕与否,关系到东晋兴亡。然其人并无实战经验,受命北伐,大败而归,废黜为民。殷浩之人生沉浮,至少可以有两点启示:一、中国文化传统中有将文采等同于施政才能的误解。一个人的文章写得好,诗作得好,或口才雄辩,就自比伊、吕,希企管、葛,及至给他一个官做,却又做不好。当然,这种歧误深与选拔制度的不完善相关,无论是汉代的察举、征聘,还是唐代后的“诗赋取士”,其中都蕴涵着自身难以克服的矛盾和危机。以今天的眼光来看,殷浩可以做律师、记者、大学教授,但古代士农工商的社会结构,使其除了做官,难以找到自身位置。二、一个国家或一个团体的安危存亡,切不可寄希望于一个人。事实证明,这样的权力运作方式既容易导致集权,又容易造成政权的极度不稳。考诸世界各国政权之风云突变与萧墙祸起,自当明了。那种“地球缺了谁都照样转”的政权体制,才是稳定、健全的。当然,在中国漫长的封建社会,特别是在魏晋门阀社会,权力集中于少数一二家族,这样的质疑显然是一种苛求。}

\lettrine{8.100} 殷中军\myidx{殷浩}道右军\myidx{王羲之}“清鉴贵要\footnote{殷中军:殷浩。右军:王羲之。清鉴贵要:地位尊贵显要,有高明的鉴赏能力。}”。{\fzxk\zihao{6}\textcolor{red}{\CJKunderwave{晋安帝纪}曰:“羲之风骨清举也。”}}

{\cangkai\zihao{5}【评】此条与本门第80条“逸少清贵人”语同,似可并为一条,评略。}

\lettrine{8.101} 谢太傅\myidx{谢安}为桓公\myidx{桓温}司马\footnote{谢太傅:指谢安。晋穆帝升平三年,谢安始出仕,为桓温司马。}。{\fzxk\zihao{6}\textcolor{red}{\CJKunderwave{续晋阳秋}曰:“初,安优游山水,以敷文析理自娱。桓温在西蕃,之(钦)其盛名,讽朝廷请为司马。以世道未夷,志存匡济,年四十,起家应务也。”}} 桓诣谢,值谢桓(梳)头,遽取衣帻\footnote{值谢桓头:“桓”当为“梳”,诸本作“梳”。帻:包头巾。古代男子包裹发髻的头巾。}。桓公云:“何烦此!”因下共语至瞑\footnote{瞑:通“暝”。天黑,日暮。}。既去,谓左右曰:“颇曾见如此人不\footnote{颇曾:可曾。}?”

{\cangkai\zihao{5}【评】故事用三个画面形象地反映了魏晋名士间的任达率真之风。桓温拜访谢安,是上级过访下级,乃不拘常礼的任性之举,下级诚惶诚恐,情在理中;于是引出了第二个画面,正在梳头的谢安,听到消息后,急忙穿戴衣服迎接。以儒家礼法来看,免冠见人为非礼之举,更何况是面对自己的顶头上司?一“遽”字状其紧张、无措。不料桓温不拘常规,“何烦此”三字,可见其尚通脱、简约的风格,一句随意的安慰,给谢安吃了定心丸,将紧张的气氛扫于无形;第三个画面,“因下共语至瞑”,凸现令人叹赏的名士之风,二人的精神交流超越了世俗功利层面。可以想见,晤谈的内容绝非家长里短、田舍稻粱,亦非杀人窃国、沽名钓誉。故千载以下,令人神往、费人思量。桓温临去,由衷地对谢安发出至高赞叹。故事通过几个镜头刻画了一代枭雄桓温旷达、简约、情深蕴雅的名士风流。}

\lettrine{8.102} 谢公\myidx{谢安}作宣武\myidx{桓温}司马\footnote{谢公:指谢安。宣武:指桓温。宣武是其死后谥号。},属门生数十人于田曹中郎赵悦子\myidx{赵悦}\footnote{属:通“嘱”。托付。门生:门人。田曹中郎:官名。即田曹从事中郎,掌农政的官吏。赵悦子:赵悦,字悦子,东晋下邳(今江苏宿县)人。}。{\fzxk\zihao{6}\textcolor{red}{伏滔\CJKunderwave{大司马寮属名}曰:“悦字悦子,下邳人。历大司马参军、左卫将军。”}} 悦子以告宣武,宣武云:“且为用半\footnote{且:姑且,暂且。}。”赵俄而悉用之,曰:“昔安石在东山\footnote{东山:谢安出仕前隐居东山。在今浙江省上虞市东部。},缙绅敦逼\footnote{缙绅:指官员、士大夫。敦逼:敦促。此指征召谢安出山为官。},恐不豫人事\footnote{豫:参与。人事:世事。}。况今目(自)乡选\footnote{乡选:就乡里选拔人才。},反违之邪?”

{\cangkai\zihao{5}【评】\CJKunderwave{论语·为政}篇有言:“为政以德,譬如北辰,居其所,众星共之。”\CJKunderwave{老子}曰:“生而不有,为而不恃,功成而弗居。夫唯弗居,是以不去。”这两处经典,用以状谢安之人生信念和处世准则,都非常确切。谢安出山以前,为士林所宗,居官后更以无为之德政使士人推服。故赵悦子甘为之前驱,敢违桓温而成全谢安之美意。故事可见谢安人格感召力量之大,所谓四时无言而大美成者也。}

\lettrine{8.103} 桓宣武\myidx{桓温}表云\footnote{桓宣武:指桓温。晋穆帝永和十二年(356),桓温北伐平洛,上表荐谢尚镇洛阳。}:“谢尚\myidx{谢尚}神怀挺率\footnote{谢尚:字仁祖,晋豫章太守谢鲲子。谢豫章:谢鲲,曾作豫章太守。刘孝标注“鲲子别见”,“子”字衍。将:携,谓携之送客。自:已经。参:参与、进入。上流:上等、上品注。神怀挺率:胸襟怀抱率易挺达。},少致民誉\footnote{少:年轻时。致:得。}。”{\fzxk\zihao{6}\textcolor{red}{温集载其\CJKunderwave{平洛表}曰:“今中州既平,宜时绥定。镇西将军、豫州刺史尚,神怀挺率,少致人誉。是以入论百揆,出蕃方司。宜进据洛阳,抚宁黎庶。谓可本官都督司州诸军事。”}}

{\cangkai\zihao{5}【评】桓温作\CJKunderwave{平洛表},此年谢尚年近五十,距死期不远,故温“少致民誉”之“少”,当指中年以前。谢尚乃谢鲲之子,自幼敏于言辞。有一次一位名士夸赞他像孔夫子的高足颜回。谢尚应声而答:“坐无尼父,焉别颜回!”这句带有调侃意味的回答,透露着翩翩少年的极大自信,赢得了客人们的喝彩。谢尚善跳一种八哥舞,这种舞蹈模拟八哥的动作并加以艺术化,为好奇尚异的名士们欣赏。王导设宴待客,请谢尚表演此舞助兴。谢尚毫不忸怩,“便著衣帻而舞。导令坐者抚掌击节,尚俯仰在中,傍若无人,其率诣如此”(\CJKunderwave{晋书}尚本传)。王导比之为王戎,常呼为“小安丰”,又聘为自己的掾属,从此步入仕途。后由名士成为将军,善骑射,与庾翼打赌,控弦中的,庾翼以鼓吹(仪仗乐队)相送。谢尚与当时一些草包将军相比,并无多少浮诞之气,相反,却能文能武,才情兼备,在名士之中,殊为难得。桓温枭雄,自具只眼,“神怀挺率,少致民誉”云云,当指此。}

\lettrine{8.104} 世目谢尚\myidx{谢尚}为“令达”\footnote{谢尚:见前则。令达:美好通达。}。阮遥集\myidx{阮孚}云\footnote{阮遥集:阮孚,阮咸次子,晋元帝世为安东参军,历侍中、吏部尚书、丹阳尹、广州刺史等。}:“清畅似达\footnote{清畅:清明晓畅。}。”或云:“尚自然令上\footnote{自然:自然天成。令上:美好卓越。}。”{\fzxk\zihao{6}\textcolor{red}{\CJKunderwave{晋阳秋}曰:“尚率易挺达,昭悟令上也。”}}

{\cangkai\zihao{5}【评】“令达”、“清畅似达”诸评,似乎集中在一个“达”字。谢尚之尚达,既是时代的大风气的濡染,更是受了乃父谢鲲的家风熏陶。谢鲲可与竹林七贤“把臂入林”,谢尚号称“小安丰”,父子承传的一脉,不难捕捉。谢尚仪容既美,又好修饰,“好衣刺文裤”,就是好穿一条绣有花纹的套裤,更显得风流佻达,与众不同,如今日青少年之怪异装束与染发文身等,引领时代潮流。又如镇寿阳期间,军务倥偬之际,尚聊发少年狂,以堂堂朝廷二品命官,春日登楼,弹唱\CJKunderwave{大道曲}:“春阳二三月,柳青桃复红。车马不相识,误落黄埃中。”引得街头士女驻足倾听,举目观望。此举非达而何?然而又不越情废礼,传达了对青春之美的留驻,属于高雅的精神追求,这与玄学末流之任诞堕落之达,有着天壤之别。故时人在“达”之前加以“令”或“清”修饰,以见春秋笔意。}

\lettrine{8.105} 桓大司马\myidx{桓温}病\footnote{桓大司马病:桓温于晋孝武帝宁康元年(373)病死。},谢公\myidx{谢安}往省病\footnote{谢公:谢安。往省病:“往”字原形残,据诸本校补。省,问候。},从东门入。{\fzxk\zihao{6}\textcolor{red}{温时在姑熟(孰)。}} 桓公遥望叹曰:“吾门中久不见如此人!”

{\cangkai\zihao{5}【评】桓温对谢安,既爱敬其名士风流,又恨其刚忠,视安为自己篡权路上的阻碍,因此对其时亲时疏,态度无常。但总体上,对谢安尚有爱敬之意。温病笃,不久死去。临终之际,当不免参透了名利纷争、看破了权力拼杀,一生问鼎逐鹿、激荡风云,最后还不是要将自己交付与这一掊黄土?倒不如风流名士,落得个生也潇洒、去也随缘。思来想去,“豪华落尽见真淳”,发为咏叹,其言也善。“吾门中久不见如此人”与前之“颇曾见如此人不”两相发明,视昔日的“异己分子”谢安如云山江水,景仰之情遂定格为生命弥留之际的临终感言。殷殷赤诚,不含一点矫饰,也活脱了桓温纯情士子的一面。}

\lettrine{8.106} 简文\myidx{司马昱}目敬豫\myidx{王恬}为“朗豫”\footnote{简文:指晋简文帝司马昱。豫:王恬字敬豫,王导次子。朗豫:开朗快乐。}。{\fzxk\zihao{6}\textcolor{red}{王恬,已见。\CJKunderwave{文字志}曰:“恬识理明贵,为后进冠盖也。”}}

{\cangkai\zihao{5}【评】王恬为王导次子,因尚武而不为导喜爱。王导见了乖孩子——长子王悦眉开眼笑,见了舞刀动枪的王恬则怒从中来。有晋一代守文尚柔,贵族家风则好吟风月、舞文弄墨,动静行藏之际以流露出大贵族的雍容华贵为高,故王恬之打打杀杀、灰头土脸为导所恶。王恬心理健康,并未因家庭里的“歧视”和冷眼而影响其性格成长,而是一天生的乐天派。“朗豫”,乐观开朗貌。王恬字敬豫,简文目其“朗豫”,运用谐音双关手法。刘辰翁曰:“此一字连其人名,如谑如谥,更自高简。”分析巧妙。}

\lettrine{8.107} 孙兴公\myidx{孙绰}为庾公\myidx{庾亮}参军\footnote{孙兴公:孙绰。庾公:庾亮。},共游白石山\footnote{白石山:山名。在今江苏省。},卫君长\myidx{卫永}在坐\footnote{卫君长:卫永,字君长,东晋济阴成阳(今山东曹县东北)人。}。{\fzxk\zihao{6}\textcolor{red}{\CJKunderwave{卫氏谱}曰:“永字君长,成阳人,位至左军长史。”}} (孙曰):“此子神情,都不关山水\footnote{此子神情:据袁本,“此子”前脱“孙曰”二字。关:关心、注意。},而能作文。”庾公曰:“卫风韵虽不及卿诸人\footnote{风韵:风度韵致。},倾倒处亦不近\footnote{倾倒处:令人倾倒的地方。近:浅。}。”孙遂沐浴此言\footnote{沐浴:领会、涵咏。}。

{\cangkai\zihao{5}【评】孙绰是东晋著名的玄言诗人,早年与许询等人居于会稽,游弋山水。曾将自己与许询比较,云:“高情远志,弟子早已服膺;然一吟一咏,许将北面矣。”对自己吟咏山水的文学才能,是十分自信甚至自负的。故事中孙绰讽卫永“神情都不关山水,而能作文”,实是张扬己长,而暗寓讥刺,不自觉间犯了文人相轻的毛病。庾亮的回答,既盛赞了孙绰诸人的风姿韵度,更突出了卫永不同凡近的才华。可见庾亮气度雍容,能包纳各种类型人士为己所用。}

\lettrine{8.108} 王右军\myidx{王羲之}目陈玄伯\myidx{陈泰}“垒块有正骨\footnote{王右军:王羲之。陈玄伯:陈泰字玄伯,魏司空陈群子。垒块:土块,比喻胸中郁结不平之气。正骨:刚正的品格。}”。{\fzxk\zihao{6}\textcolor{red}{陈泰,已见。}}

{\cangkai\zihao{5}【评】陈泰为魏司空陈群子。太丘陈寔至纪、群、泰四世,于汉魏两朝并有重名。泰亦自我砥砺、立事立功。王右军目陈泰“垒块有正骨”,当指司马篡魏之际,陈泰所表现出的浩然正气。据\CJKunderwave{魏氏春秋}载,魏高贵乡公被杀后,陈泰枕帝尸于股,号哭尽哀。司马昭问计于泰,泰对曰:“独有斩贾充,少可以谢天下耳。”司马昭请更思他计,泰曰:“岂可使泰复发后言。”此句情感复杂,意谓“难道还要我把后面更难听的话说出来吗?”先是声讨杀人真凶贾充,后又将矛头直指司马氏集团。魏晋易代之际,士人或佯狂杜门,或望风使舵,或助纣为虐,像陈泰这样刚直难犯的臣子,确实有难能可贵的“正骨”。}

\lettrine{8.109} 王长史\myidx{王濛}云\footnote{王长史:王濛。}:“刘尹\myidx{刘惔}知我\footnote{刘尹:刘惔。},胜我自知。”{\fzxk\zihao{6}\textcolor{red}{濛别传》曰:“濛与沛国刘惔齐名,时人以濛比袁曜卿,惔比荀奉倩,而共交友,甚相知赏也。”}}

{\cangkai\zihao{5}【评】王濛与刘惔齐名友善,濛云“刘君知我,胜我自知”,当是对刘惔常称其“性至通,而自然有节”的有感而发。常言道:人贵有自知之明,苏东坡诗有“不识庐山真面目,只缘身在此山中”之人生感悟。做到自知,何其难也!作为好友,刘惔能够透过王濛任诞的外表,得出“自然有节”的结论,是知人论世的认识,是同情之理解。具备这种同情,即使有千山万水的空间阻隔,也能因绵邈的深情而产生心灵的交流和碰撞;即使跨越百代的今人和古人,也能因“读其书、诵其诗”而结成旷世神交。何况王濛、刘惔相濡以沫、息息相通,更容易涵咏对方的人生意趣、精神旨归。}

\lettrine{8.110} 王\myidx{王濛}、刘\myidx{刘惔}听林公\myidx{支遁}讲\footnote{王、刘:王濛、刘惔。林公:指支遁。遁字道林,东晋僧人。},王语刘曰:“向高坐者\footnote{高坐者:坐在上座的人。此指支道林。},故是凶物\footnote{故:本来。凶物:不吉之人。}。”复更听\footnote{更:再。},王又曰:“自是钵盂后王\myidx{王弼}、何\myidx{何晏}人也\footnote{自:原来,本来。 钵盂后:钵盂,是佛门传法之器。钵盂后,犹言如来传法之后的佛界之中。王、何:指王弼、何晏。}。”{\fzxk\zihao{6}\textcolor{red}{\CJKunderwave{高逸沙门传}曰:“王濛恒寻遁,遇祗洹寺中讲,正在高坐上。每举麈尾,常领数百言,而情理俱畅,预坐百馀人,皆结舌注耳。濛(云):‘听讲众僧,向高坐者,是钵盂后王、何人也。’”}}

{\cangkai\zihao{5}【评】\CJKunderwave{高僧传}载支遁初至京师,太原王濛甚重之,曰:“造微之功,不减辅嗣。”又载王濛因其才词,往诣遁,作数百语,后乃叹曰:“实缁钵之王、何也。”与本篇内容互相补充。王濛对支遁由误解到理解并最终发出叹赏的过程,正符合文学阅读接受的一般心理机制。王濛首先是带着一定的“阅读经验期待视野”,去解读支遁的讲经谈玄。支遁宣讲的内容与方式,带着鲜明的异教痕迹和个性化色彩,大大刺激东土士人的眼球和耳膜,因而其“期待视野”遇挫,产生抗拒与抵触情绪在所难免;随着接受活动的不断深入,王濛很快为支遁指示的豁然开朗的境界而振奋,并因扩充和丰富了“期待视野”而感到欣悦与满足(即如支遁注\CJKunderwave{逍遥游}与向郭义之外别立新解,新天下耳目)。“山重水复疑无路,柳暗花明又一村”,王濛便在遇挫与开悟交替的精神活动中,体验到了支遁玄谈的无尽魅力,产生精神上的共鸣,并进而由衷地赞叹“自是钵盂后王、何人也”。}

\lettrine{8.111} 许玄度\myidx{许询}言\footnote{许玄度:许询,有才藻,善为五言诗。}:“\CJKunderwave{琴赋}所谓‘非至精者\footnote{\CJKunderwave{琴赋}:嵇康所作,今存\CJKunderwave{昭明文选}卷一八。精:明细。},不能与之析理’,刘尹\myidx{刘惔}其人\footnote{刘尹:刘惔。};‘非渊静者\footnote{渊静:沉静恬淡。},不能与之闲止\footnote{闲止:闲居。}’,简文\myidx{司马昱}其人\footnote{简文:晋简文帝司马昱。}。”{\fzxk\zihao{6}\textcolor{red}{嵇叔夜\CJKunderwave{琴赋}也。刘惔真长,丹阳尹。}}

{\cangkai\zihao{5}【评】嵇康有很深的音乐造诣,有\CJKunderwave{声无哀乐论}、\CJKunderwave{琴赋}等论文,专门探讨音乐的艺术哲学。许询引用嵇康之言,实际上,已经触及了艺术接受活动中,非常普遍的“共鸣”现象。音乐种类、形式多样,人各有偏好,“非至精者,不能与之析理”,“非渊静者,不能与之闲止”,就是马克思所说的“对于没有音乐感的耳朵来说,最美的音乐也毫无意义”(\CJKunderwave{1844年经济学哲学手稿})。共鸣的大前提是主客体的双向契合,缺一不可。正如苏轼\CJKunderwave{琴诗}云:“若言琴上有琴声,放在匣中何不鸣。若言声在指头上,何不于君指上听。”否则,就陷入我们常说的“对牛弹琴”之境。}

\lettrine{8.112} 魏隐\myidx{魏隐}兄弟\myidx{魏逷}少有学义\footnote{魏隐兄弟:魏隐,字安时,东晋会嵇(稽)会上虞(今属浙江)人,安帝隆安中为义兴太守,孙恩陷会稽,隐弃职而逃。其弟魏逷,仕黄门郎。学义:学识。},{\fzxk\zihao{6}\textcolor{red}{\CJKunderwave{魏氏谱}曰:“隐字安时,会嵇(稽)上虞人,历义兴太守、御史中丞。弟逷,黄门郎。”}} 总角诣谢奉\myidx{谢奉}\footnote{总角:古代男女未成年前束发为两结,形状如角,故称总角。谢奉:字弘道,晋会稽山阴人。},奉与语,大说之\footnote{说:通“悦”,喜爱。},曰:“大宗虽衰\footnote{大宗:始祖的嫡长子为大宗,其他为小宗。},魏氏已复有人。”

{\cangkai\zihao{5}【评】谢奉于魏氏兄弟为乡邦长者,总角一见辄叹“大宗虽衰,魏氏已复有人”。谢奉的超拔激赏,以及\CJKunderwave{世说}所载众多长辈对童蒙小儿的赏识,均可见出魏晋士人对儿童早期教育的重视。现代教育学已然揭示,人生观、价值观,及诸多行为习惯,均萌蘖于童稚期,其未来成长方向,均与儿时的教养有莫大关联。“三岁看老”并不是一个被夸张的命题。贤明长者三言两语的夸赞,甚至是一个赞许的表情,既是对小儿的肯定,又为其以后的成长指示了方向。俗语云“鼓励能使猪上树”,几句夸赞,言者无心,听者有意,情商高的孩子能因此产生人生的顿悟。如俄国文豪托尔斯泰的父母爱好文艺,时常教育孩子阅读俄罗斯古典文学作品,一次小托尔斯泰放开喉咙,朗诵普希金的\CJKunderwave{致大海}。父亲看到孩子对作品有一定的理解,就对其报以一个赞许的微笑。这个愉快的印象在托尔斯泰的心灵深处一直保持到晚年。又如我国著名画家朱屺瞻小时候苦练绘画,有一次他父亲过生日,他画了一幅\CJKunderwave{清供图}以祝贺生日。他父亲深情地看了他一眼,这不平凡的一瞥,成了朱屺瞻一生艺术追求的永恒动力。种种事例,不胜枚举,教育者可不慎欤?}

\lettrine{8.113} 简文\myidx{司马昱}云\footnote{简文:晋简文帝司马昱。}:“渊源\myidx{殷浩}语不超诣简至\footnote{渊源:殷浩,字渊源。超诣:高超卓越。简至:简明通达。},然经纶思寻处\footnote{经纶:整理丝缕,理出头绪,此指思路条理。思寻:思考。},故有局陈\footnote{故:确实。局陈:即“局阵”。格局阵势,言人论谈布置有法,犹如棋局兵阵。}。”

{\cangkai\zihao{5}【评】简文评殷浩之“言”与“思”,涉及了中国哲学的言意之辨问题,也可以说是涉及了人类思维的内外转换问题。中国文化传统中比较占优势的观点是“言不尽意”论。\CJKunderwave{庄子·天道}说:“语之所贵者意也,意有所随。意之所随者,不可以言传也。”\CJKunderwave{易传·系辞}中借孔子之口说:“书不尽言,言不尽意。”也承认言不能完全传达意,但又补充说:“圣人立象以尽意,设卦以尽情伪。”表现出调和的态度。陆机\CJKunderwave{文赋}则从人类一般思维规律的角度,阐发创作中“意不称物,文不逮意”的苦恼,其实是指从内语言(意或思)到外语言转换过程中的心理现象。殷浩长于思而非精于言,了然于胸而不能了然于口与手,于此病最为典型。或许正如唐大圆\CJKunderwave{文赋注}云:“意虽善构,若无词藻以达之,则又患在学俭。”解救之法,惟在勤学,可使由俭而博。}

\lettrine{8.114} 初,法汰\myidx{法汰}北来\footnote{法汰:竺法汰。北来:从北方来。},未知名,{\fzxk\zihao{6}\textcolor{red}{专(车)频\CJKunderwave{秦书}曰:“释道安为慕容俊所掠,欲投襄阳,行至新野,集众议曰:‘今遭凶年,不依国主,则法事难举。’仍分僧众,使竺法汰诣扬州,曰:‘彼多君子,上胜可投。’法汰遂渡江,至扬土焉。”}} 王领军\myidx{王洽}供养之\footnote{王领军:王洽,字敬和。王导第三子。供养:供给生活所需。}。{\fzxk\zihao{6}\textcolor{red}{\CJKunderwave{中兴书}曰:“王洽字敬和,丞相导第三子。累迁吴郡内史,为士民所怀。征拜中领军,寻加中书令,不拜。年二(三)十六而卒。”}} 每与周旋行\footnote{周旋:亲密往来。},来往名胜许\footnote{名胜:名流,著名人士。许:处,处所。},辄与俱;不得汰,便停车不行。因此名遂重。{\fzxk\zihao{6}\textcolor{red}{\CJKunderwave{名德沙门题目}曰:“法汰高亮开达。”孙绰为汰赞曰:“凄风拂林,明泉映壑,爽爽法汰,校德无怍。事外萧洒,神内恢廓。实从前起,名随后跃。”\CJKunderwave{泰元起居注}曰:“法汰以十五(二)年卒,烈宗诏曰:‘法汰师丧逝,哀痛伤怀,可赠钱十万。’”}}

{\cangkai\zihao{5}【评】竺法汰在东晋士人中间享有至高的声誉。王洽之心驰神往、以礼相待,已如前述。又如,汰在桓温座共语,因病不堪久坐而提前退场,桓温听说后,匆匆抛开宾客,亲自将其接回。简文皇帝对他亦深相敬重,亲到瓦官寺听讲。这些人,上至帝王下至公卿,为何对不名一文、毫无权力,又丝毫不关涉国计民生的方外僧人心醉神迷呢?此中可透视出魏晋士人追求人生高致、探讨宇宙真谛、不汲汲于实用的无功利心态和超拔的精神取向,在历史长河中留下个性鲜明的一笔。这种态度,是治学求知最应有的态度,也最利于真理的产生。近代大学者王国维描述的学问“不分古今、不分中外、不分有用无用”的超迈境界,是这种态度的一脉延伸。只可惜,这种心态在中国文化传统重实用理性的强大潮流中,显得非常微弱,惜哉!}

\lettrine{8.115} 王长史\myidx{王濛}与大司马\myidx{桓温}书\footnote{王长史:王濛。大司马:指桓温。},道渊源\myidx{殷浩}识致安处\footnote{渊源:殷浩,字渊源。此盖为隐居墓所、未为扬州刺史之时。识致:见识情致。安处:平日居处。},足副时谈\footnote{副:符合。时谈:时人的评说。}。

{\cangkai\zihao{5}【评】故事当发生在殷浩尚隐居墓所时。故时人对其出、处期望甚大。这里,我们暂时抛开王濛评价殷浩的真伪不谈,做出这样的推论,就是很多魏晋士人是名不副实的。有孔稚圭\CJKunderwave{北山移文}中所揭示的“形在江海之上,心存魏阙之下”的假隐士;有像王衍那样口中雌黄而大节有亏的空谈家;当然也有像谢万那样善于邀名窃誉而庸碌无能的纨绔子弟,等等,不一而足。魏晋文化是自由的,多元的,形形色色的人等粉墨登场,寻找其生存空间。“人格面具的过度膨胀者”(荣格语)所在多有,给人物品评带来较大难度,那种一蹴而就、盖棺定论式的品藻,实在要独具心眼才行。多数时候,察其言、观其行,经过从实践到认识,再从认识到实践的反复,是非常必要的。事实证明,王濛虽然观察了殷浩的见识情致、日常起居,自以为所见不差,其实最终还是看走了眼。}

\lettrine{8.116} 谢公\myidx{谢安}云\footnote{谢公:谢安。}:“刘尹\myidx{刘惔}语审细\footnote{审细:周密严谨。}。”{\fzxk\zihao{6}\textcolor{red}{孙绰为惔诔叙曰:“神犹渊镜,言必珠玉。”}}

{\cangkai\zihao{5}【评】故事与本门第八十三条王长史谓刘惔“致言处自寡耳”,有异曲同工之妙,可谓出言矜慎不凡。此处谢安又评其出言周密详细,众口一词,想必刘惔言谈确有审、慎特色。}

\lettrine{8.117} 桓公\myidx{桓温}语嘉宾\myidx{郗超}\footnote{桓公:桓温。嘉宾:郗超,小字嘉宾,郗超:任桓温大司马,深得信任,立简文为帝后,迁中书侍郎,实代桓温监督朝廷而权重当时。在直:在宫中值班。}:“阿源\myidx{殷浩}有德有言\footnote{阿源:殷浩,字渊源。 有德有言:有德望,有名言。},向使作令仆\footnote{向:先前。令仆:尚书令、尚书仆射,为综理朝政之官。},足以仪刑百揆\footnote{仪刑:示范。百揆:百官。},朝廷用违其才耳\footnote{违:与……不相称。}!”{\fzxk\zihao{6}\textcolor{red}{嘉宾,郗超小字也。阿源,殷浩也。}}

{\cangkai\zihao{5}【评】桓温、殷浩是总角之交,用今天北京人话讲是“发小”,长期耳鬓厮磨,相知甚深。后因人生追求不同,而成为分属于不同政治阵营的政敌。桓温一代枭雄,气魄宏大,虎视晋鼎;殷浩则成为简文帝所倚重的重臣,其实是用与桓温抗衡。历史和命运有时就是这样无情,它使昔日的至交好友,成为各有所主的冤家对头。然而,他们又不愧是滋养了魏晋风度的名士,即便明争暗斗,心底还时时流露不泯的人性温情。即如桓温,这一次还在为殷浩忧虑,向郗超念叨着,殷浩是仪刑百揆的廊庙之才,却不是指挥三军的好统帅。晋穆帝永和中,任用殷浩为中军将军,统师北伐,结果大败而回,浩也被废黜。用违其才,正指此事。为泄怨气,殷浩北伐失败后,桓温上疏请求严惩,落井下石不遗馀力;后又旧情复萌,荐举殷浩作尚书令。魔鬼与天使、刚狠老辣与绵邈情深,完美地交织成一曲人性善恶的二重奏。桓温诸流,恰如林下名士王戎所谓的“情之所钟,正在我辈”。}

\lettrine{8.118} 简文\myidx{司马昱}语嘉宾\footnote{简文:晋简文帝司马昱。嘉宾:郗超,见前则。}:“刘尹\myidx{刘惔}语末后亦小异,回复其言\footnote{回复:回味,反复思考。},亦乃无过。”

{\cangkai\zihao{5}【评】刘惔自视言谈为第一流,与王濛俱为简文座上客。孙盛曾作\CJKunderwave{易象妙于见形论},殷浩与辩不胜,简文命人迎请刘惔,孙理遂屈。好像专家出诊,手到病除。殷浩为清谈名家,然相形之下,逊惔一筹。言谈清辩,有时论难往还、汪洋恣肆,难免前后失应,语杂小痴;炉火纯青如刘惔者,也会犯“语末后亦小异”的毛病。然而,只需义理要点一以贯之,不枝不蔓,就不会影响大局。晋人尚通脱,不过分株守章句、拘泥文辞小异,支遁玄讲取“九方皋相马,略其玄黄而取其俊逸”(谢安语)的态度,以及陶渊明“不求甚解”的读书方法,正可作为通脱的例证。}

\lettrine{8.119} 孙兴公\myidx{孙绰}、许玄度\myidx{许询}共在白楼亭\footnote{孙兴公:孙绰。许玄度:许询。白楼亭:亭名,在今浙江绍兴附近。},{\fzxk\zihao{6}\textcolor{red}{\CJKunderwave{会稽记}曰:“亭在山阴,临流映壑也。”}} 共商略先往名达\footnote{商略:商讨,议论。先往名达:先前有名望的贤达。}。林公\myidx{支遁}既非所关\footnote{林公:支遁字道林,东晋僧人,人称林公。},听讫云:“一(二)贤故自有才情\footnote{一贤故自有才情:据文义及诸本,“一”当为“二”。故自,确实,真是。才情,才华。}。”

{\cangkai\zihao{5}【评】人物品藻虽是对被谈论对象的评价,而在商略、鉴赏的过程中,品目者本人的风度、才情也会借此洋洋洒洒地挥发出来。孙绰、许询这次在白楼亭探讨先往名达,因超越了现实人际利害关系的纠葛,而显得空明纯净、无拘无束。支道林因事不关己,以佛家的法眼在一旁静观,更有一种“众鸟高飞尽,孤云独去闲”的悠然体察。其评判孙、许才情,是不急不躁的智者之言。故事揭示出人物品藻过程中,评人与被评均属正常。正是:哪个背后不说人?哪个不被旁人说?要想逃脱被人评说的境地,也是枉然。此情此境,恰如现代诗人卞之琳先生在\CJKunderwave{断章}中所描绘的:“你站在桥上看风景,看风景的人在楼上看你。明月装饰了你的窗子,你装饰了别人的梦。”}

\lettrine{8.120} 王右军\myidx{王羲之}道东阳\myidx{王临之}\footnote{王右军:王羲之。东阳:王临之,小字阿林,官东阳太守,指东阳太守王临之。}:“我家阿林\footnote{阿林:王临之。刘注:“林”应为临。},章清太出\footnote{章:通“彰”,明亮。清:高洁。太出:很突出,很杰出。}。”{\fzxk\zihao{6}\textcolor{red}{“林”应为“临”。\CJKunderwave{王氏谱}曰:“临之字仲产,琅邪人,仆射彪之子,仕至东阳太守。”}}

{\cangkai\zihao{5}【评】王临之为羲之同宗侄辈,故羲之称“我家阿林(临)”。清,是晋人对人的至高赞美之词,“章清太出”,当指其心理、才思极其清朗澄明,故羲之有此激赏。惜史不详载。}

\lettrine{8.121} 王长史\myidx{王濛}与刘尹\myidx{刘惔}书\footnote{王长史:王濛。 刘尹:刘惔。},道渊源\myidx{殷浩}触事长易\footnote{渊源:殷浩。触事:办事,处事。长:通“常”,经常。易:平易;平和。}。

{\cangkai\zihao{5}【评】刘辰翁评此则为“费辞说”,因词有歧义而影响意思之理解。一说以为“触”通“处”,“长”通“常”,意谓殷浩处事经常很平易。一说以为“长”,上声,语由\CJKunderwave{易·系辞}“触类而长”化用而出。原文曰:“引而申之,触类而长之,天下之能事毕矣。”“触长”之义,指一卦六爻之动可变为六十四卦,六十四卦可变成四千零九十六卦。天下万物皆如此例,各以类增长,则天下所能之事,法象皆尽。如此,则王长史称道殷浩对于玄言义理能活学活用,不拘一隅,不局一器,上下周流,触类皆长。}

\lettrine{8.122} 谢中郎\myidx{谢万}云\footnote{谢中郎:谢万。}:“王修载\myidx{王耆之}乐托之性\footnote{王修载:王耆之。王𢋸(廙)子,王胡之弟。乐托:同“落拓”。形容不拘小节,放荡不羁。},出自门风\footnote{门风:家风。}。”{\fzxk\zihao{6}\textcolor{red}{\CJKunderwave{王氏谱}曰:“嗜(耆)之字修载,琅邪人,荆州刺史𢋸(廙)弟(笫)三子。历中书郎、鄱阳太守、给事中。”}}

{\cangkai\zihao{5}【评】魏晋间以落拓不羁为名士本色,故谢万评价王修载落拓,临川乃列之于“赏誉”。王修载落拓之举,史所不载,但\CJKunderwave{晋书}载其父王𢋸(廙)行止,可资一观:“𢋸(廙)性隽率,尝从南下,旦自浔阳,迅风飞帆,暮至都,倚舫楼长啸,神气甚逸。”从社会学和教育学的角度看,一个人的性格、情致和行为习惯的形成,确实有家族积淀、传承等因素在内。虽不能如基因遗传规律那样屡试不爽,然言传身教耳濡目染之功,不可小觑。}

\lettrine{8.123} 林公\myidx{支遁}云\footnote{林公:支道林。}:“王敬仁\myidx{王修}是超悟人\footnote{王敬仁:王修,字敬仁。超悟:高超颖悟。}。”{\fzxk\zihao{6}\textcolor{red}{\CJKunderwave{文字志}曰:“修之少有秀令之称。”}}

{\cangkai\zihao{5}【评】王修少年颖悟,世称“秀出”。十二岁作\CJKunderwave{贤全论},为刘惔所称;其辞章学问,挺拔出众,谢尚称其“文学镞镞,无能不新”。支道林又称其“超悟人”。综合各家印象,可见名不虚传。然二十四岁而夭,令人唏嘘!临终叹曰:“无愧古人,年与之齐矣。”其言颇费解,是否指青年夭亡的玄学天才王弼?待考。但有一点可以肯定,以王修之秀出、超悟,必自高自砥砺,取法乎上,这位古人一定是位青年夭折的贤人。}

\lettrine{8.124} 刘尹\myidx{刘惔}先推谢镇西\myidx{谢尚}\footnote{刘尹:刘惔。推:推崇;推许。谢镇西;谢尚,谢豫章:谢鲲,曾作豫章太守。刘孝标注“鲲子别见”,“子”字衍。将:携,谓携之送客。自:已经。参:参与、进入。上流:上等、上品注。},谢雅重刘\footnote{谢雅重刘:袁本“谢”下多一“后”字,于义更优。雅重,极为推重。},曰:“昔尝北面\footnote{北面:旧时君见臣,尊长见卑幼,南面而坐,故以“北面”指向人称臣或居于人下。}。”{\fzxk\zihao{6}\textcolor{red}{案:谢尚年长于惔,神颖夙彰。而曰北面于刘,非可信。}}

{\cangkai\zihao{5}【评】刘孝标注以为谢尚年长于刘惔,而曰北面于刘,非可信。凌濛初曰:“推重耳何足致疑。况刘亦堪此,勿论年长。”凌说有理。唐韩愈\CJKunderwave{师说}以为“吾师道也,夫庸知其年之先后生于吾乎?是故无贵无贱,无长无少,道之所存,师之所存也”。为追求学问真知,应该有一种“虽千万人吾往也”的态度和勇气,更何况名士间的倾心,本为超越身份、地位、年龄等一些外在附加性因素的高级精神活动,谢尚是潇洒不羁的名士,想必不会有倚老自重的蓬塞之心。}

\lettrine{8.125} 谢太傅\myidx{谢安}称王修龄\myidx{王胡之}曰\footnote{谢太傅:谢安。王修龄:王胡之。}:“司州可与林泽游\footnote{司州:王胡之曾任司州刺史,故称。林泽:山林水泽。山水胜境,是隐者所居。}。”{\fzxk\zihao{6}\textcolor{red}{\CJKunderwave{王胡之别传}曰:“胡之常遗世务,以高尚为情,与谢安相善也。”}}

{\cangkai\zihao{5}【评】谢安与王胡之乃林泽友,逯钦立先生辑\CJKunderwave{先秦汉魏晋南北朝诗}载二人交友酬唱诗各一首,可见一斑。谢安\CJKunderwave{与王胡之诗}六章,第六章曰:“朝乐朗日,啸歌丘林。夕玩望舒,入室鸣琴。五弦清澈,南风披襟。醇醪淬虑,微言洗心。幽畅者谁,在我赏音。”称王胡之为可以把臂入林的“赏音”。王\CJKunderwave{答谢安诗}八章,中述“畴昔宴游,缱绻苕龄。或方童颜,或始角巾”。又曰:“今也华发,卑高殊韵。形迹外乖,理畅内润。”从总角到垂老,二人保持长期的友谊。在这里,我们要为这群名士庆幸,崇尚自然的时代风气,造就了一大批光鲜雅洁的名士,从容不迫的心态保证他们在多元的社会文化分层中,找到属于自己的群体,而不必忍受倾诉无门的悲哀!}

\lettrine{8.126} 谚曰:“扬州独步王文度\myidx{王坦之}\footnote{扬州:州名。东晋时治所在建康。独步:独一无二,一时无两。比喻杰出的人才。王文度:王坦之。},后来出人郗嘉宾\myidx{郗超}\footnote{后来:后辈,晚辈。出人:超越众人。郗嘉宾:郗超,郗超:任桓温大司马,深得信任,立简文为帝后,迁中书侍郎,实代桓温监督朝廷而权重当时。在直:在宫中值班。}。”{\fzxk\zihao{6}\textcolor{red}{\CJKunderwave{续晋阳秋}曰:“超少有才气,越世负俗,不循常检,时人(为)一代盛誉者语曰:‘大才槃槃谢家安,江东独步王文度,盛德日新郗嘉宾。’其语小异,故详录焉。”}}

{\cangkai\zihao{5}【评】\CJKunderwave{晋书}王坦之传所载为“盛德绝伦郗嘉宾,江东独步王文度”。与此稍异。王坦之、郗超俱为名公之子,少有重名,在东晋晚期的政治舞台上均扮演了重要的政治角色。王坦之是简文帝死后与谢安辅助幼主的比肩重臣,对谢安的某些浪漫而不合礼俗的举动予以不遗馀力的抵制,相较之下,显其刚直近乎板正的儒者面孔;郗超有绝顶的聪明和超群的才华,在他身上矛盾的对立因素能够和谐共存。一方面他有优雅的风度,善于交友,拔寒素、轻金钱,为名士所推;另一方面,他又违背了忠孝仁义的古训,鬼使神差做了桓温的死党,有失名士节慨。或许他心目中的“大树”,就该是桓温这样既雅于深情,又能刻石立功的实干家吧!只可惜一失足成千古恨,成为朝廷的敌对势力,聪明反被聪明误。王、郗这两位当年雄姿英发的贵公子,因人生选择路径不同,写就了自己在历史上一正一邪的形象,成为后世殷鉴。故事运用了谐音双关的艺术技巧。如“步”与“度”,“人”与“宾”俱叶韵。}

\lettrine{8.127} 人问王长史\myidx{王濛}江霦\myidx{江虨}兄弟群从\footnote{王长史:王濛。江霦:据诸本作“江虨”,是。群从:指同族子弟。}。王答曰:“诸江皆复足自生活\footnote{诸江:指江家兄弟子侄。皆:都。复,词缀,无实义。足自:完全能够。生活:生存,谓立足于世。}。”{\fzxk\zihao{6}\textcolor{red}{霦及弟淳(惇)、从灌,并有德行,知名于世。}}

{\cangkai\zihao{5}【评】东晋高门世族中,陈留江家是奉儒守正的传统士人家庭,\CJKunderwave{晋书}载江氏子弟行止,皆以任道而行、刚正不阿著称于世。这样的家风虽然少了些浪漫潇洒的气质,在魏晋时代风气之下,显得有点落落寡合,却因其实践躬行的精神,为国家做出了实际的贡献。长史王濛与江霦游处情好,雅相钦重,出于对江家的了解,称赞江氏群从皆足以立足于世。一个大家族后来的守成者,若不是躺在祖先的基业上坐吃山空,也不是将父辈“老革命”的资格作为资本而讨价还价、邀功请赏,而是自己打拼挣得一份生活的资本和做人的尊严,何其难能可贵!中国有“人过留名,雁过留声”的古训,江氏兄弟可谓以自己的德业留名青史了。}

\lettrine{8.128} 谢太傅\myidx{谢安}道安北\myidx{王坦之}\footnote{谢太傅:谢安。安北:王坦之,死后赠安北将军,故称。}:“见之乃不使人厌,然出户去,不复使人思。”{\fzxk\zihao{6}\textcolor{red}{安北,王坦之也。\CJKunderwave{续晋阳秋}曰:“谢安初携幼稚同好,养志海滨,襟情超畅,尤好声律。然抑之以礼,在哀能至。弟万之丧,不听丝竹者将十年。及辅政,而修室第园馆,丽车服,虽期功之惨,不废妓乐,王坦之因苦谏焉。”案:谢公盖以工(王)坦之好直言,故不思尔。}}

{\cangkai\zihao{5}【评】刘孝标注以为谢安因王坦之“好直言,故不思尔”。朱铸禹\CJKunderwave{汇校集注}不以为然,驳之曰:“坦之性情冲悒平淡,故云见之不使人厌;而又无风流韵度,故云去不复使人思。盖谢似嘉其能而矜持自洁,不慕纷华耳。信如注云以苦谏而不思,则当如会孟所评使人畏,见之亦使人厌,且不当入赏誉矣。”二人所评,刘重政治道德,朱重风度品格。细加揣摩,当以朱评为是,以见晋人精神。与王坦之不同,谢安是一个懂得生活享受又魅力四射的中心型人物。谢、王二人性情不同,但却能以君子之交相忘于江湖的态度,共辅幼主,以济时艰。以宰相肚里能撑船的谢安,焉能包容不下王坦之的直谏而耿耿于怀呢?}

\lettrine{8.129} 谢公\myidx{谢安}云\footnote{谢公:谢安。}:“司州\myidx{王胡之}造胜遍决\footnote{司州:王胡之。造胜:进入胜境。指探究玄理。遍决:遍释群难。}。”{\fzxk\zihao{6}\textcolor{red}{\CJKunderwave{宋明帝文章志}曰:“胡之性简,好达玄言也。”}}

{\cangkai\zihao{5}【评】谢安谓老朋友王胡之谈玄论理能诣胜境,又能全面释疑解难,给人以很高评价。士林中有“攀安提万”之评,王胡之的清谈水平,当与安相去未远。}

\lettrine{8.130} 刘尹\myidx{刘惔}云\footnote{刘尹:刘惔。}:“见何次道\myidx{何充}饮酒\footnote{何次道:何充,字次道,晋康帝时为骠骑将军。},使人欲倾家酿\footnote{家酿:家中自制的酒。而陆游\CJKunderwave{老学庵笔记}则曰:“晋人所谓见何次道令人倾家酿,犹云欲竭家资以酿酒饮之。”所解可另备一说。}。”{\fzxk\zihao{6}\textcolor{red}{充饮酒能温克。}}

{\cangkai\zihao{5}【评】在乡土中国,特别是在酒风醇浓的中国北方的民间酒宴上,有“酒品看人品”的不成文规矩。这并非一般对北方民风、民德有所隔膜的南土人士所认为的那样,以为北方人喝酒就是要把人灌醉。真正酒德高尚的人,一定是生活中的道德君子,说到底,是追求一种与人和谐、其乐融融的人生境界。酒桌上,是否偷奸耍滑或者憨厚仗义,明眼人往往因酒德而判定其人德。这亦是一种见微知著的人物品目方法。在酒桌上,还经常会遇到耍小聪明的人士,采用偷梁换柱、装疯卖傻等小伎俩。酒桌上的小人,在现实生活中绝非君子,此言万世不易。儒家\CJKunderwave{礼记·乡饮酒义}有云“吾观于乡,而知王道之易易也”,即通过乡间饮酒之尊贤尚齿而见王道教化。不过,儒家之饮酒礼实在太烦琐,最理想状态是合于“酒以成礼”的精神实质,不拘泥于形式上的束缚,同时尽可能展示饮者的个性风采,以收其乐融融之效。大概何充就是属于善饮而不乱,又颇具风度的类型。故人愿倾家酿,陪同畅谈。}

\lettrine{8.131} 谢太傅\myidx{谢安}语真长\myidx{刘惔}\footnote{谢太傅:谢安。真长:刘惔。}:“阿龄\myidx{王胡之}于此事,故欲太厉\footnote{阿龄:王胡之,字修龄。故:确实。欲:好像。厉:严厉。}。”{\fzxk\zihao{6}\textcolor{red}{修龄,王胡之小字也。}} 谢曰:“亦名士之高操者\footnote{谢曰:谢,诸本作“刘”,是。高操者:品格高尚的人。}。”{\fzxk\zihao{6}\textcolor{red}{\CJKunderwave{胡之别传}曰:“胡之治身清约,以风操自居。”}}

{\cangkai\zihao{5}【评】谢安所谓太厉者,当指\CJKunderwave{方正}门第52则陶胡奴送米,而王曰:“王修龄若饥,自当就谢仁祖索食,不须陶胡奴米。”王胡之严格士庶之别,拒绝陶范(胡奴)的友情馈赠,殊违人情之常,使人一腔热血化冰。同样一件事,谢安认为王胡之自矜太甚,刘惔却认为正反映出名士高格。相较之下,谢安心胸涵容,正合“王者不却众庶,故能明其德”(李斯\CJKunderwave{谏逐客疏})之意。刘惔则与王胡之一样,其门阀意识已深入骨髓,对于名士的理解,已入刁钻偏狭一路。}

\lettrine{8.132} 王子猷\myidx{王徽之}说\footnote{王子猷:王徽之,字子猷,王羲之子。王子猷、子敬:王羲之二子。说:评论;评说。}:“世目士少\myidx{祖约}为朗\footnote{士少:祖约字士少,晋范阳人。朗:高洁开朗。},我家亦以为傲朗\footnote{我家:我,说话人称自己。我家亦以为傲朗:傲,诸本作“彻”。彻朗,通达爽朗。}。”{\fzxk\zihao{6}\textcolor{red}{\CJKunderwave{晋诸公赞}曰:“祖约少有清称。”}}

{\cangkai\zihao{5}【评】王徽之评,当在祖约死后几十年,可见魏晋士人不以成败论英雄的心态。\CJKunderwave{世说}人物品目,“朗”是使用频率极高的一个词,属于上乘的评价。\CJKunderwave{说文解字}释朗为“明”,可见被誉为“朗”者,其性情气度必有透亮、开朗、舒展的特点。世人评祖约为“朗”,王徽之在“朗”前加一修饰词,一字之增,见出魏晋之赏誉识鉴,态度严谨,非率意为之。}

\lettrine{8.133} 谢公\myidx{谢安}云\footnote{谢公:谢安。}:“长史\myidx{王濛}语甚不多\footnote{长史:王濛。},可谓有令音\footnote{令音:佳美言辞。}。”{\fzxk\zihao{6}\textcolor{red}{\CJKunderwave{王濛别传}曰:“濛性和畅,能清言,谈道贵理中,简而有会。商略古贤显默之际,辞旨劭令,往往有高致。”}}

{\cangkai\zihao{5}【评】\CJKunderwave{晋书}王濛本传载孙绰与简文商略风流人物,亦称濛“能言理,辞简而有令”,与谢安所见略同。晋人清谈辞令尚简尚通,谈道贵理中,不以锦饰文采、铺扬弘丽为高。从另外一个角度看,由繁入简,由雕饰返自然,自能虚室生白,易有令音佳言。}

\lettrine{8.134} 谢镇西\myidx{谢尚}道敬仁\myidx{王修}\footnote{谢镇西:谢尚,曾作镇西将军。谢豫章:谢鲲,曾作豫章太守。刘孝标注“鲲子别见”,“子”字衍。将:携,谓携之送客。自:已经。参:参与、进入。上流:上等、上品注。敬仁:王修,字敬仁。}:“文学镞镞\footnote{文学:辞章学问。镞镞:挺拔出众。},无能不新\footnote{新:更新,创新。}。”{\fzxk\zihao{6}\textcolor{red}{\CJKunderwave{语林}曰:“敬仁有异才,时贤皆重之。王右军在郡,迎敬仁,叔仁辄同车,常恶其迟,后以马迎敬仁。虽复风雨,亦不以车也。”}}

{\cangkai\zihao{5}【评】“若无新变,不能代雄。”(萧子显\CJKunderwave{南齐书·文学传论})王敬仁于辞章学问博才出众,且在各方面都能有所建树,有所创新。这类人于知识、学问触类而长,举一反三,其思维方法值得学习。相反,迂夫子则皓首穷经,缺少变通而“白发死章句”,最终不过是两脚书橱。\CJKunderwave{文心雕龙}曰:“文律运周,日新其业。变则其久,通则不乏。”其实不仅文律,人间万事无不以变通周流获其新生,这也正符合\CJKunderwave{易}之生生不息的精神旨归。}

\lettrine{8.135} 刘尹\myidx{刘惔}道江道群\myidx{江灌}“不能言而能不言\footnote{刘尹:刘惔。江道群:江灌字道群。}”。{\fzxk\zihao{6}\textcolor{red}{江灌,已见。}}

{\cangkai\zihao{5}【评】刘尹此评,只将一词顺序颠倒而生新意,可谓巧妙。“不能言”指不善言辞,是语言天赋欠缺;“能不言”则是缄默之道,是后天的人格修养。江灌能够化不能言为能不言,实是化短为长,属于棋高一招的人生智慧。他之所以这样做,可能出于如下的原因:一、是一种品德修养,秉承了儒家“知之为知之,不知为不知”的态度。古往今来,夸夸其谈、不懂装懂者不可胜数,最终不免露出破绽,为世人耻笑。如今之学术界有一种“语不懵人死不休”的恶俗,搞新名词轰炸,“言必称希腊”,实则如“七宝楼台,碎拆下来,不成片段”。就是做不到“能不言”。二、可能出于明哲保身的考虑而三缄其口,有似于阮嗣宗的“终身履薄冰,谁知我心焦”。这又是一种生存智慧。}

\lettrine{8.136} 林公\myidx{支遁}云\footnote{林公:支道林。}:“见司州\myidx{王胡之}\footnote{司州:王胡之。},警悟交至\footnote{警悟:机敏聪慧。},使人不得住\footnote{住:停止。指王胡之谈锋引人入胜,牵着人随他的思路走。},亦终日忘疲。”{\fzxk\zihao{6}\textcolor{red}{\CJKunderwave{王胡之别传}曰:“胡之少有风尚,才器率举,有秀悟之称。”}}

{\cangkai\zihao{5}【评】玄言清谈何以使人欲罢不能,终日忘疲?这一方面说明王胡之言语有一种令人着迷的魅力,另一方面说明了魏晋士人对于高雅的精神生活的热衷。这种热情,近于王国维独标之治学三境界的第二重“衣带渐宽终不悔,为伊消得人憔悴”。为探求真知义理,言听双方抛开一切尘世间的俗事,达到忘饥、忘疲的境界,长此以往,蔚然成风,可收移风易俗之功,民族的整体素质必然随之得以提高。这样的场景,人或讥之为痴,然而,痴情正是发明创新的原动力之一。反之,一个国家、民族,若集体性地沉陷于金钱的追逐和感性欲望的放纵之中,排斥宝贵文化遗产,漠视精神世界建设,乃是饮鸩止渴的慢性自杀行为!}

\lettrine{8.137} 世称苟子\myidx{王修}秀出\footnote{苟子:王修字敬仁,小字苟子,王濛子。秀出:超群出众。},阿兴\myidx{王蕴}清和\footnote{阿兴:王蕴,字叔仁,小字阿兴,王濛子。清和:清静和平。}。{\fzxk\zihao{6}\textcolor{red}{苟子,已见。阿兴,王蕴小字。}}

{\cangkai\zihao{5}【评】“秀出”,超群出众意也;“清和”,清净平和貌。二子同出而气性有异,但俱为太原王氏庭前的芝兰玉树。}

\lettrine{8.138} 简文\myidx{司马昱}云\footnote{简文:晋简文帝司马昱。}:“刘尹\myidx{刘惔}茗柯有实理\footnote{刘尹:刘惔。茗柯:昏懵的样子。}。”{\fzxk\zihao{6}\textcolor{red}{“柯”一作“朾”,又作“仃”,又作“打(艼)”。}}

{\cangkai\zihao{5}【评】各家解释皆以为“茗柯”为“茗艼”之误,即酩酊,后转为懵懂。此言刘惔看似精神懵懂,发言却往往有实理。老子曰“大巧若拙”,又曰“俗人昭昭,我独昏昏。俗人察察,我独闷闷”。刘惔貌似懵懂,却不妨其真气内充,洞若观火。这就是人世的辩证法,世人大抵为表面现象欺骗,无法得出真相。即如今日的偶像崇拜,徒悦外貌形体或一些细碎、毫无内在意义的所谓“个性化”的酷言酷行,却对给人灵魂以深刻启迪、对社会进步做出重大贡献的人和事嗤之以鼻。其实无论是偶像,还是追星族(即今日所谓“粉丝”),伸出舌头,内里大多空空荡荡。这种对个性化的理解,实走入荒唐可笑的误区,是逃避崇高精神、放弃理性思考的自甘肤浅的偷懒行为。刘辰翁评曰:“五字最妙。大道之极,昏昏默默。”深得其中滋味。}

\lettrine{8.139} 谢胡儿\myidx{谢朗}作箸作郎\footnote{谢胡儿:谢朗小字胡儿,谢据长子,谢安侄。著作郎:官名,掌编修国史。著作郎始到职,必为名臣一人撰传。},尝作\CJKunderwave{王堪传}\footnote{\CJKunderwave{王堪传}:文篇名,谢朗撰。},{\fzxk\zihao{6}\textcolor{red}{\CJKunderwave{晋诸公赞}曰:“堪字世胄,东平寿张人。少以高亮义正称。为尚书左丞,有准绳操。为石勒所害,赠太尉。”}} 不谙堪是何似人\footnote{谙:熟悉。何似:什么样的。},咨谢公\myidx{谢安}\footnote{咨:问。谢公:谢安。}。谢公答曰:“世胄\myidx{王堪}亦被遇\footnote{世胄:王堪字。被遇:被赏识。指受朝廷赏识。}。堪,烈\myidx{王烈}之子\footnote{烈之子:王烈的儿子。}。{\fzxk\zihao{6}\textcolor{red}{\CJKunderwave{晋诸公赞}曰:“烈字阳秀,早知名魏朝,为治书御史。”}} 阮千里\myidx{阮瞻}姨兄弟\footnote{阮千里:阮瞻,字千里,晋陈留人。阮咸子。姨兄弟:姨表兄弟。},潘安仁\myidx{潘岳}中外\footnote{潘安仁:潘岳,字安仁。西晋荥阳中牟(今河南)人。诗文名家,工诗赋,辞藻艳丽,善为哀诔之体。\CJKunderwave{秋兴赋}为其重要作品。中外:中表兄弟。},安仁诗所谓‘子亲伊姑,我父唯舅’\footnote{“安仁诗所谓”句:潘岳有\CJKunderwave{北芒送别王世胄诗}五章,此处所引为第一章中两句。据诗,王堪之母为潘岳之姑母,潘岳之父为王堪之舅父,故王堪与潘岳为中表兄弟。}。是许允\myidx{许允}婿\footnote{许允(?—254):字士宗,三国魏高阳(今属河北)人。魏明帝时为吏部郎,后为晋景王司马师所杀。}。”{\fzxk\zihao{6}\textcolor{red}{岳集曰:“堪为成都王军司马,岳送至北如(邙)别,作诗曰:‘微微发肤,受之父母。峨峨王侯,中外之首。子亲伊姑,我父唯舅。’”}}

{\cangkai\zihao{5}【评】\CJKunderwave{晋书·职官志}曰:“著作郎始到职,必撰名臣传一人。”侄儿谢据作\CJKunderwave{王堪传},却不了解传主是何许人,向“活化石”叔父谢安讨教。谢安的回答,并未叙述王堪平生功业,却大谈特谈其社会关系。乍一看,令人如坠云里雾里。什么姨表亲、姑舅亲,好像是交代社会履历,但正是在这复杂的亲族坐标系中,王堪的身份逐渐凸显。谢安娓娓道来,王堪的社会关系图谱便基本鲜明了。中国人重裙带关系,晋人更重门第及血亲,谱系绵长,瓜瓞交错,历经千载,至今依稀可见此传统一脉相传的历史印痕。这在魏晋门阀社会中,谱牒学是一大学问。刘辰翁评曰:“作文不知来历,害事。谢公似不通。”刘氏未能深思谢安用心所在。}

\lettrine{8.140} 谢太傅\myidx{谢安}重邓仆射\myidx{邓攸}\footnote{谢太傅:谢安。邓仆射:邓攸,字伯道,官至尚书仆射。},常言:“天地无知,使伯道无儿\footnote{伯道无儿:\CJKunderwave{晋书·邓攸传}载:永嘉之乱,邓攸南逃,步行担其儿及弟之子,计不能两全,乃舍己子而保全弟之子,后竟无嗣。时人语曰:“天道无知,使邓伯道无儿!”}。”{\fzxk\zihao{6}\textcolor{red}{\CJKunderwave{晋阳秋}曰:“邓攸既弃子,遂无复继嗣,为有识伤惜。”}}

{\cangkai\zihao{5}【评】\CJKunderwave{晋书}邓攸本传载攸逃往途中,步行担其儿及弟之子,计不能两全,乃舍己子而保全弟之子,系子于树而去。关于此事,历来有两种评价。一是当时人的观点,时人叹曰:“天道无知,使邓伯道无儿。”表达了赞赏与同情。二是传后的“史臣”之论,表达了相异的观点,以为邓攸把儿子系于树而去,“斯岂慈父仁人之所用心也?卒以绝嗣,宜哉!勿谓天道无知,此乃有知矣”。分歧端在所持之伦理道德视角不同。我们跳出其争论的圈子,试从另外角度观照。中国文化传统重香火子嗣、传宗接代,故\CJKunderwave{孟子}赵岐注有“不孝有三,无后为大”的训诫。这是以血统相传的后代。另一方面,春秋时期鲁国叔孙豹提出了“太上立德,其次立功,其次立言”的所谓“三不朽说”。不言自明,以德、业、言相传的才是禄之大者,可垂万世不朽。生理上的后代虽可瓜瓞绵长,但并不可靠。多少君子之泽,二世、三世而斩。纨绔子弟、不肖儿孙,虽多奚益?惟立德、立功、立言之人却没世而不绝。释迦、孔子、老子、耶稣、华盛顿、李白、杜甫诸人,史不绝书,均不以子孙传。从这个角度看,邓伯道的有儿无儿,就显得不是那么重要了。刘辰翁评曰:“诔语如此,千古如生。”谢安、刘辰翁对邓攸的赞叹仅止于其身,何关乎有儿无儿呢?}

\lettrine{8.141} 谢公\myidx{谢安}与王右军\myidx{王羲之}书曰\footnote{谢公:谢安。王右军:王羲之。}:“敬和\myidx{王洽}栖托好佳\footnote{敬和:王洽字敬和,王导子。栖托:身心所寄托。}。”{\fzxk\zihao{6}\textcolor{red}{\CJKunderwave{中兴书}曰:“洽于公子中最知名,与颍川荀羡俱有美称。”}}

{\cangkai\zihao{5}【评】“栖托好佳”意为有安身立命的雄厚资本。王洽在王导诸子中最为知名,晋穆帝评价他“清裁贵令”(\CJKunderwave{晋书}本传)。虽出身王公巨卿之门,却毫无纨绔子弟的恶劣习气,其身心内外具有安身立命之资,为谢安赏识,殊为难得。相反,历朝历代都不乏养尊处优、不思进取得“八旗子弟”,不仅本人毫无生存本钱,就连祖上遗留的家业也败得精光,则又是等而下之了。\CJKunderwave{世说}中有诸多士人家庭教育的良好范本,值得那些自身风光无限、子女们窝囊至极的家长们深思!}

\lettrine{8.142} 吴四姓旧日(目)云\footnote{吴四姓:吴郡顾、陆、朱、张四姓。旧日:据文义及诸本当为“旧目”,即从前的品评。}:“张\myidx{张昭}文\footnote{张:张昭之族。文:文才。},朱\myidx{朱然}武\footnote{朱:朱然、朱桓之族。 武:武功。},陆\myidx{陆逊}忠\footnote{陆:陆逊之族。忠:忠诚。},顾\myidx{顾雍}厚\footnote{顾:顾雍之族。厚:宽厚。}。”{\fzxk\zihao{6}\textcolor{red}{\CJKunderwave{吴录士林}曰:“吴郡有顾、陆、朱、张为四姓,三国之间,四姓盛焉。”}}

{\cangkai\zihao{5}【评】不同家族由于其开创者独特的治家之道,而显示出同中有异的门风。今人萧华荣先生著书,以“簪缨世家”和“华丽家族”为题目分别概括王、谢家族的特点。王氏家族与谢氏家族相比,更追求权势;反之,谢氏家族更倾向于文采和浪漫。当然,这只是就主导倾向和代表人物而言,并不能概括其所有分歧。“门风”是在长辈的价值观念、生活取向、人生态度影响下形成的传统,既有传承性,也有变异。以四字概括四大家族传统,文字简约有味。张、朱、陆、顾为吴郡四大家族,文、武、忠、厚为各家族相对的区别性特征,张昭之文才,朱然之武功,陆逊之忠勇,顾雍之宽厚,大体而言是成立的。若说陆逊之武功,朱然之忠勇也未为不可。可见人物品目在力求精当到位的同时,亦无法排除含混、朦胧的特性。}

\lettrine{8.143} 谢公\myidx{谢安}语王孝伯\myidx{王恭}\footnote{谢公:谢安。王孝伯:王恭。(?—398):孝武帝后兄,安帝舅父。与殷仲堪、桓玄等,二次兴兵清君侧,兵败被诛。会稽:郡治在今浙江绍兴市。}:“君家蓝田myidx{\}举体无常人事\footnote{蓝田:指王述,袭爵蓝田侯。举体:全身,浑身。}。”{\fzxk\zihao{6}\textcolor{red}{案:述虽简而性不宽裕,投火怒蝇,方之未甚。若非太傅虚相褒饰,则\CJKunderwave{世说}谬设斯语也。}}

{\cangkai\zihao{5}【评】谢安谓蓝田“举体无常人事”,指其质性通透澄明,如浑金璞玉自然可爱。刘孝标注与谢安评语意忤,以为蓝田性急,如怒踏鸡卵,不当得此佳评。王世懋亦以为“注驳是”。余意以为,君子“不以一眚而掩大德”,性情急躁与宽裕属于天生气质范畴,如能加强后天修养,如古人佩韦、佩弦以自缓、自急,则微瑕不足以掩盖美玉之姿。\CJKunderwave{晋书}载蓝田有自知之明,主动加强自身修养,“既跻重位,每以柔克为用。谢奕性粗,尝忿述,极言骂之。述(蓝田)无所应,面壁而已。居半日,奕去,始复坐”。\CJKunderwave{论语}云“君子之过也,如日月之蚀焉。过也,人皆见之;更也,人皆仰之”。王述善于改过自省的做法,难道不值得肯定和赞扬吗?}

\lettrine{8.144} 许掾\myidx{许询}尝诣简文\myidx{司马昱}\footnote{许掾:许询字玄度,东晋名士。诣:拜访。简文:晋简文帝司马昱。},尔夜风恬月朗\footnote{尔夜:此夜。},乃共作曲室中语\footnote{曲室:密室,幽深隐秘的地方。}。襟情之咏\footnote{襟情之咏:抒发情怀的吟咏。谓作诗。襟情,襟怀。},偏是许之所长,辞寄清婉\footnote{偏:更,最。辞寄:言辞兴寄。清婉:清丽婉约。},有逾平日。简文虽契素\footnote{契素:意趣投合。},此遇尤相咨嗟,不觉造膝\footnote{造膝:促膝,形容亲切。},共叉手语\footnote{叉手:双手相握。叉,原刻形似“义”。},达于将旦。既而曰:“玄度才情,故未易多有许\footnote{故:确实。许:句末语气词。}。”{\fzxk\zihao{6}\textcolor{red}{\CJKunderwave{续晋阳秋}曰:“询能言理,曾出都迎姉(姊)。简文皇帝、刘真长,说其情旨及襟怀之咏,每造膝赏对,夜以系日。”}}

{\cangkai\zihao{5}【评】故事描述的乃是一次名士间高雅的晤谈。月夜清风,主贤客雅,谈诗论文,切磋琢磨,这样的场面可遇而不可求。良辰美景,赏心乐事,贤主嘉宾,有如辐辏,正如王勃\CJKunderwave{滕王阁序}所描绘的“四美具,二难并”的佳会盛况。这次晤谈的主客双方一为帝王,一为布衣隐士,身份悬殊而能促膝而坐,握手而谈,打破世俗界限,犹相咨嗟,是对封建礼教等级制的解构,契合庄子“齐物”之义。此情此景,在重事功、重利益而鲜有人情味的时代,殊为难得,或许已成为可艳羡而遥不可及的绝响。言谈忘疲,秉烛达旦,超脱日常生活之累,恰似“逍遥”的人生境界。故事从一个侧面反映出了晋人的生命观、审美观——在最动乱、最富于悲剧意味的时代里,却弹奏出了最浓情、最自由的华彩乐章,不能不令物质生活极丰富,而精神追求极贫乏的后人感到汗颜。}

\lettrine{8.145} 殷允\myidx{殷允}出西\footnote{殷允:字子思,东晋陈郡长平(今河南西华东北)人,恭素谦退,有儒者风。 出西:往西边去。指到京城去。},郗超\myidx{郗超}与袁虎\myidx{袁宏}书云\footnote{郗超:郗超:任桓温大司马,深得信任,立简文为帝后,迁中书侍郎,实代桓温监督朝廷而权重当时。在直:在宫中值班。袁虎:袁宏字彦伯,小字虎。}:“子思求良朋,托好足下\footnote{托好:寄托友情。谓相结交。足下:对同辈的敬称。},勿以开美求之\footnote{开美:开朗美好。求:要求,责求。}。”{\fzxk\zihao{6}\textcolor{red}{\CJKunderwave{中兴书}曰:“允字子思,陈郡人,太常康弟(第)六子。恭素谦退,有儒者之风。历吏部尚书。”}} 世目袁为“开美”,故子敬\myidx{王献之}诗曰\footnote{子敬:王献之,(344—388),出于琅邪王氏家族。曾任谢安长史,官至中书令,故称王令或王大令。据\CJKunderwave{晋书·后妃传},尚简文帝女新安公主。少有令名,“风流一时之冠”。其书法已造神境,与父羲之并称“二王”。病笃:病重。}:“袁生开美度\footnote{度:风度。}。”

{\cangkai\zihao{5}【评】殷允恭素谦退,有儒者风范;袁宏开朗爽彻,沾溉了任放之气。郗超欲玉成二人之结交,却担心袁宏持论太厉,不交非类,而致书开导劝慰。拳拳之心令人感叹。郗超所虑不无道理,以己之长、轻人所短,正是人类弱点,自非豁达之士不能免此通病。临川收入\CJKunderwave{赏誉}门,固称袁之开美,然亦赏殷之谦退,尤赏郗超玉成之美。}

\lettrine{8.146} 谢车骑\myidx{谢玄}问谢公\myidx{谢安}\footnote{谢车骑:谢玄,谢安侄。车骑:此指谢玄,谢安侄,死后追赠车骑将军。谢公:谢安。}:“真长\myidx{刘惔}性至峭\footnote{真长:刘惔。峭:严厉苛刻。},何足乃重\footnote{何足:哪里值得。乃:如此,这样。}?”答曰:“是不见耳\footnote{是不见耳:谓谢玄未曾见过刘惔。刘惔死时,谢玄才六七岁。}。阿见子敬\myidx{王献之}\footnote{阿:我。子敬:王献之,(344—388),出于琅邪王氏家族。曾任谢安长史,官至中书令,故称王令或王大令。据\CJKunderwave{晋书·后妃传},尚简文帝女新安公主。少有令名,“风流一时之冠”。其书法已造神境,与父羲之并称“二王”。病笃:病重。},尚使人不能已\footnote{不能已:情不能已。指使人钦敬。已,止。}。”{\fzxk\zihao{6}\textcolor{red}{\CJKunderwave{语林}曰:“羊驎因酒醉,抚谢左军谓太傅曰:‘此家讵复后镇西?’太传曰:‘汝阿见子敬,便沐浴为论兄辈。’”推此言意,则安以玄不见真长,故不重耳。见子敬尚重之,况真长乎?}}

{\cangkai\zihao{5}【评】程炎震以为,刘惔卒时,谢玄才六七岁,故不相见也。世传刘惔风操太厉,玄未晓叔父谢安何以雅相推重,故有此问。谢安回答,避其词锋,而以子敬尚使人钦敬,烘托真长之风度魅力。一“尚”字用意深长,情感天平倾向不言自明。“真长性至峭”,当指其严格士庶之别、名士调门高而言。不过,狂士如阮籍者,虽待世人以清白眼,可还有“青眼聊因美酒横”的时候,刘惔之“峭”,也是有一定范围的。对于大名鼎鼎的妹夫谢安,刘惔恐不会太过放肆吧,或许一物降一物也说不定;另一方面,对于已经作古的名士,惔为安之妻兄,婚姻至亲,谢安总该有意回护才是。}

\lettrine{8.147} 谢公\myidx{谢安}领中书监\footnote{谢公:谢安。领:以职位较高的身份兼任较低的官职。中书监:官名。中书设监、令各一人,并掌机密。},王东亭\myidx{王珣}有事\footnote{王东亭:王珣,曾封东亭侯,王导孙。},应同上省\footnote{上省:到省台。}。王后至,坐促\footnote{坐促:座位窄狭。},王、谢虽不通\footnote{王、谢虽不通:王谢两家不相交往。王珣、王珉兄弟都是谢氏婿,以猜嫌致隙离婚,于是两家成仇衅。},太傅\myidx{谢安}犹敛膝容之\footnote{太傅:谢安。敛膝:收拢膝盖,留出馀地。}。{\fzxk\zihao{6}\textcolor{red}{王、谢不通事,别见。}} 王神意闲畅\footnote{神意:神态。闲畅:闲适舒畅。},谢公倾目\footnote{倾目:注目。}。还谓刘夫人曰\footnote{刘夫人:谢安妻刘氏,刘惔之妹。}:“向见阿苽\myidx{王珣}\footnote{阿苽:王珣,小名阿苽。},故自未易有\footnote{故自:的确,确实。},{\fzxk\zihao{6}\textcolor{red}{案:王询(珣)小字法护,而此言阿瓜(苽),未为可解,傥小名有两耳。}} 虽不相关,正自使人不能已已\footnote{正自:真是,确实。 已已:停止,休止。}。”

{\cangkai\zihao{5}【评】谢安、王珣佳翁美婿,两家门当户对!却因家族利益而猜嫌致隙,遂致离婚成愁,令人情郁于中,难以释怀。不料冤家路狭,翁婿恰巧同坐,不知有何好戏上演!不料,谢安“敛膝容之”,颇有宽厚长者之风;王珣“神意闲畅”,毫无尴尬不安之态。回家后,谢安又向夫人盛赞昔日女婿的不凡举止,远非小肚鸡肠之人所能望其项背。故事通过生活中一突发事件,为我们展示了东晋士人超越门户私怨,将生活中琐事甚至憾事,化成雍容高雅的精神情趣,展示了不凡的胸襟和爱美的慧眼。后来,谢安死而王珣前去恸哭,同样是对名士风度发自内心的尊重,与此相映成趣,合成双美!}

\lettrine{8.148} 王子敬\myidx{王献之}语谢公\myidx{谢安}\footnote{王子敬:王献之,(344—388),出于琅邪王氏家族。曾任谢安长史,官至中书令,故称王令或王大令。据\CJKunderwave{晋书·后妃传},尚简文帝女新安公主。少有令名,“风流一时之冠”。其书法已造神境,与父羲之并称“二王”。病笃:病重。谢公:谢安。}:“公故萧洒\footnote{故:确实。萧洒:豁达不拘谨,超逸脱俗。}。”谢曰:“身不萧洒\footnote{身:我。},君道身最得\footnote{君:你,侪辈之间称“君”。道:品题,评论。得:满意;得意。},身正自调畅\footnote{正自:确实,真正。调畅:和谐舒畅。}。”{\fzxk\zihao{6}\textcolor{red}{\CJKunderwave{续晋阳秋}曰:“安弘雅有器,风神调畅也。”}}

{\cangkai\zihao{5}【评】王子敬称赞谢安潇洒,谢言不敢自谓潇洒。他人之评皆皮相,唯子敬所评最为相得,故其胸襟自然畅快。谢安所言道出了一常见的心理现象,即人们面对鼓励与表扬时舒畅惬意的内心反应。贤人君子驭之以道,则能最大限度地发挥人之主观能动性;小人利用此人类弱点逢迎主上,以达到卑劣的目的。教育者善用此道,则“鼓励能使猪上树”。优点、缺点全在掌控之中。谢安虽有人之常情一面,不过还能保持清醒头脑,认为自己并不潇洒,殊为难能可贵。故王世懋有“谢公自知”之评。}

\lettrine{8.149} 谢车骑\myidx{谢玄}初见王文度\myidx{王坦之}\footnote{谢车骑:谢玄,车骑:此指谢玄,谢安侄,死后追赠车骑将军。王文度:王坦之,字文度。},曰:“见文度,虽萧洒相遇\footnote{萧洒:偶然,无意地。},其复\xpinyin*{愔愔}竟夕\footnote{愔愔:安详和悦貌。竟夕:终日,整天。}。”

{\cangkai\zihao{5}【评】愔愔,\CJKunderwave{康熙字典}释为“安和”、“深静”貌。王坦之与谢玄虽偶然相见,却整晚一直保持安和、深静的意态。主要原因是王坦之虽濡染了魏晋玄风潇洒不羁的一面,但总体上还是以儒者的面貌出现。王曾著\CJKunderwave{废庄论},\CJKunderwave{晋书}本传载其“尤非时俗放荡,不敦儒教”。儒家待人接物注重居处恭、执事敬、治事勤;独处不愧屋漏,出门如见大宾。王坦之虽与晚辈偶然相逢,并非正式会面,还是展现了视听言动、一秉于礼的儒者风范。当代学者季羡林先生被请去吃饭,已经走至半路,觉得穿着随意不礼貌,折回去更换衣服,后被人强劝作罢(引自余秋雨\CJKunderwave{借我一生})。季先生之举实与王文度之“愔愔竟夕”出于同一鹄的,都是儒家礼仪长期熏陶而化成内心自律的必然反映。这样也就可以理解,王坦之何以对谢安任情越礼的声色享受,总是予以坚决地抵制。}

\lettrine{8.150} 范豫章\myidx{范宁}谓王荆州\myidx{王忱}\footnote{范豫章:范宁,曾任豫章太守,豫章:郡名,辖境为今江西省大部分地区,郡治在今南昌。此指豫章太守。王荆州:王忱,曾任荆州刺史,(?—398):孝武帝后兄,安帝舅父。与殷仲堪、桓玄等,二次兴兵清君侧,兵败被诛。会稽:郡治在今浙江绍兴市。}:{\fzxk\zihao{6}\textcolor{red}{范宁、王忱,并已见。}} “卿风流隽望\footnote{风流:指人的仪容俊美,气度不凡。隽望:俊逸而有名声。},真后来之秀。”王曰:“不有此舅,焉有此甥。”

{\cangkai\zihao{5}【评】\CJKunderwave{方正}门第66则载王忱在舅氏范宁家遇吴士张玄,自恃门第,不与之语,曰:“张希祖欲相识,自可见诣。”本则舅甥间对话接此而来。细品语气,二人似有调侃味道,于诙谐谈笑间已见出名士间的互相标榜。范宁奉儒之家,一生功业志在兴学;王忱任性不拘,嗜酒裸游。舅甥之间,价值取向不同,而能互相包容,见出时代意识形态和而不同、多元共存的特点。}

\lettrine{8.151} 子敬\myidx{王献之}与子猷\myidx{王徽之}书\footnote{子敬:王献之,(344—388),出于琅邪王氏家族。曾任谢安长史,官至中书令,故称王令或王大令。据\CJKunderwave{晋书·后妃传},尚简文帝女新安公主。少有令名,“风流一时之冠”。其书法已造神境,与父羲之并称“二王”。病笃:病重。子猷:王徽之,献之之兄,王子猷、子敬:王羲之二子。},道:“兄伯萧索寡会\footnote{兄伯:兄长。 萧索寡会:落落寡合,孤寂不合群。},遇酒则酣畅忘反,乃自可矜\footnote{乃自:确实。可矜:值得夸耀。}。”

{\cangkai\zihao{5}【评】王献之评徽之语,主观上褒,客观上却起了贬的效果。王家子弟以其高贵的门第而骄傲狂妄,以自我为中心,目空一切,俯视世人。由于目中无人,当然会落落寡合,只能借酒浇愁了。献之兄弟名士旨趣相通,故有此评。}

\lettrine{8.152} 张天锡\myidx{张天锡}世雄凉州\footnote{张天锡:字纯嘏,小字独活。世雄凉州:世代称雄于凉州。凉州:指今甘肃、宁夏和青海地区。},以力弱诣京师,虽远方殊类\footnote{殊类:异族。},亦边人之桀也\footnote{桀:通“杰”,才能出众者。}。{\fzxk\zihao{6}\textcolor{red}{天锡,已见。}} 闻皇京多才\footnote{皇京:京都。此指建康。},钦羡弥至。犹在渚住\footnote{渚:小洲。住,停泊。言过江后尚停留在洲渚。},司马箸(著)作往诣之\footnote{司马箸(著)作:当时姓司马官著作之人。},{\fzxk\zihao{6}\textcolor{red}{未详。}} 言容鄙陋,无可观听。天锡心甚悔来,以遐外可以自固\footnote{以:认为。遐外:边远之地。自固:自己固守。}。王弥\myidx{王珉}有隽才美誉\footnote{王弥:王珉,小字僧弥。},当时闻而造焉。{\fzxk\zihao{6}\textcolor{red}{\CJKunderwave{续晋阳秋}曰:“珉风情秀发,才辞富赡。”}} 既至,天锡见其风神清令\footnote{风神:风度神采。清令:清雅美好。},言话如流,陈说古今,无不贯悉\footnote{贯悉:通晓。}。又谙人物氏族\footnote{谙:熟悉。人物氏族:指名士风流及门阀望族。},中来皆有证据\footnote{中来:诸本皆作“中来”,与影宋本同。不可解。李慈铭云:“中来当是中表之误。”魏晋重婚姻门望,故重中表亲。}。天锡讶服。

{\cangkai\zihao{5}【评】京都因人物荟萃、资源集中,历来为各地人士向往,正所谓,越陌度阡,有如辐辏;争相趋鹜,有如过江之鲫。东晋虽偏安江左,可衣冠南渡,依然维系着繁荣。张天锡作为“边人之杰”,就是抱着这样一种仰视的心情来到皇都建康。所见二人,一言容鄙陋,无可观听;一风神清令,言话如流。导致一悔一服,心情迥异。王珉为王导之孙,总算为京都士人争回了颜面。故事从一个侧面说明了中心与边缘不可绝对化,“山外青山楼外楼”,\CJKunderwave{庄子·秋水}篇中海若关于河伯、海若、九州、宇宙的深层次思考就颇能说明这个道理。中心人物若自恃地缘或机缘优势而夜郎自大、不思进取,最终将逐渐失去优势,成为望洋兴叹的可笑角色;而边缘人物扬长避短,发挥自身优势,最终反而能跻身或开创某个中心。}

\lettrine{8.153} 王恭\myidx{王恭}始与王建武\myidx{王忱}甚有情\footnote{王恭:(?—398):孝武帝后兄,安帝舅父。与殷仲堪、桓玄等,二次兴兵清君侧,兵败被诛。会稽:郡治在今浙江绍兴市。王建武:王忱,官建武将军,(?—398):孝武帝后兄,安帝舅父。与殷仲堪、桓玄等,二次兴兵清君侧,兵败被诛。会稽:郡治在今浙江绍兴市。王忱是王恭同族长辈。},后遇袁悦之\myidx{袁悦}间\footnote{袁悦之:袁悦,字元礼,东晋陈郡阳夏(今河南太康)人。王恭与王忱友善,由于袁悦离间,王恭怀疑王忱谗害他,王忱又无以自明,二人遂失和。间:离间。},遂致疑隙\footnote{疑隙:因猜疑而产生的感情裂痕。}。{\fzxk\zihao{6}\textcolor{red}{\CJKunderwave{晋安帝纪}曰:“初,忱与族子恭少相善,齐声见称;及并登朝,俱为主相所待,内外始有不咸之论。恭独深忧之,乃告忱曰:‘悠悠之论,颇有异同,当由骠骑简于朝觐故也,将无从容切言之邪?若主相谐睦,吾徒得勠力明时,复何忧哉?’忱以为然。而虑弗见用,乃令袁悦具言之。悦每欲间恭,乃于三(王)坐嗔让恭曰:‘卿何妄生同异,疑误朝野!’其言切厉。恭虽惋怅,谓忱为构己也。忱虽心不负恭,而无以自亮。于是情好大离,而怨隙成矣。”}} 然每至兴会\footnote{兴会:兴致因有所感而被触发。},故有相思时\footnote{故:仍然,还。}。恭尝行散至京口射堂\footnote{行散:服五石散后散步调适,称“行散”。京口:今江苏镇江。 射堂:讲武演习射艺之所。},于时清露晨流,新桐初引\footnote{引:此谓抽芽,发芽。}。恭目之\footnote{目:品评。},曰:“王大\myidx{王忱}故自濯濯\footnote{王大:王忱,小字佛大。故自:确实。濯濯:鲜明而有光泽的样子。}。”

{\cangkai\zihao{5}【评】故事中王恭运用比兴之体,以清晨露水中刚抽嫩芽的梧桐喻王忱光鲜亮洁,意象优美生动,如在目前。又状王恭“每至兴会,故有相思时”,昔日友情时时涌上心头,令人感动。王恭、王忱原本情意绸缪,虽因小人挑拨乖谬情好,仍能奉行“君子绝交不出恶言”之义,为对方的气质才情发出由衷的赞叹。与常人之“爱之欲其生,恶之欲其死”(\CJKunderwave{论语·颜渊}),或爱之欲上青天,恨之欲堕地狱的两极做法,有本质差异,于细微处见出名士宽广的胸襟和脱俗的精神修养。凌濛初因而感叹曰:“疑隙而相思,后世亦往往有之。然未易能。”}

\lettrine{8.154} 司马太傅\myidx{司马道子}为二王\myidx{王恭}\myidx{王忱}目曰\footnote{司马太傅:司马道子,(364—403),简文帝子。孝武帝时,官太子太傅、扬州刺史、都督中外诸军事,宰执朝政。二王:王恭、王忱。目:品评。}:“孝伯亭亭直上\footnote{孝伯: 王恭, 字孝伯。亭亭: 耸立挺拔的样子。},阿大罗罗清疏\footnote{阿大:王忱,小字佛大。罗罗:疏朗放达的样子。清疏:爽朗疏放。}。”{\fzxk\zihao{6}\textcolor{red}{恭,正直亢烈;忱,通朗诞放。}}

{\cangkai\zihao{5}【评】王恭为王忱族子,二人情好绸缪如同兄弟,堪称佳话。故司马道子以二王并称。恭正直亢烈,深存节义,故有“亭亭直上”之目;忱任达不拘,放酒疏诞,恰合“罗罗清疏”之旨。王恭于司马道子为政敌,但道子仍为其气质发出由衷的叹美,于此见其风度。}

\lettrine{8.155} 王恭\myidx{王恭}有清辞简旨\footnote{王恭:(?—398):孝武帝后兄,安帝舅父。与殷仲堪、桓玄等,二次兴兵清君侧,兵败被诛。会稽:郡治在今浙江绍兴市。清辞简旨:清丽的言辞,简约的旨意。},能叙说而读书少,颇有重出\footnote{重出:重复出现。}。{\fzxk\zihao{6}\textcolor{red}{\CJKunderwave{中兴书}曰:“恭虽才不多,而清辩过人。”}} 有人道孝伯常有新意,不觉为烦。

{\cangkai\zihao{5}【评】读书之道在精不在多。晋人更注重删繁就简,由博返约。迂夫子满腹经纶,却舍本逐末、贪多不化,故鲜能有所创新,不过为世间多一两脚书橱而已;王恭读书虽少,而能清辞简旨,常有新意,当与其善于思考、提纲挈领的思维方法有关。这样的方法,与魏晋玄学以无为本、探本求源的玄学精神暗合。若方法正确,又加以博学之功,则学问将精进一层。}

\lettrine{8.156} 殷仲堪\myidx{殷仲堪}丧后\footnote{殷仲堪:(?—399):善清谈,当时与韩康伯齐名。},桓玄\myidx{桓玄}问仲文\myidx{殷仲文}\footnote{桓玄:(369—404),袭父温之爵南郡公,故称。安帝时任江州刺史、都督荆州八郡诸军事,率军东下,篡晋自立,建国号楚。旋被刘裕击败,斩首京师。杨广(?—399):曾官淮南太守,南蛮校尉,后与弟佺期俱被桓玄攻杀。殷荆州:指殷仲堪。仲文:殷仲文,仲堪从弟。}:“卿家仲堪,定是何似人\footnote{定:究竟。何似:怎样。}?”仲文曰:“虽不能休明一世\footnote{休明:美好清明。},足以映彻九泉\footnote{九泉:犹言黄泉。}。”{\fzxk\zihao{6}\textcolor{red}{\CJKunderwave{续晋阳秋}曰:“仲堪,仲文之从兄也,少有美誉。”}}

{\cangkai\zihao{5}【评】故事当发生在桓玄袭取荆州后得意之时。殷仲文为仲堪从弟,仲堪为桓玄袭杀,故仲文与桓玄有家族血仇。桓玄之问真假参半,骄恣、戏弄成分不可排除,仲文回答不卑不亢,恰到好处。仲堪兵败而身死人手,徒留英雄失路之悲;平生功业虽不及桓玄,然固是一时名士,身死之后,当以其立言不朽而光景常新;在当时人眼里,桓玄窃国大盗,虽立功业,但其“不朽”,乃是遗臭万年,并非千古流芳,用殷仲文语,即不能“映彻九泉”也。现代思想家胡适在古人之“三不朽”学说的基础上,加以发展创造,他认为:善言善行固永垂青史而不朽,而恶德恶行因烙刻在历史的耻辱柱上,也可以不朽。但在历史的天平上,二者一为正值,一为负值,并不具有等量齐观的意义。殷仲文评价一语而双关,与胡适之“新不朽观”相通。}







%%% Local Variables:
%%% mode: latex
%%% TeX-engine: xetex
%%% TeX-master: "../Main"
%%% End:
