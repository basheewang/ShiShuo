%% -*- coding: utf-8 -*-
%% Time-stamp: <Chen Wang: 2025-12-06 17:00:21>

% ○ ◎ ‧ 「 」 『 』 々 ( ) “ ” ■ ^[一-龥]
% 【\([^】][^】][^】]+\)】 → {\\fzxk\\zihao{6}\\textcolor{red}{\1}}
% \(【评】.*\) → {\\cangkai\\zihao{5}\1}
% \(【题解】.*\) → {\\cangkai\\zihao{5}\1}
% 《\([^》]+\)》 → \\CJKunderwave{\1}
% ^\([0-9]+.[0-9]+\) → \\lettrine{\1}
% {\\fzxk\\zihao{6}\\textcolor{red}{[^o]*}}


\setlength{\parindent}{0pt}

\part{下卷}

\chapter{容止第十四}


{\cangkai\zihao{5}【题解】 \CJKunderwave{容止}一门,顾名思义,指的是人的外在容貌和风度举止。外在容貌,原是父母遗传的天生资质,这是难以改变的,是一种静态的自然之美;而风度举止,则是可以通过人的学识修养来加以培植锻炼的动态之美。魏晋时代欣赏人的“容止”,一方面是通过外在的可见的容貌举止,来研究人体本身,并把人当作天地生存的小自然,进一步发现和探索人体之美;另一方面,又是由外而内,由有形转向无形的世界,来形象地揭示和展现魏晋士人那超凡脱俗的品格风度及精神气质。这是因为,外形之美是与人的内在才性资质密切相关,观察人的外在“容止”,就是架起了沟通与认识人的无形才性的桥梁。人的天生外貌虽然难以变换(按:现代的易容化妆则另当别论),但其喜怒哀乐和内在修养学识,却可以通过外貌作媒介来具体传达。因此,魏晋士大夫欣赏人的“容止”,不仅涉及人体的外貌形式之美,更重要的是展现其时代的精神风度。精神升华是魏晋士人欣赏“容止”之美的关键。这与今天娱乐圈中的选美活动,恐怕还是有所区别的。}

\lettrine{14.1} 魏武\myidx{曹操}将见匈奴使\footnote{魏武:指曹操。汉末建安年间,丞相曹操专擅朝政,汉献帝形同傀儡。但曹操终其一生,并未篡汉自代。其魏武帝之号,乃曹魏篡汉开国后追谥。匈奴:北方游牧民族。当时匈奴有南、北之别,南匈奴臣服于汉。曹操当政时,又分南匈奴为左、右、南、北、中五部。},自以形陋,不足雄远国\footnote{雄:作动词用,称雄,威服。},{\fzxk\zihao{6}\textcolor{red}{\CJKunderwave{魏氏春秋}曰:“武王姿貌短小,而神明英发。”}} 使崔季珪\myidx{崔琰}代\footnote{崔季珪:崔琰(?—216),字季珪。为人梗直敢言,后被谗自杀。},帝自捉刀立床头\footnote{帝:指曹操。捉刀:持刀。后引申为代人作文之义。床:坐榻。}。既毕,令间谍问曰:“魏王何如\footnote{魏王:指曹操。建安二十一年封魏王。}?”匈奴使答曰:“魏王雅望非常\footnote{雅望:美好的容仪。非常:不同寻常。}。{\fzxk\zihao{6}\textcolor{red}{\CJKunderwave{魏志}曰:“崔琰,字季珪,清河东武城人。声姿高畅,眉目疏朗,须长四尺,甚有威重。”}} 然床头捉刀人,此乃英雄也。”魏武闻之,追杀此使。

{\cangkai\zihao{5}【评】古代的政治家都很会演“戏”。譬如三国时的曹操、刘备和孙权这三个顶尖的人物,无不是优秀的“演员”。特别是曹操,青梅煮酒论英雄,更是把人生当戏台,演出了一幕幕有声有色的戏剧。在演“戏”中,政治家非常重视自我公众形象的塑造。这则故事,据\CJKunderwave{三国志·魏书·武帝纪},建安二十一年(216)五月,曹操进爵魏王,南匈奴来贺,同年,崔琰被谗杀。如果实有其事,则故事发生于该年崔琰被害之前。但唐代史论家刘知几因其事形同儿戏而疑其真伪,这属于历史问题,留待历史家考证。倒是文学批评家刘辰翁直视为“戏”,“谓追杀此使,乃小说常情”,评说得体。如果转换视角,从文学的艺术真实角度看问题,则故事虽短,但内涵丰富,其人物刻画生动,颇为符合曹操的性格与为人。曹操是乱世奸雄、治国能臣,有“宁我负人,毋人负我”的名言。他以美男帅哥崔琰代己接见外国使节,正是着意塑造自己高大美好形象的心理反应,虽然以假乱真,却是行之毫无愧怍之色。但当他了解匈奴使者见识过人,本领不小,是个人才又不为我用之时,从事业出发,在自己力量强盖匈奴之时,不顾邦交礼节而追杀此使。其流氓无赖手段令人咋舌,但其中又透露了他那雄才大略政治家的别样思考。}

\lettrine{14.2} 何平叔\myidx{何晏}美姿仪\footnote{何平叔:何晏,字平叔。魏正始时吏部尚书,又是当时清谈玄学家领袖。姿仪:姿貌容仪。},面至白。魏明帝\myidx{曹叡}疑其傅粉\footnote{魏明帝:讳叡,字元仲。但据\CJKunderwave{太平御览}诸书称引,“明帝”作“文帝”,指曹丕。另备一说。按:刘注谓晏“与帝相长”,则指文帝无疑。}。正夏月,与热汤䴵\footnote{热汤䴵:热汤面。䴵,同“饼”。}。既噉\footnote{噉:吃。},大汗出,以朱衣自拭,色转皎然\footnote{转:更加。皎然:洁白明亮。}。{\fzxk\zihao{6}\textcolor{red}{\CJKunderwave{魏略}曰:“晏性自喜,动静粉帛不去手,行步顾影。”按此言,则晏之妖丽本资外饰。且晏养自宫中,与帝相长,岂复疑其形姿,待验而明也?}}

{\cangkai\zihao{5}【评】何晏人生的中晚期,是曹魏晚期学界士林的领袖人物,因此,一贯注意自己的公众形象。“晏性自喜,动静粉帛不去手”,\CJKunderwave{魏略}所言,说明何晏对于自我修饰的重视。而修饰是给人看的,正是一种爱美心理的条件反射。魏明帝“疑其傅粉”,也透露出士夫贵族对于外貌修饰之美非常重视的信息,后来发展成为一代贵族男性青少年逐步女性化的一种妆饰。如东晋谢玄身佩香囊,就引起了叔父谢安的不满。当然,这个故事更说明人体自然之美令人歆羡,肤色洁白,面目皎然,也是一种重要的社会资本。其实不仅古代如此,今天的社会更是变本加厉。你看那流行歌星,个个都是俊男靓女,只要脸蛋漂亮,再加上娱记的炒作,无不成为青少年追星的偶像——至于其音乐艺术,则我辈大都不敢恭维。}

\lettrine{14.3} 魏明帝\myidx{曹叡}使后弟毛曾\myidx{毛曾}与夏侯玄\myidx{夏侯玄}共坐\footnote{毛曾:毛嘉子,河内人。因毛后暴贵,而出身寒门,故为士族所轻。夏侯玄(209—254):字太初,谯(今安徽亳县)人。官征西将军,后被司马集团诛杀。他是早期玄学清谈领袖之一。},时人谓“蒹葭倚玉树\footnote{蒹葭:蒹,荻。葭,芦苇。蒹葭泛指一般常见水草,这里借喻人之出身微贱。玉树:美好仙树,喻形貌秀出,才能出众。}。”{\fzxk\zihao{6}\textcolor{red}{\CJKunderwave{魏志}曰:“玄为黄门侍郎,与毛曾并坐,玄甚耻之,曾(不)说形于色。明帝恨之,左迁玄为羽林监。”}}

{\cangkai\zihao{5}【评】在曹魏时代,夏侯家族与曹氏家族,同为皇族出身,门第高华;而毛曾则出身微贱寒门,因毛后之故而成为政治上的暴发户,以此为士族所轻。为了提高小舅子的社会地位,明帝特命毛曾与夏侯玄同坐一席,史称“玄耻之,不说形之于色”,因为不给皇帝面子,所以“明帝恨之”,立即报复,把玄贬官。这说明实行九品中正制度之后,门阀士族势力正腾腾上升,甚至皇帝都无法控制。夏侯玄耻与毛曾“并坐”,除其门第寒微之外,还因毛曾言行举止的粗鄙,这与其所受的教育及内在学识修养直接相关。一旦与名门子弟“并坐”,立即喜形于色而有受宠若惊之感,其内在志趣之粗俗,可见一斑。玄耻与并坐,不亦宜乎!“蒹葭倚玉树”之喻,艺术对比,反差强烈,给人印象深刻;形象生动,言约旨远,而又意在言外,说明汲汲高攀,只有徒取其辱而已。}

\lettrine{14.4} 时人目夏侯太初\myidx{夏侯玄}“朗朗如日月之入怀”\footnote{目:品评。夏侯太初:即夏侯玄。朗朗:光明磊落的样子。},李安国\myidx{李丰}“颓唐如玉山之将崩”\footnote{颓唐:精神萎靡懒散的样子。}。{\fzxk\zihao{6}\textcolor{red}{\CJKunderwave{魏略}曰:“李丰字安国,卫尉李义子也。识别人物,海内注意。明帝得吴降人,问江东闻中国名士为谁?以安国对之。是时丰为黄门郎,改名宣。上问安国所在,左右公卿即具以丰对。上曰:‘丰名乃被于吴、越邪?’仕至中书令,为晋王所诛。”}}

{\cangkai\zihao{5}【评】汉末清议,发展到三国时代,其重点仍在人物品评方面,这与“汝南月旦”是一脉相承的。其所品目,大多运用意象思维的手法,重视象征比喻等修辞手法,这对于文学的语言艺术的发展,是有一定的促进作用。不过,在曹魏末期正始时代前后,因讨论才性问题的理论发展,有“四本论”的出现,成了后来魏晋玄学家清谈的重要题目之一。其中,李丰是“四本论”中持“才性异”的理论代表。经过正始玄家的发展,由人物清议而逐渐把重点转向理论思辨的清谈,等待的只是时间而已。}

\lettrine{14.5} 嵇康\myidx{嵇康}身长七尺八寸,风姿特秀\footnote{风姿:风度姿容。特秀:秀美拔俗。}。{\fzxk\zihao{6}\textcolor{red}{\CJKunderwave{康别传}曰:“康长七尺八寸,伟容色,土木形骸,不加饰厉而龙章凤姿,天质自然。正尔在群形之中,便自知非常之器。”}} 见者叹曰:“萧萧肃肃\footnote{萧萧肃肃:象声词,原指风声,借喻人之风度潇洒。},爽朗清举\footnote{爽朗清举:明朗爽快,清高飘逸。}。”或云:“肃肃如松下风,高而徐引\footnote{高而徐引:高远而绵绵不绝。}。”山公\myidx{山涛}曰:“嵇叔夜之为人也\footnote{嵇叔夜:指嵇康,字叔夜。},岩岩若孤松之独立\footnote{岩岩:高峻的样子。};其醉也,傀俄若玉山之将崩\footnote{傀(ɡuī归)俄:倾倒的样子。}。”

{\cangkai\zihao{5}【评】爱美是人之天性,女人爱美,男人也爱美。在魏晋风流人物中,嵇康是个公认的美男子。不仅是因其身长七尺八寸的魁梧身材,而且更重要的是他的“风姿特秀”,也就是其外貌之美,传达其内在的秀美风神。其潇洒风度,来自内在的品质及其学识修养,此所谓“不加饰而龙章凤姿,天质自然”,是内质的自然流露,来不得半点的装腔作势。这与今人的选美作秀是很不相同的。魏晋士人的审美意识,不仅重外貌,更重其内在品质之自然脱俗。山涛与嵇康同是竹林七贤的代表人物,他品评好友,认为嵇康是“岩岩若孤松之独立”,以高峻挺拔之孤松,象征其超凡脱俗而特立独行之人格;又以“傀俄若玉山之将崩”描绘其醉态,正见其对无拘无束生活自由的追求。山涛此评,可谓嵇康知音。}

\lettrine{14.6} 裴令公\myidx{裴楷}目王安丰\myidx{王戎}\footnote{裴令公:裴楷曾任中书令,故称。王安丰:王戎,封安丰侯。}:“眼烂烂如岩下电\footnote{烂烂:光明灿烂。岩下电:山岩下的闪电。}。”{\fzxk\zihao{6}\textcolor{red}{王戎形状短小,而目甚清炤,视日不眩。}}

{\cangkai\zihao{5}【评】仅从外貌看,王戎个子矮小,与嵇康相距甚远,当然与美男帅哥的称号无缘。但他忝列竹林七贤之末,想也并非一般人物,其目光炯炯有神如山间闪电,如今人之所谓“电眼”,同样令人叹美。俗话说,眼睛是心灵的窗户,王戎的目光眼神,自然流露其内在的机智与风流之性。这就弥补其外貌之不足而有幸登上\CJKunderwave{容止}门。外不掩内,这是魏晋士人的审美眼光。}

\lettrine{14.7} 潘岳\myidx{潘岳}妙有姿容,好神情\footnote{姿容:姿色容貌。神情:神态风情。}。{\fzxk\zihao{6}\textcolor{red}{\CJKunderwave{岳别传}曰:“岳姿容甚美,风仪闲畅。”}} 少时挟弹出洛阳道\footnote{挟弹:手持弹弓。洛阳:西晋都城。道:街道。},妇人遇者,莫不连手共萦之\footnote{连手:手拉手。萦:围绕。}。左太冲\myidx{左思}绝丑\footnote{左太冲:即左思。},{\fzxk\zihao{6}\textcolor{red}{\CJKunderwave{续文章志}曰:“思貌丑顇,不持仪饰。”}} 亦复效岳游遨\footnote{游遨:游玩。},于是群妪齐共乱唾之\footnote{妪:妇女的泛称。},委顿而返\footnote{委顿:萎靡狼狈的样子。}。{\fzxk\zihao{6}\textcolor{red}{\CJKunderwave{语林}曰:“安仁至美,每行,老妪以果掷之满车。张孟阳至丑,每行,小儿以瓦石投之,亦满车。”二说不同。}}

{\cangkai\zihao{5}【评】这则故事生动有趣,但在古代评论家中,却引起了一场争论。刘辰翁认为:“理不犯群妪,何至委顿?”也就是说,左思并没有触犯街道上的妇女观众,为什么会被她们唾口水而狼狈不堪呢?王世懋支持刘氏,发挥说:“太冲纵丑,未闻丑人必为群妪所唾,好事者之谈也。”但凌濛初则不同意上述见解,他转换视角,另出新见,说:“要之,借彼形此,不足多辩。”两种对立的批评,如从不同的角度看,各有合理的一面。刘、王二人是从现实生活中的情理角度看问题,人长得丑,这是父母天生,何罪之有?为什么丑人会遭到唾口水而委顿不振呢?可说绝无此理。但凌濛初则是从艺术审美的另一角度来分析,以丑来衬托美,形象对比鲜明,产生了强烈的审美效果。所评言简意赅,切中要害而启人思考。另外,在魏晋人的心目中,嵇康和潘岳都是令人欣赏的美男子,不过,前者具阳刚之美,后者则见阴柔之美。与嵇康相比,潘岳的美貌多少带点女性化的特点,于此可见魏晋士人审美情趣之所在。}

\lettrine{14.8} 王夷甫\myidx{王衍}容貌整丽\footnote{王夷甫:即王衍。整丽:端庄漂亮。},妙于谈玄\footnote{谈玄:玄学清谈。魏晋士人重在\CJKunderwave{老}、\CJKunderwave{庄}、\CJKunderwave{易}三玄之理。},恒捉白玉柄麈尾\footnote{麈(zhǔ主)尾:魏晋时助清谈的器具,形似羽扇,上圆下方,兼拂尘和扇子的功用,士人执持以示风雅。},与手都无分别。

{\cangkai\zihao{5}【评】皮肤白皙,是人体美受人欣赏注目的一个条件。魏晋贵族有傅粉的嗜好,说明了当时士人修饰增白的审美要求。其实,中外一理,据最近的报刊披露,古罗马的妇女常用含铅的增白膏粉末擦脸和皮肤,铅有毒,为了一时之美也顾不了许多。但同样是“白”,魏晋士人似乎更欣赏的是自然肤色,如何晏皮肤的光洁皎白。当然,天然洁白与人为修饰结合得天衣无缝,则是美上加美,达到理想境界,如本则故事所示。王衍之手的肤色,与白玉麈尾争相辉映,其肤色之洁白光亮,温润如玉,加上想象其执麈尾潇洒清谈的神采飞扬,其“整丽”容貌由内而外,富有活力而愈加动人。}

\lettrine{14.9} 潘安仁\myidx{潘安}、夏侯湛\myidx{夏侯湛}并有美容\footnote{潘安仁:即潘安。夏侯湛(243—291),字孝若。幼负盛才,颇富文学。},喜同行,时人谓之“连璧”\footnote{连璧:并体成双的玉璧,喻同样人物佳好。}。{\fzxk\zihao{6}\textcolor{red}{\CJKunderwave{八王故事}曰:“岳与湛箸契,故好同游。”}}

{\cangkai\zihao{5}【评】潘安是西晋一代的美男子,不仅吸引异性妇女环绕投果,更是同性崇拜偶像。夏侯湛之美貌,能与潘安并称“连璧”,则其姿质之美,当非凡响。其实,生活中美男甚多,为什么标榜潘安、夏侯呢?恐怕与其才华有关。在诗歌方面,潘(安)、陆(机)为太康之英;而夏侯则“颇窥六经之文,览百家之学”,曾独逍遥于养生,而雍容于艺文,其所著论,史称“别为一家之言”,都是少负盛名的文学才子,再加上他们的外在美好容貌,成为公众赞美的“明星”偶像,并非偶然。}

\lettrine{14.10} 裴令公\myidx{裴楷}有隽容姿\footnote{裴令公:指裴楷,曾任中书令,故称。隽:美好,出众。},一旦有疾,至困,惠帝\myidx{司马衷}使王夷甫\myidx{王衍}往看。裴方向壁卧\footnote{向壁:向壁卧床。},闻王使至\footnote{王使:皇帝派出的使者。},强回视之。王出,语人曰:“双眸闪闪若岩下电\footnote{双眸闪闪若岩下电:一作裴楷品评王戎语,见本门第六则故事。},精神挺动\footnote{精神挺动:旧注多训为倦怠、迟滞。但徐震堮\CJKunderwave{校笺}引\CJKunderwave{吕氏春秋·忠廉}“不足以挺其心矣”注,训“挺”为“动”,以为“精神挺动承上语来,下句乃另作转语”。其说可从。精神挺动,指目光眼神之灵动有风采。},体中故小恶\footnote{小恶:小病。恶,疾。}。”{\fzxk\zihao{6}\textcolor{red}{\CJKunderwave{名士传}曰:“楷病困,诏遣黄门郎王夷甫省之。楷回眸属夷甫云:‘竟未相识。’夷甫还,亦叹其神隽。”}}

{\cangkai\zihao{5}【评】西晋一代,裴楷也是有名的美男模范。这则故事说明裴楷容貌之美,主要在于眸子精光闪烁的眼神。患病之时,尚且目光炯炯有神,更何况是平时,其目光如电,一定更加摄人心魄。眼睛是人类心灵的窗户,写一眼神而光彩照人,形象尽出,其成功描绘,可资借鉴。}

\lettrine{14.11} 有人语王戎\myidx{王戎}曰:“嵇延祖\myidx{嵇绍}卓卓如野鹤之在鸡群\footnote{嵇延祖:即嵇绍,字延祖。卓卓:超拔特立的样子。}。”答曰:“君未见其父\myidx{嵇康}耳\footnote{其父:指绍父嵇康。}。”{\fzxk\zihao{6}\textcolor{red}{康,已见上。}}

{\cangkai\zihao{5}【评】“鹤立鸡群”的成语,即从这则故事中化出。在中国传统文化的意象批评中,鹤已成为一种具有高洁品格而特立独行、超凡脱俗的文化象征。闲云野鹤中的“野鹤”,尤为鹤群中之高洁者,其无拘无束、自由自在的生活更成为高人雅士追求的理想境界。在这方面,绍承家风,但不及乃父远甚,故王戎有此评,也是实事求是之言。嵇康的偶像作用,非嵇绍可替代。}

\lettrine{14.12} 裴令公\myidx{裴楷}有隽容仪,脱冠冕\footnote{冠冕:古代帝王、贵族及士大夫所穿戴的礼帽、礼服。},麤服乱头皆好\footnote{麤服乱头:粗劣的衣服,披着散乱的头发。麤,通“粗”字。},时人以为“玉人”。见者曰:“见裴叔则如玉山上行\footnote{裴叔则:即裴楷,字叔则,曾任中书令,故又称裴令公。},光映照人。”

{\cangkai\zihao{5}【评】本门第十则故事,表现的是裴楷的目光精神之美。这则故事,则重在其“麤服乱头”而毫不修饰的自然之美。裴楷是当时著名玄家,于此可见其“任自然”的生活。另外,古代之“玉”,别有文化内涵。\CJKunderwave{礼记·玉藻}称为“古之君子必佩玉”,“君子于玉比德焉”。玉之温润光洁,正是君子盛德的光辉外现。因此,作者由外貌而传达其内在风神,以“玉山”、“玉人”喻其自然姿质之美,实际上是给予很高的审美评价。}

\lettrine{14.13} 刘伶\myidx{刘伶}身长六尺,貌甚丑悴\footnote{丑悴(cuì萃):丑陋衰弱。},而悠悠忽忽\footnote{悠悠忽忽:形容如醉酒迷离而自由飘忽的样子。},土木形骸\footnote{土木形骸:身体如同土木一般质朴无华。}。{\fzxk\zihao{6}\textcolor{red}{梁祚\CJKunderwave{魏国统}曰:“刘伶,字伯伦。形貌丑陋,身长六尺,然肆意放荡,悠焉独畅,自得一时,常以宇宙为狭。”}}

{\cangkai\zihao{5}【评】刘伶是个身材矮小而形貌丑陋之人,与美男帅哥毫不搭界,但他却与矮小的王戎,同样有幸进入\CJKunderwave{容止}门。这是因为魏晋人赏美,不仅在“容”,还在于“止”——即其风度举止,用行动来表现其内在风神。刘伶神态举止是“悠悠忽忽,土木形骸”,这种无拘无束的自由自在,正见其竹林名士那“越名教而任自然”的美学情趣。当时玄学思潮对于审美观念的影响,于此可见一斑。}

\lettrine{14.14} 骠骑王武子\myidx{王济}是卫玠\myidx{卫玠}之舅\footnote{骠骑王武子:即王济,字武子,卒后追赠骠骑将军,故称。},隽爽有风姿\footnote{隽爽:隽杰豪爽。风姿:风度姿容。}。见玠辄叹曰:“珠玉在侧,觉我形秽。”{\fzxk\zihao{6}\textcolor{red}{\CJKunderwave{玠别传}曰:“骠骑王济,玠之舅也。尝与同游,语人曰:‘昨日吾与外生共坐,若明珠之在侧,朗然来照人。’”}}

{\cangkai\zihao{5}【评】在\CJKunderwave{容止}门中,第14、16和19等三则故事的主人公都是卫玠。在两晋之交,若论容貌风度而选美男,卫玠当获桂冠无疑。这则故事,并不正面描绘卫玠之美,而是通过其舅父来加映衬。王济其人,在两晋是个有名的风流人物,史称其“风姿英爽,气盖一时,好弓马,勇力绝人”,是个有阳刚气质的美男子。其性颇为狂妄,很少看得起人。由他自己来作比照,更可见出外甥之美,难有比拟。其所形容,不仅可信,而且更见卫玠的照人光彩。后来,“自惭形秽”的成语,由此演化而来。}

\lettrine{14.15} 有人诣王太尉\myidx{王衍}\footnote{王太尉:指王衍。},遇安丰\myidx{王戎}、大将军\myidx{王敦}、丞相\myidx{王导}在坐\footnote{安丰:指安丰侯王戎。大将军:指王敦。丞相:指王导。};往别屋,见季胤\myidx{王诩}、平子\myidx{王澄}\footnote{季胤:王诩。平子:王澄。按:诩、澄俱为衍弟。}。{\fzxk\zihao{6}\textcolor{red}{石崇\CJKunderwave{金谷诗叙}曰:“王诩,字季胤,琅邪人。”\CJKunderwave{王氏谱}曰:“诩,夷甫弟也。仕至修武县令。”}} 还,语人曰:“今日之行,触目见琳琅珠玉\footnote{琳琅:美玉。}。”

{\cangkai\zihao{5}【评】魏晋人喜欢用“珠玉”来形容人体美,不仅因其作为装饰之物,价值昂贵,更在其质地温润,晶莹透彻,精光迷人。其外貌之美,肇自内质之秀。这则故事,重在描绘西晋时琅邪王家精英,看来王家贵少之美,自有遗传基因,后来如东晋的王羲之、献之父子亦然。外貌离不开风度,而风度则是内在神气的外现。}

\lettrine{14.16} 王丞相\myidx{王导}见卫洗马\myidx{卫玠}\footnote{王丞相:指王导。卫洗马:指卫玠,曾任太子洗马。},曰:“居然有羸形\footnote{居然:显然。羸(leí雷)弱:瘦弱的身体。},虽复终日调畅\footnote{调畅:调和舒畅。},若不堪罗绮\footnote{不堪罗绮:无法承受丝质罗衣之轻。罗绮,质地轻柔的丝织品。}。”{\fzxk\zihao{6}\textcolor{red}{\CJKunderwave{玠别传}曰:“玠素抱羸疾。”\CJKunderwave{西京赋}曰:“始徐进而羸形,似不胜乎罗绮。”}}

{\cangkai\zihao{5}【评】本则故事,写的仍是人们眼中的卫玠之美。卫玠以其瘦弱而不胜罗衣之轻的阴柔之美,成为一种魏晋贵族男子愈趋女性化的象征。但王导乃一代名士,所言卫玠“终日调畅”云云,却透露其精神风度之一斑。史称卫玠“好言玄理”,时与亲友清谈,听者“无不咨嗟”。内质充裕温润,更增其珠玉琳琅之光鲜。本书\CJKunderwave{品藻}门第42则描绘“卫虎(玠)奕奕神气”,同样是从其“神清”方面来见其容止之美。}

\lettrine{14.17} 王大将军\myidx{王敦}称太尉\myidx{王衍}\footnote{王大将军:指王敦。太尉:指王衍。称:称誉,赞美。}:“处众人中,似珠玉在瓦石间。”

{\cangkai\zihao{5}【评】王衍也是晋时的一个美男子,史称其“神情明秀,风姿详雅”,主要就其精神风度而言;但论其外貌,当也不差。故其童年见山涛,山涛叹美曰:“何物老妪,生此宁馨儿!”此则故事,通过王敦之口,加以称美,以珠玉与瓦石作对比,形成强烈艺术反差,令人想象其神明风度之脱俗超群。}

\lettrine{14.18} 庾子嵩\myidx{庾敳}长不满七尺\footnote{庾子嵩:即庾敳。好\CJKunderwave{老}\CJKunderwave{庄},喜清谈玄理,与王衍齐名,为士林所重。尺:晋时一尺相当于今天的24.2厘米。其七尺相当于现在的1.69米的中矮个子。},腰带十围\footnote{围:古代一种模糊统计的长度单位,据称一围近于今之五寸。十围极言其腰围之肥大。},颓然自放\footnote{颓然自放:形容精神委散、自由舒展。}。

{\cangkai\zihao{5}【评】魏晋之人,以身材魁梧为美,但也并不是绝对以形取人。庾敳貌不惊人,甚至从身材来说,其矮胖之躯,可说比例失调。但因其“雅有远韵”的风度神情,便可掩其“腰带十围”的外形之丑,而入于令人歆羡的\CJKunderwave{容止}门,成为审美对象。“容止”之动静变化,其审美标准与内在风姿神态直接相关,而非仅是静态的外貌须眉身材之美而已。}

\lettrine{14.19} 卫玠\myidx{卫玠}从豫章至下都\footnote{卫玠:见前注。豫章:郡名,治所在今江西南昌市。下都:与西晋京师洛阳相比,称东晋首都建康为下都。},人久闻其名,观者如堵墙\footnote{观者如堵墙:形容围观的人群密集犹如树了一堵墙壁。后世“观者如堵”的成语,即由此演化而来。}。玠先有羸疾\footnote{羸疾:瘦弱多病。},体不堪劳,遂成病而死,时人谓看杀卫玠。{\fzxk\zihao{6}\textcolor{red}{\CJKunderwave{玠别传}曰:“玠在群伍之中,实有异人之望。龆龀时,乘白羊车于洛阳市上,咸曰:‘谁家璧人?’于是家门州党号为璧人。”案:\CJKunderwave{永嘉流人名}曰:“玠以永嘉六年五月六日至豫章,其年六月二十日卒。”此则玠之南度豫章四十五日,岂暇至下都而亡乎?且诸书皆云玠亡在豫章,而不云在下都也。}}

{\cangkai\zihao{5}【评】看杀卫玠,以形容其美貌对人的强大魅力,这是修辞艺术的夸张说法。事实是,卫玠身材一贯瘦弱,体质很差,平日娇生惯养,犹如温室里的花朵,虽然很美,却难抗风暴摧残。在永嘉大动乱中,为求生存,他必须奉母南渡,千里奔波,因而不胜劳累“成病而死”,当在情理之中。他是当朝名士,一代美男“冠军”,名声很大,因此沿途围观的群众很多,也可能是事实。于此可见魏晋士人的唯美主义倾向,从欣赏阳刚之美的壮伟大丈夫,逐渐转向欣赏具有娇嫩软媚女性化的贵族男性的天平增重分量。这一审美情趣的微妙变化,随着东晋士族的活跃,愈趋明显。这与贵族当日生活方式及士人心态密切相关。另:中古是以男性为中心的社会,因此\CJKunderwave{世说}中几乎难见对于女性之美的直接描绘,但通过男人审美的女性化倾向,多少传达了若干信息。男人弱不禁风、不胜罗衣之轻,成为美的偶像,可能就与变相的异性相吸的生理及心理需求有关。}

\lettrine{14.20} 周伯仁\myidx{周顗}道\footnote{周伯仁:周顗,字伯仁。}:“桓茂伦\myidx{桓彝}嵚崎历落\footnote{桓茂伦:桓彝,字茂伦。嵚崎(qīn qí钦奇):山石高峻貌。历落:即磊落。},可笑人\footnote{可笑人:可爱的人。但余嘉锡引李治\CJKunderwave{敬斋古今顗}四曰:“盖顗谓彝为人不群,世多忽之,所以见笑于人耳!”别作一解。}。”或云谢幼舆言\myidx{谢鲲}\footnote{谢幼舆:谢鲲,字幼舆。}。

{\cangkai\zihao{5}【评】在人体美欣赏中,魏晋士人逐步完善了神内形外的理论观念,并逐步加重了“传神”的审美分量。史称桓彝“少孤贫,随箪瓢,处之晏如”,性格通达,早获盛名。苏峻叛乱时,任宣城内史,部下劝其自保,但他坚决抗战而视死如归,厉色慷慨而言:“今社稷危逼,义无晏安。”其“嵚崎历落”的坦荡神态,历历如画,故下述“可笑人”,必非贬词,而当属褒义而谓其如此可爱也。}

\lettrine{14.21} 周侯\myidx{周顗}说王长史\myidx{王濛}父\footnote{周侯:指周顗,弱冠,袭父浚爵为武城侯,故称。王长史:指王濛。据\CJKunderwave{晋书·外戚}濛传,濛曾祖黯,祖佑,父讷,与刘注引\CJKunderwave{王氏谱}不同。}:{\fzxk\zihao{6}\textcolor{red}{\CJKunderwave{王氏谱}曰:“讷字文渊(开),太原人。〔曾〕祖默,尚书。父祐,散骑常侍。顗始过江,仕至新淦令。”}} 形貌既伟\footnote{伟:魁梧伟壮。},雅怀有概\footnote{雅怀有概:胸怀高雅有气度节概。},保而用之,可作诸许物也\footnote{诸许物:诸多事情。}。

{\cangkai\zihao{5}【评】王濛之父讷,字文开。其形貌伟壮丈夫,雅怀有概,当属阳刚美男之列。人称如能保持发挥其形、貌方面的优势,则可成就诸多事情。于此可见“容止”的社会作用不可小觑。今日招聘,许多公司企业为公关需要,对俊男靓女情有独钟,其“容止”占了不少便宜。这在\CJKunderwave{世说}时代,早开其端。王讷形貌之美,可由其子王濛之言见其端倪。史称王濛“美姿容,尝览镜自照,称其父字曰:‘王文开生如此儿邪!’”其美虽后来居上,但正见其得之遗传。}

\lettrine{14.22} 祖士少\myidx{祖约}见卫君长\myidx{卫永}\footnote{祖士少:祖约,字士少。祖逖之弟。卫君长:卫永,字君长。},云:“此人有旄杖下形\footnote{旄杖:旄节仪仗。旄节为将帅信物。}。”

{\cangkai\zihao{5}【评】言外之意,见出卫永具有拥旄节雄视一方的将帅之风度气象。}

\lettrine{14.23} 石头事故\footnote{石头事故:咸和二年(327),因不满庾亮等专擅朝政,历阳太守苏峻发动叛乱,兵陷京师建康,迁晋成帝于石头城软禁。后苏峻叛乱被陶侃、温峤等击灭。},朝廷倾覆,{\fzxk\zihao{6}\textcolor{red}{\CJKunderwave{晋阳秋}曰:“苏峻自姑熟至于石头,逼迁天子。峻以仓屋为宫,使人守卫。”\CJKunderwave{灵鬼志·谣征}曰:“明帝末,有谣歌曰:‘侧力,放马出山侧,大马死,小马饿。’后峻迁帝于石头,御膳不具。”}} 温忠武\myidx{温峤}与庾文康\myidx{庾亮}投陶公云\myidx{陶侃}\footnote{温忠武:指温峤,时任江州刺史。卒谥忠武,故称。庾文康:指庾亮,卒谥文康。他当时是辅政大臣。陶公:指陶侃,时任荆州刺史,掌控长江上游雄兵。按:“陶公”下袁本有“求救,陶公”四字,于义较合。}:“肃祖\myidx{司马绍}顾命不见及\footnote{肃祖:指晋明帝司马绍,驾崩后庙号肃祖。顾命:临终遗命,指皇帝遗诏。不见及:没有提到我。},且苏峻\myidx{苏峻}作乱,衅由诸庾\footnote{衅:衅乱,祸乱,引申为罪责。诸庾:指庾亮、庾冰诸掌朝政之人。},诛其兄弟,不足以谢天下。”{\fzxk\zihao{6}\textcolor{red}{徐广\CJKunderwave{晋纪}曰:“肃祖遗诏,庾亮、王导辅幼主而进大臣官,陶侃、祖约不在其例。侃、约疑亮寝遗诏也。”\CJKunderwave{中兴书}曰:“初,庾亮欲征苏峻,卞壸不许。温峤及三吴欲起兵卫帝室,亮不听,下制曰:‘妄起兵者诛!’故峻得作乱京邑也。”}} 于时庾在温船后闻之,忧怖无计。别日,温劝庾见陶,庾犹豫未能往,温曰:“溪狗我所悉\footnote{溪狗:对陶侃的蔑称。当时中原士人称江西人为“傒”,“溪狗”即“傒狗”,因陶侃为豫章郡人,出身寒门,故云。},卿但见之\footnote{但:尽管。},必无忧也。”庾风姿神貌,陶一见便改观,谈宴竟日\footnote{谈宴:宴饮畅谈。竟日:一整天。},爱重顿至\footnote{顿至:立刻,一下子。}。

{\cangkai\zihao{5}【评】这则故事,发生于咸和二年(327),写得非常生动,无论是叙事,或是对话,以及心理刻画,都很成功。虽是写实,但论其艺术,却犹如一篇动人的小小说。在平定苏峻叛乱以复兴国家朝廷的重大事件中,陶侃、温峤和庾亮,都是重要角色,在故事中形象无不栩栩如生。以人物语言为例,陶侃之言,一方面显露其内心压抑已久的气愤,对于朝廷的不公与腐败,进行猛烈的抨击,所以加大政治筹码,以便压制诸庾。诛之是假,迫使诸庾就范、以便朝廷改正错误是真。因而所言义形于色,充分显示了正义与自信。温峤说话,也是洞彻对方心理的智慧之言,同时又具有鲜明的个性特征。对于陶侃,出于高门士族的门第偏见,称之为“溪狗”,但同时又有求于他,后来又推陶侃为联军盟主(统帅),他判断陶侃对于朝廷国家的忠心,因而有“必无忧也”的明确判断。至于庾亮,因其“风姿神貌”而顿令陶侃泯灭恩怨,看似偶然,其实是他们有共同政治目标所致,庾亮风神,起了一个治病处方之药引的作用,同样不可小觑。}

\lettrine{14.24} 庾太尉\myidx{庾亮}在武昌\footnote{庾太尉:指庾亮,卒赠太尉,故称。在平定苏峻乱后,出京任江、荆、豫三州刺史,镇武昌。},秋夜气佳景清,佐吏殷浩\myidx{殷浩}、王胡之\myidx{王胡之}之徒登南楼理咏\footnote{南楼:武昌楼名。理咏:调理吟咏。理,治也。}。音调始遒\footnote{遒:刚劲,指音调高亢。},闻函道中有屐声甚厉\footnote{函道:楼梯。屐:底下有齿的木鞋。厉:急促。},定是庾公\footnote{庾公:指庾亮。}。俄而率左右十许人步来\footnote{俄:一会儿。十许人:十馀人。},诸贤欲起避之。公徐云:“诸君少住,老子于此处兴复不浅\footnote{老子:犹“老夫”,年纪大的人常作第一人称代词的“我”来用。少住:稍作停留。}!”因便据胡床\footnote{据:靠、坐。胡床:从西域胡地传进来的轻便交椅。},与诸人咏谑\footnote{咏谑:歌咏笑谑。},竟坐甚得任乐\footnote{竟坐:终坐,从头坐到最后一刻。任乐:自由欢乐。}。后王逸少\myidx{王羲之}下与丞相\myidx{王导}言及此事\footnote{王逸少:王羲之,字逸少。下:从上游顺流到下游之地,京师建康在武昌下游,故云。丞相:指王导。},丞相曰:“元规\myidx{庾亮}尔时风范\footnote{元规:庾亮,字元规。风范:风度规范。},不得不小颓\footnote{小颓:小损。}。”右军答曰:“唯丘壑独存\footnote{右军:指王羲之,他曾任右军将军,故称。丘壑:泛指山水。}。”{\fzxk\zihao{6}\textcolor{red}{孙绰\CJKunderwave{庾亮碑文}曰:“公雅好所托,常任尘垢之外。虽柔心应世,蠖屈其迹,而方寸湛然,固以玄对山水。”}}

{\cangkai\zihao{5}【评】这是一篇成功的叙事“小小说”。因与琅邪王导在朝中的矛盾,为鄢陵庾氏家族长远利益计,掌控长江上游的强大武装力量是一种重要的权力保障。因而陶侃死后,庾亮代其镇武昌,任征西将军,江、荆、豫三州刺史,时在咸和九年(334),其人生已进入晚年。庾亮是个儒玄双修的人物。史称其“美姿容,善谈论,性好\CJKunderwave{老}\CJKunderwave{庄},风格峻整,动由礼节,闺门之内,不肃而成”。但他在入朝取代王导执政之时,一改王导“宽和得众”的宽松团结政策,变成“任法裁物”的严整之政,因此颇失人心。当时庾氏家族挟外戚帝室之威,逐渐取代琅邪王氏家族的地位,以权势风范自尊。故\CJKunderwave{晋阳秋}曰:“亮端拱巍然,郡人惮之,觐接者数人而已。”端足了严肃架势,成为礼仪标本。但在苏峻叛乱之后,庾氏家族作为政权支柱,遭受挫折,出为江、荆、豫三州刺史。虽仍控重兵而实权在握,但挫折使人变得聪明。庾亮风范“小颓”,转向“丘壑独存”的玄家理趣追求,因而能够与众同乐,表现了“玄对山水”的审美情趣,故其精神风度,转近脱俗之自然,此所以能使大名鼎鼎的风流名士王羲之为之折服,正见其不同凡近的公众形象光彩。}

\lettrine{14.25} 王敬豫\myidx{王恬}有美形\footnote{王敬豫:王导次子王恬,字敬豫。},问讯王公\myidx{王导}\footnote{问讯:问安,问候。王公:指王导。按:“王公”下袁本重“王公”二字,是。“抚其肩”者,乃王公也。},抚其肩,曰:“阿奴\footnote{阿奴:长辈对小辈的昵称。此为父称子。},恨才不称。”又云\footnote{又云:朱铸禹\CJKunderwave{汇校集注}云:“此为临川(按:指刘义庆)附记当时人之评论。”李慈铭按:“‘又云’字有误,上文乃导自谓其子之语,下不得作‘又云’也,当是他人品目之语。”}:“敬豫事事似王公。”{\fzxk\zihao{6}\textcolor{red}{\CJKunderwave{语林}曰:“谢公云:‘小时在殿廷会见丞相,便觉清风来拂人。’”}}

{\cangkai\zihao{5}【评】魏晋人心目中,光有外貌“美形”,还是不够的。王导对其次子王恬(敬豫),常会生气不满意,就是因他有一副美丽仪容,但却尚武少文,不喜欢学习,因其内在的才华欠缺而与外貌“美形”不相称副,此王导所以生“嗔”而常感遗憾。参见\CJKunderwave{德行}第29则故事。至于王恬“事事似王公”,处处模仿乃父,缺乏创造与个性,也是王导生“嗔”的原因之一。处处似人,则“我”何在哉?缺乏自我个性,当然无法潇洒、闲逸与超拔,这当然为魏晋士人所不屑。}

\lettrine{14.26} 王右军\myidx{王羲之}见杜弘治\myidx{杜乂}\footnote{王右军:即王羲之。杜弘治:杜乂,字弘治。杜预之孙。},叹曰:“面如凝脂\footnote{凝脂:凝冻的油脂,喻其光滑洁白。},眼如点漆\footnote{点漆:眼睛瞳仁又黑又亮。},此神仙中人。”{\fzxk\zihao{6}\textcolor{red}{\CJKunderwave{江左名士传}曰:“永和中,刘真长、谢仁祖共商略中朝人士,或曰:杜弘治清标令上,为后来之美;又面如凝脂,眼如点漆,粗可得方诸卫玠。”}} 时人有称王长史\myidx{王濛}形者\footnote{王长史:指王濛,曾为司徒左长史,故称。形:外貌。按:此称其美姿容。},蔡公\myidx{蔡谟}曰\footnote{蔡公:对蔡谟的尊称。}:“恨诸人不见杜弘治耳。”

{\cangkai\zihao{5}【评】“面如凝脂,眼如点漆”,亲切可见,令人神往。读\CJKunderwave{诗经·卫风·硕人},则知其美之所自。凝脂以喻面色洁白温润,娇嫩无比,吹弹即破;而点漆形容眼睛黑白分明,顾盼生姿,神采奕奕。二句言简意赅,绘出了人体美的形神俱在。杜乂是当时堪与卫玠媲美而各有特点的美男子。史称其“性纯和,美姿容,有盛名于江左”。又据前\CJKunderwave{赏誉}门第68则,庾亮称“弘治至羸,不可以致哀”,则其姿貌仪表,与卫玠同样属阴柔之美的典型。}

\lettrine{14.27} 刘尹\myidx{刘惔}道桓公\myidx{桓温}\footnote{刘尹:指刘惔,曾官丹阳尹,故称。他是东晋中期的清谈领袖人物之一。}:鬓如反猬皮\footnote{鬓如反猬皮:形容其鬓毛须发刚硬竖起,如反转的刺猬皮。},眉如紫石稜\footnote{眉如紫石稜:双眉如紫石英的棱角刚健。稜,同“棱”。按:朱铸禹\CJKunderwave{汇校集注}引\CJKunderwave{晋书}温传,以为“眉”当作“眼”,另备一说参考。},自是孙仲谋、司马宣王一流人\footnote{自:本来。孙仲谋:指三国吴主孙权,字仲谋。司马宣王:指司马懿。}。{\fzxk\zihao{6}\textcolor{red}{宋明帝\CJKunderwave{文章志}曰:“温为温峤所赏,故名温。”\CJKunderwave{吴志}曰:“孙权字仲谋,策弟也。汉使者刘琬语人曰:‘吾观孙氏兄弟,虽并有才秀明达,皆禄祚不终。唯中弟孝廉,形貌魁伟,骨体不恒,有大贵之表。’”\CJKunderwave{晋阳秋}曰:“宣王天姿杰迈,有英雄之略。”}}

{\cangkai\zihao{5}【评】“反猥皮”、“紫石稜”云云,描绘桓温相貌奇伟,不可以寻常观之,颇有一代枭雄之相。史称其“姿貌甚伟”,大致不差;但“面有七星”,则是夸饰之词。其相貌当以\CJKunderwave{世说}为是,属于“豪爽有风概”的阳刚丈夫。如是外貌丑陋,则不可能被皇帝看上,选尚南康公主。年轻时的桓温,毫无权势可言,作为驸马,除家庭背景和内在修养外,当然也要有一定的外貌条件。}

\lettrine{14.28} 王敬伦\myidx{王劭}风姿似父\footnote{王敬伦:王劭,字敬伦,王导第五子。},作侍中\footnote{作侍中:事实疑误,劭未曾作侍中,而时桓温作侍中、太尉。按:疑“作侍中”前脱漏“桓公”二字。},加授桓公\myidx{桓温}公服\footnote{加授桓公公服:史称桓温授太尉时,固让,旬月之中,使者八至。当时官阶不同,则服饰有异。公服,此指太尉官阶的上朝官服,太尉在晋是三公之一。桓公,指桓温。},从大门入,桓公望之,曰:“大奴固自有凤毛\footnote{大奴:对王劭的昵称。凤毛:六朝人习惯称儿子似乃父者为凤毛,即得老凤之羽毛也。}。”{\fzxk\zihao{6}\textcolor{red}{大奴,王劭也,已见。\CJKunderwave{中兴书}曰:“劭美姿容,持仪也。”}}

{\cangkai\zihao{5}【评】故事发生在永和八年(352)。琅邪王家,特别是王导父子,史多有“美姿容”之称誉,或与遗传基因有关。但桓温“凤毛”之赏,则重在精神风度,与其家族文化传统的长期积淀有关,而非一朝一夕之功。如前述魏明帝妻舅毛曾,就是个暴发户,其粗俗可鄙之性,也是长期积累所致,而非装腔作势可加掩盖。习惯成自然,风度气质的改变是长期的事。}

\lettrine{14.29} 林公\myidx{支遁}道王长史\myidx{王濛}\footnote{林公:即支遁,字道林,东晋名僧。道:品评。王长史:指王濛。},敛衿作一来\footnote{敛衿:整饬衣襟,表示恭敬。作一来:有所动作时。},何其轩轩韶举\footnote{何其:多么。轩轩:轩昂,气宇非凡。韶举:美好举止。}!{\fzxk\zihao{6}\textcolor{red}{\CJKunderwave{书(语)林}曰:“吾(王)仲祖有好仪形,每览镜自照,曰:‘王文开那生如馨儿!’时人谓之达也。”}}

{\cangkai\zihao{5}【评】这是对于王濛神情风度的赞美。王濛照镜而自言自语:“王文开(其父讷)那生如馨儿!”直呼父名,自叹自怜。真是脱尽俗气,而自然可爱,形象地见魏晋风度之一斑。}

\lettrine{14.30} 时人目王右军\myidx{王羲之}\footnote{目:目品,品评。王右军:即王羲之。}:“飘如游云,矫若惊龙\footnote{矫:矫健。惊:迅疾貌。}。”

{\cangkai\zihao{5}【评】\CJKunderwave{晋书}本传以此语品评王羲之书法艺术,称其“尤善隶书,为古今之冠,论者称其笔势以为飘若浮云,矫若惊龙”。后人或以为“与此状其容止者不同”。其实,人们常说“文如其人”、“诗如其人”,书法艺术何尝不是如此。对王羲之来说,书法艺术与其为人的风度精神,早已融入自然而合二为一。游云(浮云)在天,清风徐拂,飘荡漫无目标,以喻无拘无束的自由自在。矫若惊龙,则由慢转快,由浮云之虚,转为惊龙之实,夭矫雄健,见其力量,迅速变化,不可端倪而又焕若神明。人乎?书法乎?相得益彰。}

\lettrine{14.31} 王长史\myidx{王濛}尝病\footnote{王长史:指王濛。},亲疏不通\footnote{亲疏不通:拒绝亲友探视问候。}。林公\myidx{支遁}来\footnote{林公:支道林,东晋名僧。},守门人遽启之\footnote{遽(jù巨):急忙。},曰:“一异人在门\footnote{异人:不同寻常之人。},不敢不启。”王笑曰:“此必林公。”{\fzxk\zihao{6}\textcolor{red}{案:\CJKunderwave{语林}曰:“诸人尝要阮光禄共诣林公,阮曰:‘欲闻其言,恶见其面。’”此则林公之形,信当丑异。}}

{\cangkai\zihao{5}【评】王濛拒绝了一切亲友的探望,以便安心养病。但对林公,却因其为“异人”,破例接待。其所以“异”,不仅因其形貌不同寻常的丑陋,而更在于其神明气概方面之“异”——即超凡脱俗的名士风度,从而化丑为美,令人心向往之。}

\lettrine{14.32} 或以方谢仁祖\myidx{谢尚}\footnote{方:比拟。谢仁祖:谢尚,字仁祖。},不乃重者\footnote{不乃重者:不太重视。乃,犹甚。余嘉锡\CJKunderwave{笺疏}曰:“言有比人为谢尚者,其意乃实轻之。若曰‘某不过谢仁祖之流耳’。”}。桓大司马\myidx{桓温}曰\footnote{桓大司马:指桓温,曾官大司马,故称。}:“诸君莫轻道\footnote{道:论议,品评。莫轻道,即不要轻易地说三道四。},仁祖企脚北窗下弹琵琶\footnote{企脚:踮脚,跷脚。},故自有天际真人想\footnote{真人:道教中不食人间烟火的神仙一流人物。}。”{\fzxk\zihao{6}\textcolor{red}{\CJKunderwave{晋阳秋}曰:“尚善音乐。”\CJKunderwave{裴子}云:“丞相尝曰:‘坚石挈脚枕琵琶,有天际想。’”坚石,尚小名。}}

{\cangkai\zihao{5}【评】谢尚是谢安的堂兄,东晋陈郡阳夏谢氏家族的代表人物。不过,自其父鲲死后,谢氏家族势力中衰,因此有轻忽之者,或因政治权势不足之故。实际上,史称尚年轻时,“开率颖秀,辨语绝伦,脱略细行,不为流俗之事”,加以“善音乐,博综众艺”,自是不同凡响之人。因此深为王导赏识。在佳集胜会中,导邀其即席作\CJKunderwave{鸲鹆舞},“尚俯仰在中,傍若无人”。晋士之艺术人生,大抵与玄家那越名教而任自然的观念相关,故其风度神情率诣如此。其企脚北窗弹琵琶,令桓温有“天际真人想”之叹,完全可以想象。}

\lettrine{14.33} 王长史\myidx{王濛}为中书郎\footnote{王长史:指王濛。中书郎:中书省官名,即中书侍郎或郎中。},往敬和\myidx{王洽}许\footnote{敬和:王洽。王导第三子。史称“导诸子中最知名,与荀羡俱有美称”。许:处所,地方。}。{\fzxk\zihao{6}\textcolor{red}{敬和,王洽。已见。}} 尔时积雪,长史从门外下车,步入尚书省\footnote{尚书省:朝廷官署,总理政务。}。敬和遥望叹曰:“此不复似世中人\footnote{世中:人世之中。}。”

{\cangkai\zihao{5}【评】王濛容仪之美,已见前述。这则故事,并不正面实写王濛容貌,而只从他人眼中见出,加以侧笔描绘而已。作者聪明地运用了虚实相生的传统手法,给读者留下了大片的艺术空白,以驰骋其丰富的艺术想象。故事大约发生在晋康帝建元年间(343—344)王濛任中书郎时。当时王洽是个二十馀岁而富有想象力的青年,他的由衷叹美:“此不复似世中人!”令人对于王濛那超世拔俗、闲逸潇洒的风神气度,产生了真切的向往,从而留下了美好而深刻的印象。}

\lettrine{14.34} 简文\myidx{司马昱}作相王时\footnote{简文:指晋简文帝司马昱。相王:司马昱即位前封会稽王,于太和元年(366)为丞相、录尚书事,故称。},与谢公\myidx{谢安}共诣桓宣武\myidx{桓温}\footnote{谢公:指谢安。桓宣武:指桓温,死后谥号宣武。}。王珣\myidx{王珣}先在内\footnote{王珣:祖导,父洽,谢安侄婿。弱冠入桓温幕府。},桓语王:“卿尝欲见相王,可住帐里。”二客既去,桓谓王曰:“定何如\footnote{定:究竟。}?”王曰:“相王作辅\footnote{辅:辅政大臣。},自湛若神君\footnote{湛:清明澄澈的样子。神君:言人贤明如神。}。{\fzxk\zihao{6}\textcolor{red}{\CJKunderwave{续晋阳秋}曰:“帝美风姿,举止安详。”}} 公亦万夫之望,不如,仆射何得自没\footnote{自没:自我埋没。}?”{\fzxk\zihao{6}\textcolor{red}{仆射,谢安。}}

{\cangkai\zihao{5}【评】故事发生在太和四年(369),年轻的王珣进入桓温幕府。当时桓温东征西讨,南北进剿,声威赫赫,而野心勃勃,正是权势熏天而废立自专之时。王珣所评论的相王司马昱及桓温诸人,都是东晋权力金字塔上顶尖的人物。王世懋评王珣之言曰:“此东亭(王珣)媚语,安石恐未肯便没。”批评了王珣谄媚桓温的言行,并指出谢安亦是一代伟杰,岂肯在桓温面前自甘认输而埋没沉沦呢?其实,王询言外之意,王世懋没有读懂。刘孝标以为仆射指谢安,误。当时谢安任侍中,而仆射是珣之族叔王彪之。桓温因不满王彪之,曾于兴宁三年(365)劾罢之,后遇赦复职不久。因此,王珣一方面巴结桓温,称其“万夫之望”,百姓“救星”;一方面又借机为族叔彪之开脱,如朱铸禹\CJKunderwave{汇校集注}所云:“王珣此言盖正以彪之未从简文行,巽言以解其被劾之前嫌,不仅取媚于桓也。”小小年纪,于心计中见智慧。}

\lettrine{14.35} 海西\myidx{司马奕}时\footnote{海西:指晋废帝司马奕,太和六年被废,降为海西公,故称。},诸公每朝,朝堂犹暗,唯会稽王\myidx{司马昱}来\footnote{会稽王:指司马昱,后被桓温立为帝,史称简文帝。},轩轩如朝霞举\footnote{轩轩:气宇轩昂貌。举:升起。}。

{\cangkai\zihao{5}【评】海西公奕为帝时,桓温作为大司马,不仅掌控兵权,而且专擅朝政。史称“桓温有不臣之心,欲先立功河朔,以收时望。及枋头之败,威名顿挫,遂潜谋废立,以长威权”(\CJKunderwave{晋书}卷八\CJKunderwave{海西公纪})。其废立自专的野心,日渐付诸实践。因此,朝廷人心浮动,恐恐然不知所措,此所以“诸公每朝,朝堂犹暗”,非自然阳光不照殿堂,而是诸公卿朝官心理阴郁所致。唯有会稽王司马昱到,则朝廷诸公眼前一亮,不仅因其气宇轩昂的仪容风度,更因其作相王辅政,朝廷视为遏制桓温野心的希望。因诸公心存光明之想,故誉其“轩轩如朝霞举”,理想境界,自然升起了万道霞光,眩人耳目。可惜在政治拳击台上,司马昱并非一流高手,面对桓温这一重量级的冠军,只能“对之悲泣”,乞求他不要篡位而已。故其“轩轩朝霞”之光,犹如美丽的七色彩虹一样,瞬间即灭。}

\lettrine{14.36} 谢车骑\myidx{谢玄}道谢公\myidx{谢安}\footnote{谢车骑:指谢玄,曾官车骑将军,故称。道:称道。谢公:指谢安,玄叔。}:“游肆复无乃高唱\footnote{游肆:上街游玩。复无乃高唱:更无须高声吆喝。},但恭坐捻鼻顾睐\footnote{恭坐:端坐不动。捻鼻:轻捏鼻子。据说谢安有鼻疾,音浊。其作“洛生咏”——用中原洛阳口音吟咏诗歌,轰动一时,人争仿效。顾睐:顾盼风姿。顾,回视。睐,旁视。},便自有寝处山泽间仪\footnote{寝处山泽:栖卧山林,指隐退。仪:仪态,容仪。}。”

{\cangkai\zihao{5}【评】谢玄之父奕早卒,叔父安待如己出。因此,侄儿对叔父的理解,非一朝一夕之功。但是,谢玄对于谢安的称美,并非一味歌功颂德,而只是随意从其生活琐事出发,抓住其典型细节,来刻画其容仪姿态,精神气度,因而更加可信、动人。的确,谢安虽是东晋陈郡阳夏谢氏家族首屈一指的代表人物,是一代名相,是一位杰出的政治家。但他与众不同之处,就是同时又是一位情趣高尚的玄学清谈家,早已勘破人生,看重功成身退后的归隐山林的自由生活,淡泊名利、闲云野鹤,成为一种理想追求,而并不恋栈。其遨游市肆大街之上,又何须叫卖自己来加以炒作宣传呢?“捻鼻顾睐”,确为画龙点睛的神来之笔。}

\lettrine{14.37} 谢公\myidx{谢安}云见林公\myidx{支遁}\footnote{谢公:指谢安。林公:指名僧支遁,即支道林。},双眼黯黯明黑\footnote{黯(àn暗)黯:深幽黑亮的样子。明黑:形容眼睛黑白分明,炯炯有神。}。孙兴公\myidx{孙绰}见林公\footnote{孙兴公:指孙绰,东晋著名文学家。},稜稜露其爽\footnote{稜稜:高起突兀貌。爽:豪爽。}。

{\cangkai\zihao{5}【评】在魏晋时代,要跻身上流社会,除了士族的高贵门第之外,名士风流也是重要条件。谢安是东晋一代的风流名相,士林领袖,他形人容止,和大画家顾恺之一样,直指要害,所谓“传神写照,正在阿堵中”——也即注重人之眼光神采。谢安心目中的名僧支遁,超拔物累而自然脱俗,因而“双眼黯黯明黑”,目光黑白分明而神彩流溢。其风神令人向往。而孙绰心目中的林公,则别是一个威严爽朗而豪气干云的人物。因为孙氏出身庶族寒门,跻身上流贵族社会甚属不易,因而仰头见名士,故有此评。视角不同,林公风采有异,但同为一活生生的支遁无异。}

\lettrine{14.38} 庾长仁\myidx{庾统}与诸弟入吴\footnote{庾长仁:庾统,字长仁,小字赤玉,亮之族子。见前\CJKunderwave{赏誉}89注。吴:郡名,今苏州一带。},欲住亭中宿\footnote{亭:驿亭,供人休息住宿。},诸弟先上,见群小满屋\footnote{群小:指一般百姓。当时士大夫蔑称百姓庶民为“小人”。},都无相避意。长仁曰:“我试观之。”乃策杖将一小儿\footnote{策杖:扶着手杖。将:携带。小儿:年轻仆人。},始入门,诸客望其神姿,一时退匿\footnote{退匿:退避躲藏。}。{\fzxk\zihao{6}\textcolor{red}{长仁,已见。一说是庾亮。}}

{\cangkai\zihao{5}【评】“欲住亭中宿”,朱铸禹\CJKunderwave{汇校集注}谓“住”疑为“往”之形讹,似可从。又据史载,庾统未有诸弟,所以刘注以此为庾亮故事。庾统英年早逝,而庾亮风流乃一代领袖人物,其神态气度自然不同凡响。魏晋是个士族门阀社会,其出身虽然并不写在脸上,但其风度神气,乃是长期的文化积淀形成,因此而有容仪气度方面的“君子”与“小人”之别,这也是现实生活所致。“小人”避让“君子”,虽属社会变异扭曲现象,但却是无可奈何的历史存在。}

\lettrine{14.39} 有人叹王恭\myidx{王恭}形茂者云\footnote{王恭:字孝伯,晋孝武帝王皇后之兄。形茂:形貌美好。}:“濯濯如春月柳\footnote{濯濯:清新明媚的样子。春月:春天,春季。}。”

{\cangkai\zihao{5}【评】“濯濯如春月柳”,运用了意象批评的修辞手法,从整体姿态神气着眼,而非理性的五官描写,因而人物形象更富生趣而有生命活力。其鲜活明亮、洁净清新,如春天新绿而随风飘拂的婀娜柳条,那风神气象,自然而然,而无须任何装腔作势的矫饰,这是俗世所无,而如天上神仙一般。魏晋名士风度之美,令人千载之下,想象不尽。}




%%% Local Variables:
%%% mode: latex
%%% TeX-engine: xetex
%%% TeX-master: "../Main"
%%% End:
