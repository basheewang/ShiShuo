%% -*- coding: utf-8 -*-
%% Time-stamp: <Chen Wang: 2025-12-06 19:45:17>

% ○ ◎ ‧ 「 」 『 』 々 ( ) “ ” ■ ^[一-龥]
% 【\([^】][^】][^】]+\)】 → {\\fzxk\\zihao{6}\\textcolor{red}{\1}}
% \(【评】.*\) → {\\cangkai\\zihao{5}\1}
% \(【题解】.*\) → {\\cangkai\\zihao{5}\1}
% 《\([^》]+\)》 → \\CJKunderwave{\1}
% ^\([0-9]+.[0-9]+\) → \\lettrine{\1}
% {\\fzxk\\zihao{6}\\textcolor{red}{[^o]*}}


\setlength{\parindent}{0pt}


\chapter{伤逝第十七}



{\cangkai\zihao{5}【题解】 伤,悲伤;逝,消逝,逝世。“伤逝”连称,则含有悲伤哀悼已经消逝的可爱生命之义。江淹\CJKunderwave{别赋}云:“黯然销魂者,唯别而已矣!”人生必然有生离死别,生离,当然催人泪下;但死别则人鬼幽隔,痛彻肝肠而徒唤奈何。江淹\CJKunderwave{恨赋}结语云:“已矣哉!……绮罗毕兮池馆尽,琴瑟灭兮丘垄平。自古皆有死,莫不饮恨而吞声。”因此,思考人类的生死问题,就成为文学的永恒主题。故王羲之\CJKunderwave{兰亭集序}有“生死亦大矣,岂不痛哉”之言。后之视今亦犹今之视昔,面对亲朋好友的消逝,怎能不情动于中而形于言咏呢?悲悼死者,不仅是对古人的怀念,同时是一种对于自我生命价值的现实思考。本门共十九则,故事不一,而无不情真以抒痛。故王世懋批评说:“\CJKunderwave{世说}惟‘伤逝’独妙,无一语不解损神。”你看魏晋名士,无论帝王士庶,甚至是说“空”名僧,悲悼故人,几乎个个是情种,如第四则故事中王戎所说:“圣人忘情,最下不及情。情之所钟,正在我辈!”一旦能动真情,则毫无顾忌地加以表达。或痛哭流涕,或好作驴鸣,率性任情而自由不拘,其怪诞言行正是越名教而任自然的魏晋风度的又一生动展现。}

\lettrine{17.1} 王仲宣\myidx{王粲}好驴鸣\footnote{王仲宣:王粲(177—217),字仲宣,汉末山阳高平(今山东邹县西南)人。曾避乱南下荆州依刘表,后为曹操侍中。文学为建安七子之一。},{\fzxk\zihao{6}\textcolor{red}{\CJKunderwave{魏志}曰:“王粲字仲宣,山阳高平人。曾祖袭、〔祖〕父畅,皆为汉三公。粲至长安见蔡邕,奇之,倒屣迎之,曰:‘此王公孙,有异才,吾不及也。吾家书籍尽当与之。’避乱荆州,依刘表,以粲貌寝通脱,不甚重之。太祖以从征吴,道中卒。”}} 既葬,文帝\myidx{曹丕}临其丧\footnote{文帝:指魏文帝曹丕。但王粲死于建安二十二年春,冬,曹丕为魏太子。曹丕篡汉称帝,是粲死后之事。此为\CJKunderwave{世说}作者追述所称。},顾语同游曰\footnote{同游:同行之人。}:“王好驴鸣,可各作一声以送之。”赴客皆一作驴鸣\footnote{赴客:赴丧吊唁之客。}。{\fzxk\zihao{6}\textcolor{red}{案:戴叔鸾母好驴鸣,叔鸾每为驴鸣以说其母。人之所好,傥亦同之。}}

{\cangkai\zihao{5}【评】此事发生在建安二十二年(217),王粲跟随曹操征吴,道中死于疫疾。当时曹丕贵为魏王曹操的接班人,其权势日盛,与王粲地位悬殊。可见曹丕亲临王粲之丧而加以吊唁,并非从政治地位出发,而是重在真挚的友情。丕与建安七子的关系非同一般,其中包括乃父政敌孔融,都因其文学而亲近叹美,其\CJKunderwave{与吴质书}云:“昔年疾疫,亲故多离其灾。徐、陈、应、刘,一时俱逝,痛可言邪!昔日游处,行则连舆,止则接席,何曾须臾相失。每至觞酌流行,丝竹并奏,酒酣耳热,仰而赋诗,当此之时,忽然不自知乐也。……何图数年之间,零落略尽,言之伤心!……追思昔游,犹在心目,而此诸子化为粪壤,可复道哉!”这是以文学家的视角,来抒发其诚挚悼念之情。追悼会的气氛,按照礼教之制,应是悲哀肃穆。但在王粲临葬的追悼会上,曹丕作为一人之下,万人之上的“准”太子,却因王粲生前的爱好,建议大家各作驴鸣以悼念,并且自己带头作驴鸣。追悼会上作驴鸣,古人以为违反礼教,今人以为滑稽可笑。但是,只要出于真心真情,抒泄思友悲痛,“准”太子曹丕就可以不顾礼教规范而行之无所顾忌。越名教而抒真情,无拘无束,旷达任诞,正见其时魏晋名士精神风度之一斑。另外,不仅是王粲、曹丕等好驴鸣,后面故事中的王济、孙楚还有戴叔鸾,均好驴鸣,这是为什么?张万起等先生解释说:“魏晋文人多以歌啸为行气修炼的养生术,力求气之拉长盛壮。而且吟啸也足以表现文人的风度逸态。吟啸之声或若鸾凤之音,或若高柳之蝉、巫峡之猿,等等。好驴鸣盖亦此类。”(见其\CJKunderwave{译注})。驴为生活中常见之物,其体虽小于骡马,但其鸣号,力大声宏,而传播甚远,用今天的话说,其共鸣之音震撼人心,而远胜于骡马之嘶。人善作驴鸣,则必擅于气功者,当然与养生术有关。张注可资参考。}

\lettrine{17.2} 王濬冲\myidx{王戎}为尚书令\footnote{王濬冲:王戎,字濬冲。参\CJKunderwave{德行}17注。尚书令:朝廷尚书省长官。},箸公服\footnote{公服:官吏礼服。},乘轺车\footnote{轺(yáo摇)车:一马拉的轻便车。},经黄公酒垆下过\footnote{黄公酒垆:酒店名。酒垆:酒肆。垆,安放酒坛的土台。}。{\fzxk\zihao{6}\textcolor{red}{韦昭\CJKunderwave{汉书注}曰:“垆,酒肆也;以土为堕,四边高似垆也。”}} 顾谓后车客:“吾昔与嵇叔夜\myidx{嵇康}、阮嗣宗\myidx{阮籍}共酣饮于此垆\footnote{嵇叔夜:嵇康字叔夜。阮嗣宗:阮籍。字嗣宗。嵇、阮为正始年间竹林七贤领袖人物。}。竹林之游\footnote{竹林之游:阮籍、嵇康、山涛、刘伶、向秀、阮咸、王戎相与友善,常在竹林宴游谈论,时人号为“竹林七贤”。},亦预其末\footnote{亦预其末:七贤中王戎年纪最轻,名列于末。}。自嵇生夭、阮公亡以来\footnote{嵇生夭:嵇康被司马昭所杀,年仅三十九,正当盛年之时,故称“夭”。阮公亡:阮籍病故,死时五十四岁,故称“亡”。},便为时所羁绁\footnote{羁绁(jī xiè机屑):原为络具,比喻束缚、拘绊。}。今日视此虽近,邈若山河\footnote{邈(miǎo眇):遥远。}。”{\fzxk\zihao{6}\textcolor{red}{\CJKunderwave{竹林七贤论}曰:“俗传若此。颍川庾爰之尝以问其伯文康,文康云:‘中朝所不闻,江左忽有此论,盖好事者为之耳。’”}}

{\cangkai\zihao{5}【评】七贤之游竹林,嵇、阮为其领袖,他们志同道合,追求独立之人格,自由之思想,旷达任诞,蔑视礼法,为魏晋名士的精神风度开了一个好头。但曾几何时,“嵇生夭,阮公亡”,在统治者挥舞屠刀之时,很快烟消云散。后来王戎为人,不敢恭维,日渐发迹而成为贵要之士;但从自由走向羁绁,总觉得人活着很不自在,在依附的生活中,失去了人之所以为人的精神。以此故地重游而触目惊心,往事历历在目,终于发出了“今日视此虽近,邈若山河”的深沉浩叹,陈梦槐以为“二语痛绝”,可谓一语中的。人或持此语与\CJKunderwave{古诗十九首}中之\CJKunderwave{迢迢牵牛星}中“河汉清且浅,相去复几许?盈盈一水间,脉脉不得语”相拟,都表现一种可望而不可即的天人阻隔的深情悲痛。如杨勇先生为范子烨\CJKunderwave{〈世说新语〉研究}作序,云:“先师伍公叔傥尝举\CJKunderwave{世说·伤逝篇}‘王濬冲为尚书令’条评之曰:居然犹美于\CJKunderwave{迢迢牵牛星},富有诗意。\CJKunderwave{世说}超秀,韵味简直是一部无韵散文诗。……真说破了\CJKunderwave{世说}近诗的奥秘。”所论启人良多,读\CJKunderwave{世说}者,须明“诗无达诂”之理。}

\lettrine{17.3} 孙子荆\myidx{孙楚}以有才\footnote{孙子荆:孙楚(?—294),字子荆。参\CJKunderwave{言语}24注。},少所推服\footnote{推服:推崇佩服。},唯雅敬王武子\myidx{王济}\footnote{王武子:王济(约240—285),字武子。参\CJKunderwave{言语}24注。}。武子丧时,名士无不至者。子荆后来,临尸恸哭,宾客莫不垂涕。哭毕,向灵床曰\footnote{灵床:停尸床。}:“卿常好我作驴鸣,今我为卿作。”体似真声\footnote{体似真声:模仿驴鸣声音逼真。体,模仿。},宾客皆笑。孙举头曰:“使君辈存,令此人死!”{\fzxk\zihao{6}\textcolor{red}{\CJKunderwave{语林}曰:“王武子葬,孙子荆哭之甚悲,宾客莫不垂涕。既作驴鸣,宾客皆笑。孙闻之,曰:‘诸君不死,而令王武子死乎!’宾客皆怒。”}}

{\cangkai\zihao{5}【评】在西晋人物中,太原王济和孙楚,都是著名的文人才子。二人皆心高气傲,少所推服。连对于像二陆(机、云)兄弟这样的大文学家,王济都曾加以嘲讽。但王对于孙,却大为推赏,他当州大中正时,自品孙楚云:“天才英博,亮拔不群。”孙妇死,作诗悼念,王读后评云:“未知文生于情,情生于文,览之凄然,增伉俪之重。”王济是孙楚的知音。孙楚之推服王济,正是惺惺相惜。济死,则有知音难遇之叹,因而临尸恸哭,为作驴鸣,纯是一片真情的自由抒发,而不顾忌礼制规范。“使君辈存,令此人死!”无限悲痛,生于五内而溢于言表。}

\lettrine{17.4} 王戎\myidx{王戎}丧儿万子\myidx{王绥}\footnote{王戎:见前注。万子:戎子绥,字万子。绥年十九卒。\CJKunderwave{赏誉}29则云:“戎子万子,有大成之风,苗而不秀。”},山简\myidx{山简}往省之\footnote{山简:山涛子,字季伦。\CJKunderwave{赏誉}29云:“涛子简,疏通高素。”简温雅有父风,官至尚书左仆射、领吏部。后出为征南将军,镇襄阳,常饮酒于高阳池,倒载醉归。省之;探望。},王悲不自胜。简曰:“孩抱中物\footnote{孩抱中物:怀抱中的幼小孩子,泛指儿童。按:王绥死年十九,与“孩抱中物”情况不符。故刘注指出:“一说是王夷甫丧子,山简吊之。”查\CJKunderwave{晋书·王衍传},衍曾丧幼子,疑是。},何至于此?”王曰:“圣人忘情,最下不及情\footnote{“圣人忘情”二句,魏晋玄学清谈命题之一,魏正始年间,何晏提出“圣人无情”论,认为圣人不同世俗常人,因其高明而无喜怒哀乐。王弼不同意,认为圣人之情同于常人,只是不为情累而已,故倡“圣人忘情”之论。}。情之所钟\footnote{钟:钟集,专注。},正在我辈。”{\fzxk\zihao{6}\textcolor{red}{王隐\CJKunderwave{晋书}曰:“戎子绥,欲取裴遁女。绥既早亡,戎过伤痛,不许人求之,遂至老无敢取者。”}} 简服其言,更为之恸。{\fzxk\zihao{6}\textcolor{red}{一说是王夷甫丧子,山简吊之。}}

{\cangkai\zihao{5}【评】王戎(或王衍)是魏晋的清谈名家,熟悉当时所讨论的玄学命题。“圣人忘情,最下不及情”,正是孔子所谓上智下愚不移论的翻版。他把自己摆在生活中的常人地位。“情之所钟,正在我辈”,乃实境妙语,其舐犊情深,催人泪下。}

\lettrine{17.5} 有人哭和长舆\myidx{和峤}曰\footnote{何长舆:和峤,字长舆。参\CJKunderwave{德行}17注。}:“峨峨若千丈松崩\footnote{峨峨:高峻貌。崩:倒。}。”

{\cangkai\zihao{5}【评】运用象征修辞手法来进行意象批评,喻和峤之死,如千丈松之倾倒。前\CJKunderwave{赏誉}第15则及\CJKunderwave{晋书}本传,谓庾敳之言。史称和峤为人厚自崇重,“有盛名于世,朝野许其能整风俗,理人伦”,其为政清简,“甚得百姓欢心”。在朝坚持风节,直言敢谏。如谓太子之痴愚“圣质如初”,令武帝不快。后太子即位,是为惠帝,责问和峤:“卿苦谓我不了家事(按即没有能力继承帝位),今日定云乎?”峤曰:“臣昔事先帝,曾有斯言。言之不效,国之福也。臣敢逃其罪乎!”仍然巧妙地坚持其气节而不为屈服。故千丈松以喻其国家栋梁之材,十分贴切。悼念和峤,实是为国惜才,流露出时人对于国家局势动荡的担心。不久,八王乱起,继之“五胡乱华”而西京沦亡。千丈松崩,正是黍离之哀的前奏曲,悲乎痛哉!}

\lettrine{17.6} 卫洗马\myidx{卫玠}以永嘉六年丧\footnote{卫洗马:指卫玠,曾官太子洗马。永嘉六年:即312年。永嘉为晋怀帝年号。},谢鲲\myidx{谢鲲}哭之\footnote{谢鲲:字幼舆。官至豫章太守。他酷喜老庄,善清谈。为一时之杰。},感动路人。{\fzxk\zihao{6}\textcolor{red}{\CJKunderwave{永嘉流人名}曰:“玠以六年六月二十日亡,葬南昌城许徵墓东。玠之薨,谢幼舆发哀于武昌,感恸不自胜。人问:‘子何恤而致哀如是?’答曰:‘栋梁折矣,何得不哀!’”}} 咸和中\footnote{咸和:晋成帝年号(326—334)。},丞相王公\myidx{王导}教曰\footnote{丞相王公:指王导。教:古代官场中,上级对下级的指令、批示、意见,均为教。}:“卫洗马当改葬,此君风流名士\footnote{风流:动人风韵和杰出才华。},海内所瞻\footnote{瞻:瞻仰,敬慕。},可修薄祭\footnote{薄祭:菲薄的祭礼。},以敦旧好\footnote{敦:敦厚,加深。}。”{\fzxk\zihao{6}\textcolor{red}{\CJKunderwave{玠别传}曰:“玠咸和中改迁于江宁。丞相王公教曰:‘洗马明当改葬。此君风流名士。海内民望。可修三牲之祭,以敦旧好。’”}}

{\cangkai\zihao{5}【评】人们悲悼卫玠,并非因其官高爵显,恰恰相反,据\CJKunderwave{晋书·职官志},太子属官中,“洗马八人,职如谒者、秘书,掌图籍”,“出则直前驱,导威仪”,干的是图书馆、仪仗队的事情。可见当时两晋士人心目中的卫玠,推为“中兴第一名士”,与政治仕途及其功业事迹无关,而是因其学问风仪之令名。玠少有名理,善通\CJKunderwave{庄}、\CJKunderwave{老},其玄学清谈,理会要妙,令人叹绝。甚至连目空一切的狂妄将军王敦也佩服得五体投地,叹美说:“昔王辅嗣(弼)吐金声于中朝,此子(卫玠)复玉振于江表,微言之绪,绝而复续。”以之上继正始之音而发扬光大。玠死鲲恸,“栋梁折矣,何得不哀”,发自五内,至诚至情,感动路人。王导一代之名相,称颂卫玠“风流名士,海内所瞻”,正是以玠作为魏晋风流之典范,其改葬祭奠,仍然是出于真情之自然。当时没有今天的音像设备,因此卫玠仪容之美及其清谈之妙,其风流随着时间的流逝而烟消云散,惜哉!}

\lettrine{17.7} 顾彦先\myidx{顾荣}平生好琴\footnote{顾彦先:顾荣,字彦先。参\CJKunderwave{德行}25注。},及丧,家人常以琴置灵床上\footnote{灵床:原指停尸床,此则为悼念死者而虚设的坐床。}。张季鹰\myidx{张翰}往哭之\footnote{张季鹰:张翰,字季鹰。参前\CJKunderwave{识鉴}10注。},不胜其恸\footnote{恸:悲痛恸哭。},遂径上床\footnote{径:径直,直接。},鼓琴作数曲竟\footnote{竟:完毕。},抚琴曰:“顾彦先颇复赏此不\footnote{颇:疑问语气词,犹如“是否”。}?”因又大恸,遂不执孝子手而出。

{\cangkai\zihao{5}【评】张翰是个性情中人,他博学能文,任性不拘,人称“江东步兵”,以比拟于阮籍。他是顾荣的生前知友,因顾生前好琴,以此不顾一切,径直走上灵床,为之鼓琴。“顾彦先颇复赏此不?”如与好友生前对语,情真意切,愈增悲悼之痛。其又大恸,不执孝子手而出,与灵床鼓琴事,皆违背丧礼制度。这些行为,人们视为怪异,而实是魏晋名士纵情率性,不拘礼法的真情自然流露,因而更加可爱感人。}

\lettrine{17.8} 庾亮儿\myidx{庾会}遭苏峻\myidx{苏峻}难遇害\footnote{庾亮儿:此指庾会(\CJKunderwave{晋书}作庾彬),字会宗,小字阿恭。参\CJKunderwave{雅量}17注。苏峻难:晋成帝咸和二年(327),庾亮辅政,欲夺历阳内史苏峻兵权。峻与祖约合谋,举兵反,次年破京师建康,迁成帝于石头,纵兵掠杀。后被陶侃、温峤等平定。}。诸葛道明\myidx{诸葛恢}女为庾儿妇\footnote{诸葛道明:诸葛恢,字道明。参\CJKunderwave{方正}25注。},既寡,将改适\footnote{改适:改嫁。},{\fzxk\zihao{6}\textcolor{red}{亮子会,会妻父(文)彪,并已见上。}} 与亮书及之。亮答曰:“贤女尚少,故其宜也\footnote{故:确实,本来。}。感念亡儿,若在初没\footnote{没:通“殁”,死。初没,刚死之时。}。”

{\cangkai\zihao{5}【评】据\CJKunderwave{晋书}庾亮传,亮三儿:彬、羲、和,彬于苏峻叛乱中遇害。但据\CJKunderwave{雅量}第17则刘注引\CJKunderwave{庾氏谱},又言会字会宗,亮长子,咸和六年遇害。据此推测,彬、会实为一人,虞彬小名为会;或因其字会宗之“会”,而父母有“会儿”之呼。虞会(彬)被害时年十九,则其妻诸葛文彪更为年轻。会与文彪,年轻夫妻感情甚笃,会死而文彪坚拒登车改嫁。但是,十几岁的青春少女守寡终身,终是有违人性。娘家与夫家双方家长商量之后,决定促使文彪改嫁,因而有后来\CJKunderwave{假谲}第10则所载江虨智娶新寡女的故事。于此可见魏晋士人并不以寡妇改嫁为耻,这一观念,远胜于宋明理学盛行之世。在这方面,庾亮思想颇为开通,不愧一代名士之称。其回诸葛恢之信,言有馀痛,声情并茂,忆念儿死,历历如在目前,但并不因一己私情,阻其新寡儿媳改嫁、另觅美满婚姻。于此又具体可见魏晋精神气度之一斑。但类似事情,王戎的处理则大异旨趣,如本门第4则刘注引王隐\CJKunderwave{晋书}载,戎子绥“欲取裴遁女,绥既早亡,戎过伤痛,不许人求之,遂至老无敢取(娶)者”。这就将一己之悲,转化为青春少女的终身之痛,从而违背了根本之人性,这是残酷的迫害,造成了新的人生悲剧。对于王戎晚年,不敢恭维,已丧尽前期竹林名士之风度。故陈梦槐比较批评说:“王衍(按:应作王戎)丧子不许人取裴女,庾亮毕竟胜王多多。”所说甚是。}

\lettrine{17.9} 庾文康\myidx{庾亮}亡\footnote{庾文康:指庾亮,死后谥号文康,故称。},何扬州\myidx{何允}临葬云\footnote{何扬州:指何允,他曾官扬州刺史,故称。参前\CJKunderwave{言语}54注。}:“埋玉树箸土中\footnote{玉树:喻姿仪秀美、才华杰出之人。},使人情何能已已\footnote{人情:人心。已已:第一个“已”是完了、停止之义,以此第二个“已”是语气词。}!”{\fzxk\zihao{6}\textcolor{red}{\CJKunderwave{搜神记}曰:“初,庾亮病,术士戴洋曰:‘昔苏峻事,公于白石祠中许赛车下牛,从来未解,为此鬼所考,不可救也。’明年,亮果亡。”\CJKunderwave{灵鬼志·谣征}曰:“文康初镇武昌,出石头,百姓看者于岸歌曰:‘庾公上武昌,翩翩如飞鸟;庾公还扬州,白马牵旒旐。’又曰:‘庾公初上时,翩翩如飞鵶;庾公还扬州,白马牵旐车。’后连征不入,寻薨,下都葬焉。”}}

{\cangkai\zihao{5}【评】玉树温润喜人,晶莹透彻而又青春焕发,其姿仪之美,风气之佳,可以想象。埋入土中,化为污泥,物尚不堪,何况是人!“使人情何能已已”,真情悲悼,何其沉痛!}

\lettrine{17.10} 王长史\myidx{王濛}病笃\footnote{王长史:王濛,曾官司徒左长史,故称。病笃:病重,病危。},寝卧灯下,转麈尾视之\footnote{麈尾:魏晋清谈时助兴的手执用具。},叹曰:“如此人,曾不得四十\footnote{曾:竟,居然。按:濛卒于永和初年,年三十九。}!”及亡,刘尹\myidx{刘惔}临殡\footnote{刘尹:指曾任丹阳尹的刘惔。他与王濛齐名,都是当时著名清谈家。殡:殡殓,指为死者更衣入棺的丧礼。},以犀柄麈尾箸柩中\footnote{柩(jiù就):盛尸棺材。},因恸绝\footnote{恸绝:因悲痛而一时昏迷。}。{\fzxk\zihao{6}\textcolor{red}{\CJKunderwave{濛别传}曰:“濛以永和初卒,年三十九。沛国刘惔与濛至交,及卒,惔深悼之,虽友于之爱,不能过也。”}}

{\cangkai\zihao{5}【评】濛病笃时,灯下转麈尾而视,自怨自艾,叹己英年早逝,不得竟其心爱的清谈之兴,悲哉!刘惔与濛,均为一时清谈领袖,视玄谈活动为生命活力之所在。故熟知故人心思,以所爱犀柄麈尾置棺中,恸绝声悲,恨不能起死回生,与挚友再次畅谈玄理。“以犀牛麈尾著柩中”,一个简单的细节动作,却胜过千言万语之悼词。古人有“知音其难”之叹,“逢其知音,千载其一乎”(\CJKunderwave{文心雕龙·知音})。刘惔是王濛的真正知音。濛寿虽夭,但人生得一知己,足矣,又何怨乎?}

\lettrine{17.11} 支道林\myidx{支遁}丧法虔\myidx{法虔}之后\footnote{支道林:支遁,字道林,东晋名僧。},精神霣丧\footnote{霣(yǔn陨)丧:坠落丧失。},风味转坠\footnote{风味:风采韵味。转坠:日益消沉颓丧。}。{\fzxk\zihao{6}\textcolor{red}{\CJKunderwave{支遁传}曰:‘法虔,道林同学也。隽朗有理义。遁甚重之。”}} 常谓人曰:“昔匠石废斤于郢人\footnote{昔匠石废斤于郢人:见\CJKunderwave{庄子·徐无鬼}篇。斤,斧头。郢,地名,春秋时楚国都。},{\fzxk\zihao{6}\textcolor{red}{\CJKunderwave{庄子}曰:“郢人垩漫其鼻瑞,若蝇翼,使匠石运斤斫之,垩尽而鼻不伤,郢人立不失容。”}} 牙生辍弦于锺子\footnote{牙生:指俞伯牙。辍弦于锺子:因锺子期死而绝弦止琴。},{\fzxk\zihao{6}\textcolor{red}{\CJKunderwave{韩诗外传}曰:“伯牙鼓琴,锺子期听之。方鼓琴,志在太山,子期曰:‘善哉乎鼓琴,巍巍乎若太山。’莫景之间,志在㳅(流)水,子期曰:‘善哉乎鼓琴,洋洋乎若㳅(流)水。’锺子期死,伯牙擗琴绝弦,终身不复鼓之。以为在者,无足为之,鼓琴也。”}} 推己外求\footnote{推己外求:谓因己之意以推求别人之心。},良不虚也\footnote{良:确实。}。冥契既逝\footnote{冥契:指互相默契的知音。},发言莫赏,中心蕴结\footnote{中心蕴结:心中郁闷。},余其亡矣!”却后一年\footnote{却后:过后。},支遂殒\footnote{殒(yǔn陨):死。}。

{\cangkai\zihao{5}【评】故事当发生于哀帝隆和三年(365),因支遁卒于太和元年(366),法虔早其一年而逝。支遁是东晋一代名僧,佛学精深,其著\CJKunderwave{即色论},有“色不自有,虽色而空”之言,见\CJKunderwave{文学}第35则刘注。但是,名僧也是人,虽高倡“色空”之论,关键时刻,仍然无法脱离地球而忘于世外,其晚年临终前一年,因同学法虔之丧而未能忘情,即是一例。人或讥其“精神霣丧”,有失名僧风度。实则观音慈悲,不离世俗,普度众生,莫非人情。而人情之中,不情之情,最为深情,皆内心自然之流露而无可掩蔽,岂虚言哉!“冥冥既逝,发言莫赏”,与前则知音难求之叹,同一生命情结,读之催人泪下。}

\lettrine{17.12} 郗嘉宾\myidx{郗超}丧\footnote{郗嘉宾:郗超,字嘉宾。},左右白郗公\myidx{郗愔}\footnote{白:禀告。郗公:指郗愔,超父。}:“郎丧\footnote{郎:家门下人对少主人的尊称,相当于今之“郎官”,“少爷”。}。”既闻不悲,因语左右:“殡时可道。”公往临殡,一恸几绝\footnote{绝:气绝,昏绝。}。{\fzxk\zihao{6}\textcolor{red}{\CJKunderwave{中兴书}曰:“超年四十二,先愔卒。超所交友,皆一时俊乂,及死之日,贵贱为诔者四十馀人。”\CJKunderwave{续晋阳秋}曰:“超党戴桓氏,为其谋主。以父愔忠于王室,不令知之。将亡,出一小书箱付门生,云:‘本欲焚此,恐官年尊,必以伤愍为毙。我亡后,若大损眠食,则呈此箱。’愔后果恸悼成疾,门生乃如超旨,则与桓温往反密计。愔见即大怒,曰:‘小子死恨晚。’后不复哭。”}}

{\cangkai\zihao{5}【评】晚年丧子,白头人哭黑发人,其舐犊情深,乃人之常情,此郗公之所以一哭而恸绝也。据\CJKunderwave{资治通鉴},郗超死于晋武帝太元二年(377)十二月。在东晋间,郗超是个不可多得的人才,连政敌谢安也畏惮三分,对其学问义理,也甚为佩服。作为桓温谋主,温死之后,郁郁而亡,也在料中,政治生涯确是变幻莫测。人算不如天算,又可奈何!}

\lettrine{17.13} 戴公\myidx{戴逵}见林法师\myidx{支遁}墓\footnote{戴公:指戴逵,东晋高隐之士。},{\fzxk\zihao{6}\textcolor{red}{\CJKunderwave{支遁传}曰:“遁,太和元年终于剡之石城山,因葬焉。”}} 曰:“德音未远\footnote{德音:美好言辞。},而栱(拱)木已积\footnote{栱木已积:墓木成林。}。冀神理绵绵\footnote{冀:希望。神理绵绵:精神绵延不绝。},不与气运俱尽耳\footnote{气运:气数,年寿。}。”{\fzxk\zihao{6}\textcolor{red}{王珣\CJKunderwave{法师墓下诗序}曰:“余以宁康二年命驾之剡石城山,即法师之丘也。高坟郁为荒楚,丘陇化为宿莽,遗迹未灭,而其人已远。感想平昔,触物悽怀。”其为时贤所惜如比。}}

{\cangkai\zihao{5}【评】戴逵之言,触物生情,倍感凄怆。林公名僧,同样是人,人之年寿有时而尽,但冀其精神永垂不朽。林公虽逝而获此知己,则虽死犹生。}

\lettrine{17.14} 王子敬\myidx{王献之}与羊绥\myidx{羊绥}善\footnote{王子敬:王献之,字子敬。羊绥:字仲彦,官至中书侍郎。}。绥清淳简贵\footnote{清淳:清雅纯厚。简贵:简朴清高。},为中书郎\footnote{中书郎:官名,即中书侍郎,是中书省令、监的副职。},少亡。{\fzxk\zihao{6}\textcolor{red}{绥,已见。}} 王深相痛悼\footnote{痛悼:悲痛伤心。},语东亭\myidx{王珣}云\footnote{东亭:王珣,封东亭侯。}:“是国家可惜人\footnote{可惜人:值得珍惜之人。}。”

{\cangkai\zihao{5}【评】羊绥是个品格高尚、具有独立精神的人,他不仅不会为了升官发财而处心积虑去走后门,就是送上门来的官运,只要感到不太尊重的话,也是坚决拒绝。如\CJKunderwave{方正}第60则故事,载辅政的谢安知道羊绥的才干,“致意令来”——希望羊绥来见他,但羊氏“终不肯诣”。独立之人格,自由之思想,在羊绥身上,看到了魏晋名士之风度。王献之称之为“国家可惜人”以此。献之是当时的玄学清谈家,人讥玄家不婴世务,以浮华相高,而不以国事为重。但观献之为国惜才之事实,则片面之讥,不攻自破。}

\lettrine{17.15} 王东亭\myidx{王珣}与谢公\myidx{谢安}交恶\footnote{王东亭:指王珣。谢公:指谢安。}。{\fzxk\zihao{6}\textcolor{red}{\CJKunderwave{中兴书}曰:“珣兄弟皆壻谢氏,以猜嫌离婚。太傅既与珣绝婚,又离〔珉〕(原无‘珉’字,据\CJKunderwave{晋书·王珣传}校增)妻。由是二族遂成仇衅。”}} 王在东闻谢丧\footnote{在东:吴郡、会稽一带,在京师建康之东。此指会稽王家。},便出都诣子敬\myidx{王献之}\footnote{诣:拜访,访问。子敬:王献之,字子敬。参前注。},道欲哭谢公\footnote{哭:哭吊,吊唁。}。子敬始卧,闻其言,便惊起,曰:“所望于法护。”{\fzxk\zihao{6}\textcolor{red}{法护,珣小字。}} 王于是往哭。督帅刀(刁)约不听前\footnote{督帅:公府属官,刁约为谢安属吏,主办其丧事杂务。},曰:“官平生在时\footnote{官:据朱铸禹引\CJKunderwave{通鉴注}曰:“宋齐之间,义从私属以至婢仆率呼其主为官。”这是对君主、尊长的敬称。},不见此客。”王亦不与语,直前哭甚恸,不执末婢\myidx{谢琰}手而退\footnote{末婢:指谢琰。据丧礼,吊客最后应与家属握手慰问。“不执末婢手而退”,则有违礼制。}。{\fzxk\zihao{6}\textcolor{red}{末婢,谢琰小字。琰字瑗度,安少子。开率有大度。为孙恩所害,赠侍中、司空。}}

{\cangkai\zihao{5}【评】为国家一哭泯恩仇,后来\CJKunderwave{晋书}据此而载入史册。东晋门阀社会,王、谢、桓、庾四大家族与司马皇室共治天下。庾、桓二族,自庾亮、桓温死后,势力渐替;王、谢家族,成为东晋政权主要支柱。琅邪王家的珣、珉兄弟,王导嫡孙,原为谢安女婿,但翁婿之间,因事嫌猜失和而离婚,于是构成仇衅。这大不利于东晋国家政权的巩固。淝水之战以后,谢安原想乘机“混一文轨”,北伐以恢复中原,为此进行合理部署,惜天不假年,加以权奸掣肘,事业未竟而遗恨终生。在这一形势下,王珣从维护国家利益出发,不计私怨,而以哭吊的行动,说明了他对谢安的正确评价。故事虽短,但人物形象栩栩如生,内在心理刻画深刻。不仅王珣“直前哭甚恸,不执末婢手而退”,见其不拘礼法的一片真情;就是王献之始卧惊起而呼:“所望于法护!”欲弥王、谢二家恩仇以利国家之心,跃然纸上。}

\lettrine{17.16} 王子猷\myidx{王徽之}、子敬\myidx{王献之}俱病笃\footnote{王子猷、子敬:兄弟俩为羲之子。羲之生七子,第五子徽之字子猷,第七子献之字子敬。参前注。病笃:病重,病危。},而子敬先亡\footnote{子敬先亡:按刘注,献之以泰元十三年(388)卒,年四十五。但据\CJKunderwave{晋书·王珉传},珉于献之死后代其为中书令,人称献之“大令”,王珉“小令”。珉卒于太元十三年,则献之不当卒于是年。据\CJKunderwave{法书要录}引张怀瓘\CJKunderwave{书断},曰:“子敬为中书令,泰元十一年(386)卒于官,年四十三。”其说与刘注不同,疑是。参朱铸禹\CJKunderwave{汇校集注}。}。{\fzxk\zihao{6}\textcolor{red}{献之以泰元十三年卒,年四十五。}} 子猷问左右\footnote{左右:身边服侍之人。}:“何以都不闻消息\footnote{都:完全。}?此已丧矣!”语时了不悲\footnote{了不悲:全无悲痛样子。}。便索舆来奔丧\footnote{索舆:吩咐备轿。},都不哭。子敬素好琴,便径入坐灵床上,取子敬琴弹,弦既不调\footnote{弦既不调:各根琴弦彼此音调已不和谐。},掷地云:“子敬,人琴俱亡!”因恸绝良久\footnote{恸绝:悲痛之极而昏厥。},月馀亦卒。{\fzxk\zihao{6}\textcolor{red}{\CJKunderwave{幽明录}曰:“泰元中,有一师从远来,莫知所出。云:‘人命应终,有生乐代者,则死者可生;若逼人求代,亦复不过少时。’人闻此,咸怪其虚诞。王子猷、子敬兄弟特相和睦,子敬疾,属厀,子猷谓之曰:‘吾才不如弟,位亦通塞,请以馀年代弟。’师曰:‘夫生代死者,以己年限有馀,得以足亡者耳。今贤弟命既应终,君侯算亦当尽,复何所代?’子猷先有背疾,子敬疾笃,恒禁来往。闻亡,便抚心悲惋,都不得一声,背即溃裂。推师之言,信而有实。”}}

{\cangkai\zihao{5}【评】此与第七则张翰径上灵床鼓琴哭悼顾荣事相似。但相比之下,此则兄弟手足情深,其言行不同寻常而违制越轨,纯是日积月累平素生活感情累积之爆发,是一种纯真的潜意识行为,因而更加动人心弦。徽之先是“了不悲”、“都不哭”,其悲其痛,已是“麻木”;一旦麻木期过,则有“人琴俱亡”之叹,因而“恸绝良久”,“月馀亦卒”,一抑一扬,掀起情感高潮,见作者艺术匠心。}

\lettrine{17.17} 孝武\myidx{司马曜}山陵夕\footnote{孝武:指东晋孝武帝司马曜。山陵夕:指皇帝驾崩之时。山陵,原指皇帝陵墓,这里名词动化,指帝王丧葬。},王孝伯\myidx{王恭}入临\footnote{王孝伯:王恭,字孝伯。参\CJKunderwave{德行}44注。入临:入京哭吊。},告其诸弟曰:“虽榱桷惟新\footnote{榱桷(cuī jué崔决):房屋椽子。比喻身负重任之人。},便自有\CJKunderwave{黍离}之哀\footnote{\CJKunderwave{黍离}之哀:\CJKunderwave{黍离}是\CJKunderwave{诗经·王风}之诗,\CJKunderwave{毛诗序}以为是周大夫见故国宗庙一片废墟,长满禾黍,因此叹西周之沦亡。}。”{\fzxk\zihao{6}\textcolor{red}{\CJKunderwave{中兴书}曰:“烈宗丧,会稽王道子执政,宠幸王国宝,委以机任。王恭入赴山陵,故有此叹。”}}

{\cangkai\zihao{5}【评】故事发生于太元二十一年(396)九月,孝武帝被弑,其弟司马道子执政,元显专权,“竟不推其罪人”,连皇帝都算白死。其实,道子、元显父子在孝武帝时,早已窃柄弄权,史称“官以贿迁,政刑谬乱”,“势倾天下,由是朝野奔凑”,因而“朋党竞扇”而危机深重。王恭作为孝武帝王皇后之兄,帝“深相钦重”。为牵制司马道子,委恭为平北将军、兖青二州刺史、都督兖青冀幽并徐州晋陵诸军事,镇京口。帝崩,王恭自京口入都奔丧,感到乌云满天,国家将乱,故兴\CJKunderwave{黍离}之叹。\CJKunderwave{黍离}诗云:“彼黍离离,彼稷之苗。行迈靡靡,中心摇摇。知我者谓我心忧,不知我者谓我何求。悠悠苍天,此何人哉!”他以“榱桷惟新”喻司马道子父子之擅政,新人来而故人弃,国家将有大难。无限忧虑与叹息,尽在不言中。}

\lettrine{17.18} 羊孚\myidx{羊孚}年三十一卒\footnote{羊孚:字子道。参\CJKunderwave{言语}104注引\CJKunderwave{羊氏谱},谓孚“年四十六卒”,与此“年三十一卒”不同。},桓玄\myidx{桓玄}与羊欣\myidx{羊欣}书曰\footnote{桓玄:字敬道。温少子。羊欣(370—442):字敬元,孚从祖弟。曾为桓玄主簿,预机要。后称病家居而自免十馀年。入宋为新安、义兴诸郡太守,又称病自免。欣博学工书,师王献之。}:“贤从情所信寄\footnote{贤从:羊孚为羊欣从祖兄,故称。信寄:信赖寄托。},暴疾而殒\footnote{暴疾:急病。殒:没,死亡。}。{\fzxk\zihao{6}\textcolor{red}{孚,已见。\CJKunderwave{宋书}曰:“欣字敬元,太山南城人。少怀静默,秉操无竞,美姿容,善笑言,长于草隶。”\CJKunderwave{羊氏谱}曰:“孚即欣从祖(兄)。”}} 祝予之叹\footnote{祝予之叹:亡我之悲。},如何可言!”{\fzxk\zihao{6}\textcolor{red}{\CJKunderwave{公羊传}曰:“颜渊死,子曰:‘噫,天丧予!’子路亡,子曰:‘噫,天祝予!’”何休曰:“祝者,断也。天将亡夫子耳。”}}

{\cangkai\zihao{5}【评】羊孚与羊欣,皆为东晋末年名士。孚的玄学清谈及其文学创作,当时即播在人口。二人皆桓玄腹心。桓玄后来因篡晋自称楚帝而成为历史的罪人。但他并非一生下来就是反面人物。他年轻时,也是个很有叛逆性格的性情中人。年轻时,父温死后,因其“历史问题”而背上黑锅,几乎抬不起头来,他不得不在黑暗中摸索、挣扎和奋斗。羊孚就是他受压奋斗中的知友,曾给予规劝、帮助和希望。因此,羊孚暴疾而死,给玄予巨大的心理打击,“情所信寄……如何可言!”正是出自内心真情的自然爆发。其思悼之哀,远胜于追悼会上公布的悼词。孚死不久,玄将行篡晋之谋,曾矫诏自贺曰:“六合同悦,情何可言!”言辞相似,而情之真、伪不辨自明。其感人力量,何可同日而语。}

\lettrine{17.19} 桓玄\myidx{桓玄}当篡位\footnote{桓玄当篡位:桓玄于晋安帝元兴二年(403)篡晋称帝,建国号楚,年号永始。桓玄其人。当,将要。按:桓玄称帝后第二年,被刘裕等击杀。},语卞鞠\myidx{卞范}云:\footnote{卞鞠:即卞范之(?—405),字敬祖,小字鞠,故称。济阴宛句人。桓玄腹心。玄称帝,官侍中、尚书仆射。后事败被杀。}{\fzxk\zihao{6}\textcolor{red}{卞范,已见。}} “昔羊子道\myidx{羊孚}恒禁吾此意\footnote{羊子道:羊孚,字子道。禁:制止,不许。}。今腹心丧羊孚\footnote{腹心:即心腹,喻亲近信任之人。},爪牙失索元\myidx{索元}\footnote{爪牙:鸟兽以爪牙作攻击或自卫,此喻武臣干将。},{\fzxk\zihao{6}\textcolor{red}{\CJKunderwave{索氏谱}曰:“元字天保,敦煌人。父绪,散骑常侍。元历征虏将军、历阳太守。”\CJKunderwave{幽明录}曰:“元在历阳疾病。西界一年少女子姓某,自言为神所降,来与元相闻,许为治护。元性刚直,以为妖惑,收以付狱,戮之于市中。女临死曰:‘却后十七日,当令索元知其罪。’如期,元果亡。”}} 而匆匆作此诋突\footnote{诋突,抵触,冒犯。按指篡位事。},讵允天心\footnote{讵(jù距):岂,哪里。允:符合。天心:天意。}?”

{\cangkai\zihao{5}【评】篡晋之前,桓玄执政,其初至朝廷,也曾厉行改革,“黜凡佞,擢俊贤,君子之道粗备,京师欣然”,这符合百姓热望和平,“思归一统”的心理要求,曾取得了一定的政绩。但政治家的野心,在利益的驱动下,很快变质。史称“玄自篡盗之后,骄奢荒侈,游猎无度,以夜继昼”,于是众务繁兴而民不聊生,很快滑向了罪恶的深渊(见\CJKunderwave{晋书}本传)。北方后秦姚兴,曾问袁虔之:“桓玄……才度定何如父?”对曰:“玄藉世资,雄据荆楚,属晋朝失政,遂偷宰衡。安忍无亲,多忌好杀,位不才授,爵以爱加,无公平之度,不如其父远。今既握朝政,必行篡夺,既非命世之才,正可为他人驱除耳。”(见\CJKunderwave{晋书}卷一一七\CJKunderwave{载记·姚兴上})与乃父温相比,玄有野心而无其能力,聪明一时,而糊涂一世,羊孚等之规劝置之脑后,虽自知“作此诋突,讵允天心”,但却为利所诱,而自取灭亡,悲哉!}





%%% Local Variables:
%%% mode: latex
%%% TeX-engine: xetex
%%% TeX-master: "../Main"
%%% End:
