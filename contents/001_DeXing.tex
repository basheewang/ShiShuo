%% -*- coding: utf-8 -*-
%% Time-stamp: <Chen Wang: 2025-11-25 09:52:19>

% ○ ◎ ‧ 「 」 『 』 々 ( ) “ ”
% 【\([^】][^】][^】]+\)】 → {\\fzxk\\zihao{6}\\textcolor{red}{\1}}
% \(【评】.*\) → {\\cangkai\\zihao{5}\1}
% \(【题解】.*\) → {\\cangkai\\zihao{5}\1}
% 《\([^》]+\)》 → \\CJKunderwave{\1}
% ^\([0-9]+.[0-9]+\) → \\lettrine{\1}

\setlength{\parindent}{0pt}

\part{上卷}


\chapter{德行第一}

{\cangkai{}\zihao{5}【题解】今本\CJKunderwave{世说新语}(以下简称\CJKunderwave{世说})共三十六门类,人称以\CJKunderwave{论语·先进}所载孔门四科(德行、言语、政事、文学)冠其首。此话不假。但若论其区分门类的标准及其精神实质,则因作者及所录人物的生活年代不同,已具有魏晋六朝时代的新鲜风貌,而不必与先秦两汉传统儒家观念尽皆相同。魏晋六朝的时代思潮,玄风炽煽,释理禅义,熏染了一代士人。因而魏晋士人的思想面貌,道德标准及其言语行事,既继往,又开来,在旧模式中又具有新内容和新突破。本门所称“德行”,当作如是观。汉郑玄曾云:“德行,内外之称,在心为德,施之为行。”(\CJKunderwave{周礼·地官·师氏}注)所谓“德行”,顾名思义,是道德与品行,指人们的道德观念及其行为实践。但如“言皆玄远,未尝臧否人物”的阮籍一类人物,已成为魏晋士人心仪之典型,正见当时道德观念的微妙变化及其时代复杂性。}

\lettrine{1.1} 陈仲举\myidx{陈蕃}言为士则\footnote{陈仲举(?—168):陈蕃字仲举,汝南平舆(今属河南)人。性方峻,不交非类。不畏强御而直言极谏,终为宦官所害。士则:士人典则。},行为世范\footnote{世范:世人典范。},登车揽辔\footnote{登车揽辔(pèi配):登上公车而持缰驭奔。泛指出仕为官。},有澄清天下之志。{\fzxk\zihao{6}\textcolor{red}{\CJKunderwave{汝南先贤传}曰:“陈蕃字仲举,汝南平舆人。有室荒芜,不扫除。曰:‘大丈夫当为国家扫天下。’值汉桓之末,阉竖用事,外戚豪横。及拜太傅,与大将军窦武谋诛宦官,反为所害。”}} 为豫章太守\footnote{豫章:古郡名,治所在今江西南昌。},{\fzxk\zihao{6}\textcolor{red}{\CJKunderwave{海内先贤传}曰:“蕃为尚书,以忠正忤贵戚,不得在台,迁豫章太守。”}} 至,便问徐孺子\myidx{徐穉}所在\footnote{徐孺子:徐穉字孺子,汉末隐者,有“南州高士”之称。},欲先看之\footnote{看:探看,访问。}。{\fzxk\zihao{6}\textcolor{red}{谢承\CJKunderwave{后汉书}曰:“徐穉字孺子,豫章南昌人。清妙高时(他本作‘跱’),超世绝俗。前后为诸公所辟,虽不就,及其死,万里赴吊。常预炙鸡一只,以绵渍酒中,暴干以裹鸡,径到所赴冢隧外,以水渍绵,斗米饭,白茅为藉,以鸡置前。酹酒毕,留谒即去,不见丧主。”}} 主簿白\footnote{主簿:官吏名称。汉时中央及地方郡县均置此属官,职在典领文书,参与机要,办理政事。}:“群情欲府君先入廨\footnote{府君:古时太守敬称。廨(xiè械):官署。}。”陈曰:“武王式商容之闾\footnote{武王:指周武王姬发。式:通“轼”,车前横木,这里作动词用,指乘车前往。商容:传说中的殷商遗贤。闾:里巷之门,这里指家门。},席不暇煗\footnote{席不暇煗:席子尚未坐暖。煗,“暖”的异体字。}。{\fzxk\zihao{6}\textcolor{red}{许叔重曰:“商容,殷之贤人,老子师也。”车上跽曰式。}} 吾之礼贤,有何不可!”{\fzxk\zihao{6}\textcolor{red}{袁宏\CJKunderwave{汉纪}曰:“蕃在豫章,为穉独设一榻,去则悬之,见礼如此。”}}

{\cangkai\zihao{5}【评】据\CJKunderwave{后汉书·陈蕃传},同一故事,是东安太守礼遇周璆,与\CJKunderwave{世说}不同。然唐王勃\CJKunderwave{滕王阁序}有“徐孺下陈蕃之榻”的名言,充分说明了\CJKunderwave{世说}的影响及魅力。礼贤下士,是当时人们称颂的美德,与士人们匡时救国的理想密切相关。史称陈蕃性“方峻”,不交非类。但对贤人高士则不因其地位卑微而轻之,悬榻示敬,正见其评价人物以道德为先,而非以功名爵禄为准。本则故事,言约旨远,正气凛然。“澄清天下”诸语,掷地铿然有声,见后汉志士仁人力挽狂澜理想之远大。思想是行为的指南,陈蕃最后明知必死,而慷慨赴义,终成一代士人之典范。}

\lettrine{1.2} 周子居\myidx{周乘}常云\footnote{周子居:周乘字子居。汉末汝南安城(今河南正阳东北)人。\CJKunderwave{世说·赏誉}第1则刘注引\CJKunderwave{汝南先贤传}云:“天姿聪朗,高峙岳立,非陈仲举、黄叔度之俦不交也。……为太(泰)山太守,甚有惠政。”}:“吾时月不见黄叔度\myidx{黄宪}\footnote{时月:泛指数月。黄叔度:汉末汝南慎阳(今河南正阳)人。出身贫贱而德行高尚,世誉之颜渊再生。\CJKunderwave{后汉书·黄宪传}论曰:“将以道周性全,无德而称乎?”所谓“无德”,言其“德大无能名焉”。},则鄙吝之心已复生矣\footnote{鄙吝之心:浅薄庸俗的贪婪之心。}。”{\fzxk\zihao{6}\textcolor{red}{子居别见。\CJKunderwave{典略}曰:“黄宪字叔度,汝南慎阳人。时论者咸云‘颜子复生’。而族出孤鄙,父为牛医。颍川荀季和执宪手曰:‘足下,吾师范也。’后见袁奉高曰:‘卿国有颜子,宁知之乎?’奉高曰:‘卿见吾叔度邪?’戴良少所服下,见宪则自降薄,怅然若有所失。母问:‘汝何不乐乎?复从牛医儿所来邪?’良曰:‘瞻之在前,忽焉在后,所谓良之师也。’”}}

{\cangkai\zihao{5}【评】据\CJKunderwave{后汉书·黄宪传},语出同郡陈蕃、周举之口,当属传闻有异之故。但同时也从另一个方面说明,黄宪道德高尚,士人有口皆碑,故范晔传论称其“言论风旨,无所传闻”,美其道德之广大无涯,与道家无德之谓至德相似。宪一介布衣,乡间兽医之子,能够流誉人口,当与汉末士林清议或汝南月旦有关,隐约透露出激清扬浊的传统美刺精神,是医治浊世的一剂清醒剂。}

\lettrine{1.3} 郭林宗\myidx{郭泰}至汝南\footnote{郭林宗:(127—169):郭泰字林宗。汉末太原介休(今属山西)人。学识渊博,居家教授,生徒千数。尝游京师洛阳,与李膺友善,名震京师,士林宗望。还乡时,送者车千乘。以清议品题士人公卿,不入仕途而愈增其身价。},造袁奉高\myidx{袁阆}\footnote{袁奉高:袁阆字奉高,汉末汝南慎阳人。刘注引\CJKunderwave{汝南先贤传}作“袁闳”,余嘉锡\CJKunderwave{笺疏}证其误。按:袁宏字夏甫,袁安玄孙,史称安为汝阳人。},{\fzxk\zihao{6}\textcolor{red}{\CJKunderwave{续汉书}曰:“郭泰字林宗,太原介休人。泰少孤,年二十,行学至城阜屈伯彦精庐。乏食,衣不盖形,而处约味道,不改其乐。李元礼一见称之曰:‘吾见士多矣,无如休(林)宗者也。’及卒,蔡伯唱(喈)为作碑,曰:‘吾为人作铭,未尝不有惭容,唯为郭有道碑,颂无愧耳。’初,以有道君子征,泰曰:‘吾观乾象、人事,天之所废,不可支也。’遂辞以疾。”\CJKunderwave{汝南先贤传}曰:“袁闳(阆)字表(奉)高,慎阳人。友黄叔度于童齿,荐陈仲举于家巷。辟大尉掾,卒。”}} 车不停轨\footnote{车不停轨:指谈话不长即继续前行。轨,车辙。},鸾不辍轭\footnote{鸾不辍轭(è饿):车铃之声不断。辍,停止。轭,马具,状如“人”字形,驾车时套于牲口颈上以资牵引。}。诣黄叔度\myidx{黄宪},乃弥日信宿\footnote{弥日信宿:一连几天。弥日,整天。信宿,二夜连宿。}。人问其故,林宗曰:“叔度汪汪如万顷之陂\footnote{汪汪:深而广袤。万顷之陂(bēi杯):烟波浩渺,未可深测。陂,池塘,这里喻其广如湖泽。},澄之不清,扰之不浊,其器深广,难测量也\footnote{“澄之不清”以下四句:喻其人德行,高深难测。}。”{\fzxk\zihao{6}\textcolor{red}{\CJKunderwave{泰别传}曰:“薛恭祖问之,泰曰:‘奉高之器,譬诸氾滥,虽清易挹耳。’”}}

{\cangkai\zihao{5}【评】郭泰是汉末在野的士人清议领袖之一。当时的乡闾清议,一方面与举荐人才有关,一经品题,身价陡增;另一方面汉末社会极端腐败,宦官集团与外戚集团,此起彼落,争权夺利,进一步掀起党锢之祸,镇压正直士人,于是朝廷钳口,所以乡闾清议又转为对抗黑暗专制的舆论动员。郭泰离京,千乘相送,犹如一次抗议大游行,道理在此,说明人心所在。其称美黄宪“澄之不清,扰之不浊”,正是处于混浊之世,士人保持其高洁人格以示不与世俗同流合污的清醒认识。据\CJKunderwave{后汉书·郭泰传},宦官集团谋杀陈蕃等,泰悲恸而叹:“人之云亡,邦国殄瘁。”正说明当时士人对于国事的关心与无奈之心态。晋葛洪批评郭泰:“周旋清谈闾阎,无救于世道之凌迟”(见\CJKunderwave{抱朴子·正郭}篇),实在是不明形势的过激偏见。另,近人陈寅恪以为魏晋“清谈之风实由郭泰启之”,从理论思辨角度着眼,间接说明了汉末清议与魏晋清谈的渊源关系,也可另备一说(见万绳南整理\CJKunderwave{陈寅恪魏晋南北朝史讲演录},黄山书社1987年版,第45页)。}

\lettrine{1.4} 李元礼\myidx{李膺}风格秀整\footnote{李元礼(110—169):李膺字元礼,汉末颍川襄城(今属河南)人。在朝清议领袖之一,与杜密并称“李杜”。因反对宦官专政,被太学生称为“天下模楷”。后遭党锢之祸,死于狱中。},高目(自)标持\footnote{高目:诸本作“高自”,是。标持:犹标置。高自标持,即自我要求和评价都很高。},欲以天下名教是非为己任\footnote{名教:指儒家以正定名分为中心的传统礼教。}。{\fzxk\zihao{6}\textcolor{red}{薛莹\CJKunderwave{后汉书}曰:“李膺字元礼,颍川襄城人。抗志清妙,有文武隽才。迁司隶校尉,为党事自杀。”}} 后进之士,有升其堂者,皆以为登龙门。{\fzxk\zihao{6}\textcolor{red}{\CJKunderwave{三泰(秦)记}曰:“龙门,一名河津,去长安九百里。水悬绝,龟鱼之属莫能上,上则化为龙矣。”}}

{\cangkai\zihao{5}【评】\CJKunderwave{后汉书·党锢列传}序云:“逮桓、灵之间,主荒政缪,国命委于阉寺,士子羞与为伍,故匹夫抗愤,处士横议,遂乃激扬名声,互相题拂,品核公卿,裁量执政,婞直之风,于斯行矣。”于此可见汉末清议的风气及其政治影响。李膺是当时党人领袖之一,为了发扬清议以正世风,就必须注意培养人才,李膺奖拔士人,着眼于此。当时士人舆论,特别是太学生,几乎是以李膺之言为准的,所以有“李膺言出于口,人莫得违也”之说(见\CJKunderwave{太平御览}卷四四七引袁子正语)。士人得其赏识,自然身价百倍。跃登龙门之叹,比喻生动贴切。另,汉末清议影响举荐用人,后来又逐渐影响魏晋之世九品中正的品评,于是世家大族日渐成形,如经李膺品题的颍川陈寔、荀淑二氏,终成魏晋南北朝的高门士族。}

\lettrine{1.5} 李元礼\myidx{李膺}尝叹荀淑\myidx{荀淑}、锺皓\myidx{锺皓}\footnote{荀淑(83—149):汉末颍川颍阴(今河南许昌)人。好学而不为章句,见讥于俗儒。以德行及识人闻世。治事明理,人称“神君”。后弃官归隐。时贤李固、李贤等拜他为师。锺皓:汉末颍川长社(今河南长葛)人。隐居密山,敦\CJKunderwave{诗}、\CJKunderwave{书}而悦礼乐,教授门徒千馀人。与同郡陈寔、荀淑、韩韶称颍川四长。后为郡功曹,旋自劾去。公府征辟不赴。}{\fzxk\zihao{6}\textcolor{red}{,\CJKunderwave{先贤行状}曰:“荀淑字季和,颍川颍阴人也。所拔韦褐刍牧之中,执案刀笔之吏,皆为英彦。举方正,补朗陵侯相,所在流化。锺皓字季明,颍川长社人。父祖至德者(著)名。皓高风承世,除林虑长,不之官。人位不足,天爵有馀。”}} 曰:“荀君清识难尚\footnote{清识:识见清朗。尚:超过。},锺君至德可师。”{\fzxk\zihao{6}\textcolor{red}{\CJKunderwave{海内元(先)贤传}曰:“颍川先辈,为海内所师者:定陵陈锺(穉)叔,颍阴荀淑,长社锺皓。少府李膺宗此三君,常言:‘荀君清识难尚,陈、锺至德可师。’”}}

{\cangkai\zihao{5}【评】品题人物,是汉末清议的重要内容之一。此风一开,迅速蔓延,席卷了魏晋六朝。而汉末李膺、郭泰等,是其先驱。其所品题,高度概括而言约旨远。故经其品题即士林传诵,并非偶然。时代需要和个人的敏锐观察能力,都有关涉。}

\lettrine{1.6} 陈太丘\myidx{陈寔}诣荀朗陵\myidx{荀淑}\footnote{陈太丘:陈寔(104—187)字仲弓,汉末颍川许昌(今属河南)人。曾任太丘长,故云。其治政清明,百姓安业,以公正直名闻世。时人评云:“宁为刑罚所加,不为陈君所短。”党锢祸起,自请系狱。卒时远近赴吊,刊石立碑,谥文范。荀朗陵:荀淑曾任郎陵侯相,故云。},贫俭无仆役。{\fzxk\zihao{6}\textcolor{red}{\CJKunderwave{陈寔传}曰:“寔字仲弓,颍川陈(许)昌人。为闻喜令、太丘长,风化宣流。”}} 乃使元方\myidx{陈纪}将车\footnote{元方:即陈纪。以至德孝养闻。初,征辟不就。董卓入洛后,逼授五官中郎将,后官至尚书令、大鸿胪。年七十一,卒。将车:扶车前进。},{\fzxk\zihao{6}\textcolor{red}{\CJKunderwave{先贤行状}曰:“陈纪字元方,寔长子也。至德绝俗,与寔高名并著,而弟谌又配之。每宰府辟召,羔雁成群,世号三君,百城皆图画。”}} 季方\myidx{陈谌}持杖从后\footnote{季方:陈谌,寔少子。有令名而早卒。持杖:替父亲拿拐杖。},长文\myidx{陈群}尚小\footnote{长文:陈群(?—237)字。祖寔、父纪。孔融高才倨傲却与群交,由是显名。后参曹操丞相军事。入魏迁侍中、尚书,制九品官人法,形成一代门阀制度。后为司空、录尚书事,封颍阴侯,卒谥靖。},载著车中。既至,荀使叔慈\myidx{荀靖}应门\footnote{叔慈:荀靖(128—190)之字,淑第三子。少有俊才,动止以礼。卒,士人惜之,追谥玄行先生。应门:在门口应接宾客。},慈明\myidx{荀爽}行酒\footnote{慈明:荀爽(128—190)字,一名谞,淑第六子。幼而好学,早通经传,征辟不应。在荀淑八子中,人称“荀氏八龙,慈明无双”。著\CJKunderwave{诗传}、\CJKunderwave{易传}等。后官至司空。与司徒王允谋诛董卓,事未行而病卒。行酒:巡行劝酒。行,汉魏时常用语,犹赐也,即按客一一分送物品。},馀六龙下食\footnote{馀六龙:荀淑八子,人称八龙。除应门靖、行酒爽外,尚有俭、绲、焘、汪、肃、敷(旉)六人。下食:传送饭菜。},{\fzxk\zihao{6}\textcolor{red}{张璠\CJKunderwave{汉纪}曰:“淑有八子:俭、绲、靖、焘、汪、爽、肃、敷(旉)。淑居西豪里,县令苑康曰:‘昔高阳氏有才子八人。’遂署其里为高阳里。时人号曰八龙。”}} 文若\myidx{荀彧}亦小,坐箸䣛前\footnote{文若:荀彧(163—212)字。祖淑、父绲。少有才名,后为曹操的重要智囊谋士。其为人礼贤下士,知人善任,而德行兼备。官至尚书令,因忤曹操自杀。䣛:通“膝”。}。于时太史奏真人东行\footnote{太史:史官名,属太常。掌国史及天文历法。真人:得道之人。}。{\fzxk\zihao{6}\textcolor{red}{檀道鸾\CJKunderwave{续晋阳秋}曰:“陈仲弓从诸子侄造荀父子,于时德星聚,太史奏:‘五百里贤人聚。’”}}

{\cangkai\zihao{5}【评】汉魏之际,颍川人才济济,居中原之冠。如陈寔、荀淑等家族,均以德行著称,为人师表而图画百城。故事描摹二家德素,风景俨然,使一次普通的应酬宴会,化为宣扬贤人德行的“化妆”表演。所谓“太史奏真人东行”云者,不过是作者的狡狯之笔,夸显星象以应人事,目的仍在宣扬传统名教及贤人政治。但是,隐于故事背后,又是美中见刺的小说笔法,暗寓其激清扬浊的批判现实精神。还有,故事提到的两个小孙辈陈群和荀彧,后来居上,不仅官阶声名超越前辈,就是道德观念,也与父、祖有所不同。具体考察由陈寔至陈群,由荀淑至荀彧的颍川二族,又可见出“德行”标准的微妙变化。陈群和荀彧,后来都是曹操集团的重要骨干。宋朱熹于此大加挞伐:“且以荀氏一门而论之,则荀淑正言于梁氏用事之日,而其子爽已濡迹于董卓专命之朝,及其孙彧则遂为唐衡之婿、曹操之臣,而不知以为非矣。盖刚方志大之气,折于凶虐之馀,而渐图所以全身就事之计。”(见余嘉锡\CJKunderwave{笺疏}称引朱熹\CJKunderwave{答刘子澄书})按:朱子之讥,胶柱鼓瑟于汉儒传统名教之说,而无视时代的重大变化,实非的论。但其所言,却也揭示了从汉末到魏晋,有关“德行”观念的历史变化,即在祖孙之间,业已大异。以忠君为至德,弥近弥淡,故曹魏及司马二朝,儒者仕宦于篡弑相继之朝而不以为非。称孝而乏忠,这是魏晋时代道德的新油彩。}

\lettrine{1.7} 客有问陈季方\myidx{陈谌}\footnote{陈季方:即陈谌。}:{\fzxk\zihao{6}\textcolor{red}{\CJKunderwave{海内先贤传}曰:“陈谌字季方,寔少子也。才识博达。司空掾公车征,不就。”}} “足下家君太丘\myidx{陈寔}\footnote{足下家君:指谌父陈寔。足下,与人交谈或书信时用以敬称对方。太丘:指陈寔。},有何功德,而何天下重名\footnote{何:“荷”的古字,承担。诸本作“荷”。重名:高名,大名。}?”季方曰:“吾家君譬如桂树生泰山之阿\footnote{阿(ē婀):山隅,山坳。},上有万仞之高\footnote{仞:古代长度单位,一仞八尺。},下有不测之深;上为甘露所霑\footnote{霑(zhān沾):浸润。},下为渊泉所润。当斯之时,桂树焉知泰山之高、渊泉之深?不知有功德与无也!”

{\cangkai\zihao{5}【评】汉末清议品题之风,不仅盛于政坛,同时潜入家庭文化生活之中。以生动的修辞譬喻来品题父亲的高尚德行,陈谌引以自荣自傲,见其善为家族声誉做宣传。其桂生泰山之喻,高深难测之譬,形象具体而生动,有如亲临其境而目睹太丘风采。妙用意象,悟人甚多。颍川陈氏家族,寔议论不畏权贵,多直接的道德评议;而谌之品题,却开始汉末清议向魏晋审美意识方向的转化和过渡。其演变轨迹值得注意。对于诗赋文章修辞艺术的运用,也会产生积极影响。}

\lettrine{1.8} 陈元方\myidx{陈纪}子长文\myidx{陈群}有英才\footnote{陈元方:即陈纪。长文:陈群字,纪子。英才:英彦硕才。},{\fzxk\zihao{6}\textcolor{red}{\CJKunderwave{魏书}曰:“陈群字长文。祖寔尝谓宗人曰:‘此儿必兴吾宗。’及长,有识度,其所善,皆父党。”}} 与季方\myidx{陈谌}子孝先\myidx{陈忠}\footnote{季方:即陈谌。孝先:陈忠字,谌子。}{\fzxk\zihao{6}\textcolor{red}{\CJKunderwave{陈氏谙(谱)}曰:“谌子忠,字孝先。州辟不就。”}} 各论其父功德,争之不能决。咨于太丘\myidx{陈寔},太丘曰:“元方难为兄,季方难为弟。”{\fzxk\zihao{6}\textcolor{red}{一作“元方难为弟,季方难为兄”。}}

{\cangkai\zihao{5}【评】成语“难兄难弟”肇源于此。难者,不易也,这是在动态发展的行为比较中确立的概念。兄弟道德品行俱佳,原是可以同列齿并,难以轩轾。但人又是在时间的流逝中存在,犹如逆水行舟,不进则退。因此,兄弟二人都必须严格要求自己而不能有丝毫的松懈,一旦自满自傲,一方进而一方退,则优劣高下立判,怎可保持“难兄难弟”的齿列之位呢?太丘之言,富有人生哲理而颇耐咀嚼。不过,后世语词的引申发展,“难兄难弟”性质变异,另有二人同恶、难脱困境之意,“难”作“患难”解,意义由褒入贬,这又另当别论。}

\lettrine{1.9} 荀巨伯\myidx{荀巨伯}远看友人疾\footnote{荀巨伯:汉末颍川人。生平不详。},{\fzxk\zihao{6}\textcolor{red}{\CJKunderwave{荀氏家传}曰:“巨伯,汉桓帝时人也。亦出颍川,未详其始末。”}}值胡贼攻郡\footnote{胡贼:中原人对当时北方少数民族侵扰武装的蔑称。史称,桓帝永寿、延熹年间,乌桓、南匈奴、鲜卑诸部族武装多次侵袭边庭九郡,诸胡多为汉军所败,“惟鲜卑常自来自去”。故余嘉锡\CJKunderwave{笺疏}以为荀巨伯所遭遇的“胡贼”为鲜卑,疑是。具体时间、地点失载。}。友人语巨伯曰:“吾今死矣,子可去\footnote{子:你。古时第二人称代词。}。”巨伯曰:“远来相视,子令吾去。败义以求生,岂荀巨伯所行邪?”贼既至,谓巨伯曰:“大军至,一郡尽空。汝何男子,而敢独止\footnote{男子:非指男子汉,而是泛称毫无功名的白衣之士。止:停留。}?”巨伯曰:“友人有疾,不忍委之\footnote{委之:抛弃他,离开他。},宁以我身代友人命\footnote{宁:宁肯,甘愿。}。”贼相谓曰:“我辈无义之人,而入有义之国\footnote{国:此非指国家,而是泛称地方。古时多诸侯封国,故常以“国”称地方。}!”遂班军而还\footnote{班军:班师,撤军。},一郡并获全\footnote{全:安全,不破碎。}。

{\cangkai\zihao{5}【评】故事发生在汉末桓帝时,地点在北方边庭地区。故事使人联想起先儒那“舍生取义”的老话题。\CJKunderwave{孟子·告子上}云:“生,我所欲也;义,亦我所欲也。二者不可得兼,舍生而取义者也。”但古往今来,真能实行者有几?贼兵攻城,“一郡尽空”——守土有责者早已舍义求生而溜之大吉了。其实,不仅是守土有责的地方官吏,即是高高在上的朝廷名公巨卿,将军校尉,又有哪个站出来为国为民尽忠死战呢?汉末“胡贼”纵横,并非因敌人强大,而是朝廷内部腐败直接造成的,其罪责主要在这帮“败义以求生”的无耻之尤。但是天道未丧,“舍生取义”的高风亮节,在民间草莱见其薪火之传。荀巨伯之言,慷慨诚挚,掷地有声。他在战乱中坚持留下照顾病友,而全然没有考虑自己与病友,二者生命孰轻孰重的问题。在安危攸关的紧急关头,难道还要先去思考所救之人,是老、是少、是健康、抑或病人,是否值得伸出救援之手等冷酷的理念吗?唐韩愈\CJKunderwave{原道}云:“行而宜之之谓义。”只要是正义之路,就应该义无反顾地坚持到底。古人于荀巨伯,有“千古一朋”之颂,其心胸之坦荡,至今仍激动人心。}

\lettrine{1.10} 华歆\myidx{华歆}遇子弟甚整\footnote{华歆(huà xīn化欣)(157—231):汉魏之际平原高唐(今属山东禹城)人。汉献帝时官拜豫章刺史,为政清静不烦,吏民感受。后入拜尚书、侍中,代荀彧为尚书令。入魏,官至司徒。遇:对待。子弟:子侄后辈。整:严肃,整饬。},虽闲室之内\footnote{闲室:私室。},俨若朝典\footnote{俨:俨然,庄严整齐貌。\CJKunderwave{三国志}裴注引华峤\CJKunderwave{谱叙}曰:“每策大会,坐上莫敢先发言。歆时起更衣,则论议讙哗。歆能剧饮,至石馀不乱,众人微察,常以其整衣冠为异,江南号之曰‘华独坐’。”朝典:朝廷盛典。}。{\fzxk\zihao{6}\textcolor{red}{\CJKunderwave{魏志}曰:“歆字子鱼,平原高唐人。”\CJKunderwave{魏略}曰:“灵帝时,与北海邴原、管宁,俱游学相善,时号三人为一龙:谓歆为龙头,宁为龙腹,原为龙尾。”}} 陈元方\myidx{陈纪}兄弟恣柔爱之道\footnote{陈元方兄弟:指汉末以陈纪为首的颍川陈氏兄弟一家。恣:放任,听凭。}。而二门之里,两不失雍熙之轨焉\footnote{雍熙:和睦友善貌。轨:轨则,法度。}。

{\cangkai\zihao{5}【评】在三国时代,华歆是个人物。史称其“议论持平,终不毁伤人”,与昔日汉儒之直言极谏、杀身成仁异其旨趣,从而成为向魏晋“口不臧否人物”的清谈之风过渡的人物。\CJKunderwave{世说}称其“德行”,当然也就染有过渡时期的新风尚,而并非以尽忠朝廷皇帝为准的。\CJKunderwave{三国志}本传裴注引孙盛评曰:“歆既无夷、皓韬邈之风,又失王臣匪躬之操,故挠心于邪儒之说,交臂于陵肆之徒,位夺于一竖,节坠于当时,……咎孰大焉!”批判极其严厉。但华氏事二朝而“节坠”,具魏晋之特色。其“德行”为魏晋篡夺相继、弃旧迎新的时代风气使然,提倡孝而羞言忠,不足为怪。故\CJKunderwave{世说}以华歆登\CJKunderwave{德行}门,具体说明了不同时代各有其道德标准,汉儒传统观念也会产生动摇、发展和变化。}

\lettrine{1.11} 管宁\myidx{管宁}、华歆\myidx{华歆}共园中锄菜\footnote{管宁(158—241):汉魏之际北海朱虚(今山东临朐东南)人。史称其敬善陈寔。避乱辽东,聚邑讲学,“讲\CJKunderwave{诗}、\CJKunderwave{书},陈俎豆,饰威仪,明礼让,非学者无见也”。后返中原,朝廷征辟不就,以布衣终。},{\fzxk\zihao{6}\textcolor{red}{\CJKunderwave{傅子}曰:“宁字幼安,北海朱虚人。齐相管仲之后也。”}} 见地有片金,管挥锄与瓦石不异,华捉而掷去之\footnote{捉:拾起。}。又尝同席读书\footnote{同席:古时铺席而坐,“同席”即同坐一席。},有乘轩冕过门者\footnote{轩冕:古时公卿大夫所乘轩车和冕服。轩,古代一种前顶较高而有帷帐的车子,供贵族高官乘用。冕,泛指古时帝王或公卿大夫的礼帽。},宁读如故,歆废书出看。宁割席分坐\footnote{割席而坐:割断坐席,分开座位,以示志趣不同。后来引申为绝交。},曰:“子非吾友也!”{\fzxk\zihao{6}\textcolor{red}{\CJKunderwave{魏略}曰:“宁少恬静,常笑邴原、华子鱼有仕宦意。及歆为司徒,上书让宁。宁闻之,笑曰:‘子鱼本欲作老吏,故荣之耳。’”}}

{\cangkai\zihao{5}【评】汉魏之际是个大动荡的年代,各色人等,纷纷登台表演,其处世哲学,形形色色。管、华二人,各异旨趣。史称管宁睿智,料事准确,具预见性,故能避乱而善终。他一生淡泊功名,征辟不赴,而专心以讲学教育为务,从而获得了社会的尊敬。司空陈群上书朝廷颂其德行,云:“(宁)行为世表,学任人师,清俭足以激浊,贞正足以矫时。”人的一生,从小看八十。故事发生时间,当在管、华二人年轻同游京师国学之时。华志在功名,故轩冕轰然而废读出观,形象展示其歆慕富贵的内在心态;相反,管则无意功名而专心向学,其内在心境平静无波,轰然轩车和他岂生关涉?管氏后来备受尊敬,在年轻时已埋下成功的种子。\CJKunderwave{孟子·告子上}曰:“今夫弈之为数,小数也。不专心致志,则不得也。弈秋,通国之善弈者也。使弈秋诲二人弈:其一人专心致志,惟弈秋之为听;一人虽听之,一心以为鸿鹄将至,思援弓缴而射之,虽与之俱学,弗若之矣!为是其智弗若与?曰:非也。”借用孟子所讲故事来比喻管、华二人之读书,十分贴切。所谓“割席”,不必拘泥字面,如划线而坐,表示距离,也是“割席”之态。华之废书出观,思鸿鹄而射富贵,其致讥于史家,已在此见出端倪。至于锄地见金,管视而不见,非真不见,如佛家之“无心”,见其自然,故与瓦石无异;华则“捉而掷之”,其“捉”为真,“掷”则作伪作态,其恋财之心,矫饰之态,思绪流程,形象毕现。宋刘辰翁评曰:“捉掷未害其真,强生优劣,其优劣不在此。”似非的论。}

\lettrine{1.12} 王朗\myidx{王朗}每以识度推华歆\myidx{华歆}\footnote{识度:见识气度。王朗(?—228):本名严,东海郯城(今山东郯城)人。以通经拜郎中,后迁会稽太守。居郡惠爱于民。后被孙策所逐,北归曹操。入魏后官至司空,上疏劝育民省刑。曾为\CJKunderwave{易}、\CJKunderwave{春秋}、\CJKunderwave{孝经}、\CJKunderwave{周官}作传。}{\fzxk\zihao{6}\textcolor{red}{。\CJKunderwave{魏书}曰:“朗字景兴,东海郯人。魏司徒。”}} 歆蜡日\footnote{蜡(zhà诈)日:古代岁末合祭百神的重要节日,当时有会饮的风俗。},{\fzxk\zihao{6}\textcolor{red}{\CJKunderwave{礼记}曰:“天子大蜡八,伊耆氏始为蜡。蜡,索也。岁十二月,合聚万物而索飨之。”\CJKunderwave{五经要义}曰:“三代名腊:夏曰嘉平,殷曰清祀,周曰大蜡,总谓之腊。”晋博士张亮议曰:“蜡者,合聚百物索飨之,岁终休老息民也。腊者,祭宗庙五祀。\CJKunderwave{传}曰:‘腊,接也。祭则新故交接也。秦汉已来,腊之明日为初岁,古之遗语也。’”}} 尝集子侄燕饮\footnote{燕饮:宴会饮酒。燕,通“宴”。},王亦学之。有人向张华说此事\footnote{张华:范阳方城(今河北固安西北)人。博学多才,贯通今古,以诗赋文章称名于世。为晋武帝筹设灭吴方略,一统天下。惠帝时官至司空,死于八王之乱。}。张曰:“王之学华,皆是形骸之外\footnote{形骸之外:喻外在之物,而非内在实质。形骸,人的形体躯壳。},去之所以更远。”{\fzxk\zihao{6}\textcolor{red}{王隐\CJKunderwave{晋书}曰:“张华字茂先,范阳人也。累迁司空,而为赵王伦所害。”}}

{\cangkai\zihao{5}【评】在宋本中,此则与上则相连为一。但因其内容非一:上则褒管宁而贬华歆,此则誉华歆而讥王朗。毁誉不一,故据诸本分为二则。 魏晋之际,士人道德观念变化颇大。华歆之徒,于汉魏之替,威逼旧主,侍欢新朝,斯时清议,不以为异,仍然成为当时人们津津乐道的风流人物。贤如曹植,誉歆“志存太虚,安心玄妙。处平则以和养德,遭变则以义断事”(\CJKunderwave{辅臣论}),是个德义俱佳的典型。于此可见,魏晋士人于德行,另有不同于汉儒传统之标准。但是,余嘉锡\CJKunderwave{笺疏}于此大加挞伐,云:“自后汉之末,以晋六朝,诗人往往饰容止、盛言谈,小廉曲谨,以邀声誉。逮至闻望既高,四方宗仰,虽卖国求荣,又翕然以名德推之。华歆、王朗、陈群之徒,其作俑者也。……此其优劣,无足深论也。”余氏借他人之酒杯,浇自己的块垒,因其现代视角,发人生之浩叹,自有其合理成分。但移之古人,评价则未必公允,因时事推移,历史标准非一之故。如王朗,\CJKunderwave{三国志}裴注谓其高才博雅,严整慷慨,“常讥世俗有好施之名,而不恤穷贱,故用财以周急为先”。观朗为人,廉己济困,其德行较敛聚自养之徒,不可同日而语,岂能一笔抹煞?}

\lettrine{1.13} 华歆\myidx{华歆}、王朗\myidx{王朗}俱乘船避难\footnote{华、王避难事:程炎震据华峤\CJKunderwave{谱叙},以为“是献帝在长安时事。王朗方从陶谦于徐州,不得同行也”。见余氏\CJKunderwave{笺疏}称引。},有一人欲依附,歆辄难之\footnote{辄(zhé哲):则,就。难:刁难,拒绝。}。朗曰:“幸尚宽,何为不可?”后贼追至,王欲舍所携人。歆曰:“本所以疑,正为此耳。既已纳其自托\footnote{纳:接纳,接受。自托:把自己的安危托付别人。},宁可以急相弃邪\footnote{宁:岂,难道。}?”遂携拯如初\footnote{拯(zhěng整):拯救,援助。}。世以此定华、王之优劣。{\fzxk\zihao{6}\textcolor{red}{华峤\CJKunderwave{谱叙}曰:“歆为下邽令,汉室方乱,乃与同志士郑太等六七人避世。自武关出,道遇一丈夫独行,愿得与俱。皆哀许之。歆独曰:‘不可,今在危险中,祸福患害,义犹一也。今无故受之,不知其义,若有进退,可中弃乎?’众不忍,卒与俱行。此丈夫中道堕井,皆欲弃之。歆乃曰:‘已与俱矣,弃之不义。’卒共还,出之而后别。”}}

{\cangkai\zihao{5}【评】“言必信,行必果”(\CJKunderwave{论语·子路}),是孔夫子的教导,也是古人行“义”的一种传统美德。华歆阅历丰富,见多识广,做事预先估计到困难和特殊情况,而不轻于应允。此非心存不善,而如刘辰翁所评,是“阅世而后知其难”,具有先见之明。而一旦允诺,则言出如山而不可动摇,绝不能因自己有难就抛下难友而不顾,即使为此丧命,也将坚守诺言而义无反顾。小人则反之,浪言相招,急则相弃,言而无信,仁义不存。故李卓吾评云:“此君子小人之所以分也。彼平时爱买好,急则不顾。故凡买好者,皆非其心也。小人奉事不顾后,大率难以准凭,若此,国家将安得用之乎?”古人有“疾风知劲草”之言,在和平时期,可能大家相安无事;而一旦大难降临,则君子小人,各显其庐山真面目而优劣自分。}

\lettrine{1.14} 王祥\myidx{王祥}事后母朱夫人甚谨\footnote{王祥(184—268):魏晋时琅邪临沂(今属山东,琅邪,一作琅玡、琅琊)人。以至孝闻世,传统“二十四孝”有其“卧冰求鱼”故事及图画。汉末避乱庐江,后为徐州别驾。入魏官至司空,晋拜太保。祥及弟览,为琅邪王氏发达之始祖,后来如敦、导等皆其子孙。谨:恭谨,小心。}。{\fzxk\zihao{6}\textcolor{red}{\CJKunderwave{晋诸公赞}曰:“祥字休征,琅邪临沂人。”\CJKunderwave{祥世家}曰:“祥父融,娶高平薛氏,生祥。继室以庐江朱氏,生览。”\CJKunderwave{晋阳秋}曰:“后母数谮祥,屡以非理使祥,弟览辄与祥俱。又虐使祥妇,览妻亦趋而共之。母患。方盛寒冰冻,母欲生鱼,祥解衣,将剖冰求之,会有处冰小解,鱼出。”萧广济\CJKunderwave{孝子传}曰:“祥后母忽欲黄雀炙,祥念难卒致。须臾,有数十黄雀飞入其幕。母之所须,必自奔走,无不得焉。其诚至如此。”}} 家有一李树,结子殊好,母恒使守之。时风雨忽至,祥抱树而泣。{\fzxk\zihao{6}\textcolor{red}{肃(萧)广济\CJKunderwave{孝子传}曰:“祥后母庭中有李,始结子,使祥昼视鸟爵,夜则趁鼠。一夜,风雨大至,祥抱泣至晓,母见之恻然。”}} 祥尝在别床眠,母自往闇斫之\footnote{闇:通“暗”,暗中。斫:以刀斧砍杀。}。值祥私起\footnote{值:正巧。私起:起床小便。},空斫得被\footnote{空斫得被:扑空斫在被上。}。既还,知母憾之不已,因跪前请死。母于是感悟,爱之如己子。{\fzxk\zihao{6}\textcolor{red}{虞预\CJKunderwave{晋书}曰:“祥以后母故,陵迟不仕。年向六十,刺史吕虔檄为别驾。时人歌之曰:‘海、沂之康,寔(实)赖王祥;邦国不空,别驾之功。’累迁太保。”}}

{\cangkai\zihao{5}【评】史上王祥以孝著称,并因此成为魏晋显宦。魏晋之后,篡弑相继,改朝换代频繁。传统道德“忠孝”并称,至此“忠”字日渐淡出,惟留下一个“孝”字支撑道德门面。司马晋朝,因弑魏帝高贵乡公,更于“忠”君之事,讳莫如深,从此改倡“以孝治国”口号。正因时代的新需要而时来运转,王祥之孝,成为典范。王祥,\CJKunderwave{晋书}卷三三有传。其历仕三朝,政绩无闻,以“孝”名成为一个滑头政客而已。高贵乡公之难,他虽惺惺作态而口称“老臣无状”;但同时又接受司马恩典,入晋封侯拜相,依违两端而另结新欢,何“忠”之有?其临终遗命子孙,也只是“扬名显亲,孝之至也”之言,可见其意识深处,家族利益至为重要,而不见“忠孝”并称之名,这就揭示了魏晋六朝高门士族意识本质之特色。王祥之流位居台辅而“不预政事”,仅是司马氏“以孝治国”的政治标本而已。朝廷利用王祥之“孝”名,王祥同样也利用朝廷来兴盛其家族利益。后来琅邪王氏家族,衣冠极盛,而与两晋南朝相始终,祥与弟览,开创之功不可没。}

\lettrine{1.15} 晋文王\myidx{司马昭}称阮嗣宗\myidx{阮籍}至慎\footnote{晋文王:即司马昭(211—265),三国时河内温县(今属河南)人。懿次子,师同母弟。继兄师任魏之大将军,专擅朝政。灭蜀后,封晋公,加九赐,进位相国,已成篡魏开晋之势。后弑高贵乡公曹髦而立曹奂为帝,封晋王。死谥文,故称晋文王。阮籍(210—263):三国时陈留尉氏(今属河南)人。父瑀为建安七子之一,籍则为“竹林七贤”之一。曾官步兵校尉,故称“阮步兵”。当时著名的思想家及诗文名家,又是玄学清谈的代表人物。},每与之言,言皆玄远\footnote{玄远:玄妙高远。},未尝臧否人物\footnote{臧否(zāng pǐ赃痞):批评褒贬。}。{\fzxk\zihao{6}\textcolor{red}{\CJKunderwave{魏书}曰:“文王讳昭,字子上,宣帝第二子也。”\CJKunderwave{魏氏春秋}曰:“阮籍字嗣宗,陈留尉氏人,阮瑀子也。宏达不羁,不拘礼俗。兖州刺史王昶请与相见,终日不得与言。昶愧叹之,自以不能测也。口不论事,自然高迈。”李秉(康)\CJKunderwave{家诫}曰:“昔尝侍坐于先帝,时有三长史俱见,临辞出,上曰:‘为官长当清、当慎、当勤,修此三者,何患不治乎!’并受诏。上顾谓吾等曰:‘必不得已而去,于斯三者何先?’或对曰:‘清固为本。’复问吾,吾对曰:‘清慎之道,相须而成,必不得已,慎乃为大。’上曰:‘卿言得之矣。可举近世能慎者谁乎?’吾乃举故太尉荀景倩,尚书董仲达,仆射王公仲。上曰:‘此诸人者,温恭朝久,执事有恪,亦各其慎也。然天下之至慎者,其唯阮嗣宗乎!每与之言,言及玄远,而未尝评论时事,臧否人物,可谓至慎乎!’”}}

{\cangkai\zihao{5}【评】司马昭之评,见籍形骸而遗其内。籍著名\CJKunderwave{咏怀诗}嘲讽虚矫声势的伪善礼法之士云:“外厉贞素谈,户内灭芬芳。放口从衷出,复说道义方。委曲周旋仪,姿态愁我肠。”刻画入木三分。其\CJKunderwave{大人先生赋}讥伪善“君子”如寄居裤裆之群虱,“饥则啮人”,自以为有无穷之乐;而一旦“炎丘火流,焦邑灭都,群虱死于裈中而不能出,……悲夫”,又何尝不臧否人物?其内心之是非,明明白白。故当时礼法之士,疾之如仇。但内心之思想,难以作为刑法之根据,如政敌钟会之徒,数以时事问之,欲因其可否而致之罪。籍皆以醉酣无言而获免。\CJKunderwave{晋书}本传云:“籍本有济世志,属魏晋之际,天下多故,名士少有全者,籍由是不与世事,遂酣饮为常。”其外“至慎”,痛饮美酒称名士,正是出于政治上的自我保护意识,是英雄失路的一曲悲歌。司马昭誉之,实是对于当时士人“不慎”言行之警告。直至屠刀砍杀了籍友嵇康,人们才恍然大悟。}

\lettrine{1.16} 王戎\myidx{王戎}云\footnote{王戎(234—305):魏晋时琅邪人,王祥族人,当时清谈名士,“竹林七贤”之一。入晋官至尚书令、司徒。}:“与嵇康\myidx{嵇康}居二十年\footnote{嵇康(223—262):三国时谯郡铚(今安徽亳县)人。“竹林七贤”之一。曾任中散大夫,故称嵇中散。当时著名思想家、文学家、清谈名家。因其主张越名教而任自然,抨击礼法之士,不与司马氏统治集团合作,盛年被杀。},未尝见其喜愠之色\footnote{愠(yùn运):含怒,怨恨。}。”{\fzxk\zihao{6}\textcolor{red}{\CJKunderwave{康集叙}曰:“康字叔夜,谯国  人。”王隐\CJKunderwave{晋书}曰:“嵇本姓奚,其先避怨徙上虞,移谯国  县。以出自会稽,取国一支,音同本奚焉。”虞预\CJKunderwave{晋书}曰:“  有嵇山,家于其侧,因氏焉。”\CJKunderwave{康别传}曰:“康性含垢藏瑕,爱恶不争于怀,喜怒不寄于颜。所知王濬冲在襄城,面数百,未尝见其疾声朱颜。此亦方中之美范,人伦之胜业也。”\CJKunderwave{文章叙录}曰:“康以魏长乐亭主  ,迁郎中,拜中散大夫。”}}

{\cangkai\zihao{5}【评】嵇康与阮籍,史上并称“嵇阮”。\CJKunderwave{晋书}本传称康“性静寡欲,含垢匿瑕,宽简有大量。学不师受,博览无不该通,长好\CJKunderwave{老}、\CJKunderwave{庄},……故能越名教而任自然,……审贵贱而通物情”。其喜愠不形于色,并非没有自己的主张与独特个性,而是与阮籍有同样的苦衷。其内在个性之慷慨峻烈,远过于籍,却又不得不强加压抑,其内心悲愁之痛,有过于籍辈。故其\CJKunderwave{与山巨源绝交书}云:“阮嗣宗口不论人过,吾每师之而未能及,……至为礼法之士所绳,疾之如仇雠。”他临刑时作自责诗又云:“欲寡其过,谤议沸腾,性不伤物,频致怨憎。”以此,他虽与阮籍皆为一代天才,但其所遇,却没有阮籍幸运。只要统治者看不顺眼,即可杀一儆百,管你有什么德行与才干!}

\lettrine{1.17} 王戎\myidx{王戎}、和峤\myidx{和峤}同时遭大丧\footnote{和峤(?—292):魏晋时汝南西平(今属河南)人。官至中书令。为政清简得民,有风格,善礼法,朝野许其能正风俗人伦。家财富而性至吝,人称有“钱癖”。大丧:指父母之丧。据\CJKunderwave{晋书}戎传,时戎遭母丧,而峤遭父丧。},俱以孝称。王鸡骨支床\footnote{鸡骨支床:形容骨瘦如柴而憔悴倚床。},和哭泣备礼。{\fzxk\zihao{6}\textcolor{red}{\CJKunderwave{晋诸公赞}曰:“戎字濬冲,琅邪人,太保祥宗族也。文皇帝辅政,锺会荐之曰:‘裴楷清通,王戎简要。’即俱辟为掾。晋践祚,累迁荆州刺史,以平吴功,封安丰侯。”\CJKunderwave{晋阳秋}曰:“戎为豫州刺史,遭母忧,性至孝,不拘礼制,饮酒食肉,或观棋弈,而容皃(貌)毁悴,杖而后起。时汝南和峤,亦名士也,以礼法自持。处大忧,量米而食,然憔悴哀毁,不逮戎也。”}}武帝\myidx{司马炎}谓刘仲雄\myidx{刘毅}\footnote{武帝:指晋武帝司马炎(236—290)。刘仲雄(?—285):刘毅字仲雄。魏晋时东莱掖(今山东莱州市)人。官至尚书左仆射。性方正謇忠。曾当面讥晋武帝为汉之桓、灵二帝。主张废九品中正制度,未果。}:{\fzxk\zihao{6}\textcolor{red}{王隐\CJKunderwave{晋书}曰:“刘毅字仲雄,东莱不夜(掖)人,汉城阳景王后也。亮直清方,见有不善,必评论之。王公大人,望风惮之。侨居阳平,太守杜恕致为功曹,沙汰郡吏三百馀人。三魏佥曰:‘但闻刘功曹,不闻杜府君。’累迁尚书司隶校尉。”}} “卿数省王、和不\footnote{省:探望。卿:第二人称代名词。徐震堮\CJKunderwave{校笺}附录\CJKunderwave{世说新语词语简释}云:“下于己者或侪辈间亲暱而不拘礼数者称‘卿’。”}?闻和哀苦过\footnote{过:过度。据\CJKunderwave{晋书}及诸本“过”下有“礼”字。},使人忧之。”仲雄曰:“和峤虽备礼,神气不损;王戎虽不备礼,而哀毁骨立。臣以和峤生孝,王戎死孝\footnote{生孝:尽孝而无害于健康。死孝:哀毁尽孝而伤身。}。陛下不应忧峤\footnote{陛下:臣下对帝王的尊称。},而应忧戎。”{\fzxk\zihao{6}\textcolor{red}{\CJKunderwave{晋阳秋}曰:“世祖及时谈以此贵戎也。”}}

{\cangkai\zihao{5}【评】司马开晋,因其篡弑,故于“忠义”二字,讳莫如深。但作为朝廷国家,总要有其思想伦理作支撑,万般无奈之际,提倡“以孝治国”作为门面。标榜王戎辈之“生孝”、“死孝”者以此。王戎其人,虽忝为“竹林七贤”之末,但论其德行,与嵇、阮相去甚远。戎位居台辅,而性好聚敛,何德于民?史称其“与时舒卷,无謇谔之节。自经典选,未尝进寒素,退虚名,但与时浮沈,户调门选而已”。傅咸曾上疏严劾,请免其官,上不从。晋武帝重戎之“孝”,实是别有用心,为当时的士族门阀统治,修建牌坊门面而已。另外,无论是和峤备礼而泣的“生孝”,或王戎鸡骨支床的“死孝”,若与“孺子终日啼而不隘,和之至也”(\CJKunderwave{王阳明全集}卷二七\CJKunderwave{与许台仲书})之童心相比较,其虚伪炒作之内诈,自然暴露无遗了。}

\lettrine{1.18} 梁王\myidx{司马彤}、赵王\myidx{司马伦}\footnote{梁王:即司马彤(tóng同),懿子。永康初,与赵王伦共废贾后。赵王伦篡位,为阿衡。死后议谥,博士蔡克责其“国乱不能匡,主颠不能扶”,谥号曰灵。赵王:即司马伦,懿第九子。废贾后,旋即篡位称帝,兴兵与诸王战,兵败诛灭,实为八王之乱罪魁祸首。},{\fzxk\zihao{6}\textcolor{red}{朱凤\CJKunderwave{晋书}曰:“宣帝张夫人生梁孝王彤,字子徵(徽),位至太宰。栢夫人生赵王伦,字子彝,位至相国。”}} 国之近属\footnote{近属:近亲。按:梁、赵二王,皆为武帝叔父。},贵重当时。裴令公\myidx{裴楷}{\fzxk\zihao{6}\textcolor{red}{\CJKunderwave{晋诸公赞}曰:“裴楷字叔则,河东闻喜人,司空秀之从弟也。父徽,冀州刺史,有俊识。楷特精\CJKunderwave{易}义。累迁河南尹、中书令,以卒。”}} 岁请二国租钱数百万\footnote{裴令公:即裴楷,曾官中书令,故云,又称“裴令”。善\CJKunderwave{老}、\CJKunderwave{易},当时著名清谈名家。二国租钱:指从梁、赵二国税收所获钱财。},以恤中表之贫者\footnote{恤:抚恤,救济。}。或讥之曰:“何以乞物行惠\footnote{乞物行惠:乞讨钱财以施恩惠于人。}?”裴曰:“损有馀、补不足,天之道也\footnote{“损有馀”以下二句:裴楷为玄学家,精于\CJKunderwave{易}、\CJKunderwave{老}之学。\CJKunderwave{周易}中有\CJKunderwave{损}、\CJKunderwave{益}二卦,\CJKunderwave{益卦·彖传}云:“损上益下,民说无疆。”\CJKunderwave{老子}云:“天之道其犹张弓乎?高者抑之,下者举之;有馀者损之,不足者与之。天之道,损有馀而补不足。”其思想观念肇源于此。}。”{\fzxk\zihao{6}\textcolor{red}{\CJKunderwave{名士传}曰:“楷行己取与,任心而动,毁誉虽至,处之晏然。皆此类。”}}

{\cangkai\zihao{5}【评】在晋初玄家中,裴楷素有“清通”之名。所谓“清通”,就是思维清明,见识通达,言行不为传统礼法名教所拘束。这与当时玄家的理论主张及生活态度有关。在玄风熏陶之下,玄家名士对“德行”有自己的独特认识。向权贵乞讨钱物,有悖传统道德;但乞物以恤贫贱,则是替天行道而另当别论。在当时新玄家看来,损有馀以补不足,顺自然而合大道。体则天道,讲究实惠,就是至德,何羞之有?实际上,如\CJKunderwave{易·益}卦彖辞所说,适当地“损上益下”,可以达到“民说(悦)无疆”的新境界,这不是更有利于巩固封建统治吗?可惜古代的统治者大多鼠目寸光,思想如裴楷之“清通”者,寥若晨星,他们反其道而行之,大多重在“损下益上”,以掊克聚敛为急务,而置民于水火之中,天下怎能不乱?西晋速亡,原因很多,但统治者竞豪奢而务聚敛,无情敲剥百姓,也是重要原因之一。裴楷“损有馀,补不足”之言,冲口而出,情真自然,读之能无思乎!}

\lettrine{1.19} 王戎\myidx{王戎}云:“太保\myidx{王祥}居在正始中\footnote{太保:王祥官拜太保,故云。正始:魏齐王芳年号(240—248)。},不在能言之流\footnote{能言之流:特指当时如何晏、王弼之流的玄家清谈名士。}。及与之言,理中清远\footnote{理中清远:道理适中而清新玄远。中,六朝人口语,事理得当之称。},将无以德掩其言\footnote{将无:魏晋口语,与“得无”同,犹言“莫非”,表示模糊肯定之意。}!”{\fzxk\zihao{6}\textcolor{red}{\CJKunderwave{晋阳秋}曰:“祥少有美德行。”}}

{\cangkai\zihao{5}【评】如前所述,王祥身为三公而政绩无闻,是否因其能力低劣?非也。清谈领袖王戎誉其“理中清远”,并非浪言。处在当时改朝换代的大动荡年代,司马集团屠戮名士如何晏、夏侯玄、嵇康之辈,极其惨酷,绝不手软。正因明了形势,所以王祥虽具能言内质,外表却偏是拱默装呆。这实是光华敛尽以求明哲保身的人生态度。祥之“德行”,如此而已。不呆装呆,能言无言,这不仅是个人,更可见出时代的悲哀。}

\lettrine{1.20} 王安丰\myidx{王戎}遭艰\footnote{王安丰:王戎因平吴之功,封安丰侯,故称。遭艰:犹丁艰,指遭父母之丧。},至性过人\footnote{至性:指至孝之性。}。裴令\myidx{裴楷}往吊之\footnote{裴令:指裴楷。},曰:“若使一恸果能伤人\footnote{恸(tòng痛):悲痛大哭。伤人:伤害健康。},濬冲必不免灭性之讥\footnote{灭性:只因悲痛过度而伤身害命。}。”{\fzxk\zihao{6}\textcolor{red}{\CJKunderwave{曲礼}曰:“居丧之礼,毁瘠不形,视听不衰。不胜丧,乃比于不慈不孝。”\CJKunderwave{孝经}曰:“毁不灭性,圣人之教也。”}}

{\cangkai\zihao{5}【评】如前所述,王戎以“死孝”流誉士林,史称“世祖(武帝)与时谈以此贵之”,这与晋朝“以孝治国”的方略有关。上有所好,则下必甚焉,所以会有“死孝”反常现象出现。但纵观王戎一生,以自我为中心,聚敛成性,刻薄下民,怎会产生因孝致死的念头呢?王戎、裴楷,同为玄学清谈名家,但相比之下,裴之“清通”难及。站在新玄学立场,裴楷寥寥数语,一针见血地戳穿了世俗礼教的虚伪。以裴楷之分析,王戎“死孝”只能有以下两种可能:一是孝子本非真实想死,而只是做给人看,因矫饰以邀盛名,故裴楷“若使一恸果能伤人”之语,使用的是假设句,并非真有其事,言外之意,斥其虚伪;一是若真的因其至性而亡,则“不免灭性之讥”,违背人性自然,连生命都不知爱惜,还有什么好赞扬的呢?无论其“死孝”是真是假,二律背反,王戎及其所代表的世俗道德观念,在玄家的眼光中,都免不了原形毕现。}

\lettrine{1.21} 王戎\myidx{王戎}父浑\myidx{王浑(琅邪)}有令名\footnote{王浑:晋初有二王浑:一是晋阳王浑,字玄冲,官至司徒,封京陵侯;一是琅邪王浑,字长原,官凉州刺史,封贞陵亭侯。这里指后者。令名:美好声名。},官至凉州刺史\footnote{凉州:古地名,汉置十三刺史部之一,辖境相当于今甘肃、宁夏和青海湟水流域、内蒙古纳林河、穆林河流域。魏晋时治所姑臧(今甘肃武威)。}。{\fzxk\zihao{6}\textcolor{red}{\CJKunderwave{世语}曰:“浑字长原,有才望。历尚书、凉州刺史。”}} 浑薨\footnote{薨(hōng轰):古时诸侯贵族或高官显爵之死称薨。},所历九郡义故\footnote{九郡:据\CJKunderwave{晋书·地理志},武帝时凉州辖八郡:金城、西平、武威、张掖、西郡、酒泉、敦煌、西海。至惠帝元康五年,分敦煌、酒泉二郡地,别立晋昌郡,方称九郡。故余嘉锡\CJKunderwave{笺疏}据\CJKunderwave{太平御览}卷五五○引作“州郡”,疑是。义故:义从故吏,指随从部曲及故旧属吏。},怀其德惠\footnote{德惠:恩泽惠泽。},相率致赙数百万\footnote{致赙(fù付):赠送丧仪。赙,以财物助人办理丧事。},戎悉不受。{\fzxk\zihao{6}\textcolor{red}{虞预\CJKunderwave{晋书}曰:“戎由是显名。”}}

{\cangkai\zihao{5}【评】在\CJKunderwave{世说}中,王戎是个风流人物,其“成功”看来并非偶然,或者和他早知宣传自己的炒作之术有关。至少,他借死去的父母为自己拉了不少“广告”。前述“死孝”的表演,即是一例。其却父丧之赙,也有两种可能:一是戎年轻时尚未贪鄙成性,后来之贪,是生活大染缸所致,说明其性格及人生道路,有个发展过程;一是颇具远见的“广告”意识。戎性本狡狯,贪大不贪小,琅邪王家,魏晋时已成高门望族,家底丰厚,区区丧葬费之赠,又何足道哉!故却赙以邀誉,陶珙评其“第欲显名,刻意自苦”,史称戎“由是显名”,信然。于此可见,戎自年轻时即颇工心计,善于推销自己,为自己未来的“成功”作开拓和努力。这和今天社会上常见的自我“炒作”差不多,请读者不要看花了眼。}

\lettrine{1.22} 刘道真\myidx{刘宝}尝为徒\footnote{刘道真:刘宝字道真。少贫贱,“常渔草泽,善歌啸,闻名莫不留连”(见\CJKunderwave{世说·任诞})。为司马骏赏拔,后成为士人领袖而与王衍齐名,一经其品题,身价陡增。故陆机入洛之初,张华以为其所宜拜访者,“刘道真是其一”。可见当时刘宝在士林中的声望。为徒:罚为刑徒。},{\fzxk\zihao{6}\textcolor{red}{\CJKunderwave{晋百官名}曰:“刘宝字道真,高平人。”徒,罪役作者。}} 扶风王骏\myidx{司马骏}{\fzxk\zihao{6}\textcolor{red}{虞预\CJKunderwave{晋书}曰:“骏字子臧,宣帝第十七子,好学至孝。”\CJKunderwave{晋诸公赞}曰:“骏八岁为散骑常侍,侍魏齐王讲。晋受士,封扶风王,镇关中,为政最美。薨,赠武王。西土思之,但见其碑赞者,皆拜之而泣。其遗爱如此。”}} 以五百疋布赎之\footnote{扶风王骏:即司马骏。刘注谓“宣帝第十七子”,误。据\CJKunderwave{晋书}卷三八\CJKunderwave{宣王传},司马懿生九子,骏第七。十七子应为七子之讹。疋:通“匹”。赎:赎罪,这里指以布匹财物抵罪,赎回人身自由。},既而用为从事中郎\footnote{从事中郎:官名,魏晋时节镇将帅的幕僚。}。当时以为美事。

{\cangkai\zihao{5}【评】故事主角是扶风王司马骏。魏晋实行九品中正制,是个门阀社会,士庶之别,实有天渊之隔。刘宝出身贫寒,官场之上,原无置足之地。但司马骏慧眼卓识,赏拔于草莱刑徒队中,确非常人能及。后来高平刘氏,自宝之后,又衍为山东兖州的高门士族,一时传为美谈。史称司马骏幼极聪慧好学,能著论,文有可称,及长,又能“清贞自守,宗室之中最为隽望”。后因忠言直谏忤武帝意,“遂发病薨”,很可能因政见不同而被逼致死。其赏拔寒隽,拔于刑徒,冲破门阀偏见,确非容易。可见司马统治集团中人,并非尽皆昏庸,当与其好学深思有关。惜武帝不用其良,西晋速亡,不亦宜乎!}

\lettrine{1.23} 王平子\myidx{王澄}、胡毋彦国\myidx{胡毋辅之}诸人\footnote{王平子:即王澄(267—312)。出自琅邪王氏。兄衍为西晋士林清谈领袖,誉澄“阿平第一”。有士人“经澄所题者,衍不复有言,辙云‘已经平子矣’”。澄由是显名于世。澄官荆州刺史,日夜纵酒,不以军政为意。曾残杀巴蜀流民,激起民变。后因故为王敦所杀。胡毋彦国:即胡毋辅之,晋清谈名士。史称有知人之鉴。性嗜酒,任纵不拘小节。与王澄、王敦、庾敳俱为太尉王衍亲昵,号称“四友”。永嘉乱后,南渡卒于湘江刺史任上。},皆以任放为达\footnote{放任为达:以放纵率性为通达。},或有裸体者。{\fzxk\zihao{6}\textcolor{red}{\CJKunderwave{晋诸公赞}曰:“王澄字平子,有达识,荆州刺史。”\CJKunderwave{永嘉流人名}曰:“胡毋辅之字彦国,泰山奉高人,湘州刺史。”王隐\CJKunderwave{晋书}曰:“魏末,阮籍嗜酒荒放,露头散发,裸袒箕踞。其后贵游子弟阮瞻、王澄、谢鲲、胡毋辅之之徒,皆祖述于籍,谓得大道之本。故去巾帻,脱衣服,露丑恶,同禽兽,甚者名之为通,次者名之为达也。”}} 乐广\myidx{乐广}笑曰\footnote{乐广(?—304):字彦辅,南阳淯阳(今河南南阳东南)人。少孤贫,寒素为业,与物无竞。其清谈析理,与王衍并称,卫瓘以为有正始遗风。官至尚书令,八王乱中,以故忧卒。}:“名教中自有乐地\footnote{名教:指以儒家正名定分的传统礼教。乐地:快乐境地。},何为乃尔也\footnote{何为乃尔也:为什么竟然如此呢?尔,如此,这样。}?”

{\cangkai\zihao{5}【评】魏晋以后,儒家名教的思想统治日渐衰落,当时部分清谈玄家名士,以老庄自然之道相抗衡,其趋极端者,率性任诞,裸裎为达,倾向于以人生态度的离奇放荡,来表现自己所企求的超凡脱俗的浪漫情调,并且常以违名教而任自然相号召。但同样作为玄家清谈名士乐广则不走此极端,其思想立场与郭象暗合,主张儒、玄双修,折衷于儒家名教与玄学自然之间,以为儒家之学未可尽去,如能灵活对待而时出新解,则也自有“乐地”。儒、玄双修,调和名教和自然,正体现了魏晋统治者的思想要求。与郭象相比,郭以著书注\CJKunderwave{庄}在哲学及理论思辨方面影响很大;乐广则因善言辞而不便笔墨,以清谈立世,随着时间流逝,其思想消逝于历史之中。著书与口谈,优劣自显。不过,从乐广的话中,透露了当时玄家清谈阵营也甚为复杂,对待儒家名教时有不同态度和意见。玄学清谈,从行为方式到思想观念,也是千姿百态,各行其道。

另,\CJKunderwave{任诞}第13则“阮浑长成”刘注引戴逵\CJKunderwave{竹林七贤论}曰:“乐令之言有旨哉!谓彼非玄心,徒利其纵恣而已。”谓作达任诞者为不懂玄学,亦可备参考。}

\lettrine{1.24} 郗公\myidx{郗鉴}值永嘉丧乱\footnote{郗公:郗鉴(269—339),字道徽,晋高平金乡(今属山东)人。东晋初官至司徒、进位太尉,位至朝廷三公,故称。明帝时,鉴都督扬州,牵制王敦;成帝时,平祖约、苏峻有功。永嘉:晋怀帝司马炽年号(307—312)。永嘉五年,匈奴族武装攻陷京师洛阳,怀帝被俘。史称永嘉之乱,不久西晋亡。},在乡里,甚穷馁\footnote{穷馁(něi):穷困饥饿。}。乡人以公名德\footnote{名德:名望道德。},传共饴之\footnote{饴(sì四):通“饲”,饲养。}。公常携兄子迈\myidx{郗迈}及外生周翼\myidx{周翼}二小儿往食\footnote{外生:即“外甥”。}。乡人曰:“各自饥困,以君之贤,欲共济君耳,恐不能兼有所存。”公于是独往食,辄含饭着两颊边\footnote{着:置放。颊(jiá夹):面颊,脸的两侧。},还,吐与二儿。后并得存,同过江\footnote{江:指长江。按:永嘉乱后,中原士人纷纷渡江南下避乱。}。{\fzxk\zihao{6}\textcolor{red}{\CJKunderwave{郗鉴别传}曰:“鉴字道徽,高平金乡人,汉御史大夫郗虑后也。少有体正,耽思经籍,以儒雅著名。永嘉末,天下大乱,饥馑相望。冠带以下,皆割己之资供鉴。元皇征为领军,迁司空、太尉。”\CJKunderwave{中兴书}曰:“鉴兄子迈,字思远。有干世才略,累迁少府、中护军。”}} 郗公亡,翼为剡县\footnote{为剡(shàn善)县:当剡县令。剡县,在今浙江嵊州。},解职归,席苫于公灵床头\footnote{苫(shān衫):居丧草垫。},心丧终三年\footnote{心丧:心中悼念,孝子之外的一种守丧之礼。}。{\fzxk\zihao{6}\textcolor{red}{\CJKunderwave{周氏谱}曰:“翼字子卿,陈郡人。祖弈,上谷太守。父优,车骑咨议。历剡令、青州刺史、少府卿。六十四而卒。”}}

{\cangkai\zihao{5}【评】在丧乱饥馑之时,饿殍遍野,千金易得而一饭难求。而郗鉴却能含饭吐哺二儿,绝不快活独饱,这与\CJKunderwave{庄子}寓言中的相濡以沫,如出一辙,见主人公真人性。故事以一典型细节,状人物生动形象,可资创作借鉴。又刘辰翁评云:“两颊所箸能几,足哺二儿?儿非甚小,在谷气不绝耳,哀哉!”一方面道出了时代悲剧,一方面又点明了生命之顽强。因其谷气不绝而获生,一口之饭,岂是琐事!其功德胜造七级浮图。}

\lettrine{1.25} 顾荣\myidx{顾荣}在洛阳\footnote{顾荣(?—312):两晋之际江南士族领袖之一,与陆机、陆云同时入洛,时称“三俊”。南渡后,代表江南士族拥护和支持司马睿在江南开国,是为东晋。洛阳:西晋京师。},尝应人请,觉行炙人有欲炙之色\footnote{行炙人:端送烤肉之人。炙,烤肉。但后一“炙”名词动化,欲炙,想吃烤肉。},因辍己施焉\footnote{辍:停止。}。同座嗤之\footnote{嗤:嘲笑。}。荣曰:“岂有终日执之,而不知其味者乎?”后遭乱渡江,每经危急,常有一人左右己\footnote{左右:扶持,保护。}。问其所以,乃受炙人也。{\fzxk\zihao{6}\textcolor{red}{\CJKunderwave{文士传}曰:“荣字彦先,吴郡人。其先越王勾践之支庶,封于顾邑,子孙遂氏焉,世为吴著姓。大父雍,吴丞相。父穆,宜都太守。荣少朗俊机警,风颖标彻。历廷尉正。曾在省与同僚共饮,见行炙者有异于常仆,乃割炙以啖之。后赵王伦篡位,其子为中领军,逼用荣为长史。及伦诛,荣亦被执,凡受戮等辈十有馀人。或有救荣者,问其故,曰:‘某省中受炙臣也。’荣乃悟而叹曰:‘一餐之惠,恩今不忘,古人岂虚言哉!’”}}

{\cangkai\zihao{5}【评】“德行”一词,德在内而行在外,外现的行为是受内在意识的指挥。当然,思想意识又有自觉与不自觉之分。顾荣看到整天端着热腾腾、香喷喷的烤肉奔跑的人,却一口也享受不到,于是推己及人,“辍己施焉”,这一行动应该是受内在潜意识的驱动,是一种自然而然而不想回报的“无心”之举。这说明儒家传统道德中的恕道,已在他心中生根发芽,无所不在。魏晋之际,中原流行玄学,而江南则“服膺儒学”,所受学术风气熏染有异。顾荣不想回报而有好报,又说明了人性良心所在。}

\lettrine{1.26} 祖光禄\myidx{祖纳}少孤贫\footnote{祖光禄:即祖纳,曾官光禄大夫,故称。纳,一作讷。与闻鸡起舞,中流击楫的祖逖为异母兄弟。},性至孝,常自为母炊爨作食\footnote{炊爨(cuàn窜)作食:烧火做饭。}。{\fzxk\zihao{6}\textcolor{red}{王隐\CJKunderwave{晋书}曰:“祖讷字士言,范阳遒人。九世孝廉。讷诸母三兄,最治行操,能清言。历太子中庶子、廷尉卿。避地江南,温峤荐为光禄大夫。”}} 王平北\myidx{王乂}闻其佳名\footnote{王平北:即王乂,曾任平北将军,故称。佳名:美好名声。},以两婢饷之\footnote{饷:赠。},因取为中郎\footnote{中郎:官名,即从事中郎,诸王、节镇或州郡属官。}。{\fzxk\zihao{6}\textcolor{red}{\CJKunderwave{王乂别传}曰:“乂字叔元,琅邪临沂人。时蜀新平,二将作乱,文帝西之长安,乃征为相国司马,迁大尚书,出督幽州诸军事,平北将军。”}} 有人戏之者曰:“奴价倍婢\footnote{奴:一指男性奴仆,一是对人的鄙称。这里二义双关并用。}。”祖云:“百里奚亦何必轻于五羖之皮耶\footnote{“百里奚”句:祖纳称引古贤故事以自况。羖(gǔ古):黑色公羊。}!”{\fzxk\zihao{6}\textcolor{red}{\CJKunderwave{楚国先贤传}曰:“百里奚字井伯,楚国人。少仕于虞,为大夫。晋欲假道于虞以伐虢,谏而不听,奚乃去之。”\CJKunderwave{说苑}曰:“秦穆公使贾人载盐于虞,诸贾人买百里奚以五羊皮。穆公观盐,怪其牛肥,问其故,对曰:‘饮食以时,使之不暴,是以肥也。’公令有司沐浴衣冠之,公孙支让其卿位,号曰五羖大夫。”}}

{\cangkai\zihao{5}【评】详味故事,表面上是因祖纳侍母至孝而入\CJKunderwave{德行}门;但故事的重心实在后半段“有人戏之”后祖纳之自我解嘲,故明王世懋以为应“入于排调”之门。其实,祖纳称引古贤百里奚的故事以自喻,正见其内心的自我价值评价。人讥其“奴价倍婢”——你只值两个婢女之价,但祖纳不在乎人们的无知嘲讽,他心知肚明自己的价值,即在于治国安邦、辅助霸业的理想,并不以一时之屈曲而湮没自己内在的理想光辉。顽强坚持自己的道德理想,岂非至性至德!}

\lettrine{1.27} 周镇\myidx{周镇}罢临川郡\footnote{临川郡:古郡名,今属江西。罢:有二义,一是罢免,一是结束。按:据刘注谓“所在有异绩”,则非因过罢官,而是任期已到而结束郡务。后述“王丞相(导)往看之”,即是内证。都:京都,指东晋京师建业(今南京)。},还都,未及上\footnote{上:上任。一般注为上岸。但详味周镇并非犯过罢官,实是任期满而回都听候调选,前任已罢而后任未接,故称“未及上”。},住泊青溪渚\footnote{住泊:驻泊。清溪渚:清溪水岸。清溪是三国东吴所开河渠,在建业附近,是通往京师的重要漕运河道。}。{\fzxk\zihao{6}\textcolor{red}{\CJKunderwave{永嘉流人名}曰:“镇字康时,陈留尉氏人也。祖父和,故安令。父震,司空长史。”\CJKunderwave{中兴书}曰:“清约寡欲,所在有异绩。”}} 王丞相\myidx{王导}往看之\footnote{王丞相:指王导。}{\fzxk\zihao{6}\textcolor{red}{。\CJKunderwave{丞相别传}曰:“王导字茂弘,琅邪人。祖览,以德行称。父裁,侍御史。导少知名,家世贫约,恬畅乐道,未尝以风尘经怀也。”}} 时夏月,暴雨卒至\footnote{夏月:夏天。卒:通“猝”,突然。},舫至狭小\footnote{舫(fǎng仿):船。},而又大漏,殆无复坐处。王曰:“胡威\myidx{胡威}之清\footnote{清:清廉、廉洁。},何以过此!”即启用为吴兴郡\footnote{吴兴郡:古郡名,治所乌程(今浙江吴兴县)。}。{\fzxk\zihao{6}\textcolor{red}{\CJKunderwave{晋阳秋}曰:“胡威字伯虎,淮南人。父质,以忠清显。质为荆州,威自京师往省之。及告归,质赐威绢一疋。威跪曰:‘大人清高,于何得此?’质曰:‘是吾俸禄之馀,故以为汝粮耳。’威受而去。每至客舍,自放驴,取樵爨炊,食毕,复随旅进道。质帐下都督阴赍粮要之,因与为伴,每事相助经营之,又进少饭。威疑之,客诱问之,乃知都督也。谢而遣之。后以白质,质杖都督一百,除其吏名。父子清慎如此。及威为徐州,世祖赐见,与论边事及平生。帝叹其父清,因谓威曰:‘卿清孰与父?’对曰:‘臣清不如也。’帝曰:‘何以为胜汝邪?’对曰:‘臣父清畏人知,臣清畏人不知,是以不如远矣。’”}}

{\cangkai\zihao{5}【评】俗话说:“无官不贪。”概括了古代官场的贪赃腐败。在此大形势下,激清扬浊,就是政界的上等德政。想来当时周镇之“清”,是有些名声的,所以当他驻泊清溪船上时,王丞相及时赶去看望,如果他是犯错罢官,作为朝廷宰辅,王导岂有此举?东晋开国江南,以清官相号召,以便团结民众,共同抗敌,同时见出了王导的远见和德政。}

\lettrine{1.28} 邓攸\myidx{邓攸}始避难\footnote{避难:此指避永嘉之乱,时邓攸被匈奴族石勒部俘虏,后逃出。},于道中弃己子、全弟子\footnote{全:保全。}。{\fzxk\zihao{6}\textcolor{red}{\CJKunderwave{晋阳秋}曰:“攸字伯道,平阳襄陵人。七岁丧父母及祖父母,持重九年。性清慎平简。”邓粲\CJKunderwave{晋纪}曰:“永嘉中,攸为石勒所获,召见,立幕下与语,悦之,坐而饭焉。攸车所止,与胡人邻毂。胡人失火烧车营,勒吏案问胡,胡诬攸。攸度不可与争,乃曰:‘向为老姥作粥,失火延逸,罪应万死。’勒知,遣之。所诬胡厚德攸,遗其驴马,护送令得逸。”王隐\CJKunderwave{晋书}曰:“攸以路远,斫坏车,以牛马负妻子以叛。贼又掠其牛马。攸语妻曰:‘吾弟早亡,唯有遗民。今当步走,担两儿,尽死,不如弃己儿抱遗民。吾后犹当有儿。’妇从之。”\CJKunderwave{中兴书}曰:“攸弃儿于草中,儿啼呼追之,至暮复及。攸明日系儿于树而去。遂渡江。至尚书左仆射,卒。弟子绥,服攸齐衰三年。”}} 既过江,取一妾\footnote{取:通“娶”。},甚宠爱。历年后,讯其所由,妾具说是北人遭乱\footnote{具说:详细陈说。},忆父母姓名,乃攸之甥也。攸素有德业,言行无玷\footnote{德业:道德操守。玷(diàn店):玉石污点。},闻之哀恨终身,遂不复畜妾\footnote{蓄:养。}。

{\cangkai\zihao{5}【评】有关邓攸的“德行”,古今议论纷纷。刘注引\CJKunderwave{中兴书},谓“攸明日系儿于树而去”一节,更是激起公愤。刘应登云:“按邓攸弃儿全侄,局于势之不可两全耳。儿追及之,系之而去,毋乃无人心、天理乎?不复有子,于此见天道之不诬也。”俞德邻谓“追而不及,尚当怜之,追及而缚于道旁,其绝灭天理甚矣”。故王世懋一针见血地指出:“本欲颂邓公高谊,乃令成一大忍人,\CJKunderwave{中兴书}于是为不情矣。”邓攸之儿,弃后能追而及之,则已是具奔跑能力之童,弃之尚情有可宥,缚于树上,无异直接残杀生命,为人父母,于心何忍!为声名而炒作乎?又过于鲜血淋漓之残酷。晋人好名,至此极矣。\CJKunderwave{晋书·良吏}本传也记载此事,可见传闻甚广。不过,刘辰翁另有一解,云:“谓系儿树上者,喜谈全侄,而甚之也。使其追及,任所能行,何事干系?言系者谬,罪系又谬。”可备参考。}

\lettrine{1.29} 王长豫\myidx{王悦}为人谨顺\footnote{谨顺:恭谨和顺。},事亲尽色养之孝\footnote{色养:承顺父母颜色以尽孝养之道。}。{\fzxk\zihao{6}\textcolor{red}{\CJKunderwave{中兴书}曰:“王悦字长豫,丞相导长子也。仕至中书侍郎。”}} 丞相\myidx{王导}见长豫辄喜,见敬豫\myidx{王恬}辄嗔。{\fzxk\zihao{6}\textcolor{red}{\CJKunderwave{文字志}曰:“王恬字敬豫,导次子也。少卓荦不羁,疾学尚武,不为导所重。至中军将军。多才艺,善隶书,与济阳江彪(虨)以善弈闻。”}} 长豫与丞相语,恒以慎密为端\footnote{慎密:谨慎严密。端:原则,根本。}。丞相还台\footnote{台:指朝廷中央衙门。尚书省称中台。时导为尚书省长官,故云。},及行,未尝不送至车后。恒与曹夫人并当箱箧\footnote{并当:收拾,料理。箱箧:泛指箱子。}。长豫亡后,丞相还台,登车后,哭至台门;曹夫人作簏\footnote{簏(lù路):竹箱,泛指箱笼。},封而不忍开。{\fzxk\zihao{6}\textcolor{red}{\CJKunderwave{王氏谱}曰:“导娶彭城曹韶女,名淑。”}}

{\cangkai\zihao{5}【评】王悦(长豫)能登上\CJKunderwave{德行}门的光荣榜,不仅沾了祖宗太保公(祥)的光,更是两晋“以孝治国”的门面,因为他成了当时“色养”之孝的典型。随顺颜色,听大人言,作乖孩子,给父母以心里安慰,当然也是一种孝道,但并不是传统孝道的主要内容。王祥给琅邪王氏后代的临终遗嘱中有“扬名显亲,孝之至也”之言。而扬名显亲的最佳途径,就是中举做官光宗耀祖,仅是色养之孝,只满足家门之内父母的心理,是无法在外获得美爵显宦的,这是一方面。另一方面,作为父母如王导,尽管是东晋开国名相,但人非圣贤,谁能无过?如果父母有错而子女一味随顺,这不是加重矛盾与扩大过错吗?因而“色养”之道本质上考虑也不一定是真孝。儒经\CJKunderwave{周易}中有\CJKunderwave{蛊}卦,“干父之蛊,小有悔,无大咎”(九三爻辞),主张“干父之蛊,用誉”(六五爻辞),强调纠正父辈错失的独立思考。实事求是地纠正父母之过而加以改革,以便真正光大祖先事业而扬名显亲。这是传统孝道的更为重要的内容,可惜被王导忽视了。王悦一生缺乏独立思考,没有什么成就,也就不奇怪的了。不过故事写王导父子情深,倒是真挚感人而形象如画,是很值得一读的。}

\lettrine{1.30} 桓常侍\myidx{桓彝}闻人道深公\myidx{僧法深}者\footnote{桓常侍:桓彝,曾官散骑常侍,故云。道:评论,品题。深公(286—374):据\CJKunderwave{高僧传},谓出于琅邪王氏家族。但是否为王敦之弟,则无考而难信。},辄曰:“此公既有宿名\footnote{宿名:素旧声名。},加先达知称\footnote{先达:前辈名贤。知称:赏识赞许。},又与先人至交\footnote{先人:死去的父祖。至交:深厚交情。},不宜说之。”{\fzxk\zihao{6}\textcolor{red}{\CJKunderwave{桓彝别传}曰:“彝字茂伦,谯国龙亢人,汉五更桓荣十(九)世孙也。父颢,有高名。彝少孤,识鉴明朗。避乱渡江,累迁散骑。”僧法深,不知其俗姓,盖衣冠之胤也。道徽高扇,誉播山东,为中州刘公弟子。值永嘉乱,投迹扬士(土),居止京邑。内持法纲,外允具瞻,弘道之法师也。以业滋清净,而不耐风尘,考室剡县东二百里★山中。同游十馀人,高栖浩然。支道林宗其风范,与高丽道人书,称其德行。年七十有九,终于山中也。}}

{\cangkai\zihao{5}【评】观桓彝之为人,守宣城时,适值苏峻之乱,势孤力屈,而“辞气壮烈,志节不挠”,义在致死而已,彪炳忠义之举,实其内在道德之外现,皆出于平素之自然。平时不在背后随便议论先贤,也是一种道德积淀。人生于复杂的社会,好在背后议论人之是非,如长舌妇一般,蜚语流言,坏人功德,是一种极不道德的行为。桓彝则反其道而行之,事虽小而意义不小,以此入于“德行”榜,正见作者深意。}

\lettrine{1.31} 庾公\myidx{庾亮}乘马有的卢\footnote{庾公:庾亮(289—340)的敬称。他历仕东晋元、明、成三朝,作为外戚,曾执国政,显赫于朝。的卢:传说中的凶马之名,骑之不利主人。},{\fzxk\zihao{6}\textcolor{red}{\CJKunderwave{晋阳秋}曰:“庾亮字元规,颍川鄢陵人,明穆皇后长兄也。渊雅有德量,时人方之夏侯太初、陈长文之伦。侍从父琛,避地会稽,端拱嶷然,郡人严★(惮)之,觐接之者,数人而已。累迁征西大将军、荆州刺史。”伯乐\CJKunderwave{相马经}曰:“马白额入口至齿者,名曰榆雁,一名的卢,奴乘客死,主乘弃市,凶马也。”}} 或语令卖去。{\fzxk\zihao{6}\textcolor{red}{\CJKunderwave{语林}曰:“殷浩劝公卖马。”}} 庾云:“卖之必有买者,即复害其主,宁可不安己而移于它人哉?昔孙叔敖杀两头蛇以为后人\footnote{孙叔敖:★氏,名敖,字孙叔。春秋时楚令尹,助庄王成其霸业。},古之美谈,{\fzxk\zihao{6}\textcolor{red}{贾谊\CJKunderwave{新书}曰:“孙叔敖为儿时,出道上,见两头蛇,杀而埋之。归见其母,泣,问其故,对曰:‘夫见两头蛇者,必死。今出见之,故尔。’母曰:‘蛇今安在?’对曰:‘恐后人见,杀而埋之矣。’母曰:‘夫有阴德,必有阳报,尔无忧也!’后遂兴于楚朝,及长,为楚令尹。”}} 效之,不亦达乎!”

{\cangkai\zihao{5}【评】人性极其复杂。人皆有恻隐之心,见溺而思援手,这是人性的一面。但为蝇头小利而落井下石,嫁祸于人者,又比比皆是,这同样是人干的。人性之善或恶,与古代君子小人的义利之辨同在。历史上的卢马是否真能害主,不可得知。但庾亮之言:“宁可不安己而移于他人哉?”态度坚定,掷地有声。庾亮能成为东晋开国时期的风流人物,当与其高尚道德声誉有关。史称,亮葬时,何充叹曰:“埋玉树于土中,使人情何能已!”其仁德得人心,于此可见一斑。无独有偶,历史上前有刘备,汲汲于仁义而“以人为本”,故结物情而得人心,终济大业。据\CJKunderwave{三国志·蜀书·先主传}裴注引\CJKunderwave{世语},谓备寄荆州刘表时,蒯越、蔡瑁将以计杀之,备知潜逃,“所乘马名的卢,骑的卢走,坠襄阳城西檀溪水中,溺不得出。备急曰:‘的卢:今日厄矣,可努力!’的卢乃一踊三丈,遂得过”,后追者至,已不及矣。刘备借的卢马救命。马之或吉或凶,祸福相生,关键非马,而在驭者人心之所在。}

\lettrine{1.32} 阮光禄\myidx{阮裕}在剡\footnote{阮光禄:即阮裕,曾以金紫大夫征,故称。\CJKunderwave{世说}作者刘义庆为避宋武帝刘裕名讳,从不称阮裕之名。剡(shàn善):古县名,在今浙江嵊州。},曾有好车,借者无不皆给。有人葬母,意欲借而不敢言。阮后闻之,叹曰:“吾有车,而使人不敢借,何以车为\footnote{何以车为:车有何用?何……为,反诘句式。}?”遂焚之。{\fzxk\zihao{6}\textcolor{red}{\CJKunderwave{阮光禄别传}曰:“裕字思旷,陈留尉氏人。祖略,齐国内史。父顗,汝南太守。裕淹通有理识,累迁侍中。以疾,筑室会稽剡山,征金紫光禄大夫不就。年六十一卒。”}}

{\cangkai\zihao{5}【评】“德行”之“德”,得也,在得人心而与人融洽相处。正确处理人与物的关系,也是内在之德的自然表现。魏晋贵族,竞先豪奢,一饭至数百万,尚称无可下箸。如此之人,不刻剥百姓以厚自奉养,行吗?故当时聚敛财物成风,甚至有所谓“钱癖”、“马癖”之称。以此,财物积聚愈多,伤人之心愈疠。失民心则失天下。西晋之亡,即是教训。有鉴于此,阮裕等正直士人汲取历史教训,反其道而行之,不为物累而有肥遁之志。财物为人服务,与人共,弊之无憾。思借车者,以为用于送葬凶事,不吉利,所以不敢向裕开口。但裕则以为车不为人所用,是人对己的不信任,是失民心的表示。故以“何以车为”自责,毁车以自表心旌。车辆一旦成为隔离群众的难以逾越的障碍,就会失掉民心支持,毁车之举,实在是拆除障碍的远见卓识。李贽称阮裕“好名多事”,实际不然。阮裕辞征辟而就二郡太守,他性本好静,但坦言为生计出仕,言之自然真挚,岂有虚矫之态?故史称其以“德业知名”,并非浪言。}

\lettrine{1.33} 谢弈(奕)\myidx{谢奕}作剡令\footnote{谢奕(?—358):字无奕,谢安长兄,陈郡阳夏谢氏家族在东晋初期的代表人物之一。},{\fzxk\zihao{6}\textcolor{red}{\CJKunderwave{中兴书}曰:“谢弈字无弈,陈郡阳夏人。祖衡,太子少傅。父裒,吏部尚书。弈少有器鉴,辟太尉掾,剡令,累迁豫州刺史。”}} 有一老翁犯法,谢以醇酒罚之\footnote{醇酒:烈性酒。},乃至过醉而犹未已。太傅\myidx{谢安}时年七八岁,箸(著)青布绔\footnote{太傅:指谢安,卒赠太傅,故云。箸:同“著”,穿。青布绔(kù裤):黑布裤子。},在兄厀边坐\footnote{厀(xī西):通“膝”。膝盖。},谏曰:“阿兄,老翁可念\footnote{可念:可怜。},何可作此!”弈于是改容,曰:“阿奴\footnote{阿奴:长者对幼小者的爱称,有如今吴方言中的“阿囡”。这里是兄对弟的昵称。},欲放去邪?”遂遣之。

{\cangkai\zihao{5}【评】儿童的成长,其道德品行,从小看八十。谢安童蒙总角之时,即自然见其恻隐之心,这在门第高贵的陈郡谢氏家族中,应该说是无意之中开了一个好头。后来长兄奕早逝,谢安就成了东晋王、谢家族中的主要代表人物,并且非常注意孩子的童蒙教育,其子侄谢道韫及谢玄,能够成为一代风流人物,即与谢安所给予的成功的童蒙教育有关。}

\lettrine{1.34} 谢太傅\myidx{谢安}绝重褚公\myidx{褚裒}\footnote{褚公:对褚裒的敬称。褚裒(póu抔)(303—349),晋康帝皇后之父,朝廷议以“不臣之礼”,力辞执政,而赴外镇。官征北大将军。曾率军三万北伐,败后上疏自贬,忧慨发愤而卒。见\CJKunderwave{晋书·外戚传}。},常称“褚季野虽不言,而四时之气亦备\footnote{四时之气:谓春、夏、秋、冬四季冷热变换。}”。{\fzxk\zihao{6}\textcolor{red}{\CJKunderwave{文字志}曰:“谢安字安石,弈(奕)弟也。世有学行。安弘粹通远,温雅融畅。桓彝见其四岁时,称之曰:‘此儿风神秀彻,当继踪王东海。’善行书。累迁太保,录尚书事,赠太傅。”\CJKunderwave{晋阳秋}曰:“褚裒字季野,河南阳翟人。祖䂮,安东将军。父洽,武昌太守。裒少有简贵之风,冲默之称。累迁江、兖二州刺史,赠侍中、太傅。”}}

{\cangkai\zihao{5}【评】褚裒是当时颇有责任心的一代风流人物,外渊默不言,内慷慨有器识。谢安所称之言,实是形象生动的人格比喻。意谓褚裒外虽不言,内里却心知肚明而自有是非褒贬,是个德行高尚而有原则的人。四季之气运行于外,寒温冷热明白于内,大事绝不含糊。因四季之中有春有秋,又寓\CJKunderwave{春秋}褒贬之义,故史称其有“皮里阳秋”,外面虽不随便臧否人物,而心里自有是非褒贬,有独立主见而不随声附俗。}

\lettrine{1.35} 刘尹\myidx{刘惔}在郡\footnote{刘尹:刘惔,字真长,曾任丹阳尹,故称。谢安妻兄,尚明帝女庐陵公主。会稽王司马昱为相,与王濛并为其座上清谈之客。性简贵自重,与王羲之友善。卒年三十六。},临终绵惙\footnote{绵惙(chuò辍):弥留时气息绵绵欲绝,指病重或病危。一说人临死前,置绵鼻端察看气息之有否。},闻閤(阁)下祠神鼓舞\footnote{祠神鼓舞:祭神时巫师击鼓跳舞。},正色曰\footnote{正色:脸色庄重。}:“莫得淫祀\footnote{淫祀:滥设非礼之祭。}。”{\fzxk\zihao{6}\textcolor{red}{\CJKunderwave{刘尹别传}曰:“惔字真长,沛国萧人也,汉氏之后。真长有雅裁,虽筚门陋巷,晏如也。历司徒左长史、侍中、丹阳尹。为政务镇静信诚,风尘不能移也。”}} 外请杀车中牛祭神\footnote{外:指在外祭祀之吏属。杀车中牛祭神:晋人驾车用牛,乘骑以马。杀驾车牛以祭神,是晋人常事。},真长答曰:“丘之祷久矣,勿复为烦\footnote{“丘之祷久矣”二句:语出\CJKunderwave{论语·述而}篇,谓孔子病,子路请祷,子曰:“丘之祷久矣!”不许祷神。}。”{\fzxk\zihao{6}\textcolor{red}{\CJKunderwave{包氏论语}曰:“祷,请也。”孔安国曰:“孔子素行合于神明,故曰丘之祷久矣。”}}

{\cangkai\zihao{5}【评】刘惔性好老庄,放任自然而有知人之明,曾多次建议朝廷适当抑制桓温野心,颇富政治预见性,惜上不纳。其居官行道家自然无为之政,故孙绰诔辞有“居官无官官之事,处事无事事之心”之的评。他认为自己一生,光明正大而清清白白,死生乃人之自然,祀神祈祷何益于寿?表现了内在纯正坦然之心,内无怍愧,外则无惭神明,所以借\CJKunderwave{论语}中孔子的话以自况,说明自己一生行为自然合乎神明,不必另行祷神,在生死之际,犹无改于平日风流优雅之风范,表现了一种潇洒脱俗的深沉人生态度。}

\lettrine{1.36} 谢公\myidx{谢安}夫人教儿\footnote{谢夫人:谢安夫人为东晋清谈领袖人物之一刘惔之妹。父耽,沛国人。},问太傅\myidx{谢安}\footnote{太傅:指谢安。}:“那得初不见君教儿\footnote{初不:从不。}?”答曰:“我常自教儿\footnote{常自:经常,常常。}。”{\fzxk\zihao{6}\textcolor{red}{\CJKunderwave{谢氏谱}曰:“安娶沛国刘耽女。”案:太尉刘子真,清洁有志操,行己以礼。而二子不才,并渎货致罪,子真坐免官。客曰:“子奚不训道之?”子真曰:“吾之行事,是其耳目所闻见,而不放效,岂严训所变邪?”安石之旨,同子真之意也。}}

{\cangkai\zihao{5}【评】谢安答言,虽率尔而对,但态度认真。安雅善清谈,故言微旨远,令人咀嚼回味。对于童蒙教育,在魏晋时代产生了两种不同的教育方式:一是继承汉儒的传统经学教育方式,重在师承的知识积累,一篇\CJKunderwave{尚书}题目,讲解动辄十几万言,在传授知识时多采用填鸭式满堂灌的方法;一是受清谈玄家的影响,力图摒弃传统教育的经学模式,在“为什么”的反复诘难中,采用了启悟思维的新式教育方法。安之于教育,即取玄家新立场。他本人也是清谈名家,只因政绩卓著而掩其玄家之名。从这则故事看,似乎谢安轻忽了孩子的教育,所以被妻子埋怨。实则不然。主要是刘夫人以传统思维模式来衡量,一时还不理解丈夫的不言之教——一种特殊的家庭教育子女的方式。所以谢安说:我经常以自己的言行作榜样来启悟孩子。他回话理直气壮,益见其真,强调的是不言之身教。提倡以身作则的身教启悟,在孩子的独立反思中,为童蒙教育另辟一片新洞天。其子侄如谢玄,能成为德政双优的一代风流人物,当与谢公的言传身教有关。}

\lettrine{1.37} 晋简文\myidx{司马昱}为抚军时\footnote{晋简文:指晋简文帝司马昱(320—372),穆帝年幼即位,昱任抚军大将军总理政务。后来大将军桓温专擅朝政,先废海西公,后立司马昱为帝,第二年崩。},{\fzxk\zihao{6}\textcolor{red}{\CJKunderwave{续晋阳秋}曰:“帝讳昱,字道万,中宗少子也。仁明有智度。穆帝幼冲,以抚军辅政。大司马桓温废海西公而立帝,在位三(二)年而崩。”}} 所坐床上\footnote{床:坐榻。},尘不听拂\footnote{听:听任、准许。},见鼠行迹,视以为佳。有参军见鼠白日行\footnote{参军:官名,王国或军镇的重要属官。},以手板批杀之\footnote{手板:即“手版”,又称“笏”。古时官吏上朝或见上司时的狭长方形小板,质地有竹、木、象牙之不同,用以记事或备忘。},抚军意色不悦。门下起弹\footnote{门下:下属。弹:弹劾。},教曰:“鼠被害,尚不能忘怀,今复以鼠损人,无乃不可乎\footnote{教:古时王侯或长官发布的指示或命令。无乃:恐怕,表示委婉语气。}?”

{\cangkai\zihao{5}【评】人称简文清虚寡欲,尤善玄言,留心于典籍,而不以居处为意,凝尘满席,而坐处湛如,其自然之德如此。但坐床听任鼠迹白日横行,则又见其矫饰作态,于儒不合仁义,于道则损其自然,如刘辰翁之所评:“此复何足与于德行,正应弹鼠,不应弹人。”其道德岂足以号召天下而力挽狂澜哉!其见欺于桓温,实不足怪。故史称其“无济世大略”,而被谢安评为“(晋)惠帝之流”——即亡国之君也,其德其能,如此而已!其事入于\CJKunderwave{世说}之\CJKunderwave{德行}门,或是作者好奇之笔。}

\lettrine{1.38} 范宣\myidx{范宣}年八岁\footnote{范宣:\CJKunderwave{晋书·儒林传}谓字宣子,与刘注不同。东晋著名儒学教育家,当时与范宁并称“二范”。著\CJKunderwave{易论难}、\CJKunderwave{礼论难}行世。},后园挑菜,误伤指,大啼。人问:“痛邪?”答曰:“非为痛。身体发肤,不敢毁伤\footnote{“身体发肤”二句:语见\CJKunderwave{孝经}:“身体发肤,受之父母,不敢毁伤,孝之始也。”无故毁伤身体,则为不孝,此所以啼哭。},是以啼耳。”{\fzxk\zihao{6}\textcolor{red}{\CJKunderwave{宣别传}曰:“宣字子宣,陈留人,汉莱芜长范丹后也。年十岁,能诵\CJKunderwave{诗}、\CJKunderwave{书}。儿童时,手伤改容,家人以其年幼,皆异之。征太学博士、散骑常侍,一无所就。年五十四卒。”}} 宣洁行廉约。韩豫章\myidx{韩伯}遗绢百匹\footnote{韩豫章:指韩伯,时任豫章太守,故称。曾为王弼\CJKunderwave{周易注}补注\CJKunderwave{易传}之系辞、说卦、杂卦等,是当时著名玄学名家。疋:通“匹”,古时一匹四丈。},不受。{\fzxk\zihao{6}\textcolor{red}{\CJKunderwave{中兴书}曰:“宣家至贫,罕交人事。豫章太守殷羡见宣茅茨不完,欲为改室,宣固辞。羡爱之,以宣贫,加年饥疾疫,厚饷给之,宣又不受。”\CJKunderwave{续晋阳秋}曰:“韩伯字康伯,颍川人。好学善言理。历豫章太守、领军将军。”}} 减五十疋,复不受。如是减半,遂至一疋,既终不受。韩后与范同载\footnote{同载:共乘一车。},就车中裂二丈与范,云:“人宁可使妇无裈邪\footnote{妇:此专指妻子。裈(kūn昆):内裤。}!”范笑而受之。

{\cangkai\zihao{5}【评】范宣总角童年,即聪慧能言,因刀伤之疼痛而啼哭,对孩童而言,出于自然的生理反应。但当人问之,则有“身体发肤,不敢毁伤”之语,说明他早已熟读\CJKunderwave{孝经},较一般儿童,表现了一种面对突发事件的早慧和思考。晋人提倡“以孝治天下”,故作者因其与“孝”搭上关系,勉强挤入\CJKunderwave{德行}门。实际上,以之入\CJKunderwave{夙惠}门或\CJKunderwave{言语}门更为合适。至于宣之廉洁故事,则直入\CJKunderwave{德行}之堂而无愧。韩伯在故事中虽为配角,但其“人宁可使妇无裈邪”之词,言语可人,性情极真,笑谑中显出语言艺术的魅力,同时也成功地刻画了韩伯的诙谐性格,对于衬托范宣之廉洁,起了很好的作用。还有,范宣是个教书先生,在古代,教书职业清苦,没有多少人去走他的后门,这对范宣保持廉洁本性大有帮助。如果换在今天,人们急于子女成龙成凤,于是向主管教育的官员或先生搞公关、走后门,已成风气。这时要像范宣那样廉洁自律,恐怕就需要很高的定力了。}

\lettrine{1.39} 王子敬\myidx{王献之}病笃\footnote{王子敬:即王献之(344—388),出于琅邪王氏家族。曾任谢安长史,官至中书令,故称王令或王大令。据\CJKunderwave{晋书·后妃传},尚简文帝女新安公主。少有令名,“风流一时之冠”。其书法已造神境,与父羲之并称“二王”。病笃:病重。},道家上章应首过\footnote{道家:此指道教之徒,而非诸子之道家。上章:道士替病家上章奏给天帝,祈求谅解和祛病延年。首过:自首其过以忏悔。},问子敬:“由来有何异同得失\footnote{由来:历来。异同得失:偏义复词,指过失或罪错。}?”子敬云:“不觉有馀事,唯忆与郗家离婚\footnote{郗家:高平郗氏家族,特指前妻郗道茂。}。”{\fzxk\zihao{6}\textcolor{red}{\CJKunderwave{王氏谱}曰:“献之娶高平郗昙女,名道茂,后离婚。”\CJKunderwave{献之别传}曰:“祖父旷,淮南太守。父羲之,右将军。咸宁(安)中诏尚馀姚公主。迁中书令。卒。”}}

{\cangkai\zihao{5}【评】王羲之、献之父子,世事天师道教,因此,子敬病中之时,才会由道士上章首过,这大概和西方天主教徒向上帝忏悔一般,为的是求得内在的心理平衡。事虽迷信,言却真诚,故辞甚哀楚,所谓人之将亡,其言也善,展现了献之心灵世界的真实。为什么献之对离婚如此悔恨?是否因其花心再娶公主,成为陈世美般的负心人呢?非也。原来王献之与前妻郗道茂,青梅竹马,两小无猜。出妻而再娶公主实是上命难违,史称“子敬灸足以违诏”(见\CJKunderwave{宋书·后妃传}引江斅语),以自残的方式来逃婚,但没有成功。这说明献之和郗道茂,感情甚笃,其离异实出于强大的政治压力。此献之所以悔恨终身也。其悔过出于内心真情的自然流露,而无丝毫的虚假和矫饰之态。人生悲剧很多,高门士人同样无法逃脱厄运,这是时代和制度使然。}

\lettrine{1.40} 殷仲堪\myidx{殷仲堪}既为荆州\footnote{殷仲堪(?—399):善清谈,当时与韩康伯齐名。为荆州:任荆州刺史。按:当时荆州为掌控长江中上游的军事重镇。},值水俭\footnote{水俭:水涝成灾,田谷歉收。或谓“水俭”为“岁俭”之讹,岁俭,年岁歉收。见黄汝琳\CJKunderwave{世说新语补}校刊。于义亦通,可另备一说。},食常五椀盘\footnote{五椀盘:魏晋六朝流行于南方的小型成套餐具,亦称“五盏盘”,有一托盘和盘中五只小碗组成,盛菜容量有限。},外无馀肴\footnote{肴:菜肴。},饭粒脱落盘席间,辄拾以噉之\footnote{噉(dàn但):同“啖”,吃。}。虽欲率物\footnote{率物:为人表率。},亦缘其性真素\footnote{缘:因。真素:真朴自然。}。每语子弟云:“勿以我受任方州\footnote{方州:大州。方,大。},云我豁平昔时意\footnote{豁(huò或):豁散,豁免,引申为舍去。}。今吾处之不易,贫者士之常\footnote{贫者士之常:见\CJKunderwave{说苑·杂言篇}荣启期答孔子之语。},焉得登枝而捐其本\footnote{登枝捐本:因官职高升而忘本。}?尔曹其存之\footnote{尔曹:你们,指仲堪子弟辈。}。”{\fzxk\zihao{6}\textcolor{red}{\CJKunderwave{晋安帝纪}曰:“仲堪,陈郡人,太常融孙也。车骑将军谢玄请为长史。孝武说之,俄为黄门侍郎。自杀袁悦之后,上深为晏驾后计,故先出王恭为北蕃。荆州刺史王忱死,乃中诏用仲堪代焉。”}}

{\cangkai\zihao{5}【评】勤俭是传统美德,古人早有“成由勤俭破由奢”(李商隐诗)的警句。人性勤俭,则自然清虚寡欲而不为物累;不为物累,则顺天理而获民心,于事无不济矣!这是出自内心本性而以行道济世为目的,故为真勤俭,是一种美德。反之,则为虚假的勤俭,是一种失德的表现。史称殷仲堪迷信事神“不吝财贿”,但却“怠行仁义,啬于周急”,则其示俭以“率物”,一方面是其鄙吝天性的自然流露,另一方面是做表面文章,以炒作的自我宣传来博取名声而已。其爱惜粒饭,外无馀肴,而不周急于民,何德之有?以之入\CJKunderwave{俭啬}门似乎更合适。}

\lettrine{1.41} 初,桓南郡\myidx{桓玄}、杨广\myidx{杨广}共说殷荆州\myidx{殷仲堪}\footnote{桓南郡:指桓玄(369—404),袭父温之爵南郡公,故称。安帝时任江州刺史、都督荆州八郡诸军事,率军东下,篡晋自立,建国号楚。旋被刘裕击败,斩首京师。杨广(?—399):曾官淮南太守,南蛮校尉,后与弟佺期俱被桓玄攻杀。殷荆州:指殷仲堪。},宜夺殷觊\myidx{殷觊}南蛮以自树\footnote{殷觊(jì计):\CJKunderwave{晋书}“觊”作“顗”。南蛮:官名,指南蛮校尉之军职,地位仅次于将军。}。{\fzxk\zihao{6}\textcolor{red}{\CJKunderwave{桓玄别传}曰:“玄字敬道,谯国龙亢人,大司马温少子也。幼童中,温甚爱之,临终,命以为嗣。年七岁,袭封南郡公。拜太子洗马、义兴太守。不得志,少时去职,归其国。与荆州刺史殷仲堪素旧,情好甚隆。”周祇\CJKunderwave{隆安记}曰:“广字德度,弘农人,杨震后也。”\CJKunderwave{晋安帝纪}曰:“觊字伯道,陈郡人。由中书郎出为南蛮校尉。觊亦以率易才悟者(著)称,与从弟仲堪俱知名。”\CJKunderwave{中兴书}曰:“初,仲堪欲起兵,密邀觊,觊不同。杨广与弟佺期劝杀觊,仲堪不许。”}} 觊亦即晓其旨。尝因行散\footnote{行散:魏晋士大夫喜服五石散又称“寒食散”以养生,五石散药性猛烈,服后须散步调适,发泄药性,称行散或行药。},率尔去下舍\footnote{率尔:随意,随便。下舍:馆舍住所。},便不复还,内外无预知者。意色萧然\footnote{萧然:超然洒脱。},远同鬬生\myidx{穀于菟}之无愠\footnote{鬬生:指春秋时楚国穀于菟,即令尹子文。子文三为令尹无喜色,三罢令尹无愠色。愠:怨怒之色。}。时论以此多之\footnote{多:褒美,赞扬。}。{\fzxk\zihao{6}\textcolor{red}{\CJKunderwave{春秋传}曰:“楚令尹子文,鬬氏也。”\CJKunderwave{论语}曰:“令尹子文三仕为令尹,无喜色;三已之,无愠色。”}}

{\cangkai\zihao{5}【评】殷觊与仲堪为兄弟,一地为官,但二人心思却大不相同。仲堪封疆大吏,地位远高于觊,但为实现政治野心,不顾兄弟情义,屡逼殷觊就范,并设计夺其南蛮校尉以壮大自己。见利忘义,此乃小人行径。觊早已识破其心,视南蛮如弊履,去之而不复顾。史称仲堪将兴兵内伐,觊谏之曰:“夫人臣之义,慎保所守,朝廷是非,宰辅之务,岂藩屏之所图也!”仲堪不从而恨之,在形势甚明的情况下,觊又训斥仲堪:“我病不过身死,但汝病在灭门!”(\CJKunderwave{晋书}卷八三本传)世人赞美殷觊,见其为国为家之远见卓识,岂论区区南蛮官职之有无哉!}

\lettrine{1.42} 王仆射\myidx{王愉}在江州\footnote{王仆射:王愉(?—404)曾官尚书左仆射,故称。在江州:在江州刺史任上。},为殷\myidx{殷仲堪}、桓\myidx{桓玄}所逐\footnote{殷:指殷仲堪。桓:指桓玄。},奔窜豫章\footnote{豫章:郡名,治所在今江西南昌。},存亡未测。{\fzxk\zihao{6}\textcolor{red}{徐广\CJKunderwave{晋纪}曰:“王愉字茂和,太原晋阳人,安北将军坦之次子也。以辅国司马出为江州刺史。愉始至镇,而桓玄、杨佺期举兵以应王恭,乘流奄至。愉无防,惶遽奔临川,为玄所得。玄篡位,迁尚书左仆射。”}} 王绥\myidx{王绥}在都\footnote{王绥(?—404):愉子。愉、绥父子因不满刘裕见杀。},既忧戚在貌,居处饮食,每事有降。时人谓为“试守孝子\footnote{试守孝子:谓未知父之生死而先有丧容,故曰“试守”。试守,秦汉时正式用官吏之前的试用称“试守”,犹今见习。}”。{\fzxk\zihao{6}\textcolor{red}{\CJKunderwave{中兴书}曰:“绥字彦猷,愉子也。少有令誉。自王泽至坦之,六世盛德。绥又知名于时,冠冕莫与为比。位至中书令、荆州刺史。桓玄败后,与父愉谋反,伏诛。”}}

{\cangkai\zihao{5}【评】王绥“试守孝子”之行,与晋廷“以孝治天下”思想相凑泊。但其“试守”之忧,是真,是假?是忧父之存亡,抑或忧己之名位?值得思考。史称其“少有美名,厚自矜迈,实鄙而无行”,桓玄之篡,急攀为中书令。有“孝”而无“忠”,正见其道德特色。故其身死之后,“名论殆尽”,何德之有?一个贵族恶少,因其善作姿态而误入\CJKunderwave{德行}门榜,是对历史的嘲弄,悲哉!}

\lettrine{1.43} 桓南郡\myidx{桓玄}{\fzxk\zihao{6}\textcolor{red}{玄也。}} 既破殷荆州\myidx{殷仲堪}\footnote{桓南郡:桓玄。殷荆州:殷仲堪。破:击破,打败。},收殷将佐十许人,咨议罗企生\myidx{罗企生}亦在焉\footnote{咨议:全称是咨议参军。当时公府、节镇皆设此官,以参谋军事。罗企生时为殷仲堪的咨议参军。}。{\fzxk\zihao{6}\textcolor{red}{\CJKunderwave{玄别传}曰:“玄尅荆州,杀殷道护及仲堪参军罗企生、鲍季札,皆仲堪所亲仗也。”}} 桓素待企生厚,将有所戮\footnote{将有所戮:将要行刑杀人。},先遣人语云:“若谢我\footnote{谢:谢罪以求原谅。},当释罪。”企生答曰:“为殷荆州吏,今荆州奔亡,存亡未判\footnote{存亡未判:生死未明。},我何颜谢桓公!”{\fzxk\zihao{6}\textcolor{red}{\CJKunderwave{中兴书}曰:“企生字宗伯,豫章人。殷仲堪初请为府功曹,桓玄来攻,转咨议参军。仲堪多疑少决,企生深忧之,谓其弟遵生曰:‘殷侯仁而无断,事必无成。成败天也,吾当死生以之。’及仲堪走,文武并无送者,唯企生从焉。路经家门,遵生绐之曰:‘作如此分别,何可不执手?’企生回马授手,遵生便牵下之,谓曰:‘家有老母,将欲何行?’企生挥涕曰:‘今日之事,我必死之。汝等奉养,不失子道。一门之内,有忠与孝,亦复何恨!’遵生抱之愈急。仲堪于路待之,企生遥呼曰:‘今日死生是同,愿少见待。’仲堪见其无脱理,策马而去。俄而玄至,人士悉诣玄,企生独不往,而营理仲堪家。或谓曰:‘玄性猜急,未能取卿诚节,若遂不诣,祸必至矣。’企生正色曰:‘我殷侯吏,见遇以国士,不能共殄丑逆,致此奔败,何面目就桓求生乎?’玄闻,怒而收之,谓曰:‘相遇如此,何以见负?’企生曰:‘使君口血未干,而生此奸计,自伤力劣不能剪定凶逆,我死恨晚尔!’玄遂斩之,时年三十有七。众咸悼之。”}} 既出市\footnote{市:原指洛阳东市杀人之地,这里泛称杀人刑场。},桓又遣人问:“欲何言?”答曰:“昔晋文王\myidx{司马昭}杀嵇康\myidx{嵇康},而嵇绍\myidx{嵇绍}为晋忠臣\footnote{“昔晋文王”二句:晋文王指司马昭。他于魏景元四年(263)下令杀害名士嵇康。后康子绍,经山涛推荐而成为晋臣,官至侍中。八王之乱时,为保护晋惠帝而死于乱军之中,故\CJKunderwave{晋书}以之入\CJKunderwave{忠义传}。}。{\fzxk\zihao{6}\textcolor{red}{王隐\CJKunderwave{晋书}曰:“绍字延祖,谯国铚人。父康,有奇才隽辩。绍十岁而孤,事母孝谨。累迁散骑常侍。惠帝败于荡阴,百官左右皆奔散,唯绍俨然端冕,以身卫帝。兵交御辇,飞箭雨集,遂以见害也。”}} 从公乞一弟以养老母。”桓亦如言宥之\footnote{宥(yòu佑):宽恕、饶恕。}。桓先曾以一羔裘与企生母胡,胡时在豫章,企生问至\footnote{问:音问,消息。},即日焚裘。

{\cangkai\zihao{5}【评】故事发生在安帝隆安三年(399),地点在荆州治所江陵。为主尽忠,在今天看来,表现了一种人身依附的关系,是愚蠢的,但在古代,却是一种当然的道德规范。罗企生身为贫寒之士,官不过地方佐吏,但其人生态度,与王愉、绥父子之反复小人行径,大相径庭。重义轻生,节烈严霜;颈加白刃,而志不可屈。其铮铮之言,掷地有声。其母胡氏,见问至而焚裘,大义凛然,有其母乃有其儿,可见一门忠义之风。其荣获“德行”榜,正可为卖友(甚至是卖国)求荣者诫。}

\lettrine{1.44} 王恭\myidx{王恭}从会稽还\footnote{王恭(?—398):孝武帝后兄,安帝舅父。与殷仲堪、桓玄等,二次兴兵清君侧,兵败被诛。会稽:郡治在今浙江绍兴市。},{\fzxk\zihao{6}\textcolor{red}{周秖(祇)\CJKunderwave{隆安记}曰:“恭字孝伯,太原晋阳人。祖父濛,司徒左长史,风流标望。父蕴,镇军将军,亦得世誉。”\CJKunderwave{恭别传}曰:“恭清廉贵峻,志存格正。起家者(著)作郎,历丹阳尹、中书令,出为五州都督、前将军、青兖二州刺史。”}} 王大\myidx{王忱}看之\footnote{王大:即王忱,因小字佛大,故称。}。{\fzxk\zihao{6}\textcolor{red}{王忱,小字佛大。\CJKunderwave{晋安帝纪}曰:“忱字元达,平北将军坦之弟(第)四子也。甚得名于当世,与族子恭少相善,齐声见称。仕至荆州刺史。”}} 见其坐六尺簟\footnote{簟(diàn垫):竹席。},因语恭:“卿东来\footnote{卿:第二人称代词,用于上称下、尊称卑,或同辈间亲昵而不拘礼数之称呼。},故应有此物,可以一领及我\footnote{及:给予,赠予。}。”恭无言。大去后,即举所坐者送之。既无馀席,便坐荐上\footnote{荐:草垫。}。后大闻之,甚惊,曰:“吾本谓卿多,故求耳。”对曰:“丈人不悉恭\footnote{丈人:对长辈的敬称。王忱是恭的族叔,故云。},恭作人无长物\footnote{长(zhàng丈)物:多馀的东西。}。”

{\cangkai\zihao{5}【评】故事发生在王恭年轻时随父坦之从会稽到京师的时候,正可见其性情之自然。史称王恭诛后,“家无财帛,唯书籍而已,为识者所伤”。可见他虽居显宦,但一生清廉。在“无官不贪”的黑暗时代,清官总比贪官好,清廉简朴当然是一种美德。“恭作人无长物”,并非矫饰炒作,而是清新可人的本色语,诚如刘辰翁所评:“无紧无要,有襟有度。”生活琐事之中,正见其人生原则。但遗憾的是,王恭为人,志大才疏,不顾条件,屡兴晋阳之甲,事败被人反噬。悲夫!}

\lettrine{1.45} 吴郡陈遗\myidx{陈遗}\footnote{吴郡:郡名,治所在今江苏苏州。陈遗:\CJKunderwave{南史}入于\CJKunderwave{孝义传}。},{\fzxk\zihao{6}\textcolor{red}{未详。}} 家至孝\footnote{家:在家,居家。},母好食铛底焦饭\footnote{铛(chēng撑):锅。焦饭:今之锅巴。}。遗作郡主簿\footnote{主簿:官名,为朝廷台省或地方郡县属吏,主管簿籍文书。},恒装一囊\footnote{囊:口袋。},每煮食,辄贮录焦饭\footnote{贮录:收集贮藏。},归以遗母。后值孙恩\myidx{孙恩}贼出吴郡\footnote{孙恩(?—402):孙恩以东南沿海岛屿为根据地,于隆安二年(398)起事,自号征东将军,攻略东晋沿海诸郡,朝廷全力御之。恩于元兴元年(402)败亡。出:到。},{\fzxk\zihao{6}\textcolor{red}{\CJKunderwave{晋安帝纪}曰:“孙恩一名灵秀,琅邪人。叔父泰,事五斗米道,以谋反诛。恩逸逃于海上,聚众十万人,攻没郡县。后为临淮(海)太守辛昺斩首送之。”}} 袁府君\myidx{袁山松}{\fzxk\zihao{6}\textcolor{red}{山松,别见。}} 即日便征\footnote{袁府君:袁山松。袁山松:与谢混同为陈郡阳夏(今河南太康)人。曾著\CJKunderwave{后汉书}百篇,善音乐,改编\CJKunderwave{行路难}曲,酣醉纵歌,“听者莫不流涕”,与羊昙、桓伊并称乐坛“三绝”。便征:立刻出征。}。遗以聚敛得数斗焦饭,未展归家\footnote{未展:未及,来不及。},遂带以从军。战于沪渎\footnote{沪渎:水名,在今年上海市东北,即今吴淞江下游一段。},败,军人溃散,逃走山泽,皆多饿死。遗独以焦饭得活。时人以为纯孝之报也。

{\cangkai\zihao{5}【评】陈遗孝母,故事宛然有致,纯然出于一片真情。\CJKunderwave{世说}所载及于刘宋时人者,少之又少,唯有谢灵运、王谧、傅亮等少数高门士族。陈遗故事发生在晋末,而其人则卒于宋,故可与谢灵运等并称宋人。以一介江南寒士而跻身“德行”高榜,在门阀社会中,人谓三生有幸,其幸在纯孝之名而不以微贱弃之。如余嘉锡\CJKunderwave{笺疏}评云:“自中原云扰,五马南浮,虽王纲解纽,风教陵夷,而孝弟之行,独为朝野所重。……故虽江左偏安,五朝递嬗,犹能支柱二百七十馀年,不为胡羯所吞噬,……而孝乃为人之本,……岂可不加之意也哉!”世界再乱,也要有某种道德或思想作支撑,才不至于漫无方向。于此可见思想道德的重要社会作用。}

\lettrine{1.46} 孔仆射\myidx{孔安国}为孝武\myidx{司马曜}侍中\footnote{孔仆射:东晋孔安国(?—428),曾任左仆射,故称。孝武:晋孝武帝司马曜,驾崩后谥孝武,庙号烈宗。侍中:官名,晋时皇帝身边的顾问人员,参与机密大事。},豫蒙眷接\footnote{豫:通“预”。眷接:礼遇,厚待。}。烈宗\myidx{司马曜}山陵\footnote{山陵:原指皇帝陵墓,这里名词动化,指驾崩。},孔时为太常\footnote{太常:官名,即太常卿,朝廷九卿之一,掌礼仪,祭祀,并备顾问。},形素羸瘦\footnote{羸(léi雷):瘦弱。},着重服\footnote{重服:重丧孝服。},竟日涕泗流涟,见者以为真孝子。{\fzxk\zihao{6}\textcolor{red}{\CJKunderwave{续晋阳秋}曰:“孔安国字安国,会稽山阴人。车骑愉第六子也。少而孤贫,能善树节,以儒素见称。历侍中、太常、尚书,迁左仆射、特进,卒。”}}

{\cangkai\zihao{5}【评】此孔安国,非汉代\CJKunderwave{尚书}专家孔安国,但同样是一介儒生。古人云:滴水之恩,当涌泉相报。他因蒙受孝武帝的赏拔,心怀感激,故于孝武驾崩之后,身穿重丧孝服而悲戚不已。作为太常卿,是为朝廷百官表率;但更重要的是,这纯然出于内心真性情的流露,并非装腔作势的热炒。试想,“竟日涕泗流涟”,再优秀的演员也很难做到。情真而悲恸,是其德行之所在。}

\lettrine{1.47} 吴道助\myidx{吴坦之}、附子\myidx{吴隐之}兄弟\footnote{吴道助、附子兄弟:指吴坦之、隐之兄弟。},居在丹阳郡后\footnote{丹阳:郡名,治所在今江苏南京东南。郡后:郡守府廨之后。}。遭母童夫人艰\footnote{遭艰:指遭父母之丧。},{\fzxk\zihao{6}\textcolor{red}{道助,坦之小字。附子,隐之小字也。\CJKunderwave{吴氏谱}曰:“坦之,字处靖,濮阳人。仕至西中郎将功曹。父坚,取东苑(莞)童侩女,名秦姬。”}} 朝夕哭临,及思至\footnote{思至:一谓通于“缌★”,指穿守丧孝服。一谓“思至”为“周忌”形、音之讹。见余嘉锡\CJKunderwave{笺疏}引李慈铭说。},宾客吊省,号踊哀绝\footnote{号踊:边哭边顿脚。},路人为之落泪。韩康伯\myidx{韩康伯}时为丹阳\footnote{为丹阳:任丹阳尹。},母殷在郡,每闻二吴之哭,辄为悽恻,语康伯曰:“汝若为选官\footnote{选官:负责组织人事的选举之官。},当好料理此人\footnote{料理:魏晋口语,指安排,照顾。}。”康伯亦甚相知,韩后果为吏部尚书\footnote{吏部尚书:中央朝廷负责选举的最高长官。}。大吴不免哀制\footnote{哀制:礼制规定中的守丧之期。},小吴遂大贵达。{\fzxk\zihao{6}\textcolor{red}{郑缉\CJKunderwave{孝子传}曰:“隐之字处默。少有孝行,遭母丧,哀毁过礼。时与太常韩康伯邻居。康伯母,杨(扬)州刺史殷浩之妹,聪明妇人也。隐之每哭,康伯母辄辍事流涕,悲不目(自)胜,终其丧如此。谓康伯曰:‘汝后若居铨衡,当用此辈人。’后康伯为吏部尚书,乃进用之。”\CJKunderwave{晋安帝纪}曰:“隐之既有至性,加以廉洁,俸禄颁九族,冬月无被。桓玄欲革岭南之敝,以为广州刺史。去州二十里,有贪水,世传饮之者,其心无厌。隐之乃至水上,酌而饮之,因赋诗曰:‘石门有贪泉,一歃重千金。试使夷齐饮,终当不易心。’为卢循所攻,还京师。历尚书、领军将军。”\CJKunderwave{晋中兴书}曰:“旧云:往广州饮贪泉,失廉洁之性。吴隐之为刺史,自酌贪泉饮之,题石门为诗云云。”}}

{\cangkai\zihao{5}【评】吴坦之、隐之兄弟因其“死孝”入“德行”荣誉之榜。但吴坦之“不免哀制”——因悲伤过度而死于守丧之期,这是不明生、老、病、死乃人之自然的大道理。生命不存,则什么都说不上,有何价值可言?但因其言行合于晋时道德需要,所以应以历史眼光视之;如今时过境迁,后人当然不会仿效。倒是韩母恻隐之心,写得颇为生动。如刘辰翁所评:“本为二吴孝行,而韩母在焉,善观人者也。”喧宾夺主的神来之笔,于写作也是一法而可资借鉴。}



%%% Local Variables:
%%% mode: latex
%%% TeX-engine: xetex
%%% TeX-master: "../Main"
%%% End:
