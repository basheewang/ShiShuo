%% -*- coding: utf-8 -*-
%% Time-stamp: <Chen Wang: 2025-12-06 18:50:55>

% ○ ◎ ‧ 「 」 『 』 々 ( ) “ ” ■ ^[一-龥]
% 【\([^】][^】][^】]+\)】 → {\\fzxk\\zihao{6}\\textcolor{red}{\1}}
% \(【评】.*\) → {\\cangkai\\zihao{5}\1}
% \(【题解】.*\) → {\\cangkai\\zihao{5}\1}
% 《\([^》]+\)》 → \\CJKunderwave{\1}
% ^\([0-9]+.[0-9]+\) → \\lettrine{\1}
% {\\fzxk\\zihao{6}\\textcolor{red}{[^o]*}}

\setlength{\parindent}{0pt}

\chapter{自新第十五}



{\cangkai\zihao{5}【题解】 人非圣贤,谁能无过?关键在于“过则勿惮改”(\CJKunderwave{论语·学而}),传统的教训是很深刻的。知错必纠,改过自新,就是好同志,大家就应该摒弃前嫌,团结一致干事业,而不应该老是揪住人家历史的小辫子不放。人生道路漫长,谁能担保自己一点也不走弯路呢?走弯路不就是犯错误了么!人一旦知道自己迷失了方向,走错了路,回过头来重新走向正确的道路,虽然作为当事人,浪费了人生的宝贵时间,但人生道路是一个由错误到正确的不断探索的实践过程,知错改过,就为他人总结了经验教训,从而为更多的人指引了正确的方向和道路,节省了达到人生目标的大量时间。\CJKunderwave{易·系辞}有“日新之谓盛德”之言,改过自新,也是日新盛德大业之一端,其功厥伟。}

\lettrine{15.1} 周处\myidx{周处}年少时\footnote{周处(?—297):三国吴时为无难督,晋时官至御史中丞。因强直为权贵所谗,西征氐人,战殁。},凶强侠气\footnote{凶强:凶恶强暴。侠气:任侠使气。},为乡里所患\footnote{所患:视为祸害。}。{\fzxk\zihao{6}\textcolor{red}{\CJKunderwave{处别传}曰:“处字子隐,吴郡阳羡人。父鲂,吴鄱阳太守。处少孤,不治细行。”\CJKunderwave{晋阳秋}曰:“处轻果薄行,州郡所弃。”}} 又义兴中\footnote{义兴:郡名,治所在阳羡,即今之江苏宜兴市。},水中有蛟,山中有邅迹{\fzxk\zihao{6}\textcolor{red}{一作白额}} 虎\footnote{蛟:蛟龙,传说中水中凶恶神物。按,实指水中鳄鱼。邅(zhān沾)迹虎:邪足虎,即跛脚虎。但跛脚之虎,威力大减,何横之有?文句欠通。一作“白额虎”,疑是。},并皆暴犯百姓\footnote{暴犯:强暴侵犯。},义兴人谓为“三横”\footnote{横:横强,横暴的事物。},而处尤剧\footnote{剧:甚。}。或说处杀虎斩蛟,实冀“三横”唯馀其一\footnote{说:游说,说服。冀:希冀,希望。}。处即刺杀虎,又入水击蛟,蛟或浮或没行数十里,处与之俱,经三日三夜,乡里皆谓已死\footnote{乡里:指代家乡父老乡亲。},更相庆。竟杀蛟而出。闻里人相庆,始知为人情所患\footnote{人情:人心。},有自改意。{\fzxk\zihao{6}\textcolor{red}{\CJKunderwave{孔氏志怪}曰:“义兴有邪足虎,溪渚长桥有苍蛟,并大啖人,郭西周,时谓‘郡中三害。’”周即处也。}} 乃入吴寻二陆\footnote{吴:吴郡,即今江苏省苏州市。二陆:指陆机、陆云兄弟。}。平原\myidx{陆机}不在\footnote{平原:指陆机,曾任平原内史,故称。},正见清河\myidx{陆云}\footnote{清河:指陆云,曾任清河内史,故称。},具以情告,并云:“欲自修改而年已蹉𧿶\footnote{蹉𧿶:即蹉跎,即虚耗光阴,空度年华。} ,终无所成。”清河曰:“古人贵朝闻夕死\footnote{朝闻夕死:语出\CJKunderwave{论语·里仁篇}:“朝闻道,夕可死矣。”意谓一旦明白真理,即使不久即死也不感遗憾。},况君前途尚可。且人患志之不立,亦何忧令名不彰邪\footnote{令名:美好声名。}?”处遂改励\footnote{改励:改过自新而奋发图强。},终为忠臣孝子。{\fzxk\zihao{6}\textcolor{red}{\CJKunderwave{晋阳秋}曰:“处仕晋为御史中丞,多所弹纠。氐人齐万年反,乃令处距万年。伏波孙秀欲表处母老,处曰:‘忠孝之道,何当得两全?’乃进战,斩首万计,弦绝矢尽,左右劝退,处曰:‘此是吾授命之日。’遂战而没。”}}

{\cangkai\zihao{5}【评】人生在世,当然应该尽量避免或减少过错,不做犯罪之事。但如前述,漫漫人生,孰能无过?过错是一种客观存在,并不可怕,可怕的是不知有错,而麻木不仁。论出身,周处并非“高干”出身,而只是个中等的“干部”子弟。在古代,这一出身,在地方颇有势力,足以横行乡里。处父早死,年少时缺乏应有的教育,因而“凶强侠气,为乡里所患”,成为地方的三害之一。但他并不自知,反而沾沾自喜而以此傲人。其实,年轻周处横行乡里之过,在于没有接受良好教育以启发其内在的良能良知——也即人性中固有恻隐之心。一旦良心本性受外物所累,犹如明镜被尘土所掩蔽,怎能明白事理?其为害乃外在环境物累所致,一旦接受了教育,尘垢尽扫,则良心明净如镜,自当洞鉴一切而改过自新。他入吴拜二陆为师,陆云以“闻道夕死”的古训加以启发,其内在良知之心豁然顿悟,从而成就其忠义之人生。这一故事叙事生动,人物形象,特别是周处形象的塑造非常成功。斩虎刺蛟,何等凶险之事,周处之勇猛凶强可见一斑。但虎、蛟于乡,其害为次;周处改过自新,才是除却乡里最大祸害。今京剧据此编有\CJKunderwave{除三害}剧目上台演出,仍给现代观众以有益的启迪。}

\lettrine{15.2} 戴渊\myidx{戴渊}少时\footnote{戴渊:字若思,广陵(今江苏淮阴东南)人。},游侠不治行检\footnote{游侠:任侠使气,或仗义轻生,或打劫犯禁,言行非同一般。行检:品行操守。},尝在江淮间攻掠商旅\footnote{江淮间:长江、淮河之间广大地域,指今江苏、安徽北部一带。}。陆机\myidx{陆机}赴假还洛\footnote{赴假:销假。洛:洛阳,西晋首都。},辎重甚盛\footnote{辎重:行李。},渊使少年掠劫。渊在岸上,据胡床指麾左右\footnote{据:靠,坐。胡床:轻便交椅。指麾(huī挥):同“指挥”,发令调度。},皆得其宜。渊既神姿峰颖\footnote{峰颖:秀美杰出,非同凡俗。},虽处鄙事\footnote{鄙事:鄙陋之事,此指抢劫。},神气犹异。机于船屋上遥谓之曰:“卿才如此,亦复作劫邪\footnote{作劫:抢劫。}?”渊便泣涕,投剑归机。辞厉非常\footnote{辞厉:言辞慷慨。},机弥重之\footnote{弥:愈,更加。},定交\footnote{定交:确立朋友之谊。},作笔荐焉\footnote{作笔荐焉:写信推荐。}。{\fzxk\zihao{6}\textcolor{red}{虞预\CJKunderwave{晋书}曰:“机荐渊于赵王伦曰:‘盖闻繁弱登御,然后高墉之功显;孤竹在肆,然后降神之曲成。伏见处士戴渊,砥节立行,有井渫之洁;安穷乐志,无风尘之慕。诚东南之遗宝,朝廷之贵璞也。若得寄迹康衢,必能结轨骥騄,耀质廊庙,必能垂光瑜璠。夫枯岸之民,果于输珠;润山之客,列于贡玉。盖明暗呈形,则庸识所甄也。’伦即辟渊。”}} 过江,仕至征西将军\footnote{过江:渡过长江。此特指司马南渡,开国江南,史为东晋。}。

{\cangkai\zihao{5}【评】如果说前则周处是年轻无知,任侠狂妄,欺负乡亲的过错,用今天的话说,这还属于“人民内部”矛盾,还是民事纠纷范围;那么戴渊直接指挥抢劫,是明火执仗的强盗行为,其“过”已属于刑事犯罪范围内的“敌我矛盾”性质。他是否因家贫生活无着落而不得已行劫呢?看来不像。史称其“祖烈,吴左将军。父昌,会稽太守”,三代官宦子弟,家道富贵,生活无忧。其行劫乃出于“八旗”子弟仗势欺人以自我作乐的目的,至于做强盗实施抢劫是否犯罪而触犯法律,他则有恃无恐,这当与其缺乏良好教育而“游侠不治行检”的法盲直接相关。于此可见,不重视年轻人的教育,不仅会产生一般过失,甚至可能直接坠入犯罪深渊而严重干扰社会治安和安定团结。因此,重在教育青少年悔过自新,不仅是为国家培养未来人才,同时也关系到社会的安定生活。对于戴渊这一犯罪青年,陆机放长眼光,并没有一棍子打死,相反,他是治病救人,力促其悔过自新,成为国家栋梁。从故事看,这场打劫与反打劫的争斗,煞是热闹,而势均力敌。戴渊作为劫方首领,利用天时地利,布下圈套,主动出击,志在必得;而防劫一方,虽然事出突然,被动应对,但陆机兄弟是何等人物?能文能武,世代将门之后,自己又曾当过将军领兵作战,指挥打仗,受过正规训练,加以远途运输财物辎重,能不做一定防备部署吗?其登船屋从容发话,正可见防劫一方也并未慌张吃亏。在相持阶段,陆机一看戴渊的指挥调发,自然明白其内在的杰出潜能,在国家多事之秋,为国推荐人才,是义不容辞的责任。“卿才如此,亦复作劫邪?”简明尖锐,出于至诚之心,启发了戴渊的良心觉悟,恢复其报效国家朝廷的雄心壮志,这就叫御敌以攻心为上的策略,击垮对手的心理防线。戴渊也以诚相应,旋即悔过自新,终成为东晋开国功臣,最后死于王事,忠义之气,直贯日月。故事启发我们,对于犯过、甚至是犯罪的青少年,应实行给出路的政策,晓之以理,动之以情,悔过自新,则将利国利民。}




%%% Local Variables:
%%% mode: latex
%%% TeX-engine: xetex
%%% TeX-master: "../Main"
%%% End:
