%% -*- coding: utf-8 -*-
%% Time-stamp: <Chen Wang: 2025-12-06 12:29:34>

% ○ ◎ ‧ 「 」 『 』 々 ( ) “ ” ■ ^[一-龥]
% 【\([^】][^】][^】]+\)】 → {\\fzxk\\zihao{6}\\textcolor{red}{\1}}
% \(【评】.*\) → {\\cangkai\\zihao{5}\1}
% \(【题解】.*\) → {\\cangkai\\zihao{5}\1}
% 《\([^》]+\)》 → \\CJKunderwave{\1}
% ^\([0-9]+.[0-9]+\) → \\lettrine{\1}
% {\\fzxk\\zihao{6}\\textcolor{red}{[^o]*}}




\setlength{\parindent}{0pt}



\chapter{豪爽第十三}



{\cangkai\zihao{5}【题解】 豪爽者,豪放劲健与爽朗痛快也。本门所写的大多是性格雄豪而不落凡俗的士人故事。在\CJKunderwave{世说}作者看来,“豪爽”明显是一个褒义词。在魏晋黑暗的社会中,在虚伪名教旗帜的掩盖下,形势极其复杂,士人头上永远悬了一把达摩克利斯剑,一般人战战兢兢,不知何时会大难临头。在动辄得咎的现实阴影下,激起了士人的心理反弹,于是追求无拘无束的个性张扬,欣赏率真自然而敢于蔑视一切的豪爽性格,表现出超凡脱俗的痛快言行,成了士人心目中的理想境界与榜样。\CJKunderwave{世说}作者不以成败论英雄。本门13则故事中,王敦成为主角的有5则。王敦因叛逆而被钉在历史的耻辱柱上。但他在本门中,仍然成为士人心目中的“英雄”。桓温亦然。对于王、桓性格之豪爽,是因为他们手握军、政大权,有恃无恐。故应具体分析而给予恰当评价。但如祖逖性格之豪爽,则与其内心坦诚相联系。其忠肝义胆表现了爱国激情,则应予充分的肯定与颂扬。}

\lettrine{13.1} 王大将军\myidx{王敦}年少时\footnote{王大将军:王敦。年少:年轻。},旧有田舍名\footnote{田舍名:乡巴佬之称。},语音亦楚\footnote{语音亦楚:语音不雅正。楚:原指楚国方言,引申指代语音不近京洛标准音。}。武帝\myidx{司马炎}唤时贤共言伎艺事\footnote{伎艺事:音乐歌舞艺术。},人皆多有所知,唯王都无所关,意色殊恶\footnote{意色殊恶:神色很难看。}。自言知打鼓吹\footnote{鼓吹:原指吹打乐,这里指击鼓。},帝令取鼓与之。于坐振袖而起,扬槌奋击,音节谐捷\footnote{谐捷:和谐迅速。},神气豪上\footnote{豪上:豪放激越。},傍若无人\footnote{傍:通“旁”。},举坐叹其雄爽\footnote{雄爽:英雄豪爽。}。

{\cangkai\zihao{5}【评】故事发生在西晋太康年间,时武帝平吴统一中国不久,国家一片歌舞升平,骄奢淫逸之风自上而下迅速蔓延开来。当时王敦尚武帝女襄城公主不久,成为驸马都尉、太子舍人,是个可以接近皇帝的年轻宠臣。两晋之际,贵族子弟的成长有“尚文”、“尚武”两路,一般士人在和平风气下,大多“尚文”,学习文学“伎艺”,具有较高的艺术素质,助其生活享受;但如王敦之徒,倾向“尚武”,不废文学,但于“伎艺”则只图享受,实际是一窍不通。故武帝与时贤“共言伎艺事”时,王敦“意色殊恶”,因为他被人瞧不起,认为是乡巴佬。这在心理上打击很大。但是,他振袖扬槌,用谐捷劲健的鼓声,打破了压抑的气氛,发泄了胸中的愤懑,表达了雄壮的威势与一往无前的气魄,终于征服了大家,“叹其雄爽”。在这里,鼓既是打击乐器,用以指挥演奏,同时又是疆场上的战鼓,指挥三军前进。鼓声之中,神气毕现。后来的野心,在年轻时的鼓声中透露端倪。}

\lettrine{13.2} 王处仲\myidx{王敦}\footnote{王处仲:王敦字处仲。},世许高尚之目\footnote{许:称许。目:品目,品评。}。尝荒恣于色\footnote{荒恣于色:荒淫放荡而沉溺女色。},体为之弊\footnote{弊:疲惫,困倦。},左右谏之,处仲曰:“吾乃不觉尔\footnote{乃:竟,却。},如此者甚易耳。”乃开后閤\footnote{后閤:内宅。“閤”通“阁”。},驱诸婢妾数十人出路\footnote{出路:赶出上路。},任其所之。时人叹焉。{\fzxk\zihao{6}\textcolor{red}{邓粲\CJKunderwave{晋纪}曰:“敦性简脱,口不言财,其存尚如此。”}}

{\cangkai\zihao{5}【评】魏晋时妇女的命运,一方面贵族仕女具有较多的开放与自由,另一方面是奴婢的人身依附加强,主人对于奴婢可以随意占有,甚至是生杀予夺。后面\CJKunderwave{汰侈}门记载石崇杯酒杀美人的故事就是一例。魏晋士人的荒淫生活,在王敦身上有具体的反映。故事应该发生在东晋建国前后王敦得势之时。“驱诸婢妾数十人出路,任其所之”,表面是释放奴婢,使其获得自由,故人们许之以“高尚”。实际上,作为大将军的王敦昔日对于结发妻子襄城公主这个天潢之胄,尚且敢把她单身抛弃于兵荒马乱的青州,更何况是召之即来、挥之即去的婢妾呢?今日驱出数十,明日需要时即可招进数百。而且,当时形势动乱,驱赶婢妾之时,一时兴发,又没有做生活安排,诸婢妾一旦出门上路,又将如何生活?岂不沦为乞丐四处流浪,生命毫无保障。这不是变相杀人又是什么?豪爽其外,卑鄙其内,是谓王敦。}

\lettrine{13.3} 王大将军\myidx{王敦}首(自)目高朗疏率\footnote{王大将军:王敦。首目:面目。唐写本、袁本作“自目”,谓自我品评,于义更佳。},学通\CJKunderwave{左氏}\footnote{\CJKunderwave{左氏}:指\CJKunderwave{春秋}三传中的\CJKunderwave{左传}。相传为春秋时鲁国左丘明所著。}。{\fzxk\zihao{6}\textcolor{red}{\CJKunderwave{晋阳秋}曰:“敦少称高率通朗,有鉴裁。”}}

{\cangkai\zihao{5}【评】王敦年轻时即具野心,史称其“少有奇人之目”,奇人者,非常人之心可以衡量也。他吹嘘自己品格高尚、生性爽朗、言行疏放、性格率真,是士人的典范。这是一种自我炒作的舆论宣传。至于学通\CJKunderwave{春秋左氏传},更与其“尚武”精神有关。\CJKunderwave{左传}载春秋诸国征战兴亡之事甚详,可资定国安邦之借鉴。三国时蜀汉上将关羽,史称“好\CJKunderwave{左氏传},讽诵略皆上口”(见\CJKunderwave{三国志·蜀书·关羽传}裴注引\CJKunderwave{江表传})。晋初灭吴有功的征南将军杜预,自称“有\CJKunderwave{左传}癖”(见\CJKunderwave{术解}第4则刘注)。前贤之士历历在目。未来的大将军王敦“学通\CJKunderwave{左传}”,自是别有用心,而非如一般的儒生诵读经典。按:与王敦“自目”相较,当时人对年轻的王敦做出完全相反的品评,如其同事太子洗马潘滔云:“处仲蜂目已露,但豺声未振,若不噬人,亦当为人所噬。”其族弟王导亦称:“处仲若当世,心怀刚忍,非令终也!”(见\CJKunderwave{晋书}敦传)旁观者清,时人品目,更合实际。}

\lettrine{13.4} 王处仲\myidx{王敦}每酒后\footnote{王处仲:王敦。},辄咏“老骥伏枥,志在千里;烈士莫年,壮心不已\footnote{辄:总是。“老骥”四句:曹操乐府诗\CJKunderwave{步出夏门行·神龟虽寿}句。老骥,老马。枥,马厩。烈士,志存功业之士。莫,通“暮”。壮心,雄心壮志。}”。{\fzxk\zihao{6}\textcolor{red}{魏武帝乐府诗。}} 以如意打唾壶\footnote{如意:器物名。用竹、玉、骨等制成,头作灵芝或云叶形,柄微曲。供指划或观赏之用。唾壶:痰盂。},壶口尽缺。

{\cangkai\zihao{5}【评】故事发生在元帝开基江南的东晋初期,琅邪王氏敦、导兄弟等拥立有功,群从显贵,故有“王与马,共天下”之传言。当时王敦作为大将军、荆州刺史,专任阃外,掌控雄师,威权莫贰,遂萌异志而有问鼎之心。这就引起了皇帝及朝廷大臣的猜忌,元帝起用刘隗、刁协等以为心膂,力排琅邪王氏。于是君臣嫌隙遂构,引起王敦的愤怒。曹操\CJKunderwave{龟虽寿}是优秀诗篇,表达了不甘衰老而奋斗不息、建功立业的积极有为的精神。但王敦则借以表达其实现野心的意志。慷慨悲壮的歌声中,却隐约分辨出豺狼噬人的嚎叫。同一首诗,所用不同,则性质有异。王敦之歌,实是点金成铁,化神奇为腐朽。但击壶口缺的细节生动,一个奸雄的形象,脱颖而出。故如王世贞所评:“老贼故自豪”,“其人不足言,其意乃大可悯矣!”}

\lettrine{13.5} 晋明帝\myidx{司马绍}欲起池台\footnote{晋明帝:司马绍,字道畿。元帝长子。},元帝\myidx{司马睿}不许\footnote{元帝:司马睿,东晋开国皇帝。}。帝时为太子,好武养士,一夕中作池,比晓便成\footnote{比晓:到天亮。},今太子西池是也。{\fzxk\zihao{6}\textcolor{red}{\CJKunderwave{丹阳记}曰:“西池,孙登所创,\CJKunderwave{吴史}所称西苑也,明帝修复之耳。”}}

{\cangkai\zihao{5}【评】司马绍立为皇太子,在元帝太兴元年(318)三月之时。永昌元年(322)王敦举兵向阙,元帝忧愤告谢,太子绍即位。据此,故事应发生在太兴年间(318—321)。与乃父之失驭强臣,下陵上辱不同,明帝生母燕人荀氏,种族鲜卑,故其身上有一半是鲜卑“胡”人血统,性刚烈果敢,“一夕中作池,比晓便成”,见其雷厉风行之风。后来他冒险亲入王敦叛军大本营侦察,旋即击灭叛逆,其英明果断,在凿池建台的过程中已见端倪。这是其优点。但从另一角度看,当时国家草创,经济匮乏,兵凶岁饥,民生凋敝,事极艰虞。故辅政王导,每劝元帝“克己励节”,元帝也提倡勤俭兴邦,史称其“所幸郑夫人衣无文彩”,这与时代需要相合。但太子绍则反之,乖违时命而兴建楼堂馆所,开凿池台,唯见少年心性而不计全局,其享国岂可长乎!}

\lettrine{13.6} 王大将军\myidx{王敦}始欲下都\footnote{王大将军:王敦。下都:顺江直下到京师建康。},更分树置\footnote{更分树置:更动处分,另有树置。“更分”,袁本作“处分”,义止处分。当以宋本“更分”为佳。},先遣参军告朝廷\footnote{参军:军府中重要僚佐。},讽旨时贤\footnote{讽旨:委婉传达意图。}。祖车骑\myidx{祖逖}尚未镇寿春\footnote{祖车骑:祖逖死后赠车骑将军,故称。元帝时,祖逖任奋威将军、豫州刺史,北上抗敌。后退守淮南,镇寿春(今安徽寿县)。},瞋目厉声语使人曰:“卿语阿黑:{\fzxk\zihao{6}\textcolor{red}{敦小字也。}} 何敢不逊\footnote{不逊:放肆无礼。}!催摄面去\footnote{催摄面:赶快收起张牙舞爪的脸面。又,催者速也,摄者撤也。面,敬胤注引作“回”。催摄面,即赶快撤回去。别是一解,于义亦通。},须臾不尔\footnote{须臾不尔:稍有徘徊。},我将三千兵槊脚令上\footnote{将:率领。槊:似长矛的兵器。这里名词动词化,以槊刺脚。上:溯江而上,退回去。}。”王闻之而止。

{\cangkai\zihao{5}【评】祖逖之叱王敦,真将军也!语呼“阿黑”,称其小名,非亲非故,则蔑贱视之也,开口即煞大将军的威风。“不逊”云云,指王敦对朝廷的态度,“何敢”之斥,义正词严,凛然作色而情见乎辞。以下“催摄面去”四句,叱其收拾颜面赶快回去,如稍犹豫,则我自率三千精兵,以长槊戳其脚溯江而上,送尔等回老家。时王敦镇武昌,故云。活用口语方言,声喝酣畅淋漓,动作劲疾痛快,气势一往无前,说得虎虎有生气,而令王敦生畏。成功的语言艺术,生动地描述了一个忠心国事而不畏强暴的将军形象,其爱国激情义薄云天,可敬可畏。}

\lettrine{13.7} 庾稚恭\myidx{庾翼}既常有中原之志\footnote{庾稚恭:庾翼字稚恭。亮、冰之弟。亮死,代兄任荆州刺史、安西将军。中原之志:指恢复中原故国的理想。},文康\myidx{庾亮}时\footnote{文康:指庾亮,卒谥文康,故称。},权重未在己。及季坚\myidx{庾冰}作相\footnote{季坚:庾冰字季坚。苏峻乱后,继兄亮辅政为相。},忌兵畏祸,与稚恭历同异者久之\footnote{同异:偏意副词,义偏于异,即不同。},乃果行\footnote{乃果行:才得以实行。}。倾荆、汉之力,穷舟车之势,师次于襄阳\footnote{师次:军队驻扎。襄阳:城名,在今湖北省北部。},{\fzxk\zihao{6}\textcolor{red}{\CJKunderwave{汉晋春秋}曰:“翼风仪美劭,才能丰瞻,少有经纬大略。及继兄亮居方州之任,有匡维内外、扫荡群凶之志。是时杜乂、殷浩诸人盛名冠世,翼未之贵也,常曰:‘此辈宜束之高閤,俟天下清定,然后议其所任耳!’其意气如此。唯与桓友善,桓期以宁济宇宙之事。初,翼辄发所部奴及车马万数,率大军入沔,将谋伐狄,遂次于襄阳。”\CJKunderwave{翼别传}曰:“翼为荆州,雅有大志,每以门地威重,兄弟宠授,不陈力竭诚,何以报国?虽蜀阻险塞,胡负凶力,然皆无道酷虐,易可乘灭。当此时,不能罪除二寇以复王业,非丈夫也。于是征役三州,悉其帑实,成众五万,兼率荒附,治戎大举,直指魏、赵,军次襄阳,耀威汉北也。”}} 大会参佐,陈其旌甲\footnote{陈其旌甲:陈列雄师阵势。},亲授弧矢\footnote{亲授弧矢:“授”,唐写本作“援”。亲自拉弓放箭。},曰:“我之此行,若此射矣。”遂三起三叠\footnote{三起三叠:三发三中。起,发。叠,击鼓。徐震堮\CJKunderwave{校笺}谓军中阅射“中的则击鼓为号”。}。徒众属目\footnote{属目:贯注。},其气十倍。

{\cangkai\zihao{5}【评】晋康帝建元元年(343),安西将军、荆州刺史庾翼,率师北伐,屯兵襄阳。史称“翼雅有大志,欲以灭胡平蜀为己任,言论慷慨,形于辞色”(\CJKunderwave{晋书}翼传)。故事发生于是年。后来,因康帝崩,兄坚卒,家事国事,殷忧迭至,加以朝廷诸臣多有异同之论,北伐之事,不果于行,惜其无成,但不可以胜败论英雄。东晋孱弱,朝廷纷争,气自不振。庾翼师出荆汉,振臂高呼,志复中原,其阳刚之气,鼓动国家,民心振奋。故事写其北伐之事,既有概貌,如谓“倾荆汉之力,穷舟车之势,师屯襄阳”,叙事简明有序。至其亲授弧矢,三起三叠,铮铮誓言,掷地铿然有声,故三军慷慨,“其气十倍”,通过典型细节的描绘,给人以难忘的印象,一位豪爽慷慨的爱国统帅,跃然纸上。}

\lettrine{13.8} 桓宣武\myidx{桓温}平蜀\footnote{桓宣武:桓温。卒谥宣武,故称。蜀:十六国汉据蜀地割据立国,从李特起兵,至李势降晋,共四十六年。},集参僚置酒于李势殿\footnote{李势:十六国汉之第二代君主。},巴、蜀缙绅莫不来萃\footnote{巴、蜀:指巴郡、蜀郡。缙绅:指缙笏垂绅的士大夫。萃:聚集。}。桓既素有雄情爽气,加尔日音调英发,叙古今成败由人,存亡系才,其状磊落\footnote{其状磊落:样子英伟慷慨,洒脱不凡。},一坐叹赏\footnote{叹赏:叹美赞赏。}。既散,诸人追味馀言,于时寻阳周馥\myidx{周馥}曰\footnote{寻阳:郡名,治所在今九江。周馥:晋有二周馥:一是西晋周馥,字祖寅,淮南人;一是东晋周馥,字湛隐,寻阳人。参见刘注。}:“恨卿辈不见王大将军\footnote{恨:遗憾。王大将军:王敦。}!”{\fzxk\zihao{6}\textcolor{red}{\CJKunderwave{中兴书}曰:“馥,周抚孙也,字湛隐。有将略,曾作敦掾。”}}

{\cangkai\zihao{5}【评】在东晋历史上,桓温一代枭雄,集雄心与野心于一身。镇荆州、平蜀,是成就其事业及扩大势力的转折点,也就是说,桓温正处于上升时期。平蜀之战始于穆帝永和二年(346),三年蜀汉主李势投降。桓温声誉日隆,地位腾腾直上,如日中天。观其置酒李势宫,可想象其雄姿英发形象。故其雄情爽气,声威自见。其“叙古今成败由人,存亡系才”,善于总结历史兴衰治乱经验教训,志在招揽人才,为己所用,于此见其志慨、心胸与气魄,不愧是一个称雄一世的大政治家。桓温是故事的当然主角。但故事的第二主角是王敦。周馥之言:“恨卿不见王大将军!”犹如篇末点题,以王敦来衬托桓温之奸雄形象。前\CJKunderwave{赏誉}第79则载,桓温经王敦墓,连呼“可儿!可儿!”称美王大的非常之举,以之作为自己心目中追求的理想榜样。故王世贞评曰:“敦虽败,令人有馀畏,桓温所以叹为可儿。”温之心迹昭然若揭,其人生结局,亦可预料。}

\lettrine{13.9} 桓公\myidx{桓温}读\CJKunderwave{高士传}\footnote{桓公:桓温。\CJKunderwave{高士传}:晋皇甫谧著。宋本刘注“谧”误作“谥”。记载古代隐士的故事。今有辑佚本。},至於陵仲子\myidx{陈仲子}\footnote{於陵仲子:战国时齐人,后隐居于楚国於陵。详参刘注。},便掷去,曰:“谁能作此溪刻自处\footnote{溪刻:苛刻,刻薄。}!”{\fzxk\zihao{6}\textcolor{red}{皇甫谥(谧)\CJKunderwave{高士传}曰:“陈仲子字子终,齐人。兄载(戴),相齐,食禄万钟。仲子以兄禄为不义,乃适楚,居於陵。曾乏粮三日,匐匍而食井李之实,三咽而后能视。身自织屦,令妻擗煗,以易衣食。尝归省母,有馈其兄生鹅者,仲子鶂鶂曰:‘恶用此鶂鶂为哉!’后母杀鹅,仲子不知而食之。兄自外入,曰:‘鶂鶂肉邪!’仲子出门哇而吐之。楚王闻其名,聘以为相,乃夫妇逃去,为人灌园。”}}

{\cangkai\zihao{5}【评】\CJKunderwave{高士传},晋皇甫谧著,原记载上古至魏晋隐逸之士七十二人。已亡佚,今传为辑佚本,已杂入嵇康\CJKunderwave{高士传}及\CJKunderwave{后汉书}有关传记,增益为九十六人。隐逸之士,古已有之。\CJKunderwave{易}有\CJKunderwave{遁}卦,提倡隐遁而亨,与时偕行之道,思想影响很大,因而代有传人。但隐逸之士,其漱流激清,寝巢韬耀,重在淡泊名利,而修至道之乐,故悔吝弗生。而桓温奸雄,久怀异心,志向莫测,史称负其才力“欲先立功河朔,还受九锡”,所走道路,与隐逸之士背道而驰。曾中夜抚枕而叹曰:“既不能流芳后世,不足复遗臭万载邪!”志在功名富贵,社稷江山。奸雄屈人之节,隐士则不屈其节,二者相互水火,故桓温读\CJKunderwave{高士传}而格格不入,宜哉!}

\lettrine{13.10} 桓石虔\myidx{桓石虔}\footnote{桓石虔:桓温弟豁之庶长子。官至豫州刺史。},司空豁\myidx{桓豁}之长庶也\footnote{司空豁:桓豁,官荆州刺史、征西大将军,卒赠司空。长庶:庶出长子。},{\fzxk\zihao{6}\textcolor{red}{\CJKunderwave{豁别传}曰:“豁字朗子,温之弟。累迁荆州刺史,赠司空。”}} 小字镇恶。年十七八,未被举\footnote{举:指庶子正式被确定身份。},而童隶已呼为镇恶郎\footnote{童隶:年轻童仆。郎:奴仆对少主人的称呼,相当于“少爷”。}。尝住宣武\myidx{桓温}斋头\footnote{宣武:桓温卒谥宣武,故称。斋头:书斋里面。}。从征枋头\footnote{枋头:地名,在今河南浚县西南。},车骑冲\myidx{桓冲}没陈\footnote{车骑冲:桓冲,温弟,官车骑将军,故称。参前\CJKunderwave{夙惠}第7则注。陈:通“阵”。},左右莫能先救。宣武谓曰:“汝叔落贼,汝知不?”石虔闻之,气甚奋,命朱辟\myidx{朱辟}为副,策马于数万众中,莫有抗者,径致冲还\footnote{径致冲还:直接救回桓冲。},三军叹服。河朔后以其名断疟\footnote{断疟:禁断疟鬼。}。{\fzxk\zihao{6}\textcolor{red}{\CJKunderwave{中兴书}曰:“石虔有才干,有史学。累有战功,仕至豫州刺史,赠后军将军。”}}

{\cangkai\zihao{5}【评】故事发生在晋燕枋头之战,时间是废帝海西公太和四年(369),以桓温北伐大败而归告终。故事生动刻画了一位少年将军叱咤风云的英雄形象。“策马于数万众中,莫有抗者”,写活了一往无前的威势。“河朔后以其名断疟”,则是因其英勇与“镇恶”名实相符,借用民间传说,更衬托其神勇气概。与统帅桓温枋头大败相较,刘辰翁评云:“小名镇恶,遂能断疟,第不知当时桓温愧此儿不?”此问发人深思。}

\lettrine{13.11} 陈林道\myidx{陈逵}在西岸\footnote{陈林道:陈逵字林道。袭封广陵公。参前\CJKunderwave{品藻}第59则注。西岸:建康西边的长江北岸。},{\fzxk\zihao{6}\textcolor{red}{\CJKunderwave{晋阳秋}曰:“逵为西中郎将,领淮南太守,戍历阳。”}} 都下诸人共要至牛渚会\footnote{要:通“邀”。牛渚会:聚会牛渚。牛渚,山名,在今安徽当涂县西北,山脚入长江处称采石矶。},陈理既佳,人欲共言折(析)\footnote{“陈理既佳”二句:意谓陈逵精于玄理,众人想和他一起谈玄析理。折,唐写本作“㭊”,即“析”字,是。言析,谈玄析理。}。陈以如意柱颊\footnote{如意:器物名,见前第四则注。},望鸡笼山\footnote{鸡笼山:山名,在建康西北,状如鸡笼,故称。},叹曰:“孙伯符\myidx{孙策}志业不遂\footnote{孙伯符:孙策字伯符。三国时东吴孙氏政权的开基人。}!”{\fzxk\zihao{6}\textcolor{red}{\CJKunderwave{吴录}曰:“长沙桓王讳策,字伯符,吴郡富春人。少有雄姿风气,年十九而袭业,众号孙郎。平定江东,为许贡客射破其面,引镜自照,谓左右曰:‘面如此,岂可复立功乎?’乃谓张昭曰:‘中国方乱,夫以吴越之众,二(三)江之固,足以观成败。公等善相吾弟!’呼大皇帝,授以印绶,曰:‘举江东之众,决机于两陈之间,卿不如我;任贤使能,各尽其心,我不如卿。慎勿北渡!’语毕而薨,年二十有六。”}} 于是竟坐不得谈\footnote{竟坐:终坐。}。

{\cangkai\zihao{5}【评】据\CJKunderwave{晋书·穆帝本纪},永和五年八月,征北大将军褚裒北伐失败,“退在广陵,西中郎将陈逵焚寿春而遁”。据理推之,故事当发生于永和五年(349)北伐失败后,痛心疾首,无意谈玄。又前\CJKunderwave{品藻}第59则谓谢安“润于林道”,则陈逵乃一时谈玄名士,稍逊于安。陈逵虽精玄理,但一心志在恢复。牛渚山与鸡笼山,是昔日孙策战胜刘繇,决战江东的故地。陈逵坐牛渚望鸡笼山而叹孙伯符“志业不遂”,吊古抒怀,借他人之酒杯,以浇自己的块垒。朱铸禹\CJKunderwave{汇校集注}引日人尾张\CJKunderwave{世说笺本}评曰:“言可对我者,特孙伯符一人,而志业不遂,可惜哉!盖蔑视诸人,以如意柱颊,豪爽之态可见,以此诸人为其气所慑,竟不得谈也。”}

\lettrine{13.12} 王司州\myidx{王胡之}在谢公\myidx{谢安}坐\footnote{王司州:王胡之,字修龄,琅邪人。王导族子。官至司州刺史,故称。参\CJKunderwave{言语}第81则注。谢公:谢安。坐:坐席间。},咏“入不言兮出不辞,乘回风兮载云旗\footnote{“入不言兮出不辞”二句:\CJKunderwave{楚辞·九歌·少司令}之诗句。辞,告别。回风,旋风。云旗:张云为旗。}”,{\fzxk\zihao{6}\textcolor{red}{\CJKunderwave{离骚·九歌·少司命}之辞。}} 语人云:“当尔时\footnote{尔时:此时。},觉一坐无人。”

{\cangkai\zihao{5}【评】魏晋上流贵族社会,如王恭所说:“痛饮酒,熟读\CJKunderwave{离骚},便可称名士。”(\CJKunderwave{任诞}第53则)当时风气,熟读\CJKunderwave{楚辞},是成为名士的重要修养。故王胡之在谢安坐席之间,高咏\CJKunderwave{少司命}句,借古抒怀,神态如见。王胡之其人,谢安称赏,谓“可与林泽游”(\CJKunderwave{赏誉}第125则)。胡之曾至吴兴印渚观赏山水风光,叹曰:“非唯使人情开涤,亦觉日月清朗。”(\CJKunderwave{言语}第81则)于此见其艺术化的审美人生态度及其高尚脱俗情怀。其高咏“入不言兮出不辞,乘回风兮载云旗”,抒发了张扬自我、超越世俗而自由自在的精神理想。当他一旦进入了\CJKunderwave{少司命}所描述的想象世界中,为情所动,故觉一坐无人,进入了一个全新的自由人生之境界。}

\lettrine{13.13} 桓玄\myidx{桓玄}西下\footnote{桓玄:字敬道,大司马温少子。后篡晋自立,国号楚。旋即被刘裕击杀。参\CJKunderwave{德行}第41则注。西下:沿长江顺流而下。},入石头\footnote{石头:城名,在建康西边保卫京师的军事要塞。},外白司马梁王\myidx{司马珍之}奔叛\footnote{外白:役吏报告。梁王:司马珍之,晋宗室。奔叛:叛亡,逃走。}。{\fzxk\zihao{6}\textcolor{red}{\CJKunderwave{续晋阳秋}曰:“梁王珍之,字(景)度。”\CJKunderwave{中兴书}曰:“初,桓玄篡位,国人有孔璞者,奉珍之奔寻(寿)阳。义旗既兴,归朝廷,仕至太常卿,以罪诛。”}} 玄时事形已济\footnote{事形:形势,事态。济:成功。},在平乘上笳鼓并作\footnote{平乘:大型楼船。笳鼓:泛指奏乐。笳,管乐器。鼓,打击乐器。},直高咏云:“箫管有遗音,梁王安在哉\footnote{“箫管有遗音”二句:阮籍五言\CJKunderwave{咏怀}诗第31首诗句。梁王,原指战国时魏王。}?”{\fzxk\zihao{6}\textcolor{red}{阮籍\CJKunderwave{咏怀诗}也。}}

{\cangkai\zihao{5}【评】性格豪爽人有不同。以本门故事为例,如祖逖、庾翼等,发自内心坦诚,故誓言铮铮而感天动地;而如王敦、桓温、桓玄等奸雄,则因掌控雄师,操纵朝政而废立自专,其所谓豪爽,实具雄厚资本,助其一时声势,而与内心诚孚相乖。桓玄于安帝隆安三年(399)计袭荆州,杀好友殷仲堪及杨佺期后,自任荆、江二州刺史,掌控了长江中上游。元兴元年(402),又率军自江陵顺流东下,攻入京师建康。故事发生于桓玄叛逆形势在握之时,当然实具“豪爽”资本。其咏阮籍\CJKunderwave{咏怀}诗句,情景表面相似;欢宴之乐馀音尚存,但主人梁王何在?以古喻今,讽刺司马梁王的失败。但实质大异:阮诗借咏古事来抨击魏王奢靡误国而身死国灭,抒发忧国忧民之情。但桓玄则仅借“梁王”之称,以自鸣得意而已。这是古人“断章取义”的伎俩,其内在野心毕呈脸上,而与阮诗不能同日而语。}






%%% Local Variables:
%%% mode: latex
%%% TeX-engine: xetex
%%% TeX-master: "../Main"
%%% End:
