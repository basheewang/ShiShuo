%% -*- coding: utf-8 -*-
%% Time-stamp: <Chen Wang: 2025-12-02 21:47:11>

% ○ ◎ ‧ 「 」 『 』 々 ( ) “ ” ■ ^[一-龥]
% 【\([^】][^】][^】]+\)】 → {\\fzxk\\zihao{6}\\textcolor{red}{\1}}
% \(【评】.*\) → {\\cangkai\\zihao{5}\1}
% \(【题解】.*\) → {\\cangkai\\zihao{5}\1}
% 《\([^》]+\)》 → \\CJKunderwave{\1}
% ^\([0-9]+.[0-9]+\) → \\lettrine{\1}
% {\\fzxk\\zihao{6}\\textcolor{red}{[^o]*}}


\setlength{\parindent}{0pt}

\chapter{识鉴第七}


{\cangkai\zihao{5}【题解】识鉴,包括对事物的见地和对人才的赏拔。因识鉴须通过被赏者的外在言谈举止、气度风神等因素为切入点,进而做“知人论世”的内在观照,希冀把握其本质、神髓,故对鉴赏者提出了相当高的要求。自汝南“月旦”以来,乡邦贤达主持品评、鉴赏,士林间品评成风,以此确定一个人的社会声誉,甚至会影响其仕途发展。如曹操呈现在历史上的是“乱世之英雄,治世之奸贼”的枭雄形象,殊不知,曹操原是默默无闻的人物,正是靠了乔玄的这句评价,从而改变了他的社会地位。看来,名士清流一言九鼎,往往起着世俗皇权也无法干预的作用,这是一种无形的舆论力量。在老庄玄学兴盛、佛教勃兴以后,士大夫阶层更加重视人的精神、悟性,以至于整个魏晋时代,形成了崇拜天才、颖悟、神鉴的社会风气。}

{\cangkai\zihao{5}本门28则故事,多为人伦识鉴,我们可以借此了解魏晋士大夫见微知著、审时度势、料事如神的洞察力,和在关键时刻急流勇退、避祸全身的机智决断。当然,也有因识鉴力不足而或身死人手,或朋友反目成仇的事例,这些适足为反面教材,使人警醒。张季鹰的故事最为典型。季鹰为齐王冏东曹掾,在洛阳,见秋风起而思吴中羹脍,曰:“人生贵得适意尔,何能羁宦数千里以要名爵?”,遂命驾便归。其后,齐王败,被人赞叹为“见机”。张季鹰的“命驾便归”与陶渊明的“不为五斗米折腰”,在本质上都是以自由生命对抗官场樊篱,表征了不受羁绊的名士风度,故千载以来传为佳话。}

{\cangkai\zihao{5}本门还记载了少数民族领袖石勒读\CJKunderwave{汉书}的故事,三言两语而勾勒出一位虽不识字,却善于在戎马生涯和社会生活这本大书中,学习知识、总结道理的一代枭雄形象,读后令人感佩其钻研精神;书中还描绘了王胡之避司马无忌之难,而巧遇少年车胤于篱中,发出“此儿当致高名”的赞叹,堪称极富神韵的笔调。}

{\cangkai\zihao{5}总之,\CJKunderwave{识鉴}门的故事,不仅牵系个人前途,更关系到一代精英和国家人才的建设。}

\lettrine{7.1} 曹公\myidx{曹操}少时见乔玄\myidx{乔玄}\footnote{曹公:曹操。乔玄(108—183):字公祖,东汉末年梁国睢阳(今河南商丘南)人。},玄谓曰:“天下方乱,群雄虎争\footnote{群雄虎争:指东汉末年黄巾起义之后的州郡牧守、地方军阀的割据纷争局面。},拨而理之\footnote{拨而理之:指治理乱世。拨,治,治理。},非君乎?然君实是乱世之英雄,治世之奸贼\footnote{“乱世之英雄,治世之奸贼”:这两句是乔玄给曹操的评价。东汉用征辟、察举等制度来选拔人才,选拔的标准是依据乡闾宗党平日对某个人长期观察而得出的舆论鉴定,即“清议”。曹操得到乔玄的赏识,可以提高其在士林间的地位,不再加以歧视。}。恨吾老矣,不见君富贵,当以子孙相累。”{\fzxk\zihao{6}\textcolor{red}{\CJKunderwave{续汉书}曰:“玄字公祖,梁国睢阳人。少治\CJKunderwave{礼}及\CJKunderwave{严氏春秋},累迁尚书令。玄严明(有)才略,长于知人。初,魏武帝为诸生,未知名也,玄甚异之。”\CJKunderwave{魏书}曰:“玄见太祖曰:‘吾见士多矣,未有若君者。天下将乱,非命世之才不能济也。能安之者,其在君乎?’”案:\CJKunderwave{世语}曰:“玄谓太祖:‘君未有名,可交许子将。’太祖乃造子将,子将纳焉。”孙盛\CJKunderwave{杂语}曰:“太祖尝问许子将:‘我何如人?’固问,然后子将答曰:‘治世之能臣,乱世之奸雄。’太祖大笑。”\CJKunderwave{世说}所言谬矣。}}

{\cangkai\zihao{5}【评】汤用彤先生\CJKunderwave{读〈人物志〉}文以为:“英雄者,汉魏间月旦人物所有名目之一也。”汉末乡党清议,许劭是大名士,主持评论人物,每月更换,称“月旦评”。天下大乱,拨乱反正则仰仗群雄,豪杰并起,欲平定天下,均以英雄自许。曹操父嵩是大宦官曹腾的养子,史称“莫能审其出生本末”(\CJKunderwave{三国志}魏武本纪)。这样的家庭背景,当然与清流无缘;而亦步亦趋地走修、齐、治、平的传统正路,又非曹操所愿。恰好动乱的时代,为曹操实现政治理想,提供了大有作为的天地。他与群雄平定董卓之乱,镇压黄巾起义,又于官渡之战中击败袁绍,逐步统一了北中国。他一生身经百战,屡建奇功,抱负远大,知贤善任,不愧为一代英主。曹操得到乔玄赏识,又得到主持“月旦评”的许劭的首肯,由此引起士大夫的普遍注意。“乱世之英雄,治世之奸贼”一语,抓住了曹操立身行事之大体。治世推崇文质彬彬的君子,乱世则呼唤不守常规的枭雄。世运代变,不可一概而论。此所谓“时势造英雄”。曹操一生多次打破儒家繁文缛节、条条框框,不拘礼法小节,确是不可以常规标准衡量的特出之人。乔玄透过“天下方乱,群雄虎争”的纷扰世态,拔英雄于庸众,确有识鉴。}

\lettrine{7.2} 曹公问裴潜曰\footnote{曹公:曹操。裴潜:三国魏河东闻喜(今属山西)人。黄巾起义时,避乱荆州依附刘表。曹操定荆州,以他为参丞相军事。魏明帝时为尚书令。}:“卿昔与刘备共在荆州\footnote{刘备:字玄德,涿郡涿县(今河北)人。汉末天下大乱,刘备与曹操、孙权三分天下,备占据西蜀,建国蜀,在位三年。荆州:汉刺史部之一,辖境主要在今湖南、湖北两省地区。},卿以备才如何?”潜曰:“使居中国\footnote{居:占据,据有。中国:指中原地区。},能乱人,不能为治;若乘边守险\footnote{乘边守险:占据边疆。乘,防守。},足为一方之主。”{\fzxk\zihao{6}\textcolor{red}{\CJKunderwave{魏志}曰:“潜字文行,河东人。避乱荆州,刘表待以宾客礼。潜私谓王粲、司马芝曰:‘刘牧非霸王之才,而欲以西伯自处,其败无日。’累迁尚书令,赠太常。”}}

{\cangkai\zihao{5}【评】据余嘉锡\CJKunderwave{笺疏}考证,故事当发生在建安二十年冬操降张鲁、备争汉中之际。当时,曹操以“天下归心”自命,故极注重各方诸侯势力的消长盈虚,以做到“知己知彼,百战不殆”。\CJKunderwave{三国演义}中为世人熟知的“青梅煮酒论英雄”故事,意在说明曹操视刘备为中原逐鹿的真正敌手,因而有意试探。此则裴潜之言,似有超前预测能力,能预知三国鼎立形势。明王世懋亦有此问,曰:“此语似事后论人,不宜预知至此。”经前贤余嘉锡先生考证:方曹操与裴潜问答之时,潜知备才足以定蜀,而地狭兵少,必不能遽复中原。操虽强盛,而所值乃当事人杰,亦决不能并蜀。故潜预测形势而为是言,其远见卓识,虽非诸葛之比,但能令曹操信服,亦属非凡。}

\lettrine{7.3} 何晏、邓飏、夏侯玄并求傅嘏交\footnote{何晏:字平叔,三国魏人。何晏字平叔,\CJKunderwave{三国志}作何进孙。少有才,正始初为曹爽所用,名盛于天下。好老庄,与夏侯玄、王弼等倡导玄学,开魏晋清谈之风注。邓飏(?—326):字玄茂,三国魏南阳宛(今河南南阳)人。为人浮华贪贿,京师人传“以官易富邓玄茂”。因党曹爽被诛。夏侯玄:字太初,三国魏人。(209—254):字太初,三国魏人。曹爽辅政时,他以爽姑之子受重用。曹爽被诛,玄废黜。后与李丰等谋杀司马师,事败,同被诛。他是早期的玄学领袖人物。傅嘏:字兰硕,三国魏人。善言义理,好言才性。},而嘏终不许。{\fzxk\zihao{6}\textcolor{red}{\CJKunderwave{魏略}曰:“邓飏字玄茂,南阳宛人,邓禹之后也。少得士名。明帝时,为中书郎,以与李胜等为浮华,被斥。正始中,迁侍中、尚书。为人好货,臧艾以父妾与飏,得显官。京师为之语曰:‘以官易富(妇)邓玄茂。’何晏选不得人,颇由飏。以党曹爽诛。”}} 诸人乃因荀粲说合之\footnote{荀粲:字奉倩,颍川人。说合:从中介绍,把双方说到一起。},谓嘏曰:“夏侯太初,一时之杰士\footnote{一时:当代。杰士:俊杰之士,杰出的人。},虚心于子,而卿意怀不可。交合则好成,不合则致隙\footnote{致隙:造成隔阂。}。二贤若穆,则国之休\footnote{穆:通“睦”,和睦。休:吉庆,祥福。}。此蔺相如所以下廉颇也\footnote{蔺相如下廉颇:事见\CJKunderwave{史记·廉颇蔺相如列传}。此借喻傅嘏当与夏侯玄交好。}。”{\fzxk\zihao{6}\textcolor{red}{\CJKunderwave{史记}曰:“相如以功大拜上卿,位在廉颇右。颇怒,欲辱之。相如每称疾,望见,引车避匿。其舍人欲去之,相如曰:‘夫以秦王之威,而吾廷叱之。何畏廉将军哉?顾秦强赵弱,秦以吾二人,故不敢加兵于赵。今两虎斗,势不俱生。吾以公家急而后私雠也。’颇闻谢罪。”}} 傅曰:“夏侯太初,志大心劳,能合虚誉,诚所谓利口覆国之人\footnote{利口覆国:能言善辩,颠覆国家。\CJKunderwave{论语·阳货}:“恶利口之覆邦家者。”覆,倾败。}。何晏、邓飏,有为而躁,博而寡要,外好利而内无关籥\footnote{关籥:门闩之类横持门户之木,引申为检点、约束。},贵同恶异,多言而妒前\footnote{妒前:指嫉恨比自己强的人。},多言多衅,妒前无亲。以吾观之,此三贤者,皆败德之人尔。远之犹恐罹祸,况可亲之邪?”后皆如其言。{\fzxk\zihao{6}\textcolor{red}{\CJKunderwave{傅子}曰:“是时,何晏以才辩显于贵戚之间。邓飏好交通,合徒党,鬻声名于闾阎。夏侯玄以贵臣子,少有重名。皆求于嘏,嘏不纳也。嘏友人荀粲,有清识远志,然犹劝嘏结交云。”}}

{\cangkai\zihao{5}【评】何晏好利,邓飏贪货。二人并祖述老庄,鼓扇玄虚。从传统伦理礼教的角度看,二人绝非淳厚无瑕的高士,亦无守正直行的特操。故傅嘏不与其交友,看似“道不同不相与谋”不交非类,乃洁身自好的高蹈之举;实由于易代之际,残酷政治斗争的不同利益分野所致。在司马氏与曹魏夺权斗争中,傅嘏属司马氏一党。司马懿请为从事中郎,曹爽之诛,齐王之废,嘏皆参与其事。傅嘏与何晏、邓飏及夏侯玄不平,皆因其为魏之宗室或党羽,而嘏与锺会、何曾等善,皆司马氏之党羽也。又时有“才性”之争,傅嘏、锺会主才性同合,李丰、王广持才性异离。在常见的思想交锋背后,仍是政治站队的立场问题。嘏之才性论,实为司马氏篡夺行为张本。傅嘏之识鉴,非为善于论人,而是善于站队,从事态发展的风吹草动中,敏锐地嗅出了山雨欲来风满楼的时代动向。}

\lettrine{7.4} 晋武帝讲武于宣武场\footnote{晋武帝:司马炎。为晋第一代君主。宣武场:魏晋都城洛阳的讲武之所。},帝欲偃武修文\footnote{偃武修文:停止战备,修明文教。},亲自临幸,悉召群臣。山公谓不宜尔\footnote{山公:指山涛。为竹林七贤之一。车骑:此指谢玄,谢安侄,死后追赠车骑将军。}。因与诸尚书言孙、吴用兵本意\footnote{尚书:尚书省列曹长官。孙吴:孙武和吴起,古代兵法家。本意:主旨。},遂究论。举坐无不咨嗟\footnote{举坐:全座,满座。指所有人。咨嗟:赞叹。},皆曰:“山少傅乃天下名言\footnote{山少傅:即山涛。晋武帝咸宁初,山涛为太子少傅,故称。}。”{\fzxk\zihao{6}\textcolor{red}{\CJKunderwave{中(史)记}曰:“孙武,齐人;吴起,卫人。并善兵法。”\CJKunderwave{竹林士(七)贤论}曰:“咸宁中,吴既平,上将为桃林、华山之事,息弭役兵,示天下以大安。于是州郡悉去兵,大郡置武吏百人,小郡五十人。时京师犹讲武,山涛因论孙、吴用兵本意。涛为人常简默,盖以为国者不可以忘战,故及之。”\CJKunderwave{名士传}曰:“涛居魏、晋之间,无所标名。尝与尚书卢钦言及用兵本意,武帝闻之,曰:‘山少傅名言也。’”}} 后诸王骄汰\footnote{诸王骄汰:指历史上有名的八王之乱。},轻遘祸难。于是寇盗处处蚁合\footnote{蚁和:如蚁之聚合,形容数量多。},郡国多以无备,不能制服\footnote{郡国:指天下州郡和王国。此泛指地方政府。},遂渐炽盛。皆如公言。时人以谓“山涛不学孙、吴,而暗与之理会\footnote{暗:暗中。理会:见解一致。}”。王夷甫亦叹云\footnote{王夷甫:王衍字夷甫。王夷甫:王衍(256—311)字夷甫,见刘孝标注。“以清虚通理称”,为当时清谈名家,“妙悟若神”,“妙善玄言,唯谈\CJKunderwave{老}、\CJKunderwave{庄}为事”。为政多谋略,不以经国为念,而善思自全之计,然终为石勒所害。(见\CJKunderwave{晋书}本传)注。}:“公暗与道合。”{\fzxk\zihao{6}\textcolor{red}{\CJKunderwave{竹林七贤论}曰:“永宁之后,诸王构祸,狡虏欻起,皆如涛言。”\CJKunderwave{名士传}曰:“王夷甫推叹涛‘晻晻为与道合,其深不可测’。皆此类也。”}}

{\cangkai\zihao{5}【评】魏晋玄学,见于日常言谈文章,促进了时代理论思维向精深化、缜密化发展;同时对社会生活各方面,如政治、军事、文学等,均产生至深的影响。山涛力谏晋武帝不宜偃武修文,既是出于鉴古通今的史家意识,更主要的是受林下玄风的耳濡目染而作的见微知著的深刻哲学考察。孙、吴用兵本意,虽显于名相的万有,其背后本质则是大道之行。尚武与修文,本应配合运用,未可以一概而论。晋武帝鉴于汉末以来以迄西晋的动荡不已、战争频仍,提出休养生息的政策,其出发点是好的。但因未能把握玄学思维福祸倚伏的辩证之理,而误入一偏之歧途,遂导致臣强主弱及后来的八王之乱,终致国家丧亡。}

\lettrine{7.5} 王夷甫父■\footnote{王夷甫:王衍,王夷甫:王衍(256—311)字夷甫,见刘孝标注。“以清虚通理称”,为当时清谈名家,“妙悟若神”,“妙善玄言,唯谈\CJKunderwave{老}、\CJKunderwave{庄}为事”。为政多谋略,不以经国为念,而善思自全之计,然终为石勒所害。(见\CJKunderwave{晋书}本传)注。■:王■,王衍父。},为平北将军\footnote{平北将军:将军之号。汉末所设,魏晋沿置。},有公事,使行人论\footnote{公事:公案诉讼。行人:官名,掌朝觐聘问。},不得\footnote{不得:交涉理论没有结果。}。时夷甫在京师,命驾见仆射羊祜、尚书山涛\footnote{羊祜:羊祜字叔子,为尚书仆射、卫将军,出镇荆州,有政绩,人称羊公。尚书山涛:尚书省列曹长官。}。夷甫时总角\footnote{总角:指未成年时。古代男女未成年前束发为两结,形状如角,故称总角。},姿才秀异,叙致既快\footnote{叙致:陈述表达。},事加有理\footnote{事加有理:诉讼之事的理由又充分。加,又。}。涛甚奇之。既退,看之不辍,乃叹曰:“生儿不当如王夷甫邪?”羊祜曰:“乱天下者,必此子也。”{\fzxk\zihao{6}\textcolor{red}{\CJKunderwave{晋阳秋}曰:“夷甫父又(■),有简书,将免官。夷甫年十七,见所继从舅羊祜,申陈事状,辞甚俊伟。祜不然之,夷甫拂衣而起。祜顾谓宾客曰:‘此人必将以盛名处当世大位,然败俗伤化者,必此人也。’”\CJKunderwave{汉晋春秋}曰:“初,羊祜以军法欲斩王戎,夷甫又忿祜言其必败,不相贵重。天下为之语曰:‘二王当朝,世人莫敢称羊公之有德。’”}}

{\cangkai\zihao{5}【评】\CJKunderwave{晋书}衍本传亦载山涛有“误天下苍生者,未必非此人也”之语。大意略同。羊祜、山涛是久经历练的政治家,识人无算。王衍出身琅邪王氏高门,其姿才秀异,聪敏绝伦,动静举止间神态、气质,已然传达出其自恃聪明、任意雌黄的内心真实,逃不出二公明察秋毫的火眼金睛。历史证明,王衍是首鼠两端、毫无气节的士林败类。二公之识鉴,由形入神,可谓高明。然羊祜所称“乱天下”、山涛“误苍生”云云,均着眼于王衍的清谈为误国致乱之阶,就未免把个人在历史上的影响看得过重,也连带抹杀了魏晋清谈的积极意义。记载恐有小说家夸饰的成分。}

\lettrine{7.6} 潘阳仲见王敦小时\footnote{潘阳仲:潘滔(?—311),字阳仲,西晋荥阳(在今河南)人。潘岳之侄。王敦:王敦:字处仲,晋琅邪临沂(今属山东)人,王导堂兄。妻为晋武帝女襄城公主,拜驸马都尉。晋室东迁,与王导一起辅佐元帝,任要职,握重兵,镇守扬州、荆州等重镇。公元322 年起兵谋反,入京都建康。王含:见刘孝标注。光禄勋:官名,九卿之一,领管光禄、大中、中散、谏议等大夫及羽林郎、五官、虎贲、左右等中郎将注。},谓曰:“君蜂目已露\footnote{蜂目:像蜂那样的眼睛。},但豺声未振耳\footnote{豺声:像豺那样的声音。古人认为,蜂目豺声之人是凶狠残忍的人。\CJKunderwave{左传·文公元年}:“蜂目而豺声,忍人也。”}。必能食人,亦当为人所食。”{\fzxk\zihao{6}\textcolor{red}{\CJKunderwave{晋阳秋}曰:“潘滔字阳仲,荣(荥)阳人,太常尼从子也。有文学才识。永嘉未(末),为河南尹,遇害。”\CJKunderwave{汉晋春秋}曰:“初,王夷甫言东海王越,转王敦为扬州。潘滔初为太傅长史,言于大(太)傅曰:‘王处仲蜂目已露,豺声未发。今树之江外,肆其豪强之心,是贼之也。’”\CJKunderwave{晋阳秋}曰:“敦为太子舍人,与滔同僚,故有此言。”习、孙二说,便小迁异。\CJKunderwave{春秋传}曰:“楚令尹子曰(上)谓世子商臣‘蜂目而豺声,忍人也。’”}}

{\cangkai\zihao{5}【评】唐李贺\CJKunderwave{梦天}诗曰:“遥望齐州九点烟,一泓海水杯中泻。”从大九州的视角俯察人世间小九州的沧海桑田,自会比囿于其中看得透彻。王敦、桓温虽为一代枭雄,但均有远大抱负,是独具个性魅力的人杰,代表世家门阀向司马皇权发起冲击。其勃勃野心,震撼一代。王敦酒后“烈士暮年,壮心不已”之咏,桓温北伐“木犹如此,人何以堪”之叹,各具风神。惜其生不逢时,否则,其声名功业何让汉高、魏武!若固执地站在正统史家的立场上看问题,王敦、桓温,一定是口诛笔伐的对象,任由他人丑化涂抹,泼在他们身上的脏水,恐怕是永远洗不清的。蜂目、豺声之评,恐是事后附会之言,有小说家夸饰成分。正如前贤指出的,揆诸人之常情,一般不会对人当面言此。敦本传载,“眉目疏朗”,“少有奇人之目,尚武帝女襄城公主”,与“蜂目”之评相去甚远。如敦小时候就凶相毕露,武帝又怎能草率地嫁女于敦呢?}

\lettrine{7.7} 石勒不知书\footnote{石勒(274—333):羯人,十六国时后赵国主。他好文史,在军族常令儒生读史,每以其意论古帝王善恶。不知书:不识字。},{\fzxk\zihao{6}\textcolor{red}{\CJKunderwave{石勒传}曰:“勒字世龙,上党武乡人,匈奴之苗裔也。椎(雄)勇好骑射。晋元康中,流宕山东,与平原茌平人师欢家庸,耳恒闻鼓角鞞铎之音,勒私异之。初,勒乡里原上地中生石,日长,类铁骑之象;国中生人参,葩叶甚盛。于时父老相者皆云:‘此胡体貌奇异,有不可知。’劝邑人厚遇之,人多哂而不信。永嘉初,豪桀(杰)并起,与胡王阳等十八骑诣汲桑为左前督。桑败,其(共)推勒为主,攻下州县,都于襄国。后僭正号,死,谥明皇帝。”}} 使人读\CJKunderwave{汉书}\footnote{\CJKunderwave{汉书}:东汉班固撰。记述西汉一代历史,是我国第一部纪传体断代史。}。闻郦食其劝立六国后\footnote{郦食其:秦末儒生,陈留高阳(今河南)人。后为刘邦谋士,项羽、刘邦荥阳之战,曾劝刘邦立六国后裔以削弱楚。见\CJKunderwave{史记·郦生陆贾列传}。},刻印将授之,大惊曰:“此法当失,云何得遂有天下\footnote{云何:说什么。}!”至留侯谏\footnote{留侯:指汉代张良。良为汉高祖刘邦谋士,楚汉相争,良辅佐刘邦得天下,因功封留侯。见\CJKunderwave{史记·留侯世家}。谏:劝阻,此指良向刘邦陈述不宜立六国后高之事。},乃曰:“赖有此耳!”{\fzxk\zihao{6}\textcolor{red}{邓粲\CJKunderwave{晋纪}曰:“勒不知书,目不识字,每于军中令人诵读,听之,皆解其意。”\CJKunderwave{汉书}曰:“项羽急围汉王于荥阳,汉王与郦食其谋挠楚权。食其劝立六国后,王令趣刻印。张良入谏,以为不可。辍食吐哺,骂郦生曰:‘竖儒,几败乃公事!’趣令销印。”}}

{\cangkai\zihao{5}【评】汉代章句之学多拘泥而不知变通,说五字之文,至于二三万言;魏晋玄学兴起以后,读书治学尚会通玄远,发挥义理。史载阮瞻读书“不甚研求,而默识其要”(\CJKunderwave{晋书·阮籍传}附阮瞻传);支遁读书“善标宗会,而章句或有所遗”(\CJKunderwave{高僧传});乃至陶渊明读书之“不求甚解”,都是受魏晋玄风的熏染,而持通脱的读书观。天资颖悟的读者,往往能透过外在语言符号,直接以心灵去感悟、把握言说的深层含蕴。石勒会通的思维方式,与此种时代潮流暗合,他虽不知书,但能从广阔的社会生活这本大书中,汲取有益的精神滋养,融会贯通,变成治国治军的智慧。与阮瞻等人相比,更是一种超越文字、不立名相、以心传心式的思考和领悟。从民间成长起来的军事家,读透社会这所大学、生活这本大书,对于那些接受正规教育,却纸上谈兵的教条主义者而言,真是最幽默的讽刺!}

\lettrine{7.8} 卫■年五岁\footnote{卫■:字叔宝,小字虎,晋河东安邑(今山西)人。美姿容,好言玄理。官拜太子洗马。即卫筁,官拜太子洗马,故称。惨悴:忧伤憔悴的样子。左右:身边侍从人员注。},神衿可爱\footnote{神衿:神情气度、仪容丰采。}。祖太保曰\footnote{祖太保:卫瓘,字伯玉,西晋初河东安邑(今山西运城东北)人。卫■之祖父。}:“此儿有异,顾吾老\footnote{顾:只是,不过。},不见其大耳!”{\fzxk\zihao{6}\textcolor{red}{\CJKunderwave{晋诸公赞}曰:“瓘字伯玉,河东安邑人。少以明识清允称,傅嘏极贵重之,谓之宁武子。仕至太保,为楚王玮所害。”\CJKunderwave{■别传}曰:“■有虚令之秀,清胜之气,在群伍之中,有异人之望。祖太保见■五岁,曰:‘此儿神爽聪令,与众大异,恐吾年老,不及见尔!’”}}

{\cangkai\zihao{5}【评】卫■为两晋之际光鲜绝伦的人物。论形貌,堪称数一数二的美男,具备魏晋人物审美理想的阴柔之美。总角乘羊车入市,见者皆以为玉人,观之者倾都。人们用世间至美的珠玉拟其容貌。有晋一朝,堪与卫■分此殊荣的,大概只有潘安等少数人。时人对卫■的钟爱与仰慕,表征了晋人渴求、重视人物自然之美的时代风气,有似今日人们对酷男靓女“中性美”的疯狂追逐。当然二者本质大相径庭,但就疯狂度而言,还是可相比拼的。卫■不是“绣花枕头”,而是有深厚文化内涵的思想家。若仅论容貌,每一时代的帅哥美眉都何止千万,但大帅哥卫■却如天上的星辰,永远闪耀着深邃而迷人的光辉。这就说明,魏晋时代,并不是一个肤浅的时代,贵族士大夫对人物之美的欣赏,绝非片面追求天生长相,而是有更挑剔的要求。这其中包含了气质风度、言谈风采等。论玄言风采,卫■被称为“中兴名士第一”。王敦曾叹赏曰:“昔王辅嗣吐金声于中朝,此子复玉振于江表,微言之绪,绝而复续。不意永嘉之末,复闻正始之音,何平叔若在,当复绝倒。”(\CJKunderwave{晋书}本传)可见,祖父卫瓘之激赏,并非带有私人感情色彩的谬赞,而是有预见力的识鉴。}

\lettrine{7.9} 刘越石云\footnote{刘越石:刘琨字越石。西晋亡后,率军在河北抗击石勒、刘曜,志复中原。}:“华彦夏识能不足\footnote{华彦夏:华轶字。西晋末平原(在今山东)人,华歆曾孙。识能:识鉴能力。},强果有馀\footnote{强果:坚强果敢。}。”{\fzxk\zihao{6}\textcolor{red}{虞预\CJKunderwave{晋书}曰:“华轶字彦夏,平原人,魏太尉歆曾孙也。累迁江州刺史,倾心下士,甚得士欢心。以不从元皇命见诛。”\CJKunderwave{汉晋春秋}曰:“刘琨知轶必败,谓其自取之也。”}}

{\cangkai\zihao{5}【评】华轶于西晋末年天子孤危、四方瓦解之时,有匡天下之志,每遣使入洛,不失臣节。从魏晋之际士大夫只知有家、不知有国的大背景来看,华轶确实是一位有社会责任感的正直之士。然而,正如刘琨所评,其强果有馀而识能不足。永嘉之乱中,晋怀帝为匈奴刘渊所虏,群臣共推司马睿为盟主,实际上是代行皇帝权力。郡县官员劝华轶归顺司马睿,轶因未见京洛诏书,终不从命,惹恼了司马睿,派遣王敦讨伐,轶兵败被杀。封建社会的臣子,以忠于一家一姓的气节自励,华轶则甚至到了只忠于某个皇帝的程度。历史的发展有时候会出现惊人的相似,翻开一部古代史,改朝换代或王室更迭的情况可谓屡见不鲜,而充满刀光剑影的流血夺权,往往最能考验一位臣子的操守。唐玄宗天宝末年发生安史之乱,玄宗幸蜀,肃宗李亨未经父皇诏命在甘肃即位。杜甫“麻鞋见天子,衣袖露两肘”,结果“涕泪授拾遗”,高唱“流离主恩厚”!(杜甫\CJKunderwave{述怀})杜甫此举,在后世竟成为忠君的美谈,华轶却因过于固执,而身首异处。他如能有杜甫一点点灵活的脑筋,又何必死得不明不白呢?与朝廷衮衮诸公见机而动相比,华轶失之于愚,刘琨之评,中其肯綮。}

\lettrine{7.10} 张季鹰辟齐王东曹掾\footnote{张季鹰:张翰,字季鹰,西晋吴郡吴县(今江苏苏州)人。有清才,善属文,为人放达不拘,时号“江东步兵”。辟:征召。齐王:司马冏,(?—302),西晋皇族,字景治。齐王司马攸子,嗣封齐王。后为长沙王司马■所杀。东曹掾:东署的属官。},在洛,见秋风起,因思吴中菰菜羹、鲈鱼脍\footnote{吴中:指吴郡(今江苏)地区。菰菜羹:一说为“莼菜羹”。菰,茭白。鲈鱼脍:吴中名菜。脍,细切的鱼肉。后世以“莼鲈之思”指思乡之情。}。曰:“人生贵得适意尔,何能羁宦数千里以要名爵\footnote{羁宦:在异乡做官。要:求取。}?”遂命驾便归。俄而齐王败,时人皆谓为见机\footnote{见机:洞察事情变化的细微迹象。}。{\fzxk\zihao{6}\textcolor{red}{\CJKunderwave{文士传}曰:“张翰字季鹰。父俨,吴大鸿胪。有清才美望,博学善属文,造次立成,辞义清新。大司马齐王冏辟为东曹掾。翰谓同郡顾荣曰:‘天下纷纷未已,夫有四海之名者,求退良难。吾本山林间人,无望于时久矣。子善以明防前,以智虑后。’荣捉其手,怆然曰:‘吾亦与子采南山蕨,饮三江水尔。’翰以疾归,府以辄去除吏名。性至孝,遭母艰,哀毁过礼。自以年宿,不营当世,以疾终于家。”}}

{\cangkai\zihao{5}【评】西晋八王之乱,齐王冏起兵杀赵王伦,掌握朝政大权,张翰为其掾属。在齐王冏权力如日中天之时,张翰不愿跟着蹚浑水,毫无捞稻草、分杯羹之念,而是急流勇退、命驾便归,是一位深晓否泰、剥复之理的清醒士人。他见秋风起而思故乡风味,追求感官适意的背后,昭示了晋人尊重个体自适、珍惜短暂光阴的生命意识,其本质是受玄家思维的影响,是对儒家生命观的有益补充,有着积极的时代意义。南宋词人辛弃疾\CJKunderwave{水龙吟}中有“休说鲈鱼堪脍,尽西风,季鹰归未”三句,可见张翰故事的潇洒风流,在后世士大夫中有深远影响。辛弃疾力主抗金而沉沦下僚,壮志未酬,与张翰人生心态不同,有其时代原因,又另当别论。凌濛初评曰:“羹脍故可思,然亦见败机耳。”分析中肯。“见败机”云者,道出了张翰的政治远见与人生智慧。}

\lettrine{7.11} 诸葛道明初过江左\footnote{诸葛道明:诸葛恢,字明道。初过江左:谓刚从北方渡江到江南。江左:江东,长江下游以东地区。指东晋辖区。古人叙地理以东为左,以西为右。},自名道明\footnote{自名道明:诸葛道明和荀道明(名闿)、蔡道明(名谟)三人有“中兴三明”之称。},名亚王、庾之下\footnote{亚:次居第二位。王、庾:王导、庾亮。}。{\fzxk\zihao{6}\textcolor{red}{\CJKunderwave{中兴书}曰:“恢避难过江,与颍川荀道明、陈留蔡道明俱有名誉。号曰‘中兴三明’,时人为之语曰:‘京都三明各有名,蔡氏儒雅荀、葛清。’”}} 先为临沂令\footnote{临沂:县名。旧治在今山东费县东。},丞相谓曰\footnote{丞相:王导。}:“明府当为黑头公\footnote{明府:汉魏以来对太守、州牧皆称明府。黑头公:指年轻发黑而位登三公。}。”{\fzxk\zihao{6}\textcolor{red}{\CJKunderwave{语林}曰:“丞相拜司空,诸葛道明在公坐,指冠冕曰:‘君当复著此。’”}}

{\cangkai\zihao{5}【评】诸葛恢祖诞为魏司空,父靓为吴大司马,诸葛氏在汉、魏时期就建立了赫赫功勋,位至公卿,门第高华。诸葛恢一生恪尽职守,仕途顺利,为东晋元、明二帝赏识,在会稽内史任上,因政绩第一,皇帝下诏嘉奖。历任尚书右仆射、中书令、侍中等职,参与国家最高权力机构。王导以黑头公相期许,确实有知人之明。过江以后,恢名亚王、庾,心有不甘,老牌贵族心态,远非几十年的人世风雨所能冲刷尽净。王导与诸葛恢争族姓,王导曰:“人言王、葛,不言葛、王”,这是历史运道的无情事实,王导言语之间未免有得意之色。恢曰:“不言马、驴,而言驴、马,岂驴胜马邪!”反唇相讥,令人称羡其智慧和机锋,但也仅以语言机巧取胜而已,老贵族的花落水流已是无可挽回。}

\lettrine{7.12} 王平子素不知眉子\footnote{王平子:王澄,王衍之弟,乐广(?—304):字彦辅,南阳淯阳(今河南南阳东南)人。少孤贫,寒素为业,与物无竞。其清谈析理,与王衍并称,卫瓘以为有正始遗风。官至尚书令,八王乱中,以故忧卒注。知:赏识。眉子:王玄(?—313?),西晋琅邪临沂(今属山东)人,字眉子。王衍子。},曰:“志大无量,终当死坞壁间\footnote{志大无量:无,袁本作“其”,可备一说。坞壁:坞堡壁垒。一种军事防御性的小城堡。东汉末,各地坞堡林立,有的发展为武装割据势力。}。”{\fzxk\zihao{6}\textcolor{red}{\CJKunderwave{晋诸公赞}曰:“王玄字眉子,夷甫子也。东海王越辟为掾,后行陈留太守,大行威罚,为坞人所害。”}}

{\cangkai\zihao{5}【评】王澄为王衍弟,亦有重名于世,时人许以人伦之鉴。有经澄所题目者,王衍不复有言,云:“已经平子矣”,对王澄的鉴赏品味毫不怀疑,用今天的话说,凡是经过王澄检验、考核的,都属于“免检产品”。这个荣誉可是来之不易!不过,当我们知道王衍与王澄的特殊关系以后,就不必对王衍的话太过认真,因为他们是兄弟,难免自家人喝彩的吹嘘成分,说到底,是为家族利益鼓与呼。不过,这次王澄真没看走眼,出口就应验。他对衍子玄,也就是自己的亲侄子,从来看不上眼,给予评价是“志大无量,终当死坞壁间”,王玄在任梁国内史时,为政苛急,大行威罚,甚失人心,终为仇家所杀。可怜衍、澄、玄这兄弟、叔侄三人,都死得不得其所,空有识人之明,而缺自知之明,悲哉!}

\lettrine{7.13} 王大将军始下\footnote{王大将军:王敦,王敦:字处仲,晋琅邪临沂(今属山东)人,王导堂兄。妻为晋武帝女襄城公主,拜驸马都尉。晋室东迁,与王导一起辅佐元帝,任要职,握重兵,镇守扬州、荆州等重镇。公元322 年起兵谋反,入京都建康。王含:见刘孝标注。光禄勋:官名,九卿之一,领管光禄、大中、中散、谏议等大夫及羽林郎、五官、虎贲、左右等中郎将注。始下:指王敦于晋元帝永昌元年(322)以诛刘隗为名,自武昌举兵,沿江而下,进军石头城。下,指王敦顺江而下攻打建康。},杨朗苦谏不从\footnote{杨朗:字世彦,东晋弘农华阴(今属陕西)人,有器识,为王敦、谢安所赏识,历南郡太守,官至雍州刺史。苦谏:极力劝阻。},遂为王致力\footnote{致力:效力。}。乘中鸣云露平(车)径前\footnote{中鸣云露平:据袁本,“平”作“车”,是。中鸣云露车,古代打仗时用的一种指挥车。车上有望楼,并置金鼓,以指挥进退。径前:径直前来相见。},曰:“听下官鼓音,一进而捷。”王先把其手曰:“事克,当相用为荆州\footnote{荆州:指荆州刺史。}。”既而忘之,以为南郡\footnote{南郡:郡名。西晋治所在江陵(今湖北)。}。{\fzxk\zihao{6}\textcolor{red}{\CJKunderwave{晋百官名}曰:“朗字世彦,弘农人。”\CJKunderwave{杨氏谱}曰:“朗祖嚣,典军校尉。父冀州刺史。”王隐\CJKunderwave{晋书}曰:“朗有器识才量,善能当世。仕至雍州刺史。”}} 王败后,明帝收朗\footnote{明帝:指晋明帝司马绍。收:拘捕。},欲杀之。帝寻崩,得免。后兼三公\footnote{兼三公:“三公”下当有“曹”。\CJKunderwave{晋书·职官志}列曹尚书有三公曹,主典选。盖其以尚书摄职,而云兼。},署数十人为官属。此诸人当时并无名,后皆被知遇。于时称其知人。

{\cangkai\zihao{5}【评】杨朗的识鉴,包括两方面:一为人伦识鉴;一为对事理的识鉴,也就是料事如神的洞察力。故事中具体表现为:一、王敦谋反,杨朗苦谏,预知篡逆行为终将失败,尽了部下职守。主帅不听,也并不拼死力谏,这说明他能屈能伸、随缘任心,并非头脑发热而置生命于不顾的迂夫子;二、“一进而捷”云云,料定此役必胜,对交战双方形势有深入的洞察,有军事头脑;三、兼三公曹时,选官得人,知人善任。杨朗之识鉴,确实可圈可点。}

\lettrine{7.14} 周伯仁母\footnote{周伯仁:周■,字伯仁。},冬至举酒赐三子曰\footnote{冬至:二十四节气之一。古人把冬至看成节气的起点,有在这天宴饮的习尚。}:“吾本谓度江托足无所\footnote{谓:以为。托足:立足,谓容身。},尔家有相\footnote{有相:有吉祥之相,有福相。},尔等并罗列\footnote{罗列:排列。},吾复何忧!”周嵩起\footnote{周嵩:周■弟,性狷介。},长跪而泣曰\footnote{长跪:直身而跪。古人席地而坐,坐时两膝据地,以臀部着脚跟。跪则伸直腰股,以示庄重。}:“不如阿母言。伯仁为人,志大而才短,名重而识暗\footnote{暗:迟钝,不精明。},好乘人之弊\footnote{乘:利用,趁着。弊:危殆,衰败。},此非自全之道\footnote{自全:保全自己。}。嵩性狼抗\footnote{狼抗:狂妄自大。},亦不容于世。唯阿奴碌碌\footnote{阿奴:指周谟,周■、嵩之弟,性狷介。阿奴,此处为兄称弟的昵称。碌碌:随众附和,平庸无作为。},当在阿母目下耳。”{\fzxk\zihao{6}\textcolor{red}{邓粲\CJKunderwave{晋纪}曰:“阿奴,嵩之弟周谟也。”三周,并已见。}}

{\cangkai\zihao{5}【评】周家三子,周谟得享天年,■、嵩均死王敦刀下。周嵩之言,竟成谶语,似有先见之明。细思之,其言不尽确凿。周嵩刚直太过,近于褊狭,每以才气凌物。自评“性狼抗,不容于世”,倒也大体属实。诚如凌濛初之言:“自知不容于世,犹手批玄亮,火攻伯仁。”西谚有云“性格即命运”,周嵩之死,确由性格致祸。■则“性宽裕而友爱过人”、“以雅望获海内盛名”(\CJKunderwave{晋书}■本传),王敦构逆,■临危赴难,不屈而死。周嵩对其“志大才短”、“名重识暗”、“好乘人之弊”之酷评,属于片面之词。综观■之为人,似难以得出上述印象。\CJKunderwave{世说}所载嵩多次言行,足见其妒忌狭隘的性格缺陷。■、嵩兄弟都因不屈而为王敦所杀,可谓死得其所。但弟嵩与兄■相较,因性格、修养原因,不为世人所喜,影响到对其抗争价值的充分评价。另,周嵩之狼抗,亦表现在家人团聚宴饮之际,丝毫不顾及母亲的情绪及和谐的氛围,发此恶谶,连最基本的人情世故都不顾,令人生厌!}

\lettrine{7.15} 王大将军既亡\footnote{王大将军:指王敦。晋明帝时,王敦以诛奸臣为名,第二次起兵反,途中病死。人溃亡。},王应欲投世儒\footnote{王应(?—324)字安期,王含子。王敦无子,以应为嗣子。世儒:王彬(275—333),字世儒,王敦从弟,时为江州刺史。},世儒为江州\footnote{江州:指江州刺史。}。王含欲投王舒\footnote{王含:王敦兄,王敦:字处仲,晋琅邪临沂(今属山东)人,王导堂兄。妻为晋武帝女襄城公主,拜驸马都尉。晋室东迁,与王导一起辅佐元帝,任要职,握重兵,镇守扬州、荆州等重镇。公元322 年起兵谋反,入京都建康。王含:见刘孝标注。光禄勋:官名,九卿之一,领管光禄、大中、中散、谏议等大夫及羽林郎、五官、虎贲、左右等中郎将注。王舒(266?—333):字处明,王导从弟,时为荆州刺史。},舒为荆州\footnote{荆州:指荆州刺史。}。含语应曰:“大将军平素与江州云何,而汝欲归之?”应曰:“此乃所以宜往也。{\fzxk\zihao{6}\textcolor{red}{\CJKunderwave{晋阳秋}曰:“应字安期,含子也。敦无子,养为嗣,以为武卫将军,用为副贰,伏诛。”}} 江州当人强盛时,能抗同异\footnote{抗:抗论,直言不讳。同异:偏指于“异”,不同。},此非常人所行。及睹衰厄,必兴慜(愍)恻\footnote{愍恻:怜悯之心。}。{\fzxk\zihao{6}\textcolor{red}{\CJKunderwave{王彬别传}曰:“彬字世儒,琅邪人。祖览,父正,并有名德。彬爽气出侪类,有雅正之韵。与元帝姨兄弟,佐佑皇业,累迁侍中。从兄敦下石头,害周伯仁。彬与■素善,往哭其尸,甚恸。既而见敦,敦怪其有惨容而问之。答曰:‘向哭周伯仁,情不能已。’敦曰:‘伯仁自致刑戮,汝复何为者哉!’彬曰:‘伯仁清誉之士,有何罪?’因数敦曰:‘抗旌犯上,杀戮忠良。’音辞慷慨,与泪俱下。敦怒甚,丞相在坐,代为之惧。命彬曰:‘拜谢。’彬曰:‘有足疾。比来见天子,尚不欲拜。何跪之有?’敦曰:‘脚疾何如颈疾?’以亲故,不害之。累迁江州刺史、左仆射,赠卫将军。”}} 荆州守文\footnote{守文:遵守礼法。},岂能作意表行事\footnote{意表:意外。}!”含不从,遂共投舒,舒果沈含父子于江。{\fzxk\zihao{6}\textcolor{red}{\CJKunderwave{王舒传}曰:“舒字处明,琅邪人。祖览,知名。父会,御史。舒器业简素,有文武干。中宗用为比(北)中郎将、荆州刺史、尚书仆射,出为会稽太守。父名会,累表自陈。讨苏峻有功,封彭泽侯,赠车骑大将军。”}} 彬闻应当来,密具船以待之。竟不得来,深以为恨。{\fzxk\zihao{6}\textcolor{red}{含之投舒,舒遣军逆之,含父子赴水死。昔郦寄卖友见讥,况贩兄弟以求安,舒非人矣。}}

{\cangkai\zihao{5}【评】敦、含为亲兄弟,与彬、舒俱为从兄弟。故事运用两处对比手法,映衬人物性格。一是王彬、王舒人格境界对比。彬“闻应当来,密具船以待之”,可见其有情有义:又往哭周■,痛斥王敦,刀戟加颈而不变色,有侠义心肠,是正气凛然的大丈夫。王舒受王敦知遇,是敦部下。王含、王应父子来投,王舒害怕受牵连,竟遣军逆之,沉之于江。其自私、残酷,令人发指,是落井下石的小人。二、王含、王应父子识鉴对比。虽为父子,而眼光差距若天渊。王含目光短浅,头脑昏聩,害人害己;王应目光敏锐,头脑清醒,却受制其父。刘辰翁评曰:“英贤独见,为鉴后来,龟不自灵,可商可戒。”语涉沉痛,发人深思。}

\lettrine{7.16} 武昌孟嘉作庾太尉州从事\footnote{武昌:郡名,治所在武昌县(今湖北)。孟嘉:东晋江夏(今河南信阳东北)人。三国吴司空孟宗之曾孙,陶渊明外祖父。少有文才,以清操知名,性嗜酒,饮多而举止不乱,自谓得酒中真趣。庾太尉:指庾亮。州从事:指江州庐陵从事。},已知名。褚太傅有知人鉴\footnote{褚太傅:褚裒,褚公:对褚裒的敬称。褚裒(póu 抔)(303—349),晋康帝皇后之父,朝廷议以“不臣之礼”,力辞执政,而赴外镇。官征北大将军。曾率军三万北伐,败后上疏自贬,忧慨发愤而卒。见\CJKunderwave{晋书·外戚传}。鉴:照察的能力。},罢豫章还\footnote{罢豫章:免去豫章太守官职。},过武昌,问庾曰:“闻孟从事佳,今在此不\footnote{不:同“否”。}?”庾云:“试自求之。”褚眄睐良久\footnote{眄睐:目光左右流动着看。},指嘉曰:“此君小异\footnote{小异:稍有不同。},得无是乎?”庾大笑曰:“然。”于时既叹褚之默识\footnote{默识:用思深秘、暗中识人的能力。},又欣嘉之见赏\footnote{见赏:被赏识。}。{\fzxk\zihao{6}\textcolor{red}{\CJKunderwave{嘉别传}曰:“嘉字万年,江夏■人。曾祖父宗,吴司空。祖父揖,晋庐陵太守。宗葬武昌阳新县,子孙家焉。嘉少以清操知名。太尉庾亮领江州,辟嘉部庐陵从事。下都还,亮引问风俗得失,对曰:‘待还,当问从事吏。’亮举麈尾,掩口而笑,语弟翼曰:‘孟嘉故是盛德人。’转劝学从事。太傅褚裒有器识,亮正旦大会,裒问亮:‘闻江州有孟嘉,何在?’亮曰:‘在坐,卿但自觅。’裒历观久之,指嘉曰:‘将无是乎?’亮欣然而笑,喜裒得嘉,奇嘉为裒所得,乃益器之。后为征西桓温参军。九月九日,温游龙山,参寮毕集。时佐史并箸戎服,风吹嘉帽堕落,温戒左右勿言,以观其举止。嘉初不觉,良久如厕。命取还之,令孙盛作文嘲之,成,箸嘉坐。嘉还,即答,四坐嗟叹。嘉善酣畅,愈多不乱。温问:‘酒有何好,而卿嗜之?’嘉曰:‘明公未得酒中趣尔。’又问:‘听伎,丝不如竹,竹不如肉,何也?’答曰:‘渐近自然。’转从事中郎,迁长史。年五十三而卒。”}}

{\cangkai\zihao{5}【评】孟嘉一代名士,“龙山落帽”、“渐近自然”之逸闻趣事高情千古,屡为后人称道。嘉乃田园诗宗陶渊明外祖父,大概因遗传基因所及与外家风尚流播,陶渊明亦濡染外祖酣饮不乱、崇尚自然的气质风度,并最终回归田园,成为将现实生活艺术化的文学大师。褚太傅于众座中眄睐良久,终于认出了孟嘉,可见名士间气息投合,自有会心得意处那一点灵犀,和高山流水、拈花微笑般的共鸣。总之,亮之识嘉,是超越门第和身份地位等世俗因素的一种精神契合。凌濛初曰:“既是异人,复逢善鉴,安得不识。每阅此等,令人愈急知己。”主客双方非凡意趣,缺一不可。此外,太尉庾亮也是有深情雅韵之人,不然,何以会在严肃的办公场所成就一段佳话?}

\lettrine{7.17} 戴安道年十馀岁\footnote{戴安道:戴逵。},在瓦官寺画\footnote{瓦官寺:东晋佛寺名。在都城建康城西南隅。}。长史见之\footnote{长史:王濛。},曰:“此童非徒能画,{\fzxk\zihao{6}\textcolor{red}{\CJKunderwave{续晋阳秋}曰:“逵善图画,穷巧丹青也。”}} 亦终当致名。恨吾老,不见其盛时耳!”

{\cangkai\zihao{5}【评】戴安道是全方位发展的艺术天才,史载其少博学,好谈论,善属文,能鼓琴,工书画,各艺术门类无不通晓。王濛识戴安道于总角,缘于安道天资颖发,不同常童;亦由于濛和畅通脱,不拘一格。濛期以“非徒能画,亦终当致名”。揣摩其意,谓安道必当运势通达,有立功之名。不意安道忘情丘壑,栖心自然,朝廷三征而三不至,是绝意宦情的真名士。王濛地下有知,当作何想?中国古代为官本位社会,注重立功不朽,扬名立万。从事艺术创作,始终被看作是雕虫小技,正史之中屈居边角。不过安道成为一代大艺术家,为王氏子弟钦羡,也算是另一种被承认的方式吧!}

\lettrine{7.18} 王仲祖、谢仁祖、刘真长俱至丹阳墓所省殷扬州\footnote{王仲祖:王濛。谢仁祖:谢尚,谢豫章:谢鲲,曾作豫章太守。刘孝标注“鲲子别见”,“子”字衍。将:携,谓携之送客。自:已经。参:参与、进入。上流:上等、上品注。刘真长:刘惔,字真长,曾任丹阳尹,故称。谢安妻兄,尚明帝女庐陵公主。会稽王司马昱为相,与王濛并为其座上清谈之客。性简贵自重,与王羲之友善。卒年三十六。丹阳:郡名。故城在今江苏南京江宁县东。墓所:墓地。省:访问。殷扬州:殷浩,字渊源,曾为扬州刺史,故称。(?—356):见刘孝标注。浩善谈玄,负盛名,简文执政时惧桓温势盛,引浩为建武将军、扬州刺史,以对抗桓温。后因北征许洛败绩,为桓温所弹,废为庶人。},绝有确然之志\footnote{绝有确然之志:绝,袁本作“殊”,可备一说。绝,甚、颇。确然之志,坚定不移的栖隐之志。语出\CJKunderwave{周易·乾·文言}:“不易乎世,不成乎名;遁世无闷,不见是而无闷;乐则行之,忧则违之,确乎其不可拔,潜龙也。”}。{\fzxk\zihao{6}\textcolor{red}{\CJKunderwave{中兴书}曰:“浩桓(棲)迟积年,累聘不至。”}} 既反\footnote{反:同“返”。},王、谢相谓曰:“渊源不起,当如苍生何\footnote{如苍生何:把百姓怎么样呢?苍生,百姓,众生。}?”深为忧叹。刘曰:“卿诸人真忧渊源不起邪?”

{\cangkai\zihao{5}【评】殷浩少与桓温齐名,为一时谈论者所宗,至有王、谢子弟“渊源不起,当如苍生何”之叹,将其与谢安等量齐观。面对殷浩貌似坚定不移的栖隐之志,众人被蒙在鼓里,唯有刘惔一针见血地指出殷浩膺情魏阙,却故作清高的内心真实,确如凌濛初所评,“真长口角无处不可畏”。殷浩后历任建武将军、扬州刺史及中军将军,以恢复中原为己任,上疏北伐失败被黜。桓温后来拟启用浩为尚书令,浩视此为东山再起的救命稻草,将复函检查了数十遍,急欲出仕的心态委曲毕现。但因心情过度紧张反而弄巧成拙,竟寄去了一只空信封。出山之事,只得作罢。殷浩晚年似有一定的心理障碍,即今天心理学上所说的“强迫症”。滑稽举止的背后引起人们对魏晋名士风度的反思,如此患得患失,恰与玄学“以无为本”,和“一生爱好是天然”的精神实质异辙,与戴安道那样“三征而三不至”的真名士相比,其差距不可以道里计。}

\lettrine{7.19} 小庾临终\footnote{小庾:庾翼。翼与兄庾亮并有名气,人称小庾。兄亮死后,为荆州刺史,镇武昌。},自表以子园客为代\footnote{自表:自己上表章。园客:庾爰之,字仲真,小字园客。庾翼第二子。为代:此谓作为代任荆州刺史的人选。}。{\fzxk\zihao{6}\textcolor{red}{园客,爰之小字也。\CJKunderwave{庾氏谱}曰:“爰之字仲真,翼弟(第)二子。”\CJKunderwave{中兴书}曰:“爰之有父翼风,桓温徙于豫章,年三十六而卒。”}} 朝廷虑其不从命\footnote{不从命:不听从命令。},未知所遣,乃共议用桓温\footnote{桓温:桓公北征:桓温曾有三次北征,刘盼遂\CJKunderwave{世说新语校笺}考订,此次当为太和四年(369)之征。时桓温已58岁。}。刘曰\footnote{刘:刘惔,字真长,曾任丹阳尹,故称。谢安妻兄,尚明帝女庐陵公主。会稽王司马昱为相,与王濛并为其座上清谈之客。性简贵自重,与王羲之友善。卒年三十六。}:“使伊去\footnote{伊:他。此指桓温。},必能克定西楚\footnote{克定:平定。西楚:东晋称荆州一带地区。这里古属楚国,位居京师建康之西,故称。},然恐不可复制\footnote{不可复制:不能再控制。}。”{\fzxk\zihao{6}\textcolor{red}{\CJKunderwave{陶侃别传}曰:“庾翼薨,表其子爰之代为荆州。何充曰:‘陶公重勋也,临终高让。丞相未薨,敬豫为四品将军,于今不改。亲则道恩,优游散骑,未有超卓若此之授。’乃以徐州刺史桓温为安西将军、荆州刺史。”宋明帝\CJKunderwave{文章志}曰:“翼表其子代任,朝廷畏惮之。议者欲以授桓温,时简文辅政,然之。刘惔曰:‘温去,必能定西楚,然恐不能复制。愿大王自镇上流,惔请为从军司马。’简文不许。温后果如惔所算也。”}}

{\cangkai\zihao{5}【评】刘惔、桓温二人,恰似欢喜冤家,时而反目成仇,时而契若知己,然终道不同不能相谋。惔既雅重温之才能,称其为孙仲谋、晋宣王之流亚;又憎其政治野心,每劝朝廷多存戒备。庾翼死后,朝廷议用桓温为荆州,刘惔喜忧参半。东晋偏安江左,沿江多为要地,上游荆州与下游扬州尤为重镇。故继王氏、陶侃之后,庾氏兄弟以外戚之重,统治荆州十年之久,与在中央的王导相抗衡。庾翼死前,欲以子爰之代己,意在维护“庾与马,共天下”的大权独揽的局势,可见荆州战略上的重要性。若用桓温为荆州刺史,其人能力足以制庾克蜀,但长江中上游大权旁落于温,等于放虎归山,后必伤人。刘惔之隐忧,乃是出于对桓温个性及当时政治形势的透彻了解,有识人、识事之鉴。桓温统治荆州近二十年,桓氏桓豁、桓冲、桓石民等相继治荆,形成“桓氏世莅西土”的局面,而桓温之子桓玄,卒以荆州为根据地继而篡晋。可见荆州刺史一职在当时之重要。}

\lettrine{7.20} 桓公将伐蜀\footnote{桓公:桓温。桓公北征:桓温曾有三次北征,刘盼遂\CJKunderwave{世说新语校笺}考订,此次当为太和四年(369)之征。时桓温已58岁。注。蜀:此指成汉,十六国之一。},在事诸贤咸以李势在蜀既久\footnote{在事诸贤:指掌持政事的官员们。李势(?—361):十六国成汉国君,字子仁。在蜀既久:晋惠帝元康八年(298),关中连年饥荒,巴氐首领李特率流民入蜀,至李势归降,前后六世,凡数十年。},承藉累叶\footnote{承藉:继承前代事业以为凭借。累叶:接连几世,数世。},且形据上流,三峡未易可克。唯刘尹云\footnote{刘尹:刘惔。}:“伊必能克蜀。观其蒲博\footnote{蒲博:即蒲戏。古代一种赌博游戏。},不必得则不为。”{\fzxk\zihao{6}\textcolor{red}{\CJKunderwave{华阳国志}曰:“李势字子仁,洛(略)阳临渭人,本已(巴)西宕渠■人也。其先李特,因晋乱据蜀。特子雄,称号成都。势祖骧,特弟也。骧生寿,寿篡位自立。势即寿子也。晋安西将军伐蜀,势归降,迁之扬州。自起至亡,六世,三(四)十七年。”\CJKunderwave{温别传}曰:“初,朝廷以蜀处险远,而温众寡少,悬军深入,甚以忧惧。而温直指成都,李势面缚。”\CJKunderwave{语林}曰:“刘尹见桓公每嬉戏必取胜,谓曰:‘卿乃尔好利,何不焦头?’及伐蜀,故有此言。”}}

{\cangkai\zihao{5}【评】唐代诗人李白\CJKunderwave{蜀道难}诗曰:“蜀道之难,难于上青天。”蜀道易守而难攻,关键在于其“一夫当关、万夫莫开”的险要地势。桓温将伐蜀,朝中诸公咸精于地理而暗于知人,忽略了人的重要性因素,故得出否定性结论。刘惔通过观察桓温平日赌博不为则已,为则必赢等生活细节、琐事,深知桓温绝非轻率、盲动、贪功速成之辈,而是持重隐忍、多谋善断之统帅,具备成大事的基本心理素质。刘惔称温为孙权、司马懿之流亚,孙与司马皆为不世出的政治家,可见刘惔对桓温伐蜀的结果早已成竹在胸。故事可见刘惔超人的预见才能。}

\lettrine{7.21} 谢公在东山畜妓\footnote{谢公:谢安。东山:山名。在浙江上虞市西南。谢安早年隐居于此。妓:古代贵族家中主要从事歌舞、音乐表演的侍女。},简文曰\footnote{简文:晋简文帝司马昱,晋简文:指晋简文帝司马昱(320—372),穆帝年幼即位,昱任抚军大将军总理政务。后来大将军桓温专擅朝政,先废海西公,后立司马昱为帝,第二年崩。}:“安石必出。既与人同乐,亦不得不与人同忧。”{\fzxk\zihao{6}\textcolor{red}{\CJKunderwave{宋明帝文章志}曰:“安纵心事外,疏略常节,每畜女妓,携持游肆也。”}}

{\cangkai\zihao{5}【评】晋简文帝常被世人视为痴人,其实不然。简文乃脱略俗礼、大智若愚。\CJKunderwave{世说}所载,多见其吉光片羽之言。简文通过谢安东山蓄妓,推测其必出仕做官,刘辰翁评曰:“此语别见发微者也。”甚是。\CJKunderwave{孟子·梁惠王章句下}载孟子见梁惠王,问王:“独乐乐,与人乐乐,孰乐?”王曰:“不若与人。”又问:“与少乐乐,与众乐乐,孰乐?”王曰:“不若与众。”孟子通过正反对比及心理分析,得出了“今王与百姓同乐,则王矣”的乐观结论。孟子依据的正是儒家向来看重的“推己及人”的同情心。同情,不仅包括狭义理解的怜悯之心的“同情”,更是现代心理学意义上所讲的“心理移情”,即人同此心、心同此理。同情心是人际交流的前提基础,如能将其层层放大,自然能扩展为与人同其忧乐。简文料谢安终将出山,良有以也。}

\lettrine{7.22} 郗超与谢玄不善\footnote{郗超:郗超:任桓温大司马,深得信任,立简文为帝后,迁中书侍郎,实代桓温监督朝廷而权重当时。在直:在宫中值班。谢玄:车骑:此指谢玄,谢安侄,死后追赠车骑将军。}。苻坚将问晋鼎\footnote{苻坚:前秦君主。问晋鼎:谋取东晋天下。问鼎,语出\CJKunderwave{左传·宣公三年}“楚子问鼎之大小轻重焉”。三代以九鼎为传国之宝,楚子问鼎,有觊觎周室之意。},既已狼噬梁、岐\footnote{狼噬:像狼一样吞食。梁、岐:梁,指今四川、陕西等一带;岐,指今陕西一带。},又虎视淮阴矣\footnote{淮阴:此泛指淮河以南一带。}。{\fzxk\zihao{6}\textcolor{red}{车频\CJKunderwave{秦书}曰:“苻坚字永固,武都氐人也。本姓蒲,祖父洪,诈称谶文,改曰苻。言己当王,应苻命也。坚初生,有赤光流其室。及诞,背赤色,隐起若篆文。幼有美度。石虎司隶徐正(统)名知人,坚六岁时,尝戏于路,正(统)见而异焉,问曰:‘苻郎,此官街,小儿行戏,不畏缚邪?’坚曰:‘吏缚有罪,不缚小儿。’正(统)谓左右曰:‘此儿有王霸相。’石氏乱,伯父健及父雄西入关。健梦天神使者朱衣冠,拜肩头为龙骧将军。肩头,坚小字也。健即拜为龙骧,以应神命。后健僭帝号,死,子生立,凶暴,群臣杀之而立坚。坚立十五年,遣长乐公丕攻没襄阳。十九年,大兴师伐晋,众号百万,水陆俱进,次于项城。自项城至长安,连旗千里,首尾不绝。及(乃)遣告晋曰:‘已为晋君于长安城中建广夏之室,今故大举渡江相迎,克日入宅也。’”}} 于时朝议遣玄北讨,人间颇有异同之论。唯超曰:“是必济事\footnote{济事:成事。}。吾昔尝与共在桓宣武府\footnote{桓宣武府:桓温幕府。},见使才皆尽,虽履屐之间\footnote{履屐:比喻小事。},亦得其任。以此推之,容必能立勋\footnote{容:当;或许。}。”元功既举\footnote{元功:大功。指淝水之战击退前秦大军之功。举:实行;实现。},时人咸叹超之先觉\footnote{先觉:有预见。},又重其不以爱憎匿善。{\fzxk\zihao{6}\textcolor{red}{\CJKunderwave{中兴书}曰:“于时氐贼强盛,朝议求文武良将可镇靖北方者。卫大将军安曰:‘唯兄子玄可任此事。’中书郎郗超闻而叹曰:‘安违众举亲,明也。玄必不负其举。’”}}

{\cangkai\zihao{5}【评】郗超与谢玄关系不睦,或许与父辈的恩怨有一定的关系。谢安入掌机要,位高权重,郗超父愔虽出高门,却屈居下僚。愔心怀愤愤,时有怨怼之言,导致二姓交恶。后郗超为桓温入幕宾,专生杀之威,桓温几次欲诛杀谢安,均为郗超之计,不能排除其公报私仇、一石二鸟之用心。而在谢玄北伐之际,郗超却力排众议,促成壮举,表现了“先国家之急而后私仇”的雅量,并有见微知著的人伦识鉴。郗超、谢玄,合则双美,分则两伤。于此可见,政治运筹重在形成合力与平衡,尽管政见不同、交情不睦,只要秉持公心,取“和而不同”的思想立场,政治的机器仍然可以高效运转。今日西方政坛有所谓的“府、院之争”、“象、驴之争”等等现象,聚讼不休,令人有雾里看花之感,但仍可以产生高效政治,何也?读郗超与谢玄故事,应该有所启迪。}

\lettrine{7.23} 韩康伯与谢玄亦无深好\footnote{韩康伯:韩伯,时任豫章太守,故称。曾为王弼\CJKunderwave{周易注}补注\CJKunderwave{易传}之系辞、说卦、杂卦等,是当时著名玄学名家。谢玄:见前则。},玄北征后\footnote{北征:指谢玄率师北上抗击前秦军。},巷议疑其不振\footnote{巷议:里巷间人们的议论。不振:谓不能奋力作战。}。康伯曰:“此人好名,必能战。”{\fzxk\zihao{6}\textcolor{red}{\CJKunderwave{续晋阳秋}曰:“玄识局贞正,有经国之才略。”}} 玄闻之,甚忿,常于众中厉色曰:“丈夫提千兵入死地\footnote{提:带领。千兵:成千上万的兵士。死地:危殆之境。此指前线战场。},以事君亲故发\footnote{君亲:偏义复词,指君王。},不得复云为名!”

{\cangkai\zihao{5}【评】中国传统文化崇尚立德、立功、立言的三不朽,青史留名是士子们的至高追求。韩康伯评谢玄“此人好名,必能战”之语,虽有忿忿不平之意,却也道出了人之常情。而谢玄“丈夫提千兵,入死地,以事君亲故发,不得复云为名”之反唇相讥,出语冠冕堂皇,却难免几分矫饰之情,不可谓全是内心的真实。剥离了个体价值和个人追求的所谓大公无私,是虚幻的,也是不值得提倡的。正如西方思想家曼德维尔在\CJKunderwave{蜜蜂的寓言}等著作中所提出的,德行起于荣辱感,而荣辱感起于自私。英国经济学家亚当·斯密,在\CJKunderwave{国富论}书中也有类似观点。这里的自私,并非伦理学上理解的见利忘义、损公肥私的“自私”,而是对个人应得利益的正当追求,有似“经济人”概念所揭示的人的趋利本性。西哲所言,一针见血,补充了我们认识中的某些片面性。同时,韩康伯在养病期间的妒忌心理(\CJKunderwave{方正}门记其讥讽诸谢“此复何异王莽时”,与此则可互相印证),也是人类妒忌本能的变相发泄。}

\lettrine{7.24} 褚期生少时\footnote{褚期生:褚爽,字茂弘,小字期生。褚裒孙。好老庄,淡荣利。},谢公甚知之\footnote{谢公:谢安。知:知遇,欣赏。},恒云:“褚期生若不佳者,仆不复相士\footnote{仆:自称的谦辞。相:品评。}!”{\fzxk\zihao{6}\textcolor{red}{期生,褚爽小字也。\CJKunderwave{续晋阳秋}曰:“爽字茂弘(弘茂),河南人,太傅裒之孙,秘书监韶(歆)之子。太傅谢安见其少时,叹曰:‘若期生不佳,我不复论士!’及长,果俊迈有风气。好老、庄之言,当世荣誉,弗之屑也。唯与殷仲堪善。累迁中书郎、义兴太守。女为恭帝皇后。”}}

{\cangkai\zihao{5}【评】褚爽为褚裒之孙,又是生于阶庭的芝兰玉树,故为谢安激赏。“褚期生若不佳者,仆不复相士”云云,特极言其卓落特异,为加强语气之辞。“相士”一词,可见当时人物识鉴蔚成风气,谢安颇以此自负。}

\lettrine{7.25} 郗超与傅瑗周旋\footnote{郗超:郗超:任桓温大司马,深得信任,立简文为帝后,迁中书侍郎,实代桓温监督朝廷而权重当时。在直:在宫中值班。傅瑗:字叔玉,东晋北地灵州(今宁夏灵武)人。以学业知名。周旋:交往。}。瑗见其二子\footnote{见:引见。},并总发\footnote{总发:即总角。古代儿童头发束在顶上,因指代童年。},超观之良久,谓瑗曰:“小者才名皆胜,然保卿家,终当在兄。”即傅亮兄弟也\footnote{傅亮兄弟:傅亮(?—426),南朝宋人,字季友。傅瑗子。宋武帝刘裕受禅,有佐命之功。因与徐羡之等杀少帝罪被处死。傅亮之兄傅迪,字长猷。位至五兵尚书。}。{\fzxk\zihao{6}\textcolor{red}{\CJKunderwave{傅氏谱}曰:“瑗字叔王(玉),北地灵州人。历护军长史,安城太守。”\CJKunderwave{宋书}曰:“迪字长猷,瑗长子也。位至五兵尚书,赠太常。”丘渊之\CJKunderwave{丈(文)章录}曰:“亮字季友,迪弟也,历尚书令,左光禄大夫。元嘉三年,以罪伏诛。”}}

{\cangkai\zihao{5}【评】\CJKunderwave{宋书}本传有“超令人解亮衣,使左右持去,初无吝色”三句,更觉其精彩。傅亮为人,善于揣摩上意,刘裕欲行篡逆,部下如呆瓜木头,“无人会得凭栏意”,幸好傅亮想主子之所想,言人之所未发,进京游说,有创基之功;又与徐羡之诸人谋诛王室刘义真,谋废少帝,果决、残忍之性格暴露无遗。可见其当初的望风承旨,绝非出于对刘裕的忠贞,而是见风使舵的政治投机。小人之志,翻云覆雨,在上位者可不慎欤!郗超品目,察言观行,并非无根之谈,从一个人的枝叶细节间,可见其真精神。超见谢玄履屐之间咸得其任,而知其必然克立大功,亦属此类。}

\lettrine{7.26} 王恭随父在会稽\footnote{王恭:(?—398):孝武帝后兄,安帝舅父。与殷仲堪、桓玄等,二次兴兵清君侧,兵败被诛。会稽:郡治在今浙江绍兴市。王恭之父王蕴,太元年间任会稽内史。},王大自都来拜墓\footnote{王大:王忱,(?—398):孝武帝后兄,安帝舅父。与殷仲堪、桓玄等,二次兴兵清君侧,兵败被诛。会稽:郡治在今浙江绍兴市。小字佛大,故称“阿大”。都:京都。指建康。拜墓:祭扫坟墓。},{\fzxk\zihao{6}\textcolor{red}{恭父蕴、王忱,并已见。}} 恭暂往墓下看之。二人素善,遂十馀日方还。父问恭:“何故多日?”对曰:“与阿大语,蝉连不得归\footnote{蝉连:连续不断。}。”因语之曰:“恐阿大非尔之友,终乖爱好\footnote{乖:背离。爱好:友情。王恭、王忱终因志向不同而相背离,成了仇家。}。”果如其言。{\fzxk\zihao{6}\textcolor{red}{忱与恭为王绪所间,终成怨隙。别见。}}

{\cangkai\zihao{5}【评】王恭、王忱同族齐名,俱为太原王氏,并流誉一时。名士间本同气相求,加上同族血缘,自然加深了亲近。诗云“总角之宴,言笑晏晏”(\CJKunderwave{诗经·氓}),大凡少年之好,热血来潮时耳鬓厮磨,一日三秋;热情退却时烟消云散,形同陌路。由于缺少挫折的考验和理性的驾驭,友谊的马车鲜能奔向终点。王忱拜墓,王恭与其流连十馀日,相濡以沫,如胶似漆。恭父王蕴冷眼旁观,预测二人友情终将乖戾,这是经历过人生风雨沐浴后“豪华落尽见真淳”的清醒认识。后二人在何澄坐上,因逼酒致怨,竟然刀兵相向,果然不出王蕴所料。}

\lettrine{7.27} 车胤父作南平郡功曹\footnote{南平郡:郡名。治所在今湖南安乡北。功曹:官名。郡中佐吏。},太守王胡之避司马无忌之难\footnote{司马无忌:(?—350),东晋宗室,字公寿,司马丞子,位至郡守。王敦杀丞,曾假手于王。无忌为父报仇,欲攻杀王胡之,故王胡之避之。},置郡于澧阴\footnote{澧:水名,源出湖南西北,至安乡(晋南平郡治)南注洞庭湖。阴,水之南。}。是时胤十馀岁,胡之每出,尝于篱中见而异焉。谓胤父曰:“此儿当致高名。”后游集,恒命之\footnote{恒:常,总是。命:召。}。胤长,又为桓宣武所知\footnote{桓宣武:桓温。},清通于多士之世,官至选曹尚书\footnote{选曹尚书:即吏部尚书,主官吏之选拔考校任免等。}。{\fzxk\zihao{6}\textcolor{red}{\CJKunderwave{续晋阳秋}曰:“胤字武子,南平人。父育,为郡主簿。太守王胡之有知人识裁,见,谓其父曰:‘此儿当成卿门户,宜资令学问。’胤就业恭勤,博览不倦。家贫,不常得油。夏月,则练囊盛数十萤火以继日焉。及长,风姿美劭,机悟敏率。桓温在荆州,取为从事,一岁至治中。胤既博学多闻,又善于激赏。当时每有盛坐,胤必同之。皆云‘无车公不乐’。太傅谢公游集之日,开筵以待之。累迁丹阳尹、护军将军、吏部尚书。”}}

{\cangkai\zihao{5}【评】“车胤囊萤”的故事,作为激励贫寒子弟克服困难、刻苦攻读的典范而家喻户晓。车胤家贫不常得油,夏月则以绢囊盛数十萤火以照书——虽无膏油以继晷,而恒兀兀以穷年。立志笃学的莘莘学子,必与游手好闲的纨绔子弟,在精神面貌、行为习惯等方面表现殊异,王胡之于篱笆中赏识车胤,窥一斑而见全豹,而有超常之评。车胤果不负所望,成为国之栋梁。热心现代自然科学技术的清康熙皇帝,曾亲自试验,以大囊盛萤数百,以照字画,竟不能辨(事载\CJKunderwave{康熙东华录})。囊萤之事,可能是后人以讹传讹,但车胤发愤读书,想必不虚也,不然,何以一贫寒子弟而为王胡之、桓温、谢安诸名士所赏识,甚而成为孝武帝讲\CJKunderwave{孝经}时的主讲人?}

\lettrine{7.28} 王忱死\footnote{王忱:(?—398):孝武帝后兄,安帝舅父。与殷仲堪、桓玄等,二次兴兵清君侧,兵败被诛。会稽:郡治在今浙江绍兴市。,官至荆州刺史、建武将军。东晋孝武帝太和十七年(392)十月,死于任上。},西镇未定\footnote{西镇未定:指荆州刺史之官职未任命。西镇:指荆州。},朝贵人人有望\footnote{朝贵:朝廷显贵。}。时殷仲堪在门下\footnote{殷仲堪:(?—399):善清谈,当时与韩康伯齐名。注。门下:即黄门,后称门下省,直属于皇帝的顾问咨询机关,参与朝政。},虽居机要\footnote{机要:掌管机密要事的地位、职务或部门。},资名轻小\footnote{资名:资历名望。},人情未以方岳相许\footnote{人情:人心,人们的意见。方岳:四方之岳。此指地方高级长官。许:称道,赞许。}。晋孝武欲拔亲近腹心\footnote{晋孝武:孝武帝司马曜。拔:提拔。},遂以殷为荆州\footnote{以殷为荆州:太元十七年十一月,以黄门郎殷仲堪为都督荆益宁三州诸军事、荆州刺史,代王忱之职。}。事定,诏未出。王珣问殷曰\footnote{王珣。}:“陕西何故未有处分\footnote{陕西:东晋时指荆州。东晋以扬州、荆州为长江下游和上游重镇,比照周公、召公分治之陕东、陕西,称荆州为陕西。处分:处置。此指朝廷任命官吏。}?”殷曰:“已有人。”王历问公卿,咸云:“非。”王自许才地\footnote{自许才地:以才能门第自许。地,通“第”。王珣出身琅邪王氏大族,此时任尚书左仆射,以才学文章深为孝武帝所倚仗。},必应在己,复问:“非我邪?”殷曰:“亦似非。”其夜,诏出用殷。王语所亲曰:“岂有黄门郎而受如此任!仲堪此举,乃是国之亡征。”{\fzxk\zihao{6}\textcolor{red}{\CJKunderwave{晋安帝纪}曰:“孝武深为晏驾后计,擢仲堪代王忱为荆州。仲堪虽有美誉,议者未以方岳相许也。既受腹心之任,居上流之重,议者谓其殆矣。终为桓玄所败。”}}

{\cangkai\zihao{5}【评】自晋室播迁、王居建业,则以荆、扬为京师根本之所寄。荆楚地处上游,控制胡虏,为国藩屏,拟周之分陕,故有陕西之号,而荆州皆以重臣坐镇。殷仲堪为孝武赏识,荆州刺史王忱死后,孝武任用亲信殷仲堪为荆州,以加强皇权的势力。仲堪之任,孝武下诏曰:“卿去有日,使人酸然。常谓永为廊庙之宝,而忽为荆楚之珍,良以慨恨!”(\CJKunderwave{晋书}仲堪本传)可见君臣情笃。殷仲堪时为黄门侍郎,官卑言轻,但因与皇帝情好而被越级提拔。王珣自计才能门第,属意荆州刺史职于己,而轻仲堪为黄门侍郎。虽一时情急,却也无意中道出了当时政治混乱、官吏选拔临时抱佛脚的情况。后果如珣所言,仲堪死于奸雄桓玄之手,国亦败亡。余嘉锡氏考证,王忱死后,桓玄知殷仲堪弱才,易于掌控,乃遣尼妙音游说孝武以仲堪为荆州,果然实现险恶用心。可备一说。}



[1] 

[2] 

[3] 

[4] 

[5] 

[6] 

[7] 

[8] 

[9] 

[10] 

[11] 

[12] 

[13] 

[14] 

[15] 

[16] 

[17] 

[18] 

[19] 

[20] 

[21] 

[22] 

[23] 

[24] 

[25] 

[26] 

[27] 

[28] 

[29] 

[30] 

[31] 

[32] 

[33] 

[34] 

[35] 

[36] 

[37] 

[38] 

[39] 

[40] 

[41] 

[42] 

[43] 

[44] 

[45] 

[46] 

[47] 

[48] 

[49] 

[50] 

[51] 

[52] 

[53] 

[54] 

[55] 

[56] 

[57] 

[58] 

[59] 

[60] 

[61] 

[62] 

[63] 

[64] 

[65] 

[66] 

[67] 

[68] 

[69] 

[70] 

[71] 

[72] 

[73] 

[74] 

[75] 

[76] 

[77] 

[78] 

[79] 

[80] 

[81] 

[82] 

[83] 

[84] 

[85] 

[86] 

[87] 

[88] 

[89] 

[90] 

[91] 

[92] 

[93] 

[94] 

[95] 

[96] 

[97] 

[98] 

[99] 

[100] 

[101] 

[102] 

[103] 

[104] 

[105] 

[106] 

[107] 

[108] 

[109] 

[110] 

[111] 

[112] 

[113] 

[114] 

[115] 

[116] 

[117] 

[118] 

[119] 

[120] 

[121] 

[122] 

[123] 

[124] 

[125] 

[126] 

[127] 

[128] 

[129] 

[130] 

[131] 

[132] 

[133] 

[134] 

[135] 

[136] 

[137] 

[138] 

[139] 

[140] 

[141] 

[142] 

[143] 

[144] 

[145] 

[146] 

[147] 

[148] 

[149] 

[150] 

[151] 

[152] 

[153] 

[154] 

[155] 

[156] 

[157] 

[158] 

[159] 

[160] 

[161] 

[162] 

[163] 

[164] 

[165] 

[166] 





%%% Local Variables:
%%% mode: latex
%%% TeX-engine: xetex
%%% TeX-master: "../Main"
%%% End:
