%% -*- coding: utf-8 -*-
%% Time-stamp: <Chen Wang: 2025-12-09 21:02:06>

% ○ ◎ ‧ 「 」 『 』 々 ( ) “ ” ■ ^[一-龥]
% 【\([^】][^】][^】]+\)】 → {\\fzxk\\zihao{6}\\textcolor{red}{\1}}
% \(【评】.*\) → {\\cangkai\\zihao{5}\1}
% \(【题解】.*\) → {\\cangkai\\zihao{5}\1}
% 《\([^》]+\)》 → \\CJKunderwave{\1}
% ^\([0-9]+.[0-9]+\) → \\lettrine{\1}
% {\\fzxk\\zihao{6}\\textcolor{red}{[^o]*}}

\setlength{\parindent}{0pt}


\chapter{忿狷第三十一}




{\cangkai\zihao{5}【题解】 忿狷者,触事愤怒,行为褊急而毫无耐心也。这是一种常见的心理现象。在魏晋名士身上有非常突出的表现。比如讨厌苍蝇、蚊子,竟然拔剑追杀,简直发疯一般;又如本门记录的王大与王恭,因劝酒不饮,立即反目为仇,竟至动员千人以上的家丁部曲准备厮杀,简直像是战场一样热闹。鲁迅先生曾究其原因,在\CJKunderwave{魏晋风度及文章与药及酒之关系}中说:“晋朝人多是脾气很坏,高傲,发狂,性暴如火的,大约便是服药的缘故。”把原因归结为服五石散(又称寒石散),因为五石散药性燥热有毒所致。这恐怕只是一方面的物质现象,若从精神方面追根究底,则原因非常复杂。社会动荡,战祸频仍,仕途维艰,命运多乖,甚至是生命朝不保夕,长期积压在胸中的痛苦,常借愤慨发怒的方式来加以急遽发泄,以求内在心理的平衡。于此也可见出魏晋名士率性自然,毫无掩饰的一面,这是一种强调自我、张扬个性的明显表现。当然,忿狷也会坏事,名士们也注意到这一点,故性急的王蓝田(述)强行压抑而以柔克刚,不为谢奕的谩骂所动,这更是克服缺点的自然人性之升华。}

\lettrine{31.1} 魏武\myidx{曹操}有一妓\footnote{魏武:曹操。妓:通“伎”,女乐人。},声最清高\footnote{清高:清亮高亢。},而情性酷恶\footnote{酷恶:很坏。}。欲杀则爱才,欲置则不堪\footnote{不堪:无法忍受。}。于是选百人一时俱教\footnote{一时:同时。},少时果有一人声及之,便杀恶性者。

{\cangkai\zihao{5}【评】曹操执掌朝政,史称好刑名之学,似乎是个法家。但法家并不随便杀人,如若犯罪而罪不至死,则无诛杀之理。曹操的歌伎,只因“情性酷恶”——脾气不好而被杀,其法安在?每个人都有自己的性格与脾气,稍不合主人意即加诛戮,这是魏晋门阀社会中奴婢人身依附的悲剧。\CJKunderwave{史记·酷吏列传}中的杜周曾说:“三尺(按:指法律)安出哉?前主所是著为律,后主所是疏为令,当时为是,何古之法乎?”曹操灭绝人性的做法,正反映了封建法律的虚伪性,这不仅是个人行为,更是社会使然。}

\lettrine{31.2} 王蓝田\myidx{王述}性急\footnote{王蓝田:王述字怀祖,爵蓝田侯,故称。}。尝食鸡子\footnote{鸡子:鸡蛋。},以筯刺之\footnote{筯:筷子。},不得,便大怒,举以掷地。鸡子于地圆转未止,仍下地以屐齿蹍之\footnote{屐:木拖鞋,底有前、后齿。碾:蹂踏。},又不得。𧷒(瞋)甚\footnote{𧷒 :他本作“瞋”,是。瞋,发怒。},复于地取内曰(口)中,齧破即吐之\footnote{齧 (niè聂):咬。}。王右军\myidx{王羲之}闻而大笑曰\footnote{王右军:王羲之。}:“使安期\myidx{王承}有此性\footnote{安期:王承,字安期,述父。官东海太守。东晋名臣。},犹当无一豪可论\footnote{豪:通“毫”。},况蓝田邪?”{\fzxk\zihao{6}\textcolor{red}{\CJKunderwave{中兴书}曰:“述清贵简正,少所推屈,唯以性急为累。”安期,述父也,有名德。已见。}}

{\cangkai\zihao{5}【评】这是一篇优秀的小小说。通过王述吃鸡蛋不得的故事,形象描绘了一个性急之人,既生动又典型,细节刻画中的掷、蹍、齧等连续动作,层层加深了对鸡蛋之圆溜及脾性急躁的描绘,人物形象栩栩如生,犹如亲眼所见,艺术非常成功。在当时,王述的性急是有名的,但王羲之的讥评,却另有缘故,因二人素来不惬,故羲之“闻而大笑”,见其轻蔑声色,因子而讥及其父,尤见其傲慢与偏见。后羲之誓墓去官,即与其轻视王述有关,虽是后话,附带及之。}

\lettrine{31.3} 王司州\myidx{王胡之}尝乘雪往王螭\myidx{王恬}许\footnote{王司州:王胡之字修龄,廙子。曾官司州刺史,故称。参前\CJKunderwave{言语}第81则注。王螭:王恬字敬豫,小字螭虎,导次子。参前\CJKunderwave{德行}第29则注。许:处所。},{\fzxk\zihao{6}\textcolor{red}{王胡之、王恬,并已见。恬小字螭虎。}} 司州言气少有牾逆于螭\footnote{牾逆:触犯。},便作色不夷\footnote{作色:变色。不夷:不平,不高兴。}。司州觉恶,便舆床就之\footnote{舆床:搬移座席。就之:靠近他。},持其臂曰:“汝讵复足与老兄计\footnote{讵复:怎么,难道。}?”{\fzxk\zihao{6}\textcolor{red}{按\CJKunderwave{王氏谱},胡之是恬从祖兄。}} 螭拨其手曰:“冷如鬼手馨\footnote{馨:魏晋口语,般、样。},强来捉人臂!”

{\cangkai\zihao{5}【评】这是琅邪王家的内部矛盾。在东晋第一高门士族的琅邪王氏当中,王导一支最为尊贵而兴旺发达,而王廙一支则稍微逊色。王胡之(司州)是王廙子,王恬是王导子,若论大排行,则胡之为从兄,应敬重兄长,故胡之舆床、捉臂而有“汝讵复足与老兄计”之言,以上临下的教训口吻颇重。但小老弟并不买账。史称其人“少卓荦不羁,疾学尚武”,容易发性愤怒,是其性格。他自认宰相之子,名门之后,岂能容人教训,于是稍有忤逆,便即作色,而不问你是老兄前辈。“冷如鬼手馨”,口语生动,声吻毕肖,语言背后是对自家门第、自我个性的肯定和张扬。自然、真率而毫不掩饰,对王螭这个贵族子弟来说,虽然狂妄不足为训,但也有天真可爱的一面。}

\lettrine{31.4} 桓宣武\myidx{桓温}与袁彦道\myidx{袁耽}樗蒱\footnote{桓宣武:桓温,参前注。袁彦道:袁耽字彦道,陈郡阳夏人。官建威将军,司徒从事中郎。参前\CJKunderwave{任诞}第34则注。樗蒱:古代博戏之一。},袁彦道齿不合\footnote{齿:此指博齿,犹如今之骰子,上有点数。不合:点数不符。},遂厉色掷去五木\footnote{五木:樗蒱赌戏中的骰子,凡五子,故称。}。温太真\myidx{温峤}云\footnote{温太真:温峤字太真。参前\CJKunderwave{言语}第35则注。}:“见袁生迁怒\footnote{迁怒:为此生气发怒。},知颜子\myidx{颜回}为贵\footnote{颜子:颜回,孔子最得意的门生。}。”{\fzxk\zihao{6}\textcolor{red}{\CJKunderwave{论语}曰:“哀公问:‘弟子孰为好学?’孔子曰:‘有颜回者好学,不迁怒,不贰过,不幸短命死矣!’”}}

{\cangkai\zihao{5}【评】此则应与\CJKunderwave{任诞}第34则故事并读体味。故事中出现了温太真(峤),温峤卒于晋成帝咸和四年(329),时桓温年仅十七。而袁彦道(耽)卒于咸康初(335),年仅二十五岁,以此上推,则咸和四年耽年十九,袁耽、桓温二人年少相若。故事必然发生在咸和四年前,则二人为十六七岁的少年。在博戏方面,袁耽是个天才,当时享有盛名,赌界唯知袁彦道,几乎是博无不胜。刘注引\CJKunderwave{袁氏家传},谓耽“高风振迈,少倜傥不羁”,是士人心仪的名士。但故事发生时,袁、桓二人皆是少年心性,相戏相争,激怒于一时。因为“赌王”也有时运不济的时候,偶然失手,则狂呼大叫,愤掷五木,是其性格率真自然的表现,故刘辰翁评曰:“于此识彦道。”此未足深责,正见其真面目。}

\lettrine{31.5} 谢无奕\myidx{谢奕}性粗强\footnote{谢无奕:谢奕字无奕,陈郡阳夏人。安兄。官至安西将军,豫州刺史。参前\CJKunderwave{德行}第33则注。粗强:粗暴强横。},以事不相得\footnote{不相得:不相合,不投机。},自往数王蓝田\myidx{王述}\footnote{数:责备,数落。王蓝田:王述。},肆言极骂\footnote{肆言极骂:破口大骂。}。王正色面壁不敢动\footnote{正色:脸色庄重严肃。面壁:面向墙壁不敢看人。}。半日,谢去,良久,转头问左右小吏曰:“去未?”答云:“已去。”然后复坐。时人叹其性急而能有所容\footnote{叹:叹赏。}。

{\cangkai\zihao{5}【评】此则应与前面第2则并读共参,咀嚼体味。王述出于太原王氏家族,其父承(字安期)为渡江名臣,“中兴第一”,王导、庾亮、周顗诸名士甘居其下。作为名门之后,述少有清誉,为人刚正,曾为儿坦之拒婚于权臣桓温。但性急则是其缺陷,王述于此有自知之明,而不像陈郡谢奕那样张狂使性,大骂泄愤,毫无士人修养。王述性急而足蹍鸡子,只为自我泄愤,而与他人无涉;但若与人发生关系,则努力加强自我修养,以免误事。面对谢奕的泼妇骂街,血性男儿谁受得了?但述却以柔克刚,面壁不为所动,忍人之所不能忍。故王开乾评云:“蓝田食鸡子,性似不可解。故佩韦自缓,佩弦自急,因物憬悟,存乎其人。”针对缺点,加强修养以自我改造,这才是真正的大丈夫气概。}

\lettrine{31.6} 王令\myidx{王献之}诣谢公\myidx{谢安}\footnote{王令:王献之,字子敬,羲之少子,官至中书令,故称。谢公:谢安。},值习凿齿\myidx{习凿齿}已在坐\footnote{习凿齿:字彦威,襄阳人。官荥阳太守。有文史之才,撰\CJKunderwave{汉晋春秋}。参前\CJKunderwave{言语}第72则注。},当与并榻\footnote{并榻:同榻共坐。榻:座席。}。王徙倚不坐\footnote{徙倚:徘徊。},公引之与对榻。去后,语胡儿\myidx{谢朗}曰\footnote{胡儿:谢朗字长度,小字胡儿。安二兄据之长子,仕至东阳太守。擅玄谈,善文义,为谢安所赏。}:“子敬实自清立\footnote{清立:清高特立。},但人为尔,多矜咳\footnote{矜咳:沈校本作“矜硋”,意谓矜持拘执,俗称装腔作势。},殊足损其自然\footnote{殊:甚,非常。}。”{\fzxk\zihao{6}\textcolor{red}{刘谦之\CJKunderwave{晋纪}曰:“王献之性甚整峻,不交非类。”}}

{\cangkai\zihao{5}【评】王献之这个琅邪王氏的贵族子弟,不肯与习凿齿同坐议事,完全是门阀制度中士庶之别意识在作祟。习凿齿并非等闲之辈,除书法艺术外,他在文史贡献和政治才干方面,都有杰出的表现,是当时知识分子的精英。但仅仅因其“世为乡豪”,出身于襄阳乡下的豪强地主,而不是世代簪缨的上品贵族,王献之谨守士庶之别犹如天隔的观念,严格“不交非类”,此所谓“清立”、“矜咳”,既不尊重客人,也给主人谢安以难堪。这对习凿齿是明显歧视,是一种不文明、不礼貌的行为。但对献之来说,却是必然的认识。故刘辰翁评曰:“‘矜咳’二字极不成语,然极有似。”正是这一魏晋生活口语,写尽了名士的扭捏作态,形象非常生动。但作为主持朝政的一代名相谢安,其目光深远,为国家民族的利益,他必须同时与士庶保持接触,因此而批评了王献之的虚矫做作有损自然。同是上品贵族,王、谢二人认识不同,因而成就与影响自然不同。}

31. 王大\myidx{王忱}、王恭\myidx{王恭}尝俱在何仆射\myidx{何澄}坐\footnote{王大:王忱字元达,小字佛大,故称。参前\CJKunderwave{德行}第44则注。王恭:字孝伯。参前\CJKunderwave{德行}第44则注。何仆射:何澄,官尚书左仆射,故称。坐:座席,指宴会。},{\fzxk\zihao{6}\textcolor{red}{\CJKunderwave{中兴书}曰:“何澄字子玄。清正有器望,历尚书左仆射。”}} 恭时为丹阳尹\footnote{丹阳:郡名,治建康,故城在今江苏江宁东。},大始拜荆州\footnote{拜荆州:任荆州刺史。}。{\fzxk\zihao{6}\textcolor{red}{\CJKunderwave{灵鬼志·谣征}曰:“初,桓石民为荆州,镇上时(明),民忽歌\CJKunderwave{黄昙曲}曰:‘黄昙英,扬州大佛来上朋(明)。’少时,石民死,王忱为荆州。”佛大,忱小字也。}} 讫将乖之际\footnote{讫:通“迄”,到。将乖:临别。},大劝恭酒,恭不为饮,大逼强之,转苦\footnote{转苦:逼迫更厉害。}。便各以裙带绕手。恭府近千人,悉呼入斋;大左右虽少,亦命前,意便欲相杀。何仆射无计,因起排坐二人之间\footnote{排坐:挤坐。},方得分散。所谓势利之交,古人羞之\footnote{“势利之交”二句:语出\CJKunderwave{汉书·张耳陈馀传赞}。}。

{\cangkai\zihao{5}【评】二王俱出于太原晋阳王氏同一士族。这是同一家族中较势斗力的矛盾。王忱是王坦之的第四子,王恭的族叔。王恭祖父濛,一代清谈名士。父蕴,知名当世,孝武帝王皇后父。王忱与王恭,俱流誉一时。太元中,忱出为荆州刺史,都督荆益宁三州军事,建武将军,年少居方伯之任,自恃才气,任达不拘,眼中少能容物。但王恭作为皇后之兄,也是才气纵横,太元中任丹阳尹,作为外戚帝舅,也是春风得意。二人气势旗鼓相当。宴会之上,王忱以族叔身份,强行劝酒,自示尊贵;但王恭颇傲,偏不为屈,拒而不饮,使王忱大失面子。为泄一己之私愤,双方竟然立即调动千人以上人马,“便欲相杀”,气氛紧张,全然不顾国家利益和朝廷体面。宗族血亲之内,仍然以武力相见,更何况是外人呢!朝廷用人如此,国家岂能兴旺发达?小小家族纠纷,预示了东晋来日无多了。}

{\cangkai\zihao{5}另:本则如与\CJKunderwave{德行}第44则,\CJKunderwave{赏誉}第153则并读互参,则对二王恩怨性质的复杂性将会有更全面更深刻的认识。}

\lettrine{31.8} 桓南郡\myidx{桓玄}小儿时\footnote{桓南郡:桓玄,袭父爵南郡公,故称。参\CJKunderwave{德行}第41则注。},与诸从兄弟各养鹅共斗。南郡鹅每不如,甚以为忿。乃夜往鹅栏间,取诸兄弟鹅悉杀之。既晓,家人咸以惊骇,云是变怪\footnote{变怪:灾变怪异。},以白车骑\myidx{桓冲}\footnote{车骑:桓冲,温弟,曾任车骑将军,故称。见前\CJKunderwave{夙惠}第17则注。}。车骑曰:“无所致怪,当是南郡戏耳\footnote{戏:戏谑,恶作剧。}!”问,果如之。

{\cangkai\zihao{5}【评】近人吴承仕(检斋)曾评桓冲之言曰:“车骑口中,何云南郡?此记事不中律令处。”这可能有两种情况,一是后人之称,借桓冲之口道出;一是桓冲当时真实之言。桓温卒于孝武帝康宁元年(373),时少子玄五岁,温爱少子,临终,命以为嗣,袭爵南郡公。到桓玄少年斗鹅时,早有封爵之号。其叔冲继温掌控荆州,为兄故,抚爱玄胜似己出。其口称玄为“南郡”,一属事实,一是希望诸子侄对玄之恶作剧,看在其父温的面上,不要计较。古时斗鹅之风甚盛。据\CJKunderwave{新唐书}卷二〇八\CJKunderwave{宦者下·田令孜传}曰:“帝(按:指唐僖宗)冲騃,喜斗鹅走马……一鹅至五十万钱。”则斗鹅之戏延至唐末而不息。年少桓玄,一夜之间“取诸兄弟鹅悉杀之”,一只不留,只为争一时之忿,而不计后果。一个人从小看八十,小儿心胸褊狭躁急如此,长大后作为政治风云人物,岂能不败事?性格弱点,早埋下败亡之兆。}





%%% Local Variables:
%%% mode: latex
%%% TeX-engine: xetex
%%% TeX-master: "../Main"
%%% End:
