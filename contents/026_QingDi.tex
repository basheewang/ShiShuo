%% -*- coding: utf-8 -*-
%% Time-stamp: <Chen Wang: 2025-12-08 22:06:25>

% ○ ◎ ‧ 「 」 『 』 々 ( ) “ ” ■ ^[一-龥]
% 【\([^】][^】][^】]+\)】 → {\\fzxk\\zihao{6}\\textcolor{red}{\1}}
% \(【评】.*\) → {\\cangkai\\zihao{5}\1}
% \(【题解】.*\) → {\\cangkai\\zihao{5}\1}
% 《\([^》]+\)》 → \\CJKunderwave{\1}
% ^\([0-9]+.[0-9]+\) → \\lettrine{\1}
% {\\fzxk\\zihao{6}\\textcolor{red}{[^o]*}}

\setlength{\parindent}{0pt}


\chapter{轻诋第二十六}



{\cangkai\zihao{5}【题解】 轻诋者,轻视与诋毁也。“轻诋”是一个并列结构的复合动词。所谓“轻”,不仅是轻蔑、瞧不起,而且还含有轻率的意思,其言行多是不假思索而从潜意识中发出的;所称“诋”则是毁坏他人的言论,明显不是出自善意的立场。\CJKunderwave{轻诋}门的故事共33则,作者态度复杂,并非尽持批评否定的立场。这是因为故事中所轻诋的言行与对象,比较复杂,不能一概而论。如第11则,桓温掌生杀大权,居高临下,轻诋僚佐袁虎(宏)为不能“负重致远”的千斤大牛,将“烹之以飨将士”,作者以“四坐既骇,袁亦失色”加以作结,实际上对桓氏以势压人的言行是持批评的态度。袁氏一句“率尔而对”的寻常语,稍不合桓之意,差点招来杀身之祸,这不太霸道了吗?但在更多的场合,名士之间的相轻相诋,大多是抓住了被诋对象本身的言行错失或性格弱点,因而其所轻诋,常是一语中的,令人难以争辩而不得不服。如第17则谢安与孙长乐兄弟交,“言至款杂”,被安妻讥为“亡兄(刘惔)门未尝有如此宾客”。当时孙绰有才而俗,与刘惔的名士清高形成了鲜明的对照。谢安虽为大名士,面对妻子却也只能是“深有愧色”。这就表现了作者的肯定态度。但更多情况下,作者持中性立场,见仁见智,褒贬不一,留出联想的广阔空间,让广大读者自由驰骋。而且,名士相轻,大多率尔而对,出于性格之自然,其态度天真而不加掩饰,这一点同样也很可爱,显现了魏晋名士的风貌与特点。}

\lettrine{26.1} 王太尉\myidx{王衍}问眉子\myidx{王玄}\footnote{王太尉:指王衍,字夷甫,西晋时官至太尉,故称。参前\CJKunderwave{言语}第23则注。眉子:王玄字眉子,衍子。东海王越辟为掾,行陈留太守。参前\CJKunderwave{识鉴}第12则注。}:“汝叔\myidx{王澄}名士\footnote{汝叔:指玄叔王澄,字平子。衍弟。官至荆州刺史。参前\CJKunderwave{德行}第23则注。},何以不相推重\footnote{推重:推服敬重。}?”{\fzxk\zihao{6}\textcolor{red}{眉子,已见。叔,王澄也。}} 眉子曰:“何有名士终日妄语!”

{\cangkai\zihao{5}【评】王衍与澄兄弟,于西京称一代名士,衍清谈领袖人物,后被石勒所杀,致清谈误国之讥。而澄为任诞名士,史称“经澄所题目者,衍不复有言”。其为人终日“酣宴纵诞,穷欢极娱”,虽在寇戎急务军中,亦不以在怀。其侄王玄,亦是豪气干云的名家,他早看透其叔的性格弱点,故其“何有名士终日妄语”之诋,实是一语中的,并非空穴来风之言。据前\CJKunderwave{识鉴}第12则,王澄对此“逆侄”,也颇不满,有“志大其量,终当死坞壁间”之咒。叔侄之间,早有对立情绪,源于思想认识的不同。}

\lettrine{26.2} 庾元规\myidx{庾亮}语周伯仁\myidx{周顗}\footnote{庾元规:庾亮字元规。颍川鄢陵人。官至中书令。参前\CJKunderwave{德行}第31则注。周伯仁:周顗字伯仁。参前\CJKunderwave{言语}第30则注。}:“诸人皆以君方乐\footnote{方:比方,比拟。}。”周曰:“何乐?谓乐毅邪\footnote{乐毅:战国时燕国上将,曾率诸侯兵大败齐国,封昌国君。}?”{\fzxk\zihao{6}\textcolor{red}{\CJKunderwave{史记}曰:“乐毅,中山人。贤而为燕昭王将军,率诸侯伐齐。终于赵。”}} 庾曰:“不尔,乐令\myidx{乐广}耳\footnote{乐令:指西晋乐广,字彦甫,南阳人。清言名士。官至尚书令。故称。}。”周曰:“何乃刻画无盐\footnote{刻画无盐:精心美化丑女。无盐原是地名,丑女锺离春生于该地。因借称丑女。},以唐突西子也\footnote{唐突西施:冒犯美女西施。西施,春秋时越国美女。后成为中国古代四大美女的典型之一。}。”{\fzxk\zihao{6}\textcolor{red}{\CJKunderwave{列女传}:“锺离春者,齐无盐之女也。其丑无双,黄头深目,长壮大节,鼻昂结喉,肥项少发,折腰出胸,皮肤若漆。行年三十,无所容入,衒嫁不售。乃自诣齐宣王,乞备后宫,因说王以四殆。王□(拜)为正后。”\CJKunderwave{吴越春秋}曰:“越王勾践得山中采薪女子,名曰西施,献之吴王。”}}

{\cangkai\zihao{5}【评】庾亮把周顗比作乐令,本意是推许,而非轻诋。乐广其人,西晋儒、玄双修的清言领袖,并非等闲之辈。其性冲约而有远识,谈论析理,厌人之心。故当时天下言风流者,谓王(衍)、乐(广)称首。但其治国方略与政治才干却是平平,在八王之乱中,忧惧而死。这与周顗的理想并不和谐。顗之率真任放不让西京名士,但处东晋初建之际,酣饮中有清醒之识,国家之责、社稷之重,并未遗忘。其所等待,只是表现的时机。当王敦举兵向阙之时,朝廷散败,君主被囚而人人自危之时,周顗挺身而出,慷慨赴义,骂贼而死。其死重于泰山,岂乐广忧惧自身可比!故顗视庾亮比之乐令为“刻画无盐,唐突西施”,如同以丑女来冒犯美女,是对自己的轻诋。从政治功业和历史道义来看,周顗之言不仅有道理,而且见其率真之性,勃然变色之怒,溢于言表,形象非常生动。}

\lettrine{26.3} 深公\myidx{竺法深}云\footnote{深公:即竺法深,东晋高僧。参前\CJKunderwave{德行}第30则注。}:“人谓庾元规\myidx{庾亮}名士\footnote{庾元规:即庾亮。参前注。},胸中柴棘三斗许\footnote{柴棘:柴草荆棘。许:表约数。三斗许即三斗多。}。”

{\cangkai\zihao{5}【评】深公和庾亮,交游颇多,故其优劣相知亦深。史称庾亮美姿容,善谈论,性好\CJKunderwave{庄}、\CJKunderwave{老},风格峻整,是个儒、玄双修的名士。但作为外戚,他更重要的是一名政治家,“太后临朝,政事一决于亮”。从历史实际看,庾亮治国才干平庸,内诛王室宗亲,离心打击王导;外则举措失当,疑忌陶侃,并直接造成苏(峻)、祖(约)之乱,司马朝廷几乎不保。正如深公所言,庾亮“胸中柴棘三斗许”,心胸狭隘,怎能不乱?深公游于方外,却洞悉方内而抉其要害,此所以为高僧也。故王世懋评曰:“此言得其深。”一语中的。周婴\CJKunderwave{卮林}则反之,以为“元规于法深不薄,今乃发其轻诋,……岂高逸沙门哉”,其意见为尊者讳,实是对于深公的误解。}

\lettrine{26.4} 庾公\myidx{庾亮}权重\footnote{庾公:庾亮。亮妹明帝皇后,成帝时以太后临朝,政事一决于亮,故称“权重”。},足倾王公\myidx{王导}\footnote{倾:倾轧,压制。王公:王导。}。庾在石头\footnote{石头:城名,在建康西面,是当时捍卫京师的军事要塞。},王在冶城坐\footnote{冶城:地名,晋时丹阳郡治所。},大风扬尘,王以扇拂尘曰:“元规尘污人。”{\fzxk\zihao{6}\textcolor{red}{案:王公雅量通济,庾亮之在武昌,传其应下,公以识度裁之,嚣言自息。岂或回贰,有扇尘之事乎?王隐\CJKunderwave{晋书·戴洋传}曰:“丹阳太守王导,问洋得病七年,洋曰:‘君侯命在申,为土地之主。而于申上冶,火光照天,此为金火相铄,水火相炒,以故相害。’导呼冶令奕逊使启镇东徙,今东冶是也。”\CJKunderwave{丹阳记}曰:“丹阳冶城,去宫三里,吴时鼓铸之所。吴平,犹不废。”又云:“孙权筑冶城,为鼓铸之所。”既立石头大坞,不容近立此小城。当是徙县冶(治),空城而置冶尔。冶城疑是金陵本冶(治),汉高六年,令天下县邑,秣陵不应独无。}}

{\cangkai\zihao{5}【评】此则应与前则并读体味。二则都写庾亮,前则虚写,概括其心胸气量狭隘;此则实写,从王导的轻诋之言,联想到庾亮的作为与矛盾。东晋之初,民谣有“王与马,共天下”之言,琅邪王家在王敦、王导的经营下,势力腾腾直上,甚至打压司马皇室。但自王敦败亡,虽然王导忠心王室,但琅邪王家势蹙,庾氏外戚之家,几乎取而代之。围绕朝廷政权,庾、王两族展开了矛盾争斗。故事发生在成帝咸和年间,时庾亮作为江、荆、豫三州刺史,都督六州军事,掌控长江中上游雄兵,几次萌发“东下意”——进京逼迫王导罢相,但因郗鉴反对而止。以此,王导发为“元规尘污人”的慨叹,以喻颍川鄢陵庾氏气焰的甚嚣尘上。王导之言,虽属轻诋性质,却也合乎事实。言语之中,反映出庾、王二族的门阀之争及朝廷的复杂矛盾。同时,王导和庾亮治国施政不同,王导实施道玄无为之治,而庾氏则任法裁物,指导思想各异,故王导视庾之言行为精神污染。}

\lettrine{26.5} 王右军\myidx{王羲之}少时甚涩讷\footnote{王右军:王羲之,字逸少。官至右军将军,会稽内史。王导从子。涩讷:言语迟钝。},在大将军\myidx{王敦}许\footnote{大将军:王敦,字处仲,官大将军,故称。参前\CJKunderwave{文学}第20则注。许:处所,住地。},王\myidx{王导}、庾\myidx{庾亮}二公后来\footnote{王、庾二公:指王导和庾亮。},右军便起欲去,大将军留之,曰:“尔家司空、{\fzxk\zihao{6}\textcolor{red}{王丞相,已见。}} 元规\footnote{尔家司空:指王导,曾官司空,故称。王导与王敦、王羲之同是琅邪王氏家族人物,故称“尔家”。元规:庾亮,字元规。},复可所难\footnote{复可所难:又有什么难处呢?可,与“何”通借。}?”

{\cangkai\zihao{5}【评】在书法与文学方面,王羲之诚为名垂青史的艺术天才。但天才并非人尽神童。羲之少时即有言语“涩讷”并羞见于人的弱点,其为风流所宗的天纵之英,当与其后天努力密切相关。勤奋出天才,并非妄言。}

\lettrine{26.6} 王丞相\myidx{王导}轻蔡公\myidx{蔡谟}\footnote{王丞相:王导。轻:轻视,轻诋。蔡公:蔡谟,字道明,济阳孝城人。克(一作充)子。官至侍中、司徒。},曰:“我与安期\myidx{王承}、千里\myidx{阮瞻}共游洛水边\footnote{安期:王承字安期。千里:阮瞻字千里。按:王、阮二人皆为西晋名士。洛水:水名,在西晋京师洛阳。},何处闻有蔡克\myidx{蔡克}儿\footnote{蔡克儿:指蔡谟。蔡克:他本作“蔡充”。按:\CJKunderwave{晋书}卷七七\CJKunderwave{蔡谟传}作“父克”。故宋本作“克”是。}?”{\fzxk\zihao{6}\textcolor{red}{\CJKunderwave{晋诸公赞}曰:“克字子尼,陈留雍丘人。”\CJKunderwave{克别传}曰:“克祖睦,蔡邕孙也。克少好学,有雅尚,体貌尊严,莫有媟慢于其前者。高平刘整有隽才,而车服奢丽,谓人曰:‘纱縠,人常服耳。尝遇蔡子尼在坐,终日不自安。’见惮如此。是时,陈留为大郡,多人士。琅邪王澄尝经郡入境,问:‘此郡多士,有谁乎?’叓(吏)曰:‘有江应元、蔡子尼。’时陈留多居大位者,澄问:‘何以但称此二人?’吏曰:‘向谓君侯问人,不谓位也。’澄笑而止。克历成都王东曹掾,故称东曹。”\CJKunderwave{妒记}曰:“丞相曹夫人性甚忌,禁制丞相,不得有侍御,乃至左右小人,亦被检简,时有妍妙,皆加诮责。王公不能久堪,乃密营别馆,众妾罗列,儿女成行。后元会日,夫人于青疏台中,望见两三儿骑羊,皆端正可念。夫人遥见,甚怜爱之。语婢:‘汝出问,是谁家儿?’给使不达旨,乃答云:‘是第四、五等诸郎。’曹氏闻,惊愕大恚。命车驾,将黄门及婢二十人,人持食刀,自出寻讨。王公亦遽命驾,飞辔出门,犹患牛迟。乃以左手攀车兰(栏),右手捉麈尾,以柄助御者打牛,狼狈奔驰,劣得先至。蔡司徒闻而笑之,乃故诣王公,谓曰:‘朝廷欲加公九锡,公知不?’王谓信然,自叙谦志。蔡曰:‘不闻馀物,唯闻有短辕犊车,长柄麈尾。’王大愧。后贬蔡曰:‘吾与安期、千里共在洛水集处,不闻天下有蔡克儿!’正忿蔡前戏言耳。”}}

{\cangkai\zihao{5}【评】王导是东晋开国功臣,著名政治家。当时为了东晋建国初期的“统战”需要,他颇有宰相风度,地不分南北,人不分种族,都曾克制自己的贵族脾性,一概巧于周旋。但在潜意识深处,琅邪王家高贵门阀的傲慢与偏见,仍然潜伏盘旋,伺机爆发。在门阀社会中,这一故事说明了王导自有脾气和个性,不允许别人有损其自我尊严。蔡谟与他朝廷共事,也是当时中原世族中的名公巨卿,可能出于对王导的辅政的不满,曾在丞相府坐,导令作伎,谟“不悦而去,导亦不之止”,见\CJKunderwave{晋书}谟传。又因导妾雷氏“颇预政事”以纳贿,蔡讥为“雷尚书”,见\CJKunderwave{惑溺}第7则;又以“朝廷欲加公九锡”之言来讥讽王导之执政。王导因此大为恼火,而以“何处闻有蔡克儿”相诋。开玩笑中,轻慢之色,潜伏了一场政治矛盾。刘辰翁评曰:“人之轻诋,更累其父。”这不仅是个人的玩笑谑语,而且涉及家族与政治,因而连累谟父克也不得不出场“亮相”。}

\lettrine{26.7} 褚太傅\myidx{褚裒}初渡江\footnote{褚太傅:褚裒,字季野。河南阳翟人。卒赠侍中,太傅,故称。参前\CJKunderwave{德行}第34则注。},尝入东,至金昌亭\footnote{金昌亭:驿亭名,在吴县(今苏州)阊门外。},吴中豪右燕集亭中\footnote{吴中:具体指吴县(今苏州)。豪右:豪门右族之人。}。{\fzxk\zihao{6}\textcolor{red}{谢歆\CJKunderwave{金昌亭诗叙}曰:“余寻师,来入经吴,行达昌门,忽睹斯亭,傍川带河,其榜题曰‘金昌’。访之耆老,曰:‘昔朱买臣仕汉,还为会稽内史,逢其迎吏,逆旅比舍,与买臣争席。买臣出其印绶,群吏惭服自裁。因事建亭,号曰“金伤”,失其字义耳。’”}} 褚公虽素有重名,于时造次不相识别\footnote{造次:仓猝,匆忙。},敕左右多与茗汁\footnote{茗汁:茶水。},少箸粽\footnote{粽:佐茶的蜜汁瓜果,如今之蜜饯。},汁尽辄益,使终不得食。褚公饮讫,徐举手共语云:“褚季野。”于是四坐惊散,无不狼狈。

{\cangkai\zihao{5}【评】这则故事与\CJKunderwave{轻诋}无涉,入\CJKunderwave{雅量}或\CJKunderwave{豪爽}门似更合适。前\CJKunderwave{雅量}第18则有钱塘令沈充戏褚,褚举手答曰:“河南褚季野。”事异而情节略同,则传闻之异也。}

{\cangkai\zihao{5}褚裒非等闲之辈,\CJKunderwave{晋书·外戚}有传,史称其“少有简贵之风,与京兆杜乂俱有盛名,冠于中兴”,其气度神韵雅为谢安所重。但吴中豪强,偏是有眼不识泰山,趋炎附势,仗势欺生,灌饮茶水不断,使褚“终不得食”。但褚不以为意,初不报名,饮讫始举手相报,一座惊散,辱人者终自辱。世态炎凉,形于笔端,令人叹息。}

\lettrine{26.8} 王右军\myidx{王羲之}在南\footnote{王右军:王羲之。},丞相\myidx{王导}与书\footnote{丞相:王导。},每叹子侄不令\footnote{令:美好,优秀。},云:“虎㹠\myidx{王彭之}、虎犊\myidx{王彪之}\footnote{虎㹠:王彭之,字安寿,小字虎㹠。彬子,导从侄。㹠,小猪。虎犊:王彪之,字叔虎,小字虎犊。犊:小牛。彬诸子中最有名。官至尚书令。},还其所如\footnote{还其所如:恰如其小字所称。}。”{\fzxk\zihao{6}\textcolor{red}{虎㹠,王彭之小字也。\CJKunderwave{王氏谱}曰:“彭之,字安寿,琅邪人。祖正,尚书郎。父彬,卫将军。彭之仕至黄门郎。”“虎犊,彪之小字也。彪之字叔虎,彭之第三弟。年二十而头须皓白,时人谓之‘王白须’。少有局干之称。累迁至左光禄大夫。”}}

{\cangkai\zihao{5}【评】程炎震、朱铸禹诸人,以为故事当发生于“右军在江州时”。按\CJKunderwave{晋书}本传羲之任江州刺史,出于庾亮临死前的推荐,亮卒于咸康六年(340),羲之赴江州任,必在是年之后。而王导先亮一年而卒。咸康六年之后的江州刺史王羲之,怎能收到已故王导之信呢?程、朱之说明显有误。此信必然写于咸康五年之前。当时政事一决于庾氏家族,晚年王导早被架空。导忧琅邪王氏家族的地位与利益,故有此信,叹子侄不争气,而望羲之奋起光复琅邪王氏的声望。“虎㹠、虎犊,还其所如,”刘辰翁评曰:“言其真如㹠犊耳。”又余嘉锡评曰:“言彭之、彪之,生长高门,而才质凡下,羊质虎皮,恰如其名也。”所论甚是。此言“诋”侄是实,但态度非恶,而是出于恨铁不成钢的急迫心理。垂垂老矣的一代名相,为家族前途而忧心如焚,悲乎!}

\lettrine{26.9} 褚太傅\myidx{褚裒}南下\footnote{褚太傅:褚裒,参前注。},孙长乐\myidx{孙绰}于船中视之\footnote{孙长乐:孙绰字兴公,封长乐侯,故称。}。{\fzxk\zihao{6}\textcolor{red}{长乐,孙绰。}} 言次及刘真长\myidx{刘惔}死\footnote{言次:谈话之间。},孙流涕,因讽咏曰:“人之云亡,邦国殄瘁\footnote{人之云亡,邦国殄瘁:\CJKunderwave{诗经·大雅·瞻卬}诗句,意谓贤人亡失,家国衰败。亡,丧失,奔亡。殄(tiǎn舔)瘁:衰败。}。”{\fzxk\zihao{6}\textcolor{red}{\CJKunderwave{大雅}诗,毛公注曰:“殄,尽。瘁,病也。”}} 褚大怒,曰:“真长平生,何尝相比数\footnote{真长:刘惔。比数:看重,重视。},而卿今日作此面向人!”孙回泣向褚曰\footnote{回泣:止哭。}:“卿当念我\footnote{念:怜悯。}。”时咸笑其才而性鄙。

{\cangkai\zihao{5}【评】永和五年(349),石季龙死,晋徐、兖二州刺史褚裒率师北伐败归,回镇京口。故事当发生于是年,故称“褚太傅南下”。当时褚裒以外戚任封疆大吏,位重爵显,眼中容不得俗物。加以北伐失败,心绪不佳,故其“大怒”,实是乘机抒发心中郁积的一种情绪宣泄,正巧孙绰撞到了他的枪口上。不然,很难理解以诔文名世的孙绰引\CJKunderwave{诗}语来悼念刘惔,会引发这样一场不愉快。孙绰以才高性鄙闻名于世,早为褚裒所轻,\CJKunderwave{太平御览}卷六云引\CJKunderwave{语林},有褚裒游曲阿后湖,公开宣称“孙兴公多尘滓”,“便欲捉之掷水中”,号为戏谑,但轻蔑态度,显然易见。此则借机发作,对孙略加斥责,以发泄自己胸中的愤懑。}

\lettrine{26.10} 谢镇西\myidx{谢尚}书与殷扬州\myidx{殷浩}\footnote{谢镇西:谢尚,字仁祖。安从兄。曾任镇西将军、豫州刺史,故称。殷扬州:殷浩字渊源,时任扬州刺史,中军将军,故称。},为真长\myidx{刘惔}求会稽\footnote{真长:刘惔。求会稽:求任会稽郡内史。当时会稽属扬州。},殷答曰:“真长标同伐异,侠之大者\footnote{侠:通“狭”。心胸狭隘。}。常谓使君降阶为甚\footnote{谓:以为。降阶:走下台阶相迎,以示谦恭礼敬。},乃复为之驱驰邪\footnote{乃复:竟然。 驱驰:奔走效力。}?”

{\cangkai\zihao{5}【评】刘惔与殷浩,俱是东晋一代的清谈玄理名家。刘惔如何“标同伐异”,成为心胸狭隘的人物,\CJKunderwave{晋书·刘惔传}并无记载。但殷浩与之稔熟,其拒绝谢尚的请求,必有一定道理。\CJKunderwave{赏誉}第146则载谢玄有“真长性至峭”的批评。至峭,即严厉苛刻以待人。又,\CJKunderwave{识鉴}第18则载,谢尚、王濛、刘惔看望隐居墓所的殷浩,王、谢有“渊源不起,当苍生何”之叹;而刘则反之,曰:“卿诸人真忧渊源不起邪?”拆穿了殷浩内在的功名之心。可能类似的事情不一而足,早让殷浩不快,故加以拒绝,这是魏晋名士相轻的心理作用。}

\lettrine{26.11} 桓公\myidx{桓温}入洛\footnote{桓公:桓温。洛:洛阳。西晋旧京,当时被羌族姚襄占领。},过淮泗\footnote{淮泗:淮水、泗水。},践北境\footnote{北境:指中原地区。},与诸僚属登平乘楼\footnote{平乘楼:大船层楼。平乘,大船。},眺瞩中原\footnote{中原:此指河南北一带的黄河流域地区。},慨然曰:“遂使神州陆沈\footnote{神州陆沈:中国沦丧。“沈”通“沉”。},百年丘墟\footnote{百年:喻时间长久。丘墟:荒丘废墟。},王夷甫\myidx{王衍}诸人不得不任其责\footnote{王夷甫:王衍字夷甫。西晋清谈领袖。官至太尉。后为石勒所杀。}!”{\fzxk\zihao{6}\textcolor{red}{\CJKunderwave{八王故事}曰:“夷甫虽居台司,不以事物自婴,当世化之,羞言名教,自台郎以下,皆雅崇拱默,以遗事为高。四海尚宁,而识者知其将乱。”\CJKunderwave{晋阳秋}曰:“夷甫将为石勒所杀,谓人曰:‘吾等若不祖尚浮虚,不至于此。’”}} 袁虎\myidx{袁宏}率尔对曰\footnote{袁虎:袁宏字彦伯,小字虎。当时任桓温大司马记室参军。率尔:轻率而不假思索。}:“运自有废兴\footnote{运:气运,国运。},岂必诸人之过?”桓公懔然作色\footnote{作色:生气而脸色大变。},顾谓四坐曰:“诸君颇闻刘景升\myidx{刘表}不\footnote{刘景升:三国时荆州刺史刘表。}?{\fzxk\zihao{6}\textcolor{red}{\CJKunderwave{刘镇南铭}曰:“表字景升,山阳高平人。黄中通理,博识多闻。仕至镇南将军、荆州刺史。”}} 有大牛重千斤,啖刍豆十倍于常牛\footnote{啖:吃。},负重致远,曾不若一羸牸\footnote{羸牸(léi zì雷字):瘦弱的母牛。}。魏武\myidx{曹操}入荆州\footnote{魏武:指曹操。曹丕篡汉建魏后,追尊父操为武帝,故称。},烹以飨士卒\footnote{飨:犒赏。},于时莫不称快。”意以况袁\footnote{况:比拟。},四坐既骇,袁亦失色。

{\cangkai\zihao{5}【评】桓温指斥王衍等清谈误国,在当时颇有市场。如在桓温之前,庾翼遗殷浩书,批判名士“高谈\CJKunderwave{庄}、\CJKunderwave{老},说空终日……身囚胡虏,弃言所非”,应是同一社会思潮的产物。桓温是东晋政坛中一个极厉害的角色,连谢安等王、谢家族代表人物也对他畏忌三分。但作为桓温的直接下属,袁宏却率尔而对,当面顶撞,说明他是不假思索,冲口而出,是潜意识的爆发,所以一时忘记了利害。袁为清谈玄家辩护,立场一贯,是其内心思想感情的自然流露。袁宏曾作\CJKunderwave{名士传},为清谈名士树碑立传,表现了维护玄学思想理论的巨大热情,桓、袁二人,态度相反。但与袁宏的理论抗争不同,桓温之言,一箭双雕,另有政治意图。清谈名士,多出于华丽家族,指责名士清谈误国,矛头同时指向了不合作的高门贵族。因此,他对袁宏的话,会勃然变色,\CJKunderwave{通鉴}胡注所言:“温意以牛况宏,徒能糜俸禄,而无经世之用。”实际比这还严重。作为一个得意忘形的野心家,决不允许不同意见的存在,这已超出了“轻诋”的范围,故有烹杀后快之言。凌濛初因此评曰:“老贼太狠。”}

\lettrine{26.12} 袁虎\myidx{袁宏}、伏滔\myidx{伏滔}同在桓公\myidx{桓温}府\footnote{袁虎:袁宏小字虎。参前注。伏滔:字玄度,官至游击将军。当时与袁同为桓温大司马参军。参\CJKunderwave{言语}第72则注。},桓公每游燕\footnote{游燕:游乐宴饮。燕,通“宴”。},辄命袁\footnote{辄:总是,经常。袁:据袁本当为“袁伏”,指袁宏与伏滔二人,下有“与伏滔比肩”句可证。}。袁甚耻之,恒叹曰:“公之厚意,未足以荣国士\footnote{国士:国家精英。},与伏滔比肩\footnote{比肩:喻地位、声望相等。},亦何辱如之\footnote{何辱如之:还有什么耻辱能比得上这呢?}!”

{\cangkai\zihao{5}【评】此则与\CJKunderwave{宠礼}第2则为姐妹篇,当并读体味。大致仍是文人相轻陋习作祟。但又不尽然。袁、伏二人俱受桓温宠遇,不过情况有异,程度不同。\CJKunderwave{晋书·文苑}滔传称滔有才学,桓温引为参军,深加礼接,每宴集之所,必命滔同游,其宠遇不在袁下。但论其为人,袁宏性格强正亮直,“虽被温礼遇,至于辩论,每不阿屈”,令人为之担心,故凌濛初评曰:“不畏烹大牛耶?”而伏滔作\CJKunderwave{正淮}二篇,建议桓温“权不下授,威不下黩……深根固本,传之百世”,实桓氏集团谋士,后预孝武帝西堂之会,回家之后,告诉儿子说:“百人高会,天子先问伏滔在坐不,此故未易得,为人作父如此,定何如也?”小人得意之色,浮于脸上。其心胸品性难与袁宏相较量。故袁耻与比肩,也是当时名士品格气节的形象体现。}

\lettrine{26.13} 高柔\myidx{高柔}在东\footnote{高柔:字世远。官司空参军,安固令。此与三国魏之高柔字文惠者别是一人。东:此指会稽,在京师建康之东,故称。},甚为谢仁祖\myidx{谢尚}所重\footnote{谢仁祖:谢尚,字仁祖。重:重视,器重。}。既出\footnote{出:指到京师建康。},不为王\myidx{王濛}、刘\myidx{刘惔}所知\footnote{王刘:指清谈名家王濛和刘惔。}。仁祖曰:“近见高柔,大自敷奏\footnote{敷奏:陈述进奏。},然未有所得。”真长云:“故不可在偏地居,轻在角䚥{\fzxk\zihao{6}\textcolor{red}{奴角反。}} 中\footnote{角䚥(nuò诺):角落,屋角。},为人作议论。”高柔闻之,云:“我就伊无所求\footnote{就:接近。伊:他。}。”人有向真长学此言者,真长曰:“我寔亦无可与伊者\footnote{寔:的确,确实。}。”然游燕犹与诸人书:“可要安固\footnote{安固:指高柔,柔曾任安固令,故称。要:通“邀”,邀请。}。”安固者,高柔也。{\fzxk\zihao{6}\textcolor{red}{孙统为\CJKunderwave{柔集叙}曰:“柔字世远,乐安人。才理青(清)鲜,安行仁义。婚太山胡毋氏女,年二十,既有倍年之觉,而姿色清惠,近是上㳅(流)妇人。柔家道隆崇,既罢司空参军、安固令,营宅于伏川,驰动之情既薄,又爱玩贤妻,便有终焉之志。尚书令何充取为冠军参军,铚俛应命,眷恋绸缪,不能相舍。相赠诗书,清婉辛切。”}}

{\cangkai\zihao{5}【评】高柔非隐者,故曾进京“大自敷奏”,希企上知而有所作为。东晋时的清流领袖首推王(濛)、刘(惔)。柔初出伏(畎)川,不为人知。缺乏王、刘的推赏,就难以跻升当时上层社会的贵族沙龙之中,更谈不到有所作为了。刘惔之言,以为高柔长期居于偏远角落,突然进京大发一通议论,或因信息失灵,脱离现实的热点议题;或因未能及时了解官场动态,议论动辄得咎,诽毁随之。真长之言,貌似“轻诋”,却是从实际出发,企图帮助高柔立脚京师,思考人生。故朱铸禹\CJKunderwave{汇校集注}引陶珙曰:“真长对仁祖之言,大是有情,谓偏处言轻,不足为高重耳,而高不免误解。”尔后高、刘二人对话,个性鲜明,生动刻画了魏晋士人出处不同的心理人格。}

\lettrine{26.14} 刘尹\myidx{刘惔}、江虨\myidx{江虨}、王叔虎\myidx{王彪}、孙兴公\myidx{孙绰}同坐\footnote{刘尹:刘惔字真长,曾任丹阳尹,故称。江虨(bīn彬):字思玄,陈留人。统子,官至尚书左仆射,护军将军,领国子监祭酒。参前\CJKunderwave{方正}第42则注。王叔虎:王彪之字叔虎。孙兴公:孙绰字兴公,参前\CJKunderwave{言语}第84则注。},江、王有相轻色。虨以手歙叔虎云\footnote{歙(shè 射):同“摄”,捉持。一说通“胁”,恐吓,威胁。}:“酷吏!”词色甚彊\footnote{词色:声音脸色。彊:强硬。}。刘尹顾谓:“此是瞋邪\footnote{瞋(chēn琛):通“嗔”,发怒,生气。}?非特是丑言声、拙视瞻\footnote{丑言声:说话难听。拙视瞻:脸色难看。}。”{\fzxk\zihao{6}\textcolor{red}{言江此言非是丑拙,似有忿于王也。}}

{\cangkai\zihao{5}【评】这又是名士相轻的生动一幕。江虨是江统的儿子,是个儒、玄双修的名士。性格幽默风趣,围棋堪称国手,政治才干也很不错,简文帝常向他咨询政务。但与之相轻的王彪之,同样也是个政治干才,他与谢安共掌朝政,简文帝称美为“谋无遗策”。他曾任廷尉,执法严厉,近于法家,“时人比之张释之”。故江虨借故斥之为“酷吏”,并且“词色甚彊”——即声色俱厉的样子,可见其感情的激动。这早已超出了“轻诋”的范围,而达到了愤怒的程度。江之诋王,缘故何在?性格不同,抑或思想异趣?待考。但江之怒火,不顾朋友相聚的公开场合,完全出于感情之自然爆发,一点也不掩饰,这样的为人之“真”——不管是优点还是缺点,比起名教之士的虚伪矫饰,还是可爱得多。}

\lettrine{26.15} 孙绰\myidx{孙绰}作\CJKunderwave{列仙·商丘子赞}曰\footnote{\CJKunderwave{列仙}:指汉刘向\CJKunderwave{列仙传}。东晋文学家为\CJKunderwave{列仙传}中的商丘子写赞文,加以赞颂。}:“所牧何物\footnote{牧:放牧。何物:什么东西。}?殆非真猪。傥遇风云\footnote{傥:假如,如果。},为我龙\xpinyin*{摅}\footnote{龙摅:如龙飞腾。摅:舒展,腾越。}。”{\fzxk\zihao{6}\textcolor{red}{\CJKunderwave{列仙传}曰:“商丘子晋者,商邑人。好吹竽,牧豕。年七十不娶妻,而不老。问其道要,言:‘但食老木(术)、昌蒲根,饮水,如此便不饥不老耳。’贵戚富室闻而服之,不能终岁,辄止,吁(呼)将有匿术。”孙绰为赞曰:“商丘卓荦,执策吹竽。渴引(饮)寒泉,饥食昌蒲。所牧何物,殆非真猪。傥逢风云,为我龙摅。”}} 时人多以为能。王蓝曰(田)\myidx{王述}语人云\footnote{王蓝曰:诸本作“王蓝田”,是。王蓝田即王述,字怀祖,太原晋阳人。祖湛,父承,并有高名。述袭爵蓝田侯,故称。}:“近见孙家儿作文,道‘何物真猪’也。”

{\cangkai\zihao{5}【评】王述出于太原王氏,祖湛父承,俱有高名于世,连琅邪王衍也极推崇,王导以承为中兴第一。正因为出生于这样的高门望族,王述一代名士,眼中难容俗物。桓温势炽之时,欲为子求婚于述子坦之,述痛斥之,曰:“汝竟痴邪?讵可畏温面而以女妻兵也。”其门阀意识自然流露,极其真率。对于权倾朝野的权臣尚且轻之,更何况是孙绰!绰有文学才华而性鄙,又好讥调,为人有粗俗的一面。述轻诋之,也是自然之事。孙氏\CJKunderwave{商丘子赞},虽非经典之作,但自有其寓意。“傥遇风云,为我龙摅”,正是借他人酒杯,浇自己的块垒。但述轻其人,故合其前二句为“何物真猪”粗俗之句,这是化神奇为腐朽之笔,采用点金成铁法来丑诋孙绰,谓其低俗如猪,取以为讥诮耳。如此轻诋,亦是一绝。}

\lettrine{26.16} 桓公\myidx{桓温}欲迁都\footnote{桓公:桓温。迁都:温收复洛阳后,上表建议由建康迁都洛阳。},以张拓定之业\footnote{张:扩张,扩展。 拓定之业:开疆拓土,安定国家,指北伐事业。}。孙长乐\myidx{孙绰}上表谏\footnote{孙长乐:孙绰字兴公,封长乐。按:桓温迁都之表见\CJKunderwave{晋书}温传,孙绰反对迁都谏表见绰传。},此议甚有理。桓见表心服,而忿其为异\footnote{忿:恼恨。异:异议,不同意见。}。令人致意孙云:“君何不寻\CJKunderwave{遂初赋}\footnote{\CJKunderwave{遂初赋}:孙绰早年隐居会稽时所作赋,自陈放情山水,知足知止之义。},而彊知人家国事!”{\fzxk\zihao{6}\textcolor{red}{孙绰表谏曰:“中宗龙飞,实赖万里长江,画而守之耳。不然,胡马久已践建康之地,江东为豺狼之场矣。”绰赋\CJKunderwave{遂初},陈止足之道。}}

{\cangkai\zihao{5}【评】桓温轻诋孙绰,实是心知理亏而又听不得不同意见的专制意识作怪。故事发生在穆帝永和十二年(356),作为大司马的桓温率师北伐姚襄,收复洛阳,功高盖主,朝野震动。说是“震动”,朝廷诸臣既有收复失土、恢复中原一线希望的兴奋一面;同时又震慑于桓温权势的迅速膨胀,故各高门士族大姓,联合抵制桓氏集团。在此形势下,孙绰表谏之事,不知不觉中成了国家政治斗争的产物。桓之诋孙,正是针对孙“彊知人家国事”出发,令其温习其早年\CJKunderwave{遂初赋}的知足知止之义,早早退出官场为好。这是一种含蓄的政治威胁。}

\lettrine{26.17} 孙长乐兄弟\myidx{孙统}\myidx{孙绰}就谢公\myidx{谢安}宿\footnote{孙长乐兄弟:指孙统、孙绰兄弟。楚子。绰袭爵长乐侯,故称。统字承公,诞任不拘,善属文,性好山水,官余姚令,卒。 参前\CJKunderwave{品藻}第59则注。谢公:谢安。},言至欵杂\footnote{欵杂:乱七八糟。欵:即“款”,空洞。 杂:杂乱。}。刘夫人在壁后听之\footnote{刘夫人:谢安妻刘氏,沛国刘耽女,惔妹。},具闻其语。谢公明日还,问:“昨客何似\footnote{何似:怎样。}?”刘对曰:“亡兄\myidx{刘惔}门未有如此宾客\footnote{亡兄:指刘惔,字真长。}。”{\fzxk\zihao{6}\textcolor{red}{夫人,刘惔之妹。}} 谢深有愧色。

{\cangkai\zihao{5}【评】据本门第14则“刘尹、江虨、王叔虎、孙兴公同坐”,则孙绰与刘惔为友,时有聚会。刘夫人所称“亡兄门未有如此宾客”,并非事实,而是另有寓意。盖孙氏兄弟,统诞任不拘,绰通率粗鄙,与高门士族名士典雅淡远之风神,相距甚远。刘夫人出身于沛国刘氏家族,其兄惔(真长)与王濛齐名,是当时清谈玄家的领袖人物。受家族传统影响,刘夫人虽为女流,却同样具有浓厚贵族文化的高雅意识,与俗人俗世文化颇有抵忤。孙绰父楚曾上言朝廷,要求国家选贤任能,“无系世族,必先逸贱”(\CJKunderwave{晋书}楚传)。孙绰兄弟受家庭影响,其文化观念则介乎士庶雅俗之间,而非纯而又纯的贵族雅文化,所以形成“言至欵杂”的习惯。统早卒,绰后来虽然努力接近高门名士,争取融入贵族沙龙之中,但仍多次受侮,除本门见诋于褚裒(第9则)、王恭(第22则)外,如\CJKunderwave{方正}门为庾亮作诔,见拒于亮子羲(第48则)。其因“俗”见诋,当与魏晋世族根深蒂固的门阀意识有关。}

\lettrine{26.18} 简文\myidx{司马昱}与许玄度\myidx{许询}共语\footnote{简文:晋简文帝司马昱。 许玄度:许询字玄度,高阳人。清谈名士,善作玄言诗。风情简素,征辟不就。参前\CJKunderwave{言语}第69则注。},许云:“举君、亲以为难\footnote{举君亲以为难:在君主与父亲之间作一选择很困难。}。”简文便不复答,许去后而言曰:“玄度故可不至于此。”{\fzxk\zihao{6}\textcolor{red}{按\CJKunderwave{邴原别传}:“魏五官中郎将尝与群贤共论曰:‘今有一丸药 ,得济一人疾,而君、父俱病,与君邪,与父邪?’诸人纷葩,或父或君。原勃然曰:‘父子,一本也。’亦不复难。”君亲相校,自古如此。未解简文诮许意。}}

{\cangkai\zihao{5}【评】魏晋之世,政权更替,多由篡弑而来,违背了儒家忠义传统精神。故发展至西晋,“忠孝”两难,只能舍忠而提倡“以孝治国”。魏之邴原、晋之许询,其有关君亲两难的讨论,正是时代思潮的产物。在与许询共语时,简文不一定已登帝位,但作为皇室执政的相王,对许询“举君亲以为难”的意见做出迅速的反应,亦是自然之举。如果名士先亲后君,则君王何以立国施政?这是站在司马皇室立场说话,如刘辰翁所评:“似谓玄度无忠国事耳。”}

\lettrine{26.19} 谢万\myidx{谢万}寿春败后还\footnote{谢万:字万石,安弟。少有高名。官至西中郎将、豫州刺史,北伐败后,废为庶人。参前\CJKunderwave{言语}第77则注。},书与王右军\myidx{王羲之}\footnote{王右军:王羲之,字逸少。曾官右军将军,故称。},云:“惭负宿顾\footnote{宿顾:昔日的关怀。}。”右军推书曰:“此禹、汤之戒\footnote{禹汤之戒:此指帝王在困难时下罪己诏,意在收拾人心,争取支持。}。”{\fzxk\zihao{6}\textcolor{red}{\CJKunderwave{春秋传}曰:“禹、汤罪己,其兴也勃焉。”言禹、汤以圣德自罪,所以能兴。今万失律致败,虽复自咎,其可济焉。故王嘉万也。}}

{\cangkai\zihao{5}【评】故事发生于晋穆帝升平三年(359),时西中郎将、豫州刺史谢万奉命北征,因举措失当,寿春大败而归,被朝廷废为庶人。 万为陈郡谢氏家族贵游子弟,一贯矜豪傲物,啸咏自高,而不以政事为怀,受命北征,未尝抚众。羲之知其必败,曾与桓温笺,谓万可以“处廊庙,参讽议”,而非统率之才。温不听。笺见\CJKunderwave{晋书}万传。兵发之前,又曾与万书,戒其稍敛“迈往不屑之韵”,要求万“俯同群辟”而与士卒同甘共苦,古人以为美谈。书载\CJKunderwave{晋书}羲之本传。但万不听,故有此败。羲之所称“禹汤之戒”,谓万屡教不改,现在写信罪己,不过是收买人心而已,可惜悔之已晚矣!此非轻诋之言,似入\CJKunderwave{规箴}门更合适。}

\lettrine{26.20} 蔡伯喈\myidx{蔡邕}睹睐笛椽\footnote{蔡伯喈:蔡邕字伯喈,汉末名士。博学多才,是著名的文学家,又精通音乐。官中郎将,后因董卓之乱,被王允诛杀。睹睐笛椽:“笛椽”疑当作“椽笛”,即用睹睐竹椽做的笛子。},孙兴公\myidx{孙绰}听妓\footnote{孙兴公:孙绰。听妓:观赏歌女表演。},振且摆折\footnote{振且摆折:挥舞敲打而折断。}。{\fzxk\zihao{6}\textcolor{red}{伏滔\CJKunderwave{长笛赋叙}曰:“余同寮桓子野有故长笛,传之耆老,云:‘蔡邕伯喈之所制也。’初,邕避难江南,宿于柯亭之馆,以竹为椽,邕仰眄之,曰:‘良竹也。’取以为笛,音声独绝。历代传之至于今。”}} 王右军\myidx{王羲之}闻\footnote{王右军:王羲之。},大嗔曰\footnote{大嗔:大怒。}:“三祖寿{\fzxk\zihao{6}\textcolor{red}{一作台。}} 乐器\footnote{三祖寿乐器:祖上三代相传的乐器。},虺瓦吊孙家儿打折\footnote{虺(huī灰):摔,击。瓦吊:陶制纺锤。}。”

{\cangkai\zihao{5}【评】看来孙绰不仅粗率鄙俗,而且是个性情中人。观妓激动,手舞足蹈,甚至把演奏家的宝贝——三祖寿乐器睹睐椽笛,当作指挥棒挥舞,以致不慎折断。这对艺术家来说,是个无法弥补的损失。笛子的主人是桓伊,其吹笛艺术,当时江东第一。一旦失去一件得心应手的乐器,痛何如之!右军也是审美艺术专家,国家级的笛子已像陶制纺锤一样被摔击粉碎,孙绰成为风雅罪人,故羲之痛定思痛,怒斥“孙家儿打折”!“孙家儿”三字,与前褚裒语调一样,充满了轻蔑之态。}

\lettrine{26.21} 王中郎\myidx{王坦之}与林公\myidx{支遁}绝不相得\footnote{王中郎:王坦之,字文度,官至中书令。述子。参前\CJKunderwave{言语}第72则注。林公:支遁字道林,东晋高僧。时人或称支公,或称林公。不相得:合不来。}。王谓林公诡辩,林公道王云\footnote{道王:评论王坦之。}:“箸腻颜帢\footnote{颜帢:白帽,横缝以前别后。这是魏时旧制。腻:垢腻。},𦅖布单衣\footnote{𦅖布:一种粗葛布。},挟\CJKunderwave{左传},逐郑康成车后\footnote{郑康成:郑玄字康成,汉末大儒,遍注五经。}。问是何物尘垢囊\footnote{何物:什么东西。尘垢囊:佛家称肮脏俗人为“革囊盛血”之物,喻人身体是尘垢之囊。}?”{\fzxk\zihao{6}\textcolor{red}{中郎,坦之。帢,帽也。\CJKunderwave{裴子}曰:“林公云:‘文度箸腻颜,挟\CJKunderwave{左传},逐郑康成,自为高足弟子。笃而论之,不离尘垢囊也。’”}}

{\cangkai\zihao{5}【评】支遁虽是和尚,但本质却是名士。他颇有文学才能,其讥诋之言调动形象思维,运用修辞比喻的语言艺术来描绘王坦之,形象栩栩如生。“著腻颜帢”二句,从外形方面讥王谨守古代旧制,连一顶肮脏油腻的旧白帽也舍不得丢掉;从内神方面,则诋坦之子传父学,坚持儒家礼教的食古不化思想观念。原来,坦之父述,通经好儒,著\CJKunderwave{春秋旨通}十卷。坦之本人,史称“演\CJKunderwave{废庄}之论,道焕崇儒”。支遁在此为清谈家张目,故讥王氏父子为步郑玄后尘的守旧人物。支遁轻诋王坦之,实是对于坦之轻诋的反击。余氏\CJKunderwave{笺疏}曾引\CJKunderwave{语林},王坦之“为诸人谈,有时或排摈高秃,以如意注林公”。另外,史称坦之又尝作\CJKunderwave{沙门不得为高士论},态度并不友好。魏晋名士相轻相诋,自有缘由。}

\lettrine{26.22} 孙长乐\myidx{孙绰}作\CJKunderwave{王长史诔}\myidx{王濛}云\footnote{孙长乐:孙绰爵长乐侯,故称。王长史:王濛,官司徒左长史,故称。参前\CJKunderwave{言语}第66则注。诔:近似诗体的哀悼奠祭之文。徐师曾\CJKunderwave{文体明辨序说}云:“诔者,累也。累列其德行而称之也。”末寓哀伤之意。}:“余与夫子,交非势利。心犹澄水,同此玄味\footnote{“心犹澄水”二句:心若澄澈之水,同此玄妙之旨。}。”{\fzxk\zihao{6}\textcolor{red}{\CJKunderwave{礼记}曰:“君子之交淡若水,小人之交甘若醴。”}} 王孝伯\myidx{王恭}见曰\footnote{王孝伯:王恭字孝伯。父蕴,祖濛。参前\CJKunderwave{德行}第44则注。}:“才士不逊\footnote{逊:谦逊。},亡祖何至与此人周旋\footnote{周旋:交往,往来。}!”

{\cangkai\zihao{5}【评】孙绰其人,颇具才华,是当时著名的文学家。史称“于时文士,绰为其冠。温、王、郗、庾诸公薨,必须绰为碑文,然后刊石焉”。但论其审美兴味,则介乎雅俗之间,并非纯而又纯的贵族高雅文化,因此,在他努力融入上层贵族沙龙之时,不断遭受高门名士的排斥和嘲讽,即使在他死后,名士的子孙,也不放过他。这是贵族的傲慢与偏见所致,实在很不公平。故王世懋同情地说:“兴公一生受此等苦,死犹烦人。”}

\lettrine{26.23} 谢太傅\myidx{谢安}谓子侄曰\footnote{谢太傅:谢安,卒赠太傅,故称。}:“中郎\myidx{谢万}始是独有千载\footnote{中郎:指谢万,安弟,曾任抚军从事中郎,故称。参本门第19则注。独有千载:千年以来独一无二。}。”车骑\myidx{谢玄}曰\footnote{车骑:指谢玄,字幼度,小名遏。奕子,安侄。卒赠车骑将军,故称。}:“中郎衿抱未虚\footnote{衿抱:胸襟。},复那得独有?”{\fzxk\zihao{6}\textcolor{red}{中郎,谢万。}}

{\cangkai\zihao{5}【评】晋穆帝升平年间,是陈郡谢氏家族升降沉浮的关键时刻。升平元年(357)谢尚死,二年谢奕死,谢氏家族连倒两根顶梁柱。同年,安弟谢万升任西中郎将,豫州刺史,但是很快于三年北征失败废为庶人。于是,在升平四年(360)隐居东山二十馀年的谢安,为了挽救谢氏家族的利益而终于出山,这则故事应该发生于谢万荣升未败的升平二年以前。为了整个谢氏家族利益,谢安寄希望于弟万,故极力为之造舆论,张声势。万之为人,善自炫耀,早有时誉。这个特点谢安并非不知,但因兄弟情深,期望过大,为形势需要而言过其实。子侄的认识则不然。谢玄年轻,说话直率,直指谢万要害,而不以家长的是非为是非,此非轻诋,而是一种独立思考。似入\CJKunderwave{规箴}门更为合适。}

\lettrine{26.24} 庾道季\myidx{庾龢}诧谢公\myidx{谢安}曰\footnote{庾道季:庾龢字道季,亮子。官至丹阳尹、中领军。诧(chà岔):惊讶告诉。朱铸禹\CJKunderwave{汇校集注}引陶珙曰:“诧有二义:一夸耀,一诳诈,此盖夸也。”}:“裴郎\myidx{裴启}云\footnote{裴郎:裴启。一称裴期,字荣期,撰\CJKunderwave{语林}数卷,号曰\CJKunderwave{裴子}。参前\CJKunderwave{文学}第90则注。}:‘谢安谓裴郎乃可不恶\footnote{不恶:不错。},何得为复饮酒\footnote{何得:怎能。}!’{\fzxk\zihao{6}\textcolor{red}{庾龢、裴启,已见。}} 裴郎又云:‘谢安目支道林\myidx{支遁}如九方皋之相马\footnote{目:品目,品评。九方皋:春秋时善相马之人,其相马重神骏而略形色。},略其玄黄\footnote{玄黄:黑色和黄色。},取其隽逸\footnote{隽逸:超逸不群。}。’”{\fzxk\zihao{6}\textcolor{red}{\CJKunderwave{支遁传}曰:“遁每标举会宗,而不留心象喻,解释章句,或有所漏,文字之徒多以为疑。谢安石闻而善之,曰:‘此九方皋之相马也,略其玄黄而取其隽逸。’”\CJKunderwave{列子}曰:“伯乐谓秦穆公曰:‘臣所与共儋缠薪菜者有九方皋,此其于马,非臣之下也。’公使行求马,反曰:‘得矣,牝(牡)而黄。’使人取之,牝而骊。公曰:‘毛物牝牡之不知,何马之能知也?’伯乐曰:‘若皋之观马者,天机也。问其精,亡其粗;在其内,亡其外;见其所见,不见其所不见;视其所视,遗其所不视。若彼之所相,有贵于马也。’既而马果千里足。”}} 谢公云:“都无此二语\footnote{都无:完全没有。},裴自为此辞耳。”庾意甚不以为好,因陈东亭\myidx{王珣}\CJKunderwave{经酒垆下赋}\footnote{东亭\CJKunderwave{经酒垆下赋}:东亭指王导之孙王珣,爵东亭侯。据前\CJKunderwave{伤逝}第2则,王戎经黄公酒垆而伤悼嵇康、阮籍。王珣因此而作赋。}。读毕,都不下赏裁\footnote{都不:完全不。},直云\footnote{直:只。}:“君乃复作裴氏学\footnote{乃复:竟然。}!”于此\CJKunderwave{语林}遂废。今时有者,皆是先写,无复谢语。{\fzxk\zihao{6}\textcolor{red}{\CJKunderwave{续晋阳秋}曰:“晋隆和中,河东裴启撰汉魏以来迄于今时言语应对之可称者,谓之\CJKunderwave{语林}。时人多好其事,文遂㳅(流)行。后说太傅事不实,而有人于谢坐,叙其黄公酒垆,司徒王珣为之赋,谢公加以与王不平,乃云:‘君遂复作裴郎学!’自是众咸鄙其事矣。安乡人有罢中宿县诣安者,安问其归资,答曰:‘岭南凋弊,唯有五万蒲葵扇,又以非时为滞货。’安乃取其中者捉之。于是京师士庶竞慕而服焉,价增数倍,旬月无卖。夫所好生羽毛,所恶成疮痏。谢相一言,挫成美于千载;及其所与,崇虚价于百金。上之爱憎与夺,可不慎哉!”}}

{\cangkai\zihao{5}【评】王珣之赋,人称颇见才情,但谢安都不下赏裁,自有道理。一来王戎过黄公酒垆事,庾亮辨其非于前,乃俗语不实,流为丹青。王珣因之作赋,故谢安以其非真而深鄙其事。二来王珣原为谢万女婿,史称“王、谢二族以猜嫌致隙”,安绝珣婚,又离其弟珉妻(谢安女),二族遂成仇衅。事见\CJKunderwave{晋书}珣传。对于“裴氏学”的厌恶,又说明了魏晋人的小说观念。对于笔记小说,要求真实有据,而不可故作妄语诳人。不仅是志人小说,就是志怪小说,当时人也多信以为真。故干宝\CJKunderwave{搜神记序}谓:“访行事于故老”,“明神道之不诬”。裴启\CJKunderwave{语林}所记谢安之语不实,经安本人指斥为妄,不为时人所重。这一方面由于名人效应,另一方面是由于魏晋时人思想观念所致。}

\lettrine{26.25} 王北中郎\myidx{王坦之}不为林公\myidx{支遁}所知\footnote{王北中郎:指王坦之。参前第21则注。林公:支遁,字道林。参前第21则注。知:知赏。},乃箸论\CJKunderwave{沙门不得为高士论}\footnote{\CJKunderwave{沙门不得为高士论}:王坦之著,文见\CJKunderwave{晋书}坦之传。意谓佛教僧徒并非志行高洁之士。沙门:僧徒。},大略云:“高士必在于纵心调畼(畅)\footnote{纵心:适情任意,心胸舒畅。}。沙门虽云俗外\footnote{俗外:世俗之外。},反更束于教\footnote{束于教:被佛家戒律所约束。},非情性自得之谓也\footnote{情性:本性。自得:自由自在。}。”

{\cangkai\zihao{5}【评】此则与本篇第21则为姐妹篇,当并读体悟。其因果相推,犹如佛家业报。支遁讥坦之为尘垢囊,如朱铸禹所评:“口吻亦实轻薄,非禅师所宜有。”但王作\CJKunderwave{沙门不得为高士论},针对支遁,虽云俗外之人,却喜方内之游,与贵族名士关系密切,并非超凡脱俗之士,所诋在理。但若因个人行为而扩大至整个“沙门”——指佛学界,则夸大其词,打击面无限膨胀,实非笃论。佛学东渐,必经中国化之路,若不与世俗打交道,又将如何实现?故应多视角予以考察评论。}

\lettrine{26.26} 人问顾长康\myidx{顾恺之}\footnote{顾长康:顾恺之字长康。晋陵人。东晋著名画家。参前\CJKunderwave{言语}第88则注。}:“何以不作洛生咏\footnote{洛生咏:西晋洛阳一带书生讽咏语音重浊。又谢安有鼻炎,能作洛生咏,后来名流多学其咏,因音不似,以手掩鼻而吟以仿之。}?”答曰:“何至作老婢声\footnote{老婢:老年女仆。}!”{\fzxk\zihao{6}\textcolor{red}{洛下书生咏,音重浊,故云老婢声。}}

{\cangkai\zihao{5}【评】顾恺之是个艺术天才,强调的是传神写照,张扬自我,而坚决反对机械的形似模仿。但他批评洛生咏为“老婢声”,除了艺术上的原因外,可能多少寄寓了政治因素。“何至作老婢声”,如张万起、刘尚慈所评:“顾长康发此轻诋之论,所为有二:其一,顾氏世居晋陵无锡,语音清浅,鄙夷北人不屑于效仿。其二,顾为桓温挚友,温死谢安执政,有‘鱼鸟无依’之叹。而谢安善为洛生咏,故此轻诋之讥为谢安而发。”所论甚有见地。当日士之所属,地分南北,政有朋党,以此而间接影响了艺术评论。}

\lettrine{26.27} 殷顗\myidx{殷顗}、庾恒\myidx{庾恒}并是谢镇西\myidx{谢尚}外孙\footnote{殷顗:字伯通。陈郡人。与从弟仲堪并有高名。官至南蛮校尉。参前\CJKunderwave{德行}第41则注。庾恒:字敬则,龢子,亮孙。官至尚书仆射。谢镇西:指谢尚,曾任镇西将军,故称。},{\fzxk\zihao{6}\textcolor{red}{\CJKunderwave{谢氏谱}曰:“尚长女僧要适庾龢,次女僧韶适殷歆(康)。”}} 殷少而率悟\footnote{率悟:率直而聪明。},庾每不推\footnote{推:推赏,赞许。}。尝俱诣谢公\footnote{谢公:谢安。},谢公孰视殷曰:“阿巢故似镇西。”{\fzxk\zihao{6}\textcolor{red}{巢,殷顗小字也。}} 于是庾下声语曰\footnote{下声:低声,小声。}:“定何似\footnote{定:到底,究竟。}?”谢公续复云:“巢颊似镇西。”庾复云:“颊似,足作徤(健)不\footnote{足作健不:足以作为强雄之人吗?足:足够。 健:强健,强壮。}?”{\fzxk\zihao{6}\textcolor{red}{\CJKunderwave{庾氏谱}曰:“恒字敬则。祖亮,父龢。恒仕至尚书仆射。”}}

{\cangkai\zihao{5}【评】这是两个年轻表兄弟相轻之例。谢安是谢尚从弟,殷顗与庾恒的从外祖。庾不服殷,故其追问长者,步步深入,“颊似,足作健不?”认为殷只是脸形似外公,神情气质则不一定像外公那样英雄强健。此语轻而非诋,是从精神气度方面提出的更高要求。“下声问”——低声地问,小孩怕人听到秘密,小小狡狯,神态可掬。}

\lettrine{26.28} 旧目韩康伯\myidx{韩伯}\footnote{旧目:旧时品评。 韩康伯:韩伯字康伯。颍川长社人。东晋哲学家,曾续王弼\CJKunderwave{周易注}作\CJKunderwave{系辞注}、\CJKunderwave{说卦注}、\CJKunderwave{序卦注}等。官至豫章太守,领军将军。}:捋肘无风骨\footnote{捋肘:挽袖露出胳膊。无风骨:肥硕而缺乏刚健挺拔的气质。}。{\fzxk\zihao{6}\textcolor{red}{\CJKunderwave{说林}曰:“范启云:‘韩康伯似肉鸭。’”}}

{\cangkai\zihao{5}【评】魏晋人重“容止”,体貌风神是品评人物的重要内容之一。韩伯为人肥胖,故范启有“肉鸭”之讥,言其有肉无骨。“风骨”一词,原用在人物品评,后来才转移到审美场合,成为古代重要的文论概念。刘勰\CJKunderwave{文心雕龙·风骨}篇云:“怊怅述情,必始乎风;沉吟铺辞,莫先于骨。……结言端直,则文骨成焉;意气骏爽,则文风清焉。……刚健既实,辉光乃新。”虽指语言风格,但同样合于人物品评,意指刚健挺拔,风格骏爽的神态。但同一肥胖的韩伯,前\CJKunderwave{品藻}第66则称韩伯虽无骨干,“然亦肤立”——即看上去仍然挺拔,二者毁誉不同。大概“品藻”重神,而“轻诋”重形,一以才情而誉,一以外貌见诋,视角不同,故品目自异,读者自当明辨。}

\lettrine{26.29} 苻宏\myidx{苻宏}叛来归国\footnote{苻宏:前秦皇帝苻坚太子。坚为姚苌所杀,宏携母、妻降晋,官辅国将军。叛:叛逃。 归国:归顺国家,指东晋。},谢太傅\myidx{谢安}每加接引\footnote{谢太傅:谢安。接引:接待援引。}。宏自以有才,多好上人\footnote{上人:凌人之上。},坐上无折之者。适王子猷\myidx{王徽之}来\footnote{王子猷:王徽之。},太傅使共语,子猷直孰视良久\footnote{孰视:熟视。孰通“熟”。直:只是。},回语太傅云:“亦复竟不异人\footnote{亦复:也。不异人:与一般人没有差别。}。”宏大惭而退。{\fzxk\zihao{6}\textcolor{red}{\CJKunderwave{续晋阳秋}曰:“宏,苻坚太子也。坚为姚苌所杀,宏将母妻来投,诏赐田宅。桓玄以宏为将,玄败,寇湘中,伏诛。”}}

{\cangkai\zihao{5}【评】故事发生在晋孝武帝太元十年(385),时西燕慕容冲攻苻坚,坚留太子宏守长安,宏弃城,携母、妻降晋。作为前秦太子,苻宏自高身价,眼中何曾有物?但他忘记了一个最重要的因素:时间、环境变了,身份价值自然不同。机械执一,凌人之上,如同做太子时,就是苻宏人生危机之始。降晋之后,其实他那太子身上的光环早已消失。他虽跻入江东贵族沙龙社会之中,但王徽之“直孰视良久”,经过仔细观察和思考,最终结论是“亦复竟不异人”——即与普通俗人没什么两样,可说是一贬到底,难以翻身。后来苻宏叛晋被诛,当与其受人轻诋时所压抑的民族仇恨心理有关。诋人与被诋,都应该于此汲取人生教训。}

\lettrine{26.30} 支道林\myidx{支遁}入东\footnote{支道林:即支遁,参前注。入东:到会稽访问。会稽在京师建康东面,故云。},见王子猷\myidx{王徽之}兄弟\footnote{王子猷兄弟:王羲之生七子:玄之、凝之、涣之、肃之、徽之、操之、献之。老大玄之早卒。 子猷:王徽之。},还,人问:“见诸王何如?”答曰:“见一群白颈乌,但闻唤哑哑声\footnote{见“一群白颈乌”二句:余嘉锡\CJKunderwave{笺疏}以为“道林之言,讥王氏兄弟作吴言耳”,疑是。据刘盼遂\CJKunderwave{世说新语校笺}云:“晋时乌读鱼韵,哑读麻韵;鱼、模变为歌麻,行于南朝;时北人当不尽通行也。王丞相北人,喜吴语,其子弟多规效之。白颈乌,本读鱼韵,径唤作哑,读入麻韵,以取媚当时。林公诋之,盖比于颜之推诋鲜卑语也。”据音韵发展剖析,有根有据。}。”

{\cangkai\zihao{5}【评】故事当发生于王羲之作会稽内史之时,时间大概在永和六年(350)至十年(354)之间,羲之家居会稽,诸子随侍。支遁为羲之好友,曾一起共游山水,谈玄说微。但对其诸子,却是品目严厉,近乎苛刻,并不因作为其父知友而稍加宽容。这一轻诋,比喻形象生动,但说话相当刻薄,近乎轻薄的态度。对于子猷兄弟等琅邪王家子弟来说,是以其人之道还治其人之身的绝妙手段。子猷、子敬,能否醒悟一二?}

\lettrine{26.31} 王中郎\myidx{王坦之}举许玄度\myidx{许询}为吏部郎\footnote{王中郎:王坦之字文度,曾任抚军从事中郎,故称。许玄度:许询字玄度,小子阿讷。东晋清谈名士,以善作玄言诗知名于世。参前\CJKunderwave{言语}第69则注。吏部郎:吏部官员,掌管选拔官员。其地位在诸曹郎之上。},郗重熙\myidx{郗昙}曰\footnote{郗重熙:郗昙字重熙,鉴子。官至北中郎将。徐、兖二州刺史。参前\CJKunderwave{贤媛}第25则注。}:“相王\myidx{司马昱}好事\footnote{相王:指简文帝司马昱,时以会稽王进位丞相,录尚书执政,故称。好事:多事。},不可使阿讷在坐头\footnote{阿讷:许询。坐头:坐席之上。}。”{\fzxk\zihao{6}\textcolor{red}{讷,询小字。}}

{\cangkai\zihao{5}【评】简文作为相王辅政之日,正是权臣桓温势压朝野之时。司马昱除了优游华林,谈玄论道之外,几乎一无作为。相王“好事”实是出于无奈。政治斗争就是如此残酷。而许询是“相王”身边的“谈客”,并无实际政治经验,以之作为吏部郎——相王政治的左膀右臂,难以组织坚强的政治队伍而只能助长空谈之兴,对于当日严酷的政治斗争,没有一丝一毫的实际作用。郗昙轻诋,从当时的政治斗争形势出发,一箭双雕,用心良苦。}

\lettrine{26.32} 王兴道\myidx{王和之}谓谢望蔡\myidx{谢琰}“霍霍如失鹰师\footnote{王兴道:即王和之。按:刘注谓“祖翼”,翼为“廙”之讹。谢望蔡:谢琰字瑗度,小字末婢,安少子。淝水之战中有大功,封望蔡公,故称。注谓“望蔡”为琰小字,误。参前\CJKunderwave{伤逝}第15则注。霍霍:躁动不安貌。}”。{\fzxk\zihao{6}\textcolor{red}{\CJKunderwave{永嘉记}曰:“王和之,字兴道,琅邪人。祖翼(廙),平南将军。父胡之,司州刺史。和之历永嘉太守、正员常侍。”望蔡,谢琰小字也。}}

{\cangkai\zihao{5}【评】谢安以后,琅邪王氏与陈郡谢家因政见党争之故,颇有嫌隙。安离王珣、珉之婚,即是一例。王和之讥诋谢琰,是否与王、谢二族之间的政见、家庭诸多矛盾有关?待考。但和之之言,却又来自生活,合乎实际。谢琰贵游子弟,性褊急浮躁,难以容人。故和之讥为“霍霍如失鹰师”——就像一个丢掉了猎鹰的驯鹰师,除了心浮气躁之外,还有什么能耐呢?当然,琰在淝水之战中,与从兄玄等率八千北府精兵,勇往直前,大破苻坚百万之师,功勋赫赫。但对于琰言,好事变坏事,祸福转换如轮,他从此居功自傲,躺在功劳簿高枕而卧,听不得不同意见,即在镇压孙恩的战场上,作为晋军主帅,也不为备而出马击敌,意欲灭寇而后食。结果因其浮躁而败亡,不仅个人丢掉性命,也使国家蒙受巨大损失。和之又不幸言中,生活中的谢琰比失鹰师还要狼狈。}

\lettrine{26.33} 桓南郡\myidx{桓玄}每见人不快\footnote{桓南郡:桓玄,温少子,袭爵南郡公,故称。不快:谓办事不聪明,不爽快。},辄嗔云\footnote{嗔:恼怒,生气。}:“君得哀家梨,当复不烝(蒸)食不\footnote{“君得哀家梨”二句:后人据此概括为“哀梨蒸食”的成语,喻俗人不知美丑,糟蹋了美好的东西。}?”{\fzxk\zihao{6}\textcolor{red}{旧语:秣陵有哀仲家梨,甚美,大如升,入口消释。言愚人不别味,得好梨,烝(蒸)食之也。}}

{\cangkai\zihao{5}【评】桓玄是一个政治上失败的野心家。但当他在困境中挣扎奋斗之时,也曾闪现了才情与智慧的光彩。不然的话,就没有人跟随附和来摇旗呐喊了。他对办事不爽快的俗人,讥诋为蒸食哀家梨,多此一举而破坏美味,修辞比喻生动,充满了生活气息。世上此等俗人恶事,比比皆是,令人心痛。故刘辰翁云:“说得甚近人情。”}




%%% Local Variables:
%%% mode: latex
%%% TeX-engine: xetex
%%% TeX-master: "../Main"
%%% End:
