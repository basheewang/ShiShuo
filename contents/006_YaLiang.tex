%% -*- coding: utf-8 -*-
%% Time-stamp: <Chen Wang: 2025-12-02 21:30:04>

% ○ ◎ ‧ 「 」 『 』 々 ( ) “ ” ■ ^[一-龥]
% 【\([^】][^】][^】]+\)】 → {\\fzxk\\zihao{6}\\textcolor{red}{\1}}
% \(【评】.*\) → {\\cangkai\\zihao{5}\1}
% \(【题解】.*\) → {\\cangkai\\zihao{5}\1}
% 《\([^》]+\)》 → \\CJKunderwave{\1}
% ^\([0-9]+.[0-9]+\) → \\lettrine{\1}
% {\\fzxk\\zihao{6}\\textcolor{red}{[^o]*}}

\setlength{\parindent}{0pt}

\chapter{雅量第六}


{\cangkai\zihao{5}【题解】雅量是考察中古士人心灵世界的又一目标。雅量,意谓胸怀宽阔,气度宏大。自古以来,为人推崇。魏晋士大夫崇尚玄远高迈的精神世界,雅量因此被看重,成为士人竞相品题的重要题目。}

{\cangkai\zihao{5}人生最难超越的一关就是死亡。在天崩地坼的巨大悲痛和狂飙巨澜般的情感失衡之际,魏晋士人为我们展现了博大的心灵空间。顾雍丧子,神气不变,而以爪掐掌,血流沾褥。在情感的自我调控中,为我们展现了杰出政治家深沉、博大的精神世界;嵇中散临刑东市,弹奏\CJKunderwave{广陵散}而神色自若,令千古之人叹其雅量。明王世贞评此曰:“每叹嵇生琴,夏侯色,令千古他人览之,犹为不堪,况其身乎?”嵇康“宁为玉碎,不为瓦全”的抗争精神,使真正的名士风度,得到了又一次从容的展现!}

{\cangkai\zihao{5}雅量,不是上天对名士们的眷顾而天生赐予的,它需要后天的陶冶锤炼之功。魏晋政治家注重在日常的生活和山水游览中,主动觅险,以砥砺品性、涵容胸襟,培养自己的雅量。谢太傅未出山之前,与诸人泛海,风起浪涌时,众人早已大惊失色,只有谢安在船头吟咏自若,表现了处危不惊的心理承受能力。同样的,政治家庾亮,因平日训练有素,方能处乱不慌。庾亮与苏峻战,弓箭手忙中出错,误中艄公,群情不安,庾亮的一句“此手那可使著贼”,轻松地化解了一次危机。}

{\cangkai\zihao{5}此外,像王羲之“东床坦腹”、顾和“门边觅虱”等风流佳话,都是中古士人的精神气质获得最富诗意的彰显!}

\lettrine{6.1} 豫章太守顾劭\myidx{顾劭}\footnote{豫章:郡名,治所在今江西南昌。太守:郡的最高行政长官。顾劭:字孝则,晋吴郡吴(今江苏苏州)人。年二十七为豫章太守。},{\fzxk\zihao{6}\textcolor{red}{环济\CJKunderwave{吴纪}曰:“劭字孝则,吴郡人。年二十七,起家为豫章太守,举善以教民,风化大行。”}} 是雍\myidx{顾雍}之子\footnote{雍:顾雍(168—243),字元叹,三国吴郡吴人,出身江南士族。为丞相,任职十九年而卒。}。劭在郡卒。雍盛集僚属自围棋\footnote{自:正,正在。},{\fzxk\zihao{6}\textcolor{red}{\CJKunderwave{江表传}曰:“雍字元叹,曾就蔡伯喈,伯喈赏异之,以其名与之。”\CJKunderwave{吴志}曰:“雍累迁尚书令,封阳遂乡侯,拜侯还弟(第),家人不知。为人不饮酒,寡言语。孙权尝曰:‘顾侯在坐,令人不乐。’位至丞相。”}} 外启信至\footnote{启:报告。信:使者。此指送信的人。},而无儿书,虽神气不变,而心子(了)其故\footnote{心子其故:据明袁氏本,“子”作“了”,是。了,明白。}。以爪搯(掐)掌,血流沾褥。宾客既散,方叹曰:“已无延陵\myidx{季札}之高\footnote{延陵之高:延陵,指春秋时吴国季札,封于延陵,故称延陵季子。据\CJKunderwave{礼记·檀弓下}载,季子长子死,敛以时服,葬之以礼,并云:“骨肉归复于土,命也!”孔子说他合乎礼,顾雍认为自己不能忘情,做不到像季子那样旷达知命。},岂可有丧明之责\footnote{丧明之责:子夏死了儿子,哭得眼睛失明。曾子前去吊丧,曾子哭,子夏也哭。子夏说:“天乎!予之无罪也!”曾子怒责子夏,说他有三条罪过,其中之一说是丧失了儿子,又丧失了眼睛。子夏投杖认错,说:“吾过矣!吾过矣!”顾雍认为自己虽然不能做到如季子般忘情,也不能像子夏那样因丧子而毁伤身体,受到人们的指责。}!”{\fzxk\zihao{6}\textcolor{red}{\CJKunderwave{礼记}曰:“延陵季子适齐,及其反也,其长子死,葬于嬴、博之间。孔子曰:‘延陵季子,吴之习于礼者也。’往而观其葬焉。其坎深不至于泉,其敛以时服。既葬而封,广轮掩坎,其高可隐也。既封,左祖(袒),右还其封,且号者三,曰:‘骨肉归复于土,命也,若魂气则无不之也。’而遂行。孔子曰:‘延陵季子之于礼也,其合矣乎!’”“子夏丧其子而丧其明,曾子吊之曰:‘朋友丧明则哭之。’曾子哭,子夏亦哭,曰:‘天乎,予之无罪也!’曾子怒曰:‘同(商),汝何无罪也?吾与汝事夫子于洙、泗之间,退而老于西河之上。使西河之民疑汝于夫子,尔罪一也;丧尔亲,使民未有闻焉,尔罪二也;丧尔子,丧尔明,尔罪三也。’子夏投其杖而拜曰:‘吾过矣,吾过矣!’”}} 于是豁情散哀\footnote{豁情散哀:消解了哀情,排遣了愁绪。},颜色自若。

{\cangkai\zihao{5}【评】顾雍曾任吴国丞相十九年,是一位有儒家风范的名相。其性情方正,时人见惮,吴主孙权尝叹曰:“顾公在坐,使人不乐”(\CJKunderwave{吴书·顾雍传}),其动静合礼如此。顾雍为人一大特点,是雅量非常。但是,一个杰出的政治家,纵能承受住宦海的风风雨雨,又怎能禁得起痛失骨肉的凄风苦雨?儿子的噩耗传来,他正与人围棋,其神色不变,血流沾褥。白发人送黑发人之深悲巨痛,如天崩地坼,非局外人所能想象。顾雍在情感的自我调控中,为我们展现了成熟政治家深沉、博大的精神世界。但从另外一个角度看,丧子之痛,令人情不能已,放声一哭,又何损起政治家风度呢?鲁迅诗“无情未必真豪杰,怜子如何不丈夫。知否兴风狂啸者,回眸时看小於菟”,倒符合人之常情。\CJKunderwave{伤逝}门王戎丧子故事,则展现了与此截然相反的另一种情感表达方式。可见,顾雍的雅量是江东孙吴儒家礼教长期熏陶内化的结果,王戎的“情之所钟,正在我辈”,则是玄学濡染的产物。}

\lettrine{6.2} 嵇中散\myidx{嵇康}临刑东市\footnote{嵇中散:指嵇康。康曾作中散大夫。嵇康因吕安被捕受牵连,遭锺会诬陷被杀,死于魏景元三年(263)。东市:刑场。汉代在长安东市处决判死刑的人,后因以东市指刑场。},神气不变,索琴弹之,奏\CJKunderwave{广陵散}\footnote{\CJKunderwave{广陵散}:古琴曲名。嵇康善弹此曲。}。曲终,曰:“袁孝尼\myidx{袁准}尝请学此散\footnote{袁孝尼:袁准字孝尼,陈郡(今河南)人。以儒学知名,官至给事中。},吾靳固未与\footnote{靳:吝惜。},\CJKunderwave{广陵散}于今纪(绝)矣\footnote{纪:据袁本,“纪”为“绝”之形讹。}!”{\fzxk\zihao{6}\textcolor{red}{\CJKunderwave{晋阳秋}曰:“初,康与东平吕安亲善。安嫡兄逊淫安妻徐氏,安欲告逊遣妻,以谘于康,康喻而抑之。逊内不自安,阴告安挝母,表求徙边。安当徙,诉自理,辞引康。”\CJKunderwave{文士传}曰:“吕安罹事,康诣狱以明之。锺会庭论康曰:‘今皇道开明,四海风靡,边鄙无诡随之民,街巷无异口之义(议)。而康上不臣天子,下不事王侯;轻时傲世,不为物用。无益于今,有败于俗。昔太公诛华士,孔子戮少正卯,以其负才、乱群、惑众也。今不诛康,无以清洁王道。’于是录康闭狱。临死,而兄弟亲族,咸与共别。康颜色不变,问其兄曰:‘向以琴来不邪?’兄曰:‘以来。’康取调之,为\CJKunderwave{太平引}。曲成,叹曰:‘\CJKunderwave{太平引}于今绝也!’”}} 太学生三千人上书\footnote{太学生:朝廷所设置的最高学府的学生。},请以为师,不许。文王\myidx{司马昭}亦寻悔焉\footnote{文王:指晋文王司马昭。昭仕魏封晋王,死后谥文王。}。{\fzxk\zihao{6}\textcolor{red}{王隐\CJKunderwave{晋书}曰:“康之下狱,太学生数千人请之。于时豪俊皆随康入狱,悉解喻,一时散遣。康竟与安同诛。”}}

{\cangkai\zihao{5}【评】嵇康为竹林名士领袖,其“越名教而任自然”的口号,成为无数魏晋名士的精神高标。嵇康之死,表面上受吕安事牵连,为小人锺会构陷,究其实,乃是残酷政治斗争的激化形式。在司马氏高压专权之际,嵇康持不合作态度,为司马昭所不容。康临刑东市,神气不变,高情千古,盖源自平素内心涵养的工夫。常言道:俗网易脱,死关难避。魏晋士人脱略形累,希心自然,乃人性之趋利避害,故不为难事;但危难关头舍生取义,却是绝对矫饰不得。明王阳明\CJKunderwave{传习录}云:“人于生死念头,本从身命根上带来,故不易去。”正是此意。王世贞评此曰:“每叹嵇生琴,夏侯色,令千古他人览之,犹为不堪,况其身乎?与陶徵士\CJKunderwave{自祭}、\CJKunderwave{预挽},皆超脱人累,默契禅宗,得蕴空解,证无生忍者。”把嵇康说成是了悟生死、看破红尘的修行中人,未免就降低了其精神抗争的悲剧意味!嵇康“宁为玉碎,不为瓦全”的悲壮抗争,绝非“超脱人累”之人所能想象。}

\lettrine{6.3} 夏侯太初\myidx{夏侯玄}尝倚柱作书\footnote{夏侯太初:夏侯玄字太初,三国魏征西将军,以谋反罪被司马氏所杀。(209—254):字太初,三国魏人。曹爽辅政时,他以爽姑之子受重用。曹爽被诛,玄废黜。后与李丰等谋杀司马师,事败,同被诛。他是早期的玄学领袖人物。作书:写信。},时大雨,霹雳破所倚柱,衣服焦然\footnote{焦然:烧焦。},神色无变,书亦如故\footnote{书:书写,动词。}。宾客左右\footnote{左右:指主人身边的人。},皆跌荡不得住\footnote{跌荡:倾倒摇晃、立足不稳的样子。住:停。}。{\fzxk\zihao{6}\textcolor{red}{见顾恺之\CJKunderwave{书赞}。\CJKunderwave{语林}曰:“太初从魏帝拜陵,陪列于松柏下。时暴雨,霹雳正中所立之树,冠冕焦坏。左右睹之皆伏,太初颜色不改。”臧荣绪又以为诸葛诞也。}}

{\cangkai\zihao{5}【评】\CJKunderwave{论语·乡党}谓:“迅雷、风烈必变”,在这点上,魏晋士人之心理镇定,似乎不亚于儒家圣人。雅量为魏晋士人又一精神风度的追求,其表现形态虽不一而足,而面对险象能处之泰然,乃是题中应有之意。名士夏侯玄,若无面对自然界迅雷疾雨时的神色不变作基础,又怎能经得起生死考验,在面临人世灾祸时颜色不异呢?}

\lettrine{6.4} 王戎\myidx{王戎}七岁\footnote{王戎:字濬冲,竹林七贤之一。嵇康(223—262):三国时谯郡铚 (今安徽亳县)人。“竹林七贤”之一。曾任中散大夫,故称嵇中散。当时著名思想家、文学家、清谈名家。因其主张越名教而任自然,抨击礼法之士,不与司马氏统治集团合作,盛年被杀。},尝与诸小儿游。看道边李树多子折枝\footnote{折枝:使树枝弯曲。}。诸儿竞走取之\footnote{竞:争先恐后。走:跑。},唯戎不动。人问之,答曰:“树在道边而多子,此必苦李。”取之信然。{\fzxk\zihao{6}\textcolor{red}{\CJKunderwave{名士传}曰:“戎由是幼有神理之称也。”}}

{\cangkai\zihao{5}【评】王世懋以为“此自是夙慧,何关雅量”,有理。\CJKunderwave{晋书}本传载戎“幼而颖悟,神采秀彻”,小小年纪即有不凡之资,这和他自小善于思考问题,从生活中吸取有益的经验有关,故能超出同龄人。好像是上天故意跟他开了一个大玩笑,王戎一生似与小小的李子结缘——如果说小时候能识别苦李,给他带来了终生受用的“神童”美誉,可中年以后卖李钻核,则徒留尘俗吝啬鬼、守财奴的恶名。真可谓“成也李子,败也李子”!}

\lettrine{6.5} 魏明帝\myidx{曹叡}于宣武场上断虎爪牙\footnote{魏明帝:曹叡字元仲,三国魏第二代君主。曹叡,曹丕之子,魏第二代君主,死谥明皇帝。宣武场:魏都城中讲武之所。断虎爪牙:谓把虎关在牢笼里,以免以爪牙伤人。},纵百姓观之\footnote{纵百姓观之:\CJKunderwave{晋书·王戎传}谓“年六七岁,于宣武场观戏”,可知所观乃是一种人虎相搏的表演。纵,任凭。}。王戎\myidx{王戎}七岁,亦往看。虎承间攀栏而吼\footnote{承间攀栏:抓住笼子的空隙处攀着栅栏。},其声震地,观者无不辟易颠仆\footnote{辟易:惊退。颠仆:仆倒。}。戎湛然不动\footnote{湛然:冷静沉着的样子。},了无恐色。{\fzxk\zihao{6}\textcolor{red}{\CJKunderwave{竹林七贤论}曰:“明帝自阁上望见,使人问戎姓名,而异之。”}}

{\cangkai\zihao{5}【评】故事以类似电影蒙太奇的手法,展现了两幅生动的画面。老虎一声怒吼,震天动地,众人四散奔走;七岁的王戎却湛然不动,了无恐色。渐渐地,惊呼声终归平静,兽散的人群慢慢聚拢,镜头定格在骇异人群中,一张充满稚气、不惧危险的脸上。人生资质有深浅,不可力求,关键的是把握住后天的发展方向。王戎幼而颖悟,天资不凡,但长大后却未见不凡之举。非唯如此,甚至与时俯仰、苟媚取容,求田问舍、聚敛成性,适足为士林中之浊流。虽有客观原因,但究诘主观,则其自甘暴弃,本性使然,令人为之叹惋。}

\lettrine{6.6} 王戎\myidx{王戎}为侍中\footnote{侍中:官名。侍从皇帝左右,职掌傧赞礼仪、护驾陪乘等,并备应对顾问。魏晋时侍中权位颇重。},南郡太守刘肇\myidx{刘肇}遗筒中笺布五端\footnote{南郡:郡名,晋治所在江陵(今湖北)。刘肇:晋南郡太守。生平、事迹不详。遗:馈赠。筒中笺布:一种价格昂贵的细布。端:二丈(一说六丈)为一端。按\CJKunderwave{晋书·王戎传}作“筒中细布五十端”。}。戎虽不受,厚报其书\footnote{书:书信。}。{\fzxk\zihao{6}\textcolor{red}{\CJKunderwave{晋阳秋}曰:“司隶校尉刘毅奏:南郡太守刘肇以布五十疋、杂物遗前豫州刺史王戎,请槛车征付廷尉治罪,除名终身。戎以书未达,不坐。”\CJKunderwave{竹林七贤论}曰:“戎报肇书,议者佥以为讥。世祖患之,乃发口言曰:‘以戎之为士,义岂怀私?’议者乃息。戎亦不谢。”}}

{\cangkai\zihao{5}【评】王戎幼时颖悟,卓荦不群,父王浑卒,故吏赠钱数百万,戎辞而不受,能保持其纯净本色;中年以后,权势渐高,贪鄙之心渐萌,刘肇送笺布厚礼,虽心头发痒,然尚能自持;晚年时,政治混乱,他也就浑水摸鱼、与时舒卷,忘记了士大夫的社会责任。同时,还贪欲无度,聚敛财富,恒若不足。置之于世界文学守财奴画廊,亦毫不逊色。\CJKunderwave{论语}中讲“君子三戒”,第三就是“及其老也,血气既衰,戒之在得”。老年人为家族、为后人计,容易贪婪。当今社会上习见的“五十九岁现象”颇能说明问题。王戎在物质利益面前晚节不保,栽了跟斗,获讥于士林。王戎一生,从神童走向守财奴的堕落轨迹,给后人的启示是,没有人是天生爱财如命或视金钱如粪土,其观念和行为,因受外部环境影响而发生潜移默化的改变。“日三省吾身”的道德自律,加上严格的法制他律,是保证“永不变色”的不二法门。}

\lettrine{6.7} 裴叔则\myidx{裴楷}被收\footnote{裴叔则:裴楷字叔则。裴令公:即裴楷,曾官中书令,故云,又称“裴令”。善\CJKunderwave{老}、\CJKunderwave{易},当时著名清谈名家。二国租钱:指从梁、赵二国税收所获钱财。},神气无变,举止自若。求纸笔作书,书成,救者多,乃得免。后位仪同三司\footnote{仪同三司:官名。位非三公而礼仪排场与三公同。始为一时宠遇,后成为正式官职。}。{\fzxk\zihao{6}\textcolor{red}{\CJKunderwave{晋诸公赞}曰:“楷息瓒,取杨骏女。骏诛,以楷婚党,收付廷尉。侍中傅祇证楷素意,由此得免。”\CJKunderwave{名士传}曰:“楚王之难,李肇恶楷名重,收将害之。楷神色不变,举动自若。诸人请救得免。”\CJKunderwave{晋阳秋}曰:“楷与王戎,俱加仪同三司。”}}

{\cangkai\zihao{5}【评】面对死亡的从容潇洒,是魏晋士人雅量的重要表现。嵇生琴,夏侯色,早已卓绝千古;裴叔则面对死亡,没有戚戚然的惊惧,而沉着思考,以智自救。中古士人的文采风流,又一次得到了美丽的体现!汉末之际,“忧生之嗟”,一直萦绕于士人之间,挥之不去,“人生天地间,忽如远行客”、“人生非金石,岂能长寿考”,淡淡的叹逝忧伤,与浓浓的行乐情绪,弥漫在骚客文人的字里行间。继而,魏晋士人从老庄空灵、超拔的诗意人生感悟中,寻找到了莫大的精神慰藉。麈尾清谈间,死亡已不再是沉重而令人忌讳的话题,他们把死亡当作生命旅程的最后华彩诗章,以“归去”的态度从容抒写!}

\lettrine{6.8} 王夷甫\myidx{王衍}尝属族人事\footnote{王夷甫:王衍,王夷甫:王衍(256—311)字夷甫,见刘孝标注。“以清虚通理称”,为当时清谈名家,“妙悟若神”,“妙善玄言,唯谈\CJKunderwave{老}、\CJKunderwave{庄}为事”。为政多谋略,不以经国为念,而善思自全之计,然终为石勒所害。(见\CJKunderwave{晋书}本传)注。属:同“嘱”。嘱托。},经时未行\footnote{经时,多时。行:做。}。遇于一处饮燕\footnote{饮燕:饮宴,饮宴,设宴喝酒。},因语之曰:“近属尊事,那得不行?”族人大怒,便举樏掷其面\footnote{樏:一种食盒。形似盘,中有隔,每具有底有盖,谓之一沓。}。夷甫都无言,盥洗毕,牵王丞相\myidx{王导}臂,与共载去。在车中照镜,语丞相曰\footnote{丞相:指王导。导与衍为同宗兄弟。}:“汝看我眼光,乃出牛背上\footnote{出牛背上:牛背为着鞭之处,眼光出于牛背上,意指不计较挨打受辱之类的小事。}。”{\fzxk\zihao{6}\textcolor{red}{王夷甫盖自谓风神英俊,不至与人校。}}

{\cangkai\zihao{5}【评】王衍寡廉鲜耻之辈,为石勒所执,为求自免,竟厚颜无耻地劝勒称帝号,士林领袖的颜面气节丧失殆尽,实为民族败类。但在现实生活中,却以一个谦谦君子的形象炫耀自己。他为族人所伤,之所以能隐忍不发,非出于宽容雅量,主要原因在于过度地自负。既然自视甚高,又怎能与俗人一般计较呢?“车中照镜”细节,形象地传达出其极在意个人风度的微妙心理。忸怩作态之状,甚至令人作呕。又“汝看我眼光,乃出牛背上”一语,诸家众说纷纭。今人范子烨以为,牛背是挨鞭打的地方,喻俗人俗物。王衍自以为风采过人,眼光也高人一等,所以不屑于与俗人计较(\CJKunderwave{中古文人生活研究})。言之有理。}

\lettrine{6.9} 裴遐\myidx{裴遐}在周馥\myidx{周馥}所\footnote{裴遐:字叔道,河东闻喜(今山西)人。周馥:字祖宣,西晋汝南安成(今河南正阳东北)人。},馥设主人\footnote{设主人:作东道主请客。}。{\fzxk\zihao{6}\textcolor{red}{邓粲\CJKunderwave{晋纪}曰:“馥字祖宣,汝南人。代刘淮(准)为镇东将军,(镇)寿阳。移檄四方,欲奉迎天子。元皇使甘卓攻之,馥出奔,道卒。”}} 遐与人围棋,馥司马行酒\footnote{司马:军府佐吏,掌兵事,位在长史下。行酒:巡行酌酒劝饮。当时宴宾习语。}。正戏,不时为饮。司马恚,因曳遐坠地。遐还坐,举止如常,颜色不变,复戏如故。王夷甫问遐:“当时何得颜色不异?”答曰:“直是暗当故耳\footnote{暗当故耳:语义未详。或谓“暗当,暗合也,盖谓本无意而漫相当”(朱铸禹\CJKunderwave{世说新语汇校集注})。}!”{\fzxk\zihao{6}\textcolor{red}{一作“暗故当耳”,一作“真是斗将故耳”。}}

{\cangkai\zihao{5}【评】裴遐乃王衍女婿,这对翁婿之间倒有某种契合,可谓臭味相投。其共同点是极能忍耐、喜怒不形于色。宴会上被人从座位上拖到地下,是极其难堪之事。若逞一时匹夫之勇,以泄胸中恶气,则可能拳脚相见,宴会就变成了打斗场。但裴遐举止如常,颜色不变,令人称羡其雅量。盖其出身名门,高自期许,故不屑与下级官吏一般见识而自降身价。习以成俗,生活中刻意的忍耐,最终会上升为自然的雅量,遂无意间为魏晋风度添上颇具风采的一笔。文末岳父王衍向女婿探讨“忍功”,裴遐答是出于无心之自然,极富生活气息,令人忍俊不禁。}

\lettrine{6.10} 刘庆孙\myidx{刘璵}在太傅\myidx{司马越}府\footnote{刘庆孙:即刘璵。与弟刘琨齐名。历官中书郎、颍川太守。后依附东海王越,为其谋主。太傅:指东海王司马越。},于时人士多为所构\footnote{构:陷害。},唯庾子嵩\myidx{庾敳}纵心事外\footnote{庾子嵩:庾敳字子嵩。纵心:放任其心。},无迹可间\footnote{迹:形迹,迹象。间:离间。}。后以其性俭家富,说太傅令换千万\footnote{换:借支,借贷。},冀其有吝,于此可乘。{\fzxk\zihao{6}\textcolor{red}{\CJKunderwave{晋阳秋}曰:“刘舆(璵)字庆孙,中山人。有豪侠才算,善交结。为范阳王虓所昵。虓薨,太傅召之,大相委仗,用为长史。”\CJKunderwave{八王故事}曰:“司马越字元超,高密王泰长子。少尚布衣之操,为中外所归。累迁司空、太傅。”}} 太傅于众坐中问庾,庾时颓然已醉\footnote{颓然:酒醉无力的样子。},帻堕几上\footnote{帻:巾帻,包头巾。几:几案。},以头就穿取。徐答云:“下官家故可有两娑千万\footnote{故:确实。娑:三。盖古吴语。},随公所取。”于是乃服。后有人向庾道此,庾曰:“可谓以小人之虑,度君子之心。”

{\cangkai\zihao{5}【评】庾敳是魏晋士人中少有的清醒者。西晋八王之乱,权力易主有如走马,社会秩序毫无规范可言。庾敳静默无为,纵酒佯狂、聚敛积实,并非出于名士的清高与任诞;而是值天下多故、机变屡起的时代,一种无奈的掩人耳目,和自我保护之举。他纵酒和贪财,其意在示人毫无政治野心,一种无奈的心理,正如阮嗣宗“终身履薄冰,谁知我心焦”(\CJKunderwave{咏怀诗})诗句所传达出的人生意绪。故事表面上呈现给世人的是庾敳的雅量,背后蕴涵的是悲剧时代中的生存智慧。}

\lettrine{6.11} 王夷甫\myidx{王衍}与裴景声\myidx{裴邈}志好不同\footnote{王夷甫:王衍(256—311)字夷甫,见刘孝标注。“以清虚通理称”,为当时清谈名家,“妙悟若神”,“妙善玄言,唯谈\CJKunderwave{老}、\CJKunderwave{庄}为事”。为政多谋略,不以经国为念,而善思自全之计,然终为石勒所害。(见\CJKunderwave{晋书}本传)注。裴景声:裴邈,字景声,西晋河东闻喜(今属山西)人。裴頠从弟。志好:志趣爱好。},景声恶欲取之\footnote{恶欲取之:谓诋毁他而要取得的回报。},卒不能回\footnote{卒:始终。回:改变。}。乃故诣王\footnote{故:特地。},肆言极骂,要王答己,欲以分谤\footnote{分谤:分担非议。}。王不为动色,徐曰:“白眼儿遂作\footnote{白眼儿:因发怒而瞪大眼睛,眼白突出。这里指瞪着白眼的裴景声。作:发作。}。”{\fzxk\zihao{6}\textcolor{red}{\CJKunderwave{晋诸公赞}曰:“邈字景声,河东闻喜人。少有通才,从兄頠器赏之。每与清言,终日达曙。自谓理构多知,辄每谢之,然未能出也。历太傅从事中郎、左司马,监东海王军事。少为文士而经事,为将虽非其才,而以罕重称也。”}}

{\cangkai\zihao{5}【评】裴邈是沽名钓誉之辈,大概对王衍的暴得大名心存嫉妒,欲借与王衍对骂,以收既抬高自己的社会知名度,又贬损王衍公众形象的“一石二鸟”之效。此招恶毒,想必一般人都会就范。谁知王衍忍功极强,并不与其计较。裴邈泼妇骂街,千呼万唤,却唤不来敌手应战,只能自讨没趣,自掉身价。苏轼\CJKunderwave{留侯论}云:“此其所挟者甚大,而其志甚远也。”王衍向来以大名士自诩,自不屑与肆言极骂的裴邈一般见识。这就表现了不凡的名士风度,若不经平素的刻意修炼,也很难在人际纠纷时安之若素。当然,这里仅是就外在的风度而言,若论人格高下,那又另当别论。}

\lettrine{6.12} 王夷甫\myidx{王衍}长裴成公\myidx{裴頠}四岁\footnote{裴成公:指裴頠。王夷甫:王衍(256—311)字夷甫,见刘孝标注。“以清虚通理称”,为当时清谈名家,“妙悟若神”,“妙善玄言,唯谈\CJKunderwave{老}、\CJKunderwave{庄}为事”。为政多谋略,不以经国为念,而善思自全之计,然终为石勒所害。(见\CJKunderwave{晋书}本传)注。},不与相知。时共集一处,皆当时名士,谓王曰:“裴令令望何足计\footnote{令望:美好的名声。此指声望高。何足:哪里值得,不必。}?”王便卿裴\footnote{卿:“卿”本为官爵,后用为对人的美称。至魏晋南北朝时,转为上对下、长对幼之称。朋友间低于自己的或亲近而不拘礼数者也可称“卿”,但无交情者不可称“卿”。},裴曰:“自可全君雅志\footnote{全:成全。雅志:高雅的志趣。}。”{\fzxk\zihao{6}\textcolor{red}{裴顗,已见。}}

{\cangkai\zihao{5}【评】名士王衍虽矜持自许,亦偶有放松警惕、得意忘形之时。他见有人贬损裴頠堂叔裴楷的名誉,便也一时跟着起哄捡便宜,以裴頠前辈身份相称,呼裴为卿,轻慢无礼。裴頠持论“崇有”,比较注重儒家礼教,深患王衍之徒的口谈玄虚、不尊儒术。他本可借此机会小题大做,但竟不与其计较,确实有很深的修养,非矫饰可得。结尾处“自可全君雅志”,仍以“君”呼衍,见出其不温不火、处辱不乱的素养,也是对好名而善于自我包装的王衍的莫大讽刺!古往今来,世人多看不透世事沉浮盛衰之理,对外在名分过于看重、执着,结果闹得身败名裂,为天下耻笑。“桃李不言,下自成蹊”,社会声誉自有历史公论,裴頠的“不争”,是对哲学精神的大彻悟,是极明智的做法。陈梦槐评曰:“冲怀可挹,语自澹宕。”评价精当。}

\lettrine{6.13} 有往来者云:“庾公\myidx{庾亮}有东下意\footnote{往来者:指来往于京城和武昌的人。庾公:指庾亮。庾亮(289—340)的敬称。他历仕东晋元、明、成三朝,作为外戚,曾执国政,显赫于朝。的卢:传说中的凶马之名,骑之不利主人。注。陶侃薨,亮都督江、荆六州军事,镇武昌。有黜王导意,郗鉴劝止。东下:顺江东下。指从荆州东下侵犯京师的意图。}。”或谓王公\myidx{王导}\footnote{王公:王导。}:“可潜稍严\footnote{潜:暗中。稍严:略作戒备。},以备不虞\footnote{不虞:意料不到;不测。}。”王公曰:“我与元规,虽俱王臣\footnote{元规:庾亮字元规。},本怀布衣之好\footnote{布衣之好:贫贱之交,指未做官时的交情。布衣,平民百姓。}。若其欲不(来),吾角巾径还乌衣\footnote{若其欲不:袁本“不”作“来”,是。角巾:古代男子戴的方巾,为隐居者的冠饰。乌衣:指乌衣巷。晋时建康街巷名,在朱省桥南,是东晋豪门世族聚集之所。“角巾径还乌衣”,指弃官当百姓。},{\fzxk\zihao{6}\textcolor{red}{\CJKunderwave{丹阳记}曰:“乌衣之起,吴时乌衣营处所也。江左初立,琅邪诸王所居。”}} 何所稍严\footnote{何所:什么。}!”{\fzxk\zihao{6}\textcolor{red}{\CJKunderwave{中兴书}曰:“于是风尘自消,内外缉穆。”}}

{\cangkai\zihao{5}【评】俗谚有云:世上没有永恒的朋友,只有永恒的利益。此话若用在门阀世族与晋室皇权的关系上,极确切不过。衣冠南渡,东晋中兴的局面,王导兄弟有大功焉,所谓“王与马,共天下”是也。政治权力的分配须在不断调整分化中,获其大体的平衡。若权力畸轻畸重、破坏微妙的平衡机制,就极容易引起政治动荡乃至大地震。故晋元帝抑制王氏,遂致王敦之逼,明帝依靠外戚庾氏以牵制王家。名相王导在各种势力间相与周旋,步履维艰。特别是王敦难后,更加小心谨慎,以避天下怨谤。王导毕竟是久经磨难、胸有城府的超一流政治家,其应对政治事件能力令人称羡。故事中,有人告之庾公有东下意,劝王导加强防备,导之答语,既见其处变不惊的政治谋略,又可窥其在玄思濡染下,收放自如的人生心态。}

\lettrine{6.14} 王丞相\myidx{王导}主簿欲检校帐下\footnote{王丞相:王导。主簿:丞相府属官,负责文书簿籍。检校:检查。帐下:此指帐下办公人员,幕僚。},公语主簿:“欲与主簿周旋\footnote{周旋:交往,商量,打交道。},无为知人机(几)案间事\footnote{无为:不要,不必。机案间事:指案卷文牍之类的事情。}。”

{\cangkai\zihao{5}【评】东晋开国之初,元帝欲行法家之政。王导则以道家顺应自然相规劝。元帝死后,王导与外戚庾亮受遗诏并辅幼主,“王导辅正,以宽和得众,亮任法裁物,颇以此失人心”(\CJKunderwave{晋书}导本传)。可见王导治国方略,与庾氏兄弟严刑峻法不同——他以道家无为之道、顺随自然、适时而动为指导思想。王导官衙中的主簿要去检查下属办公情况,王导劝他不要去探听下级官吏的几案间事。这与阮籍任东平太守时,令人将官署壁障全部拆掉,如出一辙。日本学者吉川幸次郎评价阮籍的做法“这完全是光明正大的政治,或者更在一般所谓‘光明正大’以上。而且治理的结果是‘政令清宁’”(\CJKunderwave{中国诗史})。此评用在王导身上,也同样合适。用人不疑,疑人不用,放其手脚,方能端拱于上,收“君逸臣劳”之效。\CJKunderwave{论语}中孔子早就告诫政治家“为政以德,譬如北辰,居其所,而众星共之”。为政者不可不深思!无数的历史现实已然证明,“防民之口,甚于防川”!若兴察察之明,一定是其统治地位岌岌可危、缺乏起码自信的时候,其结果必然是道路以目,天下离心。}

\lettrine{6.15} 祖士少\myidx{祖约}好财\footnote{祖士少:祖约(?—330),字士少,东晋范阳遒县人(今河北涞水)人。祖逖异母弟。以讨庾亮为名,与苏峻起兵叛乱。后为温峤、陶侃等击败,奔后赵,为石勒所杀。},阮遥集\myidx{阮孚}好屐\footnote{阮遥集:阮孚,阮咸次子,晋元帝世为安东参军,历侍中、吏部尚书、丹阳尹、广州刺史等。屐:木屐,底上有齿的木鞋。},并恒自经营\footnote{恒自:经常。经营:筹划料理。}。同是一累\footnote{累:牵累,负担。},而未判其得失\footnote{判:分辨,判明。}。{\fzxk\zihao{6}\textcolor{red}{\CJKunderwave{祖约别传}曰:“约字士少,范阳道(遒)人。累迁平西将军、豫州刺史,镇寿阳。与苏峻反,峻败,约投石勒。约本幽州冠族,宾客填门。勒登高望见车骑,大惊。又使占夺乡里先人田地,地主多恨。勒恶之,遂诛约。”\CJKunderwave{晋阳秋}曰:“阮孚字遥集,陈留人,咸第二子也。少有智调,而无隽异。累迁侍中、吏部尚书、广州刺史。”}} 人有诣祖,见料视财物\footnote{料视:料理,查看。}。客至,屏当未尽\footnote{屏当:收拾,料理。},馀两小簏\footnote{簏:竹箱。},著背后,倾身障之,意未能平\footnote{平:舒展,平和。}。或有诣阮,见自吹火蜡屐\footnote{自:正在。吹火:以口吹气,使火加旺。蜡屐:给屐打蜡。蜡,用作动词,打蜡。},因叹曰:“未知一生当着几量屐\footnote{量:通“两”。量词。用于鞋子,犹“双”。}!”神色闲畅\footnote{闲畅:闲适安详。}。于是胜负始分。{\fzxk\zihao{6}\textcolor{red}{\CJKunderwave{孚别传}曰:“孚风韵疏诞,少有门风。”}}

{\cangkai\zihao{5}【评】祖约财迷,阮孚鞋痴。表面无异,细究大有不同:祖约聚敛无度,\CJKunderwave{祖约别传}载其“占夺乡里先人田地,地主多恨”。后因背叛故国,又见利忘义,为石勒所杀,死有馀辜;阮孚爱屐则属于个人业馀爱好,有似今天之各类收藏癖,是一种自遣自乐的无功利心态,与祖约之贪婪占有欲不同。故事以对比映衬手法,将二人心态刻画逼肖,活灵活现,可谓神来之笔。约“著背后,倾身障之,意未能平”,三言两语将一守财奴下意识的贪婪本性描摹殆尽;而阮孚“吹火蜡屐”、“神色闲畅”,特别是“未知一生当著几量屐”一语,超越功利层面,透出智者对人生短暂的理性思考。在个人业馀爱好中发人生感悟,并非玩物丧志。二人之高下立判。金代王寂\CJKunderwave{三友轩记}曰:“如谢康乐之山水,陶彭泽之田园,嵇康之锻,阮孚之屐,虽其所寓不同,亦各适其适也。”指出了阮孚“鞋痴”背后所昭示的名士风度。}

\lettrine{6.16} 许侍中\myidx{许璪}、顾司空\myidx{顾和}俱作丞相从事\footnote{许侍中:许璪,字思文,东晋义兴阳羡(今江苏宜兴)人。历丞相从事、侍中,侍至吏部侍郎。顾司空:顾和,死后追赠司空,故称。丞相从事:丞相府属官。三公和州郡均设从事。},尔时已被遇\footnote{被遇:被赏识。},游宴集聚,略无不同。{\fzxk\zihao{6}\textcolor{red}{\CJKunderwave{晋百官名}曰:“许璪字思文,义兴阳羡人。”\CJKunderwave{许氏谱}曰:“璪祖艳,字子良,永兴长。父裴,字季显,乌程令。璪仕至吏部侍郎。”}} 尝夜至丞相\myidx{王导}许戏\footnote{许:处所。},二人欢极。丞相便命使入己帐眠。顾至晓回转\footnote{回转:翻来覆去。},不得快熟\footnote{快熟:指睡得很踏实很熟。}。许上床便咍台太鼾\footnote{咍台:打鼾声。}。丞相顾诸客曰:“此中亦难得眠处。”{\fzxk\zihao{6}\textcolor{red}{顾和字君孝,少知名。族人顾荣曰:“此吾家骐骥也,必兴吾宗!”仕至尚书令。五子:治、隗、淳、履之、(台民)。}}

{\cangkai\zihao{5}【评】许璪、顾和同为丞相王导从事,并被重用。二人人生心态并不相同,许璪倒头便睡,并不因睡在上级的床上而忐忑不安,恐怕就是睡在皇帝的龙床上也并不觉得有什么不同,因胸无块垒而形神相亲,一派自然天放的名士风度;顾和则相反,心里如同打碎了五味瓶,各种感受不时涌上心头,或许是紧张,或许是感念,抑或是憧憬。“至晓回转,不得快熟”,八字引发人无限联想,又点出失眠人的真实情状。两相比较,则看出顾和心理素质稍逊一筹,恐有功利之想充斥心间。另外,王导语诸客“此中亦难得眠处”,语短情长、耐人回味,传达出其忧劳国事、鞠躬尽瘁,常因责任重大而难以成眠的生活状态。同一“难得眠处”,王之与顾,又有不同,读者细心自辨。故事通过对比艺术,生动地刻画了三位名士形象。}

\lettrine{6.17} 庾太尉\myidx{庾亮}风仪伟长\footnote{庾太尉:指庾亮。亮字元规,死后追赠太尉。庾亮(289—340)的敬称。他历仕东晋元、明、成三朝,作为外戚,曾执国政,显赫于朝。的卢:传说中的凶马之名,骑之不利主人。注。风仪:风度仪表。伟长:壮美优秀。},不轻举止,时人皆以为假。亮有大儿\myidx{庾会}数岁\footnote{大儿:庾亮长子,名会,字会宗,小字阿恭。},雅重之质,便自如此,人知是天性。温太真\myidx{温峤}尝隐幔怛之\footnote{温太真:温峤。当时追从姨夫刘琨,在并州为谋主,“琨所凭恃焉”(\CJKunderwave{晋书·温峤传})。建武元年(317)奉刘琨命出使江南,拥戴司马睿即帝位,建立东晋王朝。受司马睿重用,留为散骑常侍,后官至中书令,为东晋名臣。注。怛:恐吓,吓唬。},此儿神色恬然\footnote{恬然:安闲的样子。},乃徐跪曰:“君侯何以为此\footnote{君侯:古时称列侯为君侯。后转为对尊贵者的尊称。}?”论者谓不减亮。苏峻时遇害\footnote{苏峻时:指苏峻起兵攻京师建康时。}。{\fzxk\zihao{6}\textcolor{red}{\CJKunderwave{庾氏谱}曰:“会字会宗,太尉亮长子,年十九,咸和六年遇害。”}} 或云:“见阿恭,知元规\myidx{庾亮}非假。”{\fzxk\zihao{6}\textcolor{red}{阿恭,会小字也。}}

{\cangkai\zihao{5}【评】\CJKunderwave{晋书}记载庾亮子为庾彬,不是庾会,或为传闻异辞。故事涉及名士风度的家族文化积淀和传承问题。魏晋官学衰微,高门豪族家学、私学勃兴,这群知识储备及风度涵养俱佳的士人,期盼自身的龙章凤质传递给下一代。“芝兰玉树,生于阶前”,就是名士们传承家风、光耀门庭心理的集中反映。史载庾亮“风格峻整,动由礼节,闺门之内不肃而成”(\CJKunderwave{晋书}本传)。家族教育严于礼教,并且已经内化成全家上下习惯行为,能够自觉躬行家族传统。基于此,当我们进一步了解到,几岁小儿在温峤探测性的恐吓下,能够神色恬然,动静合礼,也就不足为奇了。冰冻三尺,由来非短,庾亮对子女的风度教育,可从小儿身上反映出来。魏晋士人以独特的教育方式传承民族文化之功,应予肯定。}

\lettrine{6.18} 褚公\myidx{褚裒}于章安令迁太尉记室参军\footnote{褚公:指褚裒。褚公:对褚裒的敬称。褚裒(póu 抔)(303—349),晋康帝皇后之父,朝廷议以“不臣之礼”,力辞执政,而赴外镇。官征北大将军。曾率军三万北伐,败后上疏自贬,忧慨发愤而卒。见\CJKunderwave{晋书·外戚传}。章安:县名,晋属临海郡,在今浙江省。太尉记室参军:太尉府记室参军,掌表章文书等。太尉:此指庾亮。},{\fzxk\zihao{6}\textcolor{red}{案:庾亮\CJKunderwave{启参佐名},裒时直为参军,不掌记室也。}} 名字已显而位微,人未多识。公东出\footnote{东出:向东去。},乘估客船\footnote{估客:贩货的行商。},送故吏数人\footnote{送故吏:为离任长官送行的佐吏。此指送褚裒之吏。},投钱唐亭住\footnote{钱唐亭:“唐”亦作“塘”。供旅客停留食宿的公舍叫亭。因在钱唐县,故称钱唐亭。}。{\fzxk\zihao{6}\textcolor{red}{\CJKunderwave{钱唐县记}曰:“县近海,为潮漂没。县诸豪姓敛钱雇人,辇土为塘,因以为名也。”}} 尔时,吴兴沈为县令\footnote{吴兴:郡名。治所在今浙江湖州。据袁本,“沈”字下脱“充”字。沈充:当时一个县令。},{\fzxk\zihao{6}\textcolor{red}{未详。}} 当送客过浙江,客出\footnote{客出:客到。},亭吏驱公移牛屋下。潮水至,沈令起彷徨,问:“牛屋下是何物人\footnote{何物:什么,什么人。}?”吏云:“昨有一伧父来寄亭中\footnote{伧父:六朝时南人称北方男子为伧。伧父犹北方佬,有轻贱之意。寄:借宿。},{\fzxk\zihao{6}\textcolor{red}{\CJKunderwave{晋阳秋}曰:“吴人以中州人为伧。”}} 有尊贵客,权移之\footnote{权:暂且。}。”令有酒色,因遥问:“伧父欲食饼不?姓何等?可共语。”褚因举手答曰:“河南褚季野\footnote{褚季野:褚裒字。}。”远近久承公名,令于是大遽\footnote{遽:惊慌。},不敢移公,便于牛屋下修刺诣公\footnote{修刺:指写名帖,作通报姓名之用。},更宰杀为馔具\footnote{馔具:饭食,酒食。},于公前鞭挞亭吏,欲以谢惭。公与之酌宴,言色无异,状如不觉。令送公至界。

{\cangkai\zihao{5}【评】有晋以来,士族之南北对立,成为一特殊的社会文化现象。分析主次矛盾,北方士族实为挑起争端的始作俑者。东晋南渡以后,南北士族力量此长彼消,随之导致南人心理优势亦潜滋暗长。河南人褚裒为太尉庾亮参军,江边待发,被南人亭吏呼为伧父,驱若鸡犬,毫无一点尊严可言。但裒言色无异,状如不觉,雅量非凡;被认出后,待若上宾,裒还是波澜不惊。非超级大名士,难以如此。裒与名相谢安性情相投,为安雅重。谢安评价裒曰:“裒虽不言,而四时之气亦备。”(\CJKunderwave{晋书}裒本传)意谓裒外无臧否,而内有是非,因修养到位,就不愿与一般心存偏见、见识短浅的小人计较。}

\lettrine{6.19} 郗太傅\myidx{郗鉴}在京口\footnote{郗太傅:郗鉴,(269—339),字道徽,晋高平金乡(今属山东)人。东晋初官至司徒、进位太尉,位至朝廷三公,故称。明帝时,鉴都督扬州,牵制王敦;成帝时,平祖约、苏峻有功。京口:古城名,今江苏镇江。},遣门生与王丞相\myidx{王导}书\footnote{门生:投靠世族的门客。王丞相:指王导。},求女壻。丞相语郗信\footnote{信:信使,使者。}:“君往东厢,任意选之。”门生归白郗曰:“王家诸郎,亦皆可嘉,闻来觅壻,咸自矜持\footnote{矜持:故意做作,不自然。}。唯有一郎,在东床上坦腹卧\footnote{东床:此指东厢房之床。坦:通“袒”。},如不闻。”郗公云:“正此好\footnote{正:只是。}!”访之,乃是逸少\myidx{王羲之}\footnote{逸少:羲之字。},因嫁女与焉。{\fzxk\zihao{6}\textcolor{red}{\CJKunderwave{王氏谱}曰:“逸少,羲之小字。羲之妻,太傅郗鉴女,名璿,字子房也。”}}

{\cangkai\zihao{5}【评】“东床坦腹”或“东床快婿”等典故出此。太傅郗鉴选婿,对诸郎而言,有利可图,是门当户对的好姻缘。一旦被选中,则等于一脚踏上通往康庄仕途的红地毯。因为涉及自身前程,子弟们咸自矜持、严阵以待,气氛紧张,令人屏息。惟有王羲之东床之上露着肚皮,照样吃喝,对选婿“钦差”视而不见、充耳不闻,似乎对此毫不感兴趣。最后竟被选为乘龙快婿,令弟兄们瞪红眼睛、百思而不得其解!故事正见出魏晋人物品评标准之神髓,乃是重自然天放、率性任真,过分的拘谨、矜持适足损害其质性。王羲之东床坦腹与阮籍青白眼、渊明无弦琴等魏晋风流的轶闻佳话,其本质都是魏晋重自然之时代主题的外在表征。}

\lettrine{6.20} 过江初\footnote{过江:指晋室南渡,建都建康。},拜官舆饰供馔\footnote{拜官:授官。舆饰供馔:大办宴席。舆,多。饰,整治。通“饬”。}。羊曼\myidx{羊曼}拜丹阳尹\footnote{羊曼(274—328):字祖延(一作延祖),东晋泰山南城(在今山东)人。历仕黄门侍郎、尚书吏部郎、晋陵太守。王敦败亡后,代阮孚为丹阳尹。苏峻作乱,城陷被杀。丹阳:郡名,治所在建业。},客来早者,并得佳设\footnote{设:饮馔,饮食。}。日晏渐罄\footnote{晏:迟,晚。罄:尽,空。},不复及精。随客早晚,不问贵贱。{\fzxk\zihao{6}\textcolor{red}{\CJKunderwave{曼别传}曰:“曼字延祖(祖延),泰山南城人。父监(暨),阳平太守。曼颓纵宏任,饮酒诞节,与陈留阮放等号‘兖州八达’(伯)。累迁丹阳尹,为苏峻所害。”}} 羊固\myidx{羊固}拜临海\footnote{羊固:字道安,东晋泰山平阳(在今山东)人。历仕临海太守、黄门侍郎,有清俭之称。善行草书,有名于时。临海:郡名,治所章安县(今浙江临海县)。},竞(竟)日皆美供\footnote{供:饭食,酒食。},虽晚至,亦获盛馔。时论以固之丰华,不如曼之真率。{\fzxk\zihao{6}\textcolor{red}{\CJKunderwave{明帝东宫僚属名}曰:“固字道安,太山人。”\CJKunderwave{文字志}曰:“固父坦,车骑长史。固善草行,著名一时。避乱渡江,累迁黄门侍郎。褒其清俭,赠大鸿胪。”}}

{\cangkai\zihao{5}【评】国人重待客之道,美食、美器相映成趣,贤主、嘉宾相得益彰,围坐之间“契阔谈宴,心念旧恩”,洋溢着欢快而和谐的气氛。石崇“金谷宴游”虽风流千古,因为缺乏诗意内涵徒留后世无形文人的艳羡和正直之士的诟病。羊曼随客早晚,不问贵贱,暗合庄子“齐物”真谛;羊固竟日美供,全力置办,亦是真诚悉心、一视同仁的待客之道。时论强分高下,抬高羊曼而贬低羊固,其出发点当是以为羊曼待客之道出于“自然”,超出羊固的尽力悉心,这是当时人对风流任真的世风的一种理解。其实,待客之道,应量体裁衣,随人之宜,只要是出于真情,气氛和谐、主客俱欢,就值得肯定。无论是精打细算还是倾囊而出,俱属美宴。中华饮食文化之核心在于追求“和”之境界,“五花马,千金裘,呼儿将出换美酒”,难道不是一种难以企及的潇洒境界吗?羊曼与羊固,无高下之别,俱是贤主。}

\lettrine{6.21} 周仲智\myidx{周嵩}饮酒醉\footnote{周仲智:周嵩字仲智,伯仁弟。性狷介。},瞋目还面谓伯仁\myidx{周顗}曰\footnote{瞋目:瞪眼。伯仁:周顗字伯仁。}:“君才不如弟,而横得重名\footnote{横:无缘无故,凭空。}!”须臾,举蜡烛火掷伯仁,伯仁笑曰:“阿奴火攻\footnote{阿奴:表示亲昵的称呼,用于长呼幼,尊呼卑,相当于第二人称代词。},固出下策耳!”{\fzxk\zihao{6}\textcolor{red}{\CJKunderwave{孙子兵法}曰:“火攻有五:一曰火人,二曰火积,三曰火车,四曰火军(库),五曰火队。凡军必知五火之变,故以火攻者,明也。”}}

{\cangkai\zihao{5}【评】周顗、周嵩兄弟二人性情大相径庭。顗以宽厚爱众,雅为世人所推;嵩以才气凌物,甚失孝悌之义。故事记载兄弟之间一次小摩擦:弟嵩心浮气躁,对兄顗的暴得大名心存耿耿,酒后耍疯,举烛火投向兄顗,说明其内心愤懑太深,这次酒后失态,是其潜意识里长期郁结的总爆发。周顗仅以一句充满智慧的玩笑话,轻松化解了极容易演变成拳脚相加的尴尬场面,反而使嵩之激愤行为变得滑稽可笑,大巧运斤之功和宽厚容人之量着实令人钦佩。仅此一点,周顗就比嵩高出远甚。雅量是魏晋名士风范的重要品性,周顗临辱不惊,将雅量演绎得可谓淋漓尽致。}

\lettrine{6.22} 顾和\myidx{顾和}始为扬州从事\footnote{顾和:字群孝,顾荣族子。死后追赠司空,故称。从事:即从事史,州郡属官。},月旦当朝\footnote{月旦:农历每月初一。朝:指朝会。月旦朝会为古代惯例。},未入顷\footnote{顷:时,时候。},停车州门外\footnote{州门:州府之门。}。周侯\myidx{周顗}诣丞相\myidx{王导}\footnote{周侯:周顗。丞相:王导。},历和车边\footnote{历:经过。}。{\fzxk\zihao{6}\textcolor{red}{\CJKunderwave{语林}曰:“周侯饮酒已醉,箸白祫、凭两人,来诣丞相。”}} 和觅虱,夷然不动\footnote{夷然:泰然自若的样子。}。周既过,反还,指顾心曰:“此中何所有?”顾搏虱如故\footnote{搏:捕捉。},徐应曰:“此中最是难测地。”周侯既入,语丞相曰:“卿州吏中有一令仆才\footnote{令仆:尚书令和仆射。泛指宰辅。}。”{\fzxk\zihao{6}\textcolor{red}{\CJKunderwave{中兴书}曰:“和有操量,弱冠知名。”}}

{\cangkai\zihao{5}【评】“扪虱清谈”是魏晋独有的名士风度,顾和门外觅虱,看似不雅,实则纵放,符合名士口味,与王羲之东床坦腹出于同一鹄的。顾和天资纵放,不守常礼,与见到上司毕恭毕敬的一般俗士大异其趣。周顗与顾和二人,并不因地位悬殊而彼此不顾,这是魏晋士人的可爱之处。周侯注意到顾和,对其车边扪虱之举颇感好奇,瞬间产生了精神上的相通和共鸣,好似磁石吸铁一般,二人的心紧紧吸附在一起。“此中何所有?”是进一步的试探、测试;“此中最是难测地”——顾和已然交出了令周顗满意的答卷。语不在多而在精,顾、周之谈,看似痴人说梦,难以索解,实则默契禅宗,暗合拈花微笑,会心得意处则相视无言。顾、周二人,一不惊宠辱,一慧眼识英,俱为不可多得的雅量,可称双美。又,此顾和与本门第十六则中“至晓回转,不得快熟”的顾和,乃同一人。为何前后表现判若两人?案:此则故事当发生在顾和初为王导从事,未被礼遇时,而本门第十六则故事则发生在此后。顾和车边觅虱被周顗发现,并推荐给王导,从而有第十六则的“尔时已被遇”。因前后地位变迁,“在其位,谋其政”,则心态可能会随之变化。}

\lettrine{6.23} 庾太尉\myidx{庾亮}与苏峻\myidx{苏峻}战\footnote{庾太尉与苏峻战:晋成帝咸和二年(327年),苏峻举兵反,次年进逼京城建康,执政庾亮督师与战,晋师败绩。},败,率左右十馀人,乘小船西奔。{\fzxk\zihao{6}\textcolor{red}{\CJKunderwave{晋阳秋}曰:“苏峻作逆,诏亮都督征讨。战于建阳门外,王师败绩。亮于陈(阵)携三弟奔温峤。”}} 乱兵相剥掠\footnote{剥掠:掠夺。苏峻攻陷建康,纵兵大掠。},射\footnote{射:庾亮的左右侍从向乱兵射箭。\CJKunderwave{晋书}所记乃庾亮射。},误中\xpinyin*{柂}工\footnote{柂工:舵工,掌舵的人。},应弦而倒,举船上咸失色分散\footnote{举船:全船。飞箭误中舵工,众人不知此箭是谁射,群情恐慌。}。亮不动容,徐曰:“此手那可使箸贼\footnote{手:射技。那:同“哪”,怎么。用于反诘,意在否定。箸:同“著”,射中。}!”众乃安。

{\cangkai\zihao{5}【评】故事发生在东晋成帝咸和二年(327年)。临危不乱的气度,构成魏晋士人雅量的一个组成方面。庾亮、苏峻两军交战,王师败绩,登舟逃奔。生死未卜之际,又险象环生,庾亮左右侍从放箭误中舵工,引起群情骚动,人人自危。阵脚危乱是军法大忌,处理不好则可能发生内部挤压踩踏的乱局。此刻,庾亮政治家的气度本能地发挥了作用,一句“此手那可使箸贼”,嘲笑中含善意的幽默,有似民间开玩笑说“你这臭手”,紧张气氛骤然间转化为滑稽,人们紧绷的神经得以平复,从而同仇敌忾、顺利脱身。庾亮临难不惊,关键之际呈现了其名士风采,这与其平素高自砥砺、“动由礼节”的严格自我要求不无关系。今人不可因人废言,仅以矫情镇物视之。}

\lettrine{6.24} 庾小征西\myidx{庾翼}尝出未还\footnote{庾小征西:指庾翼。翼为征西将军,其兄亮也为征西将军,故称翼为小征西。},妇母阮\footnote{妇母阮:妻子的母亲阮氏。},是刘万安\myidx{刘绥}妻\footnote{刘万安:刘绥字万安,晋高平人。官骠骑长史。},{\fzxk\zihao{6}\textcolor{red}{\CJKunderwave{刘氏谱}曰:“刘绥妻,陈留阮蕃女,字幼娥。”绥,别见。}} 与女上安陵城楼上\footnote{安陵:当为安陆,地名。晋时为江夏郡治所。}。俄顷,翼归,策良马\footnote{策:驾驶。},盛舆卫\footnote{舆卫:舆从护卫。}。阮语女:“闻庾郎能骑,我何由得见?”妇告翼,{\fzxk\zihao{6}\textcolor{red}{\CJKunderwave{庾氏谱}曰:“翼娶高平刘绥女,字静女。”}} 翼便为于道开卤簿盘马\footnote{卤簿:仪仗队。盘马:跨马盘旋。},始两转,坠马堕地,意色自若。

{\cangkai\zihao{5}【评】庾翼,字稚恭,庾亮弟。其人风仪秀伟,少有经纶大略。他在任尽职,公私充实,又锐意北伐,进位征西将军,称庾小征西。虽为皇亲国戚,却并非养尊处优的纨绔子弟,\CJKunderwave{晋书}本传称其“稚恭慷慨,亦擅雄声”。与桓温惺惺相惜,是一位欲有所作为的士人。故事记述庾翼妻子和丈母娘“春日凝妆上翠楼”,一瞥女婿雄姿英发的身姿,丈母娘听说女婿骑术精湛,想借此见识一下。翼欣然承应,也想在丈母娘面前一展金龟婿的风采。不料运气不佳,刚转了两圈,一时失手,从马背上跌落下地,场面相当滑稽。然翼并不觉得丢面子,照样气宇轩昂。故事从一个侧面折射出晋人的精神气度:敢于张扬个性与才华,为人处世平和而纵放;因充满自信甚至自负,故能从容接纳各种失利与变故。陈孟槐以为(岳母阮氏)“与(女)上城楼上,见庾郎归,欲观能骑,极是佳事。有母若此,堕地何惭,故添佳话”。触及了同样受玄风熏渐的晋世女性的心灵世界,贤母佳婿,可谓双美。又,故事与前则庾亮事有异曲同工之妙,盖是庾家兄弟心理素质有良好的遗传基因,亦未可知。}

\lettrine{6.25} 宣武\myidx{桓温}{\fzxk\zihao{6}\textcolor{red}{桓温。}} 与简文\myidx{司马昱}、太宰\myidx{司马晞}{\fzxk\zihao{6}\textcolor{red}{武陵王晞。}} 共载\footnote{宣武:指桓温。简文:简文帝司马昱。太宰:指武陵王司马晞。晞在晋穆帝时官太宰,有武干。为桓温所忌,简文即位,奏徙新安。},密令人在舆前后鸣鼓大叫\footnote{舆:车。},卤簿中惊扰\footnote{卤簿:仪仗队。}。太宰惶怖,求下舆;顾看简文,穆然清恬\footnote{穆然清恬:端庄安静。}。宣武语人曰:“朝廷间故复有此贤\footnote{故复:仍然,还。}。”{\fzxk\zihao{6}\textcolor{red}{\CJKunderwave{续晋阳秋}曰:“帝性温深,雅有局镇。尝与桓温、太宰武陵王晞同乘,至板桥,温密敕令无因鸣角鼓噪,部伍并惊驰。温佯骇异,晞大震,帝举止自若,音颜无变。温每以此称其德量。故论者谓温服惮也。”}}

{\cangkai\zihao{5}【评】魏晋人物品评,人们常以雅量窥测、蠡定人物精神品格之高下。有时,为了深入考察人物之内在深蕴,会故意制造一些险象进行测试。桓温似对恶作剧乐此不疲,\CJKunderwave{方正}门弹射刘惔枕即是,这次又以简文为考察对象。测试结果发现,简文不为惊险所动,穆然清恬,与兄司马晞形成鲜明对比,深为桓温折服。这种突然袭击式的测试,因当事人事先毫无精神准备,故其应激反应当是真实心理素质的折射,结果较为真实。\CJKunderwave{世说}多记简文痴言痴行,谢安甚至称其为惠帝之流,这是见其结果而不问动因。从主观识见言,简文实大智若愚,非惠、安二帝之弱智与白痴,所能望其项背。史载其清虚寡欲,尤善玄言。刘孝标注引\CJKunderwave{续晋阳秋}云:“性温深,雅有局镇。”惜生非其时,五十二岁熬上帝座,不到一年忧惧而死。简文一生历经元、明、成、康、穆、哀、海七帝之朝,目睹政治动荡及血腥杀戮无算,在长期压抑不得志的生活中,其“痴”乃是其性情、心志发生一定畸变的结果,也在情理之中。}

\lettrine{6.26} 王劭\myidx{王劭}、王荟\myidx{王荟}共诣宣武\myidx{桓温}\footnote{王劭、王荟:丞相王导第五子及幼子。宣武:桓温。},{\fzxk\zihao{6}\textcolor{red}{\CJKunderwave{劭荟别传}曰:“劭字敬伦,丞相导第五子。清贵简素,研味玄赜。大司马桓温称为‘凤𩿿’。累迁尚书仆射、吴国内史。荟字敬文,丞相最小子。有清誉,夷泰无竞。仕至镇军将军。”}} 正值收庾希\myidx{}家\footnote{收:拘捕。庾希:庾冰长子。}。{\fzxk\zihao{6}\textcolor{red}{\CJKunderwave{中兴书}曰:“希字始彦,司空冰长子。累迁徐、兖二州刺史。希兄弟贵盛,桓温忌之,讽免希官。遂奔于暨阳。初,郭璞筮冰子孙必有大祸,唯固三阳可以有后。故希求镇山阳,弟友为东阳,希自家暨阳。及温诛希、弟柔、倩闻希难,逃于海陵,后还京口聚众,事败,为温所诛。”}} 荟不自安,逡巡欲去\footnote{逡巡:有所顾虑而徘徊。}。劭坚坐不动,待收信还\footnote{信:使者。},得不定\footnote{得不定:谓得知事未定。},乃出。论者以劭为优。

{\cangkai\zihao{5}【评】桓温芟除异己,肆行杀戮,实为其最终篡逆铲平道路。故事记载温搜捕外戚庾家,二王兄弟心情是相当复杂的。首先,东晋时王、庾、桓、谢四大家族为权力而相互争斗,当年庾亮兄弟压制王导,两家仇隙甚深。而今诸庾难逃果报,王家自是心中大快;其次,人生无常,焉知血腥屠杀下一个不会降临到自己头上?兔死狐悲之情,于是油然心生。相形之下,劭、荟兄弟二人,一善良软弱,一刚毅冷静。在“政失准的”的混乱社会,无情的政治斗争不怜悯弱者的眼泪,而更欣赏强者的铁血手腕。王劭坚坐不动,毫不动情,具备参与政治角逐的基本心理素质。论者以劭为优,乃是时代高压政治风气影响使然。}

\lettrine{6.27} 桓宣武\myidx{桓温}与郗超\myidx{郗超}议芟夷朝臣\footnote{桓宣武:指桓温。郗超:字嘉宾,此时郗超为桓温谋主,参与密谋,权重一时。芟夷:铲除。},条牒既定\footnote{条牒:删除朝臣的方案。},其夜同宿。{\fzxk\zihao{6}\textcolor{red}{\CJKunderwave{续晋阳秋}曰:“超谓温雄武,当乐推之运,遂深自委结。温亦深相器重,故潜谋密计,莫不预焉。”}} 明晨超(起),呼谢安\myidx{谢安}、王坦之\myidx{王坦之}入\footnote{明晨超:据袁本,“超”作“起”,是。谢安、王坦之:简文帝初立,谢为侍中,王为左卫将军,俱是朝廷重臣。},掷疏示之\footnote{疏:条陈,指上文的“条牒”。}。郗犹在帐内。谢都无言,王直掷还,云:“多\footnote{多:太多了。}。”宣武取笔欲除\footnote{除:去掉。},郗不觉窃从帐中与宣武言。谢含笑曰:“郗生可谓入幕宾也\footnote{入幕宾:幕宾,幕府宾客。入幕宾在这里是双关语。}。”{\fzxk\zihao{6}\textcolor{red}{帐一作帷。}}

{\cangkai\zihao{5}【评】故事以简练之笔,刻画桓温、郗超、王坦之、谢安四人形象,其性情声口,跃然纸上:桓温老辣专横,郗超为虎作伥,坦之率真急躁,谢安镇静持重。谢安显然是故事的第一主角,他在危急时刻收放自如,以机智幽默的调侃缓解紧张气氛,为自己争取思考的时间。谢安之大智大勇,与其平素高自砥砺有密切关系。故事还可见\CJKunderwave{世说}的语言艺术。“入幕宾”是双关语,调笑中实含讥讽,谓郗超既是幕僚,参与机要;又登堂入室,宿桓温帐中。又,郗超字嘉宾,此处“宾”字关涉“嘉宾”与“宾客”二义。}

\lettrine{6.28} 谢太傅\myidx{谢安}盘桓东山时\footnote{谢太傅:谢安。盘桓:逗留。此指隐居东山事。东山:山名,在会稽上虞县。谢安出仕前,曾隐居东山。},与孙兴公\myidx{孙绰}诸人泛海戏\footnote{孙兴公诸人:指孙绰、王羲之、许询等人。泛海:泛舟海上。}。{\fzxk\zihao{6}\textcolor{red}{\CJKunderwave{中兴书}曰:“安,元居会稽,与支道林、王羲之、许询共游处,出则渔弋山水,入则谈说属文,未尝有处世意也。”}} 风起浪涌,孙、王\myidx{王羲之}诸人色并遽\footnote{遽:惊慌。},便唱使还\footnote{唱:高呼,高叫。}。太傅神情方王\footnote{方王:“王”通“旺”。此指情绪好,兴致高。},吟啸不言\footnote{吟:吟咏。啸:撮口发出长而清越的声音,魏晋士大夫的一种习惯。}。舟人以公貌闲意说\footnote{貌闲意说:神情闲适、愉悦。说,通“悦”。},犹去不止。既风转急,浪猛,诸人皆喧动不坐。公徐云:“如此,将无归\footnote{将无:莫非,还是。表示委婉语气。}?”众人即承响而回\footnote{承响:应声。}。于是审其量,足以镇安朝野\footnote{镇安:镇抚安定。}。

{\cangkai\zihao{5}【评】先有不世之人,后有不世之功。古来之成大事者,无不主动觅险、历险,渴望饱览险峰上的无限风光,从中感受到征服的乐趣,以陶冶情操、砥砺胸襟。故事历来为人传诵,因其准确地传达出谢安超一流政治家的优良心理素质。谢安历经人生风浪,均能处乱不惊,从容化解,与其平素积极挑战苦难的人生态度相关。故事以对比映衬手法烘托主人公谢安:风起浪涌之际,孙、王诸人大惊失色、喧动不安,与谢安吟啸自若、貌闲意悦形成极鲜明的表情反差。此是一般名士与杰出政治家的分野。}

\lettrine{6.29} 桓公\myidx{桓温}伏甲设馔\footnote{桓公:桓温。伏甲:埋伏甲兵。设馔:安排宴席。},广延朝士,因此欲诛谢安\myidx{谢安}、王坦之\myidx{王坦之}\footnote{欲诛谢安、王坦之:谢、王为简文帝倚重的大臣。桓温欲倾晋室,故欲先诛除大臣。}。{\fzxk\zihao{6}\textcolor{red}{\CJKunderwave{晋安帝纪}曰:“简文晏驾,遗诏桓温依诸葛亮、王导故事。温大怒,以为黜其权,谢安、王坦之所建也。入赴山陵,百官拜干(于)道侧,在位望者,战栗失色。”或云自此欲杀王、谢。}} 王甚遽\footnote{遽:惊慌。},问谢曰:“当作何计\footnote{计:打算。}?”谢神意不变,谓文度曰\footnote{文度:王坦之字文度。}:“晋阼存亡,在此一行。”相与俱前。王之恐状,转见于色\footnote{转:渐渐。}。谢之宽容\footnote{宽容:沉着、从容不迫的神态},愈表于貌。望阶趋席,方作洛生咏\footnote{洛生:洛阳书生吟咏时语音重浊,谢安有鼻疾,语重浊,通于洛生咏。后来的名士亦仿效其咏,不像,就用手掩鼻而吟。},讽“浩浩洪流”\footnote{讽:背诵。浩浩洪流:嵇康\CJKunderwave{赠秀才入军}诗:“浩浩洪流,带我邦畿。”}。桓选(惮)其旷远\footnote{桓选其旷远:据袁本,“选”当为“惮”。惮,怕。旷远,指心胸旷达高远。},乃趣解兵\footnote{趣(cù):赶紧。解兵:撤掉伏兵。}。{\fzxk\zihao{6}\textcolor{red}{按宋问(明)帝\CJKunderwave{文章志}曰:“安能作洛下书生咏,而少有鼻疾,语音浊。后名淬(流)多其咏,菩(弗)能及,手掩鼻而吟焉。桓温止新亭,大陈兵卫,呼安及坦之,欲于坐害之。王入失厝,倒执手版,汗流沾衣。安神姿举动不异于常,举目遍历温左右卫士,谓温曰:‘安闻诸侍(侯)有道,守在四邻,明公何须壁间箸阿蝫(堵)辈?’温笑曰:‘正自不能不尔。’于是矜庄之心顷尽,命却左右,促燕行觞,笑语移日。”}} 王、谢旧齐名,于此始判优劣。

{\cangkai\zihao{5}【评】桓温望简文临终禅位于己,但遗诏却以桓温依诸葛亮、王导故事。温大失所望,以为谢安、王坦之从中作梗。盛怒之下,设鸿门宴,伏甲壁间,欲杀王、谢,从而引出王、谢的登场。沧海横流,方显出英雄本色。谢安于个人性命千钧一发、晋室存亡危在旦夕之际,举重若轻,从容讽诵洛下书生咏。“浩浩洪流,带我邦畿”,诗句气象博大,风度旷达,表征了政治家的心胸,超越眼前性命之忧,引人入浩瀚宇宙联想。不料桓温反为其风度折服,冰释前嫌。可见,谢安身上有一种不怒自威的精神震撼力,桓温亦有名士风度的可爱一面,“异质同构”,一时引发心灵共鸣,止息了盛怒与杀机。刘辰翁曰:“桓自可人”,甚是。这种化解灾难的方式极富戏剧性,再一次表征了魏晋士人的情性和对诗意人生境界的高远追求。}

\lettrine{6.30} 谢太傅\myidx{谢安}与王文度\myidx{王坦之}共诣郗超\myidx{郗超}\footnote{谢太傅:谢安。王文度:王坦之。郗超:参看本门27、29两则。},日旰未得前\footnote{旰:晚。前:指面见。}。王便欲去,谢曰:“不能为性命忍俄顷\footnote{俄顷:一会儿。}?”{\fzxk\zihao{6}\textcolor{red}{超得宠桓温,专杀生之威。}}

{\cangkai\zihao{5}【评】刘辰翁评曰:“与前泛海各得自在”,可谓一语中的。泛海行舟是主动求险,共诣郗超是被动化险。看似两个极端,实有相通之处。主动求险是为磨炼意志,表征的是不惧危难的勇气;被动脱险是为保全生命以图将来,呈现的是能屈能伸的忍耐。二者是雅量在不同场合下的体现,其本质都是名士风度。政治斗争有时令人无奈、气短,为了将来,连谢安也不得不低下高贵的头颅,与自己平时不齿的政治投机客周旋、俯仰。又如为世人景仰的大文人苏东坡,因罹“乌台诗案”,被当权派呼来喝去,“被驱不异犬与鸡”。正如余秋雨在\CJKunderwave{苏东坡突围}中写道:小人牵着大师,大师牵着历史!谢安、苏东坡们因承受了地狱、炼狱的折磨,从而成就了富于诗意的“文化苦旅”!}

\lettrine{6.31} 支道林\myidx{支遁}还东\footnote{支道林:支遁字道林,东晋僧人。为东晋名僧,善玄理,是当时佛学“般若学”的代表人物,多才艺,长于草隶。与王洽、刘惔、殷浩、许询、郗超、王羲之、谢安等名流游好。常:同“尝”,曾经。注。还东:回到会稽去。东晋侨姓高门多在会稽一带广治田宅产业,常在此流连享乐。由于会稽一带处于建康之东,故时人常以东指称会稽。},{\fzxk\zihao{6}\textcolor{red}{\CJKunderwave{高逸沙门传}曰:“遁为哀帝所迎,游京邑久,心在故山,乃拂衣王都,还就岩穴。”}} 时贤并送于征虏亭\footnote{征虏亭:亭名,在建康石头坞,传说是征虏将军谢安所建,后成为送别之所。}。{\fzxk\zihao{6}\textcolor{red}{\CJKunderwave{丹阳记}曰:“太安中,征虏将军谢安立此亭,因以为名。”}} 蔡子叔\myidx{蔡系}前至\footnote{蔡子叔:蔡系,字子叔,蔡谟次子。前至:先到。},坐近林公;{\fzxk\zihao{6}\textcolor{red}{\CJKunderwave{中兴书}曰:“蔡系字子叔,济阳人,司徒谟弟(第)二子。有文理,仕至抚军长史。”}} 谢万石\myidx{谢万}后来\footnote{谢万石:谢万字万石,太傅谢安弟。},坐小远\footnote{小:稍微。}。蔡暂起,谢移就其处。蔡还,见谢在焉,因合褥举谢掷地\footnote{褥:坐垫。},自复坐。谢冠帻倾脱\footnote{冠帻:帽子和包头巾。},乃徐起,振衣就席\footnote{振衣:抖去衣上的灰尘。},神意甚平,不觉瞋沮\footnote{瞋沮:瞋怒沮丧。瞋,恼怒。}。坐定,谓蔡曰:“卿奇人,殆坏我面\footnote{殆:差点儿,几乎。}。”蔡答曰:“我本不为卿面作计\footnote{作计:打算,考虑。}。”其后二人俱不介意。

{\cangkai\zihao{5}【评】刘辰翁评曰:“送一僧何至争近至此,子叔小人,语更深狠。”是未能深究本原。东晋以来,玄言清谈重心转向佛教义理,当代名僧,既理趣符老庄,风神类谈客,沙门高僧成为公卿名流的坐上宾,甚至受国师之礼。流风所及,士人翕然相尚,托情道味、礼待法师为标榜之资。\CJKunderwave{弘明集}载“支之特秀,领握玄标,大业冲粹,神风清萧”。见出时人对支道林的推尊。故事中,谢万、蔡系争座次,可见时代风尚的嬗变。谢万出自高华门第,矜豪傲物,善自炫耀,岂屑与蔡系争吵而损其名士风度?谓谢万雅量未尝不可,但其自命不凡的门阀因素更占上风。}

\lettrine{6.32} 郗嘉宾\myidx{郗超}钦崇释道安\myidx{道安}德问\footnote{郗嘉宾:郗超小字嘉宾。郗超:任桓温大司马,深得信任,立简文为帝后,迁中书侍郎,实代桓温监督朝廷而权重当时。在直:在宫中值班。钦崇:敬重。道安:东晋高僧,曾师于佛图澄。晋孝武时,避乱襄阳,后入长安。一生讲学译经,以道自任,艰苦卓绝,鸠摩罗什谓是东方圣人。德问:道德声望。},{\fzxk\zihao{6}\textcolor{red}{\CJKunderwave{安和上传}曰:“释道安者,常山薄柳人。本姓卫,年十二作沙门。神性聪敏,而貌至陋,佛图澄甚重之。值石氏乱,于陆浑山木食修学,为慕容俊所逼,乃住襄阳。以佛法东流,经籍错谬,更为条章,标序篇目,为之注解。自支道林等皆宗其理。无疾卒。”}} 饷米千斛\footnote{饷:馈赠。斛:量器名。古代十斗为一斛。},修书累纸\footnote{修书:写信。累纸:好几张纸。累:重叠。},意寄殷勤。道安答直云\footnote{直:只,只是。}:“损米\footnote{损米:承蒙赠米。书札套语。},愈觉有待之为烦\footnote{有待:有所依赖。谓人须依靠物质才能生活,故称有待。语见\CJKunderwave{庄子·逍遥游}。}。”

{\cangkai\zihao{5}【评】郗超一家父子三代信仰各不相同,最为典型地体现了魏晋精神自由和思想多元化的特点。祖郗鉴博览经籍,以儒雅著名,是一儒士;父郗愔栖心绝谷,事天师道;而超奉佛甚勤,\CJKunderwave{弘明集}载超之佛学名篇\CJKunderwave{奉法要}。三代之思想信仰同时并存,在此前的汉代社会是很难想象的,又恰与魏晋玄学之大体演变轨迹相合。道安为佛教史上开辟新纪元式的高僧。汤用彤先生誉之为“能使佛教有独立之建设,艰苦卓绝,真能发挥佛陀之精神,而不全借清谈之浮华者,实在弥天释道安法师。”(\CJKunderwave{汉魏两晋南北朝佛教史})郗超赠米千斛,修书累纸,可见钦崇殷勤,视若偶像。道安复信不卑不亢,言简意赅,超脱尘累。“有待”一词,本为道家庄子\CJKunderwave{逍遥游}、\CJKunderwave{齐物论}等篇中用语,道安引用以比拟、配合佛教之义,是谓“格义”,为晋初兴起的佛家布道讲学之捷径。}

\lettrine{6.33} 谢安南\myidx{谢奉}免吏部尚书\footnote{谢安南:指谢奉。奉曾作安南将军。吏部尚书:吏部最高行政长官。},还东\footnote{还东:指免官后回会稽山阴。};{\fzxk\zihao{6}\textcolor{red}{\CJKunderwave{晋百官名}曰:“谢奉字弘道,会稽山阴人。”\CJKunderwave{谢氏谱}曰:“奉祖端,散骑常侍。父凤,丞相主簿。奉历安南将军、广州刺史、吏部尚书。”}} 谢太傅\myidx{谢安}赴桓公\myidx{桓温}司马\footnote{谢太傅:谢安。司马:军府属官,掌管兵事。},出西\footnote{出西:往西。},相遇破冈\footnote{破冈:地名。即破冈渎。水渠名,在建康东,三国时开凿。}。既当远别,遂停三日共语。太傅欲慰其失官,安南辄引以他端\footnote{引以他端:谓引开谈别的事,避免谈罢官之事。}。虽信宿中涂\footnote{信宿中涂:途中连宿两夜。涂,通“途”。},竟不言及此事。太傅深恨在心未尽,谓同舟曰:“谢奉故是奇士。”

{\cangkai\zihao{5}【评】宗白华先生说:“晋人向外发现了自然,向内发现了自己的深情。”(\CJKunderwave{美学散步})晋人虽然也重立功,想光耀门第,但多又能同时脱略形累,希心自然,为自己保留一块精神的后花园。故晋人官场失意,因另有心灵寄托,并不显得多么苦闷无端,反能促成一段诗意人生。因失意而诗意,是幸还是不幸?只有当事人的内心感受最真实。谢奉免官,无一丝惆怅意绪,听从故乡会稽的召唤,命驾便归。谢安与其三日共语,欲好言相劝竟无从置喙,奉圆融自足之心态,令千载以下之人,想见其风采。人或目之以“矫情”,实际上能令东晋第一号心理大师谢安称为“奇士”,窥不出任何破绽,这份“矫情”也该到炉火纯青的地步了吧!}

\lettrine{6.34} 戴公\myidx{戴逵}从东出\footnote{戴公:戴逵(326—396),字道安,东晋谯郡铚县(今安徽宿县西南)人。善鼓琴,精绘画,信奉佛教。从东出:从会稽往西。},谢太傅\myidx{谢安}往看之\footnote{谢太傅:谢安。}。谢本轻戴,见,但与论琴书。戴既无吝色\footnote{既:竟然。吝色:不乐意的神色。},而谈琴书愈妙。谢悠然知其量\footnote{悠然:深远貌,慢慢地。量:度量,雅量。}。{\fzxk\zihao{6}\textcolor{red}{\CJKunderwave{晋安帝纪}曰:“戴逵字安道,谯国人。少有清操,恬和通任,为刘真长所知。性甚快畅,泰于娱生。好鼓琴,善属文,尤乐游燕,多与高门风流者游。谈者许其通隐。屡辞征命,遂箸高尚之称。”}}

{\cangkai\zihao{5}【评】戴逵并非“形在江海之上、心存魏阙之下”的假道学,而是栖迟衡门、琴书相伴的真名士。朝廷三征而三不至,足见其不事王侯、高情远遁的决心。谢安虽亦官亦隐,但却轻视那些“三径就荒”的所谓纯粹隐士,这大概就是人的多面性吧。安仅以隐逸之流视戴。谢但论琴书小技,戴也将计就计,并不计较,亦不急于分辩、表现,而以一种宠辱不惊的态度谈论琴书,精彩的见解从唇齿间流出,见其胸中有丘壑,腹内能乘船。试想,戴逵对朝廷征聘尚不惊心,又如何能对一时的被误解心存耿耿呢?戴逵最终打动谢安,靠的正是大彻大悟后悠然的人生心态。}

\lettrine{6.35} 谢公\myidx{谢安}与人围棋\footnote{谢公:谢安。},俄而谢玄\myidx{谢玄}淮上信至\footnote{俄而:一会儿。谢玄:车骑:此指谢玄,谢安侄,死后追赠车骑将军。淮上:指淮水一带。信:信使。}。看书竟\footnote{书:书信。},默然无言,徐向局\footnote{徐向局:从容地转向棋局。}。客问淮上利害\footnote{淮上利害:淮水之上的胜负。383年,前秦苻坚大举南侵,企图灭晋,布阵淮河、淝水之间。谢安为征讨大都督,派遣其弟谢石、侄谢玄征讨,于淝水大败苻坚。此即历史上著名的淝水之战。},答曰:“小儿辈大破贼\footnote{小儿辈大破贼:淝水之战,谢安派遣其弟谢石、侄谢玄、子谢琰,各任将领,统军北上。故称谢玄等为“小儿辈”。}。”意色举止,不异于常。{\fzxk\zihao{6}\textcolor{red}{\CJKunderwave{续晋阳秋}曰:“初,苻坚南寇,京师大震。谢安无体(惧)色,方命驾出墅,与兄子玄围棋。夜还乃处分,少日皆办。破贼又无喜容。其高量如此。”\CJKunderwave{谢车骑传}曰:“氐贼苻坚,倾国大出,众号百万。朝廷遣诸军距之,凡八万。坚进屯寿阳。玄为前锋都督,与从弟琰等选精锐决战,射伤坚,俘获数万计,得伪辇及云母车,宝器山积,锦罽万端,牛、马、驴、骡、驼十万头。”}}

{\cangkai\zihao{5}【评】\CJKunderwave{晋书}记载此事大体相同而稍详,其后又有“既罢,还内,过户限,心喜甚,不觉屐齿之折,其矫情镇物如此”之句。\CJKunderwave{雅量}门中,大概只有顾雍丧子之际所表现出来的巨大忍耐力堪与此相比,二者俱被视为雅量的典范。美国著名汉学家马瑞志说过,“‘雅量’包括对面部、口头或者身体的任何一个部位表现出的忧虑、恐惧、兴奋或欢乐的情绪的最轻微暗示的隐藏。”(\CJKunderwave{〈世说新语〉的世界})但现代心理学的研究表明,人的心灵世界分为意识和无意识两个部分。人的所谓雅量,大体相当于意志力品质,只能控制意识的层面,而无意识的层面,还是会于不经意间流露出来。正如精神分析学家弗洛伊德指出的:任何五官健全的人必定知道他不能保存秘密。如果他的嘴唇紧闭,他的指尖会说话,甚至他身上的每一个毛孔都会背叛他。这就解释了,何以谢安过户折屐齿、顾雍掐掌流血了。看来,谢安、顾雍这样心理素质极好的人,也无法逃脱心理规则的支配。其实,谢安、顾雍等中古名士之雅量为后人折服钦佩,绝非一句“矫情镇物”所能含赅。他们的巨大忍耐中,显示了理性与非理性、意识与潜意识的强力冲撞,展现了一个深邃、幽邈的精神世界,具有极强烈的情感张力。}

\lettrine{6.36} 王子猷\myidx{王徽之}、子敬\myidx{王献之}曾俱坐一室\footnote{王子猷、子敬:王羲之二子。},上忽发火,子猷遽走避\footnote{遽:惊慌。走:奔。},不惶取屐\footnote{不惶:同“不遑”。来不及。屐:木屐,底上有齿的木底鞋。};{\fzxk\zihao{6}\textcolor{red}{\CJKunderwave{晋百官名}曰:“王徽之,字子猷。”\CJKunderwave{中兴书}曰:“徽之,羲之弟(第)五子,卓荦不羁,欲为傲达。仕至黄门侍郎。”}} 子敬神色恬然\footnote{恬然:安闲的样子。},徐唤左右扶凭而出\footnote{扶凭:扶持,搀扶。},不异平常。{\fzxk\zihao{6}\textcolor{red}{\CJKunderwave{续晋阳秋}曰:“献之虽不修常贯,而容止不妄。”}} 世以此定二王神宇\footnote{神宇:胸怀气量。}。

{\cangkai\zihao{5}【评】受世风及家风熏染,王家子弟世族门阀意识强烈,上演了许多滑稽可笑的丑剧、闹剧。在王献之的头脑中,此症尤其根深蒂固。房间起火,性命交关,王徽之尚能认清形势,灵活变通,跣足走避,可见还属常人心态;献之放不下名士的臭架子,宝贵生命抵不上名士的矜持派头,在正常人看来好像有精神障碍。“徐唤左右扶凭而出”,几字写出其装腔作势的丑态。上个世纪七十年代中期,笔者故乡海城大地震,事后听说,因灾难深夜猝至,不少人竟从被窝中钻出,赤身裸体逃向室外,虽于颜面有些尴尬,但毕竟捡回了一条宝贵的生命,死里逃生的老百姓事后谈论,也并未觉得有何不妥。献之此举,与雅量无涉,是一出喜剧。所谓喜剧,用鲁迅的话说,就是把人生无意义的东西撕碎给人看,其矛盾双方并不构成有力量的对抗,只能博人一笑而已。当然,世易时移,今人之评价标准与刘宋之考量尺度已相去甚远,不可同日而语。临川列此入雅量门,可见当日世风。}

\lettrine{6.37} 苻坚\myidx{苻坚}游魂近境\footnote{苻坚:前秦君主。游魂:似鬼魂游动不定。此为对敌人侵扰活动的蔑称。},{\fzxk\zihao{6}\textcolor{red}{坚,别目(见)。}} 谢太傅\myidx{谢安}谓子敬\myidx{王献之}曰\footnote{谢太傅:谢安。子敬:王献之。}:“可将当轴,了其此处\footnote{“可将当轴”二句:谓及我执政之时,了结苻坚于边境。当轴:谓掌握权力,位于中枢地位。}。”

{\cangkai\zihao{5}【评】诸家皆以为故事难以索解,试尽力揣摩之。献之尝为太傅长史,故发此语。谢安以社稷基石自命,以驱除鞑虏、克济时艰的历史使命自期。“可将当轴,了其此处”,意谓在其执政任上,了结苻坚于边境之上,不把难题推给后来人。刘辰翁曰:“谓及我在位时攻之。自任吞虏。”评价恰当。这种敢做敢当的勇气,在中国士人身上一脉传承。现代著名作家、学者朱自清先生,曾任清华大学中文系主任、兼图书馆馆长,在其馆长职位卸任前,将一位不合格而难缠的馆员开除职务,决不明哲保身而给下任领导留下难题,在当时传为美谈。文弱的外表下,蕴藏着“虽千万人吾往也”的浩然之气。朱自清先生开除馆员,虽不能与谢安的国之大计相提并论,但追根溯源,二者精神相近,其背后传达出士大夫的历史责任感和社会担当意识,都可谓雅量。}

\lettrine{6.38} 王僧弥\myidx{王珉}、谢车骑\myidx{谢玄}共王小奴\myidx{王荟}许集\footnote{王僧弥:王珉字季琰,小字僧弥,晋中领军王洽子,丞相王导孙。谢车骑:指谢玄,死后赠车骑将军。王小奴:王荟字敬文,小字小奴,晋丞相王导子。许:处所。},{\fzxk\zihao{6}\textcolor{red}{王珉、谢玄,并已见。小奴,王荟小字也。}} 僧弥举酒劝谢云:“奉使君一觞\footnote{使君:汉以后对州郡长官的尊称。谢玄曾为刺史,故称使君。觞:盛酒的杯子。}。”谢曰:“可尔\footnote{可尔:应该这样做。受人敬酒,不逊谢而语气倨傲如此,故王珉勃然大怒。}。”{\fzxk\zihao{6}\textcolor{red}{谢玄曾为徐州,故云使君。}} 僧弥勃然起\footnote{勃然:突然。},作色曰:“汝故是吴兴溪中钓碣耳\footnote{故:本来。吴兴:郡名,治所在乌程(今浙江吴兴县)。钓碣:钓鱼的羯奴。碣,通“羯”。谢玄喜好钓鱼,他小字羯,与碣同音,故此为双关语。},何敢诪张\footnote{诪张:强横、跋扈。}!”{\fzxk\zihao{6}\textcolor{red}{玄叔父安,曾为吴兴,玄少时从之游,故珉云然。}} 谢徐抚掌而笑曰:“卫军\footnote{卫军:指王荟。荟死后赠卫军将军。},僧弥殊不肃省\footnote{殊:颇,甚。肃省:严肃省察。谢玄于此直呼王珉小字,以示回击。},乃侵陵上国也\footnote{侵陵:侵犯欺凌。上国:春秋时称中原诸侯国为上国,与边远之国相对而言。后亦称地位高、实力强的诸侯国。}。”

{\cangkai\zihao{5}【评】故事具体发生于何时,已难查考。但大体可以推知,当在谢玄初克苻坚之后。曾几何时,王氏家族江左独步,然随着谢安拜相,谢氏子侄屡建战功,昔日高门王氏也不得不对“新出门户”抛出求好的橄榄枝。人世代谢如花开花落,英雄豪杰也只能徒唤奈何。三人中,王荟为王导子,于谢玄为长辈,王僧弥为王导孙,与谢玄同辈。王僧弥敬酒已是交好的信号,谢玄傲慢的回答刺痛了王氏子弟敏感的神经。僧弥勃然作色,语近丑诋,想见其声色俱厉;谢玄抚掌而笑,慢呼长者,“侵凌上国”云云,含自恃尊贵之调侃。临川以为,谢玄出之以戏谑语,固足称为雅量,实则是东晋四大家族中王、谢子弟间的一场心理较量。}

\lettrine{6.39} 王东亭\myidx{王珣}为桓宣武\myidx{桓温}主簿\footnote{王东亭:指王珣,王导的孙子。桓宣武:桓温。主簿:官名。古代中央或地方郡县所设属官,负责文书簿籍,掌管印鉴等。},既承藉有美誉\footnote{承藉:凭借。此指王珣出身琅邪王氏,凭此而有美誉。},公甚欲(敬)其人地\footnote{甚欲其人地:“欲”当为“敬”,据沈剑知校本。人地:人的才能和门第。},为一府之望\footnote{府:指大司马府。望:仰望、崇敬的人。}。初见谢失仪\footnote{见谢:向桓温致谢。失仪:有失礼仪。},而神色自若。坐上宾客即相贬笑\footnote{贬笑:贬低讥笑。},公曰:“不然。观其情貌,必自不凡,吾当试之。”后因月朝阁下伏\footnote{月朝:每月初一日下属对长官的朝拜。阁下伏:拜伏于衙署前。},公于内走马直出突之\footnote{走马:驰马。突:冲撞。},左右皆宕仆\footnote{宕仆:因站立不稳而倒下。},而王不动。名价于是大重\footnote{名价:声价。},咸云:“是公辅器也\footnote{公辅器:指做三公和丞相的才具。}。”{\fzxk\zihao{6}\textcolor{red}{\CJKunderwave{续晋阳秋}曰:“珣初辟大司马掾,桓温至重之,常称:‘王掾必为黑头公,未易才也。’”}}

{\cangkai\zihao{5}【评】王珣为王洽子,王导孙,与谢玄俱为桓温掾属。温预测二人皆未易才,当位登公辅。魏晋人物品评,由汉末之重道德、气节等内在精神性因素,转向对人物形貌、风神等外在气质之开掘。桓温通过对王珣家族出身、自然情貌等因素综合的衡量,得出了“必自不凡”的结论。桓温一生,品人无数;察名验实,若合符契。虽在行伍之间,而深得魏晋风流之神髓。其品评之法,大体有二:一为察言观色,一为制造惊险,突击考量。本门记述其考量简文,即用第二法。王珣则经受了二法的综合考验,终为众人折服。可见其砥砺有素,故能传承一代家风。}

\lettrine{6.40} 太元末\footnote{太元:晋孝武帝司马曜年号(376—396)。},长星见\footnote{长星:彗星。古人认为,长星出现为不吉之兆。见:现,出现。},孝武\myidx{司马曜}心甚恶之\footnote{孝武:指晋孝武帝司马曜。}。{\fzxk\zihao{6}\textcolor{red}{徐广\CJKunderwave{晋纪}曰:“泰元二十年九月,有蓬星如粉絮,东南行,历须(婺)女至央(哭)星。”案:泰元末,唯有此妖,不闻长星也。且汉文八年,有长星出东方。文颖注曰:“长星有光芒,或竟天,或长十丈,或二三丈,无常也。”此星见,多为兵革事。此后十六年,文帝乃崩。盖知长星非关天子,\CJKunderwave{世说}虚也。}} 夜,华林园中饮酒\footnote{华林园:宫苑名。西晋时洛阳有华林园。东晋就三国吴旧宫苑建园。},举杯属星云\footnote{属:通“嘱”,劝请。}:“长星,劝尔一杯酒,自古何时有万岁天子!”

{\cangkai\zihao{5}【评】晋人发现了外在于人的自然,有深沉而清醒的宇宙意识。他们将人生和命运放在无穷宇宙的大坐标系上思考,故能超越有限人世的形累,摆脱礼教世俗的枷锁,“对宇宙人生体会到至深的无名的哀感”(宗白华\CJKunderwave{美学散步})。简文之濠濮间想,孝武之举杯属星,均表征了晋人之洒脱、轻盈,而又交织着深沉悲悯的宇宙观、生命观。两晋权力交接,有如走马;萧墙之祸,史不绝书。政治上的最动荡、最混乱,孕育了士人精神的最热烈、最浓情。孝武末年长星屡见,被认为是不祥之兆。春花秋月,家国山河,这一切很快就要易主,一国之君,眼看王朝行将就木却无能为力;小楼东风,苍黄辞庙,才是自己最切实的宿命。故只能以这种极端的方式,驱遣弥漫胸中的恐惧与哀愁。}

\lettrine{6.41} 殷荆州\myidx{殷仲堪}有所识\footnote{殷荆州:指殷仲堪。仲堪曾作荆州刺史。},作赋,是束皙\myidx{束皙}慢戏之流\footnote{束皙:西晋初阳平元城人。尝作\CJKunderwave{劝农}、\CJKunderwave{饼}诸赋,时人以为鄙俗。慢戏:随意戏谑。}。{\fzxk\zihao{6}\textcolor{red}{\CJKunderwave{文士传}曰:“皙字广微,阳平元城人,汉太子太傅疏广后也。王莽末,广曾孙孟达自东海避难元城,改姓,去‘疏’之‘足’以为束氏。皙博学多识,问无不对。元康中,有人自嵩高山下得竹简一枚,上两行科斗书。司空张华以问皙,皙曰:‘此明帝显节陵中策文也。’检校果然。曾为\CJKunderwave{饼赋}诸文,文甚俳谑。三十九岁卒,元城为之废市。”}} 殷甚以为有才,语王恭\myidx{王恭}\footnote{王恭:(?—398):孝武帝后兄,安帝舅父。与殷仲堪、桓玄等,二次兴兵清君侧,兵败被诛。会稽:郡治在今浙江绍兴市。}:“适见新丈(文),甚可观。”便于手巾函中出之\footnote{手巾函:即手巾袋。古人用来放置手巾或文稿一类东西的袋子。}。王读,殷笑之不自胜\footnote{笑之不自胜:笑得不能自止。}。王看竟,既不笑,亦不言好恶,但以如意帖之而已\footnote{如意:器物名。古制长二尺许,六朝人清谈时好持之。帖:通“贴”。此处是说用如意抚平文稿。}。殷怅然自失。

{\cangkai\zihao{5}【评】细揣文意,临川当以王恭为雅量。王恭出于太原王氏,高门子弟矜持自高,又性抗直,若能从其口中取得一两句赞誉,则被评者身价陡增。太原王氏子弟不会随意赞人,故王恭看后无一句评语,但以“如意帖之”,下意识微小细节,已然背叛了其强自压抑的意识世界,说明他对文章并非否定,但认为也不是上乘佳作,以此“不笑,亦不言好恶”——实际是个不好不坏的中等之评。否则,以其人性情,定会嗤之以鼻或丢如敝屣。但是,这与殷仲堪的期盼相差甚远。故事以殷仲堪的小心翼翼取出文章、“笑不自胜”,后来又“怅然若失”等动作、神态描写,描绘了他爽朗、热情又略嫌急躁的外向气质,意在反衬王恭那充满矛盾心理的高傲、冷漠的名士风度。}

\lettrine{6.42} 羊绥\myidx{羊绥}第二子孚\myidx{羊孚}\footnote{羊绥:字仲彦,东晋泰山平阳(在今山东)人。羊忱孙。孚:见刘孝标注。羊后投桓玄,玄用为记室参军,为桓心腹注。},少有隽才,与谢益寿\myidx{谢混}相好\footnote{谢益寿:谢混,小字益寿。}。{\fzxk\zihao{6}\textcolor{red}{益寿,谢混小字也。}} 尝蚤往谢许\footnote{蚤:通“早”。},未食。俄而王齐\myidx{王齐}、王睹\myidx{王睹}来\footnote{俄而:一会儿。王齐、王睹:王恭二弟。王熙字叔和,小字齐,官太子洗马。王爽字季明,小字睹,官至给事黄门侍郎、侍中。},{\fzxk\zihao{6}\textcolor{red}{王睹,已见。齐,王熙小字也。\CJKunderwave{中兴书}曰:“熙字叔和,恭次弟,尚鄱阳公主,太子洗马,蚤卒。”}} 既先不相识,王向席,有不说色,欲使羊去。羊了不眄\footnote{了不:一点不,完全不。眄:斜视。},唯脚委几上\footnote{委:放置。几:几案。},咏瞩自若。谢与王叙寒温数语毕\footnote{寒温:寒暄。},还与羊谈赏,王方悟其奇,乃合共语。须臾食下\footnote{须臾:一会儿。食下:摆下饭菜。},二王都不得餐,唯属羊不暇。羊不大应对之,而盛进食,食毕便退。遂苦相留,羊义不住\footnote{义:坚决。},直云\footnote{直:通“只”。只是。}:“向者不得从命\footnote{向者:刚才。},中国尚虚\footnote{中国尚虚:腹中还空虚,肚子还饿着。以中国比喻腹心。}。”二王是孝伯\myidx{王恭}两弟。

{\cangkai\zihao{5}【评】故事场景好似名士派头的大比拼。开端、发展、高潮、结局各要素俱全。名士邂逅为故事的开端;二王兄弟以门第傲人,对先到的客人羊孚不假辞色,轻侮溢于言表,此为发展;羊孚反更不可一世,脚委几上,咏瞩自若,视二王如无物,令人有“一物降一物”之荒唐感,此为高潮,矛盾交锋至此似已山穷水尽。谁知,事态发展出现戏剧性的变化。二王发现羊孚才气非凡后,竟能主动迎合、前倨后恭,吃饭时,事以谦卑之礼,骄矜之气全消。这是故事令人意想不到的结局。三人似均无缘堪称雅量。相比之下,二王兄弟能主动矫枉,调整心态,有天真烂漫的一面,较羊孚之始终“举觞白眼向青天”为优。}



%%% Local Variables:
%%% mode: latex
%%% TeX-engine: xetex
%%% TeX-master: "../Main"
%%% End:
