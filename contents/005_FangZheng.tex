%% -*- coding: utf-8 -*-
%% Time-stamp: <Chen Wang: 2025-12-02 19:38:36>

% ○ ◎ ‧ 「 」 『 』 々 ( ) “ ” ■ ^[一-龥]
% 【\([^】][^】][^】]+\)】 → {\\fzxk\\zihao{6}\\textcolor{red}{\1}}
% \(【评】.*\) → {\\cangkai\\zihao{5}\1}
% \(【题解】.*\) → {\\cangkai\\zihao{5}\1}
% 《\([^》]+\)》 → \\CJKunderwave{\1}
% ^\([0-9]+.[0-9]+\) → \\lettrine{\1}
% {\\fzxk\\zihao{6}\\textcolor{red}{[^o]*}}

\setlength{\parindent}{0pt}

\part{中卷}


\chapter{方正第五}


{\cangkai\zihao{5}【题解】方正,指的是勇于坚持正道,品行刚直不阿。韩非子\CJKunderwave{解老}释“方”曰:“所谓方者,内外相应也,言行相称也”;\CJKunderwave{国语·周语下}释“正”曰:“且夫立无跛,正也。”贤良方正,历来是封建社会选士的重要标准,在汉代就明令诏举“贤良方正直言极谏者”,司马迁\CJKunderwave{史记·平准书}载:“当是之时,招尊方正贤良文学之士,或至公卿大夫。”清代科举制度中更有孝廉方正之名。可以想见,二千多年来,方正观念对于国人的精神世界产生了难以磨灭的影响。}

{\cangkai\zihao{5}本篇的66则故事大部分描写的是魏晋士大夫在各种人生考验关头和事关个人道德评判之际所表现出来的不凡举止,描绘了如诗如画的士人气质和可歌可泣的大丈夫精神。如一代大名士夏侯玄,在无情的杀戮即将来临时表现出来的名士风度,就对方正做出了最好的注脚——“临刑东市,颜色不异”,谱写了一曲杀身成仁的悲壮颂歌;如和峤不与小人荀勖同车,表现了倔强而不同流合污的决绝精神;嵇绍在齐王冏的会议上,义正词严地驳斥了群小要求其在庄严场合弹奏丝竹的无礼要求,保持人格尊严,也维护了国家官员的形象。}

{\cangkai\zihao{5}当然,随着时代的变迁,魏晋以后对方正内涵的理解已然发生了变化,此乃时势使然,英雄也徒唤奈何。如第51则载名士刘惔宁可饿肚皮也不吃“小人”置办的晚饭;王修龄缺米,却不接受陶胡奴的友情馈赠,这都是严持门阀世族制度而走入了历史的怪圈,有违人性之真淳。这在当时可能算是世族子弟潇洒出尘的惊人之举,但从今天的眼光看,则是矫饰作态,而不与方正沾边。}

{\cangkai\zihao{5}再如,本门还介绍了一些士人生活中的点滴细节,读来也颇富生活气息,令人想见古人风采。如39则载王丞相作女伎,蔡谟不悦而去,王导亦不留。这就用比较的方法生动地揭示了风流名相王导的家庭生活的侧面,以及受儒家思想濡染较深而无法融入现代社交圈子的蔡公形象;又如47则记王述转尚书令,不循传统三让之习,而是事行便拜,不为虚让,活脱了一个真率、坦荡的“赤子”,而他也只能在\CJKunderwave{世说新语}这个大社会里,才能显现出其独特魅力。于此可见,魏晋时代的“方正”观念,有其鲜活而独特的时代内容。}

\lettrine{5.1} 陈太丘\myidx{陈寔}与友期行\footnote{陈太丘:陈寔。(104—187)字仲弓,汉末颍川许昌(今属河南)人。曾任太丘长,故云。其治政清明,百姓安业,以公正直名闻世。时人评云:“宁为刑罚所加,不为陈君所短。”党锢祸起,自请系狱。卒时远近赴吊,刊石立碑,谥文范。荀朗陵:荀淑曾任郎陵侯相,故云。期行:约会同行。},期日中\footnote{日中:中午。}。过中不至,太丘舍去\footnote{舍去:不顾而离去。},去后乃至。元方\myidx{陈纪}时年七岁\footnote{元方:陈纪字元方,太丘长陈寔子,有德行,以孝著称。},门外戏。{\fzxk\zihao{6}\textcolor{red}{陈寔及纪并已见。}} 客问元方:“尊君在不\footnote{尊君:对话时尊称对方的父亲。下文自称己父为“家君”。}?”答曰:“待君久不至,已去。”友人便怒,曰:“非人哉!与人期行,相委而去\footnote{委:舍弃。}。”元方曰:“君与家君期日中。日中不至,则是无信;对子骂父,则是无礼。”友人惭,下车引之\footnote{引:拉。此处表示亲近。},元方入门不顾。

{\cangkai\zihao{5}【评】故事以精彩的对话形式,惟妙惟肖地描绘了一场长幼间的唇枪舌剑。友人失信,对子骂父,挑起事端,最后自讨没趣,败在七岁童子手下,灰溜溜地收兵。胜者元方乃汉末名士陈寔之子,寔乃乡党表率,乡里有“宁为刑法所加,不为陈君所短”的盛誉。王世懋评曰:“小儿语故自方正。”有一定道理,小儿涉世不深,淳朴未脱,不会矫饰、节制自己的喜怒哀乐,故李卓吾有言:“童心者,不失其赤子之心也。”但王氏所言,又未搔到痛处,元方的方正,更有良好的家庭教育及家族文化传承的深层原因。如\CJKunderwave{后汉书·陈寔传}言寔:“自为儿童,虽在戏弄,为等类所归”,俨然小大人形象。孩童善于自律,一定是家长教子有方。可见,小元方的方正当有其父的影子。士人的言传身教,对于儿童的健康成长,至关重要。}

\lettrine{5.2} 南阳宗世林\myidx{宗承}\footnote{南阳:郡名。治所在宛县(今河南南阳)。宗世林:宗承字世林,三国魏南阳安众(今河南获嘉北)人。征聘不就,士人争与相交,拒而不纳。曹丕称帝,征为直谏大夫。魏明帝欲以为相,以年老固辞。},魏武\myidx{曹操}同时\footnote{魏武:曹操,曹丕称帝后追尊为魏武帝,其子曹丕代汉称帝,追尊操为太祖武皇帝。},而甚薄其为人\footnote{薄:鄙薄,看不起。},不与之交。及魏武作司空\footnote{魏武作司空:曹操拥立汉献帝,于建安元年(196)任司空。司空,官名,东汉时为三公之一。},总朝政,从容问宗曰:“可以交未?”答曰:“松柏之志犹存\footnote{松柏之志:松柏傲霜凌雪,枝叶繁茂常青。此比喻坚贞之心。\CJKunderwave{论语·子罕}:“岁寒,然后知松柏之后凋也。”此处表示坚决不与曹操相交之志犹如往昔。}。”世林既以忤旨见疏\footnote{忤旨:违背意旨。见疏:被疏远。},位不配德\footnote{位不配德:官位和德行不相称,言德高而官位低。}。文帝兄\myidx{曹丕}弟\myidx{曹植}\footnote{文帝兄弟:指曹丕、曹植。造:到。},每造其门,皆独拜床下\footnote{床:坐榻,一种坐具。}。其见礼如此\footnote{见礼:被礼遇。}。{\fzxk\zihao{6}\textcolor{red}{\CJKunderwave{楚国先贤传}曰:“宗承字世林,南阳安众人。父资,有美誉。承少而修德雅正,确然不群,征聘不就。闻德而至者如林。魏武弱冠,屡造其门。值宾客猥积,不能得言。乃伺承起,往要之,捉手请交,承拒而不纳。帝后为司空,辅汉朝,乃谓承曰:‘卿昔不顾吾,今可为交未?’承曰:‘松柏之志犹存。’帝不说,以其名贤,犹敬礼之。敕文帝修子弟礼,就家拜汉中太守。武帝平冀州,从至邺,陈群等皆为之拜。帝犹以旧情介意,薄其位而优其礼,就家访以朝政,居宾客之右。文帝征为直谏大丈,明帝欲引以为相,以老固辞。”}}

{\cangkai\zihao{5}【评】心理学上有所谓的“首因效应”,实际上就是强调第一印象的重要性。宗世林对曹操的鄙视,与其对曹的第一印象欠佳有关。曹操“家庭出身”不好,其父嵩乃桓帝时大宦官曹腾的养子,对于这样的家庭,史称“未能审其出生本末”,而只能付之阙如。在日渐注重门第的汉末社会,曹操因门第不高为清流不齿,也就顺理成章了。不但如此,曹操为人还不修名行,\CJKunderwave{本纪}云:“任侠放荡,不治行业,故世人未之奇也”;\CJKunderwave{曹瞒传}云:“少好飞鹰走狗,游荡无度”,为士林所薄。宗士林不交非类,是时势风气使然。曹操发达后,宗士林面对位高权重的曹操的威吓,毫不畏惧,应声而答“松柏之志犹存”,需要方正的道德勇气和“威武不能屈”的大丈夫精神做心灵的有力支撑。然而,宗士林的处世,又有许多自相矛盾的地方。他鄙薄曹操的为人,却接受了文帝曹丕任命的官职。\CJKunderwave{晋书·王述传}载述曾祖父司空王昶语,言及宗士林晚年汲汲自励,害怕退休,前后举止判若两人,为时人所笑。可见人之思想、价值观念并非一成不变,亦未可以一时一事拘泥论之。}

\lettrine{5.3} 魏文帝\myidx{曹丕}受禅\footnote{魏文帝受禅:曹操病死,子丕嗣魏王,继任丞相。后迫使汉献帝禅位,称帝,国号魏。死后称魏文帝。受禅,接受前朝皇帝“让”给的帝位。},陈群\myidx{陈群}有慽容\footnote{陈群:字长文,陈寔孙。(104—187)字仲弓,汉末颍川许昌(今属河南)人。曾任太丘长,故云。其治政清明,百姓安业,以公正直名闻世。时人评云:“宁为刑罚所加,不为陈君所短。”党锢祸起,自请系狱。卒时远近赴吊,刊石立碑,谥文范。荀朗陵:荀淑曾任郎陵侯相,故云。慽容:愁苦的脸色。}。帝问曰:“朕应天受命\footnote{应天受命:顺应天道,承受天命。指帝王登基。},卿何以不乐?”群曰:“臣与华歆\myidx{华歆}服膺先朝\footnote{华歆:字子鱼,东汉平原高唐人。华歆拥护曹氏,曹操杀汉献帝之伏皇后,他勒兵入宫收捕皇后。曹丕称帝后,他登坛相礼,奉皇帝玺绶,以成“禅让”之仪。服膺:心悦诚服。先朝:前朝,指东汉王朝。},今虽欣圣化\footnote{圣化:圣王教化。这是赞誉当朝君主教化的谀辞,指魏朝的建立。},犹义形于色\footnote{义形于色:不忘旧主之情流露在脸上。}。”{\fzxk\zihao{6}\textcolor{red}{华峤\CJKunderwave{谱叙}曰:“魏受禅,朝臣三公以下并受爵位。华歆以形色忤时,徙为司空,不进爵。文帝久不怿,以问尚书令陈群曰:‘我应天受命,百辟莫不悦喜,形于声色,而相国及公独有不怡者,何邪?’群起离席长跪曰:‘臣与相国曾事汉朝,心虽悦喜,义干(形)其色,亦惧陛下实应见憎。’帝大悦,叹息良久,遂重异之。”}}

{\cangkai\zihao{5}【评】汉末是个大变动的时代。政治上宦官专权、外戚干政以及地方势力割据等各种社会痼疾,像毒瘤一样侵袭着国家政权,导致献帝播迁,中原板荡。社会思想也变得混乱不堪。为封建皇权所强化的儒学,此时弊端显露,日趋式微。士人的国家观念、忠君意识与封建大一统时代相比,已不可同日而语。故事的两位主人公陈群、华歆,虽忝列“方正”,但从儒家传统评价标准看,显然有亏。明王世懋评曰:“华歆以虚名居首揆,陈群以心膂当新宠,犹为此大言,宁不为荀彧地下所笑?”对二人的虚声窃誉表示鄙夷与不屑。特别是华歆,实际是曹氏心腹,无所不用其极,在曹操屠戮汉室宗亲和曹丕篡权过程中,是冲锋陷阵的急先锋。魏晋所称“方正”,其道德观念与两汉有异,应作具体分析。}

\lettrine{5.4} 郭淮\myidx{郭淮}作关中都督\footnote{郭淮(?—255):字伯济,三国魏阳曲(今山西太原)人。曹丕即帝位,他官至刺史,封射阳亭侯。镇守关中三十馀年,功绩卓著。关中:指函谷关以西(包括今陕西全境、甘肃东部、秦岭以北)广大地区。都督:武官名。掌一州或数州军事,或也兼管行政。},甚得民情,亦屡有战庸\footnote{民情:指民心、民意。战庸:战功。}。{\fzxk\zihao{6}\textcolor{red}{\CJKunderwave{魏志}曰:“淮字伯济,太原阳曲人。建安中,除平原府丞。黄初元年,奉使贺文帝践祚,而稽留不及。群臣欢会,帝正色责之曰:‘昔禹会诸侯于涂山,防风氏后至,便行大戮。今溥天同庆,而卿最留迟,何也?’淮曰:‘臣闻五帝先教,导民以德,夏后政衰,始用刑辟。今臣遭唐虞之世,是以知免防风氏之诛。’帝悦之,擢为雍州刺史,迁征西将军。淮在关中二(三)十馀年,功绩显著,迁仪同三司,赠大将军。”}} 淮妻太尉王凌\myidx{王凌}之妹\footnote{太尉:官名,魏晋时代为三公之一。王凌(172?—251):字彦云,三国魏太原祁(今属山西)人。与外甥令狐愚谋立楚王曹彪为帝,事败自杀。司马懿灭其三族。},坐凌事当并诛\footnote{坐:牵连获罪。指因王凌一案而受株连。},{\fzxk\zihao{6}\textcolor{red}{\CJKunderwave{魏略}曰:“凌字彦云,太原祁人。历司空、太尉、征东将军。密欲立楚王彪,司马宣王自讨之,凌自缚归罪。遥谓太傅曰:‘卿直以折简召我,我当不至邪!’太傅曰:‘以卿非皆(肯)逐折简者也。’遂使人送至西。凌自知罪重,试索棺钉,以观太傅意,太傅给之。凌行至项城,夜呼掾属与决曰:‘行年八十,身名俱灭,命邪!’遂自杀。”}} 使者征摄甚急\footnote{使者:指缉拿的官吏。征摄:追捕缉拿。}。淮使戒装,克日当发\footnote{戒装:准备行装。克日:限定日期。}。州府文武及百姓劝淮举兵,淮不许。至期遣妻,百姓号泣追呼者数万人。行数十里,淮乃命左右追夫人还,于是文武奔驰,如徇身首之急\footnote{州府文武:指州府的文武官员。徇:夺取,营救。身首:指性命。}。既至,淮与宣帝\myidx{司马懿}书曰\footnote{宣帝:指司马懿(179—251),字仲达,三国魏河内温(今属河南)人。其孙炎代魏称帝后,追尊他为晋宣帝。}:“五子哀恋,思念其母。其母既亡,则无五子;五子若殒\footnote{殒:死亡。},亦复无淮。”宣帝乃表特原淮妻\footnote{表:上表,指司马懿上表给皇帝。原:原宥,赦免。}。{\fzxk\zihao{6}\textcolor{red}{\CJKunderwave{世语}曰:“淮妻当从坐,侍御史往收,督将及羌胡渠帅数千人,叩头请淮上表留妻,淮不从。妻上道,莫不流涕,人人扼腕,欲劫留之。淮五子叩头流血请淮,淮不忍视,乃命追之。于是,数千骑往追还。淮以书白司马宣王曰:‘五子哀母,不惜其身。若无其母,是无五子;五子若亡,亦无淮也。今辄追还,若于法未通,当受罪于主者。’书至,宣王乃表原之。”}}

{\cangkai\zihao{5}【评】考诸史籍,郭淮戎马一生,所在治绩有功,是一位不可多得的能臣。他应对文帝曹丕的质问,有理有节,大义凛然,故未可仅以一介武夫视之。淮妻坐兄王凌事为朝廷所纠,情与理的较量,有如古希腊悲剧\CJKunderwave{安提哥涅}所描述的国家伦理和家庭伦理之间的情感张力。但他最后毅然尊重国家法律的严肃性,\CJKunderwave{诗}云:“刑于寡妻,至于兄弟,以御于家邦”,这需要高度克己的功夫。后来,他又上书司马懿,请求宽宥妻子,表现出重亲情的可爱一面。刘辰翁曰:“语甚感动,节次皆是。”先国家之急而后私人恩怨,临川列其入方正,可谓允当。}

\lettrine{5.5} 诸葛亮\myidx{诸葛亮}之次渭滨\footnote{诸葛亮(181—234):字孔明,三国蜀琅邪阳都(今山东沂南南)人。蜀汉刘备丞相。备死,他受遗诏辅佐后主刘禅,封武乡侯,领益州牧。次渭滨:军队驻扎在渭水边上。},关中震动\footnote{关中:指函谷关以西的地区。}。{\fzxk\zihao{6}\textcolor{red}{\CJKunderwave{蜀志}曰:“亮字孔明,琅邪阳都人。客于荆州,躬耕垅亩,好为\CJKunderwave{梁甫吟}。长八尺,每自比管仲、乐毅,时人莫之许也,唯博陵崔州平、颍川徐元直谓为信然。先主屯新野,徐庶见先主曰:‘诸葛孔明,卧龙也。将军岂愿见之乎?’先主曰:‘君与俱来。’庶曰:‘此人可就见,不可屈致也。’先主遂诣亮,谓关羽、张飞曰:‘孤之有孔明,犹鱼之有水也。’累迁丞相、益州牧。率众北征,卒于渭南。”}} 魏明帝\myidx{曹叡}深惧晋宣王\myidx{司马懿}战\footnote{魏明帝:曹叡字元仲,魏第二代君主,在位十馀年,谥为明皇帝。晋宣王:指司马懿。},乃遣辛毗\myidx{辛毗}为军司马\footnote{辛毗:字佐治,初从袁绍,后归曹操。以直言敢谏著称。魏明帝青龙二年,为大将军司马懿军师,使持节,监魏军与蜀军战于渭南。军司马:宋本及各本均作“军司马”,非是。据\CJKunderwave{魏书·辛毗传}及\CJKunderwave{晋书·宣帝纪}应为“军师”。当是为避晋景帝司马师讳而改“师”为 “司”,后人又以“军司”不通而添以“马”字。官名。参谋军事。}。{\fzxk\zihao{6}\textcolor{red}{\CJKunderwave{魏志}曰:“毗字佐治,颍川阳翟人。累迁卫尉。”}} 宣王既与亮对渭而陈\footnote{陈:通“阵”,隔着渭水相对列阵。},亮设诱谲万方\footnote{诱谲:引诱、诈骗。万方:千方百计。},宣王果大忿,将欲应之以重兵。亮遣间谍觇之\footnote{觇:窥视,暗中察看。},还曰:“有一老夫,毅然仗黄钺\footnote{黄钺:兵器名,状如斧。黄钺,以黄金为饰,天子所用。大臣持黄钺代皇帝行使权力。},当军门立,军不得出。”亮曰:“此必辛佐治也。”{\fzxk\zihao{6}\textcolor{red}{\CJKunderwave{晋阳秋}曰:“诸葛亮寇于郿,据渭水南原,诏使高祖拒之。亮善抚御,又戎政严明,且侨军远征,粮运艰涩,利在野战。朝廷每闻其出,欲以不战屈之,高祖亦以为然。而拥大军御侮于外,不宜远露怯弱之形,以亏大势,故秣马坐甲,每见吞并之威。亮虽挑战,或遗高祖巾帼。巾帼,妇女之饰,欲以激怒,冀获曹咎之利。朝廷虑高祖不胜忿愤,而卫尉辛毗,骨鲠之臣,帝乃使毗仗节为高祖军司马。亮果复挑战,高祖乃奋怒,将出应之。毗仗节中门而立,高祖乃止。将士闻见者,益加勇锐。识者以人臣虽拥众千万,而屈于王人。大略深长,皆如此之类也。”}}

{\cangkai\zihao{5}【评】文帝、明帝两朝,辛毗多次反对修殿舍、兴劳役,曾经拉住文帝的衣裾不放,是一位敢于抗颜直谏的耿介之士。魏明帝时诸葛亮北伐攻魏,吴主孙权后来也配合诸葛亮几次进攻合肥新城。魏明帝坚决执行曹操以来实行的战略防御方针,用满宠镇守淮南以防吴,用曹真、司马懿镇守关中以御蜀汉,目的在使对方进不得战,粮尽必退,所以他非常担心司马懿出兵应战。蜀、魏交战,魏方军事方针已定,采取固守耗敌实力,以达不攻而敌自退的军事目的。魏明帝派辛毗威慑司马懿,可谓得人。试想,连天威龙颜都敢冒犯,又何惧一司马懿?事实证明,明帝以辛毗制衡司马懿,棋高一招。明帝有知人善任之明,辛毗有不辱使命之功,司马懿免战得顺水推舟之情。贤明相逢,皆大欢喜,千古以来,传为美谈!}

\lettrine{5.6} 夏侯玄\myidx{夏侯玄}既被桎梏\footnote{夏侯玄:(209—254):字太初,三国魏人。曹爽辅政时,他以爽姑之子受重用。曹爽被诛,玄废黜。后与李丰等谋杀司马师,事败,同被诛。他是早期的玄学领袖人物。被:遭受。桎梏:手铐脚镣。},{\fzxk\zihao{6}\textcolor{red}{\CJKunderwave{魏氏春秋}曰:“玄字太初,谯国人,夏侯尚之子,大将军前妻兄也。风格高朗,弘辩博畅。正始中,护军。曹爽诛,征为太常。内知不免,不交人事,不畜笔研。及太傅薨,许允谓玄曰:‘子无复忧矣!’玄叹曰:‘士宗,卿何不见事乎?此人犹能以通家年少遇我,子元、子上,不吾容也。’后中书令李丰恶大将军执政,遂谋以玄代之。大将军闻其谋,诛丰,收玄送廷尉。”干宝\CJKunderwave{晋纪}曰:“初,丰之谋也,使告玄,玄答曰:‘宜详之尔。’不以闻也,故及于难。”}} 时锺毓\myidx{锺毓}为廷尉\footnote{锺毓:字稚叔,魏太傅锺繇长子。锺毓、锺会:魏锺繇二子,颍川长社人。毓,字稚叔,官至廷尉、青州刺史,督徐州、荆州军事,死后追赠车骑将军,谥惠侯。会,字士季,官至司徒。受命伐蜀,蜀破,欲率军谋反,内部先乱,为乱军所杀。魏以谋反论其罪。令誉:美好的声誉注。廷尉:掌刑狱的官。},锺会\myidx{锺会}先不与玄相知\footnote{锺会:字士季,锺繇少子。博学,精名理。景元中,与邓艾伐蜀,后以谋反罪,被杀。锺毓、锺会:魏锺繇二子,颍川长社人。毓,字稚叔,官至廷尉、青州刺史,督徐州、荆州军事,死后追赠车骑将军,谥惠侯。会,字士季,官至司徒。受命伐蜀,蜀破,欲率军谋反,内部先乱,为乱军所杀。魏以谋反论其罪。令誉:美好的声誉注。},因便狎之\footnote{狎:亲近。}。玄曰:“虽复刑馀之人\footnote{刑馀之人:受过刑的人。一般多用于犯人自称。},未敢闻命\footnote{未敢闻命:不敢听从你的命令。这里是婉辞,实际上是说,不愿与你交往。}。”{\fzxk\zihao{6}\textcolor{red}{\CJKunderwave{世语}曰:“玄至廷尉,不肯下辞。廷尉锺毓自临履玄。玄正色曰:‘吾当何辞,为令史责人邪?卿便为吾作。’毓以玄名士,节高不可屈;而狱当竟,夜为作辞,令与事相附,流涕以示玄,玄视之曰:‘不当若是邪!’锺会年少于玄,玄不与交。是日,于毓坐狎玄。玄正色曰:‘锺君何得如是?’”\CJKunderwave{名士传}曰:“初,玄以锺毓志趣不同,不与之交。玄被收时,毓为廷尉,执玄手曰:‘太初,何至于此?’玄正色曰:‘虽复刑馀之人,不可得交。’”按郭颁,西晋人,时世相近,为\CJKunderwave{晋魏世语},事多详覈。孙盛之徒,皆采以者(著)书,并云玄距锺会。而袁宏\CJKunderwave{名士传}最后出,不依前史,以为锺毓,可谓谬矣!}} 考掠\footnote{考掠:考问鞭打。},初无一言\footnote{初无:完全没有。初:全,都。},临刑东市\footnote{东市:汉代在长安东市处决判死刑的人,后以东市指刑场。},颜色不异。{\fzxk\zihao{6}\textcolor{red}{\CJKunderwave{魏志}曰:“玄格量弘济,临斩,颜色不异,举止自若。”}}

{\cangkai\zihao{5}【评】孔子曰:“志士仁人,无求生以害仁,有杀身以成仁。”(\CJKunderwave{论语·卫灵公})“仁”字当头,是儒者本色。仁是封建时代的最高道德规范、行为信条。曹氏皇权与司马氏一党之明争暗斗,本属统治阶级上层之间的勾心斗角、利益纷争,无所谓是非、正邪。但从正统的伦理道德观来衡量,司马氏集团显属篡夺行为。当时人语云:“司马昭之心,路人皆知”,言其不臣之迹已昭然若揭。其实,司马昭父兄之诛魏宗室曹爽、杀名士夏侯玄,都是篡魏的前奏,而终由司马炎完成阴谋的乐章。锺会是司马氏的帮凶走狗,夏侯玄鄙薄其人不与相交,临刑东市而颜色自若,除了昭示不同政治阵营“道不同,不相与谋”的立场分野外,还表现了一代名士在死亡降临之际那杀身成仁的潇洒风姿,与嵇康之“广陵曲绝”同出一辙而千古传诵。}

\lettrine{5.7} 夏侯泰\myidx{夏侯泰}初与广陵陈本\myidx{陈本}善\footnote{夏侯泰初:即夏侯玄,字太初。泰同“太”。广陵:郡名,治所在今江苏扬州。陈本:字休元。历位郡守、九卿,有统御之才。},本与玄\myidx{夏侯玄}在本母前宴饮。{\fzxk\zihao{6}\textcolor{red}{\CJKunderwave{世语}曰:“本字休元,临淮东阳人。”\CJKunderwave{魏志}曰:“本,广陵东阳人。父矫,司徒。本历郡守、廷尉,所在操纲领,举大体,能使群下自尽,有率御之才;不亲小事,不读法律,而得廷尉之称。迁镇北将军。”}} 本弟骞\myidx{陈骞}{\fzxk\zihao{6}\textcolor{red}{\CJKunderwave{晋阳秋}曰:“骞字休渊,司徒第二子。无謇谔风,滑稽而多智谋。仕至大司马。”}} 行还\footnote{本弟骞:陈骞,字休渊。晋武帝司马炎受禅,以佐命功进车骑将军。官至大司马。行还:从外边回家。},径入至堂户。泰初因起曰:“可得同,不可得而杂\footnote{可得同,不可得而杂:大意是可以礼相交,不能违礼杂处。“径入至堂户”相见,是一种失礼行为,可能是夏侯玄看不上陈骞的人品,以年辈不相当为托辞而不与相交。}。”{\fzxk\zihao{6}\textcolor{red}{\CJKunderwave{名士传}曰:“玄以乡党贵齿,本不论德位,年长者必为拜。与陈本母前饮,骞来而出,其可得同,不可得而杂者也。”}}

{\cangkai\zihao{5}【评】魏晋士风“越名教而任自然”,完全是爱憎分明,不尚矫饰。故锺会来拜而嵇康锻铁不顾,嵇喜吊孝而阮籍视之白眼,完全是一任天真的潇洒绝尘之举。夏侯玄鄙薄陈骞为人,托辞年辈不伦而拒与之交。以世俗常理视之,可谓不近人情,似乎有些做作;而以士人眼光来看,则谓名士高致,跻身“方正”。但方正与乖戾仅隔一步之遥,全在如何掌控。名士珍惜自己的清流声誉,本无可厚非,但若画地为牢,与俗流完全绝缘,渐渐走向不食人间烟火的怪圈,则其所持的方正也可能变了味道。清流、浊流并非泾渭分明,其本身也存在一个不断吐故纳新的过程:昨日之清流,今朝可能会变得俗不可耐,而为士林不齿。庶族浊流子弟经过岁月的陶冶与磨砺,反而会跻身名士之列。夏侯玄与陈本共饮而鄙薄其弟的做法,与宗士林看不起曹操而做曹丕的官,似有相近之处。}

\lettrine{5.8} 高贵乡\myidx{曹髦}公薨\footnote{高贵乡公:曹髦(241—260),字彦士,魏文帝曹丕孙,封高贵乡公。司马景王废齐王曹芳,立髦为帝。甘露五年(260),司马氏的亲信中护军贾充令太子舍人成济将其杀死,史称“高贵乡公”。薨:侯王死称薨。内外:指朝廷内外。},内外喧哗。{\fzxk\zihao{6}\textcolor{red}{\CJKunderwave{魏志}曰:“高贵乡公,讳髦,字彦士,文帝孙,东海定王霖之子也。初封郯县高贵乡公。好学夙成。齐王废,群臣迎之即皇帝位。”\CJKunderwave{汉晋春秋}曰:“自曹芳事后,魏人省彻宿卫,无复铠甲,诸门戎兵,老弱而已。曹髦见威权日去,不胜其忿。召侍中王沈、尚书王经、散骑常侍王业,谓曰:‘司马昭之心,路人所知也。吾不能坐受废辱,今日当与卿自出讨之。’王经谏,不听,乃出怀中板令投地,曰:‘行之决矣。正使死,何所恨,况不必死邪!’于是入白太后。沈、业奔走告昭,昭为之备。髦遂率僮仆数百,鼓噪而出。昭弟屯骑校尉伷入,遇髦于东止车门;左右呵之,伷众奔走。中护军贾充又逆髦战于南阙下,髦自用剑。众欲退,太子舍人成济问充曰:‘事急矣,当云何?’充曰:‘公畜汝等,正为今日。今日之事,无所问也。’济即前刺髦,刃出于背。”\CJKunderwave{魏氏春秋}曰:“帝将诛大将军,诏有司复进位相国,加九锡。帝夜自将冗从仆射李昭、黄门从官焦伯等下陵云台,铠仗授兵,欲因际会,遣使自出致讨。会雨而却。明日,遂见王经等出黄素诏于怀曰:‘是可忍也,孰不可忍!今当决行此事。’帝遂拔剑升辇,率殿中宿卫仓头官僮,击战鼓,出云龙门。贾充自外而入,帝师溃散。帝犹称天子,手剑奋击,众莫敢逼。充率厉将士,骑督成倅弟济以牙(矛)进,帝崩于师。时暴雨,雷电晦冥。”}} 司马文王\myidx{司马昭}问侍中陈泰\myidx{陈泰}曰\footnote{司马文王:指司马昭,懿子。历魏数朝,死后谥为文王。侍中:官名。侍从皇帝左右,职掌傧赞礼仪、护驾陪乘,并备应对顾问。陈泰:(?—260),字玄伯,三国魏颍川许昌人(今属河南)人。魏司空陈群子。官至侍中、左仆射。}:{\fzxk\zihao{6}\textcolor{red}{\CJKunderwave{魏志}曰:“泰字玄伯,司空群之子也。”}} “何以静之\footnote{静:平静。}?”泰云:“唯杀贾充\myidx{贾充}以谢天下\footnote{贾充:字公闾,魏末晋初人。佐司马昭执朝政,杀高贵乡公,废魏立晋,为元勋。谢天下:向天下人承认罪责。}。”文王曰:“可复下此不\footnote{下此:地位低于此人。}?”对曰:“但见其上,未见其下。”{\fzxk\zihao{6}\textcolor{red}{干宝\CJKunderwave{晋纪}曰:“高贵乡公之杀,司马文王召朝臣谋其故。太常陈泰不至,使其舅荀顗召之,告以可不。泰曰:‘世之论者,以泰方于舅,今舅不如泰也。’子弟内外咸共逼之,垂涕而入。天(文)王待之曲室,谓曰:‘玄伯,卿何以处我?’对曰:‘可诛贾充以谢天下。’文王曰:‘为吾更思其次。’泰曰:‘唯有进于此,不知其次。’文王乃止。”\CJKunderwave{汉晋春秋}曰:“曹髦之薨,司马昭闻之,自投于地曰:‘天下谓我何?’于是召百宫(官)议其事,昭垂泪问陈泰曰:‘何以居我?’泰曰:‘公光辅数世,功盖天下,谓当并迹古人,垂美于后。一旦有杀君之事,不亦惜乎!速斩贾充,犹可以自明也。’昭曰:‘公闾不可得杀也。卿更思馀计。’泰厉声曰:‘意唯有进于此耳,饮(馀)无足委者也!’归而自杀。”\CJKunderwave{魏氏春秋}曰:“泰劝大将军诛贾充,大将军曰:‘卿更思其他。’泰曰:‘岂可使泰复发后言!’遂呕血死。”}}

{\cangkai\zihao{5}【评】曹髦“司马昭之心,路人皆知”一语,见出司马氏之篡国野心已包藏不住,步步紧逼,终于酿成高贵乡公被弑之宫廷流血政变。侍中陈泰号啕尽哀,呕血而卒,力主严惩凶手。“但见其上,未见其下”,将矛头直指司马氏。在曹氏政权危如累卵、江河日下的情况下,以孤危之身面对如狼似虎的群凶,与夫“有奶就是娘”的骑墙分子,有着天壤之别。这份对故国旧君的情意,如王世懋所评:“千载凛凛,陈群有愧色矣。”群、泰父子二人的人格高下已不言自明。此外,事件发生前前后后各色人等形形色色的表现,正是世态炎凉的活画图。}

\lettrine{5.9} 和峤\myidx{和峤}为武帝\myidx{司马炎}所亲重\footnote{和峤:字长舆。晋武帝时为中书令,转侍中,甚被器重。参和峤(?—292):魏晋时汝南西平(今属河南)人。官至中书令。为政清简得民,有风格,善礼法,朝野许其能正风俗人伦。家财富而性至吝,人称有“钱癖”。大丧:指父母之丧。据\CJKunderwave{晋书}戎传,时戎遭母丧,而峤遭父丧。武帝:指晋武帝司马炎。废魏建晋,灭蜀伐吴,统一全国。在位26年,死后谥为武皇帝。},语峤:“东宫顷似更成进\footnote{语峤:明袁氏嘉趣堂本“语峤”下有“曰”字。东宫:太子所居之宫。此指皇太子。顷:近来。成进:成熟长进。},卿试往看。”还,问何如。答云:“皇太子圣质如初\footnote{圣质:指太子的资质。}。”{\fzxk\zihao{6}\textcolor{red}{\CJKunderwave{晋诸公赞}曰:“峤字长舆,汝南西平人。父逌,太常,知名。峤少以雅量称,深为贾充所知,每向世祖称之。历尚书、太子少傅。”干宝\CJKunderwave{晋纪}曰:“皇太子有醇古之风,美于信受。侍中和峤数言于上曰:‘季世多伪,而太子尚信,非四海之主。忧太子不了陛下家事,愿追思文、武之祚。’上既重长适,又怀齐王,朋党之论弗入也。后上谓峤曰:‘太子近入朝,吾谓差进,卿可与荀侍中共往言。’及顗奉诏还,对上曰:‘太子明识弘新,有如明诏。’问峤,峤对曰:‘圣质如初。’上嘿然。”\CJKunderwave{晋阳秋}曰:“世祖疑惠帝不可承继大业,遣和峤、荀勗往观察之。既见,勗称叹曰:‘太子德更进茂,不同于故。’峤曰:‘皇太子圣质如初。此陛下家事,非臣所尽。’天下闻之,莫不称峤为忠,而欲灰灭勗也。”案荀顗清雅,性不阿谀。校之二说,则孙盛为得也。}}

{\cangkai\zihao{5}【评】历代帝王册立储君,因关系到帝祚能否瓜瓞绵长,故慎之又慎。晋武帝立司马衷,即后来的惠帝,大臣和峤担忧其弱智影响执政能力,多次进谏反对。武帝出于私心杂念及惑于群小等多重原因,终未能采纳方正直言。故事截取立嫡斗争中的一个小片段,生动地刻画出一位以国家为己任的刚直不阿大臣形象。武帝请和峤考察太子,在一般人看来,这是千载难逢的巴结机会,送个顺水人情,既讨皇帝的欢心,又取悦了未来的主子,何乐而不为?但和峤听从内心良知的召唤,不给予皇帝的自私动机以丝毫的迁就,表现出了可贵之处。可惜武帝没有听从逆耳忠言,最终做出了错误决定。惠帝果然是一个任人摆布的傀儡,身死人手且不说,终酿成败坏皇基的“八王之乱”,继以“五胡乱华”而亡国,可悲。}

\lettrine{5.10} 诸葛靓\myidx{诸葛靓}后入晋\footnote{诸葛靓:字仲思,魏司空诸葛诞少子。据\CJKunderwave{三国志·吴书·孙亮传}、\CJKunderwave{三国志·魏书·诸葛诞传}载,魏司空诸葛诞于甘露二年(257)五月,起兵反抗司马氏专权,遣子靓入吴为质以求援助。靓后仕吴,为右将军、大司马。吴亡,隐居不出。后入晋:诸葛靓先在三国吴,晋灭吴,遂入晋。},除大司马\footnote{除:拜官,授任。大司马:官名。晋置大司马,与大将军、丞相共掌朝政。},召不起。以与晋室有仇\footnote{召不起:征召不受。与晋室有仇:诸葛靓父诸葛诞本为魏将,257年,诞以寿春叛,大将军司马文王率军灭之,诞被杀,夷三族。故诸葛靓与晋王室司马氏有杀父之仇。},常背洛水而坐。与武帝\myidx{司马炎}有旧\footnote{有旧:有旧交。},帝欲见之而无由,乃请诸葛妃呼靓。既来,帝就太妃间相见\footnote{太妃:即诸葛妃。诸葛妃是诸葛靓之姐,诸葛诞女,司马懿子琅邪王妃,为晋武帝司马炎叔母,故称太妃。}。礼毕,酒酣,帝曰:“卿故复忆竹马之好不\footnote{故复:仍然,还。竹马之好:指儿童时代的友情。竹马,儿童游戏,以竹竿当马。}?”靓曰:“臣不能吞炭漆身\footnote{吞炭漆身:典出\CJKunderwave{战国策·赵策一}、\CJKunderwave{史记·刺客列传}。战国时期韩、赵、魏三家攻杀智伯。智伯之门客豫让为报知遇之恩,乃吞咽木炭,用漆涂身,改变音容以刺赵襄子,事败而死。此处借以喻忍垢忍辱,矢志报仇。},今日复睹圣颜\footnote{圣颜:指皇帝的容颜。此指晋武帝。}。”因涕泗百行。帝于是惭悔而出。{\fzxk\zihao{6}\textcolor{red}{\CJKunderwave{晋诸公赞}曰:“吴亡,靓入洛,以父诞为世(袁本作‘太’,是。下同)祖所杀,誓不见世祖。世祖叔母琅邪王妃,靓之姊也。帝后因靓在姊间,往就见焉。靓逃于厕中。于是以至孝发名。时嵇康亦被法,而康子绍死荡阴之役。谈者咸曰:‘观绍、靓二人,然后知忠孝之道区以别矣。’”}}

{\cangkai\zihao{5}【评】诸葛靓本为吴臣,入晋后常背洛水而坐,以示不愿归顺之意;且不与总角之交司马炎叙旧结好,态度决绝,这与吴亡后入洛士人趋之若鹜,可谓大相径庭。是靓之忠君爱国之心超出众类吗?细细品味,殊觉未必。原来靓父诞本为魏将,257年,诸葛诞以寿春叛,为司马昭所杀,故二家有不共戴天的家族血仇。汉末以降,士人之国家意识淡出,而孝行意识被强化。靓字“仲思”,自释其义曰:“在家思孝,事君思忠,朋友思信。”其实,“思忠”一义已大打折扣,诸葛靓之方正背后,家族仇恨当占了更大比重。}

\lettrine{5.11} 武帝\myidx{司马炎}语和峤\myidx{和峤}曰\footnote{武帝:晋武帝司马炎,和峤(?—292):魏晋时汝南西平(今属河南)人。官至中书令。为政清简得民,有风格,善礼法,朝野许其能正风俗人伦。家财富而性至吝,人称有“钱癖”。大丧:指父母之丧。据\CJKunderwave{晋书}戎传,时戎遭母丧,而峤遭父丧。和峤:和峤(?—292):魏晋时汝南西平(今属河南)人。官至中书令。为政清简得民,有风格,善礼法,朝野许其能正风俗人伦。家财富而性至吝,人称有“钱癖”。大丧:指父母之丧。据\CJKunderwave{晋书}戎传,时戎遭母丧,而峤遭父丧。}:“我欲先痛骂王武子\myidx{王济}\footnote{王武子:王济字武子。亦当时豪爽之士,\CJKunderwave{晋书}卷五十六本传,言其才藻卓绝,爽迈不群,多所陵傲,缺乡曲之誉。年四十馀始仕。与王济相知甚深。注。王济妻为武帝女常山公主。和峤是王济的姐夫。},然后爵之\footnote{爵之:给他封官爵。“爵”用为动词。}。”峤曰:“武子隽爽\footnote{隽爽:性格俊迈豪爽。},恐不可屈。”帝遂召武子,苦责之,因曰:“知愧不?”{\fzxk\zihao{6}\textcolor{red}{\CJKunderwave{晋诸公赞}曰:“齐王当出藩,而王济谏请无数,又累遣常山王(主)与(甄德)妇长广公主共入,稽颡陈乞留。世祖甚恚,谓王戎曰:‘我兄弟至亲,今出齐王,自朕家计,而甄德、王济连遣妇入来生哭人邪?济等尚尔,况馀者乎?’济自此被责,左迁国子祭酒。”}} 武子曰:“尺布斗粟之谣\footnote{尺布斗粟之谣:\CJKunderwave{汉书·淮南衡山传}载,汉淮南王谋反事败,文帝流放他到蜀,路上绝食而死。百姓作民歌:“一尺布,尚可缝;一斗粟,尚可舂;兄弟二人,不能相容。”后以“尺布斗粟”比喻兄弟失和。},常为陛下耻之。{\fzxk\zihao{6}\textcolor{red}{\CJKunderwave{汉书}曰:“淮南厉王长,高祖少子也。有罪,文帝徙之于蜀,不食而死。民作歌曰:‘一尺布,尚可缝;一斗粟,尚可春(舂);兄弟二人,不能相容。’”瓒注曰:“言一尺布帛可缝而共衣,一斗米粟可春(舂)而共食;况以天子之属,而不相容也。”}} 他人能令疏亲,臣不能使亲疏,以此愧陛下。”

{\cangkai\zihao{5}【评】故事乃立储过程系列事件中的一个小插曲。表面上是岳父晋武帝与女婿王济间的斗嘴,但见微知著,涉及的实是司马衷这个傻乎乎的太子能否顺利接班的“国之大事”。王济出于太原王氏家族,朝中新贵,性极骄狂。但他同时又是富有才情的一代名士,颇见独立不拘品格,性格较为复杂。这次他敢于向皇帝岳父顶嘴,还多少有些主持正义的味道。原来在立嫡过程中,朝中大臣多属意于武帝同母弟、德才兼备的齐王司马攸。司马攸为父司马昭所爱,几乎立为太子。武帝登帝位后,封攸为齐王,声望日隆。武帝晚年,所立太子司马衷懦愚,朝臣多寄希望于齐王。王济向武帝陈请留齐王,又叫妻子常山公主进宫请求,因此触怒武帝,被责。王济却引用“尺布斗粟”之歌来讽喻武帝不容同母弟齐王。本门第九则和峤故事,已微妙地传达出这一人心所向。武帝担心自己死后,齐王攸“篡国夺权”,故采纳亲信的建议遣齐王归藩,以绝后患。王济引用汉文帝时民谣反唇相讥,在关涉国运兴衰大计时,能够“吕望大事不糊涂”,也殊为难能可贵。}

\lettrine{5.12} 杜预\myidx{杜预}之荆州\footnote{杜预(222—284):字元凯,西晋京兆杜陵(今陕西西安东南)人,西晋平吴,预有大功。博学而多谋略,时称“杜武库”。著有\CJKunderwave{春秋左氏传经传集解}。之:到……去。此指上任。荆州:治所在襄阳(今湖北)。晋武帝咸宁四年,以预为镇南大将军,都督荆州诸军事。},顿七里桥\footnote{顿: 暂时停留, 止息。七里桥: 在河南洛阳城东。},朝士悉祖\footnote{祖:原义为出门之前祭祀路神。引申为饯行,送别。}。{\fzxk\zihao{6}\textcolor{red}{王隐\CJKunderwave{晋书}曰:“预字元凯,京兆杜陵人,汉御史大夫延年十一世孙。祖畿,魏太保。父恕,幽州刺史。预智谋渊博,明于治乱,常称:‘立德者非所企及,立功、立言,所庶几也。’累迁河南尹,为镇南将军,都督荆州诸军事,镇襄阳。以平吴勋,封当阳侯。预无伎艺之能,身不跨马,射不穿札,而每有大事,辄在将帅之限。赠征南将军、仪同三司。”}} 预少贱,好豪侠,不为物所许\footnote{不为物所许:不被当时公众认可。杜预少时家道贫寒,性又豪爽,在崇尚门阀的魏晋间,难以受到推重。物,人,公众。许,赞许。}。杨济\myidx{杨济}既名氏雄俊\footnote{杨济(?—291):字文通,西晋弘农华阴(今属陕西)人。官至右卫将军、太子太傅。其兄杨骏,为晋武帝杨皇后之父,权势倾天下。名氏:名门望族。},不堪\footnote{不堪:经不起;受不了。},不坐而去。{\fzxk\zihao{6}\textcolor{red}{\CJKunderwave{八王故事}曰:“济字文通,弘农人,杨骏弟也。有才识,累迁太子太保。与骏同诛。”}} 须臾,和长舆\myidx{和峤}来\footnote{和长舆:指和峤。峤字长舆。},问:“杨右卫何在\footnote{杨右卫:指杨济。济曾作右卫将军。}?”客曰:“向来,不坐而去。”长舆曰:“必大夏门下盘马\footnote{大厦门:洛阳城门,位在城北。盘马:驰马盘旋。}。”往大夏门,果大阅骑,长舆抱内车,共载归,坐如初。

{\cangkai\zihao{5}【评】中国文化传统中,向来有立德、立功、立言的“三不朽”之说。综观杜预之一生,汲汲于功名,实现了人生不朽的目标。晋国平吴,杜预为坚定不移的倡导者和实行者;非唯如此,他还在政治、律历、史学等方面,均有杰出的建树。时人号之曰“杜武库”,绝非虚誉。杨济乃武帝杨皇后叔父,与其兄杨骏权倾天下,炙手可热。他自恃世族门高、外戚权重,在杜预的饯行会上耍起名士脾气。此举当时人或许视为率性不羁的方正风度,今人看来,这是门阀意识的偏见,毫无风度可言。王世懋评曰:“杜元凯千载名士,杨济倚外戚为豪,此何足为方正?”对此有深入的思考。可见,一个人无论地位有多高,生前多么荣光显赫,甚至为其树碑立传者不绝如缕,若不能对社会进步和历史发展起积极作用,也难逃无情历史的抛弃。}

\lettrine{5.13} 杜预\myidx{杜预}拜镇南将军\footnote{杜预:见前则。拜镇南将军:事在晋武帝咸宁四年。镇南将军,晋将军之号,征伐时所设,不常置。},朝士悉至,皆在连榻坐\footnote{连榻:榻是古代一种坐具,矮而狭长。如今之长凳或长椅之类。可坐数人者称连榻,一人坐者为独榻。独榻待客,有尊敬之意,连榻坐客,有慢待之嫌。}。{\fzxk\zihao{6}\textcolor{red}{\CJKunderwave{语林}曰:“中朝方镇还,不与元凯共坐;预征吴还,独榻,不与宾客共也。”}} 时亦有裴叔则\myidx{裴楷}\footnote{裴叔则:裴楷字叔则,博学,通\CJKunderwave{周易}。以盛德居高位。}。羊稚舒\myidx{羊琇}后至\footnote{羊稚舒:羊琇,字稚舒,晋初泰山平阳(在今山东)人。羊祜从弟。司马师妻羊氏之叔父。少与司马炎相友善,为之策画,炎得立为太子。炎即帝位后,琇甚得宠遇。},曰:“杜元凯乃复连榻坐客\footnote{杜元凯:杜预,字元凯。乃复:竟然。}!”不坐便去。{\fzxk\zihao{6}\textcolor{red}{\CJKunderwave{晋诸公赞}曰:“羊琇字稚舒,泰山人。通济有才干。与世祖同年相善,谓世祖曰:‘后富贵时,见用作领、护军各十年。’世祖即位,累迁左将军、特进。”}} 杜请裴追之,羊去数里住马,既而俱还杜许\footnote{既而:然后,过后。许:处所。}。

{\cangkai\zihao{5}【评】魏晋世族社会,“政失准的,士无特操”,高门士人上演了一幕幕荒诞丑剧。羊琇与石崇、王恺等人斗富,暴殄天物,触目惊心。羊琇乃景帝司马师夫人的堂弟,武帝司马炎的少时玩伴、立储功臣。羊琇这般国家蛀虫,虽对国家进步和人民的福祉毫无贡献,与杜预这样的实干家比起来,连垃圾也不如;但是摆起排场来,却神气活现。或许其一生立身行事实在是无可圈点,只能炫耀其贵族的头衔和外戚的血统,以满足其空虚的心理。其挟贵而骄,并非由于才、德卓荦不群,有傲人的坚实资本,而是自恃外戚与帝友的特殊身份,滋长了其不可一世的傲慢与偏见。王世懋曰:“羊琇何物,与王恺为戚里争富者,乃亦以慢镇南为方正耶?”本篇列羊琇入“方正”,与上篇列杨济入“方正”,均见世族社会对方正的理解,受时代风气制约,已陷入无可挽回的怪圈。}

\lettrine{5.14} 晋武帝\myidx{司马炎}时,荀勗\myidx{荀勗}为中书监\footnote{荀勗(?—289):字公曾,西晋颍川颍阴(今河南许昌)人,初仕曹魏。司马炎代魏称帝后,党附贾充父女,为人谄佞,为士林不齿。中书监:官名。中书省的副职。},{\fzxk\zihao{6}\textcolor{red}{虞预\CJKunderwave{晋书}曰:“勗字公曾,颍川颍阴人,汉司空爽曾孙也。十馀岁能属文,外祖锺繇曰:‘此儿当及其曾祖。’为安阳令,民生为立祠。累迁侍中、中书监。”}} 和峤\myidx{和峤}为令\footnote{和峤:和峤(?—292):魏晋时汝南西平(今属河南)人。官至中书令。为政清简得民,有风格,善礼法,朝野许其能正风俗人伦。家财富而性至吝,人称有“钱癖”。大丧:指父母之丧。据\CJKunderwave{晋书}戎传,时戎遭母丧,而峤遭父丧。令:指中书令。掌机密,传宣诏令。始设于汉,以宦官充任。后多任用有名望或亲近者。}。故事\footnote{故事:先例,旧有的典章制度。}:监、令由来共车\footnote{由来:从来,向来。共车:共乘一辆公车。}。峤性雅正\footnote{雅正:端方正直。},常疾勗谄谀\footnote{疾:憎恨。谄谀:讨好巴结奉承人。}。{\fzxk\zihao{6}\textcolor{red}{王隐\CJKunderwave{晋书}曰:“勗性佞媚,誉太子,出齐王。当时私议:损国害民、孙刘之匹也。后世若有良史,当箸\CJKunderwave{佞幸传}。”}} 后公车来,峤便登,正向前坐,不复容勗。勗方更觅车,然后得去。监、令各给车,自此始。{\fzxk\zihao{6}\textcolor{red}{曹嘉之\CJKunderwave{晋记}曰:“中书监、令常同车入朝,至和峤为令,而荀勗为监,峤意强抗,专车而坐。乃使监、令异车,自此始也。”}}

{\cangkai\zihao{5}【评】和峤不愿与荀勖共载,是鄙薄其为人,与高门狂士挟贵骄人,有着本质的区别。荀勖贵为社稷辅弼,不能止恶扬善、主持正义,却一味佞媚权贵、曲阿上意,缺乏独立的人格操守,落入唯求自保乌纱之流。和峤是坚持真理的方正之士,在武帝册立储君问题上,昭示出二人判若天地的人格差距。从西方现代民主精神角度看,知识分子非唯某一方面之专才,实乃社会的良心,应在社会事务中发出自己的声音。中国儒家传统赋予士人“为天地立心,为生民立命,为往圣继绝学,为万世开太平”的铁肩道义,“一肩挑尽古今愁”,形象地道出知识分子九死不悔的崇高追求。和峤“宁鸣而死,不默而生”,与荀勖丧尽气节两种表现,折射出古今中外知识分子两种不同的人生选择。}

\lettrine{5.15} 山公\myidx{山涛}大儿\myidx{山该}短,箸帢\footnote{山公:山涛,见\CJKunderwave{言语}注。大儿:长子。帢(qià洽):曹操创制的一种便帽,形如弁而无四角,用缣帛缝制。以颜色不同区别贵贱。},车中倚。武帝\myidx{司马炎}欲见之\footnote{武帝:晋武帝司马炎。},山公不敢辞。问儿,儿不肯行。时论乃云胜山公。{\fzxk\zihao{6}\textcolor{red}{\CJKunderwave{晋诸公赞}曰:“山该字伯伦,司徒涛长子也。雅有器识,仕至左卫将军。”}}

{\cangkai\zihao{5}【评】李慈铭、余嘉锡等前贤以为故事之主人公或为山涛第三子允,有理。\CJKunderwave{晋书}载允“少尫病,形甚短小”。山允之所以不答应武帝见面的要求,当出于自惭形秽的心理障碍,并不关涉方正。刘辰翁曰:“直自愧其矮耳,不足言胜。”可谓一语中的。汉魏以降,人物品藻风气已由昔日重操守名节转向重外在形貌及其风度气概。山允之自我封闭,一如魏武见匈奴使节,自以形貌丑陋而使美男子崔琰“捉刀”,同为“发现自我”的时势风会使然。}

\lettrine{5.16} 向雄\myidx{向雄}为河内主簿\footnote{向雄:字茂伯,西晋河内山阳(今河南修武西北)人。初仕魏为郡主簿,迁都官从事。入晋,以固谏忤晋武帝,忧愤而卒。河内:郡名。晋代治所在野王(今河南泌阳县)。主簿:中央或地方郡县设的属官。掌文书簿籍及印鉴。},有公事不及雄\footnote{公事:公家的事务,亦指公事文书。不及雄:没有送到向雄处。},而太守刘淮(准)\myidx{刘淮}横怒\footnote{刘淮:\CJKunderwave{晋书·向雄传}作刘毅。丁国钧\CJKunderwave{晋书校文}曰:“考仲雄(毅字)传,既未为河内太守,亦未迁侍中,则此文刘毅当为刘准之误。”准字君平,西晋沛国(治所在今安徽濉溪西北)人。太守:郡的最高行政长官。横怒:暴怒,没来由的发怒。},遂与杖遣之\footnote{与杖:给予杖责。遣:遣送。指罢官,赶走。}。雄后为黄门郎\footnote{黄门郎:黄门为魏晋宫内官署,黄门郎即黄门侍郎。职为侍从皇帝,传达诏命等。与侍中俱掌门下众事。},刘为侍中\footnote{侍中:官名。魏晋间常置专职者四人,侍从皇帝左右,预闻朝政,为亲信贵重之职。},初不交言。武帝\myidx{司马炎}闻之,敕雄复君臣之好\footnote{敕:皇帝的命令。君臣:东汉魏晋州郡长官与僚属之间,视为君臣关系。太守为君,府吏为臣。}。雄不得已,诣刘再拜曰\footnote{再拜:拜了又拜,古礼,拜两次,以表隆敬。}:“向受诏而来,而君臣之义绝,何如?”于是即去。武帝闻尚不和,乃怒问雄曰:“我令卿复君臣之好,何以犹绝?”{\fzxk\zihao{6}\textcolor{red}{\CJKunderwave{汉晋春秋}曰:“雄字茂伯,河内人。”\CJKunderwave{世语}曰:“雄有节概,仕至黄门郎、护军将军。”案:王隐\CJKunderwave{孙盛不与故君相闻议}曰:“昔在晋初,河内温县领牧(校)向雄,送御牺牛,不充(先)呈郡。辄随此比送洛,值天大热,郡送牛多暍(渴)死。台法甚重,太守是(吴)奋召雄与仗,雄不受杖,曰:‘郡牛者亦死也,呈牛者亦死也。’奋大怒,下雄狱,将大治之。会司隶辟雄都官从事。数年,为黄门侍郎,奋为侍中,同省,相避不相见。武帝闻之,给雄酒礼,诣奋解。雄乃奉诏。”此则非刘淮(准)也。\CJKunderwave{晋诸公赞}曰:“淮(准)字君平,沛国杼秋人。少以清正称,累迁河内太守、侍中、尚书仆射、司徒。”}} 雄曰:“古之君子,进人以礼,退人以礼\footnote{“古之君子”三句:摘自\CJKunderwave{礼记·檀弓下},谓当初刘准杖责而驱逐向雄是不合礼的。进:进用,提拔。退:斥退,罢职。}。今之君子,进人若将加诸膝,退人若将坠诸渊\footnote{“今之君子”三句:语本\CJKunderwave{礼记·檀弓}。谓刘准用人只凭私心爱憎。}。臣于刘河内\footnote{刘河内:指刘准,准曾任河内郡太守。},不为戎首\footnote{戎首:发动战争的主谋者。比喻刀兵相见。},亦已幸甚,安复为君臣之好?”武帝从之。{\fzxk\zihao{6}\textcolor{red}{\CJKunderwave{礼记}曰:“穆公问于子思曰:‘为旧君反服,古邪?’子思曰:‘古之君子,进人以礼,退人以礼,故有旧君反服之礼。今之君子,进人若将加诸膝,退人若将坠诸渊。无为戎首,不亦善乎,又何反服之有?’”郑玄曰:“为兵主来攻伐,故曰戎首也。”}}

{\cangkai\zihao{5}【评】现代心理学研究表明,人的诸多外在行为,可以从神经气质类型理论得到解释。大概太守刘准属于胆汁质类型,情绪爆发性极强,有不可遏止的冲动。他仅因一件小事,就大发雷霆,对主簿向雄施以杖责并开除革职。翻脸无情,毫无人道可言,令身边工作人员如履薄冰,又怎能激发创造性,提高办事效率呢?后来,刘、向二人狭路相逢,又成了同一官署共事的上下级。\CJKunderwave{晋书·职官志}载,黄门侍郎与侍中俱管门下众事,而侍中“备切问近对,拾遗补缺”,地位高于黄门侍郎。武帝司马炎出于安定团结的考虑,令向雄抛出和平的橄榄枝,这对于有人格尊严的士大夫而言,无异于苟媚求和、自贱身价。向雄依礼而动,据礼而言,维护了自身尊严,可钦可敬。如果没有点儿“威武不能屈”的方正品格,早就卑躬屈膝地去巴结上司了。}

\lettrine{5.17} 齐王冏\myidx{司马冏}为大司马辅政\footnote{齐王冏:司马冏(?—302),西晋皇族,字景治。齐王司马攸子,嗣封齐王。后为长沙王司马乂所杀。大司马:官名。晋置大司马与大将军和丞相共掌朝政。},{\fzxk\zihao{6}\textcolor{red}{虞预\CJKunderwave{晋书}曰:“冏字景治,齐王攸子也。少聪惠,及长,谦约好施。赵王伦篡位,冏起义兵诛伦。拜大司马,加九锡,政皆决之。而恣用群小,不复朝觐,遂为长沙王所诛。”}} 嵇绍\myidx{嵇绍}为侍中\footnote{嵇绍:字延祖,康子,官至侍中。},诣冏谘事\footnote{诣:到……去。咨事:请示公事。}。冏设宰会\footnote{设宰会:设置酒宴邀请僚属集会。宰,官员通称。},召若旟\myidx{若旟}、{\fzxk\zihao{6}\textcolor{red}{\CJKunderwave{齐王官属}名曰:“旟字虚旟,齐王从事中郎。”\CJKunderwave{晋阳秋}曰:“齐王起义,转长史。既克赵王伦,与董艾等专执威权。冏败见诛。”}} 董艾\myidx{董艾}等{\fzxk\zihao{6}\textcolor{red}{\CJKunderwave{八王故事}曰:“艾字叔智,弘农人。祖遇,魏侍中。父绥,秘书监。艾少好功名,不修士检。齐王起义,艾为新汲令,赴军,用艾领右将军。王败见诛。”}} 共论时宜\footnote{若旟:袁本及\CJKunderwave{晋书}嵇绍传均作“葛旟”,是。葛旟,西晋时齐王司马冏属官。董艾:齐王冏亲信,领右将军。时宜:指时政,适应时势的政治措施。}。旟等白冏:“嵇侍中善于丝竹\footnote{丝竹:弦乐器和管乐器。泛指乐器。},公可令操之。”遂送乐器。绍推却不受,冏曰:“今日共为欢,卿何却邪?”绍曰:“公协辅皇室,令作事可法\footnote{可法:切合法度。}。绍虽官卑,职备常伯\footnote{常伯:周代官名,相当于九卿一类的高级官职。}。操丝比竹盖乐官之事\footnote{操丝比竹:谓演奏乐器。乐官:掌管音乐的官吏。},不可以先王法服,为伶人之业\footnote{法服:礼法规定的标准服。\CJKunderwave{孝经·卿大夫}:“非先王之法服不敢服。”注:“先王制五服,各有等差,言卿大夫遵守礼法,不敢僭上逼下。”伶人:乐人,乐工。古时从事音乐的艺人被轻视。}。今逼高命\footnote{高命:尊贵的命令,敬辞。},不敢苟辞,当释冠冕\footnote{冠冕:古代帝王、官员所戴的有等级区别的礼帽。此泛指官服。},袭私服\footnote{袭:穿。私服:便服。}。此绍之心也。”旟等不自得而退。

{\cangkai\zihao{5}【评】有晋一代,“虽背恩忘义之徒不可胜载,而蹈节轻生之士无乏于时”(\CJKunderwave{晋书·忠义传})。嵇绍乃名士嵇康之子,入仕后在政治的浊流中始终能够站稳脚跟,拒绝外戚贾谧的拉拢,后又勠力王事、公忠体国而杀身成仁。与此则故事不为齐王操伶人之事,俱为方正品格做了最好的注脚。齐王冏及葛、董诸人,令嵇绍在集会上操管弄弦,从表面上看,诸人似分不清公私场合,一时糊涂。实际上则无意中暴露了这群人平日的所思所想。因心中并不把国家大事置于至高无上的地位,故对日常案牍工作无所用心,甚或醉生梦死,唯以享受为重。今晚的宴会在哪里摆?宴后安排一些什么内容?乌七八糟的东西横亘于胸,国家的事业焉有半点位置?嵇绍有理有据地给“衮衮诸公”们上了一堂职业道德课。嵇绍的刚直不阿,“激清风于万古,厉薄俗于当年”,为两晋暗淡的政治舞台增添了一丝亮色。}

\lettrine{5.18} 卢志\myidx{卢志}于众坐{\fzxk\zihao{6}\textcolor{red}{\CJKunderwave{世语}曰:“志字子通,范阳人,尚书珽少子。少知名,起家邺令,历成都王长史、卫尉卿、尚书郎。”}} 问陆士衡\myidx{陆机}\footnote{卢志:字子道,西晋范阳涿(今属河北)人。早知名。陆士衡:陆机字士衡,吴郡吴人(今苏州)人。参\CJKunderwave{晋书}本传,其为吴郡吴县华亭(今上海松江)人,当时著名的文学家。吴亡入晋后,累迁太子洗马、著作郎。曾任平原内史,故称“陆平原”。事成都王颖,颖兴兵攻掌权于洛阳的长沙王司马乂时,任陆机为后将军、河北大都督。机兵败遭谗,与弟陆云同为颖所杀。}:“陆逊、陆抗,是君何物\footnote{陆逊(183—245):字伯言,三国吴郡吴县华亭(今上海松江)人。累世为江东大族,官至丞相。陆机、陆云之祖父。陆抗:字幼节,陆逊子。历官江陵都督、大司马、荆州牧。何物:什么人。}?”{\fzxk\zihao{6}\textcolor{red}{抗已见。\CJKunderwave{吴书}曰:“逊字伯言,吴郡人,世为冠族。初领海昌令,号‘神君’。累迁丞相。”}} 答曰:“如卿于卢毓、卢珽\footnote{卢毓:字子家,卢志祖父。卢珽:字子笏。卢毓子,卢志父。}。”{\fzxk\zihao{6}\textcolor{red}{\CJKunderwave{魏志}曰:“毓字子家,涿人。父植,有名于世。累迁吏部郎、尚书。选举,先性行而后言才。进司空。珽,咸熙中为泰山太守,字子笏,位至尚书。”}} 士龙\myidx{陆云}失色\footnote{士龙:陆云字士龙,陆机弟。儒雅有俊才,善著述,官至清河内史。及兄机兵败,同被谗杀。失色:因受惊或害怕而改变脸色。},{\fzxk\zihao{6}\textcolor{red}{云别见。}} 既出户,谓兄曰:“何至如此?彼容不相知也\footnote{容:或许,可能。}。”士衡正色曰:“我父祖名播海内,宁有不知。鬼子敢尔\footnote{鬼子:骂人的话。敢尔:竟敢如此。}!”{\fzxk\zihao{6}\textcolor{red}{\CJKunderwave{孔氏志怪}曰:“卢充者,范阳人。家西三十里有崔少府墓。充先冬至一日出家西猎,见一獐,举弓而射,即中之。獐倒而复起,充逐之,不觉远。忽见一里门如府舍,门中一铃下,有唱家前。充问:‘此何府也?’答曰:‘少府府也。’充曰:‘我衣恶,那得见贵人。’即有人提襆新衣迎之。充箸,尽可体。便进见少府,展姓名。酒炙数行,崔曰:‘近得尊府君书,为君索小女婚,故相延耳。’即举书示充。充父亡时虽小,然已见父手迹,便歔叹无辞。崔即敕内,令女郎庄严,使充就东廊。充至,妇已下车,立席头共拜。三日毕,还见崔。崔曰:‘君可归矣!女有娠相,生男当以相还。生女当归自养。’敕外严车送客。崔送至门,执手零涕,离别之感,无异生人。复致衣一袭,被褥一副。充便上车,去如电逝,须臾至家。家人相见,悲喜推问。知崔是亡人,而入其墓,追以懊惋。居四年,三月三日临水戏。忽见一犊车,乍浮乍没。既上岸,充往开车后户,见崔氏女与三岁男儿共载。充见之,忻然欲捉其手。女举手指后车曰:‘府君见人。’即见少府。充往问讯,女抱儿还充。又与金碗,别,并赠诗曰:‘煌煌灵芝质,光丽何猗猗!华艳当显时,嘉异表神奇。含英未及秀,中夏罹霜萎。荣曜长幽灭,世路永无施。不悟阴阳运,哲人忽来仪。会浅离别速,皆由灵与祇。何以赠余亲,金碗可颐儿。爱恩从此别,断绝伤肝脾。’充取儿、碗及诗,忽不见二车处。将儿还,四坐谓是鬼媚,佥遥唾之,形如故。问儿:‘谁是汝父?’儿径就充怀。众初怪恶,传省其诗,慨然叹死生之玄通也。充诣市卖碗,高举其价,不欲速售,冀有识者。欻有一老婢问充得碗之由,还报其大家,即女姨也。遣视之,果是。谓充曰:‘我姨姊崔少府女,未嫁而亡,家亲痛之,赠一金碗著棺中。今视卿碗甚似。得碗本末,可得闻不?’充以事对。即诣充家迎儿。儿有崔氏状,又似充。姨曰:‘我舅生三月末间产。父曰:“春暖温也,愿休强也。”即字温休,“温休”,盖幽婚也。其兆先彰矣!’儿遂成为令器,历数邪(郡)二千石,皆著绩。其后生植。为汉尚书。植子毓,为魏司空。冠盖相承至今也。”}} 议者疑二陆忧(优)劣,谢公以此定之\footnote{谢公:谢安。定之:论定二人高下。}。

{\cangkai\zihao{5}【评】魏晋门阀制度沿袭既久,逐渐养成士族子弟傲视苍生、惟我独尊的狂妄文化心理。而西晋灭吴后,士族门阀中地方宗派的南北对立,是这种文化心理的一个重要表现。以陆机、陆云为代表的江南士族,受到中原士族的歧视,进而引发了一场南北士人的对抗。陆逊、陆抗父子,是吴国名将,战功卓著,英名盖世。在重士族门第的社会交际中,卢志怎能不晓二陆的大名?可见他是有意挑衅,以示作为战胜者的中原士族对江南士族的优越感。陆机针锋相对、反唇相讥,直道卢志父祖毓、珽之名,全然不顾后果,终为日后谗诛埋下隐患。两晋时代,南北士族对抗,力量内耗,愈演愈烈,成为加速政权灭亡的掣肘因素。李斯\CJKunderwave{谏逐客书}云“地无四方,民无异国,四时充美,鬼神降福”,这从另一方面说明了为政者应有的宽容博大心态。}

\lettrine{5.19} 羊忱\myidx{羊忱}性甚贞烈\footnote{羊忱:(?—311):一名陶,字长和,西晋泰山(在今山东)人。死于永嘉之乱。贞烈:正直刚烈。},赵王伦\myidx{司马伦}为相国\footnote{赵王伦:赵王司马伦,裴令公:即裴楷,曾官中书令,故云,又称“裴令”。善\CJKunderwave{老}、\CJKunderwave{易},当时著名清谈名家。二国租钱:指从梁、赵二国税收所获钱财。相国:宰相。},忱为太傅长史\footnote{长史:魏晋时,丞相、三公、都督府、将军府均设长史,为辅佐官吏。},乃版以参相国军事\footnote{版:书写在木板上的官府文书,凡王封官用版,称为“版官”。此指赵王伦以版诏授羊忱官职。参相国军事:相国手下的参军事官。}。使者卒至\footnote{卒:通“猝”。突然。},忱深惧豫祸\footnote{豫祸:牵连受祸。豫,通“与”。},不暇被马\footnote{被马:即鞴马,装配鞍鞯、缰勒。},于是帖骑而避\footnote{帖骑:贴身于马背,跨骑不配鞍鞯的马。}。使者追之,忱善射,左右发,使者不敢进,遂得免。{\fzxk\zihao{6}\textcolor{red}{\CJKunderwave{文字志}曰:“忱字长和,一名陶,泰山平阳人。世为冠族。父疏(‘疏’,袁各本作‘繇’,是),车骑掾。忱历太傅长史、扬州刺史,迁侍中。永嘉五年,遭乱被害,年五十馀。”}}

{\cangkai\zihao{5}【评】一个有独立思考精神的知识分子,应该知道有所为、有所不为,在大是大非面前,保持清醒头脑。羊忱就是这样一位性格贞正刚烈,深知福祸倚伏之理的清醒者。赵王伦打着“清君侧”的所谓正义之旗,诛灭贾后一党,不过是以恶制恶的政治把戏而已,并非正义之举,内心是觊觎帝位,欲行篡逆之实。羊忱挣脱司马伦的名利网罗,与那些对权力趋之若鹜的俗客形成鲜明对比。儒家\CJKunderwave{论语}中有“危邦不入,乱邦不居”、“邦有道则仕,邦无道则卷而怀之”等警语;道家智慧里亦有福祸相因、避害全生的训诫。羊忱把这些古训名言化成自己的人生指南,不能不说是智者的抉择。}

\lettrine{5.20} 王太尉\myidx{王衍}不与庾子嵩\myidx{庾敳}交\footnote{王太尉:王衍,王夷甫:王衍(256—311)字夷甫,见刘孝标注。“以清虚通理称”,为当时清谈名家,“妙悟若神”,“妙善玄言,唯谈\CJKunderwave{老}、\CJKunderwave{庄}为事”。为政多谋略,不以经国为念,而善思自全之计,然终为石勒所害。(见\CJKunderwave{晋书}本传)注。庾子嵩:庾敳。},{\fzxk\zihao{6}\textcolor{red}{王夷甫、庾敳}} 庾曰(日)卿之不置\footnote{曰:据袁本,“曰”字为衍字;或据沈剑知\CJKunderwave{世说新语校笺}(刊于\CJKunderwave{学海}第一卷1、2、3、6期及第二卷第1期)“曰”当为“日”。于义皆通。卿:对对方比较亲近而随便的称呼,相当于“你”。不置:不止,不已。}。王曰:“君不得为尔。”庾曰:“卿自君我\footnote{君我:用“君”称呼我。},我自卿卿\footnote{卿卿:用“卿”称呼你。};我自用我法,卿自用卿法。”

{\cangkai\zihao{5}【评】太尉王衍原是雅重庾敳,二人俱是不论世事、唯雅咏玄虚的玄学领袖,思想渊源是息息相通的。何以王衍后来一百八十度地大转弯,对庾敳的“卿之不置”不予理睬呢?原来“君”乃对人之尊称,而“卿”则魏晋以来对爵位较低或平辈表示亲近的称呼。王衍是极其矜持的士林领袖,“入眼平生未曾有”,是其待人处世的一贯作风。庾敳以“卿”呼衍,等于平辈相称,不经意间对其身份构成了公然挑战。是可忍,孰不可忍?但这层微妙心理又不好明言,于是王衍干脆来个沉默战术,以不回答来表示自己的抗议,直到你罢口为止。庾敳参透此中消息,但却仍然我行我素,而不顾对方的感受,在强势者面前并不低头屈曲,依然张扬个性,顽强保持自己平等的人格。魏晋士人因此称之为“方正”。}

\lettrine{5.21} 阮宣子\myidx{阮修}伐社树\footnote{阮宣子:阮修字宣子,阮籍从子,好\CJKunderwave{易}理,善清言。性简约、任诞。晋代无神论者。社:土地神或祭祀土地神的地方,如社庙、社坛、社宫。此指设立土地神坛。立社种树,作社的标志,称社树。},{\fzxk\zihao{6}\textcolor{red}{阮修,已见。\CJKunderwave{春秋传}曰:“共工氏有子曰勾龙,为后土,后土为社。”\CJKunderwave{风俗通}曰:“\CJKunderwave{孝经}称,社者土也,广博不可备敬,故封土以为社而祀之,报功也。”然则社自祀勾龙,非土之祭也。}} 有人止之。宣子曰:“社而为树,伐树则社亡;树而为社,伐树则社移矣。”

{\cangkai\zihao{5}【评】子曰:“未知人,焉知鬼”,故“不语怪力乱神”。儒家思想怀疑鬼神的存在,但却要祭鬼神,取“祭如在”的态度。社乃土地之神,在社种树为社的标志。\CJKunderwave{汉书·孔安国传}曰:“王者封五色土为社,建诸侯,则各割其方色土与之,使立社”;又\CJKunderwave{白虎通义}三\CJKunderwave{社稷}言:“人非土不立,非谷不食……故封土立社,示有土地。”社为国家政权的标志,为历代王朝所重视。玄学思想辨名析理,重视理性思考,对于鬼神的态度是检验士人理性思辨水平高低的试金石。很多人都在此处望而却步了。阮宣子是有唯物思想的清谈家,不随从流俗,坚持独立思考,确实难能可贵。}

\lettrine{5.22} 阮宣子\myidx{阮修}论鬼神有无者\footnote{阮宣子:见前则。}。或以人死有鬼,宣子独以为无,曰:“今见鬼者云,箸生时衣服;若人死有鬼,衣服复有鬼邪?”{\fzxk\zihao{6}\textcolor{red}{\CJKunderwave{论衡}曰:“世谓人死为鬼,非也。人死不为鬼,无知,不能害人。如审鬼者死人精神,人见之,宜从裸祖(袒)之形,无为见衣带被服也。何则?衣无精神也。由此言之,见衣服象人,则形体亦象人。象人,知非死人之精神也。凡天地之间有鬼,非人死之精神也。”}}

{\cangkai\zihao{5}【评】魏晋的玄学清谈,可以说是先秦百家争鸣的继承和发展,是一次因时适势的学术交锋。其具体内容,如有无、本末、言意、形神、神灭诸论,都体现了魏晋时代的新思考。鬼神观念,在中国民间文化“小传统”中有根深蒂固的思想土壤。佛教传入后,“彼岸世界”又在一定程度上强化了国人心中的冥界意识。阮修与人辩论鬼神之有无,属玄学内部的理论探讨,其回答机智幽默,持之有据。不要说在遥远的六朝时代,即便在今天,鬼神观念也还在某些人(甚至在接受过高等教育的知识分子)中间大行其道,在特定条件下,还会以一种变化了的方式沉渣泛起,成为一种兴风作浪的社会势力。南宋刘辰翁评曰:“振古绝俗,得意之名言。”可见,阮修持无神论在当时是多么的难能可贵。对于后来南朝范缜的神灭论,当起到开先河之功。}

\lettrine{5.23} 元皇帝\myidx{司马睿}既登祚\footnote{元皇帝:指晋元帝司马睿,东晋第一主。东晋第一位皇帝,在位七年,庙号“中宗”。登祚:登基,即位做皇帝。},以郑后之宠\footnote{郑后:小字阿春。建武元年,晋元帝纳为夫人,生简文帝。晋孝武时,追尊为太后。},欲舍明帝\myidx{司马绍}而立简文\myidx{司马昱}\footnote{明帝:东晋明帝司马绍(299—325),元帝长子,东晋第二主。简文:晋简文帝司马昱,晋简文:指晋简文帝司马昱(320—372),穆帝年幼即位,昱任抚军大将军总理政务。后来大将军桓温专擅朝政,先废海西公,后立司马昱为帝,第二年崩。}。时议者咸谓舍长立少,既于理非伦\footnote{舍长立少:指舍掉长子而立少子为太子皇储。非伦:不合常道。封建宗法制以立嫡立长为常道。},且明帝以聪亮英断,益宜为储副\footnote{储副:储君,太子。皇位继承人。}。周\myidx{周顗}、王\myidx{王导}诸公并苦争恳切\footnote{周、王:指周顗、王导。周、王是辅佐晋元帝之重臣。},{\fzxk\zihao{6}\textcolor{red}{\CJKunderwave{中兴书}曰:“郑太后字阿春,荥阳人。少孤,先嫁田氏,夫亡,依舅氏。时中宗敬后虞氏先崩,将纳吴氏。后与吴氏女游后园,有言之于中宗者,纳为夫人。甚宠,生简文。帝即位,尊之曰文宣太后。”}} 唯刁玄亮\myidx{刁协}独欲奉少主以阿帝旨\footnote{刁玄亮:刁协(?—322),字玄亮,东晋勃海铙安(今河北盐山南)人,晋元帝亲信近臣。协久在内朝,谙练旧事,中兴制度,多为协所建,于朝廷制度多所谋划。性刚悍,好媚上抑下。后为王敦所杀。奉:尊奉,拥戴。阿:曲从,奉迎。旨:意旨,心意。}。元帝便欲施行,虑诸公不奉诏,于是先唤周侯、丞相入\footnote{周侯:周顗。丞相:王导。},然后欲出诏付刁。周、王既入,始至阶头,帝逆遣传诏遏使就东厢\footnote{逆:预先。传诏:皇帝身边供役使差遣的人。遏:阻止。}。周侯未悟,即却略下阶\footnote{却略:倒退着走。}。丞相披拨传诏\footnote{披拨:用手拨开。},径至御床前\footnote{御床:御座,皇帝宝座。床,坐具。},曰:“不审陛下何以见臣?”帝默然无言,乃探怀中黄纸诏裂掷之。由此皇储始定。周侯方慨然愧叹曰:“我常自言胜茂弘\myidx{王导}\footnote{茂弘:王导字。},今始知不如也!”{\fzxk\zihao{6}\textcolor{red}{\CJKunderwave{中兴书}曰:“元皇以明帝及琅邪王裒,并非敬后所生,而谓裒有大成之度,胜于明帝。因从容问王导曰:‘立子以德不以年。今二子孰贤?’导曰:‘世子、宣城俱有爽明之德,莫能优劣,如此,故当以年。’于是更封裒为琅邪王。”而此与\CJKunderwave{世说}互异。然法盛采摭典故,以何为实。且从容讽谏,理或可安。岂有登阶一言,曾无奇说,便为之改计乎?}}

{\cangkai\zihao{5}【评】“五胡乱华”,司马南渡。王导拥戴司马睿创基东晋,訏谟定命,鞠躬尽瘁。史称“王与马,共天下”,非虚誉也。封建宗法社会,“立嫡以长”乃是礼法常规,元帝却因经不住郑后的枕边风,欲舍长立幼,采取反常规做法,极易成为日后祸起萧墙的导火索。深谙古今之变的政治家王导,鉴于晋室诸王同室操戈的历史教训,坚持提议聪亮英断的长子司马绍为合适人选,君臣之间为此互不相让。王导的果决,令人想起西晋开国之初的名臣和峤,二人抗颜直谏的方正品格如出一辙。另外,故事中,周顗面临突发事件时,丈二和尚摸不着头,沉着应对能力逊于王导,或许是为映衬王导的“光辉形象”而采取的“小说家言”。但从文学角度看,则从一个侧面烘托了王导多谋善断的优秀政治家形象。}

\lettrine{5.24} 王丞相\myidx{王导}初在江左\footnote{王丞相:王导。初在江左:刚到江东。西晋灭亡,王导辅佐琅邪王司马睿在建康建立东晋政权。江左:长江下游以东地区。指东晋辖区。古人叙地理以东为左,以西为右,故称江东为江左。},欲结援吴人\footnote{结援:结交以求得援助。吴人:江左本吴郡之地域,故称江左人氏为吴人。此指江东的世家大族,如吴郡的朱、张、顾、陆等。},请婚陆太尉\myidx{陆玩}\footnote{请婚:谓王导向陆玩请求通婚。陆太尉:指陆玩。吴郡吴人。卒后追赠太尉。}。对曰:“培塿无松柏\footnote{培塿:小土丘,小山。},薰莸不同器\footnote{薰莸不同器:香草与臭草不能放在同一个容器之中。薰,香草;莸,臭草。}。{\fzxk\zihao{6}\textcolor{red}{杜预\CJKunderwave{左传注}曰:“培楼(塿),小阜;松柏,大木也。薰,香草;犹(莸),臭草。”}} 玩虽不才\footnote{不才:不成材,不是人才。},义不为乱伦之始\footnote{乱伦:乱人伦。这里指门第不相当而结为婚姻。在门阀制度下,高门士族不和寒门庶族通婚。}。”{\fzxk\zihao{6}\textcolor{red}{玩已见。}}

{\cangkai\zihao{5}【评】东晋南渡之初,中原豪强的力量相对地被削弱,而江南士族豪强则因为很少受战争打击而保存并日渐扩大其影响和实力。南北士族力量此消彼长,发生了相应变化。以王导为首的中原士族之有识之士,为调动一切可能力量,采取了一系列团结南人的策略。其请婚陆太尉,就是其中事关政治大局的感情投资。太尉陆玩为江左名族,琅邪王氏为中原士族之冠,王导又是朝廷辅政大臣,王、陆通婚,陆家为攀龙附凤,何乐而不为?但事实是陆玩不想高攀而拒婚,时人以气骨而称其方正,似不无道理。但细忖度之,似乎另有更加深层的原因:南北士族对立,由来已久,在长期对抗中,南人处于被压制、被轻侮的劣势地位,而今世异时移,中原士族偏安江南,寄人篱下,南人心理优势占上风。陆玩之婉拒王导,是长期受压抑后的心理反弹,此其一;陆玩为陆机从弟,机、云兄弟惨死中原士族之手,手足情亲,记忆犹新。此其二。李贽曰:“今之恃势者,可羞也。”以陆玩为方正,似只见其表,而未窥全豹,应综合考量。}

\lettrine{5.25} 诸葛恢\myidx{诸葛恢}大女适太尉庾亮\myidx{庾亮}儿\myidx{庾会}\footnote{诸葛恢:字道明,东晋阳都(今山东浙水南)人。诸葛靓子。西晋乱,避地江左。适:嫁。庾亮儿:指庾会。会字会宗,太尉庾亮长子,娶诸葛恢女,名文彪。},{\fzxk\zihao{6}\textcolor{red}{\CJKunderwave{恢别传}曰:“恢字道明,琅邪阳都人。祖诞,司空。父靓,亦知名。恢少有令问,称为明贤。避难江左,中宗召补主簿,累迁尚书令。”\CJKunderwave{庾氏谱}曰:“庾亮子会,娶恢女,名文彪。庾会别见。”}} 次女适徐州刺史羊忱\myidx{羊忱}儿\myidx{羊楷}\footnote{羊忱:(?—311):一名陶,字长和,西晋泰山(在今山东)人。死于永嘉之乱。}。{\fzxk\zihao{6}\textcolor{red}{\CJKunderwave{羊氏谱}曰:“羊楷字道茂。祖繇,车骑掾。父忱,侍中。楷仕至尚书郎,娶诸葛恢次女。”}} 亮子被苏峻\myidx{苏峻}害\footnote{亮子被苏峻害:晋明帝崩,庾亮掌朝政,苏峻素疑庾亮欲加害,故以讨亮为名起兵反,攻陷京城。晋成帝咸和六年,庾亮子庾会被杀。},改适江虨\myidx{江虨}\footnote{江虨:字思玄,东晋陈留(今河南开封东北)人。江统子。为晋中兴大臣,累官至尚书左仆射。}。{\fzxk\zihao{6}\textcolor{red}{虨别见。}} 恢儿\myidx{诸葛衡}娶邓攸\myidx{邓攸}女\footnote{邓攸:此指避永嘉之乱,时邓攸被匈奴族石勒部俘虏,后逃出。}。{\fzxk\zihao{6}\textcolor{red}{\CJKunderwave{诸葛氏谱}曰:“恢子衡,字峻文。仕至荥阳太守。娶河南邓攸女。”}} 于时谢尚书\myidx{谢裒}求其小女婚\footnote{谢尚书:谢裒,字幼儒,东晋陈郡阳夏(今河南太康)人;谢安之父。},恢乃云:“羊、邓是世婚,江家我顾伊\footnote{世婚:世代有通婚关系的亲戚。顾;照顾:顾念。},庾家伊顾我,不能复与谢裒儿\myidx{谢石}婚。”{\fzxk\zihao{6}\textcolor{red}{\CJKunderwave{永嘉流人名}曰:“裒字幼儒,陈郡人。父衡,博士。裒历侍中、吏部尚书、吴国内史。”}} 及恢亡,遂婚。{\fzxk\zihao{6}\textcolor{red}{\CJKunderwave{谢氏谱}曰:“裒子石,娶恢小女,名文熊。”\CJKunderwave{中兴书}曰:“石字石奴,历尚书令。聚敛无厌,取讥当世。”}} 于是王右军\myidx{王羲之}往谢家看新妇\footnote{王右军:王羲之。看新妇:是古礼。晋、宋以来,初婚三日,妇见翁姑,众宾皆列观。},犹有恢之遣法\footnote{遣法:明袁氏本作“遗法”,各可成说。}:威仪端详\footnote{威仪:容貌举止。端详:端庄安详。},容服光整\footnote{容服:仪容服饰。光整:华美整洁。}。王叹曰:“我在遣女,裁得尔耳\footnote{在:于。遣女:嫁女。裁:通“才”。尔:如此。}?”

{\cangkai\zihao{5}【评】东晋渡江之初,王、葛为著姓。于时王氏为将军,而诸葛兄弟并居显要。阮思旷讥谢万为“新出门户,笃而无礼”,可见时人尚不以陈郡谢氏为高门世族。至于王、谢并称,则自谢安时始。诸葛氏婚配的挑挑拣拣,是出于保持自身门第高华的考虑,与后来嫁女谢氏一样,均是时代风气使然。以坚持婚姻的门当户对为方正,正是魏晋时代士人的认识。后谢氏兴起而诸葛衰微,高门显第之绝代风华,终于花谢水流,不得不倚赖嫁女以攀高枝,从而达到重振家声的目的。从人世白云苍狗的沧桑变幻和士族门户的运势消息中,依稀可见出历史的无情和人心的无奈。回想上个世纪六七十年代青年人谈婚论嫁,以根正苗红的工农兵出身为首选对象,这一择偶标准,在今天看来似乎不可思议,但当时人们却严肃地奉行着,这正是那个年代理解的“门当户对”。余嘉锡先生“冢中枯骨,未可尽恃”一语,可谓的论。}

\lettrine{5.26} 周叔治\myidx{周谟}作晋陵太守\footnote{周叔治:周谟,字叔治。周顗弟。晋陵:郡名。治所在今江苏常州。辖境相当于今江苏镇江、常州、无锡三市及附近地区。},周侯\myidx{周顗}、仲智\myidx{周嵩}往别\footnote{周侯:指周顗。顗字伯仁,周氏三兄弟,伯仁为长。仲智:周嵩字仲智,周顗弟,性狷介。}。叔治以将别,涕泗不止。仲智恚之曰\footnote{恚:生气,恼怒。}:“斯人乃妇女,与人别,唯啼泣。”便舍去\footnote{舍去:离去。}。{\fzxk\zihao{6}\textcolor{red}{邓粲\CJKunderwave{晋纪}曰:“周谟字叔治,顗次弟也,仕至中护军。嵩字仲智,谟兄也,性狡直果侠,每以才气凌物。顗被害,王敦使人吊焉。嵩曰:‘亡兄天下有义人,为天下无义人所杀,复何所吊?’敦甚衔之,犹取为从事中郎。因事诛嵩。”\CJKunderwave{晋阳秋}曰:“嵩事佛,临刑犹诵经。”}} 周侯独留与饮酒言话,临别流涕,抚其背曰:“阿奴好自爱\footnote{阿奴:是晋宋时代常用语,表示一种亲昵称呼,用于长呼幼、尊呼卑,相当于第二人称。平辈表示亲昵,有时也可称对方为“阿奴”。犹今吴方言中“阿囡”。}。”{\fzxk\zihao{6}\textcolor{red}{阿奴,谟小字。}}

{\cangkai\zihao{5}【评】常言道:一母生九子,九子各不同。从先天遗传基因和气质类型理论来看,此言有理。根据周顗兄弟三人的表现,截然分属两种不同的类型。周顗、周谟较为情绪化,是性情中人。分别之际,顗、谟话别流涕,表现得一往情深。\CJKunderwave{言语}门第三十一则载过江诸人新亭对泣的故事,周顗中坐而叹:“风景不殊,正自有山河之异”,正是其多愁善感的性情,触惹了众人的故国之思,于此可为佐证。周嵩则表现殊异,见兄弟周谟涕泣不止而大怒,数落一番之后便扬长而去。看似方正,实则无情。史载周嵩“狷直果侠,每以才气陵物”(\CJKunderwave{晋书}本传)。处理兄弟关系尚且如此不近人情,对待他人更可想而知。“无情未必真豪杰”,周嵩之决绝似大可不必。刘辰翁曰:“一样兄弟,厚薄如此。少年陵物,大有人以为方正。奇矫取名,最害心术,亦不得不辨。”对周嵩的矫情举止做了理性考辨。此外,故事抓取了兄弟三人生活中具有表现力的点滴瞬间加以定格,刻画其不同的精神面貌和情感世界,可谓神来之笔。}

\lettrine{5.27} 周伯仁\myidx{周顗}为吏部尚书\footnote{周伯仁:周顗。吏部尚书:吏部长官,魏时称选曹尚书,主管官员的任免、铨叙、考绩、升降等。},在省内\footnote{省:官署。此指尚书省。},夜疾危急。时刁玄亮\myidx{刁协}为尚书令\footnote{刁玄亮:刁协字玄亮。尚书令:尚书省长官。负责政令。},营救备亲好之至\footnote{备:犹“尽”。},良久小损\footnote{小损:稍微减轻。损:差减。}。{\fzxk\zihao{6}\textcolor{red}{虞预\CJKunderwave{晋书}曰:“刁协字玄亮,渤海饶安人。少好学,虽不研精,而多所博涉。中兴制度,皆禀于协。累迁尚书令。中宗信重之。为王敦所忌,举兵讨之。奔至江南,为人所杀。”}} 明旦,报仲智\myidx{周嵩}\footnote{明旦:第二天早晨。仲智:周嵩,见前则。},仲智狼狈来\footnote{狼狈:急速。}。始入户,刁下床对之大泣\footnote{床:坐榻。},说伯仁昨危急之状。仲智手批之\footnote{批:排开。},刁为辟易于户侧\footnote{辟易:退辟。}。既前,都不问病,直云:“君在中朝\footnote{中朝:晋室南渡后称渡江前的西晋为中朝。},与和长舆\myidx{和峤}齐名\footnote{和长舆:和峤字长舆,晋武帝时为中书令,后转侍中,甚被器重,为一代名臣。},那与佞人刁协有情!”迳便出。

{\cangkai\zihao{5}【评】史载刁协“性刚悍,与物多忤,崇上抑下”,历史评价似乎不高。然其对晋室“悉力尽心,志在匡救”(\CJKunderwave{晋书}本传),与那些逢迎拍马的势利小人又有区别。他与刘隗同被元帝倚为腹心,其实是制衡王氏家族的力量。周氏兄弟与王导情好,处于同一利益集团,周嵩曾上疏为王导开脱。因此,周嵩借题发挥辱骂刁协为“佞人”,既是其平素以才气陵物性格的必然反应,同时也有借机打压敌对势力气焰的深层用心。不管出于何种目的,周嵩对兄长病情毫不关心,对救命恩人疾言厉色、拳脚相加,便违背了人之常情。其下意识的过激反映,暴露出他胸中的利益观念、敌我意识,要远远高于手足之情。这就从一个侧面昭示,六朝时代一部分狭隘士人对方正的理解,与方正本义已南辕北辙,虽其出发点在于纠偏,但过犹不及,只能产生令人厌恶的效果。}

\lettrine{5.28} 王含\myidx{王含}作庐江郡\footnote{王含:字处弘,大将军王敦兄。王敦:字处仲,晋琅邪临沂(今属山东)人,王导堂兄。妻为晋武帝女襄城公主,拜驸马都尉。晋室东迁,与王导一起辅佐元帝,任要职,握重兵,镇守扬州、荆州等重镇。公元322 年起兵谋反,入京都建康。王含:见刘孝标注。光禄勋:官名,九卿之一,领管光禄、大中、中散、谏议等大夫及羽林郎、五官、虎贲、左右等中郎将注。庐江郡:晋代郡名。治所在舒县(今安徽舒城)。},贪浊狼籍\footnote{狼籍:一作“狼藉”。纵横散乱的样子,引申为破败不可收拾。此处指声名败坏。}。王敦\myidx{王敦}护其兄\footnote{护:庇护。},故于众坐称\footnote{众坐:大庭广众。}:“家兄在郡定佳,庐江人士咸称之。”时何充\myidx{何充}为敦主簿\footnote{何充:字次道,晋庐江人。字次道,晋康帝时为骠骑将军。主簿:中央或地方郡县所设属官。},在坐,正色曰:“充即庐江人,所闻异于此。”敦默然。旁人为之反侧\footnote{反侧:不安的样子。},充晏然神意自若\footnote{晏然:平静的样子。}。{\fzxk\zihao{6}\textcolor{red}{\CJKunderwave{中兴书}曰:“王敦以震主之威,收罗贤隽,辟充为主簿。充知敦有异志,逡巡疏外。及敦称含有惠政,一坐畏敦,击节而已,充独抗之。其时,众人为之失色。由是忤敦,出为东海王文学。”}}

{\cangkai\zihao{5}【评】故事极具典型意义:好比某单位开会,领导在上面大肆吹嘘自己的亲信如何德才兼备,可堪重用。这正如同明代宗臣\CJKunderwave{报刘一丈书}中以漫画笔法描绘的“上下相孚”的世风。群众虽然对实情心知肚明,但慑于淫威,对于“皇帝的新装”不敢戳穿。这时忽有一人因不甘心充当被耍弄的“阿斗”,站起来拆穿谎言。这需要何等的勇气!何充当众顶撞领导,是其方正人格面对虚伪和谎言时的必然反弹,绝不是一时头脑发热的鲁莽之举。王敦一代枭雄,气魄远大,他“舍得一身剐,敢把皇帝拉下马”,对于区区一介何充,当然不会放在眼里!他之所以隐忍不发,当出于对名士风度的相对尊重,以及自身长期修炼的名士底蕴,这就高出那些气量狭小、睚眦必报的上司百倍!}

\lettrine{5.29} 顾孟箸(著)\myidx{顾显}尝以酒劝周伯仁\myidx{周顗}\footnote{顾孟箸(著):顾显,字孟著,东晋吴郡(今江苏苏州)人。顾荣侄子。少有令名。周伯仁:周顗。},伯仁不受。顾因移劝柱,而语柱曰:“讵可便作栋梁自遇\footnote{讵:岂,难道。作栋梁自遇:自己把自己看作栋梁。遇,对待。}!”周得之欣然,遂为衿契\footnote{衿契:情意相投的朋友。}。{\fzxk\zihao{6}\textcolor{red}{徐广\CJKunderwave{晋纪}曰:“顾显字孟箸(著),吴郡人,骠骑荣兄子。少有重名,泰兴中为骑郎。蚤卒,时为悼惜之。”}}

{\cangkai\zihao{5}【评】魏晋玄学中人,有一类属于纵酒任心的,周顗就是此类中人。他嗜酒如命,“为仆射,略无醒日,时人号为‘三日仆射’。”(\CJKunderwave{晋书}本传)据说他能饮酒一石,一石合十斗,今日之一百升,就是喝水,肚子也很难承受得下,何况是酒?虽史家夸饰,不可尽信,然而必有一定依据,并非空穴来风。顾显仰慕中原名士周顗之雅望与海量,欲劝酒以交好于顗。周顗可能没有瞧得起这位后生晚辈,使顾很没面子。顾借酒佯狂,对柱而讥周,颇有后世李太白“举杯邀明月”的绝尘风姿。不料此招竟引起周顗的好感,遂与顾显结为挚友。大概名士相交,更看重反常举止背后折射出的气质风度吧,顾显借酒装疯,周顗却听出弦外之音,报以青眼。可见任情率性的名士自有一套同气相求的“话语系统”,不可以常人常理推求。刘辰翁曰:“劝柱、语柱自佳,语又佳。”抓住了故事神髓,堪称慧眼。}

\lettrine{5.30} 明帝(元帝)\myidx{司馬睿}在西堂,会诸公饮酒\footnote{明帝:明帝未即位,顗已为王敦所杀。据\CJKunderwave{晋书}本传,此当指元帝。西堂:东晋皇宫太极殿的西厅。诸公:群臣。}。未大醉,帝问:“今名臣共集,何如尧舜\footnote{尧、舜:唐尧、虞舜,传说中的两个远古帝王。尧舜时代,被视为圣明时代。且在位时有众多贤臣。}?”时周伯仁\myidx{周顗}为仆射\footnote{仆射:即尚书省主事官员,为尚书令之副。魏晋时代,或置左右仆射,或置尚书仆射。尚书仆射职权仅次于尚书令。周顗于晋元帝时任尚书左仆射。},因厉声曰:“今虽同人主,复那得等于圣治\footnote{人主:人君。圣治:圣明时代。}!”帝大怒,还内,作手诏满一黄纸,遂付廷尉令收\footnote{廷尉:掌管刑狱的官。收:逮捕。},因欲杀之。{\fzxk\zihao{6}\textcolor{red}{案:明帝未即位,顗已为王敦所杀。此说非也。}} 后数日,诏出周\footnote{诏:动词,下诏。}。群臣往省之\footnote{省:看望。},周曰:“近知当不死\footnote{近知:早就知道。近,原先、当初。当不死:该不会死。},罪不足至此。”

{\cangkai\zihao{5}【评】故事中的第二主角明帝,据\CJKunderwave{晋书}顗传作元帝,与刘注合,可从。周顗饮酒无度,人称“三日仆射”,但后人切不可以酒鬼视之。他“留一半清醒留一半醉”,在大是大非面前决不糊涂。元帝自比尧、舜,周顗偏不给皇帝面子,敢于揭短、捅马蜂窝。与诸王公大臣唯唯诺诺相比,勇气可嘉。中国古代社会,虽然不具备西方现代意义上的民主自由意识,但也不乏独立不倚的思想因子。范仲淹\CJKunderwave{灵乌赋}有“宁鸣而死,不默而生”,这就是一种诤谏的自由,是一种不平则鸣的抗争精神。周顗敢于和皇帝顶嘴,不愿睁眼说瞎话,不能将其看成是一时的酒后失言,而是其平素“善养浩然之气”的必然结果。}

\lettrine{5.31} 王大将军\myidx{王敦}当下\footnote{王大将军:指王敦。当下:谓将起兵东下。王敦于晋元帝永昌元年(322),以讨伐刘隗为名,在武昌起兵反,顺江东下。},时咸谓无缘尔\footnote{无缘尔:没有缘由如此。}。伯仁\myidx{周顗}曰:“今主非尧、舜,何能无过?且人臣安得称兵以向朝廷?处仲\myidx{王敦}狼抗刚愎\footnote{狼抗:狂妄自大。刚愎:傲慢固执。},王平子\myidx{王澄}何在\footnote{王平子:王澄字平子,太尉王衍弟。澄素有盛名,在王敦之上,兼勇力过人,为王敦所惧,后为王敦所杀。}?”{\fzxk\zihao{6}\textcolor{red}{\CJKunderwave{顗别传}曰:“王敦计(讨)刘隗,时温太真为东宫庶子,在承华门外,与顗相见曰:‘大将军此举有在,义无有滥?’顗曰:‘君年少,希更事。未有人臣若此而不作乱,共相推戴数年而为此者乎!处仲狼抗而强忌,平子何在?’”\CJKunderwave{晋阳秋}曰:“王澄为荆州,群贼并起,乃奔豫章。而恃其宿名,犹陵侮敦。敦伏勇士路戎等搤而杀之。”\CJKunderwave{裴子}曰:“平子从荆州下,大将军伺欲杀之。而平子左右有二十人,甚健,皆持铁楯、马鞭。平子恒持玉枕。大将军乃搞(犒)荆州文武,二十人积饮食,皆不能动。乃借平子玉枕,便持下床。平子手引大将军带绝,与力士斗甚苦,乃得上屋上,久许而死。”}}

{\cangkai\zihao{5}【评】王敦于元帝永昌元年(322)以清君侧讨刘隗、刁协为名,起兵东下,实则窥探神器,问鼎晋室。朝臣心存侥幸幻想,以为王敦不过是“清君侧”而已,像汉景帝出卖晁错一样,朝廷交出几个替罪羊就可以换来天下太平。只有周顗明察秋毫,一眼看穿王敦的狼子野心。“王平子何在”一语,引王敦杀戮旧事,惊破士大夫的迷梦,令人警醒。周顗勠力勤王,为国殉难。顗少有重名,晚节不亏,一生行止有为有守,虽以酒失为人所讥,然瑕不掩瑜,是一个可堪大任的杰出政治家。}

\lettrine{5.32} 王敦\myidx{王敦}既下\footnote{王敦:见前则。既下:谓兴兵东下之后。},住船石头\footnote{住:停。石头:即石头城。在京城建康西。因形势险要、地处交通要道,为东晋军事重镇。},欲有废明帝\myidx{司马绍}意\footnote{废明帝意:王敦攻入石头城,时司马绍为太子,聪明有胆略,为朝野所望,敦忌之,有废太子意。}。宾客盈坐,敦知帝聪明,欲以不孝废之。每言帝不孝之状,而皆云:“温太真\myidx{温峤}所说\footnote{温太真:温峤字太真,东晋名臣。当时追从姨夫刘琨,在并州为谋主,“琨所凭恃焉”(\CJKunderwave{晋书·温峤传})。建武元年(317)奉刘琨命出使江南,拥戴司马睿即帝位,建立东晋王朝。受司马睿重用,留为散骑常侍,后官至中书令,为东晋名臣。注。}。温尝为东宫率\footnote{东宫:皇太子所居之宫。率:皇太子官属,主门卫。温峤曾任太子中庶子,事太子司绍,即明帝。},后为吾司马,甚悉之。”须臾,温来,敦便奋其威容,问温曰:“皇太子作人何似?”温曰:“小人无以测君子。”敦声色并厉,欲以威力使从己,乃重问温:“太子何以称佳?”温曰:“钩深致远\footnote{钩深致远:物在深处,能够取之;物在远方,能招致之。语出\CJKunderwave{易·系辞上}:“探赜索隐,钩深致远,以定天下之吉凶,成天下之亹亹者,莫大乎耆龟。”后用以指人才力、学识广博精深。},盖非浅识所测。然以礼侍亲,可称为孝。”{\fzxk\zihao{6}\textcolor{red}{刘谦之\CJKunderwave{晋纪}曰:“敦欲废明帝,言于众曰:‘太子子道有亏,温司马昔在东宫,悉其事。’峤既正言,敦忿而愧焉。”}}

{\cangkai\zihao{5}【评】王世懋评曰:“叙事如画”,可谓得之。故事中王敦废帝之心急不可耐,故声色俱厉;温峤护翼之情披肝沥胆,亦豪气干云。二人性格刻画无遗,其性情、声吻跃然纸上。温峤为刘琨外甥,早年随琨将兵讨石勒,是一位入则画奇谋,出则建军功的双料人才。戎马生涯和庙堂经历,练就了其良好的心理素质和沉着机智的应对能力,面对王敦威逼质问,他应声而答,正气凛然,维护了朝廷的威严,从而成为历史上方正人物的典型。温峤私下里是明帝做太子时的布衣至交,曾利用其特殊身份,多次箴规讽谏,并非一味护短。可见,温峤是一位懂得进退屈伸、不可多得的中兴名臣。}

\lettrine{5.33} 王大将军\myidx{王敦}既反\footnote{王大将军既反:王敦反,事在永昌元年(322)。},至石头\footnote{石头:即石头城。},周伯仁\myidx{周顗}往见之\footnote{周伯仁:周顗,当时任尚书左仆射,率军抗王敦,大败,奉诏去见王敦。}。谓周曰:“卿何以相负\footnote{相负:即负我。晋愍帝建兴元年(313)周顗为益州流民起事领袖杜弢所困,投奔王敦,故敦以为有德于顗。}?”对曰:“公戎车犯正\footnote{戎车:兵车,泛指军队。犯正:指叛朝廷。},下官忝率六军\footnote{下官:郡国内的属吏对其长官及国主的自称。忝:谦词。表示行为于人有辱或于己有愧。六军:周制天子有六军,诸侯国有三军、二军、一军不等。后作军队统称,此处指王师。},而王师不振,以此负公\footnote{负公:负阁下,周顗此语,是以反言讥刺王敦。}。”{\fzxk\zihao{6}\textcolor{red}{\CJKunderwave{晋阳秋}曰:“王敦既下,六军败绩。顗长史郝嘏及左右文武劝顗避难。顗曰:‘吾备位大臣,朝廷倾挠,岂可草间求活,投身胡虏邪?’乃与朝士诣敦。敦曰:‘近日战有馀力不?’对曰:‘恨力不足,岂有馀邪?’”}}

{\cangkai\zihao{5}【评】周顗曾为杜弢所困,投奔王敦,敦以有恩德于顗,故有“何以相负”之诘。王敦将国家大事与个人恩怨纠缠在一起,实非明智之问,恰成为周顗反唇相讥的口实。周本可避其锋芒以求自保,但他早已做好了以身殉国的准备。常言道“无欲则刚”,一个人能够将生死置之度外,又何惧之有?“疾风知劲草,烈火见真金”,顗纵酒废职之微瑕,难掩临难不屈之玉质,沧海横流之际,更显出士大夫本色!故余嘉锡\CJKunderwave{世说新语笺疏}云:“\CJKunderwave{世说·方正}篇之目,惟伯仁、太真及锺雅数公可以无愧焉。其他诸人之事,虽复播为美谈,皆自好者优为之耳。”可谓盖棺定论之言。}

\lettrine{5.34} 苏峻\myidx{苏峻}既至石头\footnote{“苏峻”句:晋成帝咸和二年(327),苏峻举兵反,攻入都城建康,迁成帝于石头。后被陶侃、温峤等率军击败,峻被杀。},百僚奔散,{\fzxk\zihao{6}\textcolor{red}{王隐\CJKunderwave{晋书}曰:“峻字子高,长广掖人。少有才学,仁(仕)郡主簿,举孝廉。值中原乱,招合流、旧六千馀家,结垒本县,宣示王化,收葬枯骨,远近咸(感)其恩义,咸共宗焉。讨王敦有功,封公,迁历阳太守。峻外营将表曰:‘鼓自鸣。’峻自斫鼓曰:‘我乡里时,有此则空城。’有顷,诏书征峻。峻曰:‘台下云我反,反岂得活邪?我宁山头望廷尉,不能廷尉望山头。’乃作乱。”\CJKunderwave{晋阳秋}曰:“峻率众二万,济自横江,至于蒋山,王师败绩。”}} 唯侍中锺雅\myidx{锺雅}独在帝侧\footnote{侍中:侍从皇帝左右、备应对顾问的官。锺雅:字彦胄,东晋颍川长社(今河南)人。官至侍中,苏峻作乱,被杀。}。或谓锺曰:“见可而进,知难而退\footnote{“见可而进,知难而退”:语出\CJKunderwave{左传·宣公十二年}:“见可而进,知难而退,军之善政也。”可,适宜。是说作战须见机行事,后泛指应变能力,量力而行。},古之道也。君性亮直,必不容于寇雠。何不用随时之宜,而坐待其弊邪?”锺曰:“国乱不能匡\footnote{匡:纠正。},君危不能济\footnote{济:救助。},而各逊遁以求免,吾惧董狐将执简而进矣\footnote{“吾惧董狐”句:言史官将记录下大臣们面临危难而逃跑的可耻行为。董狐,春秋时晋国史官,以直书不隐,被称为古代良史。简,古代用以书写的竹片。}。”

{\cangkai\zihao{5}【评】中国古代社会,向有重史之传统,与此相关的是士人对不朽人格的追求。一方面,如春秋良史董狐者,秉笔直书,虽刑辟加身而不废其职守;另一方面,如赵宋忠臣文天祥者,前赴后继,虽粉身碎骨,而欲使青史流芳。二者呼应成趣、相得益彰。锺雅就是这万千“留取丹心照汗青”的仁人志士中的一员。这也可以说明,魏晋清谈末流虽有祖述浮夸、崇尚虚谈的一面,可脚踏实地的实干家还是前仆后继,代不乏人;儒家思想虽有式微之势,而服膺儒学、克己躬行者亦薪尽火传。若以为儒学在魏晋时全然退出历史舞台,就未免是皮相之谈,是对历史的误读。周顗、锺雅诸人知难而进、舍生取义,就是极有力的明证。}

\lettrine{5.35} 庾公\myidx{庾亮}临去\footnote{庾公:指庾亮。庾亮(289—340)的敬称。他历仕东晋元、明、成三朝,作为外戚,曾执国政,显赫于朝。的卢:传说中的凶马之名,骑之不利主人。},顾语锺\myidx{锺雅}后事\footnote{顾语:谓顾念嘱托。锺: 锺雅, 时为侍中。后事: 走后之事。},深以相委\footnote{委:托付。}。锺曰:“栋折榱崩\footnote{栋折榱崩:指房屋倒塌。栋,房梁。榱,屋椽。比喻国家倾覆。此时苏峻起兵攻入京师建康,故曰栋折榱崩。},谁之责邪?”庾曰:“今日之事,不容复言,卿当期克复之效耳\footnote{当:将,将会。克复:收复失地。此指平定叛乱收复京师,迎帝还都。}。”锺曰:“想足下不愧荀林父耳\footnote{荀林父:春秋时晋国大臣。率师击楚以救郑,败绩而归。荀林父归,请死,晋侯听士贞子之谏,容而不问,仍任其为将。后荀林父果攻灭赤狄,有功于晋。事见\CJKunderwave{左传·宣公十五年}。}。”{\fzxk\zihao{6}\textcolor{red}{\CJKunderwave{春秋传}曰:“楚庄王围郑,晋使荀林父率师救郑,与楚战于邲,晋师败绩。柏(桓)子归,请死,晋平公将许之,士贞子谏而止。后林父败赤狄干(于)曲梁,赏柏(桓)子、狄臣千室,亦赏士伯以爪衍之佰(县),曰:‘吾获狄田(土),子之功也。微子,吾丧伯戍(氏)矣。’”}}

{\cangkai\zihao{5}【评】故事当发生在明帝咸和二年(327)苏峻叛逆时。苏峻之乱,庾亮负有一定的责任,故亮亲自率兵抵挡,想为自己争回颜面,给国家一个交代。临行前深以国事相托,这是一种非同寻常的信任,他看中的正是锺雅正直较真的品格。\CJKunderwave{晋书}本传评锺雅“正直当官”、“直法绳违”,用今天的话说,是一位坚持原则、不徇私情的好干部。锺雅又犯了执拗的老脾气,责问庾亮“今日之事谁之咎?”既而,又对即将出师的庾亮寄予了凯旋大捷的厚望,可见其深明大义、富有人情味。朝廷同僚之间如果都能同仇敌忾,必收“其利断金”之效。}

\lettrine{5.36} 苏峻\myidx{苏峻}时\footnote{苏峻时:指苏峻举兵占据京师建康时。},孔群\myidx{孔群}在横塘\footnote{孔群:字敬林(一作“休”),东晋会稽山阴(今浙江绍)人。横塘:三国吴时筑,在今南京西南。},为匡术\myidx{匡术}所逼\footnote{匡术:东晋成帝时人,苏峻起兵,甚得宠信,峻迁成帝入石头城,逼城中居民尽聚后苑,令匡术守之,咸和四年(329)春,苏峻死,匡术以苑城降。逼:\CJKunderwave{晋书·孔群传}载:“苏峻入石头,时匡术有宠于峻,宾从甚盛。群与从兄愉同行于横塘,遇之。愉止与语,而群初不视术,术怒欲刃之。愉下车,抱术曰:‘吾弟发狂,卿为我宥之!’乃获免。”孔群在横塘为匡术所逼事。}。王丞相\myidx{王导}保存术\footnote{王丞相:王导。保存:庇护;保全。},{\fzxk\zihao{6}\textcolor{red}{\CJKunderwave{会稽后贤记}曰:“群字敬休,会稽山阴人。祖竺,吴豫章太守。父弈(奕),全椒令。群有智局,仕至御史中丞。”\CJKunderwave{晋阳秋}曰:“匡术为阜陵令,逃亡无行。庾亮征苏峻,术劝峻诛亮,遂与峻同反。后以宛(苑)城降。”}} 因众坐戏语,令术劝群酒,以释横塘之憾\footnote{释:消解。憾:仇恨,怨恨。此指匡术欲杀害孔群事。}。群答曰:“德非孔子,厄同匡人\footnote{德非孔子,厄同匡人:\CJKunderwave{孔子家语}说孔子到宋国去,匡简子以甲士围之,子路怒,奋戟将战,孔子止之,命子路弹剑而歌,孔子自和之。曲三终,匡人解甲。此借“孔”、“匡”二姓以讥刺匡术。厄,困厄、迫害。}。{\fzxk\zihao{6}\textcolor{red}{\CJKunderwave{家语}曰:“孔子之宋,匡简子以甲士围之。子路怒,奋戟将战,孔子止之,曰:‘夫\CJKunderwave{诗}、\CJKunderwave{书}之不讲,礼乐之不习,是丘之过也。若述先王之道,而为咎者,非丘罪也。命也夫!歌,予和汝。’子路弹剑,孔子和之。曲三终,匡人解甲罢。”}} 虽阳和布气\footnote{阳和布气:谓仲春时天气和暖。阳和:温和。},鹰化为鸠\footnote{鹰化为鸠:一年有二十四节气,古人将每一节气分为三候,每一候都有着与之应时而出的物候。春季惊蛰节气的三候是桃始华、仓庚鸣,鹰化为鸠。这里用“鹰化为鸠”比喻恶人放下屠刀。鸠,布谷鸟。},至于识者,犹憎其眼。”{\fzxk\zihao{6}\textcolor{red}{\CJKunderwave{礼记·月令}曰:“仲春之月,鹰化为鸠。”郑玄曰:“鸠,播谷也。”\CJKunderwave{夏小正}曰:“鹰则为鸠。鹰也者,其杀之时也;鸠也者,非杀之时也。善变而之仁,故具之。”}}

{\cangkai\zihao{5}【评】苏峻之乱时,孔群在横塘受过匡术的死亡威逼。后匡术举苑城投诚,王导保全了匡术。王世懋评曰:“丞相末年大不满人意,在保存诸叛贼,盖渠于节义二字不大分晓。”究其实,世懋对丞相的政治策略缺乏深刻认识。东晋开基不久,国运未稳,王敦、苏峻之乱接踵而至。匡术弃暗投明,王导从建立“统一战线”的大局着眼,化干戈为玉帛,给予其改过自新的机会,实是出于不得已的良苦用心。千载以下的王世懋将“节义”的理解模式化、狭隘化,倒也情有可原;至于孔群,他不是不理解王导的政治策略,所以并没有在宴席上拔刀直向仇敌。这正是对大乱甫定之后王导“统一战线”策略的支持。但他作为一位忠义之士,因潜意识的作用,对过去的仇恨耿耿于怀而不能释憾,胸中郁积的情感,瞬间自然爆发,而有“尤憎其眼”之讥,比喻生动形象。时人因其是非分明而誉入方正之门,宜哉!}

\lettrine{5.37} 苏子高\myidx{苏峻}事平\footnote{苏子高:苏峻。事平:指苏峻叛乱已平定。},{\fzxk\zihao{6}\textcolor{red}{\CJKunderwave{灵鬼志·谣征}曰:“明帝初,有谣曰:‘高山崩,石自破。’高山,峻也;硕,峻弟也。后诸公诛峻,硕犹据石头,溃散而逃,追斩之。”}} 王、庾诸公欲用孔廷尉\myidx{孔坦}为丹阳\footnote{孔廷尉:孔坦字君平,会稽山阴(今浙江绍兴)人。善\CJKunderwave{春秋},有文才,历太子舍人、尚书郎、丞,官至廷尉卿。丹阳:郡名。晋时治所在建业。晋南渡后,丹阳成为护卫京都的重要地区,设丹阳尹之职。}。{\fzxk\zihao{6}\textcolor{red}{孔坦。}} 乱离之后,百姓凋弊\footnote{凋敝:破败。此指生计艰困。苏峻攻建康时,因风放火,官署民房,一时荡尽,有“黍离”之态。}。孔慨然曰:“昔肃祖\myidx{司马绍}临崩\footnote{肃祖:晋明帝司马绍,死后庙号肃祖。},诸君亲升御床,并蒙眷识\footnote{眷识:垂爱器重。晋明帝临亡召司徒王导、尚书令卞壸、车骑将军郗鉴、护军将军庾亮等人受遗诏,辅太子。},共奉遗诏。孔坦疏贱,不在顾命之列\footnote{顾命:本为\CJKunderwave{尚书}篇名。这里指受遗诏的大臣,即顾命大臣。}。既有艰难,则以微臣为先。今犹俎上腐肉\footnote{俎:肉案,砧板。},任人脍截耳\footnote{脍截:切割、宰割。脍,细切的肉,这里用为动词。}!”于是拂衣而去,诸公亦止。{\fzxk\zihao{6}\textcolor{red}{案:王隐\CJKunderwave{晋书}:“苏峻事平,陶侃欲将坦上,用为豫章太守,坦辞母老不行。台以为吴郡,吴郡多名族;而坦年少,乃授吴兴内史。”不闻尹京。}}

{\cangkai\zihao{5}【评】东晋一朝,孔坦算是个不可多得的人才。他不仅为人正直,多次抗颜直谏;且具有军事天才,讨伐王敦之乱时已崭露头角,后在平定苏峻时,运筹帷幄、言兵多中。但朝政大权掌握在王、庾诸公手中,他的一些合理化建议多因决策人物的轻轻一句否定而遭流产的厄运。在门阀世族时代,孔坦虽能力超群,却难以跻身上流。丹阳于建康为京畿,丹阳尹则为京尹。苏峻攻建康时,因风放火,官署民房,一时荡尽;城破之后,又纵兵大掠。因此,民生凋敝,百姓流离,有黍离之态。王、庾诸公用孔坦为京尹,名义上是因京畿重地,尹应首选,但在客观上是把烂摊子推给孔坦。孔坦不是任人拿捏之辈,义正词严予以回绝。这既是对王导把持朝政、庾亮刚愎招祸的不满,其更深层用意,是把矛头指向不公正的人才选拔制度。其言掷地有声,催人警醒!代表了一部分有才而不尽其用的庶族知识分子的胸中愤懑!}

\lettrine{5.38} 孔车骑\myidx{孔愉}与中丞\myidx{孔群}共行\footnote{孔车骑:孔愉(268—342),字敬康,东晋会稽山阴(今浙江绍兴)人。中丞:指孔群。},{\fzxk\zihao{6}\textcolor{red}{\CJKunderwave{孔愉别传}曰:“愉字敬康,会稽山阴人。初辟中宗参军,讨华轶有功,封馀不亭侯。愉少时,尝得一龟,放于馀不溪中,龟中路左顾者数过。及后铸印,而龟左顾,更铸犹如此。印师以闻,愉悟,取而佩焉。累迁尚书左仆射,赠车骑将军。”中丞,孔群也。}} 在御道,逢匡术\myidx{匡术}\footnote{御道:皇帝车驾经由的道路。\CJKunderwave{晋书·孔群传}作“于横塘遇之”。盖横塘有御道。},宾从甚盛。因往与车骑共语。中丞初不视,直云:“鹰化为鸠,众鸟犹恶其眼\footnote{“鹰化为鸠”二句:一年有二十四节气,古人将每一节气分为三候,每一候都有着与之应时而出的物候。春季惊蛰节气的三候是桃始华、仓庚鸣,鹰化为鸠。这里用“鹰化为鸠”比喻恶人放下屠刀。鸠,布谷鸟。}。”术大怒,便欲刃之。车骑下车抱术曰:“族弟发狂,卿为我宥之!”始得全首领\footnote{全首领:保全性命。首领,头和颈,此指性命。}。

{\cangkai\zihao{5}【评】故事与本门第36则,乃同一件事、同一句话,但时空移易,未审孰是?根据英美新批评的理论,文本的表达方式不同,前因后果倒置,则其呈现出的意义也因之不同。第36则中,故事发生在匡术投降后,匡在宴会上处于被动地位,孔群之骂,虽然解气,未免有打落水狗之嫌,可以说是胜之不武;此则故事发生在匡术助逆得势之时,匡助峻为虐,处于优势地位,孔群被困处劣势地位,其讥骂匡术,忠义可感,当以方正视之。}

\lettrine{5.39} 梅颐(赜)\myidx{梅颐}尝有惠于陶公\myidx{陶侃}\footnote{梅颐:字仲真,东晋汝南(在今河南)人。但杨勇\CJKunderwave{世说新语校笺}作梅赜,疑是。陶公:陶侃。\CJKunderwave{晋纪}、\CJKunderwave{晋书}谓有恩于陶侃者乃梅颐之弟梅陶,并非梅颐。},后为豫章太守,有事\footnote{豫章太守:豫章郡行政长官。豫章,郡治在南昌。有事:指犯事。},王丞相\myidx{王导}遣收之\footnote{王丞相:王导。遣收之:派人拘捕他。}。侃曰:“天子富于春秋\footnote{富于春秋:未来的年华尚多。这是一种委婉说法。},万机自诸侯出\footnote{万机:繁多的日常政务。诸侯:原指中央政权分封各国国君,后泛称高级官员。}。王公既得录\footnote{录:逮捕。},陶公何为不可放?”乃遣人于江口夺之\footnote{江口:渡口。}。{\fzxk\zihao{6}\textcolor{red}{\CJKunderwave{晋诸公赞}曰:“颐字仲真,汝南西平人。少以学隐退,而才实进止。”\CJKunderwave{永嘉流人名}曰:“颐,领军司马。颐弟叔真。”邓粲\CJKunderwave{晋纪}曰:“初有谮侃于王敦者,乃以从弟廙代侃为荆州,左迁侃广州。侃文武距廙而求侃,敦闻,大怒。及令侃将莅广州,过敦,敦陈兵欲害侃,敦谘议参军梅陶谏敦,乃止,厚礼而遣之。”王隐\CJKunderwave{晋书}亦同。案:二书所叙,则有惠于陶,是梅陶,非颐也。}} 颐见陶公拜,陶公止之。颐曰:“梅仲真膝,明日岂可复屈邪\footnote{“梅仲真膝”二句:谓梅仲真不肯轻易向人屈膝。}!”

{\cangkai\zihao{5}【评】据刘孝标注,有惠于陶公者,乃梅陶(叔真)。陶公之救梅赜(仲真),乃感梅陶之意,而假手其兄以报之耳。所谓弟弟栽树,哥哥乘凉也。陶侃知恩图报,自是人伦美德。然从王导手中夺走梅赜,其理由依据颇耐人寻味。陶侃虽出身寒素,但凭借实干精神,为屏蔽东晋半壁江山,立下汗马功劳,终成一代名臣。\CJKunderwave{晋书}本传评曰:“元规(庾亮)以戚里之崇,挹其膺而下拜;茂弘(王导)以保衡之贵,服其言而动色。望隆分陕,理则宜然。”可见陶侃是王、庾都要推敬三分的扛鼎式人物。然明帝驾崩,侃不在顾命之列,其政治地位虽然名列二品,但出自庶族寒门,朝中望族士人,并未给予其充分尊重。正所谓“驴打江山马坐殿”,朝政依然由诸豪门大姓把持。陶侃心怀耿耿,从王导手里夺走梅赜是一次小的情绪爆发。你王导是中央高官,可以抓人,我陶侃是地方军政领袖,为什么不能放人?其胸中块垒,于不经意间的几句话流露无遗。故事入“方正”,其称道的主角是知恩图报的陶侃,而王世懋以为“虽有一言,宁便足称方正”?其讥评当是没有读懂故事的弦外之音、味外之味。}

\lettrine{5.40} 王丞相\myidx{王导}作女伎\footnote{作:安置,安排。女伎:歌女,舞女。伎,同“妓”。},施设床席\footnote{床席:床榻坐席。}。蔡公\myidx{蔡谟}先在坐\footnote{蔡公:蔡谟(281—356):字道明,东晋陈留考城(今河南民权东北)人。},不悦而去,王亦不留。{\fzxk\zihao{6}\textcolor{red}{\CJKunderwave{蔡司徒别传}曰:“谟字道明,济阳考城人。博学有识,避地江左。历左光禄,录尚书事,扬州刺史。薨,赠司空。”}}

{\cangkai\zihao{5}【评】蔡谟在玄学扇炽的东晋一代,一直保持其儒生本色。史载,谟性方雅笃慎,每事必为过防。时人云:“蔡公过浮航,脱带腰舟。”有点像俄国契诃夫笔下的“套中人”。儒家提倡克己复礼,动静要合于礼,但若讲究过分,走向教条化,就会面目可憎,失去其存在的合理性。王导作女妓,相当于今天的观赏歌舞表演,是为了缓解工作压力。站在还原历史的角度看,此乃人之常情,本无可厚非。蔡谟不悦而去,一副严肃冷峻的面孔,拒人于千里之外。王导不挽留他,亦是知人的明智之举——蔡谟在座,恐令人不乐也。明李贽评曰:“无味。”到底无味谓何?朱铸禹\CJKunderwave{世说新语汇校集注}以为:“本条殊简略,宜李贽评为‘无味’。”恐值得商榷。李贽所云“无味”者,当指蔡谟古板无味。李贽作为狂禅教主,其理论非圣无法、喝佛骂祖,最能吸引人眼球的当是其与上流社会女性交往的传闻。他每入书院讲学,往往会对诸生开玩笑说:此时正不如携歌妓舞女,浅斟低唱。以如此态度处世,当然会斥责蔡谟这个老古董“无味”。}

\lettrine{5.41} 何次道\myidx{何充}、庾季坚\myidx{庾冰}二人并为元辅\footnote{何次道:何充,字次道,晋康帝时为骠骑将军。庾季坚:庾冰。元辅:宰相。以其辅佐皇帝而居大臣首位,故称元辅。}。{\fzxk\zihao{6}\textcolor{red}{\CJKunderwave{晋阳秋}曰:“庾冰字季坚,太尉亮之弟也。少有检操,兄亮常器之曰:‘吾家晏平仲。’累迁车骑将军、江州刺史。”}} 成帝\myidx{司马衍}初崩\footnote{成帝:指晋成帝司马衍。},于时嗣君未定\footnote{嗣君:继承帝位的君主。}。何欲立嗣子\footnote{嗣子:嫡长子。依封建宗法制度“父死子继”的原则,嫡长子当继承祖业,称嗣子。},庾及朝议以外寇方强\footnote{朝议:朝廷上的评议、商议。外寇:主要指当时北方少数民族建立的后赵、前燕等政权。东晋偏居江左,长江以北广大地区为少数民族建立的政权,与晋成对峙局面。},嗣子冲幼\footnote{冲幼:年幼。},乃立康帝\myidx{司马岳}\footnote{康帝:晋康帝司马岳(321—344),成帝司马衍同母弟。成帝病,中书令庾冰自以舅氏当朝,倘立成帝之子,则戚属将疏,乃以岳为嗣。}。{\fzxk\zihao{6}\textcolor{red}{\CJKunderwave{中兴书}曰:“帝讳岳,字世同,成帝同母弟也。成帝崩,即位,年二十二。”}} 康帝登祚,会群臣,谓何曰:“朕今所以承大业,为谁之议?”何答曰:“陛下龙飞\footnote{龙飞:指君主即位。\CJKunderwave{周易·乾卦}:“飞龙在天,利见大人。”},此是庾冰之功,非臣之力。于时用微臣之议,今不睹盛明之世。”{\fzxk\zihao{6}\textcolor{red}{\CJKunderwave{晋阳秋}曰:“初,显宗临崩,庾冰议立长君。何充谓宜奉皇子。争之不得。充不自安,求处外任。及冰出镇武昌,充自京驰还,言于帝曰:‘冰不宜出。昔年陛下龙飞,使晋德再隆者,冰之勋也,臣无与焉。’”}} 帝有惭色。

{\cangkai\zihao{5}【评】此与\CJKunderwave{晋书}记载有异:\CJKunderwave{晋书}记载议立嗣君,成帝尚在;又记康帝即位,曰:“朕嗣鸿业,二君之力也。”两相比较,\CJKunderwave{晋书}词义为优。成帝时,庾氏以舅氏当朝,成为东晋辅政的四大家族之一。成帝崩,倘立成帝之子,则庾氏由舅氏变为外家,亲属关系疏远,辅政的理由不充分,可能大权旁落;若以帝弟司马岳为嗣,则庾氏仍为舅氏,主掌朝政,名正言顺。庾冰是出于维护高门显第门户私计的考虑,意在继续维持“庾与马、共天下”政局。而何充与其分歧之处,在于以社稷为己任,“不以私恩树亲戚”(\CJKunderwave{晋书}本传)。最终,还是庾氏兄弟愿望得逞。高门势力可以左右帝王的继嗣,晋朝皇权之羸弱可见一斑。何充回答康帝问话,不卑不亢,软中带硬,貌似颂扬,实见出对庾氏的讥讽。座忤王敦已如前载,今又抗衡贵戚,充之方正不虚也。}

\lettrine{5.42} 江仆射\myidx{江虨}年少\footnote{江仆射:指江虨。},王丞相\myidx{王导}呼与共棋\footnote{王丞相:王导。}。王手尝不如两道许\footnote{手:指下棋的技能、手段。两道:围棋盘上的格道。借指围棋子。犹两子。许:助词,置于数词后表约数。},而欲敌道戏\footnote{敌道戏:对等地下棋,不饶子。},试以观之。江不即下。王曰:“君何以不行?”江曰:“恐不得尔\footnote{尔:如此。}。”{\fzxk\zihao{6}\textcolor{red}{徐广\CJKunderwave{晋记}曰:“江虨字思玄,陈留人。博学知名,兼善弈,为中兴之冠。累迁尚书左仆射、护军将军。”}} 傍有客曰:“此年少,戏迺不恶\footnote{不恶:不坏,不错。}。”王徐举首曰:“此年少,非唯围棋见胜\footnote{非唯:不仅是。见胜:胜我。}。”{\fzxk\zihao{6}\textcolor{red}{范汪\CJKunderwave{棋品}曰:“虨与王恬等棋第一品,导弟(第)五品。”}}

{\cangkai\zihao{5}【评】范汪\CJKunderwave{棋品}曰:“江虨与王恬等,棋第一品,导第五品。”可见江虨是一流的围棋高手。王导棋艺逊色,每次都要江虨让两子。但导屡战屡败,又屡败屡战,精神可嘉。这次王导不要让棋,虨却不肯走棋,“恐不得尔”一语,看似谦逊,实则疏狂,意在表明王导不是其对手,刘辰翁曰:“丞相雅量,此年少不让,小伎自多,宜戒。”甚是。与江虨之少年气盛相比,王导倒显得雍容大度,不但毫不觉丢面子,还对客夸奖江虨非惟以围棋见长,还有诸多胜处。其奖掖后辈之宽广胸怀堪称雅量,当在江虨内心掠起一丝波澜吧。}

\lettrine{5.43} 孔君平\myidx{孔坦}疾笃\footnote{孔君平:孔坦。疾笃:病重。},庾司空\myidx{庾冰}为会稽\footnote{庾司空:庾冰。为会稽:任会稽内史。},省之\footnote{省:探问,拜访。},{\fzxk\zihao{6}\textcolor{red}{庾冰。}} 相问讯甚至\footnote{问讯:问候。甚至:备至,诚恳。},为之流涕。庾既下床,孔慨然曰:“大丈夫将终,不问安国宁家之术,迺作儿女子相问\footnote{儿女子:小女子,妇人。}!”庾闻回谢之\footnote{谢:谢罪,道歉。},请其话言。{\fzxk\zihao{6}\textcolor{red}{王隐\CJKunderwave{晋书}曰:“坦方直而有雅望。”}}

{\cangkai\zihao{5}【评】孔坦不愿做丹阳尹,获讥于刘辰翁、王世懋诸评家。刘、王之论,恐非“知人论世”的“理解之同情”(陈寅恪语),如能与本则互看,方能了解一完整的,有血有肉、敢爱敢恨的孔坦。孔坦疾笃时致书庾亮,抒其“身往名灭,朝恩不报”的遗憾(\CJKunderwave{晋书}本传),又责备庾冰作儿女相问。他对国事一定有许多独到的思考,要向执政大臣庾冰交代。孔坦心忧天下,临终前没有遗产分配的意见,也没有葬礼规格的要求,置门户私计于度外,堪称古代优秀士大夫的表率,与宋代陆放翁之“死去原知万事空,但悲不见九州同”,同样精诚可感,其方正品格当“不废江河万古流”!}

\lettrine{5.44} 桓大司马\myidx{桓温}诣刘尹\myidx{刘惔}\footnote{桓大司马:桓温,桓公北征:桓温曾有三次北征,刘盼遂\CJKunderwave{世说新语校笺}考订,此次当为太和四年(369)之征。时桓温已58岁。注。诣:到……去。刘尹:指刘惔。字真长,曾任丹阳尹,故称。谢安妻兄,尚明帝女庐陵公主。会稽王司马昱为相,与王濛并为其座上清谈之客。性简贵自重,与王羲之友善。卒年三十六。惔曾作丹阳尹。},卧不起。桓弯弹弹刘枕\footnote{弯弹:拉弯弹弓。},丸迸碎床褥间\footnote{迸碎:爆裂成碎片。古代枕头有以石、玉或陶制者。}。刘作色而起曰\footnote{作色:改变脸色,表示恼怒。}:“使君,如馨地\footnote{使君:汉晋时对刺史、太守的敬称。桓温曾作徐州、荆州刺史,故称之为使君。如馨:这样,像这样。},宁可斗战求胜\footnote{宁:难道。斗战:战斗。意谓桓温是个武夫。}!”{\fzxk\zihao{6}\textcolor{red}{\CJKunderwave{中兴书}曰:“温曾为徐州刺史,沛国属徐州,故呼温使君。斗战者,以温为将也。”}} 桓甚有恨容。{\fzxk\zihao{6}\textcolor{red}{刘尹,真长已见。}}

{\cangkai\zihao{5}【评】刘惔乃是与王羲之诸人流连往还的名流,自视为第一流人物。刘惔虽奇桓温之才,但清醒地洞明其勃勃野心。魏晋士人多不泯天真童趣,桓温见刘晨睡不起,一时童趣复萌,搞恶作剧捉弄他。弹射靠枕,着实危险。幸亏桓温技艺高超,如果射在头上,岂不脑浆崩裂?惹恼刘惔,也在情理之中。一般人甚至可能因此与之绝交,不与“虐待狂”为伍。刘情急之下,口无遮拦,呼温为武夫,既是常人常情,但同时又是其潜意识中傲慢与偏见的自然流露,轻视桓温出身武夫!晋时重文轻武,故有此言。桓温虽戎马倥偬,却以名士自命,北伐经金城,有“木犹如此,人何以堪”之叹,见其一往情深。刘惔呼其为“武夫”,岂不正刺其心头隐痛?故事可见晋人之任诞,与方正毫无关涉。刘辰翁评曰:“如怒如笑”,掩卷可想见桓温之情绪变化。}

\lettrine{5.45} 后来年少多有道深公\myidx{竺法深}者\footnote{后来:后辈。年少:年轻人。深公:指东晋名僧竺法深,曾官散骑常侍,故云。},深公谓曰:“黄吻年少\footnote{黄吻年少:犹黄口小儿。口边称吻,雏鸟嘴黄,因以喻幼童。},勿为评论宿士\footnote{宿士:老名士,老前辈。}。昔尝与元\myidx{司马睿}、明\myidx{司马绍}二帝、王\myidx{王导}、庾\myidx{庾亮}二公周旋\footnote{元、明二帝:指晋元帝司马睿和晋明帝司马绍。王、庾二公:王导和庾亮。周旋:交往,打交道。}。”{\fzxk\zihao{6}\textcolor{red}{\CJKunderwave{高逸沙门传}曰:“晋元、明二帝,游心玄虚,托情道味,以宾友礼待法师;王公、庾公,倾心侧席,好同臭味也。”}}

{\cangkai\zihao{5}【评】六朝玄学,前期以儒道思想的交锋、碰撞为主,有\CJKunderwave{老}、\CJKunderwave{庄}、\CJKunderwave{易}“三玄”之说;后期则有佛教义理的加入。佛教的“空静”追求,因其思维方式及人生境界与玄学有契合之处,而获得在士人中间的生命力。另一方面,佛教思想初来乍到,要想在中土扎根,就不得不依赖王公贵族的庇护与宣扬;僧徒在权门间游走,饮食起居亦有保障,可以说是一举多得。竺法深就是这样一位以帝王卿相座上宾面目出现的“高僧大德”。但其托豪门显贵以自高,恰恰是自降身价,落入俗人境地。唐李太白笑傲王侯、睥睨卿相,其诗曰:“昔在长安醉花柳,五侯七贵同杯酒”,看似豪气冲天,实则浅薄庸俗。竺法深亦如此。故王世懋讥曰:“道人乃借人主名卿拒人,口吻宁视方正。”一语暴露其乖违佛学教义而强拉大旗做虎皮的媚俗本质。}

\lettrine{5.46} 王中郎\myidx{王坦之}年少时\footnote{王中郎:指王坦之。坦之曾领北中郎将,故称。},{\fzxk\zihao{6}\textcolor{red}{坦之,已见。}} 江虨\myidx{江虨}为仆射\footnote{江虨:字思玄,东晋陈留(今河南开封东北)人。江统子。为晋中兴大臣,累官至尚书左仆射。仆射:官名。晋尚书省设左右仆射。},领选\footnote{领选:兼任选拔官吏之事。},欲拟之为尚书郎\footnote{拟:拟定;安排。之:指代王坦之。尚书郎:尚书省属官。初任称郎中,满一年者称尚书郎。}。有语王者,王曰:“自过江来,尚书郎正用第二人\footnote{正:只。第二人:第二流的人,指寒素之门的人。},何得拟我\footnote{何得拟我:怎么可以安排我。}!”江闻而止。{\fzxk\zihao{6}\textcolor{red}{案:\CJKunderwave{王彪之别传}曰:“彪之从伯导谓彪之曰:‘选曹举汝为尚书郎,幸可作诸王佐邪!’此知郎官寒素之品也。”}}

{\cangkai\zihao{5}【评】中朝以来,祖尚虚浮,士人们不以物自婴,口不论世事;以遗事为高,以任职为俗。“居官无官官之事,处事无事事之心”竟成名士风范。偏安江左,此弊未歇;高门子弟,尤当其首。东汉时尚书郎多以孝廉或博士高第为之,为清望要职。但魏晋以后,名士谈玄之风兴,士人崇尚虚浮,以遗落世事为高,以担任实职为俗,东晋尚沿袭此风。尚书郎主文书起草,无吏部之权势,有刀笔之烦劳,名士均不屑受尚书郎之职。坦之为太原王氏子弟,虽厌憎时俗放荡,然自负门第德望第一流,难以抵挡时代风尚之吹熏。坦之先有不屑尚书郎之举,其子国宝亦步其后尘,对任职挑三拣四,看似偶然巧合,实蕴涵着历史的必然。以之入方正,正见门阀社会的特殊认识,时过境迁则自然烟消云散。}

\lettrine{5.47} 王述\myidx{王述}转尚书令\footnote{王述:字怀祖,太原晋阳人。为人真率性急,袭爵蓝田侯。转:调动官职。尚书令:尚书省长官。},事行便拜\footnote{事行便拜:谓授官诏书下达就立即接受。}。文度\myidx{王坦之}曰\footnote{文度:王坦之字文度,王述子。}:“故应让杜、许\footnote{故:或许。杜、许:不详何人。一说杜预、许璪,但不知何据。}。”蓝田\myidx{王述}云:“汝谓我堪此否\footnote{蓝田:即王述,封蓝田侯,故称。堪:能够胜任。}?”文度曰:“何为不堪,但克让自是美事\footnote{克让:克己让人。},恐不可阙\footnote{阙:同“缺”。意谓至少从形式上谦让一下也是应该的。}。”蓝田慨然曰:“既云堪,何为复让?人言汝胜我,定不如我。”{\fzxk\zihao{6}\textcolor{red}{\CJKunderwave{述别传}曰:“述常以谓人之处世,当先量己而后动,义无虚让,是以应辞便当固执。其贞正不逾,皆此类。”}}

{\cangkai\zihao{5}【评】虚伪矫饰,古今人生一大通病,为害极烈。魏晋士人,多重性情之自然,率性任真,言行直揭矫饰之伪,很值得人们尊敬。王述言行则尤为人称道。述尝于王导座中,指斥同僚对王导的肉麻吹捧,“人非尧舜,何得每事尽善”,语虽简略,却发人深思。又,述转尚书令,事行便拜,不为虚让,毫无官场的虚浮习气,是真名士作风。王述与众人的差异,令人想起安徒生“皇帝新装”故事里所描述的,天真无邪的儿童世界,与老于世故的成人社会之别。但在门阀社会的官场中,似王述这样的“老顽童”,凤毛麟角,何济于事?惜哉!}

\lettrine{5.48} 孙兴公\myidx{孙绰}作\CJKunderwave{庾公诔}\footnote{孙兴公:孙绰。\CJKunderwave{庾公诔}:关于庾亮的诔文。},文多托寄之辞\footnote{托寄:攀附寄托,又云寄托深情厚谊的话语。}。{\fzxk\zihao{6}\textcolor{red}{绰集载诔文曰:“咨予与公,风流同归。拟量托情,视公犹师。君子之交,相与无私。虚中纳是,吐诚诲非。虽实不敏,敬佩弦韦。永戢话言,口诵心悲。”}} 既成,示庾道恩\myidx{庾羲}\footnote{庾道恩:庾羲,字叔和,小名道恩。庾亮之子。少有时誉。任吴国内史。}。庾见,慨然送还之,曰:“先君与君,自不至于此\footnote{先君:称死去的父亲,犹亡父、先父。自:本来。}。”{\fzxk\zihao{6}\textcolor{red}{道恩,庾羲小字。徐广\CJKunderwave{晋纪}曰:“羲字叔和,太尉亮第三子。拔尚率到,位建威将军、吴国内中(史)。”}}

{\cangkai\zihao{5}【评】孙绰为东晋名士,少与许询俱有高尚之志,王羲之发起著名的“兰亭之游”,绰亦预其列。孙绰又是东晋时期著名的玄言诗人,\CJKunderwave{晋书}本传称“于时文士,绰为其冠”。正是由于这种名人的社会效应,温、王、郗、庾诸公之死,必须经绰为碑文,然后乃刊石。绰为庾亮作\CJKunderwave{庾公诔},文多寄托深情厚谊之辞,原本无可厚非。但孙绰年辈,晚于庾亮多多,二人并无深交,故诔文中的“风流同归”、“君子之交”等语,未免借名流来渲染自己,有托死人以自炫之嫌,有违“铭诔尚实”(\CJKunderwave{典论·论文})之义。庾道恩作为东晋四大家族中庾氏嫡系传人,瞧不起出身不高的孙绰,在门阀社会,也在料中。但更重要的是,他不虚伪应酬,而是据实以对,慨然送还诔文,毫不留情面,其明察秋毫的“疾虚妄”之举,是率性而行的名士风度的表现。谓之方正,未为不可。}

\lettrine{5.49} 王长史\myidx{王濛}求东阳\footnote{王长史:指王濛。东阳:东阳郡,治所在今浙江金华。},抚军\myidx{司马昱}不用\footnote{抚军:指晋简文帝司马昱。他即位前,以会稽王任抚军大将军,掌朝政。用:任用。}。{\fzxk\zihao{6}\textcolor{red}{简文。}} 后疾笃\footnote{疾笃:病重。},临终,抚军哀叹曰:“吾将负仲祖\footnote{负:辜负,对不起。}!”于此命用之。长史曰:“人言会稽王痴\footnote{会稽王:指司马昱。},真痴。”{\fzxk\zihao{6}\textcolor{red}{王濛已见。}}

{\cangkai\zihao{5}【评】晋简文帝司马昱之各种“痴言痴行”前已多见。他做皇帝很失败——谢安称其为惠帝之流,谢灵运以其为周之赧王、献王之辈,实在是不堪造就。他有才能,有抱负,但在主弱臣强的门阀政治中,却无从施展,这是其抑郁而终的客观外部原因。其为人并非毫无可取,对于玄言清谈、文采风流亦时有会心。故事中王濛求东阳而不许,后因疾笃而任命,乃其痴之一例。王濛为简文布衣之好、入室之宾。简文不许王濛求任东阳之请,乃是出于一个“情”字——因情深义重,故难舍难分;后王濛病重将死,简文任命濛为东阳太守,还是出于“情”的考虑。朋友临终,满足其生前愿望,让其平静地离去,此一份用心令人感动。王濛说简文“真痴”,是朋友间的善意嘲讽,毫无反感、批评之意。临川列入方正,是没有参透名士间的特定语言符号。以今人眼光看来,亲人或友人临终,存者尽量满足其生前没有实现的愿望,不也是人之常情吗?}

\lettrine{5.50} 刘简\myidx{刘简}作桓宣武\myidx{桓温}别驾\footnote{刘简:字仲约,东晋南阳(今属河南)人。桓宣武:桓温。别驾:州刺史的佐使。},后为东曹参军\footnote{东曹参军:东曹掾属一类的官。晋诸王公及开府位相当于王公者,设东西曹,分科办事,曹有掾一类官吏,掌府内诸事。},{\fzxk\zihao{6}\textcolor{red}{\CJKunderwave{刘氏谱}曰:“简字仲约,南阳人。祖乔,豫州刺史。父挺,颍川太守。简仕至大司马参军。”}} 颇以刚直见疏\footnote{见疏:被疏远。}。尝听讯,简都无言。宣武问:“刘东曹何以不下意\footnote{下意:提出或发表意见。}?”答曰:“会不能用\footnote{会:反正,终究。}。”宣武亦无怪色。

{\cangkai\zihao{5}【评】桓温,有杀伐专断之威。刘简以一属员,于众座公然讽刺、批评桓温的霸道作风,须有“威武不能屈”的过人勇气。在专制时代,正确的言论意见不见天日,实在是一种悲哀,故事从一个角度折射出桓温为人的专横;但桓温毕竟是名士,不同于胸无点墨、动辄以杀人显淫威的屠伯,他对敢于提意见的刘简竟无怪色,其心胸之宽广,远非鼠肚鸡肠的领导所能望其项背。宜哉,其为一代枭雄也!故事见出刘简之方正,同时又折射出桓温的胸襟,可谓双美。}

\lettrine{5.51} 刘真长\myidx{刘惔}、王仲祖\myidx{王濛}共行\footnote{刘真长:刘惔,字真长,曾任丹阳尹,故称。谢安妻兄,尚明帝女庐陵公主。会稽王司马昱为相,与王濛并为其座上清谈之客。性简贵自重,与王羲之友善。卒年三十六。王仲祖:王濛。},日旰未食\footnote{日旰:天晚。}。有相识小人贻其餐\footnote{小人:对平民百姓的蔑称。魏晋时期门阀制度森严,士族阶级轻视奴仆、吏役以及各行各业的普通百姓,一概目之为“小人”。贻:赠送。},肴案甚盛\footnote{肴案:菜肴的几案,此指菜肴。},真长辞焉。仲祖曰:“聊以充虚\footnote{充虚:充饥。},何苦辞\footnote{苦:竭力。}?”真长曰:“小人都不可与作缘\footnote{都:完全。作缘:来往,发生联系。}。”{\fzxk\zihao{6}\textcolor{red}{孔子称:“唯女子与小人为难养,近之则不逊,远之则怨。”刘尹之意,盖从此言也。}}

{\cangkai\zihao{5}【评】古代社会,严格“大人”与“小人”之别,等级差别森严,难以逾越。樊须问稼,孔子斥其为小人;孟子则提出“有大人之事,有小人之事。劳心者治人,劳力者治于人”。魏晋以降,士大夫每以门第自矜,士庶差别有若天渊。九品中正制严于士庶之别,还属于统治阶级内部之争,而九品中正制以外的不入流者,则为民,属于被统治阶级范围,官民更有根本性质之异。对于平民百姓,士大夫视为“小人”,实际上已经蔑视其社会存在。刘惔死要面子活受罪,宁可饿肚皮,也决不吃“小人”置办的菜肴,将士大夫贵族阶层的矜持、高傲,演绎得淋漓尽致,看似维护了尊严,实则违背了人性的真淳。临川以其入方正,是时代的局限,今天看来只有滑稽可笑。}

\lettrine{5.52} 王修龄\myidx{王胡之}尝在东山\footnote{王修龄:王胡之字修龄。东山:在今浙江上虞,东晋名士常隐居于此。},甚贫乏。{\fzxk\zihao{6}\textcolor{red}{司州,已见。}} 陶胡奴\myidx{陶胡奴}为乌程令\footnote{陶胡奴:陶侃之子,小字胡奴。乌程:县名。晋属吴兴郡(今浙江吴兴)。郡治在今浙江湖州。},{\fzxk\zihao{6}\textcolor{red}{胡奴,陶范小字也。\CJKunderwave{陶侃别传}曰:“范字道则,侃第十子也,侃诸子中最知名。历尚书、秘书监。”何法盛以为第九子。}} 送一船米遗之\footnote{遗:馈赠。}。却不肯取,直答语:“王修龄若饥,自当就谢仁祖\myidx{谢尚}索食\footnote{谢仁祖:谢尚字仁祖,谢豫章:谢鲲,曾作豫章太守。刘孝标注“鲲子别见”,“子”字衍。将:携,谓携之送客。自:已经。参:参与、进入。上流:上等、上品注。},不须陶胡奴米\footnote{须:需要。}。”

{\cangkai\zihao{5}【评】故事与上则大同而小异,可见六朝门阀世族社会严格区分、强化士庶之别。陶氏出身寒微,陶侃虽立大功,早已跻升二品之列,而王、谢子弟犹不免以老兵视之,故王胡之羞与其子陶范为伍。常言道“官不打送礼的”,王胡之连做人的基本道德底线都不顾,其自命清高连世族高门中之清醒者谢安都看不过眼,尝语曰:“阿龄于此事故欲太厉。”过犹不及,门阀世族制度由于反人性的一面,故其甫一产生,就埋下了速朽的根苗。“旧时王谢堂前燕,飞入寻常百姓家”,刘禹锡以精炼传神之笔描画了华丽家族终于卸下矜持的浓妆油彩,以“无可奈何花落去”的失落心情淡出了历史舞台!}

\lettrine{5.53} 阮光禄\myidx{阮裕}{\fzxk\zihao{6}\textcolor{red}{阮裕,已见。}} 赴山陵\footnote{阮光禄:阮裕字思旷,即阮裕,曾以金紫大夫征,故称。\CJKunderwave{世说}作者刘义庆为避宋武帝刘裕名讳,从不称阮裕之名。剡(shàn 善):古县名,在今浙江嵊州。注。曾作金紫光禄大夫,故称。山陵:帝王陵墓,引申指帝王丧事。},至都,不往殷\myidx{殷浩}、刘\myidx{刘惔}许\footnote{至都:到京都建康。殷、刘:殷浩、刘惔。当时都是为清谈者所崇尚的大名士。许:处所。},过事便还。诸人相与追之\footnote{相与:一起。}。既亦知时流必当逐己\footnote{时流:当代名流。},乃遄疾而去\footnote{遄疾:疾速。迅速。},至方山不相及\footnote{方山:山名。在江苏江宁东南,六朝时为交通要道,商旅聚集处。}。{\fzxk\zihao{6}\textcolor{red}{\CJKunderwave{中兴书}曰:“裕终日颓然,无所错综,而物自宗之。”}} 刘尹时索会稽\footnote{刘尹时索会稽:索,索求之意。刘惔生平并未作会稽郡太守,或求而未得。},乃叹曰:“我入,当泊安石渚下耳\footnote{我入:\CJKunderwave{晋书}“我入”下有“东”字。东指会稽。泊:停船,停靠。安石渚:指谢安居处。谢安,谢奕(?—358):字无奕,谢安长兄,陈郡阳夏谢氏家族在东晋初期的代表人物之一。时谢安与阮裕同居会稽,谢安为刘惔妹婿。渚:小洲。},不敢复近思旷傍\footnote{思旷:阮裕字思旷。傍:通“旁”。}。伊便能捉杖打人不易\footnote{伊:他。此指阮裕。捉:持,握。}。”

{\cangkai\zihao{5}【评】阮裕在乱世中始终能保持难得的清醒,与一般自命不凡的名士异趣。为王敦主簿时,他以一双慧眼洞察世事,终日酣觞,以酒废职,然竟以酒免敦难。与族兄阮籍之佯狂自保如出一辙。后去职还家,居会稽东山,有肥遁之志。虽多次被征,而屡以疾辞。王羲之称曰:“近不惊宠辱,虽古之沉冥,何以过此!”殷浩、刘惔或言行不一,或清高太过,均非名士正解。阮裕之过事便还,不愿与之结缘,并非沽名钓誉,而是从花开花落、云卷云舒的自然变相中感悟人世的沉浮废兴,有隐士风度。刘辰翁评曰:“更无伦理。”认为阮裕不近人情,并非的论。}

\lettrine{5.54} 王\myidx{王濛}、刘\myidx{刘惔}与桓公\myidx{桓温}共至覆舟山看\footnote{王、刘:指王濛、刘惔。桓公:桓温。覆舟山:山名。在今南京东北。},酒酣后,刘牵脚加桓公颈\footnote{牵:引。此谓伸过来。加:放在上面。},桓公甚不堪,举手拨去。既还,王长史语刘曰:“伊讵可以形色加人不\footnote{“伊讵”句:他难道可以拿脸色强加于人吗?伊,他。讵,难道。形色加人,指对人发怒逞威。}?”{\fzxk\zihao{6}\textcolor{red}{\CJKunderwave{温别传}曰:“温有豪迈风气也。”}}

{\cangkai\zihao{5}【评】王濛簪缨世家,刘惔风流名士,桓温出身行伍,故王、刘视温为老兵,骨子里持轻蔑态度。刘惔将脚架在桓温脖颈上,荒谬绝伦令人作呕。温不堪忍受,举手拨去,竟惹得王濛勃然作色。“伊讵可以形色加人不?”出于王濛之口,意谓桓温卑微老兵,不当以声色凌人,只能默默忍受。魏晋风流,以喜怒不见于形色为上。王、刘的言行中流露出对桓的轻视之意,临川取以为方正之言,可见六朝人所理解的方正,其内涵是何等的混乱!刘惔牵脚桓公颈,桓温弹射刘惔枕,人类基本的尊重底线都弃之不顾,其不相礼敬如此,有何资格大谈名士尊严!魏晋名士越名纵礼,对儒家礼教有批判作用,但矫枉过正则同样陷入怪圈。}

\lettrine{5.55} 桓公\myidx{桓温}问桓子野\myidx{桓伊}\footnote{桓公:桓温。桓子野:桓伊,字叔夏,小字子野、野王,东晋谯国铚(今安徽西南)人。}:“谢安石\myidx{谢安}料万石\myidx{谢万}必败\footnote{谢安石:谢安字安石。万石:谢万字万石,安弟。},何以不谏\footnote{谏:直言规劝。此时桓伊为桓温参军,故有此问。}?”{\fzxk\zihao{6}\textcolor{red}{子野,桓伊小字也。\CJKunderwave{续晋阳秋}曰:“伊字叔夏,谯国铚人。父景,护军将军。伊少有才艺,又善声律,加以标悟省率,为王濛、刘惔所知。累迁豫州刺史,赠右将军。”}} 子野答曰:“故当出于难犯耳\footnote{故当:或许,可能。难犯:难以触犯。}。”桓作色\footnote{作色;改变脸色。}曰:“万石挠弱凡才\footnote{挠弱:懦弱无能。},有何严颜难犯\footnote{严颜:威严的容颜。}!”

{\cangkai\zihao{5}【评】谢万为谢安之弟,虽聪明俊秀,善于炫耀,而其名气去栖迟东山的谢安远甚,时人“攀安提万”之说可证。谢万虽于玄言清谈头头是道,但实际缺乏探本求源的哲学头脑,毫无决胜千里的军事实干。在只看门第不重能力的魏晋时代,才用非人的乖谬事情屡见不鲜。谢万作为新任北伐统帅,矜持高傲,饮酒啸歌,称诸将为兵卒,激起不满情绪,后又误判形势,溃不成军,单骑败归。谢万虽给国家造成严重损失,但后来又被易地授官,渎职之罪无人承担!千百万士兵鲜活的生命死得不明不白,滔滔劣迹竟被轻描淡写掩饰过去,怎能不令正直士人扼腕切齿!门阀制度之弊,专制统治之恶,可见一斑。桓温称其“挠弱凡才”,正见其知人之明。但他明知谢万无能而料其必败,却拒绝王羲之谏,置国家利益于不顾,作为大将军而赞同任命谢万为北伐统帅。这是为什么?这说明为了谯国桓氏家族利益,准备看陈郡谢氏的笑话,从而打击王、谢,为“桓与马,共天下”扫平障碍。如王羲之辈的一二清醒之士纵有先知先觉,亦无力扭转世风,只能任半壁江山被雨打风吹去。}

\lettrine{5.56} 罗君章\myidx{罗含}曾在人家\footnote{罗君章:罗含,字君章,东晋桂阳耒阳(今属湖南)人,谢尚、桓温称他为“江左之秀”。为桓温别驾,致仕还家,阶庭忽兰菊丛生。},主人令与坐上客共语\footnote{共语:一起谈话。}。答曰:“相识已多\footnote{相识:相知,相互了解。},不烦复尔。”{\fzxk\zihao{6}\textcolor{red}{\CJKunderwave{罗府君别传}曰:“罗含字君章,桂阳枣(耒)阳人。盖楚熊姓之后,启土罗国,遂氏族焉。后寓湘境,故为桂阳人。含,临海太守彦曾孙,荥阳太守绥少子也。桓宣武辟为别驾。以官廨喧扰,于城西池小州(洲)上立茅茨,伐木为床,织苇为席,布衣蔬食,晏若有馀。桓公尝谓众坐曰:‘此自江左之清秀,岂唯荆楚而已。’累迁散骑常侍、廷尉、长沙相,致仕中散大夫,门施行马。含自在官舍,有一白雀栖集堂宇。及致仕还家,阶庭忽兰菊挺生。岂非至行之征邪?”}}

{\cangkai\zihao{5}【评】罗君章藻思超群,桓温称其为“江左之秀”。然性喜静,不胜尘世应对往来之苦。“相识已多,不烦复尔”一语,见出其以简对繁的做人风格。此举非标榜清高、故作矜持,与陶渊明“少无适俗韵,性本爱丘山”一样,都是质性使然。又尝以官舍喧扰,于江中小洲上立茅屋,伐木为床,织苇为席而居,布衣蔬食,晏如也。正可与此则互相印证。人在官场,强颜欢笑,逢场作戏虽出无奈,但多不能了悟。罗君章特立独行,为自己开辟一片心灵的自由天地,是符合玄学精神本质的方正之举。}

\lettrine{5.57} 韩康伯\myidx{韩伯}病\footnote{韩康伯:韩伯字康伯,时任豫章太守,故称。曾为王弼\CJKunderwave{周易注}补注\CJKunderwave{易传}之系辞、说卦、杂卦等,是当时著名玄学名家。},柱杖前庭消摇\footnote{前庭:庭前。消摇:同“逍遥”。谓闲适不拘,怡然自得。},{\fzxk\zihao{6}\textcolor{red}{韩伯,已见。}} 见诸谢皆富贵\footnote{诸谢:指谢安、谢奕、谢万、谢石等。谢氏自谢尚、谢安始发迹。安为尚书仆射、中书令。谢石、谢玄屡建战功,兄弟叔侄并得荣升,显赫一时。},轰隐交路\footnote{轰隐:众车行走的声音。},叹曰:“此复何异王莽时\footnote{王莽:汉平帝时为大司马,封安汉公。莽以外戚掌权,亲族皆拜官封侯。后王莽竟篡汉,建立新朝。}!”{\fzxk\zihao{6}\textcolor{red}{\CJKunderwave{汉书}曰:“王莽宗族,凡十侯、五大司马,外戚莫盛焉。”}}

{\cangkai\zihao{5}【评】名士阮裕曾讥谢万“新出门户,笃而无礼”,曾几何时,随着谢安的东山再起,成为风流名相后,王、谢家族齐名,谢家已今非昔比了。其时谢石、谢玄、谢琰屡建战功,叔侄三人同时受封,正是谢家繁华鼎盛之际。王莽宗族十侯、五大司马;而根据台湾学者毛汉光的研究,两晋南北朝时期,谢氏家族任五品以上官吏者70人,其中一品4人,为皇后者1人,尚公主者3人,其荣华显要恐不让王氏。谢家此时门庭若市、轰隐交路,也足见出人世运道的无常,所谓“时来天地皆同力,运去英雄不自由”!康伯虽身罹重病,但仍头脑清醒。据\CJKunderwave{建康实录},康伯卒于孝武帝太元五年(380),未及见淝水之战(发生于公元383年)后陈郡谢氏全盛之事。但其为人性格强直方正,能“澄世所不能澄,而裁世所不能裁”(见\CJKunderwave{晋书}本传)。他作为\CJKunderwave{易}理玄家,鉴往知今,洞识未来,故借王莽外戚之盛,以古讽今,并预感到陈郡谢氏家族盛极必衰的阴阳消伏之机,指出了国家社稷的前途隐忧。事入方正,宜哉!}

\lettrine{5.58} 王文度\myidx{王坦之}为桓公\myidx{桓温}长史\footnote{王文度:王坦之。桓公:桓温。长史:官名。此指桓温军府之长史。},桓为儿求王女,王许咨蓝田\myidx{王述}\footnote{许:答应。咨:询问。蓝田:王述字怀祖,官至尚书令,袭爵蓝田侯。}。{\fzxk\zihao{6}\textcolor{red}{王坦之、王述,并已见。}} 既还,蓝田爱念文度\footnote{爱念:疼爱,喜欢。},虽长大,犹抱著膝上。文度因言桓求己女婚。蓝田大怒,排文度下膝\footnote{排:拨开,推开。},曰:“恶见文度已复痴\footnote{恶:怎么。已复:副词,竟然。},畏桓温面?兵\footnote{兵:此指桓温。桓为武将,又家世不在名门之列,王述自恃太原王氏为高门,轻视桓温。},那可嫁女与之!”文度还报云:“下官家中先得婚处\footnote{下官:属吏对其长官自称下官。先得婚处:谓先前为女儿订了婚家。}。”桓公曰:“吾知矣,此尊府君不肯耳\footnote{尊府君:尊称对方的父亲。}。”后桓女遂嫁文度儿\footnote{桓女遂嫁文度儿:桓温虽掌兵权,但属寒门,故王述不愿文度嫁女与桓温儿。但寒门女可嫁士族儿。}。{\fzxk\zihao{6}\textcolor{red}{\CJKunderwave{王氏谱}曰:“坦之子恺(愉),娶桓温第二女,字伯子。”\CJKunderwave{中兴书}曰:“恺字茂仁,历吴国内史、丹阳尹,赠太常。”}}

{\cangkai\zihao{5}【评】桓温虽靠个人奋斗而位极人臣,但桓氏祖上名位不昌,不在名门贵族之列。时人鄙其地寒,不以上流处之。王述、谢奕、刘惔等呼为老兵,桓温也只能徒唤奈何。桓温为子求婚于王坦之,坦之父述因门第不相匹偶而顿时勃然作色,断然拒绝,全然不顾桓温情面,不计桓家小伙子的形貌、才情和未来发展潜质,更不考虑孙女的个人感受,完全是从家族名誉着眼。王述意见决绝,不容坦之置喙,可见门第观念多么难以逾越!于时风俗,寒门之女,可适高门之子;而名门之女,必不可下嫁寒族也。王述坚拒桓氏之请,本为门户私计,魏晋士人视为方正,此乃时代风气使然。}

\lettrine{5.59} 王子敬\myidx{王献之}数岁时\footnote{王子敬:王献之,王羲之子,(344—388),出于琅邪王氏家族。曾任谢安长史,官至中书令,故称王令或王大令。据\CJKunderwave{晋书·后妃传},尚简文帝女新安公主。少有令名,“风流一时之冠”。其书法已造神境,与父羲之并称“二王”。病笃:病重。},尝看诸门生\xpinyin*{樗蒱}\footnote{门生:魏晋六朝时,仕宦者允许各募部曲,谓之义从,其在门下亲侍者,则谓之门生。樗蒱:一种游戏。}。见有胜负,因曰:“南风不竞\footnote{南风不竞:语出\CJKunderwave{左传·襄公十八年}。竞,强,强劲。}。”{\fzxk\zihao{6}\textcolor{red}{\CJKunderwave{春秋传}曰:“楚伐郑。师旷曰:‘不害,吾骤歌\CJKunderwave{南风},\CJKunderwave{南风}不竞,多死声,楚必无功。’”杜预曰:“歌者次律以咏八风,南风音微,故曰不竞也。”}} 门生辈轻其小儿,乃曰:“此郎亦管中窥豹,时见一斑\footnote{郎:魏晋时少年的通称,相当于后世的“少年”。管中窥豹:从管中看豹。比喻所见狭小,看不到全面。}。”子敬瞋目曰\footnote{瞋目:瞪眼。}:“远惭荀奉倩\myidx{荀粲}\footnote{荀奉倩:荀粲字奉倩,三国魏颍川(今河南)人。魏太尉荀彧子。},近愧刘真长\myidx{刘惔}\footnote{刘真长:刘惔字真长,晋名士。}。”遂拂衣而去。{\fzxk\zihao{6}\textcolor{red}{荀、刘已见。}}

{\cangkai\zihao{5}【评】王献之乃王氏家族的芝兰玉树,良好的家庭艺术氛围的熏陶,使其成为杰出的书法家,踵武乃父而雄视千古。然世风所及,献之自小就具有浓厚的门第、等级观念。故事中,献之提到了荀粲(奉倩)、刘惔(真长),二人均严于择友,不与“小人”往来,为名士中持论褊狭者,这就在不经意间流露出王家小孩鄙视仆隶的门第观念,与荀、刘气味相投。献之嗔目怒斥,拂袖而去,俨然高傲的小主人,殊足损其童稚天性。门阀世族的傲慢与偏见,已构成遗传因子,深深扎根于幼小一代的头脑。事实证明,王家所谓佳子弟者,除艺术造诣外,大多缺乏治国安邦的实际才干,倒是其傲视世人、目空一切的空疏作风,将成为其永远抹不去的家族胎记。}

\lettrine{5.60} 谢公\myidx{谢安}闻羊绥\myidx{羊绥}佳\footnote{谢公:谢安。羊绥:字仲彦,东晋泰山平阳(在今山东)人。羊忱孙。},致意令来\footnote{致意:把自己的心意传达给别人。},终不肯诣。{\fzxk\zihao{6}\textcolor{red}{\CJKunderwave{羊氏谱}曰:“绥字仲彦,太山人。父楷,尚书郎。绥仕至中书侍郎。”}} 后绥为太学博士\footnote{太学博士:官名。太学设博士、助教,其员额因时代而异。},因事见谢公,公即取以为主簿。

{\cangkai\zihao{5}【评】谢安投桃,羊绥并不报李,对于千载难逢的飞腾机会似乎并不热中,这就与那些费尽心计、主动巴结权贵的名利客大异旨趣,反映出了真名士极其看重自己的人格尊严,决不会轻易放下架子而授人以口实,谓之方正,不无道理。人与人之间生来平等,都是赤条条来去,本无所谓高低贵贱。士人是社会的中流砥柱,理应看重自己、挺直了腰杆做人。但我们在滚滚红尘中,看到的更多是毫无灵魂操守的士林丑类的卑劣表演。这些人无论飞升或沉沦,都适足以污染环境,造成公害。如果羊绥这样的士人再多一些,权门势豪的威风就无从施展,社会空气岂不是更纯净一些!西谚有云:“有什么样的人民,就有什么样的政府。”言下之意,政府的恶习是被宠出来的,这不能不说有一定道理。故事的另一方面,谢安为国求贤,以才不以门,可谓不拘一格,有知人善任之明;又不因一时触忤己意而睚眦必报,有“宰相肚里能行船”的宽广胸怀,不愧为一代名相。\CJKunderwave{孟子·公孙丑}云:“故将大有为之君,必有所不召之臣。”羊绥的不应召,从另一个角度看,不也凸现了名相谢安的“大有为”吗?}

\lettrine{5.61} 王右军\myidx{王羲之}与谢公\myidx{谢安}诣阮公\myidx{阮裕}\footnote{王右军:王羲之。谢公:谢安。阮公:阮裕,即阮裕,曾以金紫大夫征,故称。\CJKunderwave{世说}作者刘义庆为避宋武帝刘裕名讳,从不称阮裕之名。剡(shàn 善):古县名,在今浙江嵊州。},{\fzxk\zihao{6}\textcolor{red}{阮思旷也。}} 至门,语谢:“故当共推主人\footnote{故当:当然,一定。推:推崇,推许。主人:指阮裕。}。”谢曰:“推人正自难\footnote{正自:的确、实在。}。”

{\cangkai\zihao{5}【评】阮裕、谢安企慕风流、不惊宠辱,均有高情远致;且二人都曾有隐居会稽的经历。同声同气之情,惺惺相惜之意,二人已具备相交的大前提。对于谢安的未肯推重老阮,明王世懋评曰“意未肯降”,朱铸禹以为此乃谢安自负所致。二家理由欠充分,于意未稳。谢安若无崇敬之情,何以与右军登门拜诣阮公?盖其善于矫情镇物,内心虽急于见到老前辈,而外表并不动容,发以“推人正自难”之语,足见其无时无刻不极力呈现世人以成熟、稳重的“安石”形象。}

\lettrine{5.62} 太极殿始成\footnote{太极殿:东晋宫殿名。},{\fzxk\zihao{6}\textcolor{red}{徐广\CJKunderwave{晋纪}曰:“孝武宁康二年,尚书令王彪之等启改作新宫。太元三年二月,内外军六千人始营筑,至七月而成。太极殿高八丈,长二十七丈,广十丈。尚书谢万监视,赐爵关内侯;大匠毛安之,关中侯。”}} 王子敬\myidx{王献之}时为谢公\myidx{谢安}长史\footnote{长史:三公所设辅佐官吏。},谢送版使王题之\footnote{版:做匾额用的木板。题:书写。}。王有不平色,语信云\footnote{信:使者,信使。}:“可掷箸门外。”谢后见王曰:“题之上殿何若\footnote{何若:怎么样。}?昔魏朝韦诞诸人,亦自为也\footnote{魏朝:公元220年,魏文帝曹丕废汉称帝,建立魏朝。韦诞:字仲将,三国魏书法家。}。”王曰:“魏作所以不长\footnote{魏作:据沈剑知校本,“作”为“祚”之形讹,是。魏祚,魏朝国运。}。”谢以为名言。{\fzxk\zihao{6}\textcolor{red}{宋明帝\CJKunderwave{文章志}曰:“太元中,新宫成,议者欲屈王献之题榜,以为万代宝。谢安与王语次,因及魏时起陵云阁,忘题榜,乃使韦仲将县凳上题之。比下,须发尽白,裁馀气息。还语子弟云:‘宜绝楷法!’安欲以此风动其意,王解其旨,正色曰:‘此奇事。韦仲将魏朝大臣,宁可使其若此,有以知魏德之不长。’安知其心,乃不复逼之。”}}

{\cangkai\zihao{5}【评】宫殿题榜,乃国之大事,堪为万代墨宝。如此扬名立万之事,何乐而不为?又子敬为谢安长史,论辈分为晚生。子敬严拒题榜,于理不合。推其原因,大概谢安派遣使者命令其题榜,而不亲自与其打招呼,惹恼了子敬孤芳自赏的紧绷神经,于是耍起了名家子的高傲脾气。子敬、子猷兄弟门第观念甚严,本书各门所记多可印证。高门名士使性,连执政大臣也奈何不得,导致政令不通、上下壅滞、官不得人,也就不足为奇了。晋朝政治之混乱,于此可见一斑。}

\lettrine{5.63} 王恭\myidx{王恭}欲请江卢奴\myidx{江敳}为长史\footnote{王恭:字孝伯,晋光禄大夫王蕴子。(?—398):孝武帝后兄,安帝舅父。与殷仲堪、桓玄等,二次兴兵清君侧,兵败被诛。会稽:郡治在今浙江绍兴市。江卢奴:江敳。东晋人,字仲凯,小字卢奴。江统孙,江虨子。注中“祖正”,乃刘孝标作注时避昭明太子萧统名讳改“统”为“正”。},晨往诣江,江犹在帐中。王坐,不敢即言,良久乃得及。江不应,{\fzxk\zihao{6}\textcolor{red}{卢奴,江敳小字也。\CJKunderwave{晋安帝纪}曰:“敳字仲凯,济阳人。祖正(统),散骑常侍。父彪(虨),仆射。并以义正器素,知名当世。敳历位内外,简退箸称。历黄门侍郎、骠骑谘议。”}} 直唤人取酒,自饮一碗,又不与王。王且笑且言:“那得独饮?”江云:“卿亦复须邪\footnote{须:需要。}?”更使酌与王。王饮酒毕,因得自解去。未出户,江叹曰:“人自量,固为难!”{\fzxk\zihao{6}\textcolor{red}{\CJKunderwave{宋书}曰:“敳,即湘州江夷之父也。夷字茂远,湘州刺史。”}}

{\cangkai\zihao{5}【评】王恭才地高华,少有美誉,有人伦之望。江卢奴平庸之士,一生行事史不详载,盖凭借其父祖之功而妄自托大,不通人情常理。故事以白描手法传神地勾勒出主客的鲜明形象:主人江卢奴傲慢狂妄、目中无人,是一个自大狂;客人王恭委曲求全、唯唯诺诺,俨然一个可怜虫。王恭本是一个仗气使才、不甘人下的血性之人,为何此处表现得柔弱无骨、任人拿捏?真是费人思量。二人均与方正品格风马牛不相及。王世懋评曰:“此亦仅得简傲耳”,堪称的评。}

\lettrine{5.64} 孝武\myidx{司马曜}问王爽\myidx{王爽}\footnote{孝武:东晋帝司马曜。王爽(?—398):东晋人,字季明,小字睹。王恭弟。王恭起兵,爽参军事,事败被诛。}:“卿何如卿兄\myidx{王恭}?”王答曰:“风流秀出\footnote{风流:指风采神韵。秀出:优秀杰出。},臣不如恭。忠孝亦何可以假人\footnote{何:怎。假:借与,给予。这句话意思是忠孝应属于王爽自己。}!”{\fzxk\zihao{6}\textcolor{red}{\CJKunderwave{中兴书}曰:“爽忠孝正直。烈宗崩,王国宝夜开门入,为遗诏。爽为黄门郎,距之曰:‘大行晏驾,太子未立,敢有先入者斩!’国宝惧,乃止。”}}

{\cangkai\zihao{5}【评】恭、爽兄弟并为太原王氏佳子弟。王恭美姿仪,人目之云:“濯濯如春月柳。”尝被鹤氅裘,涉雪而行,名士孟昶叹为“神仙中人”(\CJKunderwave{晋书}本传),故爽评恭“风流秀出”之语不虚。王爽以忠孝自评,实蕴涵了高自期许之意。孝武帝崩,奸佞王国宝欲夜入宫廷篡改遗诏,爽为黄门侍郎,声色俱厉,坚拒之于门外,堪称危难见忠良。王爽答孝武帝语,运用魏晋人物品评惯用的意象批评模式,刘辰翁评曰“善对”,大体抓住了人物的神韵。王恭不仅风流秀出,还清操过人,深存节义。其抗节奸佞司马道子、王国宝之徒,何必减于乃弟王爽?盖其名士风流掩其忠臣之质耳。}

\lettrine{5.65} 王爽\myidx{王爽}与司马太傅\myidx{司马道子}饮酒\footnote{王爽(?—398):东晋人,字季明,小字睹。王恭弟。王恭起兵,爽参军事,事败被诛。司马太傅:指会稽王司马道子,简文帝子,进位太傅。},太傅醉,呼王为“小子”\footnote{小子:对人的蔑称。}。王曰:“亡祖长史\footnote{亡祖长史:爽祖父王濛,曾作司徒左长史。},与简文皇帝为布衣之交\footnote{布衣之交:平民朋友。特指官僚贵族未显贵时的结交、友情。};亡姑、亡姊,伉俪二宫\footnote{伉俪:夫妻。用为动词。伉俪二官,即做两宫皇后。}。何小子之有?”{\fzxk\zihao{6}\textcolor{red}{\CJKunderwave{中兴书}曰:“王濛女,讳穆之,为哀帝皇后。王蕴女,讳法惠,为孝武皇后。”}}

{\cangkai\zihao{5}【评】王爽之答语中似夹杂着刚直与自负两种情绪。一方面,刚直不阿是王氏家风,乃祖王濛、乃兄王恭,均是抗节直行禀性。其父王蕴,态度稍微缓和,然亦是正道直行、务存仁爱。因此,王爽机智地回应司马道子“何小子之有”,是情在理中;另一方面,其家族显赫的门第及与皇室千丝万缕的关联,为其自负心理奠定了坚实基础。凌濛初评爽“真是卖弄”,只说对了一半,其实卖弄中藏着自负、高傲,又暗含了与皇族平起平坐的微妙心理。}

\lettrine{5.66} 张玄\myidx{张玄}与王建武\myidx{王忱}先不相识\footnote{张玄:即张玄之,字祖希。王建武:指王忱,(?—398):孝武帝后兄,安帝舅父。与殷仲堪、桓玄等,二次兴兵清君侧,兵败被诛。会稽:郡治在今浙江绍兴市。},{\fzxk\zihao{6}\textcolor{red}{张玄,已见。建武,王忱也。\CJKunderwave{晋安帝纪}曰:“忱初作荆州刺史,后为建武将军。”}} 后遇于范豫章\myidx{范宁}许\footnote{范豫章:指范宁。曾任豫章太守,故称。豫章:郡名,辖境为今江西省大部分地区,郡治在今南昌。此指豫章太守。许:处,处所。},范令二人共语。{\fzxk\zihao{6}\textcolor{red}{范甯,已见。}} 张因正坐敛衽\footnote{正坐敛衽:形容态度严肃、恭敬。敛衽,提起衣襟表示恭敬。},王熟视良久,不对。张大失望,便去,范苦譬留之\footnote{苦:极力,竭力。譬:晓喻,劝喻。},遂不肯住。范是王之舅,{\fzxk\zihao{6}\textcolor{red}{\CJKunderwave{王氏谱}曰:“王坦之娶顺阳郡范汪女,名盖,即甯妹也。生忱。”}} 乃让王曰\footnote{让:以辞相责。}:“张玄,吴士之秀\footnote{吴士:吴地士人。秀:优秀者。},亦见遇于时\footnote{见遇于时:受人敬重,得志于时。},而使至于此,深不可解。”王笑曰:“张祖希若欲相识,自应见诣\footnote{见诣:拜见我。见,表示动作偏指一方。}。”范驰报张,张便束带造之\footnote{束带:谓整饬,以表庄重。造:登门拜访。}。遂举觞对语,宾主无愧色。

{\cangkai\zihao{5}【评】王忱于舅氏范甯处遇吴地之秀张玄,因自以名门名流,矜持自高,而不与交语。“若欲相识,自应见诣”一语,意谓张玄应正式登门拜访,以示其诚意,座上相识,殊有损己之尊严。盖高门子弟因身份意识过于强烈,时时横亘胸中,发为言行,过分在意形式,以满足其如玻璃般易碎的虚荣心。真正的玄学精神,是与繁文缛节相背离的。试想,玄学尚清崇简,希心高远,又怎会在意一些鸡毛蒜皮的细枝末节?王忱激烈过火的举止背后,适足以说明其学得的,仅是名士风度的皮毛而已。我们进而可以揣测,吴人张玄所追慕的,并不是什么不可企及的高标,王忱这样的“名士”都可使其甘心自降身价,“束带造之”,其人品味也就不言自明了。}




%%% Local Variables:
%%% mode: latex
%%% TeX-engine: xetex
%%% TeX-master: "../Main"
%%% End:
