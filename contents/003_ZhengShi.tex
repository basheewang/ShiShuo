%% -*- coding: utf-8 -*-
%% Time-stamp: <Chen Wang: 2025-11-30 15:48:51>

% ○ ◎ ‧ 「 」 『 』 々 ( ) “ ” ■
% 【\([^】][^】][^】]+\)】 → {\\fzxk\\zihao{6}\\textcolor{red}{\1}}
% \(【评】.*\) → {\\cangkai\\zihao{5}\1}
% \(【题解】.*\) → {\\cangkai\\zihao{5}\1}
% 《\([^》]+\)》 → \\CJKunderwave{\1}
% ^\([0-9]+.[0-9]+\) → \\lettrine{\1}

\setlength{\parindent}{0pt}

\chapter{政事第三}


{\cangkai\zihao{5}【题解】政事,就是布政治事,倘依孔门“政事:冉有、季路”(\CJKunderwave{论语·先进})之意,也指治事理政的才能。本门26则虽不多,却折射了汉末、魏晋时期政治的纷纭变化及理政的才士们,既循古贤风范又因时而变的治事特色。}

{\cangkai\zihao{5}通贯于\CJKunderwave{政事}的一个主旨,就是以儒家的仁爱之德为根本去治事理民。不论在世风日下的汉末,还是在动荡不安的两晋,上到秉国之钧的丞相,下到宰守一方的郡主、县官,其政之美,都体现了为官者的仁爱之心,于是在\CJKunderwave{政事}一门便流溢着至为动人的人性之美。本门人物,或忠孝仁爱,矫风厉俗;或为政勤勉,不倦于职事;或妙识贤才,选无遗俊;或执法严肃,除恶果决,种种侧面,将身居要津的才士风貌演绎而出。就中最为动人的是王导、谢安的宰相风范。他们通识达变,谋深思远,没有循规蹈矩,一任宰辅而求天下之全,东晋王朝尽享了他们的流惠。\CJKunderwave{政事}把这些为政者的风采,表现得生动如绘。}

\lettrine{3.1} 陈仲弓\myidx{陈寔}为太丘长\footnote{陈仲弓:陈寔,(104—187)字仲弓,汉末颍川许昌(今属河南)人。曾任太丘长,故云。其治政清明,百姓安业,以公正直名闻世。时人评云:“宁为刑罚所加,不为陈君所短。”党锢祸起,自请系狱。卒时远近赴吊,刊石立碑,谥文范。荀朗陵:荀淑曾任郎陵侯相,故云。太丘长:太丘,县名。大县长官为令,小县为长。长,据\CJKunderwave{后汉书·百官志},万户以下县的长官。},时吏有诈称母病求假。事觉收之\footnote{收:拘捕。},令吏杀焉。主簿请付狱,考众奸\footnote{主簿:官名,中央或地方机构所设的属官,掌管文书、庶务等的官员。考:审问。众奸:其他的不法行为。}。仲弓曰:“欺君不忠,病母不孝\footnote{病:把母亲说成有病。等于诅咒。}。不忠不孝,其罪莫大。考求众奸,岂复过此?”{\fzxk\zihao{6}\textcolor{red}{陈寔已别见。}}

{\cangkai\zihao{5}【评】在汉末,陈寔以“德”名动天下,为道德楷模一类的人物。民间有言:“宁为刑罚所加,不为陈君所短”(\CJKunderwave{后汉书}本传),可见他道德风范的影响。作为太丘长,他的行政,一以道德为衡量是非曲直的标准。“忠”、“孝”是道德的根本,而汉末道德式微,王纲不振,所谓“声教废于上”,其结果便是世风混浊。为吏无信而取“诈”,就典型地表露了当时吏治的混乱状况。陈寔抓住因其“诈”中“不忠不孝”的关键环节,严厉惩治,力矫世风。\CJKunderwave{后汉书}评说陈寔这类人,在汉末的作用非同一般,因为他们的努力而使“风俗清乎下”。这除了他们个人注重道德修养外,更重要的,恐怕还是因为这类人善于运用手中的权力去力矫时弊。他们的努力确实是难能可贵的。}

{\cangkai\zihao{5}凌濛初曰:“恐亦未免矫枉。”因“诈”即杀,确有过甚之嫌,然而联系到当时世风以及陈寔这类清流对世风的切齿痛恨,似可理解他的决断。}

\lettrine{3.2} 陈仲弓\myidx{陈寔}为太丘长\footnote{陈仲弓 陈太丘:陈寔(104—187)字仲弓,汉末颍川许昌(今属河南)人。曾任太丘长,故云。其治政清明,百姓安业,以公正直名闻世。时人评云:“宁为刑罚所加,不为陈君所短。”党锢祸起,自请系狱。卒时远近赴吊,刊石立碑,谥文范。荀朗陵:荀淑曾任郎陵侯相,故云。太丘长:太丘,县名。大县长官为令,小县为长。长,据\CJKunderwave{后汉书·百官志},万户以下县的长官。},有劫贼杀财主\footnote{劫贼:强盗。},主者捕之\footnote{主者:此指主管缉捕盗贼的县尉。}。未至发所\footnote{发所:发生事情的场所,即“现场”。},道闻民有在草不起子者\footnote{在草:产妇分娩。不起子:生了孩子不养育,即溺杀婴儿。},回车往治之。主簿曰:“贼大,宜先按讨\footnote{按讨:审查惩治。}。”仲弓曰:“盗杀财主,何如骨肉相残\footnote{何如:怎么比得上。}?”{\fzxk\zihao{6}\textcolor{red}{按后汉时贾彪有此事,不闻寔也。}}

{\cangkai\zihao{5}【评】刘孝标注疑为贾彪事,按\CJKunderwave{后汉书·贾彪传}:“(彪)初仕州郡,举孝廉,补新息长……城南有盗劫害人者,北有妇人杀子者。彪出案发,而掾吏欲引南。彪怒曰:‘贼寇害人,此则常理。母子相残,逆天违道!’遂驱车北行,案验其罪。”余嘉锡先生\CJKunderwave{笺疏}以为:“仲弓、伟节(彪字伟节)同时并有此事,何其相类之甚也?疑为陈氏子孙剽取旧闻,以为美谈,而临川误以为实。”陈寔本传无此事,孝标致疑,事出有因。}

{\cangkai\zihao{5}此事原有一个前因,即贾彪本传所记:当地“小民困贫,多不养子,彪严为其制,与杀人同罪”。骨肉相残,天理人性何在?乱世盗贼固然可恶,非治不可,但比起人性的普遍泯没,后者更令人堪忧。听任此风流行,则人心大坏,所以把它当作政事之要。对于“志节慷慨”的贾彪说来,这一举动,见出了他的慷慨果敢性格和敢于矫正风气的魄力。他在这一点上的政绩,\CJKunderwave{后汉书}说:“数年间,人养子者千数,佥曰:‘贾父所长’,生男名为‘贾子’,生女名为‘贾女’。”}

{\cangkai\zihao{5}\CJKunderwave{后汉书·百官志}说:“县万户以上为令,不满为长。”贾彪为长的新息县,不足万户,竟有千数在他的严令下“养子”,也就是说不再杀子了,可见杀子之风的严重。\CJKunderwave{世说新语}选载此事,一方面突现了故事主人公的才干,另一面也旌扬着汉末清流之士的人性之美。}

\lettrine{3.3} 陈元方\myidx{陈纪}年十一时\footnote{陈元方:陈纪,太丘长陈寔之子,(104—187)字仲弓,汉末颍川许昌(今属河南)人。曾任太丘长,故云。其治政清明,百姓安业,以公正直名闻世。时人评云:“宁为刑罚所加,不为陈君所短。”党锢祸起,自请系狱。卒时远近赴吊,刊石立碑,谥文范。荀朗陵:荀淑曾任郎陵侯相,故云。},{\fzxk\zihao{6}\textcolor{red}{陈纪已见。}} 候袁公\footnote{袁公:不详。}。袁公问曰:“贤家君\myidx{陈寔}在太丘\footnote{家君:父亲。此“贤家君”为敬称别人的父亲,指陈寔。},远近称之,何所履行\footnote{履行:实行。}?”元方曰:“先父在太丘,强者绥之以德\footnote{先父:袁本作“老父”。绥:安抚。},弱者抚之以仁\footnote{抚:安慰。},恣其所安\footnote{恣:听任。},久而益敬。{\fzxk\zihao{6}\textcolor{red}{袁宏\CJKunderwave{汉纪}曰:“寔为太丘,其政不严而治,百姓敬之。”}} 袁公曰:“孤往者尝为邺令\footnote{孤:王侯自称。邺:县名。东汉属魏郡,故址在今河北临漳县西南。},正行此事。不知卿家君法孤\footnote{法:效法。}?孤法卿父?”{\fzxk\zihao{6}\textcolor{red}{检众\CJKunderwave{汉书}袁氏诸公,未知谁为邺令。故阙其文,以待通识者。}} 元方曰:“周公、孔子,异世而出,周旋动静\footnote{周旋:运筹、谋划。动静:举措,行止。},万里如一。周公不师孔子,孔子亦不师周公。”

{\cangkai\zihao{5}【评】据余嘉锡先生考证,此事“必魏、晋间好事者之所为,以资谈助,非事实也”(\CJKunderwave{世说新语笺疏})。}

{\cangkai\zihao{5}本则或“非事实”,然其反映的陈仲弓风格,却与传记所载颇相吻合。陈寔治政,只要不违反儒家道德规矩,则听其自然。责己甚严,为人表率。“无为而治者,其舜也与!夫何为哉?恭己正南面而已矣。”(\CJKunderwave{论语·卫灵公})正己而正人,以身作则,而民自化之,这正是治政事者的最高水平;无争无讼,也正是一部\CJKunderwave{论语}所憧憬的境界。其子对父亲的政事没有张大夸饰,却自有动人处。对袁公的攀附比拟,陈元方的回答,机敏妥帖,不卑不亢,既未抬举眼前的王公大人,于父亲亦无丝毫损伤。本则不入\CJKunderwave{言语},而选在\CJKunderwave{政事},或表达了临川动情于汉末混乱情形下的士人形象。王朝早已不堪,而士子为社会坚守着基本的道德准则。同时,本则故事又是儒家治政理想的演绎,所以,无论其事实与否,它的价值却是值得肯定的。另从艺术角度看,不仅问答二人,就是被议论的陈寔,其人物形象皆渲染如画。}

\lettrine{3.4} 贺太傅\myidx{贺邵}作吴郡\footnote{作吴郡:作吴郡太守。吴郡:治所在吴县,今苏州市。},初不出门\footnote{初:到任之初的一段时间。}。吴中诸强族轻之,乃题府门云\footnote{门:郡府大门。}:“会稽鸡\footnote{会稽鸡:指贺邵,会稽山阴人。},不能啼。”{\fzxk\zihao{6}\textcolor{red}{环济\CJKunderwave{吴纪}曰:“贺邵字兴伯,会稽山阴人。祖齐,父景,并历吴官。邵历散骑常侍,出为吴郡太守。后迁太子太傅。”}} 贺闻,故出行\footnote{故:特意。},至门反顾,索笔足之曰:“不可啼,杀吴儿。”于是至诸屯邸\footnote{屯邸:势家豪族在屯田基础上形成的庄园。当时顾、陆为大族,其子弟多将兵屯戍在外,而居舍庄园在吴郡,故称屯邸。},检校诸顾、陆役使官兵、及藏逋亡\footnote{检校:考校查寻。逋亡:逃亡。当时,势家豪族收纳逃亡人口隐匿不报而在庄园用为仆役,以及用所领政府兵为私家役使,都是违反法令的。},悉以事言上\footnote{悉:全部。},罪者甚众\footnote{罪:获罪。}。陆抗\myidx{陆抗}时为江陵都督,{\fzxk\zihao{6}\textcolor{red}{\CJKunderwave{吴录}曰:“抗字幼节,吴郡人,丞相逊子,孙策外孙也。为江陵都督,累迁大司马、荆州牧。”}} 故下请孙皓\myidx{孙皓},然后得释。

{\cangkai\zihao{5}【评】史称贺邵“奉公贞正,亲近所惮”(\CJKunderwave{三国志}本传)。本则就突现了这位奉公贞正的干才形象。}

{\cangkai\zihao{5}吴中向来旧族聚居,不乏贤才,其乡俗亦颇藐视外来户。三国时,吴国割据江表,吴中大族更是从容发展,势力非凡。即以陆氏而言,本世为江东大族,陆逊时“部曲已有二千馀人”,逊又为孙权所倚重,权嫁以孙策女,其后子孙亦与孙氏结为姻亲。吴中顾、陆两家亦为姻亲。当地势家豪族成盘根错节之势。有这样一些原委,会稽人贺邵到此,虽为堂堂大吏,却不免遭受书之府门的明言挑衅,就不难理解了。}

{\cangkai\zihao{5}庄园经济的发展中,储运转贩,耕渔之利皆可生财,但强势之家,收纳逃亡,藏匿户口,役使官府兵丁,这些违法之举也成了他们习以为常的生财之道。邵确为干才,对这些豪强的痼疾早了如指掌。所以一经按查,他们都纷纷败露。由此见出贺邵刚直果烈的性格和他善于发现要害,治理痼疾的聪明才智。能把吴郡陆氏治得向孙晧求情,正可见出贺邵“奉公贞正,亲近所惮”的性格与才能了。}

\lettrine{3.5} 山公\myidx{山涛}以器重朝望\footnote{器:才能。朝望:在朝中卓有声望。},年逾七十,犹知管时任\footnote{知管:掌管。时任:时政。这里主要指当时官吏的任免。}。{\fzxk\zihao{6}\textcolor{red}{虞预\CJKunderwave{晋书}曰:“山涛字巨源,河内怀人。祖本,郡孝廉。父曜,冤句令。涛蚤孤而贫,少有器量,宿士犹不慢之。年十七,宗人谓宣帝曰:‘涛当与景、文共纲纪天下者也。’帝戏曰:‘卿小族,那得此快人邪?’好\CJKunderwave{庄}、\CJKunderwave{老},与嵇康善。为河内从事,与石鉴共传宿,涛夜起蹋鉴曰:‘今何等时而眠也!知太傅卧何意?’鉴曰:‘宰相三日不朝,与尺一令归第,君何虑焉?’涛曰:‘咄!石生,无事马蹄间也。’投传而去,果有曹爽事,遂隐身不交世务。累迁吏部尚书、仆射、太子少傅、司徒。年七十九薨,谥康侯。”}} 贵胜年少\footnote{贵胜:地位显贵。年少:年轻人。},若和\myidx{和峤}、裴\myidx{裴楷}、王\myidx{王济}之徒,并共宗咏\footnote{和:和峤(?—292):魏晋时汝南西平(今属河南)人。官至中书令。为政清简得民,有风格,善礼法,朝野许其能正风俗人伦。家财富而性至吝,人称有“钱癖”。大丧:指父母之丧。据\CJKunderwave{晋书}戎传,时戎遭母丧,而峤遭父丧。;裴:裴楷,裴令公:即裴楷,曾官中书令,故云,又称“裴令”。善\CJKunderwave{老}、\CJKunderwave{易},当时著名清谈名家。二国租钱:指从梁、赵二国税收所获钱财。;王:王济。时和峤任中书令,裴楷、王济任侍中。宗咏:尊奉赞美。}。有署阁柱曰\footnote{署:题写。阁:官署,此指尚书省。}:“阁东有大牛,和峤鞅,裴楷鞦,王济剔嬲不得休\footnote{“阁东有大牛”:此四句为一歌谣,“牛、鞦、休”叶韵。大牛喻山涛。鞅:驾车时套在牲口脖子上的皮带。鞦:驾车时套在牲口股后的皮带。剔嬲(niǎo鸟):疏导纠缠。此说山涛如驾车大牛,而和峤是鞅,裴楷是鞦,辅助山涛。王济善清谈,围绕山涛,忙个不停。此四句具有嘲讽意味。}。”{\fzxk\zihao{6}\textcolor{red}{王隐\CJKunderwave{晋书}曰:“初,涛领吏部,潘岳内非之,密为作谣曰:‘阁东有大牛,王济鞅,裴楷鞦,和峤刺促不得休。’”\CJKunderwave{竹林七贤论}曰:“涛之处选,非望路绝,故贻是言。”}} 或云潘尼\myidx{潘尼}作之。{\fzxk\zihao{6}\textcolor{red}{\CJKunderwave{文士传}曰:“尼字正叔,荥阳人。祖勗,尚书左丞。父满,平原太守。并以文学称。尼少有清才,文词温雅。初应州辟,终太常卿。”}}

{\cangkai\zihao{5}【评】故事发生在晋武帝泰始十年(274),时山涛任吏部尚书,正处于全国统一的前夕。为天下一统储备人才,是其重要职责。山涛有识人之才,年过七十始领吏部,后长居选职十数年,所选举之人皆为俊才。本则涉及的和峤、裴楷、王济,并“有名当世”。他们各有特长。和峤“朝野许其能整风俗,理人伦”。人赞誉其才为,森森如千丈松,有栋梁之用(\CJKunderwave{晋书·和峤传});裴楷博涉群书,特精理义,以盛德居位,见者肃然改容;王济“少有逸才,风姿英爽,气盖一时”,“善\CJKunderwave{易}及\CJKunderwave{庄}、\CJKunderwave{老},文词俊茂,伎艺过人”,性格峻厉,处理事物以“明法绳之”(\CJKunderwave{晋书·王济传})。政事的基础是人才,山涛位居选职,能为朝廷网罗如此人才,说明了他的从政才干,而且是不可多得的干才。他能为当时颇有个性、才气的这班才士所认可,也说明了他自身德才的感召力。这里的歌谣具有嘲讽意味,余嘉锡先生谓:“以大牛比山涛,言其为人所牵制,不能自主也。”(见\CJKunderwave{世说新语笺疏})以今见\CJKunderwave{晋书}考之,未见山涛受峤、楷、济等所制约,这或是对在朝的年少才俊,围绕、拥戴山涛及对这些年少显贵之士本身的嫉妒心理,但它从反面描绘出了山涛的声望和人才济济的朝堂。\CJKunderwave{世说·政事}选此,是在识人选才的为政关键处,标举了山涛这样一位具有能量、魅力的人物,让人们从这一角度,对这样一个人物,加以品味。}

{\cangkai\zihao{5}王世懋评曰:“嵇、阮以识推山公,此是也。”说到了故事的关键处。}

\lettrine{3.6} 贾充\myidx{贾充}初定律令\footnote{定律令:指定法律和条令。司马昭为晋王在曹魏执掌朝政时,曾召集贾充、郑冲、荀勖、羊祜、裴秀等十四人,就\CJKunderwave{汉律}九章,增二十篇成\CJKunderwave{晋律},于晋武帝泰始三年(267)颁行。},{\fzxk\zihao{6}\textcolor{red}{\CJKunderwave{晋诸公赞}曰:“充字公闾,襄陵人。父逵,魏豫州刺史。尚书,迁廷尉,听讼称平。晋受禅,封鲁郡公。充有才识,明达治体,加善刑法,由此与散骑常侍裴楷,共定科令,蠲除密网,以为\CJKunderwave{晋律}。薨,赠太宰。”}} 与羊祜\myidx{羊祜}共咨太傅郑冲\myidx{郑冲}\footnote{羊祜。咨:咨询、请教。}。{\fzxk\zihao{6}\textcolor{red}{王隐\CJKunderwave{晋书}曰:“冲字文和,荥阳开封人。有核练才,清虚寡欲,喜论经史,草衣缊袍,不以为忧,累迁司徒、太保。晋受禅,进太傅。”}} 冲曰:“皋陶严明之旨\footnote{皋陶(yáo摇):虞舜时掌管刑狱的要臣。},非仆暗懦所探\footnote{暗懦:昏庸无能。探:测知。}。”羊曰:“上意欲今小加弘润\footnote{上意:皇上的意思。弘润:扩充润色。}。”冲乃粗下意\footnote{粗:粗略。下意:提出意见。}。{\fzxk\zihao{6}\textcolor{red}{\CJKunderwave{续晋阳秋}曰:“初,文帝命荀勗、贾充、裴秀等分定礼仪律令,皆先咨郑冲,然后施行。”}}

{\cangkai\zihao{5}【评】贾充为司马氏的亲信,也是司马氏政权的重要勋臣,其为人或不被肯定——“无公方之操,不能正身率下,专以谄媚取容”,然其确有才干,“雅长法理”。律令、礼仪是知人安民,治理天下的工具,律令、礼仪是否合于时代需要,也是一个王朝为政水平的重要体现。因此,司马炎抓住这一要害,令擅长法理的亲信贾充亲自董理此事,并命咨询郑冲。郑冲为当时“博究儒术及百家之言”的大学者,曾为高贵乡公亲授\CJKunderwave{尚书}。他深明先贤治国之术,敬畏皋陶这样的能臣,而其性格又是“清恬寡欲”、“任真自守,不要乡曲之誉”(见\CJKunderwave{晋书}本传),所以,他慎言行,不作狂妄迂阔之论,然一经他参与意见,事情就稳妥得多。在这样的严谨董理下,与父祖辈杀戮名士相较,这部\CJKunderwave{晋律}刑宽禁简,足当时用。本则叙述了修法过程中的一个小小的片段。就这一片段说来,足见当时这班臣子,奉旨唯谨,修订律令严肃不苟。贾充、郑冲的言行都显现着他们的个性,充分体现了\CJKunderwave{世说}注重表达人物神采、个性的旨趣。从\CJKunderwave{政事}的角度看,本则所记,在前台表演的是臣子,背后却是司马炎。它表现了司马炎的董理天下之志和善抓纲要,总领天下的为政意识以及举大事而善用人的为政才能。}

{\cangkai\zihao{5}刘辰翁认为本则:“亦非政事,”更像“言语”。但如前述,它和“言语”篇比,内在表达的实是政事。}

\lettrine{3.7} 山司徒\myidx{山涛}前后选\footnote{山司徒:指山涛,曾任司徒。选:选拔官吏。},殆周遍百官\footnote{殆:几乎。},举无失才。凡所题目\footnote{题目:品题、评价。},皆如其言。唯用陆亮\myidx{陆亮},是诏所用\footnote{诏:皇帝的命令。},与公意异,争之不从。亮亦寻为贿败\footnote{寻:不久。}。{\fzxk\zihao{6}\textcolor{red}{\CJKunderwave{晋诸公赞}曰:“亮字长兴,河内野王人,太常陆乂兄也。性高朗而率烈至,为贾充所亲待。山涛为左仆射领选,涛行业既与充异,自以为世祖所敬,选用之事,与充谘论,充每不得其所欲。好事者说充:‘宜授心腹人为吏部尚书,参同选举。若意不齐,事不得谐,可不召公与选,而实得叙所怀。’充以为然。乃启亮公忠无私。涛以亮将与已(己)异,又恐其协情不允,累启亮可为左丞相,非选官才。世祖不许,涛乃辞疾还家。亮在职果不能允,坐事免官。”}}

{\cangkai\zihao{5}【评】\CJKunderwave{晋书·山涛传}说:“涛所奏甄拔人物,各为题目,时称‘山公启事’。”涛有识人之才(参见本篇5评),又廉洁谨慎,所以能“举无失才”,其品评具有权威性。本则从另一个角度,突出了山公的知人之明。依刘孝标注,陆亮之选,是朝廷官场权势之争的结果。贾充作为皇帝亲信,自然有其优势,虽是任人唯亲,培养羽翼,但山涛却争执不过。不过陆亮后来的结局,印证了山涛在选人中间的德才。}

\lettrine{3.8} 嵇康\myidx{嵇康}被诛后\footnote{嵇康:被司马昭所杀。嵇康(223—262):三国时谯郡铚 (今安徽亳县)人。“竹林七贤”之一。曾任中散大夫,故称嵇中散。当时著名思想家、文学家、清谈名家。因其主张越名教而任自然,抨击礼法之士,不与司马氏统治集团合作,盛年被杀。},山公\myidx{山涛}举康子绍\myidx{嵇绍}为秘书丞\footnote{康子绍:嵇康之子嵇绍,字延祖。二十八岁出仕,性刚烈,敢直谏,忠于晋室,八王乱时,随惠帝与成都王司马颖战,身翼惠帝,被箭而血染帝衣。晋元帝时谥“忠穆”,载\CJKunderwave{晋书·忠义传}。秘书丞:官名。秘书监属官,掌管宫中的图籍文书, 位高于秘书郎。}。{\fzxk\zihao{6}\textcolor{red}{\CJKunderwave{山公启事}曰:“诏选秘书丞。涛荐曰:‘绍平简温敏,有文思,又晓音,当成济也。犹宜先作秘书郎。’诏曰:‘绍如此,便可为丞,不足复为郎也。’\CJKunderwave{晋诸公赞}曰:‘康遇事后二十年,绍乃为涛所拔。’”王隐\CJKunderwave{晋书}曰:“时以绍父康被法,选官不敢举。年二十八,山涛启用之,世祖发诏,以为秘书丞。”}} 绍谘公出处\footnote{出处: 出仕或隐退。},{\fzxk\zihao{6}\textcolor{red}{\CJKunderwave{竹林七贤论}曰:“绍惧不自容,将解褐,故咨之于涛。”}} 公曰:“为君思之久矣,天地四时,犹有消息\footnote{消息:此句语本\CJKunderwave{周易·丰卦}“日中则昃,月盈则食;天地盈虚,与时消息,而况于人乎?"消,灭;息,生,指盛衰变化。},而况人乎?”{\fzxk\zihao{6}\textcolor{red}{王隐\CJKunderwave{晋书}曰:“绍字延祖,雅有文才,山涛启武帝云云。”}}

{\cangkai\zihao{5}【评】作为司马氏的罪臣之子,嵇绍欲出仕司马氏王朝,当然心里不踏实,问询于山涛。涛的回答,不只是辞面上的那些推论,而是包含着其厚德为人和清醒而深刻的见识。}

{\cangkai\zihao{5}嵇康之得罪司马氏,充其量是不合作的态度,并没有真的实质性的问题。嵇康才情非凡而为人散澹,自言“但欲守陋巷,教养子孙,时时与亲旧叙离阔,陈说平生,浊酒一杯,弹琴一曲,志意毕矣”;其抨击礼法之士,鼓吹“越名教而任自然”,也不过\CJKunderwave{老}、\CJKunderwave{庄}之旨,虽不合于朝廷名教,但时风如此,显不出他个人对晋氏之教有多大破坏作用。“竹林七贤”一班人,过从往还,大抵口发玄言,深邃幽眇,少及时政。钟会构难,向司马昭陈说嵇康罪状,看去危言耸听,其实没有多少经得起推敲的东西。所以尽管司马昭为了一时的政治需要,杀了嵇康,但稍一冷静便“悟而恨焉”(以上均见\CJKunderwave{晋书·嵇康传})——后悔遗憾了。因而嵇绍作为“罪臣”之子,其父之“罪”对他影响实不甚大,况且时过境迁,晋武帝开国,政权早已稳固。此为一层。}

{\cangkai\zihao{5}山公厚德,是“可以托六尺之孤”的人,虽说嵇康与之出处态度不同,但对其为人是深知而认可的,所以临被刑诛之前,告诉嵇绍:“巨源在,汝不孤矣”(\CJKunderwave{晋书·山涛传})两人相知之深可见。这样一位父辈,嵇绍遇到难处,当然可以依赖,所以向其咨询。此为另一层。}

{\cangkai\zihao{5}合这两层内容可见,山涛之推举嵇绍,除嵇绍本身贤而可举,还有的,怕是对嵇绍释褐入仕的可能性及其前程早有周密的考虑了,这是长者、智者、仁者之心。所以,山涛在嵇绍面前只是讲了一个道理,让他放心出仕,为国效力。对老友嵇康,山涛不负所托;对晋室他保举了一个忠直贤才。作为王朝重臣,山涛为政风格就在这具体料理事物的过程中生动地展现。}

{\cangkai\zihao{5}\CJKunderwave{政事}篇选了本则,突现的是为政之人的德性、见识,同时也描画出了山涛的长者形象。}

\lettrine{3.9} 王安期\myidx{王承}为东海郡\footnote{东海郡:晋郡名,治所在郯(今山东郯城)。},{\fzxk\zihao{6}\textcolor{red}{\CJKunderwave{名士传}曰:“王承字安期,太原晋阳人。父湛,汝南太守。承冲淡寡欲,无所修尚。累迁东海内史,为政清静,吏民怀之。避乱渡江,是时道路寇盗,人怀忧惧,承每遇艰险,处之怡然。元皇为镇东,引为从事中郎。”}} 小吏盗池中鱼,纲纪推之\footnote{纲纪:州郡主簿一类的官,综理府事。推:追究。}。王曰:“文王之囿,与众共之\footnote{文王:周文王。囿:苑囿,猎场。}。{\fzxk\zihao{6}\textcolor{red}{\CJKunderwave{孟子}曰:“齐宣王问:‘文王之囿,方七十里,有诸?若是其大乎?’对曰:‘民犹以为小也。’王曰:‘寡人之囿,方四十里,民犹以为大,何邪?’孟子曰:‘文王之囿,刍荛者往焉,与民同之,民以为小,不亦宜乎?今王之囿,杀麋鹿者,如杀人罪,是以四十里为穽于国中也,民以为大,不亦宜乎?’”}} 池鱼复何足惜\footnote{复:还、又。}!”

{\cangkai\zihao{5}【评】王承弱冠知名,太尉王衍“雅贵异之”,东海王司马越推许为“人伦之表”。他的特点是言事辨物,只明其指要而不饰文辞,人服其约而能通;对人“推诚接物,尽弘恕之理”(\CJKunderwave{晋书}本传)。本则说他任东海太守时,其为政是简约弘恕,正体现了他的为人特点,其底韵,便是儒家的“爱人”思想。能身体力行,将“爱人”落到真实从政的实践中,没有一种深厚的人格修养,是很难做到的。这里王承与“纲纪”之吏形成了明显的对比,更可见王承的不同凡响——他不是以权力之威去惩治人,而是以“爱人”之心的人格去感召人,引导人。如此人物,被人推重,就是必然的了。本传说“众咸亲爱焉。渡江名臣王导、卫玠、周顗、庾亮之徒皆出其下”,可见司马越“人伦之表”的赞许,非虚语也。}

{\cangkai\zihao{5}故事以具体笔墨,刻画主人公的内心世界,是其艺术成功的一笔。}

\lettrine{3.10} 王安期\myidx{王承}作东海郡,吏录一犯夜人来\footnote{录:拘捕。犯夜:触犯宵禁的禁令。}。王问:“何处来?”云:“从师家受书还,不觉日晚。”王曰:“鞭挞宁越\myidx{宁越}以立威名,恐非致理之本\footnote{致理:余嘉锡先生曰“致理当作致治,唐人避讳改之耳”。}。”{\fzxk\zihao{6}\textcolor{red}{\CJKunderwave{吕氏春秋}曰:“宁越者,中牟鄙人也。苦耕稼之劳,谓其友曰:‘何为可以免此苦也?’其友曰:‘莫如学也。学二十岁则可以达矣。’宁越曰:‘请以十五岁。人将休,吾不敢休,人将卧,吾不敢卧。’学十五岁而为周成(威)公之师也。”}} 使吏送令归家。

{\cangkai\zihao{5}【评】参见前“评”。前则“纲纪”之吏究查小民盗渔,本则小吏逮捕犯了宵禁的书生,安期皆变通处之。其遵循的原则,就是看到鞭挞行威,不是治政之本,真正的治理应该是惠爱人民,以德为政,这才是至关紧要的根本办法。\CJKunderwave{世说·为政}选了安期这两事,就是标举“为政以德”的儒家基本理念。}

\lettrine{3.11} 成帝\myidx{司马衍}在石头\footnote{石头:城名。在都城建康西,三国吴时筑,因山为城,因江为池,地势险要,处交通要冲,为军事重镇。},{\fzxk\zihao{6}\textcolor{red}{\CJKunderwave{晋世谱}曰:“帝讳衍,字世根,明帝太子。年二十二崩。”}} 任让\myidx{任让}在帝前录(戮)侍中锺雅\myidx{锺雅}、{\fzxk\zihao{6}\textcolor{red}{\CJKunderwave{晋阳秋}曰:“让,乐安人,诸任之后。随苏峻作乱。”\CJKunderwave{雅别传}曰:“雅字彦胄,颍川长社人,魏太傅锺繇弟仲常曾孙也。少有才志,累迁至侍中。”}} 右卫将军刘超\myidx{刘超}。{\fzxk\zihao{6}\textcolor{red}{\CJKunderwave{晋阳秋}曰:“超字世踰,琅邪人,汉成阳景王六世孙。封临沂慈乡侯,遂家焉。父微为琅邪国上将军。超为县小吏,稍迁记室掾、安东舍人。忠清慎密,为中宗所拔。自以职在中书,绝不与人交关书疏,闭门不通宾客,家无担石之储。讨王敦有功,封零阳伯,为义兴太守,而受拜及往还朝,莫有知者,其慎嘿如此。迁右卫大将军。”}} 帝泣曰:“还我侍中!”让不奉诏,遂斩超、雅。{\fzxk\zihao{6}\textcolor{red}{\CJKunderwave{雅别传}曰:“苏峻逼主上幸石头,雅与刘超并侍帝侧匡卫,与石头中人密期拔至尊出,事觉被害。”}} 事平之后,陶公\myidx{陶侃}与让有旧\footnote{陶公:陶侃。在平苏峻之乱中,被推为诸军统帅,因功封长沙郡公。有旧:有老交情。},欲宥之\footnote{宥:赦免。}。许柳\myidx{许柳}{\fzxk\zihao{6}\textcolor{red}{\CJKunderwave{许氏谱}曰:“柳字季祖,高阳人。祖允,魏中领军。父猛,吏部郎。”刘谦之\CJKunderwave{晋纪}曰:“柳妻,祖逖子涣女。苏峻招祖约为逆,约遣柳以众会。峻既克京师,拜丹阳尹。后以罪诛。”}} 儿思妣\myidx{许永}者至佳,诸公欲全之。{\fzxk\zihao{6}\textcolor{red}{\CJKunderwave{许氏谱}曰:“永字思妣。”}} 若全思妣,则不得不为陶全让\footnote{全:保全。},于是欲并宥之。事奏,帝曰:“让是杀我侍中者,不可宥!”诸公以少主不可违\footnote{少主:年轻的君主。},并斩二人。

{\cangkai\zihao{5}【评】成帝幼年即位,由王导、庾亮辅政,未几即遇乱。咸和二年(327)历阳内史苏峻以讨伐庾亮为名,兴兵为乱。次年攻陷京都,焚烧宫室,并“逼迁天子于石头,帝哀泣升车,宫中痛哭”(\CJKunderwave{晋书·成帝纪}),是年成帝八岁。侍中锺雅、右卫将军刘超护卫成帝,至石头,锺、刘谋欲奉帝出逃,事泄,二人被苏峻司马任让所收。“帝抱持悲泣曰:‘还我侍中、右卫!’任让不奉诏,因害之。”(\CJKunderwave{晋书·刘超传})锺、刘不仅是这个八岁孩子的依靠,而且刘超还兼着皇帝的老师,“虽在幽厄之中,超犹启授\CJKunderwave{孝经}、\CJKunderwave{论语}”(刘超本传)。对于成帝说来,在绝路之中,这一打击是极其沉重的。无论是皇权还是情感,成帝都不能容忍任让。任让、许柳都是助峻叛乱的干将,陶侃欲全任,诸公欲全许柳之子,虽各有缘故,但诸公最后还是奉诏了结了此事。}

{\cangkai\zihao{5}临川在\CJKunderwave{政事}里选择了这一公案,无非是在宣扬为政的一个重要归结点,是忠诚、敬畏皇权,主虽少亦不可欺。“让不奉诏”是乱臣贼子;诸公不违少主,是天理所归。但本则的精彩,却在于将小皇帝描绘得如闻其声,如见其人,虽位在君主,但童稚之言,声情宛然。}

\lettrine{3.12} 王丞相\myidx{王导}拜扬州\footnote{王丞相:王导。拜扬州:晋元帝时,导受扬州刺史职。},宾客数百人并加沾接\footnote{沾接:亲近,予以惠爱。},人人有悦色。唯有临海一客姓任\myidx{任颙}{\fzxk\zihao{6}\textcolor{red}{\CJKunderwave{语林}曰:“任名颙,时宦在都,预王公坐。”}} 及数胡人为未洽,公因便还到过任边云\footnote{因:趁着。便:小便。}:“君出,临海便无复人\footnote{临海:郡名。治所在章安,今浙江临海县东南。}。”任大喜悦。因过胡人前弹指云\footnote{弹指:捻指发响。佛家风习,弹指以示欢喜、许诺、警戒等。}:“兰阇,兰阇\footnote{兰阇(shě舍):梵语音译,意约为寂静无苦恼烦乱。王导弹指、说梵语都是对胡人的尊敬、褒誉。}。”群胡同笑,四坐并欢。{\fzxk\zihao{6}\textcolor{red}{\CJKunderwave{晋阳秋}曰:“王导接诱应会,少有迕者。虽疏交常宾,一见多输写款诚,自谓为导所遇,同之旧昵。”}}

{\cangkai\zihao{5}【评】当西晋王朝风雨飘摇时,王导即先行辅助元帝司马睿经营江左,准备退守半壁江山,建立东晋朝廷。这位“少有风鉴,识量清远”(\CJKunderwave{晋书·王导传})的将相之器,在这样的大业中,抓取的最要害问题,就是吸引、延揽人才。赖他的远识和努力,江左大族的巨子顾荣、贺循、纪瞻、周玘等纷纷归附,成为过江政权得以立足的基础,司马睿开东晋王朝,称帝江左。当王导官拜“右将军、扬州刺史、监江南诸军事”时,他所关注的仍是吸引、延揽人才,更深更细地构建王朝基础。本则就是一幕生动的展演。他细腻、谦恭、潇洒,照顾到在座的每一个人。个人特点的不同、胡汉的差异,他都举重若轻,打点得“人人有悦色”而“四坐并欢”,看似随便为之,实则用心良苦。从经营大族巨子,到这些基层“宾客”,均显王导的见识和才能。对人的经营是大才的事业,是为政的根本,王导达到如此境界,李贽叹之为:“第一美政。只少人知。”}

{\cangkai\zihao{5}本则的细节描绘,出神入化。对任的一句话,令其快慰非常,犹如画龙点睛一样,将王导善于直取人心事的非凡洞见力活画出来;一个弹指细节,征服了举坐胡人,又将其学养见识,烘托得清晰如画。在这些细节的点染中,让一个富于人格魅力,举手投足都可征服人的贤相跃然纸上。}

\lettrine{3.13} 陆太尉\myidx{陆玩}诣王丞相\myidx{王导}谘事\footnote{陆太尉:陆玩。见刘孝标注,卒后追赠太尉。谘事:汇报、商议事情。},过后辄翻异\footnote{翻异:更改。}。王公怪其如此,后以问陆。{\fzxk\zihao{6}\textcolor{red}{\CJKunderwave{陆玩别传}曰:“玩字士瑶,吴郡吴人。祖(瑁)父英,仕郡有誉。玩器量淹雅,累迁侍中、尚书左仆射、尚书令,赠太尉。”}} 陆曰:“公长民短\footnote{公长民短:长,尊;短,卑。民,晋时某地的人,对地方长官说话,自称曰“民”,时王导领扬州刺史,玩为吴人,属扬州管辖,故自称“民”。},临时不知所言,既后觉其不可耳。”

{\cangkai\zihao{5}【评】陆氏为吴地大族,代出公卿,其族人骨子里的高傲是不加掩饰的,陆玩也不例外。王导虽是高贵权臣,陆玩并不屈就。本传记载,王导几次与玩交好,都未能惬意。陆玩本身弱冠即有美名,器量淹雅,处事清平允当,有他自己的见识和才干。本则记其与王导议事,常不照议定的意见办,这大概不是阳奉阴违、口是心非。王导苦心为元帝经营江左,玩不会不解,而玩也是晋室的忠臣,受晋室之官,尽心谋晋室之事,在平定苏峻叛乱中立过大功,后位登公辅。可见玩之不从王导的意见,并非故意与王过不去。作为吴地土著,他的想法和作为可能更符合当地情况,而有些东西是无法和王导这刚过江的北人商议的,当地的风习、心理等等也不是一时说得清的,所以动辄“翻异”,而未见其把事情办坏,王导只是不解其“翻异”,而未怪其办坏了事。就“政事”角度说,本则一方面窥见南北士人的微妙矛盾和相互试探的心理,另一方面又显示了陆玩从政办事的才干,让北来的王朝服江南的水土。从描画陆玩其人的角度看,他辞面谦和的回答之下,实际上显示了其个性和主见,不唯上,不唯官,不苟且。一则小小的故事,展现的是其人一生的脾气秉性。}

\lettrine{3.14} 丞相\myidx{王导}尝夏月至石头看庾公\myidx{庾亮}\footnote{丞相:王导。}。庾公正料事\footnote{料事:处理事务。},丞相云:“暑可小简之\footnote{小简:稍微疏略。}。”庾公曰:“公之遗事\footnote{遗事:遗漏政务。},天下亦未以为允\footnote{允:妥当。}。”{\fzxk\zihao{6}\textcolor{red}{\CJKunderwave{殷羡言行}曰:“王公薨后,庾冰代相,网所刑岐(峻)。羡时行,遇收捕者于途,慨然叹曰:‘丙吉问牛,似不尔!’旧(尝)从容谓冰曰:‘卿辈自是网目不失,皆是小道小善耳。至如王公,故能行无理事。’谢安石每叹咏折唱。庾赤玉曾问羡:‘王公治何似?谁是所长?’羡曰:‘其馀令责,不复称论。然二捉三治,一休三败。’”}} \footnote{刘孝标注有版本异同。“网所刑岐”,别本作“网密刑峻”,“岐”,通“峻”。“称”,别作“喘”。“旧”,别作“尝”。“折唱”,别作“此唱”。“谁”,别作“讵”。“令责”,别作“令绩”。“二捉”,别作“三捉”。“一休”,别作“三休”。}

{\cangkai\zihao{5}【评】庾亮、庾冰兄弟作为国舅,一是自身修养甚好,“亮以名德流训,冰以雅素垂风”(\CJKunderwave{晋书}本传),二是深惧汉代姻党外戚之祸,所以勤政唯谨。然而,这隅居江南,仅半壁江山的王朝,内忧外患,稳定如恐不及,庾氏兄弟偏皆崇尚刑威,“亮任法裁物”,“冰颇任刑威”而颇以此失人心。以此当国理政,愈是勤劳王事,积怨就愈多,甚至搞得皇帝都兢危忧惧,成帝对庾怿说:“大舅已乱天下,小舅复欲尔邪!”吓得庾怿饮鸩自尽。这是庾氏兄弟的思维定式,可以说不谙时势,所操之术,有点南辕北辙意味。王导见庾冰用功勤勉,苦操其术,便委婉地提醒他。可两人观点、方法有根本的差异,所以庾冰不能接受王导的告诫,反而反唇相讥。}

{\cangkai\zihao{5}从政事角度讲,这一幕现象的背后,包含着这样的历史教训:操国权柄的重臣,对时势判断准确,把握大局,尤为重要。从记人的角度看,表面是看到王、庾观点不同,深层次却是表现了王导高于庾氏一筹的睿智和仁厚,友情提醒,以免其深构祸端。从当时王、庾两势族的地位看,王导是与得势发展的庾氏势族争衡,以保全他们琅邪王氏的地位。}

\lettrine{3.15} 丞相\myidx{王导}末年,略不复省事\footnote{略:全,几乎。不复:不再。省事:处理政务。},正封箓诺之\footnote{正:仅,只。封箓:封好的簿籍文书。诺:此指在公文上画诺,表示同意。}。自叹曰:“人言我愦愦\footnote{愦愦:糊涂。},后人当思此愦愦\footnote{思:怀念。}。”{\fzxk\zihao{6}\textcolor{red}{徐广\CJKunderwave{历纪}曰:“导阿衡三世,经纶夷险,政务宽恕,事从简易,故垂遗爱之誉也。”}}

{\cangkai\zihao{5}【评】王导历辅元、明、成三帝。元帝甫过江,草创东晋朝廷,北方权贵进驻江南,面临土著豪强,有能否安家落户的大问题;明帝在位三年而崩,成帝六岁即位,所谓“主幼时艰”。三朝天下,内忧外患,动荡不安。王导在这样的情势下辅政,“务存大纲,不拘细目”,延揽人才、网罗土著豪强,“以宽和得众”(\CJKunderwave{晋书·庾亮传})。其政宽惠,甚至宽到顾和所云“网漏吞舟”的程度,是王导理解时事而得出的稳定东晋王朝所必须的基本纲领,也是他三朝辅政的一贯为政风格。不求细节,不苛责于人,在不理解的人看来,就是不勤政、不精明,甚至是“愦愦”。这一风格就为庾亮所不解、不容,曾“率众黜导”,陶侃也欲起兵废导,都因郗鉴“不许”而作罢。郗鉴在两晋之际的动乱中饱经忧苦,就是因为宽惠得人心而济险越难,他深知宽惠对于稳定天下的作用。王导式的宽惠风格,如元帝的描述:德重勋高,约己冲心,以身率众(见\CJKunderwave{晋书·王导传})。正因他以如此风格的宽纵施政,才会有一种很好的绥靖效果,流惠所及,以至延长了东晋百馀年的国祚。另外,成帝之朝,外戚庾氏家族当政,“任法裁物”,琅邪王氏权势衰落,在士族门户争斗中,琅邪王家已呈衰势,导不“愦愦”,又当如何?因而“后人当思此愦愦”,其言大有意味。}

\lettrine{3.16} 陶公\myidx{陶侃}性检厉\footnote{陶公:陶侃,见刘孝标注。刘孝标注:“所曰”,别作“所由”。“二军”,别作“三军”,“诸匿”,别作“诸君”,别本是。检厉:检束严厉,办事认真。},勤于事。{\fzxk\zihao{6}\textcolor{red}{\CJKunderwave{晋阳秋}曰:“侃练核庶事,勤务稼穑,虽戎陈武士,皆劝厉之。有奉馈者,皆问其所曰(由),若力役所致,欢喜慰赐;若它所得,则呵辱还之。是以军民勤于农稼,家给人足。性纤密好问,颇类赵广汉。尝课营种柳,都尉夏施盗拔武昌郡西门所种。侃后自出,驻车施门,问:‘此是武昌西门柳,何以盗之?’施惶怖首伏,二(三)军称其明察。侃勤而整,自强不息。又好督劝于人,常云:‘民生在勤,大禹圣人,犹惜寸阴,至于凡俗,当惜分阴。岂可游逸,生无益于时,死无闻于后,是自弃也。又\CJKunderwave{老}、\CJKunderwave{庄}浮华,非先王之法言而不敢行。君子当正其衣冠,摄以威仪,何有乱头养望,自谓宏达邪?’”\CJKunderwave{中兴书}曰:“侃尝捡校佐吏,若得樗蒲博弈之具,投之曰:‘樗捕,老子入胡所作,外国戏耳。围棋,尧舜以教愚子。博弈,纣所造。诸匿(君)国器,何以为此?若王事之暇,患邑邑者,文士何不读书?武士何不射弓?’谈者无以易也。”}} 作荆州时\footnote{作荆州时:任荆州刺史时。},敕船官悉录锯木屑\footnote{敕官船:命令负责造官船的官员。录:收集、收藏。},不限多少,咸不解此意\footnote{咸:都、全。}。后正会\footnote{正会:农历正月初一,大会群僚。也称元会。},值积雪始晴,听事前除雪后犹湿\footnote{听事:厅堂。指官府办公的大堂。前除:堂前台阶。},于是悉用木屑覆之,都无所妨。官用竹皆令录厚头,积之如山。后桓宣武\myidx{桓温}伐蜀\footnote{桓宣武:桓温,桓公北征:桓温曾有三次北征,刘盼遂\CJKunderwave{世说新语校笺}考订,此次当为太和四年(369)之征。时桓温已58岁。伐蜀:西晋惠帝永兴二年(306)李雄在四川称帝,国号“大成”,东晋成帝时(338),李寿改为“汉”,史称“成汉”。东晋穆帝永和二年(346)冬,荆州都督桓温率军沿江而上,直捣蜀中“成汉”,次年春,成汉国灭。},装船悉以作钉\footnote{装:修造、装配。}。又云:尝发所在竹篙,有一官长连根取之,仍当足\footnote{仍:乃,于是。当足:用坚硬的竹根当作篙下的铁脚。},乃超两阶用之\footnote{超两阶用之:超越两级提拔任用此人。}。

{\cangkai\zihao{5}【评】陶侃的特点是,不仅性聪敏,而且毕生厉志不辍,勤于职事,恭谨细心,对部下管束严整,从而成就了他的功业名声。}

{\cangkai\zihao{5}本则三事,都是他勤于政事的点滴记录,也活画出其人的风格特色。他能于平常人毫不介意处,发现有大用的小事小物,并都把它们派在了紧要处,这非有为职事思深虑远,细心留意的习性是办不到的。对陶侃说来,这一颇具特色的习性的养成,怕是有两个重要原因。一是他的厉志不辍。见诸\CJKunderwave{晋书}本传和\CJKunderwave{世说}刘孝标注,他向以大禹的精神砥砺自己、教育部下,立志生益于时,死闻于后,十分珍惜人生价值。所以他能不务浮华,不尚逸豫,竞惜光阴,对职事心细纤密。在魏晋以饮酒任达相高的时尚中,他能“饮酒有定限”,绝不荒醉误职,自我约束如此,均见其与众不同。这些无不是厉志的结果。二是出身寒微。因其早孤微寒,尝到了寒素底层的艰辛,所以能勤俭而惜人、惜物。在\CJKunderwave{庾亮传}中有一细节:陶侃招待庾亮吃韭菜,亮留下了韭菜的根白部分。陶问留此何用?亮答可以种。陶大加赞赏,说亮“非惟风流,兼有为政之实”,这“实”便是惜物。此与本则三事旨趣无异。能细到木屑、竹头都惜而致用,奖赏能变废为用的小吏,其实反映了他的寒素本色,这与世家大族出身之官僚习惯于挥霍铺张,暴殄天物,大异其趣。正因为如此种种,陶侃在魏晋人物中,便显出了拔出群英的独特的动人风采。}

\lettrine{3.17} 何骠骑\myidx{何充}作会稽\footnote{何骠骑:何充,见刘孝标注,何骠骑:何充,字次道,晋康帝时为骠骑将军。作会稽:任会稽内史。},{\fzxk\zihao{6}\textcolor{red}{\CJKunderwave{晋阳秋}曰:“何充字次道,庐江人。思韵淹通,有文义才情。累迁会稽内史、侍中、骠骑将军、扬州刺史,赠司徒。”}} 虞存\myidx{虞存}弟謇\myidx{虞謇}作郡主簿\footnote{主簿:官名。古代中央或地方郡县所设属官,掌管文书簿籍。},{\fzxk\zihao{6}\textcolor{red}{孙统\CJKunderwave{存诔叙}曰:“存字道长,会稽山阴人也。祖阳,散骑常侍。父伟,州西曹。存幼而卓拔,风情高逸,历卫军长史、尚书吏部郎。”范汪\CJKunderwave{棋品}曰:“謇字道直,仕至郡功曹。”}} 以何见客劳损,欲断常客,使家人节量,择可通者,作白事成以见存\footnote{白:禀报。白事:下对上陈说事情的文书。}。存时为何上佐\footnote{上佐:长官的高级助手,如别驾、长史、司马、治中等。},正与謇共食,语云:“白事甚好,待我食毕作教\footnote{作教:做出批复、指示。教:上对下的发出的命令、指示。}。”食竟,取笔题白事后云:“若得门亭长如郭林宗\myidx{郭泰}者\footnote{门亭长:州郡属吏,掌传达、接待之事。},当如所白。{\fzxk\zihao{6}\textcolor{red}{\CJKunderwave{泰别传}曰:“泰字林宗,有人伦鉴识,题品海内之士,或在幼童,或在里肆,后皆成英彦六十馀人。自著一卷,论取士之本,未行,遭乱亡失。”}} 汝何处得此人?”謇于是止。

{\cangkai\zihao{5}【评】余嘉锡先生\CJKunderwave{笺疏}说:“充之为人,乃不择交友者。其作会稽时,必已如此。虞謇盖嫌其宾客繁猥,欲加以节量,不独虑其劳损而已。”\CJKunderwave{晋书}本传亦言:何充“所昵庸杂,信任不得其人”。正因其不能识人,庸杂交往,所以才会泥沙俱下,什么人都造府拜访,主簿不唯忧心其劳,亦忧其交往不慎,轻而损毁名声,甚而可得祸患。虞存引郭泰,虽借以喻门亭长的水平,客观上也与何充做了一个鲜明对比。郭泰在汉末是以“有人伦识鉴”而享誉天下的。其所品评人物,无一不中肯应验。所誉之人,多在卑微,然如其所言,终或显达、或饮誉于时;所否定之士,虽当时有名,终至祸亡、或身败名裂。而他自己,交友谨慎,在当时险恶的政治环境下,善得其终。倘若司客的门亭长能有如此鉴识,方可副虞謇白事之望,然主人尚无此识鉴水平,怎望得此属吏呢?}

{\cangkai\zihao{5}本则记的是两个近侍僚属谈论主子事情,实则从这一侧面,反映了为政识人的重要。另外也反映了历任显官的何充,为政勤勉,接遇宾客,不辞辛劳。其人也颇有些感召力,曾获“有万夫之望”的赞誉。}

\lettrine{3.18} 王\myidx{王惔}、刘\myidx{刘惔}与林公\myidx{支遁}共看何骠骑\myidx{何充}\footnote{王:王惔。刘:刘惔,字真长,曾任丹阳尹,故称。谢安妻兄,尚明帝女庐陵公主。会稽王司马昱为相,与王濛并为其座上清谈之客。性简贵自重,与王羲之友善。卒年三十六。林公:支遁,为东晋名僧,善玄理,是当时佛学“般若学”的代表人物,多才艺,长于草隶。与王洽、刘惔、殷浩、许询、郗超、王羲之、谢安等名流游好。常:同“尝”,曾经。何骠骑:见前则。},骠骑看文书不顾之\footnote{文书:公文。}。{\fzxk\zihao{6}\textcolor{red}{\CJKunderwave{晋阳秋}曰:“何充与王濛、刘惔好尚不同,由此见讥于当世。”}} 王谓何曰:“我今故与林公来相看\footnote{故:特意。相看:看你。},望卿摆拨常务\footnote{摆拨:摆脱。},应对共言\footnote{应对:答对。共言:袁本作“玄言”。三位都是健谈玄言的名士,特意访何充,所欲谈,无非玄言。},那得方低头看此邪\footnote{方:尚。}?”何曰:“我不看此,卿等何以得存\footnote{存:存活,生存。}?”诸人以为佳。

{\cangkai\zihao{5}【评】何充是“风韵淹雅,文义见称”的才士,但性好释典,迷恋于佛教。王濛、刘惔是清谈名士,崇尚\CJKunderwave{易}、\CJKunderwave{老}、\CJKunderwave{庄},支遁虽在沙门,却也和王、刘一样,健谈玄学。也许是所尚不同,何充无意与他们清谈。但他们与何相交甚熟。何充善饮,惔每云:“见次道饮,令人欲倾家酿。”且何充好士,甚喜接遇宾客(参见前篇),但在这里,他却埋头公务,客人不解,以为不近人情。他的一句回答:“我不看此,卿等何以得存?”道出了这一形象的根本面貌。作为东晋王朝的重臣,他兢兢于政务,来协助朝廷苦撑半壁江山。诸名士被讥却反称充言“为佳”,则在其言寓理精深有味,合于玄言旨趣。}

\lettrine{3.19} 桓公\myidx{桓温}在荆州\footnote{桓公:桓温,桓公北征:桓温曾有三次北征,刘盼遂\CJKunderwave{世说新语校笺}考订,此次当为太和四年(369)之征。时桓温已58岁。,又见刘孝标注。},全欲以德被江、汉\footnote{全:极。被:覆盖,遍及。江、汉:长江和汉水相接的地区,即荆州地区。},耻以威刑肃物\footnote{威刑:威权刑法。肃:整肃,整治。物:人、众人。}。{\fzxk\zihao{6}\textcolor{red}{\CJKunderwave{温别传}曰:“温以永和元年自徐州迁荆州刺史,在州宽和,百姓安之。”}} 令史受杖\footnote{令史:低级官吏名,掌文书或庶物。},正从朱衣上过\footnote{正:只。朱衣:指官服。}。桓式\myidx{桓歆}年少,从外来,{\fzxk\zihao{6}\textcolor{red}{式,桓歆小字也。\CJKunderwave{桓氏谱}曰:“歆字叔道,温第三子,仕至尚书。”}} 云:“向从閤下过\footnote{向:刚才。閤下:官府前。},见令史受杖,上捎云根,下拂地足\footnote{捎:略过。云根:云边。地足:地脚。两句谓,杖刑时,杖不着人身。}。”意讥不著。桓公云:“我犹患其重\footnote{患:担心。}。”

{\cangkai\zihao{5}【评】本则\CJKunderwave{渚宫旧事·五}作桓冲事。桓温在永和元年(345)治荆州;桓冲在太元二年(377),也自徐州迁荆州。余嘉锡先生\CJKunderwave{笺疏}谓:“云耻以威刑肃物,在州宽和,殊不类温之为人。桓式语含讥讽,亦不类子对父,似此事本属桓冲,\CJKunderwave{旧事}别有所本。\CJKunderwave{世说}属之桓温,乃传闻异辞,疑不能明,俟更详考。”但桓温是个集雄心与野心于一身的一代枭雄,其谋事之初,以宽和之政争取士心民意,如\CJKunderwave{黜免}第2则,其入蜀时部下杀三峡猿子,温怒,“命黜其人”。其待猿如此,待人可想而知。以此,事出桓温,亦属可能。总之,因传闻异辞,史家亦难以考明确为某人之事,但该事之主旨是明朗的。临川录此于\CJKunderwave{政事},正为称赏宽仁之政,在骨子里,还是崇尚儒家为政以德的从政精神。}

\lettrine{3.20} 简文\myidx{司马昱}为相\footnote{简文:东晋简文帝,晋简文:指晋简文帝司马昱(320—372),穆帝年幼即位,昱任抚军大将军总理政务。后来大将军桓温专擅朝政,先废海西公,后立司马昱为帝,第二年崩。简文即皇帝位前,于穆帝永和元年(345),任抚军大将军,录尚书六条事,掌管朝政,故称“为相"。太宗为其庙号。},事动经年\footnote{动:动辄,动不动。},然后得过\footnote{过:做完。}。桓公\myidx{桓温}甚患其迟,常加劝勉。太宗\myidx{司马昱}曰:“一日万机,那得速!”{\fzxk\zihao{6}\textcolor{red}{\CJKunderwave{尚书·皋陶谟}:“一日万机。”孔安国曰:“几,微也。言当戒惧万事之微。”}}

{\cangkai\zihao{5}【评】王世懋曰:“简文能言,谢安石以为惠帝之流,其当坐此。”谢安因简文为政无能,论其为惠帝之流,本则辞面似就表现了其无能。其实,简文与惠帝,别如天壤。惠帝本白痴,无所谓能与不能。简文则有其聪明,人多见其能言而不首肯其为政。其实,他执政时,表面是“桓与马,共天下”,但门阀政治的天秤,已向桓氏严重倾斜。因受权臣桓温威逼,使这位原本聪明能清言的皇室执政,日日如坐针毡、如履薄冰。如果按照桓温的催促,提高办事效率,恐怕除了加速政权从司马皇室向桓氏集团转移之外,不会有其他结果。因此“事动经年”,实是一种拖延以待时变的政治游戏,“一日万机”之说,不过是出于拖延战术的托词而已。简文答辞,意在言外,令人深思。}

{\cangkai\zihao{5}非其不勤,亦非百无一能,正是历史的尴尬,令简文从为相到做皇帝,都暗然无亮色,只能以一个无能的形象展演于世。悲乎!}

\lettrine{3.21} 山遐\myidx{山遐}去东阳\footnote{去:离开。指离任,卸任。东阳:郡名,治所在长山县,今浙江金华。此指东阳太守。},王长史\myidx{王濛}就简文\myidx{司马昱}索东阳云\footnote{王长史:王濛,曾做过司徒长史,故称。索东阳:求为东阳太守。索,求、要。}:“承藉猛政\footnote{承藉:继承凭借。},故可以和静致治\footnote{致治:达到清明安定。}。”{\fzxk\zihao{6}\textcolor{red}{\CJKunderwave{东阳记}云:“遐字彦林,河内人。祖涛,司徒。父简,仪同三司。遐历武陵王友、东阳太守。”\CJKunderwave{江惇传}曰:“山遐之为东阳,风政严苛,多任刑杀,郡内苦之。惇隐东阳,以仁恕怀物,遐感其德,为微损威猛。”}}

{\cangkai\zihao{5}【评】山遐前为馀姚令时,浙东偏远,法禁宽弛,豪族多藏匿户口。遐施以峻法,整肃地方,颇有效果。及为东阳太守,仍为政严猛,虽多罪人,然郡境肃然。施峻法以治,可以威镇,却因触及了豪强既得利益,令其切齿愤恨,也存在着不安定因素。所以王濛想承遐之后,以清静温和之政来完善东阳的地方治理。这是王濛的预想,也是他的特点。濛以“清约”见称,为人温润恬和,这次想到地方实践他的治政理想。结果是简文没答应他,王濛为此,至死都留着遗憾。\CJKunderwave{世说·政事}选此条是在着意推举儒家“温而厉,威而不猛”、中和平衡的治政理想。如果将山遐的做法与王濛的理想相糅合,就形象地表达出了儒家的治政图画。}

\lettrine{3.22} 殷浩\myidx{殷浩}始作扬州\footnote{殷浩(?—356):见刘孝标注。浩善谈玄,负盛名,简文执政时惧桓温势盛,引浩为建武将军、扬州刺史,以对抗桓温。后因北征许洛败绩,为桓温所弹,废为庶人。},{\fzxk\zihao{6}\textcolor{red}{\CJKunderwave{浩别传}曰:“浩字渊源,陈郡长平人。祖识,濮阳相。父羡,光禄勋。浩少有重名,仕至扬州刺史、中军将军。”\CJKunderwave{中兴书}曰:“建元初,庾亮兄弟、何充等相寻薨,太宗以抚军辅政,征浩为扬州,从民誉也。”}} 刘尹\myidx{刘惔}行\footnote{刘尹:刘惔,字真长,曾任丹阳尹,故称。谢安妻兄,尚明帝女庐陵公主。会稽王司马昱为相,与王濛并为其座上清谈之客。性简贵自重,与王羲之友善。卒年三十六。惔作丹阳时, 殷浩任扬州刺史, 丹阳属扬州, 惔为浩属下官员。},日小欲晚\footnote{小:稍稍。},便使左右取襆\footnote{襆(fú伏):包袱。指用布帛包扎的衣被等物。},人问其故,答曰:“刺史严,不敢夜行\footnote{不敢夜行:晋律禁夜行。}。”

{\cangkai\zihao{5}【评】故事发生在穆帝永和四年刘惔任丹阳尹之时。殷浩久负盛名,屡被推举,简文也多次征辟,终于在晋穆帝永和二年三月任扬州刺史。因朝廷器重,所以浩到任后认真执政。扬州治所在建康,丹阳故城在今江苏江宁之东,相去很近。刘惔善清谈,是简文的座上宾,与殷浩曾在简文处论学谈玄。以两地距离之近,两人同为简文所重,且旧时相熟,惔尚不敢夜行赶路,而稍违法度,可见殷浩执政果然严厉。刘辰翁说惔“大是乖汉”。惔之乖,也正印证了殷浩作扬州刺史法禁之严。}

\lettrine{3.23} 谢公\myidx{谢安}时\footnote{谢公:谢安,(?—358):字无奕,谢安长兄,陈郡阳夏谢氏家族在东晋初期的代表人物之一。,在孝武帝时,安为丞相,掌朝政。},兵厮逋亡\footnote{兵厮:兵士和奴仆。逋亡:逃跑、逃亡。},多近窜南塘下诸舫中\footnote{窜:躲藏。南塘:地名,在东晋都城建康淮河南岸。舫:此泛指船。}。或欲求一时搜索\footnote{求:请求。一时:同时。},谢公不许,云:“若不容置此辈,何以为京都\footnote{京都:京城。}?”{\fzxk\zihao{6}\textcolor{red}{\CJKunderwave{续晋阳秋}曰:“自中原丧乱,民离本域,江左造创,豪族并兼,或客寓流离,名籍不立。太元中,外御强氏,蒐简民实,三吴颇加澄检,正其里伍。其中时有山湖遁逸,往来都邑者。后将军安方接客,时人有于坐言宜纠舍藏之失者。安每以厚德化物,去其烦细。又以强寇入境,不宜加动人情。乃答之云:‘卿所忧,在于客耳!然不尔,何以为京都?’言者有惭色。”}}

{\cangkai\zihao{5}【评】谢安当国之时,前秦苻坚收拾北方,师逼东晋,声势日益浩大。苻坚曾形容他的部众,“以吾之众旅,投鞭于江(长江),足断其流”。\CJKunderwave{晋书·谢安传}说“时强敌寇境,边书续至,梁、益不守,樊、邓陷没”,可谓大敌当前,国势日危。当此形势,需要的是内部安定,拿出主要精力,运筹抗敌。所以谢安的政策是:“镇以和靖,御以长算。”其风格是“不存小察,弘以大纲,威怀外著,人皆比之王导,谓文雅过之”(见\CJKunderwave{晋书·谢安传})。}

{\cangkai\zihao{5}两晋之际,北方寇乱,江北今山东、苏北、河北、皖北地区的流民便多逃至今江苏南京、镇江、常州一带。他们流离失所,依附江南世家大族,而大族也藏匿户口以增财力。\CJKunderwave{政事}21则说到的山遐,就曾严法以处置藏匿户口与国争利的豪族。这些可说是积久难办的老问题、大问题,而且极易引起不安定的内乱。这里,人劝搜索隐匿,此举虽是处理眼前逃亡的兵士、仆役之类,弄不好也会由此牵起势族隐匿户口之事,所以“谢公不许”。这是谢安把握“镇以和靖”、“不存小察”的大原则、大方向。不过谢安的回答却委婉而精彩——“若这些人都不能容纳安置,怎么称得起京都呢?”京都,又称京师。\CJKunderwave{公羊传·桓公九年}:“京师者何?天子之居也。京者何?大也。师者何?众也。天子之居,必以众大之辞言之。”京师本身就当雍容大度,藏几个逋亡小民又算得了什么呢?一句回答,飞扬着谢安的远见卓识和才华气度,正显现了力求把握国家命运的良相风貌。}

\lettrine{3.24} 王太(大)\myidx{王忱}为吏部郎\footnote{王大:王忱,即王忱,因小字佛大,故称。吏部郎:官名。魏晋时,专主官吏的选拔、考核、任免等,朝廷特重视该职人选,位也在诸曹郎之上。},{\fzxk\zihao{6}\textcolor{red}{王忱已见。}} 尝作选草\footnote{选草:准备选任的官员名单草案。},临当奏\footnote{当:将要。},王僧弥\myidx{王珉}来,聊出示之\footnote{聊:姑且、随便。}。{\fzxk\zihao{6}\textcolor{red}{僧弥,王珉小字也。\CJKunderwave{珉别传}曰:“珉字季琰,琅邪人,丞相导孙,中领军洽少子。有才艺,善行书,名出兄珣右,累迁侍中、中书令。赠太常。”}} 僧弥得便以己意改易所选者近半,主人甚以为佳,更写即奏\footnote{更写:改写。}。

{\cangkai\zihao{5}【评】王忱出自太原王氏,王珉则为琅邪王氏嫡派,二王家族之间时有利益矛盾,但王忱与王珉之间则泯恩仇而共为朝廷着想。王忱是个耿介放达之士,连以雄豪著称的桓玄都对他“惮而服焉”。他任吏部郎,主持选官,权重位尊,能够听任别人更改他选拟的名单近半,实属不易。余嘉锡先生的评析至为确当:“此见王珉意在奖拔贤能,不以侵官为虑。而王忱亦能服善,惟以才为急,不以侵己之权为嫌。为王珉易,为王忱难。”(见\CJKunderwave{世说新语笺疏})}

\lettrine{3.25} 王东亭\myidx{王珣}与张冠军\myidx{张玄之}善\footnote{王东亭:王珣。封东亭侯,故称。张冠军:张玄之。为冠军将军,故称。}。{\fzxk\zihao{6}\textcolor{red}{张玄之,已见。}} 王既作吴郡\footnote{作吴郡:作吴郡太守。},人问小令\myidx{王珉}曰\footnote{小令:指王珉。王献之为中书令,后王珉代之,世称大、小王令。}:{\fzxk\zihao{6}\textcolor{red}{\CJKunderwave{续晋阳秋}曰:“王献之为中书令,王珉代之,时人曰‘大、小王令。’”}} “东亭作郡,风政何似\footnote{风政:教化、政绩。}?”答曰:“不知治化何如,唯与张祖希情好日隆耳。”

{\cangkai\zihao{5}【评】刘孝标注\CJKunderwave{言语}引\CJKunderwave{续晋阳秋}称张玄之:“少以学显,历吏部尚书,出为冠军将军、吴兴太守。会稽内史谢玄同时之郡,论者以为南北之望。玄之名亚谢玄,时亦称南北二玄。”可见张玄是时人十分推服的贤人雅望。王珉这里不直接述说家兄政绩、风教如何,只是点染了他与贤人情好日隆,看似文不对题,实则画龙点睛般描述出了其兄的善政。李贽评曰:“此是一等治化。”一心尊贤敬能,其风政不问可知。}

\lettrine{3.26} 殷仲堪\myidx{殷仲堪}当之荆州\footnote{殷仲堪:(?—399):善清谈,当时与韩康伯齐名。当:将要。荆州:州名,治所在江陵。此指荆州刺史。},王东亭\myidx{王珣}问曰\footnote{王东亭:王珣,见前则。}:“德以居全为称\footnote{德:德政。全:完整。称:声誉。},仁以不害物为名\footnote{仁:仁爱。名:扬名。}。方今宰牧华夏\footnote{宰牧:掌管、治理。华夏:古代以华夏称中国或中原地区,此指荆州地区。},处杀戮之职,与本操将不乖乎\footnote{本操:素来的操守。将不:莫不、岂不是。乖:违。}?”殷答曰:“皋陶造刑辟之制\footnote{皋陶:舜时大臣,掌刑狱。刑辟:刑法。},不为不贤;{\fzxk\zihao{6}\textcolor{red}{\CJKunderwave{古史考}曰:“庭坚号曰皋陶,舜谋臣也。舜举之于尧,尧令作士,主刑。”}} 孔丘居司寇之任\footnote{司寇:春秋时官名,掌刑狱、纠察。},未为不仁。”{\fzxk\zihao{6}\textcolor{red}{\CJKunderwave{家语}曰:“孔子自鲁司空为大司寇,七日而诛乱法大夫少正卯。”}}

{\cangkai\zihao{5}【评】王珣和殷仲堪是老熟人,并以才学文章见昵于孝武帝。王珣记性好,善识人,以他对仲堪的了解,深知其有才学而“怠行仁义”的特点,所以当仲堪将就荆州重职、主宰一方的时候,提醒他仁义乃为政之本。仲堪亦确为能言才士,一席回答,说出了仁义与刑辟之关系的大道理。仁义的根本是“爱人”,而真正达到令大多数人都能受到爱护,确保其安定、利益,那就非有严肃的秩序不可。刑法制度就是人们安定的保障,所以古圣贤自觉地建设刑法制度,并严肃地执行它。可见,以仁义为“本操”,同时也不能忽略刑法的健全。汉宣帝早已声明,汉家制度“本以霸王道杂之”(\CJKunderwave{汉书·元帝纪}),是仁义与刑法两手并重。魏晋时亦然。本则的这一问对,其实表达了选编\CJKunderwave{世说}者,在对政事中仁义与刑法关系的认识。}




%%% Local Variables:
%%% mode: latex
%%% TeX-engine: xetex
%%% TeX-master: "../Main"
%%% End:
