%% -*- coding: utf-8 -*-
%% Time-stamp: <Chen Wang: 2025-12-07 19:42:11>

% ○ ◎ ‧ 「 」 『 』 々 ( ) “ ” ■ ^[一-龥]
% 【\([^】][^】][^】]+\)】 → {\\fzxk\\zihao{6}\\textcolor{red}{\1}}
% \(【评】.*\) → {\\cangkai\\zihao{5}\1}
% \(【题解】.*\) → {\\cangkai\\zihao{5}\1}
% 《\([^》]+\)》 → \\CJKunderwave{\1}
% ^\([0-9]+.[0-9]+\) → \\lettrine{\1}
% {\\fzxk\\zihao{6}\\textcolor{red}{[^o]*}}



\setlength{\parindent}{0pt}


\chapter{任诞第二十三}




{\cangkai\zihao{5}【题解】 任诞者,任达与荒诞之谓也。任达主要指无形精神世界的识见旷达,率性任情,意志自由自在而回归天性之自然;荒诞则由虚入实,着重于言语行为方面的乖戾无常,突破世俗常规而不守名教礼法。任诞之风,后人视以为怪,封建名教之士更是嫉之如仇。但如安放到魏晋的特殊时代背景中考察,则是见怪不怪,自然而然。魏晋任诞之风,经由正始名士何(晏)、王(弼)的提倡,竹林七贤嵇(康)、阮(籍)胜流的理论发挥和行为实践,高倡“越名教而任自然”,于是风行漫卷,膨胀扩张,迅速流行而成为新的时髦,也可说是一种“流行病”。任诞之风,其源何在?一是与所受老庄及玄学思想的影响有关,玄家以无为本,主张天真无为,宅心玄远,回归自然,不为物累而张扬个性;一是由于魏晋社会篡弑相继,人们高唱礼教而自相屠杀,正直士人在虚伪礼教屠刀下,朝不保夕,由此引发了“叛逆”与任诞的思想反弹。“任诞”已成为魏晋士人的一种特殊风度而见其人格之美,如宗白华\CJKunderwave{论〈世说新语〉和晋人的美}所说:“从中国过去一个同样混乱、同样黑暗的时代中,了解人们如何追求光明,追寻美,以救济和建立他们的精神生活,化苦闷而为创造,培养壮阔的精神人格。”(\CJKunderwave{宗白华全集},安徽教育出版社1996年版,第286页)当时人们以“任诞”言行来表达自由个性和精神解放的理想和愿望,这既是一种扭曲的智慧,也是一种重压之下生命热情的迸发。因此,在任诞之风诞生和发展阶段中,这是一种无奈的抗争,具有一定的历史进步意义。但是,后学则不尽然。当任诞之风已成为魏晋名士的一种特殊身份象征之后,人人都来仿效作达,而不问其原因和条件。其实,当阮浑欲效父兄作达之时,阮籍对自己的儿子说了真心话:“卿不得复尔!”可见任诞之风,因时间和条件,后学之辈,已有真达、假达之分。任诞后来成为一件流行商品,冒牌赝品和伪劣水货,充斥市场,岂非咄咄怪事。人们读\CJKunderwave{世说}时,自应分辨真伪,以便给予恰当的历史评价,从而汲取有益的思想启迪。}

{\cangkai\zihao{5}本门54则故事,其中一半以上与酒有关,但却不感单调而颇能显现当时士人精神风貌。要知其成功奥秘,可详参鲁迅\CJKunderwave{魏晋风度及文章与药及酒之关系}一文。}

\lettrine{23.1} 陈留阮籍\myidx{阮籍}、谯国嵇康\myidx{嵇康}、河内山涛\myidx{山涛}三人\footnote{陈留阮籍:籍字嗣宗,陈留(今河南开封)人。见前\CJKunderwave{德行}第15则注。谯国嵇康:康字叔夜,谯国铚(今安徽亳县)人。见前\CJKunderwave{德行}第16则注。河内山涛:涛字巨源,河内怀(今属河南)人。见前\CJKunderwave{政事}第5则注。},年皆相比\footnote{相比:相近,相仿。},康年少亚之\footnote{年少亚之:年龄稍小一些。按:阮籍生于汉建安十五年(210),山涛生于建安二十年(215),嵇康生于魏黄初四年(223)。}。预此契者\footnote{预:参预,参与。契:交游聚会。},沛国刘伶\myidx{刘伶}、陈留阮咸\myidx{阮咸}、河内向秀\myidx{向秀}、琅邪王戎\myidx{王戎}\footnote{沛国刘伶:伶字伯伦,沛(今安徽濉溪西北)人。阮咸:字仲容,籍侄。向秀:字子期。琅邪王戎:戎字濬冲,琅邪(今山东临沂北)人。}。七人常集于竹林之下,肆意酣畅\footnote{肆意酣畅:尽情饮酒。},故世谓竹林七贤\footnote{竹林七贤:指阮籍、嵇康、山涛、刘伶、阮咸、向秀、王戎七人,是当时著名士人一个比较松散而自由交游的文化聚会。}。{\fzxk\zihao{6}\textcolor{red}{\CJKunderwave{晋阳秋}曰:“于时风誉扇于海内,至于今咏之。”}}

{\cangkai\zihao{5}【评】踵武正始名士,竹林七贤在魏晋间相继出现。当时朝廷中的司马集团,掌控政权,代替曹魏皇族发号施令,推行虚伪名教统治,为其篡弑夺权制造舆论,因而出现了顺者昌、逆者亡的思想专制局面。在高压形势下,正直士人产生了严重的危机心理,并且迅速反弹。于是就有了竹林七贤文士交游聚会的出现。竹林七贤不是一个有组织的政治社团,不过因其不满现实,愤世嫉俗,不拘礼法,口无遮拦,其所议论,或许多少会牵扯某些政治现象。竹林七贤以嵇(康)、阮(籍)为核心,多数是文学家,是继建安七子之后在中国文学史上影响颇大的文人群体。但因其不与司马统治集团同流合污,所以鲁迅说他们“差不多都是反抗旧礼教的”。他们议论风发,诗文联翩,产生了极大的社会影响。竹林名士,“肆意酣畅”,开怀痛饮美酒,一方面刺激创作的兴奋点,另一方面也是借酒浇胸中块垒,麻痺自己的神经,醉酒胡话,当权者又能怎样?但统治者不这么想,他们不管你诗文作的多好,思想多么精深,只要不随顺不合作,就是讽刺与诋毁,都在该打该杀之列。于是先逮个嵇康来杀鸡儆猴,用他的头颅来祭旗开刀,看看是你的笔尖利,还是我的刀子快!至于罪名则是莫须有,就像曹操杀孔融一样,理由莫明其妙。此后,竹林七贤发生分化,或谨慎缄口,或改换门庭。人是社会的人,面对头上的屠刀,这也不奇怪。不过,竹林七贤饮酒作达的任诞之风,并没有因统治者挥舞屠刀而消失,反而愈演愈烈,成了两晋名士作达的象征,身份的表现。名士痛饮美酒,比比皆是,甚至趋于极端荒诞,这也是人们所始料不及的。}

{\cangkai\zihao{5}这则故事并非具体描写,而只是\CJKunderwave{任诞}门54则故事的“序言”或开篇,好戏尚在后面。}

\lettrine{23.2} 阮籍\myidx{阮籍}遭母丧,在晋文王\myidx{司马炎}坐\footnote{晋文王:司马炎篡魏开晋后,追尊其父司马昭为文帝,史或称文王。坐:通“座”。},进酒肉。司隶何曾\myidx{何曾}亦在坐\footnote{何曾:司马集团重要腹心大臣,时任司隶校尉,掌管察举百官及京师治安。},{\fzxk\zihao{6}\textcolor{red}{\CJKunderwave{晋诸公赞}曰:“何曾,字颖考,陈郡阳夏人。父夔,魏太仆。曾以高雅称,加性仁孝。累迁司隶校尉,用心甚正,朝廷惮之。仕晋至太宰。”}} 曰:“明公方以孝治天下,而阮籍以重丧显于公坐饮酒食肉,宜㳅(流)之海外\footnote{㳅 :即流,流放。海外:四海之外,泛指边疆不毛之地。},以正风教\footnote{正:端正。风教:风俗教化。}。”文王曰:“嗣宗毁顿如此\footnote{嗣宗:阮籍字。毁顿:哀伤困顿而伤害身体健康。},君不能共忧之,何谓?且有疾而饮酒食肉,固丧礼也\footnote{且有疾而饮酒食肉,固丧礼也:按\CJKunderwave{礼记·曲礼}记载,“居丧之礼,……有疾则饮酒食肉,疾止复初”。}。”籍饮啖不辍\footnote{啖(dàn淡):吃。辍(chuò绰):停止。},神色自若。{\fzxk\zihao{6}\textcolor{red}{干宝\CJKunderwave{晋纪}曰:“何曾尝谓阮籍曰:‘卿恣情任性,败俗之人也。今忠贤执政,综核名实,若卿之徒,何可长也!’复言之于太祖,籍饮啖不辍。故魏、晋之间,有被发夷傲之事,背死忘生之人,反谓行礼者,籍为之也。”\CJKunderwave{魏氏春秋}曰:“籍性至孝,居丧,虽不率常礼,而毁几灭性。然为文俗之士何曾等深所雠疾。大将军司马昭爱其通伟,而不加害也。”}}

{\cangkai\zihao{5}【评】魏晋时篡弑相继,故统治者讳言“忠”。但是,一个国家政权总要有点道德支撑,因此,改为“以孝治天下”。阮籍诸人何等冰雪聪明,把这一政治把戏看得非常透彻。他在母丧期间,和世俗不一样,并不严守丧礼条文,只要身体需要,照样在公开场合“进酒肉”。司隶校尉何曾要把他流放海外,正是严惩不合作者,甚至是持不同政见者的一种残酷手段,目的无非是敲山震虎,传达政治信号。而司马昭对阮籍的宽容,似乎多了点人性,其实是他对阮籍并未全然失望,尚列于争取的名单。在夺权斗争中,争取高门士人的支持非常重要。因此,阮籍行为再怪,他也可以见怪不怪,对“孝”的解释权全在统治者的口中。故事中的主角阮籍高唱怪诞之调,若没有司马昭和何曾二人红脸白脸政治双簧的衬托,也不会那么生动形象。}

\lettrine{23.3} 刘伶\myidx{刘伶}病酒\footnote{刘伶:字伯伦,见前注。病酒:因饮酒过量而身体不适。},渴甚,从妇求酒。妇捐酒毁器\footnote{捐:丢弃,倒掉。},涕泣谏曰:“君饮太过,非摄生之道\footnote{摄生:养生。},必宜断之!”伶曰:“甚善,我不能自禁,唯当祝鬼神自誓断之耳\footnote{祝:祈祷,祷告。}。便可具酒肉\footnote{具:准备,备办。}。”妇曰:“敬闻命。”供酒肉于神前,请伶祝誓。伶跪而祝曰:“天生刘伶,以酒为名\footnote{以酒为名:视饮酒为生命。名,通“命”。},一饮一斛\footnote{斛:古代量器名,一斛十斗。},五斗解酲\footnote{解酲(chénɡ呈):醒酒。酲,醉酒失态。}。{\fzxk\zihao{6}\textcolor{red}{毛公注曰:“酒病曰酲。”}} 妇人之言,慎不可听。”便引酒进肉,隗然已醉矣\footnote{隗(wěi伟):醉酒之状。}。{\fzxk\zihao{6}\textcolor{red}{见\CJKunderwave{竹林七贤论}。}}

{\cangkai\zihao{5}【评】故事虽短,却是波澜曲折而见变化之妙。细节描写具有典型性,一个生动的清醒醉汉形象,跃动在字里行间,可说是古代一篇神形兼备的小小说,令人百读不厌。刘伶祝神“断”酒之辞,纯为古代四言铭颂之调,合于典雅之体,渴酒之际,冲口而出,又见竹林酒仙的修养和急智。\CJKunderwave{晋书·刘伶传}称伶“常乘鹿车,携一壶酒,使人荷锸而随之,谓曰:‘死便埋我。’其遗形骸如此。”刘伶“以酒为名(命)”,正是一种看透社会人生虚伪礼教的一种清醒认识和满腔无奈。社会腐败黑暗,个人无力回天,不醉酒又将如何!}

\lettrine{23.4} 刘公荣\myidx{刘昶}与人饮酒\footnote{刘公荣:刘昶,字公荣。官至兖州刺史。性通达,有知人之名,曾与阮籍、王戎等相友善。},杂秽非类\footnote{杂秽:杂乱鄙秽。非类:指身份、教养非同一层次之人。}。人或讥之,答曰:“胜公荣者,不可不与饮;不如公荣者,亦不可不与饮;是公荣辈者,又不可不与饮。”故终日共饮而醉。{\fzxk\zihao{6}\textcolor{red}{\CJKunderwave{刘氏谱}曰:“昶字公荣,沛国人。”\CJKunderwave{晋阳秋}曰:“昶为人通达,仕至兖州刺史。”}}

{\cangkai\zihao{5}【评】刘昶妙语解颐。嗜酒之人,不管对象,岂问理由?刘昶饮酒,以自己为坐标来衡量。“胜公荣者”、“不如公荣者”、“是公荣辈者”,以逻辑外延划分,则包涵了所有的人。但如实话实说,称可与一切人共饮,则语直乏味。只有如此分类比较,方才说来有味。刘昶嗜酒而不糊涂,酒性无妨其知人之称,这说明他是颇知酒趣的高人,其酒中妙语,已浮现其清醒的智慧,所以他的官做到了内史、刺史之类的方面大员,并不奇怪。}

\lettrine{23.5} 步兵校尉缺\footnote{步兵校尉:官名,汉置五校尉之一,魏晋沿之,领宿卫营兵,秩比二千石。},厨中有贮酒数百斛\footnote{厨:厨房。贮:储藏。},阮籍\myidx{阮籍}乃求为步兵校尉。{\fzxk\zihao{6}\textcolor{red}{\CJKunderwave{文士传}曰:“籍放诞有傲世情,不乐仕宦。晋文帝亲爱籍,恒与谈戏,任其所欲,不迫以职事。籍常从容曰:‘平生曾游东平,乐其土风,愿得为东平太守。’文帝说,从其意。籍便骑驴径到郡,皆坏府舍诸壁障,使内外相望,然后教令清宁,十馀日,便复骑驴去。后闻步兵厨中有酒三百石,忻然求为校尉。于是入府舍,与刘伶酣饮。”\CJKunderwave{竹林七贤论}又云:“籍与伶共饮步兵厨中,并醉而死。”此好事者为之言。籍景元中卒,而刘伶太始中犹在。}}

{\cangkai\zihao{5}【评】阮籍任性而行,求官为酒,这与时人跑官、求官、买官而为权钱交易,大相径庭,因而人以为怪诞。当时司马昭为争取士人对其统治的支持,暂时还把阮籍这个名人摆放在“可以改造”者的行列。须知,不仅因为阮籍有学问,诗歌文章也写得好,个人社会影响大,而且还因为他的族望和门第,他有个名声不小的好爸爸——建安七子之一的陈留阮瑀。在魏晋推行九品中正制度的门阀社会里,作为陈留阮氏家族的子孙,本身就是一件可以傲视世人的资本。司马昭放宽条件,拉拢阮籍,正是为争取更多士人的支持拥护以稳固统治计,对于阮籍,司马昭表面宽容大量,其实内里包藏了另一番心计。}

\lettrine{23.6} 刘伶\myidx{刘伶}尝纵酒放达\footnote{放达:不拘礼法的放纵行为。},或脱衣裸形在屋中。人见讥之,伶曰:“我以天地为栋宇\footnote{栋宇:泛指房屋。},屋室为裈衣\footnote{裈 (kūn昆):裤子。},诸君何为入我裈中?”{\fzxk\zihao{6}\textcolor{red}{邓粲\CJKunderwave{晋纪}曰:“客有诣伶,值其裸袒,伶笑曰:‘吾以天地为宅舍,以屋宇为裈衣,诸君自不当入我裈中,又何恶乎?’其自任若是。”}}

{\cangkai\zihao{5}【评】率情任性,以适意为准的,是刘伶张扬自我个性的表现。但为反对礼法,赤身裸体而一丝不挂,虽然矫枉过正,但在当时也属惊世骇俗之举。不过,伶在己家“脱衣裸形”,而非公众场合作秀,这与今日世俗歌儿舞女或“行为艺术家”的裸体表演,旨趣大异。}

\lettrine{23.7} 阮籍\myidx{阮籍}嫂尝还家\footnote{还家:已婚女子回娘家。},籍见与别。或讥之,{\fzxk\zihao{6}\textcolor{red}{\CJKunderwave{曲礼}:“嫂叔不通问。”故讥之。}}籍曰:“礼岂为我辈设也\footnote{礼:礼法制度。}!”

{\cangkai\zihao{5}【评】按\CJKunderwave{曲礼}云:“嫂叔不通问。”郑玄注:“皆为重别防淫乱。”据古礼,籍因嫂归宁而“见与之别”,不合礼法规范,以此遭讥俗世。但以今视之,此一游戏规则,实属不通人情。一门之内,尚且防患;社会上数不清的男男女女,又将如何相见呢?须知,世界除了男人,妇女是半爿天,男女不相见,彼此如防盗贼一般,又怎能在一个社会中有正常的共同生活呢?关于这一问题,战国时孟子早就提出了“嫂溺则援之以手乎”的问题,孟子断言:“嫂溺不援,是豺狼也。”(\CJKunderwave{孟子·离娄上})即便是实行礼法,也还有从权变通的时候。更何况魏晋名教中人,表面上标榜礼法,内心却男盗女娼者比比皆是。籍于其\CJKunderwave{大人先生传}中予以无情嘲讽:“汝君子之礼法,诚天下残贼、乱危、死亡之术耳,而乃目之为不易之道,不亦过乎!”因此,针对礼法之虚伪,籍发出“礼岂为我辈设也”的批判反诘,成为千古名言。话中用“我辈”,而非一介之“我”,说明他是代表了一大批愤世嫉俗之士在发言,而非仅是个人的恩怨和认识。对于世俗之讥,嗣宗怎会买账?其回答掷地锵然作响,对阴一套阳一套的虚伪礼教极其不屑。以此,礼法之士嫉之如仇,恨不得拿他祭旗开刀。}

{\cangkai\zihao{5}当时有关礼法之争,鲁迅深刻指出其问题实质,说:“魏晋时代,崇奉礼教的看来很不错,而实在是毁坏礼教,不信礼教的。表面上毁坏礼教者,实则倒是……太相信礼教。”(\CJKunderwave{魏晋风度及文章与药及酒之关系})嵇、阮之辈,因相信而爱之太切,一旦发现受骗上当,美好理想轰然崩溃,于是就产生了“任诞”不拘礼法的反弹,这也是出于自然。}

\lettrine{23.8} 阮公\myidx{阮籍}邻家妇有美色\footnote{阮公:即阮籍。},当垆酤酒\footnote{当垆酤酒:在酒店卖酒。垆,置酒瓮的土台,指代酒店。酤(ɡū姑),卖。}。阮与王安丰\myidx{王戎}常从妇饮酒\footnote{王安丰:指王戎,入晋封安丰侯,故称。},阮醉,便眠其妇侧。夫始殊疑之\footnote{殊疑:非常怀疑。},伺察\footnote{伺察:暗中观察。},终无他意\footnote{他意:此指不良意图。}。{\fzxk\zihao{6}\textcolor{red}{王隐\CJKunderwave{晋书}曰:“籍邻家处子有才色,未嫁而卒。籍与无亲,生不相识,往哭,尽哀而去。其达而无检,皆此类也。”}}

{\cangkai\zihao{5}【评】世人昏昏我独醒。在男女关系问题上,籍心怀坦荡,只有人皆有之的爱美之心,而无一星半点的狎邪之念。因此,他醉卧卖酒女郎之侧,鼾然而眠,心安理得,内外合一,而非作态摆样。刘注所称为邻家美少女哭丧尽哀,亦当作如是观。如今日大街之上,时装美女作模特儿展览,人们多看几眼,有何不可?但王隐\CJKunderwave{晋书}却因此批评阮籍“达而无检”,在今人眼中,反而成为世俗迂夫的奇谈怪论。}

\lettrine{23.9} 阮籍\myidx{阮籍}当葬母\footnote{当:将要。},蒸一肥豚\footnote{肥豚:肥美小猪。},饮酒二斗,然后临诀\footnote{临诀:去作告别。},直言“穷矣\footnote{直言:只说。直,通“只”。穷矣:完了,绝望了。按唐长孺说,魏晋时“孝子唤奈何、唤穷,疑为洛阳及其附近风俗,盖父母之丧,孝子循例要唤‘穷’也”(见其\CJKunderwave{魏晋南北朝史论丛})。}!”都得一号\footnote{都:只。一号:大哭一声。},因吐血,废顿良久\footnote{废顿:昏迷。}。{\fzxk\zihao{6}\textcolor{red}{邓粲\CJKunderwave{晋纪}曰:“籍母将死,与人围棋如故,对者求止,籍不肯,留与决赌。既而饮酒三斗,举声一号,呕血数升,废顿久之。”}}

{\cangkai\zihao{5}【评】葬母时饮酒食肉,违反礼法,故名教之士讥贬为“天地不容”的禽兽之行,其“罪”至大,怎能见容于“以孝治国”的魏晋社会呢?但后人为贤者讳,如清李慈铭以为“妄诬先达”,必无此事。但余嘉锡反之,以为李氏“空言翻案,吾所不取”。事实如何,待考。但综合阮籍一生言行,不顾名教如此,也在可以理解之列。孝与不孝,不在饮酒食肉与否,而在于心中是否有真感情真悲痛。嗣宗之饮酒食肉,是否因身体健康需要?抑或是故意否定礼法的率性之举?鄙见二者兼有。\CJKunderwave{太平御览}卷三七五\CJKunderwave{血}门称:“阮步兵居丧不率礼,而志孝称。”居丧失礼,而不改其“志孝”之心。其临诀的一声大哭,呕血昏厥,岂是礼法之士假装得来?籍之真性情,于其任诞之行见其一斑。}

\lettrine{23.10} 阮仲容\myidx{阮咸}、{\fzxk\zihao{6}\textcolor{red}{咸也。}} 步兵\myidx{阮籍}居道南\footnote{阮仲容:阮咸,字仲容,籍兄之子。步兵:指阮籍。},诸阮居道北。北阮皆富,南阮贫。七月七日,北阮盛晒衣\footnote{七月七日晒衣:古时习俗,\CJKunderwave{太平御览}卷三一引\CJKunderwave{韦氏月录}曰:“七月七日晒曝革裘,无虫。”},皆纱罗锦绮\footnote{纱罗锦绮:指锦缎香罗一类高级丝织品衣物。}。仲容以竿挂大布犊鼻裈于中庭\footnote{大布:粗布。犊鼻裈:形如犊鼻的围裙。}。人或怪之,答曰:“未能免俗,聊复尔耳\footnote{聊复尔耳:姑且如此。}。”{\fzxk\zihao{6}\textcolor{red}{\CJKunderwave{竹林七贤论}曰:“诸阮前世皆儒学,善居室,唯咸一家尚道弃事,好酒而贫。旧俗:七月七日法当晒衣。诸阮庭中,烂然锦绮。咸时总角,乃竖长竿挂犊鼻裈也。”}}

{\cangkai\zihao{5}【评】贫富悬殊,自古而然,虽是士族子孙,也不免分化升沉。魏晋时世家豪族生活侈靡,夸富斗贵之事,载于史册。如晋初国舅王恺与石崇斗富的故事,见本书\CJKunderwave{汰侈}门。但与世俗夸强斗富的传统心理相反,阮咸则不怕穷,在竹竿上高挂粗布围裙,而无惧于北阮富豪的“纱罗锦绮”之比。在时人眼中,穷是羞贱者,躲都来不及,为什么还要公开暴露自己的穷酸相呢?这不是乖戾任诞又是什么?但阮咸不这么看。人生价值,不是物质财富所能衡量,自己虽然穷得衣衫不整、财产匮乏,但肚里的满腹经纶和才华,不正是用金钱也买不到的无穷精神财富吗?人生价值,主要在神完气旺,心理健康,而非物质上富得流油的精神乞丐。阮咸挂大布犊鼻裈,是示威,是挑战,何陋之有?其“任诞”也是一种天性之自然。}

\lettrine{23.11} 阮步兵\myidx{阮籍}{\fzxk\zihao{6}\textcolor{red}{籍也。}}丧母,裴令公\myidx{裴楷}{\fzxk\zihao{6}\textcolor{red}{楷也。}}往吊之\footnote{裴令公:指裴楷,字叔则,河东闻喜人,曾官中书令,故称。参前\CJKunderwave{德行}第18则注。}。阮方醉\footnote{方:刚。},散发坐床\footnote{床:古代坐具。},箕踞不哭\footnote{箕踞:屁股坐地,两脚前伸,状如簸箕。在古代,这是一种傲慢无礼的表现。}。裴至,下席于地\footnote{下席:走下坐榻。},哭,吊喭毕便去\footnote{吊喭:吊唁。喭,通“唁”(yàn谚)。}。或问裴:“凡吊,主人哭,客乃为礼\footnote{为礼:行礼。}。阮既不哭,君何为哭?”裴曰:“阮方外之人\footnote{方外之人:超脱礼教世俗之人。},故不崇礼制;我辈俗中人,故以仪轨自居\footnote{仪轨:礼仪规则。自居:自守,自处。}。”时人叹为两得其中\footnote{中:事理合适得当谓之中。}。{\fzxk\zihao{6}\textcolor{red}{\CJKunderwave{名士传}曰:“阮籍丧亲,不率常礼。裴楷往吊之,遇籍方醉,散发箕踞,傍若无人。楷哭泣尽哀而退,了无异色。其安同异如此。”戴逵论之曰:“若裴公之致吊,欲冥外以护内,有达意也,有弘防也。”}}

{\cangkai\zihao{5}【评】阮籍丧亲之痛,在内不在外,其守制之时,照样醉酒而箕踞不哭,不足为怪,已如前述。礼法之士无法容忍,故指责裴楷之吊,有“阮既不哭,君何为哭”之问。但裴楷是信奉老庄之道的玄学家,不以儒家名教来衡量,因而顺其信奉自然的思想,给予阮籍的行为以同情之理解。他的回答,简明有趣,妥帖得当,成为名言。故事中无论是阮籍或裴楷,无不形象生动,神采飞扬。}

\lettrine{23.12} 诸阮皆能饮酒\footnote{诸阮:指陈留阮氏家族诸人。},仲容至宗人间共集\footnote{宗人:同族人。集:聚会。},不复用常杯斟酌,以大瓮盛酒,围坐相向大酌\footnote{相向:面对面。大酌:大口喝酒。}。时有群猪来饮,直接去上\footnote{直接去上:\CJKunderwave{晋书·阮咸传}称引作“咸直接去其上”,可另备一说。},便共饮之。

{\cangkai\zihao{5}【评】因嫉视名教,不拘礼法,而“解放”到人、猪共饮的程度,实是矫枉过正的过激行为,称为乖戾荒诞,并不为过。当时乐广曾批评说:“名教中自有乐地,何为乃尔也!”(见\CJKunderwave{德行}第23则)乐广是个儒、玄双修的名家,代表了当时的某种社会舆论。人猪共饮,不仅不卫生,容易得传染病,而且也使张扬个性的自我,降低到低层次的动物世界中去,宇宙自然中天、地、人三才的核心是人,早已退化消失在人兽不分的糊涂世界中了。}

\lettrine{23.13} 阮浑\myidx{阮浑}长成\footnote{阮浑:字长成,籍子,为人清虚寡欲,器量弘旷。太康中,官太子中庶子。参\CJKunderwave{赏誉}第29则注。},风气韵度似父\myidx{阮籍}\footnote{风气韵度:风度气质等内在精神面貌。},亦欲作达\footnote{作达:放达不拘礼法。}。步兵曰\footnote{步兵:指阮籍。}:“仲容\myidx{阮咸}已预之\footnote{预之:参加。},卿不得复尔!”{\fzxk\zihao{6}\textcolor{red}{\CJKunderwave{竹林七贤论}曰:“籍之抑浑,盖以浑未识己之所以为达也。后咸兄子简,亦以旷达自居。父丧,行遇大雪,寒冻,遂诣浚仪令。令为他宾设黍臛,简食之,以致清议,废顿几三十年。是时竹林诸贤之风虽高,而礼教尚峻。迨元康中,遂至放荡越礼。乐广讥之曰:‘名教中自有乐地,何至于此!’乐令之言,有旨哉!谓彼非玄心,徒利其纵恣而已。”}}

{\cangkai\zihao{5}【评】故事的第一主角是阮籍,而非其子浑。嗣宗之言,寓意精深,值得人们玩味。父亲不希望儿子学自己,是不是否定自己所走的作达之路呢?当然不是。他曾说:“礼岂为我辈设也?”表现了对儒家虚伪礼教的蔑视和挑战,而没有丝毫悔过自新的意思。但是,其饮酒作达,是被逼出来的,并非其初衷。据史称,“籍本有济世志,属魏晋之际,天下多故,名士少有全者,籍由是不与世事,遂酣饮为常”。其酣醉狂饮之作达,实际上一方面是抒泄愤懑,更重要的是出于无奈的自我政治保护。生于天下多故之秋,如果太清醒,一方面会增加痛苦,另一方面将可能死无葬身之地。而长醉不醒,才不会对统治者的权势构成威胁。但是,如此方式的活着,太辛苦,太委屈,太痛苦。而且,没有本事也学不来。阮籍希望儿子不要再走自己的痛苦之路,能堂堂正正地在阳光下生活,其期望寄托在历史的变化之中。但是,他的儿子也没能等到这一天,悲哉!}

\lettrine{23.14} 裴成公\myidx{裴顗}妇\footnote{裴成公:即裴顗,字逸民,河东闻喜人,官至侍中、尚书左仆射。卒后谥成。故称。参前\CJKunderwave{言语}第23则注。},王戎女\footnote{王戎:字濬冲,琅邪人。}。王戎\myidx{王戎}晨往裴许\footnote{许:处所,住地。},不通径前\footnote{不通:没经通报。径前:直接进来相见。}。裴从床南下,女从北下,相对作宾主,了无异色\footnote{了无异色:神态自如,没有一点难为情的样子。}。{\fzxk\zihao{6}\textcolor{red}{\CJKunderwave{裴氏家传}曰:“顗取戎长女。”}}

{\cangkai\zihao{5}【评】如前所述,魏晋时礼教清议尚峻,一家之内,叔嫂不通问,见面则有违传统礼法。但作为竹林七贤之一的王戎,则不拘礼法,连起码的招呼也不打,清晨径进女婿家,打扰小夫妇的清梦。幸亏裴顗也是个思想较为开通的清谈名家,不以为忤,因而宾主相对才能“了无异色”。玄学家的生活行为及其心理状态的确与传统礼教异其旨趣。细加分辨,有助于对魏晋士人内心世界的了解。}

\lettrine{23.15} 阮仲容\myidx{阮咸}先幸姑家鲜卑婢\footnote{阮仲容:阮咸。幸:宠幸。鲜卑:中国古代少数民族东胡族的一支。}。及居母丧,姑当远移,初云当留婢\footnote{当:将要。},既发,定将去\footnote{定:终究,到底。按:“定”沈校本作“迺”,亦通。但以魏晋用语习惯,作“定”为是。将去:带走。}。仲容借客驴,箸重服\footnote{驴:\CJKunderwave{晋书}咸传作“马”,驴难兼载,马可“累骑”,于义为胜。重服:最严重的丧服,父母丧时所穿。},自追之,累骑而返\footnote{累骑:两人并乘一骑。按\CJKunderwave{资治通鉴·魏纪}注:“累,重也,两人共马,谓之累骑。”考虑到驴体小力弱,兼乘为难,而骡马之类,则“累骑”不难。},曰:“人种不可失\footnote{人种:孕妇腹中婴儿。}!”即遥集\myidx{阮孚}之母也\footnote{遥集:阮孚字。孚,咸之次子。见前\CJKunderwave{文学}第76则注。}。{\fzxk\zihao{6}\textcolor{red}{\CJKunderwave{竹林七贤论}曰:“咸既追婢,于是世议纷然。自魏末沈沦闾巷,逮晋咸宁中始登王途。”\CJKunderwave{阮孚别传}曰:“咸与姑书曰:‘胡婢遂生胡儿。’姑答书曰:‘\CJKunderwave{鲁灵光殿赋}曰:“胡人遥集于上楹。”可字曰遥集也。’故孚字遥集。”}}

{\cangkai\zihao{5}【评】遥集是阮孚字,即咸与胡婢所生之子。孚在东晋历官侍中、吏部尚书、丹阳尹。其人智商颇高,妙悟玄理,精赏文学,颇富阮氏家学风流。在他身上,奔流着一半“胡”族血统,同样为中华民族输入了一股鲜活的血液。这是题外话。此则故事犹如纪实的小小说,情节紧张,故事曲折,非常生动。作为竹林七贤之一的阮咸,风流不减乃叔。作为故事的主人公,其言行常是不假思索的潜意识爆发,即开罪名教礼法也不顾。因其追婢事,俗议纷然,他终于为自己的感情生活,付出了沉重的代价——几十年沉沦闾巷,被禁锢于仕途之外。但阮咸不为礼法禁锢而低下那高傲的头颅。借客驴(马)追婢,“累骑而返”,公然招摇过市,这不是向传统礼法的又一次挑战又是什么!怪诞中见其精神风流。}

\lettrine{23.16} 任恺\myidx{任恺}既失权势\footnote{任恺(约220—约280):字元裒,乐安博昌(今山东博兴南)人。仕魏,尚明帝女,入晋历侍中、吏部尚书。有经国才干。因贾充谗毁而夺官,郁郁以终。},不复自检括\footnote{检括:约束检点。}。或谓和峤\myidx{和峤}曰\footnote{和峤(?—292):字长舆。汝南西平(今属河南)人。官至中书令、太子少傅。参前\CJKunderwave{方正}第9则注。}:“卿何以坐视元裒败而不救?”和曰:“元裒如北夏门\footnote{北夏门:洛阳城门之一,又称大夏门,在城北。},拉攞自欲坏\footnote{拉攞(luǒ裸):同义复合词,断裂。拉,拉折;攞,撕裂。},非一木所能支。”{\fzxk\zihao{6}\textcolor{red}{\CJKunderwave{晋诸公赞}曰:“恺字元裒,乐安博昌人。有雅识国干,万机大小多综之。与贾充不平,充乃启恺掌吏部,又使有司奏恺用御食器,坐免官。世祖情遂薄焉。”}}

{\cangkai\zihao{5}【评】故事虽小,却牵涉西晋朝廷中朋党之争的大事。史称任恺有经国之才,“性忠正,以社稷为己任”,与庾纯、张华、温颙、何秀、和峤之徒友善,而恶权倖贾充之为人。而贾充则与荀勖、冯紞承间浸润谗毁任恺。贾、任二党几经争斗,以任恺失败告终。政治失败之后,于是任恺“不复自检括”,“纵酒耽乐”,甚至是一食万钱,“犹云无可下箸处”。可见任恺本身的言行,也为政敌提供了炮弹。其事入于“任诞”,亦是持之有故。在朋党斗争中,和峤与任恺友善,本属同党,他极鄙贾充、荀勖之流。在任恺失势之时,本应谏争相救,但在义与利的方面加以衡量,终于舍义趋利,以钳口不言来拯救自己。“拉攞自欲坏,非一木所能支”,不过是怕引发自身政治危机的借口而已。在严酷的政治斗争中,舍义取利,这才是真正的荒诞之事。}

\lettrine{23.17}刘道真\myidx{刘宝}少时\footnote{刘道真:刘宝,字道真。高平人,见前\CJKunderwave{德行}第22则注。},常渔草泽\footnote{渔:捕鱼。草泽:杂草丛生的荒野水边。},善歌啸\footnote{歌啸:又称为“啸”或“啸咏”,以口哨吹奏歌吟。参前\CJKunderwave{栖逸}第1则注。},闻者莫不留连\footnote{留连:舍不得离开。}。有一老妪\footnote{老妪:老妇人。},识其非常人,甚乐其歌啸,乃杀豚进之\footnote{豚:小猪。}。道真食豚尽,了不谢\footnote{了不谢:一点也没感谢。了,完全。}。妪见不饱,又进一豚。食半馀半,乃还之。后为吏部郎\footnote{吏部郎:吏部的郎官。},妪儿为小令史\footnote{小令史:掌文书小吏。},道真超用之\footnote{超用:越次提拔重用。}。不知所由,问母,母告之。于是赍牛酒诣道真\footnote{赍(jī鸡):携带。诣:拜访。},道真曰:“去,去!无可复用相报。”{\fzxk\zihao{6}\textcolor{red}{刘宝,已见。}}

{\cangkai\zihao{5}【评】魏晋士人,在玄学风潮鼓动下,许多人性格随顺自然,只求内心的适意,而不在乎表面的应酬客套,此世俗所以称之为怪诞也。刘宝虽然年少贫寒,但后来一跃为中原士人之望,其品评人物,与王衍齐名,所以陆机初入洛时,要先去看望刘宝,以求推扬。刘宝虽亦通经之士,但儒、玄双修,慕竹林之遗风,任达放诞自居,其行事外表虽诞,但内心明白坦荡。食妪之豚而“了不谢”,甚至是“食半馀半”——把吃剩下的肉还给主人,即在今日交际,也属不礼貌的荒唐之举,但老妪却毫不在意,宾主二人彼此无言,这实是一场配合默契的内在真心的交流。刘宝发达后,超拔妪儿小令史,正是一种从心底自然流出的真诚举动,是日久积聚于胸的潜意识的爆发,纯是超功利的回报。因此,当小令史循俗赍牛酒拜谢时,他会连声呵斥,“去,去!无可复用相报”,活脱地画出了主人公超凡脱俗精神境界之可爱。}

\lettrine{23.18} 阮宣子\myidx{阮修}常步行\footnote{阮宣子:阮修字宣子,陈留尉氏人。阮咸族子。见前\CJKunderwave{文学}第18则注。\CJKunderwave{晋书}附于\CJKunderwave{阮籍传}后。},以百钱挂杖头\footnote{杖:手杖。},至酒店,便独酣畅\footnote{酣畅:开怀痛饮。},虽当世贵盛\footnote{贵盛:豪门显贵。},不肯诣也\footnote{诣:拜访。}。{\fzxk\zihao{6}\textcolor{red}{\CJKunderwave{名士传}曰:“修性简任。”}}

{\cangkai\zihao{5}【评】阮修出于陈留阮氏家族,承籍、咸先辈竹林遗风,成为当时著名玄家,“好\CJKunderwave{易}、\CJKunderwave{老},善清言”,大概是受家学的熏染。史称其“性任简,不修人事。绝不善见俗人,遇便舍去。意有所思,率尔褰裳,不避晨夕,至或无言,但欣然相对”。其行为可做此则故事的注脚。不诣贵盛,正见其笑傲世俗品格之高洁,未受污染,何诞之有!}

\lettrine{23.19} 山季伦\myidx{山简}为荆州\footnote{山季伦:山简,字季伦。父涛。见前\CJKunderwave{赏誉}第29则注。为荆州:担任荆州刺史。},时出酣畅,人为之歌曰:“山公时一醉\footnote{山公:指山简。时:不时,经常。},径造高阳池\footnote{径造:径直造访。高阳池:池名,在襄阳。原为汉侍中习郁所修鱼池。山简则意指酒池,因高阳在古代是酒徒代名词。}。日莫倒载归\footnote{莫:通“暮”。},茗艼无所知\footnote{茗艼:酩酊,大醉之态。}。复能乘骏马,倒箸白接篱\footnote{接篱:一种帽子。或作睫䍦。}。举手问葛强\myidx{葛强},何如并州儿\footnote{并州:在今山西一带。}?”高阳池在襄阳,强是其爱将,并州人也。{\fzxk\zihao{6}\textcolor{red}{\CJKunderwave{襄阳记}曰:“汉侍中习郁,于岘山南,依范蠡养鱼法作鱼池。池边有高隄,种竹及长楸,芙蓉、菱芡覆水,是游燕名处也。山简每临此池,未尝不大醉而还,曰:‘此是我高阳池也。’襄阳小儿歌之。”}}

{\cangkai\zihao{5}【评】\CJKunderwave{任诞}之门,多述酒事。中国文化传统,有属贵族精英的文化,称大传统;又有属下层民间的俗文化,称小传统。二者虽有雅、俗之分,但相互影响渗透,而各有其文化价值。荆州民歌写山简醉酒,形象栩栩如画而呼之欲出。“倒载”、“茗艼”、“倒箸白接篱”,无一不见酣醉似眠、憨态可掬。但结尾“何如并州儿”二句,则醉中也偶有清醒之时。古时北方幽、并之地,地逼边疆,为民族杂居之境,健儿多习鞍马骑射,葛强善骑,可以想象。但简醉中,却敢向他“挑战”,戏问吾之骑技,与尔等并州健儿相较如何?其问豪爽,见真精神。}

\lettrine{23.20} 张季鹰\myidx{张翰}纵任不拘\footnote{张季鹰:张翰,字季鹰。见前\CJKunderwave{识鉴}第10则注。纵任:放纵性情。},时人号为“江东步兵\footnote{江东步兵:江东的阮籍。步兵,指阮籍。}”。或谓之曰:“卿乃可纵适一时\footnote{纵适:放纵适意。},不为身后名邪\footnote{不为身后名邪:袁本作“独不为身后名邪”,亦通。身后,死后。}?”答曰:“使我有身后名,不如即时一杯酒\footnote{即时:当下,眼前,现在。}。”{\fzxk\zihao{6}\textcolor{red}{\CJKunderwave{文士传}曰:“翰任性自适,无求当世,时人贵其旷达。”}}

{\cangkai\zihao{5}【评】张翰个性,深受竹林七贤遗风的影响,其愤世嫉俗,见几避祸之预见性,堪称阮籍的好学生,故人称其“江东步兵”,他也以此为荣。他与顾荣及陆机、陆云兄弟,于吴亡后进入京师洛阳,原为事业理想而来,也即人们所称的“求名”,他也曾做到齐王冏的东曹掾。但他知几见微,早已预料天下将乱,故在八王乱前对同郡顾荣等说:“天下纷纷未已,夫有四海之名者,求退良难。吾本山林间人,无望于时久矣。子善以明防前,以智虑后。”(见\CJKunderwave{识鉴}第10则刘注)辞归江南而终。他不是不要“名”,不是缺乏理想,而是预见大乱将起,避之惟恐不及。其纵适饮酒,透过任诞之表,正见其人生智慧之光。王世懋评曰:“季鹰此意甚远,欲破世间啖名客耳。渠亦那能尽忘!”洞见深微之言。痛饮美酒而淡泊功名,实在也是一种理想破灭后的悲哀与无奈。}

\lettrine{23.21} 毕茂世\myidx{毕卓}云\footnote{毕茂世:毕卓,字茂世。两晋间新蔡鲖阳(今河南新蔡东北)人。少希放达,常饮酒废职。过江后卒于平南长史任上。}:“一手持蟹螯\footnote{蟹螯:螃蟹的第一对似钳大脚,味甘美。},一手持酒杯,拍浮酒池中\footnote{拍浮:游泳。},便足了一生。”{\fzxk\zihao{6}\textcolor{red}{\CJKunderwave{晋中兴书}曰:“毕卓字茂世,新蔡人。少傲达,为胡毋辅之所知。太兴末,为吏部郎,尝饮酒废职。比舍郎酿酒熟,卓因醉,夜至其瓮间取饮之。主者谓是盗,执而缚之,知为吏部也,释之。卓遂引主人讌瓮侧,取醉而去。温峤素知爱卓,请为平南长史,卒。”}}

{\cangkai\zihao{5}【评】嗜食蟹脚美味,古今皆有其人,今故美国总统尼克松即是一例。毕卓之异,不在手持蟹螯,而在于“拍浮酒池”以了残生,恨不得终日在酒池中濡首游泳,则非常人所能,以此见其张狂与荒唐。但生当两晋乱世,局势难以收拾,加以礼法严酷,令人见名教之虚伪。此时此刻,部分士人失望颓唐,也不足怪。古代礼法,呆板教条,压抑人性而黑暗冤屈者,不在少数。毕卓沉湎酒池,荡佚礼法、蔑视伦常,其惊世骇俗之举,除了消极颓唐之外,也还有值得深思的地方。毕卓应温峤辟而为平南将军长史,卒于任上。温峤是史上著名“死不忘忠”、勤瘁国事之能臣。若毕卓只是濡首酒池之醉鬼,温峤能给予信任吗?魏晋名士之诞,并非生活的全部。}

\lettrine{23.22} 贺司空\myidx{贺循}入洛赴命\footnote{贺司空:贺循,字彦先,会稽山阴(今浙江绍兴)人。卒赠司空,故称。参前\CJKunderwave{规箴}第13则注。洛:指西晋首都洛阳。赴命:应召为官。},为太孙舍人\footnote{太孙:晋惠帝永康元年,废愍怀太子,立其子为皇太孙。于是原东宫属官转为太孙属官。太孙舍人即由原太子舍人转化而来。},经吴昌门\footnote{吴昌门:即今苏州阊门,在城西。},在船中弹琴。张季鹰\myidx{张翰}本不相识\footnote{张季鹰:指张翰。},先在金昌亭\footnote{金昌亭:亭名,在阊门内。},闻弦甚清\footnote{弦甚清:琴声甚为清亮。},下船就贺,因共话,便大相知说\footnote{说:通“悦”。}。问贺:“卿欲何之?”贺曰:“入洛赴命,正尔进路。”张曰:“吾亦有事北京\footnote{北京:江南人称西晋洛阳为北京。},因路寄载\footnote{因路寄载:沿路搭船同行。}。”便与贺同发。初不告家\footnote{初不:全不。},家追问乃知。

{\cangkai\zihao{5}【评】张翰是可人,贺循也不赖。听琴赏音,共话知音,原本不相识的张贺二人,“大相知悦”。正因为有了思想情趣的共鸣,因此,不管北京洛阳远在数千里之遥,张翰“因路寄载”。其决定虽是一时萌发的奇想,随意性极强;但若论其心理及人格魅力,则宾主二人早有相见恨晚之叹,也属水到渠成之自然。宾求寄而主喜交,一个“愿打”,一个“愿挨”,与周瑜黄盖的故事是否有某些相似之处?}

\lettrine{23.23} 祖车骑\myidx{祖逖}过江时\footnote{祖车骑:指祖逖,字士雅,范阳人。卒赠车骑将军,故称。见前\CJKunderwave{赏誉}第43则注。},公私俭薄\footnote{俭薄:资财匮乏。},无好服玩\footnote{服玩:衣服珍玩。}。王\myidx{王导}、庾\myidx{庾亮}诸公共就祖\footnote{王、庾:指王导、庾亮等。},忽见裘袍重叠,珍饰盈列\footnote{珍饰盈列:满摆珍贵饰品。}。诸公怪问之,祖曰:“昨夜复南塘一出\footnote{南塘:地名,在建康秦淮河南。}。”祖于时恒自使健儿鼓行劫钞\footnote{鼓行劫钞:公开抢劫。钞,通“抄”。},在事之人,亦容而不问\footnote{在事之人:负责治安察举之人。}。{\fzxk\zihao{6}\textcolor{red}{\CJKunderwave{晋阳秋}曰:“逖性通济,不拘小节。又宾从多是桀黠勇士,逖待之皆如子弟。永声(嘉)中,流民以万数,扬士(土)大饥。宾客攻剽,逖辄拥护全卫。谈者以此少之,故久不得调。”}}

{\cangkai\zihao{5}【评】抢劫犯科,属于犯罪,祖逖岂能不知?故古人于其纵健儿行劫事,早有“未闻嵇(康)、阮(籍)作贼”之讥(王世懋评)。但故事置于永嘉之乱、中原丧败之后,又别有一解。作为胸怀大志的忠义之士祖逖,为复兴大计,不拘小节,招宾客义从暴桀之士,待如兄弟,从而建立起一支抗敌队伍。时京畿一带大饥,东晋朝廷虽任逖为奋威将军、豫州刺史,但仅是空衔而乏铠仗军实。试想,饿着肚子、缺乏武器装备的军队能打仗吗?所以尽管祖逖有闻鸡起舞、击楫中流的动人故事,让他领导这样一支饿着肚子、缺乏武器的军队,能有恢复中原的力量和希望吗?为部队的生存和发展,出此下策,实属无奈,也是朝廷逼出来的。史称祖逖击败石勒、收复河南之后,“躬自俭约,劝督农桑,克己务施,不畜资产,子弟耕耘,负担樵薪”,百姓无不感悦。于此可见,富贵贪赃,岂是祖逖本心!英雄义士也有犯错的时候,但评价人物应视其荦荦大节。}

\lettrine{23.24} 鸿胪卿孔群\myidx{孔群}好饮酒\footnote{孔群:字敬林,山阴(今浙江绍兴市)人。参前\CJKunderwave{方正}第36则注。鸿胪卿:官名,掌朝廷庆吊礼仪。},王丞相\myidx{王导}语云\footnote{王丞相:指王导。}:“卿何为恒饮酒\footnote{恒:经常,总是。}?不见酒家覆瓿布\footnote{瓿(bù部):瓮。},日月糜烂\footnote{日月:喻时间久长。}?”群曰:“不尔\footnote{不尔:不然,不是这样。}。不见糟肉,乃更堪久\footnote{堪久:耐久。}?”群尝书与亲旧:“今年由得七百斛秫米\footnote{由:袁本作“田”,是。斛:量器名,十斗一斛。秫米:高粱。},不了麴糵事\footnote{麴糵(qū niè曲孽):发酵用的酒母,喻酿酒。不了:不能解决。}。”{\fzxk\zihao{6}\textcolor{red}{群,已见上。}}

{\cangkai\zihao{5}【评】孔群官鸿胪卿,司掌国家庆吊祭祀。古时国家大事,惟祀与戎。故其居官,职责重大。祭祀必有酒礼。群嗜酒如命,多少与责任有关。以此,丞相王导怕他耽湎于酒而误国家大事,同时也为其身体,故有劝诫之言。而孔群则颇具独立人格精神,他不因上司批评而畏首畏尾,故有拒劝之论。不过,因彼此均属有身份有地位的人物,故无论劝或被劝二者皆以修辞比喻来作委婉生动的表述,于此见其精神风味与文学修养。孔群虽“好饮酒”而非酗酒。苏峻叛,其党徒匡术拔刃胁群,群不为所动。后峻平,匡术失势,因众坐行酒释憾,群当众峻拒之,斥谓“识者犹憎其面目”,其风节正直如此,何尝以酒乱事!事实说明,酒不醉人而人自醉,关键还在于人自己。}

\lettrine{23.25} 有人讥周仆射\myidx{周顗}与亲友言戏\footnote{周仆射:周顗,字伯仁,汝南名士。官至尚书仆射,故称。参前\CJKunderwave{言语}第30则注。言戏:说笑戏乐。},秽杂无检节\footnote{秽杂:污秽粗鄙。无检节:行为不检点。}。{\fzxk\zihao{6}\textcolor{red}{邓粲\CJKunderwave{晋纪}曰:“王导与周顗及朝士诣尚书纪瞻观伎,瞻有爱妾能为新声,顗于众中欲通其妾,露其丑秽,颜无怍色。有司奏免顗官,诏特原之。”}} 周曰:“吾若万里长江,何能不千里一曲\footnote{千里一曲:古称黄河九曲,此以黄河之曲来想象万里长江,千里必有一曲,喻人生小处之失。}!”

{\cangkai\zihao{5}【评】周顗为人,本门第28则称其“风德雅重”,\CJKunderwave{晋书}本传谓“虽时辈新狎,莫能媟也”,故人或疑其事之有无。如清李慈铭以为绝无此事,是其政敌“王敦、王导之徒,衔其强直,造此顗辞”。但余嘉锡则以为晋人蔑视礼法,放荡无检,习俗移人,贤者不免。周顗“好饮狂药,昏醉之后,亦复何所不至?固不可以一眚掩其大德,亦不必曲为之辩,以为必无此事也”。李、余正反二说,当以余说为是。葛洪\CJKunderwave{抱朴子·疾谬}谓当时“轻薄之人,……入他堂室,观人妇女,指玷修短,评论美丑”;沈约\CJKunderwave{宋书·五行志}亦称贵游子弟,“对弄婢妾”,不以为耻,可资佐证。周顗一代名士,情之所之,率性而行,不足为训,故王世懋讥评云:“达人先须去欲,周顗、谢鲲何乃以色为达。”但顗之可爱,在其万里长江,“千里一曲”之辩,既公开承认自己“一曲”是错误,同时又不以一曲小误而损己大节。修辞比喻,生动形象,而自持原则立场如故。观其面斥王敦之叛而视死如归,则无法怀疑此语态度之真诚。位重名高之人,敢于公开认错者,古往今来,又有几人!读此能无思乎?}

\lettrine{23.26} 温太真\myidx{温峤}位未高时\footnote{温太真:温峤,字太真,太原祁县(今属山西)人。东晋中兴名臣。参前\CJKunderwave{言语}第35则注。},屡与扬州、淮中估客樗蒱\footnote{扬州:指东晋建康京畿一带。淮中:淮河一带。估客:行商。樗蒱(chū pú出蒲):古代的一种流行赌博游戏。},与辄不竞\footnote{不竞:不胜,输掉。}。尝一过,大输物\footnote{一过:一局,一场。大输物:下大赌注。},戏屈\footnote{戏屈:赌输。},无因得反\footnote{反:通“返”,回还。}。与庾亮\myidx{庾亮}善,于舫中大唤亮曰\footnote{舫:有舱室的船。}:“卿可赎我!”庾即送直\footnote{直:通“值”。},然后得还。经此数四\footnote{数四:再三再四,喻其次数之多。}。{\fzxk\zihao{6}\textcolor{red}{\CJKunderwave{中兴书}曰:“峤有隽朗之目,而不拘细行。”}}

{\cangkai\zihao{5}【评】魏晋名士,不拘细行,樗蒱赌博,恶习成风,不足为训。但从另一方面看,赌徒心理,勇于冒险,意志力异于常人,如能加以正确引导,则可变坏事为好事,改造为一个有用的新人。温峤即是一例。赌博是一种冒险。但世上诸事,多有风险。如今之股票市场,以及民间的彩票之类,又何尝不是一种变相的赌博?又如战争,不仅有军事输赢的风险,更兼有政治风险。如美国打伊拉克,取得了军事上的胜利,但却带来了国际政治关系的被动,时至今日,仍深陷泥潭而难以脱身。赌之输赢,如果去其操纵因素,则赌者的坚强心理及其智慧判断,是其胜负的重要原因。温峤后来改邪归正,终成东晋复国的中兴名臣,或多或少与此心理素质训练有关。故事情节生动,跌宕有致,人物形象声口毕肖,不仅画出了温峤的无赖,即第二主人公庾亮,慷慨“送直”,也是可人一个。}

23.温公\myidx{温峤}喜慢语\footnote{温公:指温峤。朱铸禹\CJKunderwave{汇校集注}以为指桓温。按:卞壸大桓温31岁,卞死于晋成帝咸和三年(328),桓温生于312年,卞卒时桓年仅16岁,则二人相互剖击,当在更早之年。一个朝中重臣,何事与一个十几岁的孩子过不去呢?故朱说不可信。慢语:轻慢不严肃的说话。},卞令\myidx{卞壸}礼法自居\footnote{礼法:礼教法度。卞令:指卞壸,字望之,济阴冤句人。曾任尚书令,故称。参前\CJKunderwave{赏誉}第54则注。}。{\fzxk\zihao{6}\textcolor{red}{\CJKunderwave{卞壸别传}曰:“壸正色立朝,百僚严惮,贵游子弟,莫不祗肃。”}} 至庾公\myidx{庾亮}许\footnote{庾公:指庾亮。许:处所。},大相剖击\footnote{剖击:批评攻击。},温发口鄙秽\footnote{发口:开口。鄙秽:粗俗脏话。},庾公徐曰:“太真终日无鄙言\footnote{太真:温峤,字太真。}。”{\fzxk\zihao{6}\textcolor{red}{重其达也。}}

{\cangkai\zihao{5}【评】温峤年轻时,曾较多接触下层民间的生活,颇受民俗影响。加以他性格豪爽,不拘细行,有任诞放达之风,因而开口发言,俗语粗鄙,而与上层礼教之士风格异趣,自不待言。他在庾亮处与卞壸相互剖击,则当在南渡发达之后,却仍无改其旧日习性。其戏言慢语,犹如今天的善于说笑,见其生动风趣。史称温峤“美于言谈,见者皆爱悦之”,可见是一个善于言辞的人,其所不屑者在言语矫饰,故反其道而行之,名教之士以此斥之为“发言鄙秽”。庾亮儒、玄双修,故折衷于温、卞之间。但就其倾向而言,其同情的天平,似向温峤倾斜,而不以放诞为非。庾是太真知友,他在其戏言慢语中,洞悉其真情深意,故谓“太真终日无鄙言”。刘注云:“重其达也。”放达率性之言,何须遵循礼法名教来唱老调。一个国家领导人,能有这样开通的见识,也是难能可贵的。}

\lettrine{23.28} 周伯仁\myidx{周顗}风德雅重\footnote{周伯仁:周顗,字伯仁,见前注。风德:风操品德。雅重:正派庄重。},深达危乱\footnote{深达:深刻认识。}。过江积年\footnote{过江:指永嘉乱后,士大夫大批南渡长江。积年:多年。},恒大饮酒\footnote{恒:经常,总是。},尝经三日不醒\footnote{三日不醒:此谓醉酒三日不醒。但\CJKunderwave{太平御览}卷四九七\CJKunderwave{酣醉门}引\CJKunderwave{语林}作“三日醒”,\CJKunderwave{南史·陈暄传}:“与兄子秀书曰:‘昔闻周伯仁渡江,唯三日醒。’”据此,可备一说。},时人谓之三日仆射\footnote{三日仆射:谓周顗惟于姑、姊丧时三日醒,馀皆沉酣醉乡。“三日仆射”,讥其为只会饮酒而不办事的宰相。}。{\fzxk\zihao{6}\textcolor{red}{\CJKunderwave{晋阳秋}曰:“初,顗以雅望获海内盛名,后屡以酒失。庾亮曰:‘周侯末年,可谓凤德之哀也。’”\CJKunderwave{语林}曰:“伯仁正有姊丧,三日醉,姑丧二日醉。大损资望。每醉,诸公常共屯守。”}}

{\cangkai\zihao{5}【评】周顗“深达危乱”,颇具忧患意识,原本是个有理想有志气的人物。但因忧患而痛哭相对,或终日饮酒而长醉不醒。前述新亭聚会,周顗中坐而叹:“风景不殊,正自有河山之异!”王导愀然变色,谓当努力“克复神州”,饮酒痛哭,能把敌人赶出中原而恢复故国吗?当然,称周顗为“三日仆射”,也是夸大之辞,实属讥贬过当。王敦之叛,当众痛斥,慨然赴义,其铮铮风节,又岂是惟三日醒之仆射!读书不可只见其一端。}

23.29 卫君长\myidx{卫永}为温公\myidx{温峤}长史\footnote{卫君长:卫永,字君长。成阳(今属山东)人。官左军长史。参前\CJKunderwave{赏誉}第107则注。温公:指温峤。朱铸禹\CJKunderwave{汇校集注}谓指桓温,误。},温公甚善之\footnote{善:亲善,友好。},每率尔提酒脯就卫\footnote{率尔:随意。脯:干肉。就卫:看望卫永。就,到……去。},箕踞相对弥日\footnote{箕踞:古时席地而坐,岔开两腿,形似簸箕,是一种随意而傲慢的坐姿。弥日:整天。}。卫往温许\footnote{许:处,所。},亦尔\footnote{亦尔:亦然,一样。}。{\fzxk\zihao{6}\textcolor{red}{卫永,已见。}}

{\cangkai\zihao{5}【评】在官本位的封建社会中,温(峤)、卫(永)二人,无视上下等级之别,提酒携脯,箕踞相对,人以为怪。实际上,上司与下级也是人,除了政治工作关系外,人们之间还有亲情、友情等其他关系存在。不然的话,人就成了政治机器中的一只齿轮或螺丝钉了,永远在主人的操纵下作机械呆板的运作。如此之人,虽然俗世称是,但又有什么人生意味和价值呢?温卫之率尔自然,正见魏晋士人内心毫无粉饰之可爱,何“诞”之有?}

23.30 苏峻\myidx{苏峻}乱\footnote{苏峻:字子高,长广掖人。参前\CJKunderwave{方正}第34则注。乱:叛乱。按晋成帝咸和二年(327),苏峻与祖约以诛执政庾亮为名,举兵向阙,后败亡。},诸庾逃散\footnote{诸庾:指庾亮兄弟子侄。}。庾冰\myidx{庾冰}时为吴郡\footnote{庾冰:字季坚。亮弟。时为吴郡内史。王导卒后,继之为相。参前\CJKunderwave{方正}第41则注。},单身奔亡。民吏皆去,唯郡卒独以小船载冰出钱塘口\footnote{钱塘口:钱塘江口。},蘧篨覆之\footnote{蘧篨(qú chú衢除):粗竹席、芦席之类。}。时峻赏募觅冰\footnote{赏募:悬赏。觅:此指捉拿。},属所在搜检甚急\footnote{搜检:搜索检查。}。卒舍船市渚\footnote{市渚:到小洲上买东西。渚,水中小洲。},因饮酒醉,还,舞棹向船曰\footnote{棹:船桨。}:“何处觅庾吴郡\footnote{庾吴郡:指庾冰,时任吴郡内史。}?此中便是!”冰大惶怖,然不敢动。监司见船小装狭\footnote{监司:检查站负责人。},谓卒狂醉,都不复疑。自送过淛江\footnote{自:表已然的副词。淛江:浙江。淛,“浙”的异体字。},寄山阴魏家\footnote{山阴:县名,今浙江绍兴市。},得免。{\fzxk\zihao{6}\textcolor{red}{\CJKunderwave{中兴书}曰:“冰为吴郡,苏峻作逆,遣军伐冰,冰弃郡奔会稽。”后事平,冰欲报卒 }}{\fzxk\zihao{6}\textcolor{red}{\footnote{报:回报,报答。},适其所愿 }}{\fzxk\zihao{6}\textcolor{red}{\footnote{适:满足。}。卒曰:“出自厮下 }}{\fzxk\zihao{6}\textcolor{red}{\footnote{厮下:地位低下的仆役。},不愿名器 }}{\fzxk\zihao{6}\textcolor{red}{\footnote{名器:指官职爵位。}。少苦执鞭 }}{\fzxk\zihao{6}\textcolor{red}{\footnote{执鞭:喻供人驱遣。},恒患不得快饮酒;使其酒足馀年 }}{\fzxk\zihao{6}\textcolor{red}{\footnote{其:此为第一人称“我”的代词。馀年:残生,下半生。},毕矣 }}{\fzxk\zihao{6}\textcolor{red}{\footnote{毕矣:满足了。},无所复须 }}{\fzxk\zihao{6}\textcolor{red}{\footnote{无所复须:更无他求。}。”冰为起大舍,市奴婢,使门内有百斛酒,终其身。时谓此卒非唯有智,且亦达生 }}{\fzxk\zihao{6}\textcolor{red}{\footnote{达生:懂得生活,不为物累。参\CJKunderwave{庄子·达生}篇。}。}}

{\cangkai\zihao{5}【评】这是一篇以小小说面貌出现的纪实文学作品。故事情节跌宕起伏,于变化中见其章法层次。郡卒醉酒,舞棹大呼而庾冰惶怖,戏剧化的一幕,以悬念带动情节发展,把矛盾导向高潮,令人怦然心跳不已。作者誉郡卒,“非惟有智,且亦达生”,信然。其醉酒之呼,如诸葛亮唱空城计。所不同者,诸葛清醒用计,内心实具理性之紧张,故司马懿退兵后,冷汗直下;郡卒则醉中无意识行为,何来畏惧?更有甚者,事成之后,并不居功,更不贪图名器,拒绝非分之想。这是理解生活而抛弃物累的达生之举,“小人”之智,胜于贪恋功名的“大人先生”多多!}

\lettrine{23.31} 殷洪乔\myidx{殷羡}作豫章郡\footnote{殷洪乔:殷羡字洪乔。陈郡长平(今河南西华东北)人,浩父。官至光禄勋。作豫章郡:担任豫章太守。豫章郡,地名,在今江西省地。},{\fzxk\zihao{6}\textcolor{red}{\CJKunderwave{殷氏谱}曰:“羡,字洪乔,陈郡人。父识,镇东司马。羡仕至豫章太守。”}}临去,都下人因附百许函书 \footnote{因附:因便附寄捎带。百许函书:百来封信。许,表约数。函,量词,用于书信等。}。既至石头\footnote{石头:石头城在建康西,形势险要,是捍卫京师的军事要地。},悉掷水中,因祝曰:“沉者自沉,浮者自浮,殷洪乔不能作致书邮\footnote{致书邮:送信邮差。}!”

{\cangkai\zihao{5}【评】“殷洪乔不能作致书邮”,乃魏晋士人高自身份的标榜之辞。儒重然诺,言必有信。殷羡乖违传统信念,不愿为他人捎带书信,公开表明,并无不可。但应允之后,却大量毁弃他人书信,如在今天,违反邮政之法,属犯罪行为;即在古代,其所作“达”,则也属流氓无赖行径,是不道德的事情。此类事情,古代时有发生。唐代刘餗\CJKunderwave{隋唐嘉话}载:“梁常侍徐陵聘于齐,时魏收文学北朝之秀,收录其文集以遗陵,令传之江左。陵还,济江而沉之。从者以问,陵曰:‘吾为魏公藏拙。’”弃人书信文集,实是出于蔑视他人之心,其狂且诞,何足为训!史称殷羡任长沙太守时,其上司荆州刺史庾翼斥之“在郡贪残,……江州所统一二十郡,唯长沙最恶。恶而不黜,与杀督监者复何异耶”?(载\CJKunderwave{晋书·庾翼传})殷羡之灵魂丑陋,导致了行为之怪诞,本性如此,何足道哉!}

\lettrine{23.32} 王长史\myidx{王濛}、谢仁祖\myidx{谢尚}同为王公\myidx{王导}掾\footnote{王长史:指王濛,曾任司徒左长史,故称。谢仁祖:谢尚,字仁祖。鲲子。官至尚书左仆射、镇西将军、豫州刺史。掾:僚佐。},{\fzxk\zihao{6}\textcolor{red}{\CJKunderwave{王濛别传}曰:“丞相王导辟名士时贤,协赞中兴。旌命所加,必延俊乂。辟濛为掾。”}} 长史云:“谢掾能作异舞\footnote{异舞:风姿奇异的舞蹈。}。”谢便起舞,神意甚暇\footnote{神意甚暇:神态悠然自得。暇,悠闲貌。}{\fzxk\zihao{6}\textcolor{red}{。\CJKunderwave{晋阳秋}曰:“尚性通任,善音乐。”\CJKunderwave{语林}曰:“谢镇西酒后,于槃案间为洛市肆上鸲鹆舞,甚佳。”}} 王公熟视\footnote{王公:指王导。熟视:注目细看。},谓客曰:“使人思安丰\myidx{王戎}\footnote{安丰:指王戎。戎封安丰侯,故称。}。”{\fzxk\zihao{6}\textcolor{red}{戎性通任,尚类之。}}

{\cangkai\zihao{5}【评】魏晋士人追求艺术化人生的努力,在这则故事中有形象的启示和表现。魏晋陈郡阳夏谢家,经历了长期的发展,到东晋时,已成为一个世代簪缨的华丽家族,并实现了由硕儒到达士的转化,也就是说,儒、玄双修成了谢氏这一世家望族的新风尚。尚父鲲不仅好\CJKunderwave{老}、\CJKunderwave{易},善清谈,而且放达不拘,“能歌善鼓琴”。鲲之艺术细胞,通过家学传给了儿子谢尚。史称谢尚开率颖秀,聪明绝伦,脱略细行,不为流俗之事,“善音乐,博综众艺”,极富艺术家气质。故事所称“异舞”,即民间市肆的\CJKunderwave{鸲鹆舞}。当时谢尚“便著衣帻而舞,(王)导令坐者抚掌击节,尚俯仰在中,傍若无人”。魏晋士人那超凡脱俗的艺术人生,其气氛之热烈,给人以亲临其境一般的强烈感染。王导以谢尚为“小安丰”,不仅因王戎是其家族先辈,更可看出竹林遗风对于东晋士风的无形影响。}

\lettrine{23.33} 王\myidx{王濛}、刘\myidx{刘惔}共在杭南\footnote{王、刘:王指王濛,刘指刘惔,皆为东晋玄学清谈名士。王濛参前\CJKunderwave{言语}第66则注。刘惔参前\CJKunderwave{德行}第35则注。杭南:杭,同“航”,指东晋京师建康之朱雀航。乌衣巷距朱雀桥不远,则杭南指当日王、谢家族聚居之地。},酣宴于桓子野\myidx{桓伊}家\footnote{桓子野:桓伊,小字子野。参前\CJKunderwave{方正}第5则注。}。{\fzxk\zihao{6}\textcolor{red}{伊,已见。}} 谢镇西\myidx{谢尚}往尚书\myidx{谢裒}墓还\footnote{谢镇西:谢尚曾任镇西将军,故称。尚书:指尚叔谢裒,即谢安父。},葬后三日反哭\footnote{反哭:古代丧礼,葬后奉神主返庙。}。诸人欲要之\footnote{要:通“邀”,邀请。},初遣一信\footnote{信:使者。},犹未许,然已停车;重要,便回驾。诸人门外迎之,把臂便下\footnote{把臂:捉臂以示亲切。}。裁得脱帻\footnote{帻:发巾。按:疑“帻”与“帽”二字误倒。其连续动作应是脱帽著帻,以示自由随意。},箸帽酣宴。半坐\footnote{半坐:宴会进行了一半。},乃觉未脱衰\footnote{乃:方才。衰:丧服。}。{\fzxk\zihao{6}\textcolor{red}{尚书,谢裒,尚叔也,已见。宋明帝\CJKunderwave{文章志}曰:“尚性轻率,不拘细行。兄葬后,往墓还。王濛、刘惔共游新亭,濛欲招尚,先已问惔曰:‘计仁祖正当不为异同耳。’惔曰:‘仁祖韵中自应来。’乃遣要之。尚初辞,然已无归意。乃再请,即回轩焉。其率如此。”}}

{\cangkai\zihao{5}【评】有其父乃有其子,谢鲲放达任诞之风,与其艺术细胞混杂,全面地传给了儿子谢尚。二人所不同的,以时代关系,谢鲲之达因世乱而包含了几分伤心无奈,故纵酒调戏邻女,其所谓“达”,多了若干粗鄙庸俗之气;而谢尚则时过境迁,生活安定而官运亨通。他与堂弟奕、安、万等,为任诞放达行为作精神升华,呈现了超越世俗的“雅人深致”气象。反哭之日,未脱丧服而饮宴,当然有违传统礼教。但对魏晋名士来说,只是小事一桩,何必大惊小怪!心理变化了,行为自然不同,这也是越名教而任自然的一种名士气派。}

\lettrine{23.34} 桓宣武\myidx{桓温}少家贫\footnote{桓宣武:指桓温,卒谥宣武,故称。},戏大输\footnote{戏:赌博,此指樗蒱之戏。},债主敦求甚切。思自振之方\footnote{自振:自救。振。振兴。},莫知所出。陈郡袁躭\myidx{袁躭}俊迈多能\footnote{袁躭:字彦道。王导参军,历官云阳太守、从事中郎。俊迈多能:超迈杰出,多才多艺。},{\fzxk\zihao{6}\textcolor{red}{\CJKunderwave{袁氏家传}曰:“躭字彦道,陈郡阳夏人。魏中郎令涣曾孙也。魁梧爽朗,高风振迈。少倜傥不羁,有异才,士人多归之。仕至司徒从事中郎。”}} 宣武欲求救于躭。躭时居艰\footnote{居艰:守丧期间。},恐致疑\footnote{致疑:引起犹豫,迟疑。},试以告焉,应声便许,略无嫌吝\footnote{嫌吝:疑惑顾惜。}。遂变服怀布帽\footnote{变服:脱下丧服。},随温去,与债主戏。躭素有艺名\footnote{艺名:技艺高超的声名。},债主就局\footnote{就局:上了赌台。},曰:“汝故当不办作袁彦道邪\footnote{故当:当然,可能。表肯定的拟测之词。不办:不会,不可能。}?”遂共戏。十万一掷,直上百万数,投马绝叫\footnote{马:樗蒱之码,赌时投掷决胜负。绝叫:高声喊叫。},傍若无人,探布帽掷对人曰:“汝竟识袁彦道不?”{\fzxk\zihao{6}\textcolor{red}{\CJKunderwave{郭子}曰:“桓公樗蒱,失数百斛米,求救于袁躭。躭在艰中,便云:‘大快,我必作采。卿但大唤。’即脱其衰,共出门去。觉头上有布帽,掷去,著小帽。既戏,袁形势呼袒,掷必卢雉,二人齐叫,敌家顷刻失数百万也。”}}

{\cangkai\zihao{5}【评】虽然东晋一代,桓温声名显赫。但在这一故事中,第一主角却让给了袁躭。躭居丧之日,脱掉丧服而直上赌台,这有违传统礼教。但为脱友之困,他应温之求而毫不迟疑。“应声便许,略无嫌吝”,下意识的自然反应中,正见其倜傥不羁的干云义气。“投马绝叫,傍若无人”,又见其专精此道,而心无旁骛,气势已压倒对手。最后脱帽细节,也极神气:“汝竟识袁彦道不?”声口毕肖,细腻刻画了主人公内心之自许自负。故事情节跌宕起伏,形象描绘极其生动,士人通脱,味之有致。}

\lettrine{23.35} 王光禄\myidx{王蕴}云\footnote{王光禄:指王蕴,字叔仁,小字阿兴。濛子。官至尚书左仆射、镇军将军、会稽内史。卒赠光禄大夫,故称。}:“酒,正使人人自远\footnote{自远:忘掉自我,心怀高远。}。”{\fzxk\zihao{6}\textcolor{red}{光禄,王蕴也。\CJKunderwave{续晋阳秋}曰:“蕴素嗜酒,末年尤甚,及在会稽,略少醒日。”}}

{\cangkai\zihao{5}【评】据\CJKunderwave{晋书·外戚·王蕴传},蕴有外戚之贵,但政绩甚佳。荒年开仓救灾而存活百姓,因违科而免官。但他坦然表示:“行仁义而败,无所恨也。”以此百姓歌之。其晚年嗜酒,虽然略少醒日,但心中明白,为政仍然“以和简为百姓所悦”。这说明他嗜酒并非醉生梦死,而是知酒趣而不忘百姓疾苦。所谓“酒,正使人人自远”,“自远”者并非醉死忘我之谓,而是提高人的精神境界,自然具超凡脱俗情致。这与\CJKunderwave{郭子}所称相似:“三日不饮酒,觉形神不复和,酒自引人入胜地耳。”酒趣美妙如此,但视饮者如何耳!}

\lettrine{23.36} 刘尹\myidx{刘惔}云\footnote{刘尹:指东晋清谈名家刘惔,曾任丹阳尹,故称。}:“孙承公\myidx{孙统}狂士\footnote{孙承公:孙统,字承公,太原人。参前\CJKunderwave{品藻}第59则注。},每至一处,赏玩累日\footnote{累日:多日。},或回至半路却返。”{\fzxk\zihao{6}\textcolor{red}{\CJKunderwave{中兴书}曰:“承公少诞任不羁。家于会稽,性好山水。及求鄞县,遗心细务,纵意游肆,名阜胜川,靡不历览。”}}

{\cangkai\zihao{5}【评】孙统祖楚弟绰,在两晋皆为著名文学家。统善属文,自有家学渊源。其好山水,出游赏玩,累日不返,则出于时代玄风之熏染及其本性之自然。任诞不拘之士,厌弃官场丑陋,独钟情于不染世故的佳山胜水,究其实质,是在自然人化的审美观赏中,努力寻找失去的自我。篇末“或回至半路却返”,沉浸在尚未被世俗功利所污染的一片风霜高洁、纯洁无瑕的大自然新天地中。魏晋山水的精神,原是植根在风雅之士那脱俗自然的心境福田之中。}

\lettrine{23.37} 袁彦道\myidx{袁躭}有二妹\footnote{袁彦道:袁躭,字彦道。参见本门第34则注。}:一适殷渊源\myidx{殷浩}\footnote{适:嫁。殷渊源:殷浩,字渊源。},一适谢仁祖\myidx{谢尚}\footnote{谢仁祖:谢尚,字仁祖。}。{\fzxk\zihao{6}\textcolor{red}{\CJKunderwave{袁氏谱}曰:“躭大妹名女皇,适殷浩;小妹名女在(正),适谢尚。”}}语桓宣武云\footnote{桓宣武:指桓温,卒谥宣武,故称。}:“恨不更有一人配卿{\fzxk\zihao{6}\textcolor{red}{\footnote{恨:遗憾,可惜。配:许配。}。”}}

{\cangkai\zihao{5}【评】“恨不更有一人配卿”,今张万起、刘尚慈\CJKunderwave{译注}评云:“按袁躭言此,不合君子之风,有失礼仪,实际上是为了讨好桓温。作者置于\CJKunderwave{任诞},是认为袁躭言行任放不羁。”所论给人以启迪。但“讨好桓温”之说,则不敢苟同,本门第34则,桓温年轻时好赌大输,幸亏袁躭出手翻本救之。二人年轻时本是惺惺相惜的好友。桓温年轻时的地位与处境,与后来威权显赫的大司马不可同日而语。在温年轻论婚嫁时,袁躭为什么要“讨好桓温”呢?袁躭短命,待桓温发达时,早已墓木栱矣。而且,一旦真是“讨好桓温”,则是功利之所在,又岂是“任放不羁”之言行?讨论问题,应注意时间和环境所在。故事入\CJKunderwave{任诞}门,正说明作者的认识。}

\lettrine{23.38} 桓车骑\myidx{桓冲}在荆州\footnote{桓车骑:指桓冲,字幼子。温弟。曾任荆州刺史、车骑将军,故称。荆州:州名,时治所在江陵。},张玄\myidx{张玄}为侍中\footnote{张玄:又作张玄之,字祖希。官至吴兴太守、冠军将军。参前\CJKunderwave{言语}第5则注。侍中:官名,侍从皇帝,并备顾问。},使至江陵\footnote{江陵:县名,今属湖北省,当时为荆州治所。},路经阳歧村。{\fzxk\zihao{6}\textcolor{red}{村临江,去荆州二百里。}} 俄见一人\footnote{俄:一会儿。},持半小笼生鱼\footnote{生鱼:活鱼。},径来造船\footnote{径:径直,直接。造:到。},云:“有鱼,欲寄作鲙\footnote{寄:委托。鲙:切碎的鱼、肉。}。”张乃维舟而纳之\footnote{维:系、栓。纳:接待。},问其姓字,称是刘遗民\myidx{刘驎之}\footnote{刘遗民:指刘驎之,字子骥。南阳人。桓冲欲聘入幕,不赴。参前\CJKunderwave{栖逸}第8则注。}。{\fzxk\zihao{6}\textcolor{red}{\CJKunderwave{中兴书}曰:“刘驎之,一字遗民。”已见。}} 张素闻其名\footnote{素:平素,一向。},大相忻待\footnote{忻:通“欣”,欢欣。}。刘既知张衔命\footnote{衔命:身负使命。},问:“谢安、王文度并佳不\footnote{谢安、王文度:东晋帝时的辅政大臣。}?”张甚欲话言,刘了无停意\footnote{了无:一点没有。}。既进鲙,便去,云:“向得此鱼\footnote{向:方才。},观君船上当有鲙具,是故来耳。”于是便去。张乃追至刘家。为设酒,殊不清旨\footnote{清旨:清醇之味。},张高其人\footnote{高:高尚,作动词用。},不得已而饮之。方共对饮,刘便先起\footnote{便:却。},云:“今正伐荻\footnote{伐荻:收割苇荻。荻之形似芦苇,可编席。},不宜久废。”张亦无以留之。

{\cangkai\zihao{5}【评】故事情节曲折有致。张玄官阶不低,刘驎之却是一介布衣隐士,二者地位悬殊,在官本位的封建社会中,本无自由交往之理。但张之可爱,在于打破官民隔阻,平等对待。更可爱的是刘驎之,并不因布衣之士而自卑身价,而是张扬自我,以我为主地待人接物,因突破礼法限制而被视为任诞。但他在高官面前,绝非仰视,相反,是在平视中多少带有些微俯视的角度。为什么?因为他的交往,绝无求人的功利目的,无欲则刚,此其所以能超脱世俗而高人一筹也。}

23.39 王子猷\myidx{王徽之}诣郗雍州\myidx{郗恢}\footnote{王子猷:王徽之,字子猷。羲之第五子。参前\CJKunderwave{雅量}第36则注。诣:到,访问。郗雍州:指郗恢,昙子。高平人。},{\fzxk\zihao{6}\textcolor{red}{\CJKunderwave{中兴书}曰:“郗恢,字道胤,高平人。父昙,北中郎将。恢长八尺,美须髯,风神魁梧。烈宗器之,以为蕃伯之望。自太子左率,擢为雍州刺史。”}} 雍州在内,见有𣰅㲪\footnote{𣰅㲪(tà dēnɡ榻登):同“毾㲪”,传自西域波斯的细羊毛毯,可作床、榻垫褥。},云:“阿乞那得此物!”{\fzxk\zihao{6}\textcolor{red}{阿乞,恢小字。}} 令左右送还家\footnote{左右:身边跟随伺候的仆役。}。郗出觅之,王曰:“向有大力者负之而趋\footnote{向:刚才。}。”{\fzxk\zihao{6}\textcolor{red}{\CJKunderwave{庄子}曰:“夫藏舟于壑,藏山于泽,谓之固矣;然有大力者负之而走,昧者不知也。”郗无忤色 }}{\fzxk\zihao{6}\textcolor{red}{\footnote{忤:抵触,不高兴。}。}}

{\cangkai\zihao{5}【评】王徽之公开做“贼”,何须推勘定案?难道他不知偷盗犯法,侵犯了别人的权益?非也。“令左右送还家”,为自己享用,而非劫富济贫,目的并不高尚。他之所以这样做,与魏晋高门望族中作达之士的特殊心理有关。第一,徽之是东晋名士,琅邪王家的子孙,光凭其门第之高贵,在当时就会产生一种俯视世人的傲慢与偏见,其言行只想到张扬自我,而不必顾及别人的想法和意见。其二,琅邪王家与高平郗家,世代儿女通婚,关系极其密切,偶然夺人所爱,想来也不至于引发至亲的激烈反应。“郗无忤色”,一方面说明郗恢之雅量,一方面也是在为至亲遮丑。第三,徽之“偷盗”有道,他引\CJKunderwave{庄子·大宗师}为自己的行为辩解,不是有玄学修养的人,能想出这种道理吗?徽之喻恢,物藏于你处或我处,均无不可,只要藏之天下,失与不失,同样通于大道。宣示偷“道”,振振有词,听来令人发噱。但是,徽之行事,无论美丑,均是内心透明而毫无遮掩,比之男盗女娼礼法之士的虚伪,又见其自然真率之可爱。}

23.40 谢安\myidx{谢安}始出西戏\footnote{戏:游玩。},失车牛,便杖策步归\footnote{杖策:扶着手杖。}。道逢刘尹\myidx{刘惔}\footnote{刘尹:指刘惔,谢安妻舅。},语曰:“安石将无伤\footnote{将无伤:大概没受伤吧。将,魏晋口语,大概,恐怕,表揣度委婉口气。}?”谢乃同载而归。

{\cangkai\zihao{5}【评】汉初缺马,自天子不能具纯驷,将相入朝,多用牛车。此后发展到魏晋,虽然经济发展,但贵族用牛车已成习俗。谢安是陈郡谢氏家族的代表人物,一代贵族之英,出门当然也用牛车。其出游时失牛车,表示连车带牛,都被人偷走了。这犹如今天的偷名牌汽车一样性质。在正常的情况下,作为一个大贵族,他身上当然不会一文莫名,破点小财,再雇部车回家,并无困难;即使一时身上没钱,车到家后再来付款,又何尝不可?但他不这么做,而是安步当车,闲庭信步,毫无惊慌恼怒之色。这充分展示了他的内在修养,徒步虽慢,却也换来了无拘无束的自由和畅快,其行为虽然人以为怪,但只要自己内心自在,又何必在乎世俗的议论呢?}

\lettrine{23.41} 襄阳罗友\myidx{罗友}有大韵\footnote{襄阳:郡名(今湖北襄樊)。大韵:很有风度、气韵。},少时多谓之痴。尝伺人祠\footnote{伺:窥伺,侦察。祠:祭祀。},欲乞食,往太蚤\footnote{蚤:通“早”。},门未开。主人迎神出见,问以非时何得在此?答曰:“闻卿祠,欲乞一顿食耳。”遂隐门侧,至晓得食便退,了无怍容\footnote{了无:丝毫没有。怍容:惭愧神色。}。为人有记功:从桓宣武\myidx{桓温}平蜀\footnote{桓宣武:桓温,卒谥宣武,故称。晋穆帝永和二年(346),桓温率军征蜀,次年蜀汉降。},按行蜀城阙观宇\footnote{城阙:城门宫阙。观宇:楼馆台榭。},内外道陌广狭\footnote{道陌:街市道路。},植种果竹多少,皆默记之。后宣武漂(溧)洲与简文\myidx{司马昱}集\footnote{漂洲:当为“溧洲”形讹。溧洲,长江中小洲,也称“洌洲”,在今南京西南。简文:指简文帝司马昱,时为会稽王。集:聚会。},友亦预焉\footnote{预:参与。}。共道蜀中事,亦有所遗忘,友皆名列\footnote{名列:一一依名条列。},曾无错漏\footnote{曾无:毫无。}。宣武验以蜀城阙簿\footnote{城阙簿:记载城池宫阙的籍册。},皆如其言,坐者叹服。谢公\myidx{谢安}\footnote{谢公:指谢安。}云:“罗友讵减魏阳元\myidx{魏舒}\footnote{魏阳元:魏舒字阳元。任城人。晋初官司徒。参前\CJKunderwave{赏誉}第17则注。}。”后为广州刺史,当之镇\footnote{当:将要。之:前往。},刺史桓豁\myidx{桓豁}语令莫来宿\footnote{桓豁:字朗子,温弟。继温任荆州刺史,官至征西大将军。莫:通“暮”。}。答曰:“民已有前期\footnote{前期:前约。},主人贫,或有酒馔之费\footnote{费:破费。},见与甚有旧\footnote{见与:犹与我。有旧:有交情。}。请别日奉命。”征西密遣人察之\footnote{征西:指桓豁。},至夕,乃往荆州门下书佐家\footnote{书佐:管理文书的佐吏。},处之怡然\footnote{怡然:融洽欢欣的样子。},不异胜达\footnote{胜达:名流贤达。}。在益州\footnote{益州:州名,其辖地在今四川一带。},语儿云:“我有五百人食器。”家中大惊,其由来清\footnote{由来:一向。清:清廉。},而忽有此物,定是二百五十沓乌樏\footnote{沓(tà踏):量词,套,副。乌樏:黑色套盒。樏,食盒,形扁中隔,可供二人共食。}。{\fzxk\zihao{6}\textcolor{red}{\CJKunderwave{晋阳秋}曰:“友,字它(宅)仁,襄阳人。少好学,不持节检。性嗜酒,当其所遇,不择士庶。之(又)好伺人祠,往乞馀食,虽复营署垆肆,不以为羞。桓过营责之,云:‘君太不逮,须食,何不就身求,乃至于此!’之(友)傲然不屑,答曰:‘就公乞食,今乃可得,明日已复无。’罗(温)大笑之。始仕荆州,后在温府,以家贫乞禄。温虽此之(以文)学遇之,而谓其诞肆,非治民才,许而不用。后同府人有得郡者,温为席赴别,友至尤晚。问之,友答曰:‘民性饮道嗜味,昨奉教旨,乃是首旦出门,于中路逢一鬼,大见捓揄,云:“我只见汝送人作郡,何以不见人送汝住郡?”民始怪,终惭,回还以解,不觉成淹缓之罪。’温虽笑其滑稽,而心颇愧焉。后以为襄阳太守,累迁广、益二州刺史。在藩,举其宏纲,不存小察,甚为吏民所安说。薨于益泊(州)。”}}

{\cangkai\zihao{5}【评】乞食之诞,只是罗友外表,他所追求的其实是内在精神之自由——即故事所称之“大韵”。其上司桓豁约请赴宴,他却因与小吏有约在先,信守然诺,委婉拒绝长官,而与书佐欢饮怡然,处之“不异胜达”。这说明他不问士庶之异,胸中自具超越功利之风韵。以此作达,既张扬自我,而不屈己随俗;嗜饮时士庶不分,平等对待,又说明了他同时还尊重别人,甚为可爱。这与王徽之只知自我而不尊重别人之诞,自有区别而有高低之分。}

\lettrine{23.42} 桓子野\myidx{桓伊}每闻清歌\footnote{桓子野:桓伊,字叔夏,字子野。参\CJKunderwave{方正}第55则注。清歌:无伴奏的挽歌。},辄唤“奈何”\footnote{辄:总是。奈何:魏晋时人吊丧,孝子循例哭唤“奈何”。}。谢公\myidx{谢安}闻之\footnote{谢公:指谢安。},曰:“子野可谓一往有深情。”

{\cangkai\zihao{5}【评】艺术化的人生,是魏晋士人的一种精神追求。东晋时羊昙善唱乐,桓伊能挽歌,袁山松喜歌\CJKunderwave{行路难},时人谓之“三绝”。艺术的精神正在于“一往情深”,方才能够动人心魄。若自己都不感动,这样的艺术是做作,是虚伪,又岂能感动他人?}

\lettrine{23.43} 张湛\myidx{张湛}好于斋前种松柏\footnote{张湛:字处度,小字驎。东晋高平人。孝武帝时仕至中书郎。精医术,著\CJKunderwave{养生要籍}。好\CJKunderwave{庄}、\CJKunderwave{列}。今存其\CJKunderwave{列子注}八卷。斋:斋屋,房舍。};{\fzxk\zihao{6}\textcolor{red}{\CJKunderwave{晋东宫官名}曰:“湛字处度,高平人。”\CJKunderwave{张氏谱}曰:“湛祖嶷,正员郎。父旷,镇军司马。湛仕至中书郎。”}} 时袁山松\myidx{袁山松}出游\footnote{袁山松:东晋陈郡人。官吴郡太守。参后\CJKunderwave{排调}第60则注。挽歌:古代出丧时,送葬者执绋挽丧车而唱哀悼之歌。},每好令左右作挽歌。{\fzxk\zihao{6}\textcolor{red}{山松别见。\CJKunderwave{续晋阳秋}曰:“袁山松善音乐。北人旧歌有\CJKunderwave{行路难}曲,辞颇疏质,山松好之,乃为文其章句,婉其节制。每因酒酣,从而歌之,听者莫不㳅(流)涕。初,羊昙善唱乐,桓伊能挽歌,及山松以\CJKunderwave{行路难}继之,时人谓之‘三绝’。”今云挽歌,未详。}} 时人谓“张屋下陈尸,袁道上行殡”。{\fzxk\zihao{6}\textcolor{red}{裴启\CJKunderwave{语林}曰:“张湛好于斋前种松,养鸲鹆。袁山松出游,好令左右作挽歌。时人云云。”}}

{\cangkai\zihao{5}【评】古人屋前斋后,多植榆柳桃李,如陶渊明\CJKunderwave{归园田居}有“榆柳荫后檐,桃李罗堂前”之言;于庐墓处则植松柏,故\CJKunderwave{古诗}有“松柏冢累累”之句。但张湛却一反传统习惯,斋前植松柏,故人有“屋下陈尸”之讥。挽歌是特殊的艺术,类似今之安魂曲,唯当送葬时歌唱,但袁山松却在平日出门之时,好令左右唱送葬曲,这同样违背传统习俗,故时人以“道上行殡”嘲之。不过,挽歌安魂曲之类,不仅是对已逝亲友的追念,更蕴藏了一股对于未来生命的理解、同情与期待。张、袁二士,只求自己的舒心适意,而置世俗的讥贬于不顾。在魏晋士人眼中,自我个性之舒适比遵循传统习俗更重要。}

\lettrine{23.44} 罗友\myidx{罗友}作荆州从事\footnote{罗友:参本门第41则注。荆州从事:荆州府衙的属官。}。桓宣武\myidx{桓温}为王车骑\myidx{王洽}集别\footnote{桓宣武:指桓温。王车骑:王洽,字敬和,王导第三子。参前\CJKunderwave{赏誉}第114则注。集别:集聚送别。},{\fzxk\zihao{6}\textcolor{red}{车骑,王洽,别见。}} 友进,坐良久\footnote{良久:很久。},辞出。宣武曰:“卿向欲咨事\footnote{向:刚才。咨事:咨询事情。},何以便去?”答曰:“友闻白羊肉美,一生未曾得吃,故冒求前耳\footnote{冒:冒昧。},无事可咨。今已饱,不复须驻\footnote{驻:停留。}。”了无惭色\footnote{惭色:惭愧之容。}。

{\cangkai\zihao{5}【评】罗友好吃,是美食家。为求美味口福,不请自来,吃饱自去,而不顾干犯上司,得罪朋友。“今已饱,不复须驻”,回答干脆利落,内心坦荡荡,而毫无虚伪遮饰。其实,不仅故事主角罗友具赤子之心,极其可爱,就是作为威权显赫的桓温,他有容忍的雅量,同样也颇可爱。魏晋士人的特殊心理,于此有形象的展现。}

\lettrine{23.45} 张驎\myidx{张湛}酒后\footnote{张驎:张湛小字驎。},挽歌甚凄苦。桓车骑\myidx{桓冲}曰\footnote{桓车骑:指桓冲,温弟。}:“卿非田横门人\footnote{田横:秦末人。韩信率军破齐,田横率五百人逃亡海岛,自立为王。后刘邦统一天下,田横羞愤自杀。},何乃顿尔至致\footnote{顿尔:突然,一下子。至致:到此地步。}?”{\fzxk\zihao{6}\textcolor{red}{驎,张湛小字也。\CJKunderwave{谯子法训}云:“有丧而歌者,或曰:‘彼为乐丧也,有不可乎?’谯子曰:‘书云:“四海遏密八音。”何乐丧之有!’曰:‘今丧有挽歌者,何以哉?’谯子曰:‘周闻之,盖高帝召齐田横,至于尸乡亭,自刎、奉首。从者挽至于宫,不敢哭而不胜哀,故为歌以寄哀音。彼则一时之为也。邻有丧,舂不相引,挽人衔枚,孰乐丧者邪?’”按\CJKunderwave{庄子}曰:“绋讴所生,必于斥苦。”司马彪注曰:“绋,引柩索也。斥,疏缓也。苦,用力也。引绋所以有讴歌者,为人有用力不齐,故促急之也。”\CJKunderwave{春秋左氏传}曰:“鲁哀公会吴伐齐,其将公孙夏命歌\CJKunderwave{虞殡}。”杜预曰:“\CJKunderwave{虞殡},送葬歌,示必死也。”\CJKunderwave{史记·绛侯世家}曰:“周勃以吹箫乐丧。”然则挽歌之来久矣,非始起于田横也。然谯氏引礼之文,颇有明据,非固陋者所能详闻。疑以传疑,以俟通博。}}

{\cangkai\zihao{5}【评】本门有好几则士人善唱挽歌的故事。看来,喜欢挽歌,是魏晋时代的一种特殊文化现象。当时人们承传正始、竹林遗风,多任诞之风,张扬自我,率情任性。因此,虽然挽歌是特定场合的送葬曲,但对魏晋士人来说,只要内心适意,突破丧礼拘束,有何不可?加以挽歌悲苦,一唱众和,以情感人,其艺术魅力,引发了某些士人的灵魂震颤和共鸣。当时贵族虽然过着灯红酒绿的优裕生活,但在残酷的政治斗争中,又有几人能常保富贵呢?不仅袁山松很快被孙恩所杀,就是桓伊等官高权重的官僚,处在东晋司马朝与王、谢、庾、桓四大家族此起彼落的斗争中,能不胆战心惊吗?士人好挽歌,正是一种特殊时代心理的表现。}

\lettrine{23.46} 王子猷\myidx{王徽之}尝暂寄人空宅住\footnote{王子猷:王徽之,字子猷,羲之子。参前注。暂寄:暂时借住。寄,寄居。},便令种竹。或问:“暂住,何烦尔?”王啸咏良久\footnote{啸咏:长啸歌咏。},直指竹曰:“何可一日无此君\footnote{此君:指竹。}!”{\fzxk\zihao{6}\textcolor{red}{\CJKunderwave{中兴书}曰:“徽之卓荦不羁,欲为傲达,放肆声色颇过度。时人钦其才,秽其行也。”}}

{\cangkai\zihao{5}【评】即使是暂住的历史瞬间,也不忘精神寄寓之所在。魏晋精神,其格调境界自然高深玄远。“何可一日无此君!”道来极有感情。拟人化的修辞运用中,人与竹浑然为一。竹之风神,已成为士人清雅高洁人格的象征。东晋士人之任达放诞,从初始裸体浴裎阶段逐渐向诗酒风流方面转化。饮酒赏竹,不仅是物质生活的享受,更属精神风流之升华。}

\lettrine{23.47} 王子猷\myidx{王徽之}居山阴\footnote{山阴:县名,在会稽山北,今浙江绍兴。},夜大雪,眠觉,开室,命酌酒,四望皎然。因起仿偟\footnote{仿偟:徘徊。},咏左思\CJKunderwave{招隐诗}\footnote{左思:西晋诗人,与陆机等并为太康之英。参前\CJKunderwave{文学}第68则注。\CJKunderwave{招隐诗}:左思作,共二首,咏隐士生活情趣。},{\fzxk\zihao{6}\textcolor{red}{\CJKunderwave{中兴书}曰:“徽之任性放达,弃官东归,居山阴也。”左诗曰:“杖策招隐士,荒涂横古今。岩穴无结构,丘中有鸣琴。白雪停阴冈,丹葩曜阳林。”}}忽忆戴安道\footnote{戴安道:戴逵字安道,谯郡铚(今属安徽)人。东晋多才多艺的著名画家。参前\CJKunderwave{雅量}第34则注。}。时戴在剡\footnote{剡:县名,晋属会稽郡,在此今浙江嵊州市。},即便夜乘小船就之\footnote{就之:拜访他,到他家去。}。经宿方至\footnote{经宿:一整夜。方:才。},造门不前而返\footnote{造门不前:到其家门而不入。}。人问其故,王曰:“吾本乘兴而行,兴尽而返,何必见戴!”

{\cangkai\zihao{5}【评】故事虽短,却生动地勾画出一个贵族子弟的率真灵魂。\CJKunderwave{任诞}篇记录了许多魏晋名士的怪诞言行,今天看来,似乎荒唐可笑,但在魏晋士林中,却屡见不鲜,反映了部分知识分子的真实心态,从而构成了魏晋风流的又一特殊风景线。试想,生活在一个丑陋而扭曲的社会中,在虚伪名教的遮掩下,正道直行者惨遭杀戮迫害,这就为丛驱雀,把部分士人驱上了“任诞”轨道,其怪诞的超常言行,正是对不正常社会的一种冷嘲热讽和消极反抗。若把王徽之的故事安放到当时的历史环境中,则自然见怪不怪。故事体现了主人公率情任性而近于自然的灵魂。在一个严重失常的环境中,王徽之却冲破一切虚伪矫饰,努力舒展自己那被扭曲的灵魂,以挽救正在沦落之中的真我。“乘兴而来,兴尽而返,何必见戴?”全无功利的目的,仅凭自我意趣的涌动,而不问旁人的感觉与议论。须知,张扬自我人性,追求自由解脱,展现超俗离尘风采,正是魏晋精神的重要内容之一。故凌濛初评曰:“读此每令人飘飘欲飞。”相较于世俗之请托走后门者,奔走于形势之途,“足将进而趦趄,口将言而嗫嚅”,人性被功利所扭曲,二者精神人格之高低优劣,形成鲜明的对比。}

\lettrine{23.48} 王卫军\myidx{王荟}云\footnote{王卫军:王荟,字敬文。王导子。仕至镇军将军,卒赠卫军将军,故称。参前\CJKunderwave{雅量}第26则注。}:“酒,正自引人箸胜地\footnote{正自:确实。箸:到,入。}。”{\fzxk\zihao{6}\textcolor{red}{王荟,已见。}}

{\cangkai\zihao{5}【评】王荟出身于琅邪王家,是王导的小儿子,门第清华高贵,但却一扫高门望族子弟对于功名爵禄的追逐,素有“夷泰无竞”的清誉。其饮酒早已超越物质享受的范围,而入于形神相和的精神品格胜境。\CJKunderwave{醉仙图记}有云:“凡醉有所宜:醉花宜昼,袭其光也;醉雪宜夜,消其洁也;醉楼宜暑,资其清也;醉水宜秋,泛其爽也。”此可为王荟之言作解。}

\lettrine{23.49} 王子猷\myidx{王徽之}出都\footnote{出都:赴京师。魏晋时习惯用语,“出”是一种由隐之显的行为,京师地位显要,故进京称“出都”。参周一良\CJKunderwave{魏晋南北朝史札记}。},尚在渚下\footnote{渚:水中小洲,此指建康东南青溪渚。}。旧闻桓子野\myidx{桓伊}善吹笛\footnote{桓子野:桓伊小字子野。淝水大战中,因功进右军将军,封永修县侯。其妙擅音乐,“为江左第一”。},{\fzxk\zihao{6}\textcolor{red}{\CJKunderwave{续晋阳秋}曰:“左将军桓伊善音乐。孝武饮燕,谢安侍坐,帝命伊吹笛,伊神色无忤,既吹一弄,乃放笛云:‘臣于筝乃不如笛,然自足以韵合歌管。臣有一奴善吹笛,且相便串,请进之。’帝赏其放率,听召奴。奴既至,吹笛,伊抚筝而歌怨诗,因以为谏也。”}}而不相识\footnote{而:却。}。遇桓于岸上过,王在船中,客有识之者,云是桓子野。王便令人与相闻\footnote{相闻:传话,通消息。},云:“闻君善吹笛,试为我一奏。”桓时已贵显\footnote{贵显:地位尊贵显赫。},素闻王名,即便回下车\footnote{回下车:转向下车。},踞胡床\footnote{踞:倚,靠。胡床:坐具,如今之交椅。},为作三调。弄毕\footnote{弄毕:演奏完毕。},便上车去,客主不交一言。

{\cangkai\zihao{5}【评】故事生动地描绘了魏晋名士的神韵风采。王徽之有恃才傲物、怪诞不近人情的一面;但另一方面,却是个性情中人,情之所至,自然而然,而绝无假饰,更不考虑别人的看法。其心思犹如水晶般透明。这种纯真童趣,俗人要学也很难。故事中的桓伊和王徽之一样可爱。二人原非相识。王的官位比桓伊低得多。在官本位的社会中,这是人际交往的一大障碍。但王徽之并未自感低人一等,桓伊也没凭权位自视高人一头。他们讲究的是艺术良心,随感觉走,于是桓为王一人专场演奏,态度认真。其所吹“三调”,据传即今日尚存的古曲\CJKunderwave{梅花三弄},描绘了梅花笑迎冰雪、傲斗严寒,从含苞待放、绽朵盛开,到清香四溢遗留人间的全过程,表现了魏晋士人对于洁白无瑕人生品格的歌颂。王世懋评:“佳境乃在末语。”其妙全在“客主不交一言”中。演奏者一丝不苟,曲终人去,馀音袅袅;欣赏者仍然沉浸在高尚的艺术境界中,甚至忘记了说声谢谢。这才是真正的知音,追求的是审美情趣的通感,而毫无功利的痕迹。}

\lettrine{23.50} 桓南郡\myidx{桓玄}被召作太子洗马\footnote{桓南郡:指桓玄,七岁时袭封南郡公,故称。参前\CJKunderwave{德行}第41则注。太子洗马:太子东宫属官。},{\fzxk\zihao{6}\textcolor{red}{\CJKunderwave{玄别传}曰:“玄初拜太子洗马。时朝廷以温有不臣之迹,故抑玄为素官。”}} 船泊荻渚\footnote{荻渚:小洲名。},王大\myidx{王忱}服散后已小醉\footnote{王大:指王忱,字元达,小字佛大。坦之子。官至荆州刺史。散:指当时士大夫常服的寒食散,又称五石散。服散为养生,但该散具毒性,服后体内燥热,须步行发散药性,称行散。又不可饮冷酒,须温酒以帮助药性发散。},往看桓。桓为设酒,不能冷饮,频语左右令“温酒来”。桓乃㳅(流)涕呜咽。王便欲去,桓以手巾掩泪,因谓王曰:“犯我家讳\footnote{家讳:玄父名温,故“温酒来”犯其家讳。},何预卿事\footnote{何预卿事:不关你事。}!”{\fzxk\zihao{6}\textcolor{red}{\CJKunderwave{晋安帝纪}曰:“玄哀乐过人,每欢戚之发,未尝不至呜咽。”}} 王叹曰:“灵宝故自达\footnote{达:放达。}。”{\fzxk\zihao{6}\textcolor{red}{灵宝,玄小字也。\CJKunderwave{异苑}曰:“玄生而有光照室。善占者云:‘此儿生有奇耀,宜字为天人。’宣武嫌其三文,复言为‘神灵宝’,犹复用三,既难重前,却减‘神’一字,名曰灵宝。”\CJKunderwave{语林}曰:“玄不立忌日,止立忌时。其达而不拘皆此类。”}}

{\cangkai\zihao{5}【评】刘义庆编撰\CJKunderwave{世说},吸取了魏晋人的观念,不以成败论英雄。桓玄篡晋自立,兵败被杀,是为逆贼,但却成为\CJKunderwave{世说}故事的主角之一。玄工心计,而非率性任情之人。此事当在其年轻有为之时。为笼络人心,行其“统战”,桓玄常是装态饰辞,故意作达,以与士大夫打成一片。犯其家父之讳而不顾,正是一种姿态。这与王徽之等超功利的任达,实有真伪之别,读者不可被他轻易骗过。}

\lettrine{23.51} 王孝伯\myidx{王恭}问王大\myidx{王忱}\footnote{王孝伯:王恭,字孝伯。太原人。蕴子,濛孙。官至中书令,出为五州都督、前将军、青兖二州刺史。参前\CJKunderwave{德行}第44则注。王大:指王忱,见前注。}:“阮籍\myidx{阮籍}何如司马相如\myidx{司马相如}\footnote{阮籍:字嗣宗,魏陈留人,与嵇康并为竹林七贤之首。司马相如:字长卿,蜀郡成都人。西汉大赋家。}?”王大曰:“阮籍胸中垒块\footnote{垒块:胸中情绪郁积难展。},故须酒浇之。”{\fzxk\zihao{6}\textcolor{red}{言阮皆同相如,而饮酒异耳。}}

{\cangkai\zihao{5}【评】司马相如和阮籍,是西汉和魏的大文学家,二人个性与成就,多有相似之处。\CJKunderwave{史记·司马相如列传}称相如之仕宦,“未尝肯与公卿国家之事,常称疾闲居,不慕官爵”。\CJKunderwave{高士传}则谓其慢世违俗,“越礼自放”。这与阮籍的任诞作达,不拘礼法,多有精神相通之处。所异者,时代精神大不相同。相如处于封建社会全面上升的开创年代,故其作达气魄宏伟,如其\CJKunderwave{难蜀中父老}辞曰:“盖世必有非常之人,然后有非常之事;有非常之事,然后有非常之功。非常者,固常(人)之所异也。”具有火红年代的恢宏气象。而阮籍则生于篡弑相继的乱世,饮酒不醉则有可能被卷入政治漩涡而丧命,所以悲剧的人生常用作达的喜剧形式加以表演。“胸中垒块,故须酒浇之”,一语破的,捕捉了魏晋时代精神之变化。}

\lettrine{23.52} 王佛大\myidx{王忱}叹言\footnote{王佛大:即王忱,小字佛大,见前注。}:“三日不饮酒,觉形神不复相亲。”{\fzxk\zihao{6}\textcolor{red}{\CJKunderwave{晋安帝纪}曰:“忱少慕达,好酒,在荆州转甚,一饮或至连日不醒,遂以此死。”宋明帝\CJKunderwave{文章志}曰:“忱嗜酒,醉辄经日,自号‘上顿’。世喭以大饮为‘上顿’,起自忱也。”}}

{\cangkai\zihao{5}【评】由于酒精刺激神经,醉酒之后犹如进入梦幻之境,悠忽飘荡,暂时忘却现实之苦难。但醉醒之后又当如何?形神相亲之我,又当截然一分为二,陷于内心矛盾的悲苦之中。后来李白有“举杯消愁愁更愁”之句,可为此语作一新的转解。}

\lettrine{23.53} 王孝伯\myidx{王恭}言\footnote{王孝伯:即王恭。参见前注。}:“名士不必须奇才,但使常得无事\footnote{但使:只要。},痛饮酒,孰读\CJKunderwave{离骚}\footnote{\CJKunderwave{离骚}:战国时楚国屈原的长诗,是其代表作。},便可称名士。”

{\cangkai\zihao{5}【评】史称王恭身为皇亲国戚,自负其才地高华,满怀理想而企望作为,故曾叹道:“仕宦不为宰相,才志何足以骋!”其积极入世的人生态度,当然与不拘礼法而超然世外的任诞之士大异旨趣。此言讽刺名士,有入木三分之妙。但应强调指出,只合为虚假名士画像,而与嵇、阮之辈先贤无涉。真名士之“痛饮酒,熟读\CJKunderwave{离骚}”,常富味外之味,语言文字背后,寓藏有深奥的文章。但假名士不过是紧跟流行的装模作样而已,岂是真知酒趣而读懂\CJKunderwave{离骚}!}

\lettrine{23.54} 王长史\myidx{王廞}登茅山\footnote{王长史:指王廞,王导孙。曾任司徒左长史,故称。茅山:山名,又称三茅山,道教圣地。在今江苏句容县境内。},大恸哭曰:“琅邪王伯舆,终当为情死!”{\fzxk\zihao{6}\textcolor{red}{\CJKunderwave{王氏谱}曰:“廞字伯舆,琅邪人。父荟,卫将军。廞历司徒长史。”周祇\CJKunderwave{隆安记}曰:“初,王恭将唱义,使喻三吴。廞居丧,拔以为吴国内史。国宝既死,恭罢兵,令廞反丧服。廞大怒,即日据吴都以叛。恭使司马刘牢之讨廞。廞败,不知所在。”}}

{\cangkai\zihao{5}【评】王廞出于正宗的琅邪王家之后,高自标榜而情怀激烈,任性而行却缺乏理性思考。其登茅山有感,发出“为情而死”之恸哭,似为日后悲剧命运埋伏笔;同时又为魏晋任诞名士悲剧作一收束,正与未来的历史发展相符若契。}




%%% Local Variables:
%%% mode: latex
%%% TeX-engine: xetex
%%% TeX-master: "../Main"
%%% End:
