%% -*- coding: utf-8 -*-
%% Time-stamp: <Chen Wang: 2025-12-09 22:53:01>

% ○ ◎ ‧ 「 」 『 』 々 ( ) “ ” ■ ^[一-龥]
% 【\([^】][^】][^】]+\)】 → {\\fzxk\\zihao{6}\\textcolor{red}{\1}}
% \(【评】.*\) → {\\cangkai\\zihao{5}\1}
% \(【题解】.*\) → {\\cangkai\\zihao{5}\1}
% 《\([^》]+\)》 → \\CJKunderwave{\1}
% ^\([0-9]+.[0-9]+\) → \\lettrine{\1}
% {\\fzxk\\zihao{6}\\textcolor{red}{[^o]*}}

\setlength{\parindent}{0pt}


\chapter{仇隙第三十六}



{\cangkai\zihao{5}【题解】 仇隙者,仇恨相敌与嫌隙交恶也。仇隙之事,何代没有?但在魏晋社会的动荡乱世中,表现尤为残酷。本门所载八则故事,所记并非大是大非的国家仇民族恨,而是专门描写属于当时统治阶级中士人之间的内部矛盾的仇隙,其中有程度较轻的嫌隙恩怨,如王羲之因与王述的纠葛,在父墓前誓不复仕;有小人谗险,为报私仇而耍弄权势,诬人叛逆而夺人金钱美女,如孙秀之杀石崇、潘岳、欧阳建;更有甚者,直接杀人而悬首于大街之上,如太傅司马道子之骂王恭。这类故事,不一而足,无不鲜血淋漓,令人发指。究其原因无不为权势、为金钱、为美女而疯狂。在统治者之间,既有谗险佞小的可恶,也有名士轻诋之狂傲,彼此争斗,相互厮杀,必欲置人死地而后快。于此可见,统治阶级内部斗争的残酷性,不亚于阶级斗争的对抗。如潘岳等一代文学天才,就是被孙秀罗织罪名而送上了断头台,悲乎,惜哉!}

\lettrine{36.1} 孙秀\myidx{孙秀}既恨石崇\myidx{石崇}不与绿珠\myidx{绿珠}\footnote{孙秀:字俊忠,琅邪人。助赵王伦篡位,任中书令。参前\CJKunderwave{贤媛}第17则注。石崇:字季伦,官至荆州刺史。参前\CJKunderwave{汰侈}第8则注。绿珠:石崇爱妾,工笛。后因孙秀抢夺,坠楼自杀。},{\fzxk\zihao{6}\textcolor{red}{干宝\CJKunderwave{晋纪}曰:“石崇有妓人绿珠,美而工笛。孙秀使人求之。崇别馆北邙下,方登凉观,临清水。使者以告,崇出其婢妾数十人以示之,曰:‘任所以择。’使者曰:‘本受命者,指绿珠也。朱(未)识孰是?’崇勃然曰:‘绿珠吾所爱,不可得也。’使者曰:‘君侯博古知今,察远照迩,愿加三思!’崇不然。使者已出,又反,崇竟不许。”}} 又憾潘岳\myidx{潘岳}昔遇之不以礼\footnote{潘岳:字安仁。荥阳人。西晋著名诗人。官至黄门侍郎。参前\CJKunderwave{文学}第70则注。}。后秀为中书令\footnote{中书令:朝廷中书省长官。},岳省内见之,因唤曰:“孙令,忆畴昔周旋不\footnote{畴昔:昔日。周旋:打交道,应酬。}?”秀曰:“中心藏之,何日忘之\footnote{“中心藏之”二句:\CJKunderwave{诗经·小雅·隰桑}诗句。}!”岳于是始知必不免。{\fzxk\zihao{6}\textcolor{red}{王隐\CJKunderwave{晋书}曰:“岳父文德为琅邪大守,孙秀为小史,给使,岳数蹴蹋秀,而不以人遇之也。”}} 后收石崇、欧阳坚石\myidx{欧阳建},同日收岳\footnote{收:逮捕。}。{\fzxk\zihao{6}\textcolor{red}{\CJKunderwave{晋阳秋}曰:“欧阳建字坚石,渤海人。有才藻,时人为之语曰:‘渤海赫赫,欧阳坚石。’初,建为冯翊太守,赵王伦为征西将军,秀腹心挠乱关中,建每匡正,由是有隙。”王隐\CJKunderwave{晋书}曰:“石崇、潘岳与贾谧相友善。及谧废,惧终见危,与淮南王谋诛伦,事泄,收崇及亲期以上皆斩之。初,岳母诫岳以止足之道。及收,与母别曰:‘负阿母!’崇家河北,收者至,曰:‘吾不过流徙交、广耳。’及车载东市,始叹曰:‘奴辈利吾家之财。’收崇人曰:‘知财为害,何不蚤散?’崇不能答。”}} 石先送市\footnote{市:东市刑场。},亦不相知\footnote{不相知:不知彼此情况。}。潘后至,石谓潘曰:“安仁,卿亦复尔邪\footnote{尔:这样。}?”潘曰:“可谓‘白首同所归’!”{\fzxk\zihao{6}\textcolor{red}{\CJKunderwave{语林}曰:“潘、石同刑东市,石谓潘曰:‘天下杀英雄,卿复何为?’潘曰:‘俊士填沟壑,馀波来及人。’”}} 潘\CJKunderwave{金谷诗集序}云:“投分寄石友\footnote{投分:意气相投。石友:情坚如金石的朋友。},白首同所归。”乃成其谶\footnote{谶:预言吉凶的语言文字。}。

{\cangkai\zihao{5}【评】小人谗险,其心难测。孙秀因年轻时潘岳未予善待,伺机报复。“中心藏之,何日忘之!”引经据典,以文绉绉的典雅之言,来发露长期潜伏心中的杀机,令人不寒而栗。一代文学天才,为此付出了生命的代价,悲哉!至于石崇,孙秀与之无冤无仇,只是贪其亿万家财及爱妾绿珠之色而丧尽天良,何其恐怖!史称孙秀“既执机衡,遂恣其奸谋,多杀忠良,以逞私欲……于是京邑君子不乐其生矣”(\CJKunderwave{晋书·赵王伦传}),后果极其严重,整个国家,深陷八王之乱的复仇夺权血海之中,直至灭亡。故事启人深思,对于小人,不可轻易结怨,一旦相仇,则没完没了,非常可怕。若斗争无计可避,则不可以小人之乞怜而加以轻饶。孔子早有“远小人”之教,确是经验之言。}

\lettrine{36.2} 刘玙\myidx{刘玙}兄弟\myidx{刘琨}少时为王恺\myidx{王恺}所憎\footnote{刘玙兄弟:指刘玙、刘琨。琨,字越石,中山人。西晋末官至大将军,并州刺史,后为段匹磾所害。参前\CJKunderwave{言语}第35则注。玙,\CJKunderwave{晋书}作“舆”,字庆孙。官至中书侍郎。兄弟二人,隽朗有才局,著名于时,京都民谣称“洛中奕奕,庆孙越石”。},尝召二人宿,欲默除之。令作坑,坑毕,垂加害矣\footnote{垂:将要。}。石崇\myidx{石崇}素与玙、琨善\footnote{石崇:参前\CJKunderwave{汰侈}第1则注。},闻就恺宿,知当有变\footnote{变:变故,事故。},便夜往诣恺\footnote{诣恺:到王恺家。},问二刘所在。恺卒迫不得讳\footnote{卒迫:紧急,急迫。卒,通“猝”。},答云:“在后斋中眠\footnote{后斋:后面书斋。}。”石便径入,自牵出\footnote{牵:拉。},同车而去,语曰:“少年,何以轻就人宿!”{\fzxk\zihao{6}\textcolor{red}{刘澯(邓粲)\CJKunderwave{晋纪}曰:“琨与兄玙俱知名,游权贵之间,当世以为豪杰。”}}

{\cangkai\zihao{5}【评】西晋刘玙(舆)、刘琨兄弟,生性隽朗,皆以雄豪著名。二人颇富文才,与石崇、陆机兄弟、潘岳等俱在贾谧二十四友之列。刘琨之诗,悲慨激越,青史传诵。但他们在青少年时代,不拘细行,游权门间,因而差点被王恺诱骗而坠入死亡深渊,如果不是友人相援,早已化作冤魂而哀游九泉。幼稚无知与缺乏人情经验,几乎让他俩上当受骗而付出生命的代价。王恺何许人也?他是晋武帝的舅父,自少骄狂无赖,性复豪侈,恃仗皇亲国戚,“所欲之事无所顾惮”,故死后谥号为丑。王恺为朝廷恶霸,诱杀二位少年,犹如踩死两只蚂蚁一般。人心险恶,世路维艰。加强法治以保护青少年的健康成长,惩处摧残青少年的罪犯,时代无论古今,都是历史的责任。}

\lettrine{36.3} 王大将军\myidx{王敦}执司马愍王\myidx{司马丞}\footnote{王大将军:王敦。司马愍王:司马丞,\CJKunderwave{晋书}“丞”作“承”。原封谯王,时官湘州刺史,卒谥愍王。},夜遣世将\myidx{王廙}载王于车而杀之\footnote{世将:王廙字世将,琅邪人。王敦,王导从弟。时官荆州刺史。},当时不尽知也\footnote{不尽知:并不都知道事实真相。}。{\fzxk\zihao{6}\textcolor{red}{\CJKunderwave{晋阳秋}曰:“司马丞(承)字元敬,谯王逊子也。为中宗相(湘)州刺史。路过武昌,王敦与燕会,酒酣,谓丞曰:‘大王笃实佳士,非将御之才。’对曰:‘焉知铅刀不能一割乎?’敦将谋逆,召丞为军司马。丞叹曰:‘吾其死矣。地荒民解,势孤援绝。赴君难,忠也;死王事,义也。死忠与义,又何求焉!’乃驰檄诸郡丞赴义。敦遣从母弟魏文(乂)攻丞,王廙使贼迎之,薨于车。敦既灭,追赠骠骑,谥曰愍王。”}} 虽愍王家亦未之皆悉\footnote{悉:了解,知道。},而无忌\myidx{司马无忌}兄弟皆稚。{\fzxk\zihao{6}\textcolor{red}{\CJKunderwave{无忌别传}曰:“无忌字公寿,丞子也。才器兼济,有文武干。袭封谯王,卫军将军。”}} 王胡之\myidx{王胡之}与无忌长甚相昵\footnote{王胡之:字修龄,王廙子。官至西中郎将,司州刺史。长:长大。 昵:亲近。},胡之尝共游。无忌入告母,请为馔\footnote{馔:饭菜。}。母流涕曰:“王敦昔肆酷汝父,假手世将\footnote{假手世将:借王廙之手加以杀害。}。{\fzxk\zihao{6}\textcolor{red}{\CJKunderwave{司马氏谱}曰:“丞娶南阳赵民(氏)女。”\CJKunderwave{王廙别传}曰:“廙字世将,祖览,父正。廙高朗豪率,王导、庾亮游于石头,会廙至。尔日迅风飞颿,廙倚船楼长啸,神气甚逸。导谓亮曰:‘世将为复识事。’亮曰:‘正足舒其逸耳。’性倨傲,不合己者面距之,故为物所疾。加平南将军,薨。”}} 吾所以积年不告汝者,王氏门强\footnote{王氏门强:指东晋初琅邪王氏家族势力强盛。},汝兄弟尚幼,不欲使此声箸\footnote{声箸:声张开来。箸,同“著”。},盖以避祸耳。”无忌惊号,抽刃而出,胡之去已远。

{\cangkai\zihao{5}【评】这是一则形象刻画世代结怨仇杀的政治故事。王敦任大将军时,心怀异志,举兵向阙,王廙附逆,二王共谋杀害了当时兴兵赴义的司马丞。这是强势豪族对司马皇室成员举起屠刀,也反映出世家望族与皇族之间争权的残酷性。但丞子司马无忌,与廙子王胡之,二人年幼不知此事,因而“长甚相昵”,年轻人关系很好。一旦得知实情,则杀父之仇,不共戴天,王廙已死,报复其子。故请客吃饭,顿时化作杀人宴席,司马无忌“抽刃而出”,昔日情谊顿时烟消云散。幸亏王胡之及时逃避,不然早已一命呜呼。下一代并不知情,何罪之有?但在封建时代,老一辈的仇怨,代代相传而难解。家族血缘的利害,也是时代仇杀的动因之一。}

\lettrine{36.4} 应镇南\myidx{应詹}作荆州\footnote{应镇南:应詹曾任镇南将军,故称。},{\fzxk\zihao{6}\textcolor{red}{王隐\CJKunderwave{晋书}曰:“应詹字思远,汝南顿人。璩曾孙(\CJKunderwave{晋书}本传作‘孙’,是)也。为人弘长有淹度,饰之以文才。司徒何充叹曰:‘所谓文质之士。’累迁江州刺史、镇南将军。”}} 王修载\myidx{王耆之}、谯王\myidx{司马丞}子无忌\myidx{司马无忌}同至新亭与别\footnote{王修载:王耆之字修载,琅邪人。廙子。官鄱阳太守、给事中。无忌:即司马无忌,丞子,袭封谯王。参前则注。新亭:地名,在京师建康南。}。坐上宾甚多,不悟二人俱到。有一客道:“谯王丞(承)致祸\footnote{谯王丞:司马丞,原封谯王,故称。致祸:被害。},非大将军意\footnote{大将军:王敦。},正是平南\myidx{王廙}所为耳\footnote{平南:指曾任平南将军的王廙。}。”无忌因夺直兵参军刀\footnote{直兵参军:值班参军。},便欲斫修载。走投水,舸上人接取得免\footnote{舸:船。接取:捞救。}。{\fzxk\zihao{6}\textcolor{red}{\CJKunderwave{中兴书}曰:“褚裒为江州,无忌于坐拔刀斫耆之,裒与桓景共免之。御史奏无忌欲专杀害,诏以赎论。”前章既言无忌母告之,而此章复云客叙其事。且王廙之害司马丞,遐迩共悉,修龄兄弟岂容不知?法盛之言,皆实录也。}}

{\cangkai\zihao{5}【评】胡之、耆之兄弟,因其父王廙参与杀害谯王司马丞,屡为丞子无忌追杀。在提倡以孝治国的两晋时代,为父报仇,虽属犯法,却也情有可原。无忌酒宴斫杀之事,御史中丞车灌劾之,成帝宽谅,诏曰:“王敦作乱,闵(愍)王遇祸,寻事原情,今王何责!”听以赎论。以后又予升迁,以便维护皇族权益,立场清晰可见。司马丞、无忌父子,可称忠孝两全。但究其实,丞之节义,名垂青史,当之无愧。而无忌欲取胡之、耆之兄弟性命,则属仇杀,法理何在?胡之兄弟当时年幼,并不知情,又有何罪?欲杀无辜之人,此风断不可长,冤冤相报,几代可了?岂不天下大乱!故成帝之诏又云:“然公私宪制,亦已有断,王当以国体为大,岂可寻绎由来,以乱朝宪!”诏诫之语,既顾人情,又循法典,可谓周全。}

\lettrine{36.5} 王右军\myidx{王羲之}素轻蓝田\myidx{王述}\footnote{王右军:王羲之。蓝田:王述。}。蓝田晚节论誉转重,右军尤不平。蓝田于会稽丁艰\footnote{丁艰:为守父母丧而辞官家居。},停山阴治丧\footnote{治丧:办理丧事。}。右军代为郡\footnote{代为郡:代替王述出任会稽内史。},屡言出吊\footnote{出吊:前往吊唁。},连日不果\footnote{不果:没有实现。}。后诣门自通,主人既哭,不前而去\footnote{不前:没有前去与丧主见面。},以陵辱之\footnote{陵辱:凌辱。}。于是彼此嫌隙大构\footnote{嫌隙大构:大结怨恨。}。后蓝田临杨(扬)州\footnote{临扬州:任扬州刺史。},右军尚在郡。初得消息,遣一参军诣朝廷,求分会稽为越州。使人受意失旨\footnote{受意失旨:错误领会其意图。},大为时贤所笑。蓝田密令从事数其郡诸不法\footnote{从事:州郡属官。数:责备。},以先有隙,令自为其宜。右军遂称疾去郡\footnote{去郡:辞去郡内史职务。},以愤慨致终。{\fzxk\zihao{6}\textcolor{red}{\CJKunderwave{中兴书}曰:“羲之与述志尚不同,而两不相能。述为会稽,艰居郡境。王羲之后为郡,申尉而已,初不重诣,述深以为恨。丧除,征拜扬州,就征,周行郡境,而不历羲之。临发,一别而去。羲之初语其友曰:‘王怀祖免丧,正可当尚书,投老可得为仆射。更望会稽,便自邈然。’述既显授,又检校会稽郡,求其得失,主者疲于课对。羲之耻慨,遂称疾去郡,墓前自誓不复仕。朝廷以其誓苦,不复征也。”}}

{\cangkai\zihao{5}【评】据羲之誓墓绝仕之辞,发生在穆帝永和十一年(355)前后。二王的嫌隙矛盾,可能肇自上品贵族高贵门第的傲慢与偏见。二人虽同一“王”字,但族望不同。王述出身太原王氏,王昶、王浑之后,世代显贵。因其门第高华,王述曾拒绝桓温求婚,以为门户不当。羲之则出身琅邪王氏,两晋之间,簪缨世家,东晋之时,一跃为第一贵族。琅邪王氏贵游子弟,眼中何曾有人?羲之傲慢,其子徽之、献之更成为本书\CJKunderwave{任诞}、\CJKunderwave{简傲}诸门的主角。二王相轻结怨,正是门第观念及其傲慢偏见在作祟。就事论事,二人都有责任,但羲之的错误更多一些。故其誓墓绝官之辞也显示了后果之严重:“自今之后,敢渝此心,贪冒苟进,是有无尊之心而不子也。子而不子,天地所不载,名教所不得容。信誓之诚,有如皦日!”为自己的傲慢付出了代价。不过,这在\CJKunderwave{仇隙}门中是程度最轻的。}

\lettrine{36.6} 王东亭\myidx{王珣}与孝伯\myidx{王恭}语\footnote{王东亭:王珣字元琳,琅邪人。导孙,洽子。封东亭侯,故称。参前\CJKunderwave{言语}第102则注。孝伯:王恭字孝伯。时任青、兖州刺史,镇京口。语:交谈,交换意见。},后渐异\footnote{异:指意见不合。}。孝伯谓东亭曰:“卿便不可复测\footnote{不可复测:难以捉摸。}。”答曰:“王陵廷争,陈平从默\footnote{“王陵廷争”二句:汉惠帝崩,吕后欲王诸吕,右相王陵当廷谏阻,左相陈平却阿顺吕后。但后来陈平与周勃设计尽诛诸吕,迎立文帝,此所谓“全社稷,安刘氏”也。},但问克终云何耳\footnote{但:只。克终:最后结局。云何:怎样。}。”{\fzxk\zihao{6}\textcolor{red}{\CJKunderwave{汉书}曰:“吕后欲王诸吕,问左(右)相王陵,以为不可。问左丞相陈平,平曰:‘可。’陵出,让平,平曰:‘面折廷争,臣不如君;全社稷,定刘氏,君不如臣。’”\CJKunderwave{晋安帝纪}曰:“初,王恭赴山陵,欲斩国宝,王珣固谏之,乃止。既而恭谓珣曰:‘此(比)日视君,一似胡广。’珣曰:‘王陵廷争,陈平从默,但问克终如何也。’”}}

{\cangkai\zihao{5}【评】故事发生在孝武帝崩的太元二十一年(396)。时会稽王道子执政,王国宝得宠,权倾朝野,“谋黜旧臣”。史称“王恭赴山陵,欲杀国宝,珣止之曰:‘国宝虽终为祸乱,要罪逆未彰,今便先事而发,必大失朝野之望。况拥强兵,窃发于京辇,谁谓非逆!国宝若遂不改,恶布天下,然后顺时望除之,亦无忧不济也。’恭乃止”。(见\CJKunderwave{晋书}珣传)反对司马道子宠昵小人王国宝,王恭、王珣并无不同。但如何除去王国宝则二人方法有异。恭性刚烈,欲直接兵诛于京师;珣则以为应等时机,待其恶贯满盈,则“顺时望除之”。二人意见分歧以此。实践证明,珣言是也。当时左卫将军豫州刺史庾楷等党于国宝,士马甚盛,王恭也有所畏惧,故兵诛之计暂止。作为政治家,王珣继承乃祖王导之风,更为成熟,而王恭则因其幼稚而付出了生命代价。“但问克终云何耳”,有味哉!}

36. 王孝伯\myidx{王恭}死\footnote{王孝伯:王恭。},县其首于大桁\footnote{县:通“悬”,悬挂。大桁:即朱雀桥,在建康城南朱雀门外,是当时秦淮河上最长的浮桥,故称。桁,通“航”。}。司马太傅\myidx{司马道子}命驾出至标所\footnote{标所:树标之处所。标,设于刑场的高柱,杀人之后,悬首标上。},熟视首\footnote{熟视:仔细看。},曰:“卿何故趣欲杀我邪\footnote{趣:通“促”,急。}?”{\fzxk\zihao{6}\textcolor{red}{\CJKunderwave{续晋阳秋}曰:“王恭深惧祸难,抗表起兵。于是遣左将军谢琰讨恭。恭败,走曲阿,为湖浦尉所擒。初,道子与恭善,欲载出都,面相折数,闻西军之逼,乃令于儿(倪)塘斩之,枭首于东桁也。”}}

{\cangkai\zihao{5}【评】这则与前则是时间相续的姐妹篇。故事发生在安帝隆安二年(398)。太元二十一年(396)王恭参加孝武帝葬礼返回京口后,即联络荆州殷仲堪及桓玄等,抗疏起兵清君侧,朝廷为诛王国宝和王绪。不久,王恭再次联合西师,东西夹击,兴兵向阙,反对执政司马道子和谯王尚之等权倖。兵者凶事,大军过处,百姓残灭,除非万不得已,岂可一动再动?其天时、地利、人和三者皆失,不知有理、有利、有节,视战争为儿戏,此王恭所以悬首大桁也。王恭贵族名士,虽自矜贵,但与下悬隔,又不闲用兵,尤信佛道,调役百姓,修营佛寺,务在壮观,史称“士庶怨嗟”。失却军心民心,又岂可再战乎?但贵游子弟不明此理,反观司马道子,挟天子以令诸侯,号令天下,王恭硬是往枪口上撞。杀王恭者,司马道子也,但更是王恭自己。即使自信真理在握,斗争也要讲究方法策略,一味刚猛冒进,终必“亢龙有悔”,败可立待。是耶?非耶?读者深思。}

\lettrine{36.8} 桓玄\myidx{桓玄}将篡\footnote{篡:篡位,废帝自立。},桓修\myidx{桓修}欲因玄在修母许袭之\footnote{桓修:字承祖,冲子,官至抚军大将军。许:住处。}。庾夫人云\footnote{庾夫人:桓冲续弦夫人,修母。}:“汝等近\footnote{汝等近:你们是近亲。按,修与玄是嫡堂兄弟。},过我馀年\footnote{过我馀年:等我死后,撒手不问。},我养之,不忍见行此事\footnote{此事:指桓修计划在家袭杀桓玄之事。}。”{\fzxk\zihao{6}\textcolor{red}{\CJKunderwave{桓氏谱}曰:“恒(桓)冲后娶颍川庾蔑女,字姚。”\CJKunderwave{晋安帝纪}曰:“修少为玄所侮,言论常鄙之,修深憾焉。密有图玄之意,修母曰:‘灵宝视我如母,汝等何忍骨肉相图!’修乃止。”}}

{\cangkai\zihao{5}【评】这则故事,应与\CJKunderwave{排调}第65则并读互参,反映了血亲家族谯国桓氏内部的血腥斗争。两晋之交,桓彝五子:温、云、豁、秘、冲。经过温等的共同努力,谯国桓氏终于成为东晋四大家族之一,参与朝廷的夺权斗争。但与琅邪王氏一样,谯国桓氏也非铁板一块,五子政治志向并不尽同。桓冲等继承父彝忠义气节,一心为国,“尽忠王室”;而温等则野心膨胀,觊觎帝位,后来少子玄终于了却温之未遂心愿,篡位建楚,但旋即覆灭。如\CJKunderwave{晋书·桓彝传}史臣曰:桓茂伦(彝)“怀不挠之节”,“冲逡巡于内辅,豁陵厉于上游”,“外有扞城之用,里无末大之嫌,求之名臣,抑亦可算”。“而温为亢极之资,玄遂履霜之业。”两种政治倾向,犁然有别。修为冲子,从小受玄欺侮,早已怀恨在心,加以受父影响,娶妻武昌公主,对晋室尚有感情。故知桓玄将篡之谋,担心遗祸家族,定计家中诛杀,惜为母阻而计不行。但玄得志后,修为名利驱动,为了家族利益,投怀送抱,封王拜将,终为义军所诛。一念之差,自取灭亡,呜呼哀哉!}





%%% Local Variables:
%%% mode: latex
%%% TeX-engine: xetex
%%% TeX-master: "../Main"
%%% End:
