%% -*- coding: utf-8 -*-
%% Time-stamp: <Chen Wang: 2025-12-06 11:26:25>

% ○ ◎ ‧ 「 」 『 』 々 ( ) “ ” ■ ^[一-龥]
% 【\([^】][^】][^】]+\)】 → {\\fzxk\\zihao{6}\\textcolor{red}{\1}}
% \(【评】.*\) → {\\cangkai\\zihao{5}\1}
% \(【题解】.*\) → {\\cangkai\\zihao{5}\1}
% 《\([^》]+\)》 → \\CJKunderwave{\1}
% ^\([0-9]+.[0-9]+\) → \\lettrine{\1}
% {\\fzxk\\zihao{6}\\textcolor{red}{[^o]*}}


\setlength{\parindent}{0pt}



\chapter{规箴第十}



{\cangkai\zihao{5}【题解】 规箴者,规讽劝告与警醒敕戒也。人的一生,错误不断,必须有人随时规箴劝诫,以免陷入泥潭而愈坠愈深。接受正确友好的劝告警诫,常常可以悬崖勒马,扭转人生航向,驶向胜利的彼岸。因此,不仅规箴主体是智慧之人,受规箴者也同样可以是智者而获益匪浅。反之,拒谏饰非,知错不改,则可能把人生引向万劫不复的深渊和地狱。本门27则故事,广泛涉及社会人生的方方面面,有臣谏君,下级规劝上级,老师告诫学生,甚至是亲友和夫妇之间,或直言以谏,或委婉规劝,无不表现其诚挚之情。当然,在魏晋动荡乱世,规诫暴君或权臣,不仅要有胆有识,而且必须高度运用语言修辞艺术,委婉讽谏,无懈可击,以便产生最佳的效果。如京房谏汉元帝,陆凯之讽吴主孙晧,若语稍不慎,必将大祸临头。其规箴人主,风险尤大,但其忧国忧民的耿耿忠心,则天日可表。}

\lettrine{10.1} 汉武帝\myidx{刘彻}乳母尝于外犯事\footnote{汉武帝:西汉第五代皇帝刘彻。犯事:犯法。},帝欲申宪\footnote{申宪:依法惩办。宪,法律。},乳母求救东方朔\myidx{东方朔}\footnote{东方朔(前154—前93年):字曼倩。西汉文学家,武帝时官至太中大夫。性诙谐好谑。\CJKunderwave{史记}有传。}。{\fzxk\zihao{6}\textcolor{red}{\CJKunderwave{汉书}曰:“朔字曼倩,平原厌次人。”\CJKunderwave{朔别传}曰:“朔,南阳步广里人。”\CJKunderwave{列仙传}曰:“朔是楚人。武帝时,上书说便宜,拜郎中。宣帝初,弃官而去,共谓岁星也。”}} 朔曰:“此非唇舌所争\footnote{唇舌:喻言辞。},尔必望济者\footnote{济:救助。},将去时,但当屡顾帝\footnote{顾:回顾,回头看望。},慎勿言,此或可万一冀耳\footnote{冀:希望。}。”乳母既至,朔亦侍侧\footnote{侍侧:一旁侍候。},因谓曰:“汝痴耳!帝岂复忆汝乳哺时恩邪\footnote{忆:回想,记得。}?”帝虽才雄心忍\footnote{才雄心忍:雄才大略而心性刚狠。},亦深有情恋\footnote{情恋:依恋不舍之情。},乃悽然愍之\footnote{愍之:哀怜她。},即敕免罪。{\fzxk\zihao{6}\textcolor{red}{\CJKunderwave{史记·滑稽列传}曰:“汉武帝少时,东武侯母尝养帝,后号大乳母。其子孙从奴横暴长安中,当道夺人衣物,有司请徙乳母于边,奏可。乳母入辞。帝所幸倡郭舍人,发言陈辞虽不合大道,然令人主和说。乳母乃先见,为下泣。舍人曰:‘即入辞,勿去,数还顾。’乳母如其言,舍人疾言骂之曰:‘咄,老女子!何不疾行?陛下已壮矣,宁尚须乳母活邪!尚何还顾邪?’于是人主怜之,诏止母(毋)徙,罚请者。”}}

{\cangkai\zihao{5}【评】故事反映的是情与法的矛盾及其取舍。事载\CJKunderwave{史记·滑稽列传},但事主非东方朔,而是郭舍人。盖朔为史上著名机辩滑稽之雄,故误以事归之。武帝乳母自恃帝宠,公然纵奴横暴长安,有司依法惩治,该是罪有应得。朔著\CJKunderwave{答客难},是位愤世嫉俗的正直之士,岂能为犯科权倖而施呈智辩?郭舍人则另当别论。史称时公卿大臣“皆敬重乳母”,更主要的是,武帝本人内心“不忍致之法”。郭舍人侍帝左右,明白其所想,故有此谏,帝借机赦乳母而“罪谪谮之者”——即倒打一耙而惩罚依法申宪之有司。在专制的社会中,情大于法,是非颠倒,黑白混淆,岂可作鉴?但临川选此作为\CJKunderwave{规箴}开篇,或是借古讽今而另有寓意。刘宋之初,文帝屠戮皇室手足,无情倾轧,故临川借武帝之动情而讽之乎?}

\lettrine{10.2} 京房\myidx{京房}与汉元帝\myidx{刘奭}共论\footnote{京房:西汉\CJKunderwave{易}学有二京房:一受学于杨何,传梁丘贺\CJKunderwave{易},官太中大夫,齐郡太守。一是本则故事主角的京房。今文经学\CJKunderwave{京氏易传}的开创者,原姓李,字君明,东郡顿丘(今河南清丰)人,官魏郡太守。其\CJKunderwave{易}学以灾异推论时政得失。后为权倖石显谗杀。 汉元帝:刘奭,宣帝子。西汉第八位皇帝。},因问帝:“幽、厉之君何以亡\footnote{幽、厉:指周幽王和周厉王。幽王荒淫乱政,厉王暴政,皆亡其国。何以:为什么。}?所任何人?”答曰:“其任人不忠。”房曰:“知不忠而任之,何邪?”曰:“亡国之君各贤其臣\footnote{各贤其臣:各以其臣为贤。},岂知不忠而任之?”房稽首曰\footnote{稽首:跪拜。}:“将恐今之视古,亦犹后之视今也。”{\fzxk\zihao{6}\textcolor{red}{\CJKunderwave{汉书}:“京房字君明,东郡顿丘人。尤好钟律,知音声。以孝廉为郎。是时,中书令石显专权,及友人五鹿充宗为尚书令,与房同经,论议相是非。而此二人用事。房尝宴见,问上曰:‘幽、厉之君何以亡?所任何人?’上曰:‘君亦不明,而臣巧佞。’房曰:‘知其巧佞而任之邪?将以为贤邪?’上曰:‘贤之。’房曰:‘然则今何以知其不贤?’上曰:‘以其时乱而君危知之。’房曰:‘是任贤而理,任不肖而乱,自然之道也。幽、厉何不觉悟而蚤纳贤,何为卒任不肖以至亡?’于是上曰:‘乱亡之君各贤其臣,令皆觉悟,安得乱亡之君?’房曰:‘齐桓、二世,何不以幽、厉卜之,而任竖刀、赵高,政治日乱邪?’上曰:‘唯有道者能以往知来耳。’房曰‘自陛下即位,盗贼不禁,刑人满市’云云,问上曰:‘今治邪?乱也?’上曰:‘然愈于彼。’房曰:‘前二君皆然。臣恐后之视今,犹今之视前也。’上曰:‘今为乱者谁?’房曰:‘上所亲与图事帷幄中者。’房指谓石显及充宗。显等乃建言,宜试房以郡守。遂以房为东(魏)郡。显发其私事,坐弃市。”}}

{\cangkai\zihao{5}【评】举贤任能,惩奸退恶,涉及国家用人的重大原则,关系到世运兴衰。京房问元帝:“幽、厉何以亡?”给当时的最高统治者敲响了警钟。矛头直指皇帝,其规谏胆识非常人能及。后来京房之死,直接原因是被权幸石显所杀,但生杀大权操在皇帝之手。京房忠言直谏,早为元帝不满,只是一时不便发作,以免拒谏恶名。杀京房者,实元帝本人。西汉亡于平帝,但元、成、哀、平,宦官外戚,擅权专政,一代不如一代,其亡国祸根实始于元帝。“将恐今之视古,亦犹后之视今也”,京房名言,极具警醒意义,惜昏庸的统治者终不觉悟,故致亡国,哀哉!}

\lettrine{10.3} 陈元方\myidx{陈纪}遭父丧\footnote{陈元方:陈纪字元方,寔长子。参\CJKunderwave{德行}第6则注。},哭泣哀恸,躯体骨立\footnote{躯体骨立:躯体消瘦只剩骨架。}。其母愍之\footnote{愍之:可怜他。},窃以锦被蒙上\footnote{锦被蒙上:盖上漂亮的锦被。}。郭林宗\myidx{郭太}吊而见之\footnote{郭林宗:郭太字林宗,汉末名士。参\CJKunderwave{德行}第3则注。},谓曰:“卿海内之隽才\footnote{隽才:俊杰才士。},四方是则\footnote{四方是则:各地之人把你当作效仿的榜样。},如何当丧\footnote{当丧:居丧守制期间。},锦被蒙上?孔子曰:‘衣夫锦也,食夫稻也\footnote{“衣夫锦也”二句:即锦衣玉食。锦,锦绣衣裳。 稻,糯稻精粮。},于汝安乎?’{\fzxk\zihao{6}\textcolor{red}{\CJKunderwave{论语}曰:“宰我问:‘三年之丧,期已久矣。’子曰:‘食夫稻,衣夫锦,于汝安乎?夫君子居丧,食旨不甘,闻乐不乐,居处不安,故不为也。今汝安,则为之。’”}} 吾不取也。”奋衣而去\footnote{奋衣:甩开衣服。}。自后宾客绝百所日\footnote{百所日:百馀日。}。{\fzxk\zihao{6}\textcolor{red}{“所”一作“许”。}}

{\cangkai\zihao{5}【评】郭太,后汉名士,史称其为人“天子不得臣,诸侯不得友”,是当时在野的士林清议领袖,性明知人,好奖训士类,经其品题,身价陡增;一旦批评,则人皆去之。陈纪父丧而盖锦被,致讥郭泰而“宾客绝百所日”,影响之大,可见一斑。但郭太卒于建宁二年(169),陈寔卒于中平四年(187)。郭太先寔之死十八年,何由见寔之丧?可谓绝非事实,而是张冠李戴,因震于郭太之名而误以事归之。陈纪之孝,当世有名,其蒙锦被,乃母所为,致讥士林舆论,实亦蒙冤。故凌濛初评曰:“无意中受谤,莫自可解,古来同恨。”}

\lettrine{10.4} 孙休\myidx{孙休}好射雉\footnote{孙休(233—263):三国东吴第三位君主,字子烈。权第六子,在位七年,崩谥景皇帝。},至其时,则晨去夕反\footnote{反:通“返”。},群臣莫不止谏:“此为小物,何足甚耽\footnote{耽:沉迷,贪恋。}?”休曰:“虽为小物,耿介过人\footnote{耿介:正直有操守。},朕所以好之。”{\fzxk\zihao{6}\textcolor{red}{环济\CJKunderwave{吴纪}曰:“休字子烈,吴大帝弟(第)六子。初封琅邪王,梦乘龙上天,顾不见尾。孙琳废少主,迎休立之。锐意典籍,欲毕览百家之事。颇好射雉,至春晨出莫反,唯此时舍书。崩,谥景皇帝。”\CJKunderwave{条列吴事}曰:“休在位烝烝,无有遗事,唯射雉可讥。”}}

{\cangkai\zihao{5}【评】畋猎射雉之风,古已有之。如春秋时贾大夫赴如皋射雉以乐其美妻,事载于\CJKunderwave{左传}昭公二十八年。三国时君主,多乐此不疲,如曹操父子及吴主孙权,无不皆然。魏文帝丕曾称射雉之乐,侍中辛毗谏曰:“于陛下甚乐,而于群下甚苦。”丕默然,遂为之稀出(见\CJKunderwave{三国志·魏书·辛毗传})。休父权同样好射雉,潘濬强谏,“见雉翳故在,手自撤坏之。权由是自绝,不复射雉”(见\CJKunderwave{三国志·吴书·潘濬传}注引\CJKunderwave{江表传})。开国之君,意在天下,胸怀较宽,从容纳谏,知错辄改,故国事可为。而休则年轻气盛,傲对臣下,饰辞拒谏,与乃父相形,其心胸气量不可同日而语。休言因雉性“耿介过人”而好猎,果真如此,射杀仁禽,岂非残酷?射雉小事,但因此而文过饰非,拒谏斥贤,宠任奸佞,虽好读书,却用来饰辞拒谏,又何益救乱乎?潘岳\CJKunderwave{射雉赋}云:“若乃耽槃流遁,放心不移。忘其身恤,司其雌雄,乐而无节,端操或亏。此则老氏所诫,君子不为。”曲终奏雅,有味哉,斯言。}

\lettrine{10.5} 孙皓(晧)\myidx{孙皓}问丞相陆凯\myidx{陆凯}曰\footnote{孙皓:三国时吴国末代君主。字元宗,权孙。降晋后封归命侯。参\CJKunderwave{排调}第5则注。 陆凯:陆凯于宝鼎元年(266)官拜左丞相。}:“卿一宗在朝有几人\footnote{宗:宗族。在朝:在朝为官。}?”陆曰:“二相、五侯、将军十馀人\footnote{二相:陆逊、陆凯。五侯:指陆胤等。将军十馀人:指陆抗等。}。”皓(晧)曰:“盛哉!”陆曰:“君贤臣忠,国之盛也;父慈子孝,家之盛也。今政荒民弊,覆亡是惧\footnote{覆亡是惧:担心亡国。},臣何敢言盛?”{\fzxk\zihao{6}\textcolor{red}{\CJKunderwave{吴录}曰:“凯字敬风,吴人,丞相逊族子。忠鲠有大节,笃志好学。初为建忠校尉,虽有军事,手不释卷。累迁左丞相。时后主暴虐,凯正直强谏,以其宗族强盛,不敢加诛也。”}}

{\cangkai\zihao{5}【评】故事当发生于陆凯拜相的宝鼎元年(266)至建衡元年(269)凯卒三年之间。时主昏政乱,虽凯等尽其忠言嘉谋,仍无救于吴国灭亡之趋势。凯卒前曾上表谏晧二十事,文殊切直,非晧所能容忍。其所谏争,知其不可为而为之,常言人之不敢言。如\CJKunderwave{三国志}本传注引\CJKunderwave{江表传},凯上表云:“臣虽愚,暗于天命,以心审之,败不过二十稔也。臣常忿亡国之人夏桀、殷纣,亦不可使后人复忿陛下也。”直指孙晧为桀、纣暴君,并料国家必亡。后果如所料,十馀年后吴为晋所亡。“今政荒民弊,覆亡是惧”,诤诤忠言,精贯日月。但昏君不觉悟,又将奈何!}

\lettrine{10.6} 何晏\myidx{何晏}、邓飏\myidx{邓飏}令管辂\myidx{管辂}作卦\footnote{何晏:字平叔。三国魏时南阳宛人。正始名士。官至吏部尚书。后为司马懿诛。参\CJKunderwave{言语}第14则注。邓飏:字玄茂。南阳宛人。邓禹之后,少得士名。至侍中、尚书。后为司马懿诛。参\CJKunderwave{识鉴}第3则注。管辂:魏时术数解\CJKunderwave{易}卦师。官至少府丞。},云:“不知位至三公不\footnote{三公:古时以太尉、司徒、司空为三公,领袖朝廷百官。}?”卦成,辂称引古义\footnote{古义:古代故事义理。},深以戒之。飏曰:“此老生之常谈\footnote{老生之常谈:喻毫无新义。老生,老书生。}。”{\fzxk\zihao{6}\textcolor{red}{\CJKunderwave{辂别传}曰:“辂字公明,平原人也。明\CJKunderwave{周易},声发徐州。冀州刺史裴徽举秀才,谓曰:‘何、邓二尚书,有经国才略,于物理无不精也。何尚书神明清彻,殆破秋毫,君当慎之!自言不解\CJKunderwave{易}中九事,必当相问。比至洛,宜善精其理。’辂曰:‘若九事比王义(“比王义”,袁本作“皆至义”),不足劳思。若阴阳者,精之久矣。’辂至洛阳,果为何尚书问九事,皆明。何曰:‘君论阴阳,此世无双也。’时邓尚书在,曰:‘此君善\CJKunderwave{易},而语初不论\CJKunderwave{易}中辞义,何邪?’辂答曰:‘夫善\CJKunderwave{易}者不论\CJKunderwave{易}也。’何尚书含笑赞之曰:‘可谓要言不烦也。’因谓辂曰:‘闻君非徒善论\CJKunderwave{易},至于分蓍思爻,亦为神妙。试为作一卦,知位当至三公不?又梦青蝇数十来𦤓(鼻)头上,驱之不去,有何意故?’辂曰:‘鸱,天下贱鸟也,及其在林,食其桑椹,则怀其好音。况辂心过草木,注情葵藿,敢不尽忠!唯察之尔。昔元、凯之相重华,宣慈惠和,仁义之至也。周公之翼成王,坐以待旦,敬慎之至也。故能流光六合,万国咸宁。然后据鼎足而登金铉,调阴阳而济兆民。此履道之休应,非卜筮之所明也。今君侯位重东岳,势若雷霆,望云赴景,万里驰风。而怀德者少,畏威者众,殆非小心翼翼多福之士。又𦤓(鼻)者,艮也,此天中之山,高而不危,所以长守贵也。今青蝇,臭恶之物,而集之焉。位峻者颠,轻豪者亡,必至之分也。夫变化虽相生,极则有害;虚满虽相受,溢则有竭。圣人见阴阳之性,明存亡之理,损益以为衰,抑进以为退。是故山在地中曰\CJKunderwave{谦},雷在天上曰\CJKunderwave{大壮}。谦则裒多益寡,大壮则非礼不履。伏愿君侯上寻文王六爻之旨,下思尼父彖象之义,则三公可决,青蝇可驱。’邓曰:‘此老生之常谈。’又曰:‘夫老生者见不生,常谈者见不谈也。’”}} 晏曰:“知几其神乎\footnote{知几其神乎:\CJKunderwave{易·系辞下}句。知几:掌握事物变化征兆。神,神妙。},古人以为难;交疏而吐诚\footnote{交疏而吐诚:交情疏远而言辞诚恳。},今人以为难。今君一面,尽二难之道,可谓‘明德惟馨’\footnote{明德惟馨:\CJKunderwave{尚书·君陈}句,意谓德义流芳。}。\CJKunderwave{诗}不云乎,‘中心藏之,何日忘之’\footnote{“中心藏之”二句:\CJKunderwave{诗经·小雅·隰桑}诗句,意谓牢记心中,永以为念。}。”{\fzxk\zihao{6}\textcolor{red}{\CJKunderwave{名士传}曰:“是时曹爽辅政,识者虑有危机。晏有重名,与魏姻戚,内虽怀忧,而无复退也。箸五言诗以言志曰:‘鸿鹄比翼游,群飞戏太清。常畏大网罗,忧祸一旦并。岂若集五湖,从流妾(唼)浮涩(萍)。永宁旷中怀,何为怵惕惊?’盖因辂言,惧而赋诗。”}}

{\cangkai\zihao{5}【评】何晏在魏文帝、明帝时,无所事任,或为冗官,颇受曹氏父子压抑。齐王芳正始年间,曹爽与司马懿辅政,晏、飏尝为爽之腹心,乃复进叙,任尚书要职。故事当发生在正始年间。时司马懿集团与曹魏集团明争暗斗,势同水火而决战在即。晏党曹魏,立场明显,故为司马集团所疾。管辂以\CJKunderwave{周易}算卦谏之,实是以学术为政治斗争作解。管辂旁观者清,对司马集团的韬晦示羸之智及其政治实力,有所洞察,故称引古义而“深以为戒”。邓飏贪墨,傲慢无理,故讥辂“老生常谈”,实不知时局之艰危。何晏反之,“知几其神”,思理明辨,洞其言微,但又无计相回避,故有“中心藏之,何日忘之”之叹。明王世贞评曰:“何晏悦而不绎,差胜邓飏无救败亡。”晏之“悦而不绎”,知而不行,关乎整个政局,客观形势如此,区区个人,何力回天,悲哉!}

\lettrine{10.7} 晋武帝\myidx{司马炎}既不悟太子\myidx{司马衷}之愚\footnote{晋武帝:晋朝第一代皇帝司马炎,字安世。昭长子。崩谥武皇帝,庙号世祖。参\CJKunderwave{言语}第19则注。太子:指司马衷,字正度。即位后史称惠帝。},必有传后意\footnote{传后意:意在传授帝位。},诸名臣亦多献直言。帝尝在陵云台上坐\footnote{陵云台:台名,在魏晋京城洛阳。},卫瓘\myidx{卫瓘}在侧\footnote{卫瓘:字伯玉,魏晋间河东安邑人。晋时官至尚书令。后为贾后及楚王诛杀。},欲申其怀,因如醉,跪帝前,以手抚床曰\footnote{床:坐榻。}:“此坐可惜!”帝虽悟,因笑曰:“公醉邪?”{\fzxk\zihao{6}\textcolor{red}{\CJKunderwave{晋阳秋}曰:“初,惠帝之为太子,咸谓不能亲政事,卫瓘每欲陈启废之而未敢也。后因会醉,遂跪床前曰:‘臣欲有所启。’帝曰:‘公所欲言者何邪?’瓘欲言而复止者三,因以手抚床曰:‘此坐可惜!’帝意乃悟,因谬曰:‘公真大醉也!’帝后悉召东宫官属大会,令左右赍尚书处事以示太子,令处决,太子不知所对。贾妃以问外人,代太子对,多引古词义。给使张弘曰:‘太子不学,陛下所知,宜以见事断,不宜引书也。’妃从之。弘具草奏,令太子书呈,帝大说,以示瓘。于是贾充语妃曰:‘卫瓘老奴,几败汝家!’妃由是怨瓘,后遂诛之。”}}

{\cangkai\zihao{5}【评】司马衷于泰始三年(267)立为太子,故事当发生于斯年之后。此则巧用行为配合的隐喻修辞艺术,故事生动,言简而意深,人物声吻、动作及其内在心理,无不如画呈现。在封建时代,建嗣立太子属国之大事,稍有不慎,卷入夺权斗争漩涡,常有死无葬身之地之患。卫瓘“如醉”而谏者以此。晋惠帝痴呆之愚,属低能儿。时天下荒乱,百姓饿死,谓“何不食肉糜?”其蒙蔽之愚如此,能不亡乎?即位后,史称“政出群下,纲纪大坏,货赂公行,势位之家,以贵陵物,忠贤路绝,谗邪得志”,不久即八王乱起,“五胡乱华”,国家沦丧。这虽是后事,但早在卫瓘料中,故抚帝床而叹:“此坐可惜!”“坐”谓帝座、帝位,不仅关系个人,更为国家社稷及天下苍生计。但封建帝王视国家为私人财产,武帝不惜,诸名臣又奈他何!后卫瓘因此被贾妃所杀,其智慧之谏,不仅白白浪费,更使后人复为之叹息也。}

\lettrine{10.8} 王夷甫\myidx{王衍}妇\footnote{王夷甫:王衍,字夷甫。参\CJKunderwave{言语}第23则注。},郭泰宁\myidx{郭豫}女,{\fzxk\zihao{6}\textcolor{red}{\CJKunderwave{晋诸公赞}曰:“郭豫字太宁,太原人。仕至相国参军。知名蚤卒。”}} 才拙而性刚,聚敛无厌,干豫人事\footnote{干豫:干预,干涉。}。夷甫患之而不能禁。时其乡人幽州刺史李阳\myidx{李阳}\footnote{幽州:汉十三刺史部之一,晋时州治涿县(今属河北省)。},京都大侠\footnote{京都:指魏晋京师洛阳。},{\fzxk\zihao{6}\textcolor{red}{\CJKunderwave{晋百官名}曰:“阳字景相,高平人。武帝时为幽州刺史。”\CJKunderwave{语林}曰:“阳性游侠,盛暑,一日诣数百家别,宾客与别,常填门,遂死于几下,故惧之。”}} 犹汉之楼护\myidx{楼护},{\fzxk\zihao{6}\textcolor{red}{\CJKunderwave{汉书·游侠传}曰:“护字君卿,齐人。学经传,甚得名誉。母死,送葬车三千两。仕至天水太寺(守)。”}} 郭氏惮之\footnote{惮:惧怕。}。夷甫骤谏之\footnote{骤谏:屡次言语劝阻。},乃曰:“非但我言卿不可,李阳亦谓卿不可\footnote{谓:以为。}。”郭氏小为之损\footnote{小:稍微。 损:减损,收敛。}。

{\cangkai\zihao{5}【评】琅邪王衍,出身望族,官至太尉,位居宰辅,总领群臣,但无法约束自己夫人之不法,却是为何?史称“衍妻郭氏,贾后之亲,藉中宫之势,刚愎贪戾,聚敛无厌”云云,可证王衍之惧内,非本性如此,而是畏权惧势也。人畏权势,故乏謇谔忠节。一旦大军压境,刀架头上,又岂能不变节?临死前衍劝石勒称尊号,即为明证。惧内事小,但因小见大,可推其本末。宰辅如此,晋之败丧,亦在料中。}

\lettrine{10.9} 王夷甫\myidx{王衍}雅尚玄远\footnote{王夷甫:王衍。雅尚:崇尚。玄远:玄虚远俗的精神境界。},常嫉其妇贪浊\footnote{嫉:厌恶,讨厌。贪浊:贪婪污浊。},口未尝言“钱”字。{\fzxk\zihao{6}\textcolor{red}{\CJKunderwave{晋阳秋}曰:“夷甫善施舍,父时有假贷者,皆与焚券,未尝谋货利之事。”王隐\CJKunderwave{晋书}曰:“夷甫求富贵得富贵,资财山积,用不能消,安须问钱乎?而世以不问为高,不亦惑乎!”}} 妇欲试之,令婢以钱绕床,不得行。夷甫晨起,见钱阂行\footnote{阂(hé合):碍。},呼婢曰:“举却阿堵物\footnote{举却:拿走,搬开。阿堵物:这个东西。后引申喻钱。}!”

{\cangkai\zihao{5}【评】在\CJKunderwave{世说}故事中,王衍是主角,而在清谈玄家中,更是主角中的主角。在魏晋名士中,的确有人忘却高官厚禄的物质诱惑,一心追求率性自然、超凡脱俗的精神生活。但王衍不属此类,在国家多事之秋,营狡兔三窟之计,而忘杀身报国之仁,其所关注,正在一己私利之物欲。细加推敲,知其为人,能言善辩,却无实际内容,仅一只“绣花枕头”而已。又要做士林领袖,就必须养就一身做“秀”的本领。口不言钱,似乎高雅之极,但在专制社会中,权就是钱。明王世贞之评,断言“王隐此言非也”,认为人性“廉贪不系贫富”。此乃泛泛之论;若具体衡量王衍,鄙意乃以王隐为是,因王世贞忘记了王衍是个善于做“秀”的人物。故宋刘辰翁评曰:“但意不在钱,言钱何害?”一针见血,见识不凡。}

\lettrine{10.10} 王平子\myidx{王澄}年十四五\footnote{王平子:王澄,字平子。衍弟。西晋清谈名家。参\CJKunderwave{德行}第23则注。},见王夷甫妻郭氏贪欲\footnote{贪欲:贪婪。},令婢路上儋粪\footnote{儋粪:挑粪。儋,通“担(担)”。}。平子谏之,并言不可。郭大怒,谓平子曰:“昔夫人临终\footnote{夫人:指澄母,即郭氏之婆母。},以小郎嘱新妇\footnote{小郎:小叔子。嘱:嘱托,交代。新妇:魏晋已婚妇女自称。},不以新妇嘱小郎。”{\fzxk\zihao{6}\textcolor{red}{\CJKunderwave{永嘉流人名}曰:“澄父乂,第三取乐安任氏女,生澄。”}} 急捉衣裾\footnote{衣裾:衣襟。},将与杖\footnote{与杖:杖责,打棍子。}。平子饶力\footnote{饶力:多力,力气大。},争得脱,踰窗而走\footnote{踰窗而走:跳窗逃走。}。

{\cangkai\zihao{5}【评】王澄生于晋武帝泰始三年(267),顺推十五年,则故事发生在武帝太康二年,是平吴后的次年,时天下一统,国力大增。处此歌舞升平的繁荣时期,一个琅邪王家的贵妇人,却贪蝇头小利,公然令婢女路上挑粪,实在有损世家望族的贵族颜面。澄谏以此。但郭氏倚伏皇亲之势,丈夫尚“不能禁”,更何况是尚未成年的小叔子。故事虽短,却是有矛盾,有情节,跌宕起伏,令人眼花缭乱。大怒,痛骂,捉衣,与杖,连续动作干脆利落,一个凶悍泼妇的形象,呼之欲活。}

\lettrine{10.11} 元帝\myidx{司马睿}过江犹好酒\footnote{元帝:指晋元帝司马睿,字景文。东晋开国第一位皇帝。参\CJKunderwave{言语}第29则注。江:长江。},王茂弘\myidx{王导}与帝有旧\footnote{王茂弘:王导字茂弘。参\CJKunderwave{德行}第27则注。旧:老交情。},常流涕谏。帝许之\footnote{许:答允。},命酌酒一酣\footnote{酌酒一酣:唐写本“一酣”作“一唾”,周祖谟引敬胤注曰:“旧云酌酒一喢,因覆杯写(泻)地,遂断也。”据此,则唐写本“一唾”为“一喢”之形讹。“喢”通“歃”。酌酒一喢,酌酒泻地以盟誓。},从是遂断\footnote{断:戒酒。}。{\fzxk\zihao{6}\textcolor{red}{邓粲\CJKunderwave{晋纪}曰:“上身服俭约,以先时务。性素好酒,将渡江,王导深以谏。帝乃令左右进觞,饮而覆之,自是遂不复饮。克己复礼,官修其方,而中兴之业隆焉。”}}

{\cangkai\zihao{5}【评】元帝创业之始,举贤授能而从谏如流,故能龙兴江东。嗜酒贪杯,原是个人生活爱好,无足深责。但作为一国之君,贪杯废务,则关系国计民生,并非小事。如陆凯之谏孙晧云:“夫酒以成礼,过则败德,此无异商辛长夜之饮也。”殷纣王湎首酒池,奢靡荒淫,自丧其国。王导以此泣谏,目光深远。元帝喢酒盟誓而遂断,正见其复国之决心。君明臣贤,鱼水相谐,故有东晋之中兴。一般人戒酒,记载与否,无足轻重;但此乃帝王之戒,“酌酒一喢”而“遂断”,则故事有致,而意义深远。}

\lettrine{10.12} 谢鲲\myidx{谢鲲}为豫章太守\footnote{谢鲲:字幼舆,陈郡阳夏人。官豫章太守。参\CJKunderwave{文学}第20则注。豫章:郡治在南昌(今属江西)。},从大将军\myidx{王敦}下至石头\footnote{大将军:指王敦。参\CJKunderwave{文学}第20则注。石头:城名,在建康西,是捍卫京师的军事重镇。}。敦谓鲲曰:“余不得复为盛德之事矣!”鲲曰:“何为其然\footnote{何为其然:为什么这样说呢?}?但使自今已后,日亡日去耳\footnote{日亡日去:意谓时间流逝,冲淡昔日嫌隙而遗忘之。}。”{\fzxk\zihao{6}\textcolor{red}{\CJKunderwave{鲲别传}曰:“鲲之讽切雅正,皆此类也。”}} 敦又称疾不朝\footnote{不朝:不上朝觐见皇上。},鲲谕敦曰:“近者,明公之举\footnote{明公:尊称在上位者,这里指王敦。},虽欲大存社稷,然四海之内,实怀未达\footnote{实怀未达:内心未能理解。}。若能朝天子,使群臣释然,万物之心于是乃服\footnote{万物之心:喻万民心思。}。仗民望以从众怀,尽冲退以奉主上\footnote{“仗民望以从众怀”二句:意谓随顺民意而谦虚奉君。},如斯,则勋侔一匡\footnote{勋侔一匡:意谓功劳与管仲相似。史称管仲辅助齐桓公,“霸诸侯,一匡天下”(\CJKunderwave{论语·宪问})。侔,等同。匡,匡正。},名垂千载。”时人以为名言。{\fzxk\zihao{6}\textcolor{red}{\CJKunderwave{晋阳秋}曰:“鲲为豫章太守,王敦将肆逆,以鲲有时望,逼与俱行。既克京邑,将旋武昌,鲲曰:‘不就朝觐,鲲惧天下私议也。’敦曰:‘君能保无变乎?’对曰:‘鲲近日入觐,主上侧席,迟得见公,宫省穆然,义无不虞之虑。公若入朝,鲲请侍从。’敦曰:‘正复杀君等数百,何损于时!’遂不朝而去。”}}

{\cangkai\zihao{5}【评】故事发生在元帝永昌元年(322),大将军王敦以清君侧为名,起兵武昌,师指建康,四月破石头城,朝廷溃败。这是东晋初建不久的一场大规模的叛乱,不久元帝忧患而崩。这是一篇以对话叙事为特点的故事。“余不得复为盛德之事矣!”王敦不臣之心,溢于言表。这正是谢鲲所忧虑的。他是当日清谈名家,虽任诞作达,却颇孚时望,故王敦持之东下,以收士心。王敦将叛,谢鲲再三讽谏,企望挽狂澜于既倒,但终无济于事,其苦口婆心之心血,最后化为泡影幻灭。事虽不可为,但其忠言嘉谋,却是精诚感人,义薄云天。“仗民望以从众怀,尽冲退以奉主上”,虽尽忠国事,却无力回天,但确是千古名言。}

\lettrine{10.13} 元皇帝\myidx{司马睿}时\footnote{元皇帝:晋元帝司马睿。},廷尉张闿\myidx{张闿}{\fzxk\zihao{6}\textcolor{red}{葛洪\CJKunderwave{富民塘颂}曰:“闿字敬绪,丹阳人,张昭孙也。”\CJKunderwave{中兴书}曰:“闿,晋陵内史,甚有威德,转至廷尉卿。”}} 在小市居\footnote{廷尉:朝廷中掌治安刑狱之官。张闿:因平苏峻乱,功封宜阳伯,转廷尉卿。小市:都城中贸易集中之地,因其规模而有大市与小市之别。},私作都门\footnote{都门:都中里门。},早闭晚开,群小患之\footnote{群小:喻百姓。}。诣州府诉,不得理\footnote{理:审理。};遂至挝登闻鼓\footnote{挝:敲击。登闻鼓:朝堂府衙前鸣冤上诉之鼓。},犹不被判\footnote{判:判决。}。闻贺司空\myidx{贺循}出\footnote{贺司空:贺循官太常卿,卒赠司空,故称。},至破冈\footnote{破冈:水渠名,即破冈渎,在句容县南。},连名诣贺诉。{\fzxk\zihao{6}\textcolor{red}{\CJKunderwave{贺循别传}曰:“循字彦先,会稽山阴人。本姓庆,高祖纯避汉帝讳,改为贺氏。父劭,吴中书令,以忠正见害。循少婴家祸,流放荒裔,吴平乃还。秉节高举,元帝为安东王,循为吴国内史。”}} 贺曰:“身被征作礼官,不关此事。”群小叩头,曰:“若府君复不见治\footnote{治:治理。},便无所诉。”贺未语,令且去:“见张廷尉当为及之。”张闻,即毁门,自至方山迎贺\footnote{方山:地名,在江宁县东南,时为交通要道。}。贺出见,辞之\footnote{出见辞之:唐写本作“出辞见之”,是,意谓贺把百姓诉辞拿给张看。},曰:“此不必见关\footnote{见关:与我相关。},但与君门情\footnote{门情:通家世家之情谊。},相为惜之。”张愧谢曰\footnote{谢:愧谢,谢罪。}:“小人有如此,始不即知,早已毁坏。”

{\cangkai\zihao{5}【评】故事发生在东晋初建不久。\CJKunderwave{晋书·贺循传}言及此事缘由:“廷尉张闿住在小市,将夺左右近宅以广其居,乃私作都门,早闭晏(晚)开,人多患之,讼于州府,皆不见省。”张闿之心,在于抢夺民宅以广己居,作为廷尉,执法犯法,但却官官相护而不见省,老百姓连上诉的地方都没有,于是只有求助于清官个人了。以制度论,贺循非执法官吏,无权干预诉讼。但他却巧妙地动之以情,以世交“门情”来打动张闿,使问题终于获得圆满解决。张之“愧谢”,说明贺之人格,为人敬服。但问题不是依法治理,而是循情以决,却也说明了专制社会所留下的无穷祸患,至今难绝。}

\lettrine{10.14} 郗太尉\myidx{郗鉴}晚节好谈\footnote{郗太尉:郗鉴字道徽,高平金乡人。官至太尉,故称。参\CJKunderwave{德行}第24则注。好谈:喜欢谈论。},既雅非所经\footnote{雅非所经:不是他平素所长。雅,素来。经,擅长。},而甚矜之\footnote{矜:矜持,自负。}。{\fzxk\zihao{6}\textcolor{red}{\CJKunderwave{中兴书}曰:“鉴少好学博览,虽不及章句,而多所通综。”}} 后朝觐\footnote{朝觐:晋见皇帝。},以王丞相\myidx{王导}末年多可恨\footnote{王丞相:王导。以下“王公”,同指王导。},每见,必欲苦相规诫。王公知其意,每引作佗言\footnote{佗:同“他”。}。临还镇\footnote{还镇:返回军镇之所。},故命驾诣丞相\footnote{故:特意。},丞相翘须厉色\footnote{丞相翘须厉色:唐写本无“丞相”二字,是。考其主语,翘须厉色者,当是郗鉴,“丞相”二字,承上而衍。},上坐便言:“方当乖别\footnote{方当乖别:将要离别。},必欲言其所见\footnote{必欲:一定要。}。”意满口重\footnote{意满口重:气盛言重。},辞殊不流\footnote{辞殊不流:说话不流畅。}。王公摄其次\footnote{摄其次:及时抓紧时机。},曰:“后面未期\footnote{后面未期:后会不知何时。},亦欲尽所怀\footnote{欲尽所怀:希望能开怀畅谈。},愿公勿复谈。”郗遂大瞋\footnote{瞋:生气,怒。},冰衿而出\footnote{冰衿:唐写本作“冰矜”,是。冰矜,脸色冷若冰霜,而有矜奋之容。},不得一言。

{\cangkai\zihao{5}【评】在政治上,郗鉴与王导同党同心。故事当发生在晋成帝咸康初年,时庾亮代陶侃任荆州刺史,掌控长江中上游诸军事。继陶侃欲起兵废导,庾亮“又欲率众黜导”,以此咨鉴,鉴不许而止。鉴时任车骑将军、都督徐兖青三州军事、兖州刺史,镇广陵,后又加徐州刺史,镇京口,权任甚重,可抗衡庾亮,制止废导之谋。在治国方略上,王导行道玄无为之治,取镇静之说;外戚庾氏(亮、冰等)为政则“任法裁物”,“颇任威刑”,以此失人心。但当时庾太后临朝,政事一决于庾氏,王导虽贵为丞相,也只能受制庾氏,“正封箓诺之”。王导晚年,实权已失,地位岌岌可危,不愦愦又将如何?鉴之性格与导异,是个知其不可为而为之的忠义之士,故以导晚年为恨而欲强谏之。但导综观全局,明知其无可奈何,故干脆“先发制人”,巧妙地剥夺了郗鉴的发言机会。故事中“翘须厉色”、“意满口重”、“冰矜而出”,几个典型细节,生动地刻画了一个爱国老帅的内心世界。而王导之“摄其次”,终令老帅不得一言,又见其心知肚明的智者形象。作者写来,郑重可怀,其叙事情状及人物对话,生动如画。}

\lettrine{10.15} 王丞相\myidx{诣}为杨(扬)州\footnote{王丞相:王导。为扬州:任扬州刺史。},遣八部从事之职\footnote{八部从事:扬州下辖八郡:丹阳、会稽、吴、宣城、吴兴、东阳、临海、新安。每郡设部从事一人,直属刺史,督察属郡。}。顾和\myidx{顾和}时为下传还\footnote{顾和:字君孝,吴郡人。顾荣族子。参\CJKunderwave{言语}第33则注。下传还:作为使者乘驿车下郡视察返回州府。},同时俱见,诸从事各奏二千石官长得失\footnote{二千石:指郡守。},至和独无言。王问顾曰:“卿何所闻?”答曰:“明公作辅\footnote{明公:对王导的敬称。辅:宰辅。},宁使网漏吞舟\footnote{网漏吞舟:渔网漏掉吞舟之鱼,以喻法网宽大。},何缘采听风闻\footnote{风闻:不可靠的传闻。},以为察察之政\footnote{察察之政:苛酷琐细之政。}?”丞相咨嗟称佳\footnote{咨嗟:叹赏。},诸从事自视缺然也\footnote{自视缺然:自感缺失而有所不如。}。

{\cangkai\zihao{5}【评】故事发生在东晋草创而王导任扬州刺史之时,扬州是东晋的京畿地区,刺史职权极其重要。派遣八部从事巡察所部诸郡,应是行使职权的表现。但汇报之时,情况却出人意料:汇报者“自视缺然”;而顾和“独无言”,即无所汇报,却独获王导的“咨嗟称佳”。这是为什么?明王世懋不解而评曰:“如此,何遣从事为?”他不明故事发生的历史环境,以及王导施政的良苦用心,故所问未能明于言外之理。东晋之初,江东士民骚然,元帝欲行法家之政,建康街头犹如刑场,血流飘杵,故郭璞借\CJKunderwave{易}卦占筮上疏谏之。王导辅政,则极力扭转这一不利团结建国的倾向,而以道家无为镇静、顺应自然相规劝。作为刺史,遣八部从事之职,是例行公事;但听汇报时,他不爱听好言人失的小报告,若是专伺“见闻”,无异于严酷的特务统治,部属又将如何行使职权?缺乏下属士民的支持拥护,国家又怎能安定团结?顾和独受表扬,正见王导从全局出发的政治家胸怀。}

\lettrine{10.16} 苏峻\myidx{苏峻}东征沈充\myidx{沈充}\footnote{苏峻:字子高。因讨王敦、征沈充之功封公爵。官历阳太守,拥兵自重而反。后被陶侃、温峤联军击败,斩于阵前。沈充:吴兴豪族,王敦叛乱谋主。},{\fzxk\zihao{6}\textcolor{red}{\CJKunderwave{晋阳秋}曰:“充字士居,吴兴人。少好兵,謟事王敦。敦克京邑,以充为车骑将军、领吴国内史。明帝伐王敦,充率众就王含,谓其妻曰:‘男儿不建豹尾,不复归矣!’”}} 敦死,充将吴儒斩首于京都。请吏部郎陆迈\myidx{陆迈}与俱\footnote{吏部郎:吏部属官,主管官吏选拔。}。{\fzxk\zihao{6}\textcolor{red}{陆碑曰:“迈字功高,吴郡人。器识清敏,风检澄峻。累迁振威太守、尚书吏部郎。”}} 将至吴\footnote{吴:吴郡(今江苏苏州)。},密敕左右\footnote{敕:命令。},令入阊门放火以示威。陆知其意,谓峻曰:“吴治平未久\footnote{吴治平未久:自东吴孙晧降晋至晋明帝太宁初年,四十馀年。},必将有乱,若为乱阶\footnote{乱阶:祸端,祸乱来由。},请从我家始。”峻遂止。

{\cangkai\zihao{5}【评】有学者据此断言“知苏峻有反意”,余谓不然。关键在于时间,苏峻反叛有其过程。故事发生在晋明帝太宁二年(324),当时苏峻不仅未反,而且是个忠于国事的将军。\CJKunderwave{晋书}峻传称,王敦曾遣人说峻曰:“富贵可坐取,何为自来送死?”峻不从,遂率众赴京师,大败贼兵。后又奉命率军东征沈充。攻吴之时,正在东征途中,仍为朝廷作战,岂有反意?其密令入阊门放火示威,激民愤而乱贼志,乘乱攻贼,则敌之败亡可立待也。这是具体战术问题,但主意并不高明。水火无情,伤害毁灭的是人民及其财产,可说是“一将功成万骨枯”。陆迈谏止,救民于水火之中,意义在此。苏峻功成之后,日渐骄横,但作为北来的流民帅,却一直不被朝廷中诸姓衮衮诸公所信任,最后又因执政庾亮处置不当,被逼走上了反叛不归之路,虽为事实,却是后来之事,不可与此混为一谈。}

\lettrine{10.17} 陆玩\myidx{陆玩}拜司空\footnote{陆玩:字士瑶。吴人。参\CJKunderwave{政事}第13则注。司空:官名,朝廷三公之一。},{\fzxk\zihao{6}\textcolor{red}{\CJKunderwave{玩别传}曰:“是时王导、郗鉴、庾亮相继薨殂,朝野忧惧,以玩德望,乃拜司空。玩辞让不获,乃叹息谓朋友曰:‘以我为三公,是天下无人矣。’时人以为知言。”}} 有人诣之\footnote{诣:拜访。},索美酒,得便自起,泻箸梁柱间地,祝曰\footnote{祝:祷辞。}:“当今乏才,以尔为柱石之用\footnote{柱石:房柱下之基石。},莫倾人栋梁。”玩笑曰:“戢卿良箴\footnote{戢:藏,引申为牢记。箴:箴言,规诫之言。}。”

{\cangkai\zihao{5}【评】魏晋之时,行九品中正官人法,士庶之别,天渊之隔。而在高门士族之中,又有南、北之别,中原之士与江南之士,有时势同水火,矛盾尖锐。这一形势发展到东晋,虽因中原士族南渡立国的“统战”需要,有所缓和,但在潜意识深处,仍是根深蒂固,时有爆发。王导辅政,为争取江南士民的支持,曾多次对陆玩示好,尽力争取吴郡陆氏家族的支持。但时东晋草创,陆玩对中原士人的友好表示,半信半疑,如对王导提出的子女联姻的要求,以“薰莸不同器”予以婉拒(\CJKunderwave{方正}第24则)。但经过多次考验,最终明白王导的真诚,因而与兄陆晔一起,“事君如父,忧国如家”,维护了国家的统一和安定团结。面对中原士人“倾人栋梁”的挑衅嘲讽,陆玩答辞不亢不卑,既不张狂,又不畏缩,而是勇敢肩负重担,既对国家和民族负责,同时也表明了自己继承王导那泯灭士族南北歧见的团结路线的决心。故王世贞评曰:“即此是亦可作司空。”宰相肚量,的确非同一般。}

\lettrine{10.18} 小庾\myidx{庾翼}在荆州\footnote{小庾:指庾翼,字稚恭,亮弟。颍川鄢陵人。时接替兄亮任荆州刺史。参\CJKunderwave{言语}第53则注。},公朝大会\footnote{公朝大会:府衙官吏大聚会。},问诸僚佐曰\footnote{僚佐:僚属辅佐的官吏。}:“我欲为汉高、魏武\footnote{汉高:汉高祖刘邦,创汉朝数百年基业。魏武:曹操,为子孙开魏朝帝业。子丕篡汉建魏后,尊为武皇帝,故称。},何如?”{\fzxk\zihao{6}\textcolor{red}{翼别见。宋明帝\CJKunderwave{文章志}曰:“庾翼名辈,岂应狂涓(狷)如此哉!若有斯言,亦传闻者之谬矣。”}} 一坐莫答。长史江霦(虨)\myidx{江虨}曰\footnote{长史:官名,朝廷丞相、三公及军督府衙的重要僚佐。江霦(虨)(bīn彬):字思玄,陈留人。统子。官至尚书左仆射、护军将军、领国子祭酒。参\CJKunderwave{方正}第42则注。}:“愿明公为桓、文之事\footnote{明公:指庾翼。桓、文:指春秋五霸中的齐桓公、晋文公。},不愿作汉高、魏武也。”

{\cangkai\zihao{5}【评】明帝咸康六年(340)庾亮卒,弟翼代其任荆州刺史、安西将军、都督江荆司雍梁益六州诸军事,镇武昌。故事当发生于是年之后。作为臣子而公言“我欲为汉高、魏武”,曹操篡汉奸相,刘邦开汉先帝,欲为汉高魏武之事业,言外即篡弑夺国之叛逆。公朝大会,如此明目张胆,虽王敦、桓温不臣枭雄尚无是言,更何况是一贯勤于王事而忠心国家之庾翼乎?\CJKunderwave{豪爽}第13则谓“庾稚恭既常有中原之志”,史称“翼雅有大志,欲以灭胡平蜀为己任,言论慷慨,形于辞色”,皆可为证。刘注引宋明帝言以辩其诬,甚是。当时为何有此谬传?鄙见以为与当时庾亮死后,作为东晋四大家族之一的外戚世家鄢陵庾氏,已从权力巅峰开始下滑,这是其他士族政敌造谣,阴谋逼迫庾氏交出权力。为了权力,借助流言,如此无耻,令人齿寒。}

\lettrine{10.19} 罗君章\myidx{罗含}为桓宣武\myidx{桓温}从事\footnote{罗君章:罗含,字君章,桂阳耒阳人。官至侍中、廷尉、长沙相。参\CJKunderwave{方正}第56则注。桓宣武:桓温卒谥宣武,故称。从事:此指部从事,州府属官。},{\fzxk\zihao{6}\textcolor{red}{\CJKunderwave{含别传}曰:“刺史庾亮初命含为部从事,桓温临州,转参军。”}} 谢镇西作江夏\footnote{谢镇西:谢尚,字仁祖,父鲲。曾任镇西将军,故称。时任江夏相,属荆州府辖。},往检校之\footnote{检校:检查校核。}。{\fzxk\zihao{6}\textcolor{red}{\CJKunderwave{中兴书}曰:“尚为建武将军、江夏相。”}} 罗既至,初不问郡事\footnote{初不:完全不。},径就谢数日饮酒而还\footnote{径:直接。}。桓公问:“有何事?”君章云:“不审公谓谢尚\myidx{谢尚}何似人\footnote{不审:不知。公:指桓温。谓:认为。何似人:怎样的人。}?”桓公曰:“仁祖是胜我许人\footnote{我许:我辈。}。”君章云:“岂有胜公人而行非者?故一无所问。”桓公奇其意而不责也。

{\cangkai\zihao{5}【评】故事写的是罗含、谢尚与桓温三人的交往与友谊,当发生在桓温代庾翼镇荆州的穆帝永和元年(345),时罗含仍为部从事,不久即被桓温转为参军。当时地方政府,州府下辖若干郡,每郡设郡从事一人,直属刺史,代其督察属郡。时罗含尚未转官而仍为部从事,督察江夏郡。而江夏相谢尚,出身陈郡谢氏家族,是个知名士人。谢尚与罗含为方外之好,尚称含为“湘中之琳琅”。含与尚惺惺相惜,其检校江夏,“径就谢数日饮酒而还”,只叙友情,而不问郡事。而桓温于谢尚,也是称赏不置,曾上表朝廷称赞云:“谢尚神怀挺率,少致民誉。”而对于罗含,桓温美之“江左之秀”(见\CJKunderwave{晋书·罗含传}),颇为赏识。其派含检校谢尚,不过是例行公事,是走形式,故于含“不问郡事”而不责也。含对三人关系,心里明白,故化被动为主动,“岂有胜公而行非者?”既称赞上司英明,同时誉谢尚之非同凡响,可称一石二鸟,皆大欢喜。}

\lettrine{10.20} 王右军\myidx{王羲之}与王敬仁\myidx{王修}、许玄度\myidx{许询}并善\footnote{王右军:王羲之曾任右军将军,故称。王敬仁:王修字敬仁,小字苟子,太原晋阳人。濛子。少有美称,善隶行书,号“流奕清举”。任琅邪王文学,转中军司马,年二十四卒。 许玄度:许询字玄度,清谈玄家。参\CJKunderwave{言语}第69则注。},二人亡后,右军为论议更剋\footnote{论议:评论。更剋:变为苛刻。}。孔岩\myidx{孔岩}诫之曰\footnote{孔岩:字彭祖,会稽山阴人。官丹阳尹、吴兴太守。}:“明府昔与王、许周旋有情\footnote{周旋:交往。},及逝没之后,无慎终之好\footnote{无慎终之好:不能善始善终。},民所不取\footnote{民:孔岩为会稽山阴人,时王羲之任会稽内史,故尊之为明府,自谦称“民”。}。”右军甚愧。

{\cangkai\zihao{5}【评】故事当发生在羲之任会稽内史期间,具体在永和九年(353)至十一年(355)之间,因永和九年羲之作\CJKunderwave{兰亭序},许询同游;而十一年,他与扬州刺史王述闹矛盾,誓墓挂冠,此则孔岩称“民”,则羲之尚在任内,故下限在离任之前。王修、许询并有高名,但才华未获充分展现,即英年早逝。羲之不惜,反而多有讥贬而“论议更剋”。时羲之晚年,思想定型,正说明琅邪王氏簪缨世家贵族的傲慢与偏见,根深蒂固,伤人感情。贤如右军,仍不免俗,故有孔岩“慎终”之诫,如王世贞所评:“此规大有益于交道。”篇末“右军甚愧”,知错能改,则又恢复人性而无损其贤名。后人于此,能无思乎!}

\lettrine{10.21} 谢中郎\myidx{谢万}在寿春败\footnote{谢中郎:谢万,字万石。安弟。曾任豫州刺史、西中郎将,故称。参\CJKunderwave{言语}第77则注。寿春:县名,属淮南郡,今安徽寿县。},临奔走,犹求玉帖镫\footnote{玉帖镫:玉饰马镫。帖,同“贴”。}。太傅\myidx{谢安}在军\footnote{太傅:指谢安,卒赠太傅,故称。},前后初无损益之言\footnote{初无:毫无,从无。}。尔日犹云\footnote{尔日:这一日。}:“当今须烦此\footnote{须烦此:唐写本作“岂复烦此”,袁本作“岂须烦此”,语义更明。}!”{\fzxk\zihao{6}\textcolor{red}{案:万未死之前,安犹未仕,高卧东山,又何肯轻入军旅邪?\CJKunderwave{世说}此言,迂谬已甚。}}

{\cangkai\zihao{5}【评】故事发生在穆帝升平二年(358),豫州刺史监司豫冀并四州军事谢万,受命北征败归之时。谢万出于陈郡阳夏谢氏家族,门第高贵又早著时誉,是个浮华空谈的贵游子弟,其恃才傲物之狂,世罕其匹。故万衔命北征之时,王羲之与桓温笺,谏朝廷所用违才,并料其必败。其败归之时,“犹求玉帖镫”,生活仍然奢侈豪华,讲究排场,连马镫也必须用玉装饰,身份不肯稍降,完全不想自己作为败军之将给国家带来的屈辱和破坏。故谢安有“当今(岂)须烦此”之诫。但“太傅在军,前后初无损益之言”,则非实之辞。\CJKunderwave{晋书}万传称,安“深忧之,自队主将帅已下,安无不慰勉。谓万曰:‘汝为元帅,诸将宜数接对,以悦其心,岂有慠诞若斯而能济事也!’”万拒谏而败,废为庶人,郁郁而终,可谓咎由自取,为贵族的傲慢付出了惨重的代价。}

\lettrine{10.22} 王大\myidx{王忱}语东亭\myidx{王珣}\footnote{王大:王忱字文达,小字佛大,坦之子。官至荆州刺史。参\CJKunderwave{德行}第44则注。东亭:王珣,字法护。导孙。爵东亭侯,故称。参\CJKunderwave{言语}第102则注。}:“卿乃复论成不恶\footnote{乃复:竟然。论成不恶:评价不错,声名不赖。论成,犹定评。},那得与僧弥戏\footnote{僧弥:王珉字季琰,小字僧弥,珣弟。 戏:开玩笑,挑逗。}!”{\fzxk\zihao{6}\textcolor{red}{\CJKunderwave{续晋阳秋}曰:“珉有隽才,与兄珣并有名,而声出珣。故时人为之语曰:‘法护非不佳,僧弥难为兄。’”}}

{\cangkai\zihao{5}【评】王忱出于太原王氏,珣、珉兄弟出于琅邪王氏,俱是东晋高门士族。魏晋士人重声名,不仅是指官爵地位方面的政治才干,更重要的是精神品质方面的修养。在东晋中晚期,王忱与王珣、王珉兄弟并有声名美誉。若从政治才干及其业绩看,王珣早有“黑头公”的美称,官至尚书令。三人中成就最大,但王忱并不看重。珣著名于世,如桓玄所称,是因其“神情朗悟,经史明彻,风流之美,公私所寄”(见\CJKunderwave{晋书}珣传),重在精神品格之美。在这方面,弟珉“名出珣右”,在艺术化的审美人生,以及清谈论议的理论思辨方面,与乃兄相较,珉悟性更高。兄弟同听提婆讲\CJKunderwave{毗昙经},讲未半,珉已解,即是明证。王忱语珣曰:“那得与僧弥戏!”劝王珣不要挑战乃弟,正说明当时士人所重视是对于精神生活的追求。}

\lettrine{10.23} 殷觊(顗)\myidx{殷顗}病因(困)\footnote{殷觊:\CJKunderwave{晋书}本传作“殷顗”,字伯通,陈郡人。时任南蛮校尉。参\CJKunderwave{德行}第41则注。病因:唐写本作“病困”,是。病困,病重也。},看人政见半面\footnote{政:通“正”,只。}。殷荆州\myidx{殷仲堪}兴晋阳之甲\footnote{殷荆州:殷仲堪时任荆州刺史,故称。兴晋阳之甲:起兵以清君侧。甲,甲兵。晋阳之甲,参刘注。},{\fzxk\zihao{6}\textcolor{red}{\CJKunderwave{春秋·公羊传}曰:“晋赵鞅取晋阳之甲,以逐荀寅、士吉射;寅、吉射者,君侧之恶人。”}} 往与觊(顗)别,涕零属以消息所患\footnote{属:嘱付。消息所患:将养病体。消息,调养。}。觊(顗)答曰:“我病自当差\footnote{差:通“瘥”,痊愈。},正忧汝患耳\footnote{正:只。}。”{\fzxk\zihao{6}\textcolor{red}{\CJKunderwave{晋安帝纪}曰:“殷仲堪举兵,觊(顗)弗与同,且以己居小任,唯当守局而已,晋阳之事,非所宜豫也。仲堪每邀之,觊(顗)辄曰:‘吾进不敢同,退不敢异。’遂以忧卒。”}}

{\cangkai\zihao{5}【评】\CJKunderwave{晋书}本传称殷顗“性通率,有才气,少与从弟仲堪俱知名”。实际上,当时殷仲堪作为荆州刺史,是殷顗的上司,并且获孝武帝宠信,其“能清言,善属文”的名声更大。但当时的太子少傅王雅曾在孝武帝前批评其无当世之才,不可大任,断言云:“仲堪虽谨于细行,以文义著称,亦无弘量,且干略不长。……若道不常隆,必为乱阶矣。”(见\CJKunderwave{晋书}雅传)在其赴荆州藩屏之任前,早已料其败丧。殷顗虽与仲堪同一家族至亲,但一忠于国事,一则利己谋私,政治品格相互乖背。顗答仲堪之问,不忧己病,而“正忧汝患”,正是为国为家而尽最后之忠谏。惜仲堪为私利蒙蔽眼睛而逞其野心,常怀成败之计,寡谋不断,故很快被桓玄击杀,顗言不幸言中。}

\lettrine{10.24} 远公\myidx{慧远}在庐山中\footnote{远公:即慧远(334—416),东晋名僧,“公”是敬称。俗姓贾,雁门楼烦人。世为冠族。后师释道安。参\CJKunderwave{文学}第61则注。庐山:山名,在今江西九江市南。},{\fzxk\zihao{6}\textcolor{red}{\CJKunderwave{豫章旧志}曰:“庐俗字君孝,本姓匡,夏禹苗裔东野王之子。秦末,百越君长与吴芮助汉定天下,野王亡军中,汉八年,封俗鄢阳男,食邑兹部,印曰‘庐君’。俗兄弟七人,皆好道术,遂寓于洞庭之山,故世谓庐山。孝武元封五年,南巡狩,浮江,亲睹神灵,乃封俗为大明公,四时秩祭焉。”远法师\CJKunderwave{庐山记}曰:“山在江州寻阳郡,左侠(挟)彭泽,右傍通川。有匡俗先生出自殷、周之际,遁世隐时,潜居其下。或云:匡俗受道于仙人,而共游其岭,遂室崖岫,即岩成馆,故时人谓为神仙之庐而命焉。”法师\CJKunderwave{游山记}曰:“自托此山,二十三载,再践石门,四游南岭。东望香炉峰,北眺九江,传闻有石井、方湖,中有赤鳞踊出。野人不能叙,直叹其奇而已矣。”}} 虽老,讲论不辍\footnote{不辍:不停止。}。弟子中或有堕者\footnote{堕:通“惰”,怠惰。},远公曰:“桑榆之光\footnote{桑榆之光:太阳馀晖落在桑树、榆树之上。喻已入人生暮年。},理无远照;但愿朝阳之晖\footnote{朝阳之晖:远公借以喻弟子的青春年华。},与时并明耳。”执经登坐,讽诵朗畅\footnote{朗畅:爽朗流畅。},词色甚苦\footnote{词色甚苦:言辞恳切。苦,努力,恳切。}。高足之徒,皆肃然增敬。

{\cangkai\zihao{5}【评】这是慧远在其主持的庐山东林寺中为僧众讲学的情况,时间当在晋孝武帝太元十一年(386)以后,因慧远于太元六年入庐山,十一年,江州刺史桓伊为立东林寺。当时慧远年届六十,故以桑榆之光晚年暮景自况。史称其六十岁后,即拒绝世俗一切诱惑,“不复出山”而专心讲学传教。于此可见其弘扬佛学和专心教育的巨大热情。魏晋官学教育,由于时代动乱之故,遭受破坏,于是士族家学及民间私学,适应时代需要乘机兴起。东晋王朝在历经浩劫之馀,现实的苦难也触发了无数士庶皈依佛教的热情。慧远讲学东林,其宗教学校正是在时风众势下,应运而生。其实,听慧远讲学者不仅是僧众,其高足中也有俗世之士,如儒者雷次宗、画家宗炳等,都是著名人物。最令人敬服者,是慧远一生忠于教育的敬业精神。“桑榆之光”比喻确切,态度诚恳,发自肺腑的由衷之言,启迪了怠惰的学生,引发了莘莘学子一心向学的朝晖之明。“讲论不辍”、“词色甚苦”,言传身教,尽心尽力而不知老之将至,实在令人感动。这与后世办教育向钱看的颓风,不可同日而语。}

\lettrine{10.25} 桓南郡\myidx{桓玄}好猎\footnote{桓南郡:桓玄袭封南郡公,故称。参\CJKunderwave{德行}第41则注。},每田狩\footnote{田狩:狩猎。},车骑甚盛,五六十里中,旌旗蔽隰\footnote{隰:原指低湿之地,此泛指原野。},骋良马,驰击若飞,双甄所指\footnote{双甄:军阵之左右二翼。},不避陵壑\footnote{陵壑:丘陵沟坎。}。或行陈不整,麏兔腾逸\footnote{麏(jūn君):獐。腾逸:逃遁。},参佐无不被系束\footnote{参佐:僚属。系束:捆绑问罪。}。桓道恭\myidx{桓道恭}\footnote{桓道恭:刘注道恭为桓彝同堂弟,论辈分似误。所引\CJKunderwave{桓氏谱},疑“道恭字”下有漏,下当为“祖猷,彝同堂弟也”。},玄之族也,{\fzxk\zihao{6}\textcolor{red}{\CJKunderwave{桓氏谱}曰:“道恭字祖猷,彝同堂弟也。父赤之,太学博士。道恭历淮南太守、伪楚江夏相。义熙初伏诛。”}} 时为贼曹参军\footnote{贼曹参军:府衙属官,掌治安捕盗。},颇敢直言。常自带绛绵绳箸腰中\footnote{绛:大红色。},玄问:“此何为?”答曰:“公猎,好缚人士,会当被缚\footnote{会当:要是。},手不能堪芒也\footnote{芒:粗绳芒刺。}。”玄自此小差\footnote{小差:稍减,小损。}。

{\cangkai\zihao{5}【评】桓玄“性好畋猎”,史上有名。其年轻寄寓荆州之时,即曾向当时荆州刺史王忱借数百人出猎,见\CJKunderwave{晋书}忱传。古时狩猎作用有二:一是军事演练,一是生活享受。桓玄之猎,二者兼具。故事称其田狩“车骑甚盛,五六十里中,旌旗蔽隰,骋良马,驰击若飞,双甄所指,不避陵壑”,动态地描绘了古代的一次田猎行动之雄伟声势。据描写,当是桓玄夺取荆、江刺史,都督八州军事之后,这实是一场篡国夺权前的大规模军事演习,其野心与声威毕呈。玄建伪楚而登帝位后,史称“骄奢荒侈,游猎无度,以夜继昼”,则畋猎成为其奢靡的生活享受,以此而“百姓疲苦,朝野劳瘁,怨怒思乱者十室八九焉”,事见\CJKunderwave{晋书}玄传。但道恭委婉之谏,劝其爱护将士,虽减少参佐痛苦,主观上实为奸雄争取军心民心,于事何补?后道恭因助玄而伏诛,实是咎由自取。}

\lettrine{10.26} 王绪\myidx{王绪}、王国宝\myidx{王国宝}相为唇齿\footnote{王绪:见刘注。王国宝之从弟。王国宝:平北将军王坦之第三子。与王绪同为相王司马道子宠任,弄权朝廷,引发诸侯起兵清君侧,伏诛。相为唇齿:相互信赖。},并上下权要\footnote{上下权要:玩弄权术,操纵国政。“上下”,唐写本作“弄”,是。竖写“𠧗”是“弄”的异体字。}。{\fzxk\zihao{6}\textcolor{red}{\CJKunderwave{王氏谱}曰:“绪字仲业,太原人。祖延,父又(乂),抚军。”\CJKunderwave{晋安帝纪}曰:“绪为会稽王从事中郎,以佞邪亲幸。王珣、王恭恶国宝与绪乱政,与殷仲堪尅期同举,内匡朝廷。及恭表至,乃斩绪以悦诸侯。国宝,平北将军坦之弟(第)三子。太傅谢安,国宝妇父也,恶而抑之不用。安薨,相王辅政,迁中书令。有妾数百。从弟绪有宠于王,深为其说,国宝权动内外。王珣、王恭、殷仲堪为孝武所待,不为相王所眄。恭抗表讨之,车胤又争之。会稽王玘(既)不能拒诸侯兵,遂委罪国宝,付廷尉赐死。”}} 王大\myidx{王忱}不平其如此\footnote{王大:王忱,字元达,小字佛大,故称。坦之少子,国宝之弟。参本门第22则注。不平其如此:痛恨其所作所为。},乃谓绪曰:“汝为此欻欻\footnote{欻欻(xū须):轻举躁动貌。},曾不虑狱吏之为贵乎\footnote{曾不虑:竟然不顾忌。}?”{\fzxk\zihao{6}\textcolor{red}{\CJKunderwave{史记}曰:“有上书告汉丞相欲反,文帝下之廷尉。勃既出,叹曰:‘吾常将百万之军,安知狱吏之为贵也?’”}}

{\cangkai\zihao{5}【评】二王(国宝、绪)奸佞之人。国宝乃谢安之婿,安“恶其倾侧,每抑而不用”;其舅父中书郎范宁,儒雅方直,“疾其阿谀”而劝帝(孝武)黜之;王大(忱)其弟,“不平其如此”,同样痛恨其所作所为,借斥王绪而谏兄。二王之恶,时人无不知晓,但如此小人,却能弄权朝廷,势倾内外,几乎灭亡国家,能量极大,这却是何道理?这又回到本门第2则所谓“知不忠而任之”的问题,关键在于治国执政视国家为私有财产,而不为天下苍生着想,是封建专制制度使然。王忱之言:“曾不虑狱吏之为贵乎?”借古讽今,喻二王当思日后下狱治罪之惨酷,而不可因徼一时富贵而为非作歹。言语警醒,惜二王不悟而自赴断头之台。}

\lettrine{10.27} 桓玄\myidx{桓玄}欲以谢太傅\myidx{谢安}宅为营\footnote{桓玄:温少子。篡晋建楚,被刘裕诛杀。谢太傅:谢安,卒赠太傅,故称。为营:作为军营。},谢混\myidx{谢混}曰\footnote{谢混:字叔源,陈郡阳夏人。安孙,琰子。东晋末著名诗人,官至中书令、尚书右仆射。后因党刘毅被刘裕所诛。}:“召伯\myidx{姬奭}之仁\footnote{召伯:姬奭,西周初人。成王时为太保,与周公共同辅政,分陕而治。},犹惠及甘棠\footnote{甘棠:召伯于甘棠树下听讼,人思其惠,作\CJKunderwave{甘棠}之诗以颂之,诗见\CJKunderwave{诗经·召南}。}。{\fzxk\zihao{6}\textcolor{red}{\CJKunderwave{韩诗外传}曰:“昔周道之隆,召伯在朝,有司请召民。召伯曰:‘以一身劳百姓,非吾先君文王之志也。’乃暴处于棠下而听讼焉。诗人见召伯休息之棠,美而歌之曰:‘蔽芾甘棠,勿剪勿伐,召伯所茇。’”}} 文靖\myidx{谢安}之德\footnote{文靖:谢安卒谥文靖,故称。},更不保五亩之宅\footnote{五亩之宅:周行井田制,一夫之宅为五亩。此指一户居宅之所。}?”玄惭而止。

{\cangkai\zihao{5}【评】桓玄欲以谢安居宅为兵营,时当安帝元兴元年(402),桓玄兵入京师建康,加己总揆,都督中外诸军事、丞相,陵侮朝廷,幽摈宰辅,权势方炽,但尚未废晋称帝,仍须争取高门士族的支持,故安孙谢混得以进谏。玄自称帝之后,骄奢荒侈,脾性急暴,无复朝廷之体,“玄惭而止”之事,无复出现。在篡国夺权的斗争中,桓玄一方面拉拢士族,其“惭而止”的行为,是安慰如王谢家族一类高门士族的表面文章。但另一方面,则是利用一切机会,打压王谢家族,以树立桓家天下之威信,这才是本质行为。京师房宅何其多,为什么偏要以谢安宅为兵营呢?须知擒贼擒王,陈郡谢氏家族是当时高门士族的代表,打击谢家,则威信大增。桓玄曾对谢道韫批评谢安高隐东山而不终,见\CJKunderwave{排调}第26则刘注引\CJKunderwave{妇人集}。\CJKunderwave{品藻}门第87则玄又问刘瑾:“我何如谢太傅?”得势之日,大庭广众之中,咄咄逼人,形象刻画了东晋王、谢、庾、桓四大家族的矛盾斗争及其兴衰。}




%%% Local Variables:
%%% mode: latex
%%% TeX-engine: xetex
%%% TeX-master: "../Main"
%%% End:
