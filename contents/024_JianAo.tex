%% -*- coding: utf-8 -*-
%% Time-stamp: <Chen Wang: 2025-12-07 20:12:41>

% ○ ◎ ‧ 「 」 『 』 々 ( ) “ ” ■ ^[一-龥]
% 【\([^】][^】][^】]+\)】 → {\\fzxk\\zihao{6}\\textcolor{red}{\1}}
% \(【评】.*\) → {\\cangkai\\zihao{5}\1}
% \(【题解】.*\) → {\\cangkai\\zihao{5}\1}
% 《\([^》]+\)》 → \\CJKunderwave{\1}
% ^\([0-9]+.[0-9]+\) → \\lettrine{\1}
% {\\fzxk\\zihao{6}\\textcolor{red}{[^o]*}}


\setlength{\parindent}{0pt}

\chapter{简傲第二十四}



{\cangkai\zihao{5}【题解】 简傲者,简慢高傲之谓也。所谓简傲,对人而言,轻忽怠慢不礼貌;对己而言,则是看自己一朵花,视他人如粪土,自高自大,狂妄无知。在魏晋门阀社会中,这是许多贵游子弟的通病。如置于世界文化中,则西方中世纪贵族的傲慢与偏见与之近似。魏晋之简傲,已成为当时贵族的“流行病”。奇怪的是,以后两晋名士,如王、谢家族的领袖人物王导和谢安诸人,对此却大多给予理解与同情,这就无形中大大加速了简傲思想言行的流传,并终于成为魏晋时代的一种特殊文化现象。要了解魏晋士人的心态与人格,就必须对\CJKunderwave{简傲}篇加以研究。}

{\cangkai\zihao{5}在一般情况下,“简傲”是一个贬义词;但也不尽然,应视具体情况作具体分析。首先,简傲主体是否具有胜人一筹的傲人“资本”?其次,其所轻怠傲视的对象是谁?如魏晋之际的竹林七贤嵇(康)、阮(籍)之辈,他们原是满怀激情与理想,但却被虚伪礼法击得粉碎,于是转向高倡庄、老玄风,追慕旷达狂放,超越世俗而蔑视名教,放荡不拘而高自风标。嵇、阮自身资质之秀,加以傲视的是名教中的伪君子,故清者自清,浊者自浊,实有一定的社会批判价值,并具一定的历史意义。如阮籍箕踞傲对百官,当作如是解。当时的统治者司马氏集团,在篡弑夺权之前,还能对士人的简傲言行作某些容忍,以争取士人的支持。久而久之,时过境迁,贵游子弟则不问自身条件及面对的环境,也不问国家的治乱兴衰,以仿效竹林遗风而相互标榜,只知自我高傲,而践踏他人尊严,以玩忽职守为清高,以门阀傲人作矫饰,其简傲之思想言行,性质不同而应予细辨。}

\lettrine{24.1} 晋文王\myidx{司马昭}功德盛大\footnote{晋文王:指司马昭,魏咸熙元年(264)封晋王。},坐席严敬\footnote{严敬:庄严敬重。},拟于王者。{\fzxk\zihao{6}\textcolor{red}{\CJKunderwave{汉晋春秋}曰:“文王进爵为王,司徒何曾与朝臣皆尽礼,唯王祥长揖不拜。”}} 唯阮籍\myidx{阮籍}在坐,箕踞啸歌,酣放自若。

{\cangkai\zihao{5}【评】阮籍卒于景元四年(263),在司马昭封晋王前一年。因此,称“晋文王”,当是后人追称。故事发生在司马氏集团着手准备篡魏的前夕,称“拟于王者”,说明是尚未封王而以王者临朝自居。但“坐席严敬,拟于王者”的主语是谁?一解谓司马昭坐席间庄严肃穆,神情与帝王相似,主语是司马昭。一解谓坐席之群臣神色庄敬严肃,如在帝王驾前,主语是席间群臣。二解俱通,但相比而言,似后解更佳。因阮籍的态度,并非与“晋文王”作比,而是与群臣的奴态形成鲜明的对照。一个“唯”字,很说明问题,群臣趋炎附势,自失人格;而阮籍则我行我素,意态舒适。当日坐席之上,阮籍“箕踞啸歌”,不守礼节;“酣放自若”,行为轻慢。这是一种不与统治者合作的傲慢。他不想为将来的升官发财而丧失自我的人格尊严。但司马昭却原谅了他,因为他想搞“统战”,争取各方面士人代表的广泛支持,以便最终实行篡位夺权的阴谋。这对阮籍之徒,是幸,还是不幸?值得思考。}

\lettrine{24.2} 王戎\myidx{王戎}弱冠诣阮籍\myidx{阮籍}\footnote{王戎:字濬冲,魏晋间琅邪人。官至晋司徒。曾与阮籍友善,为竹林七贤之一。弱冠:古时男子二十成人,初加冠。诣:到,拜访。},时刘公荣\myidx{刘昶}在坐\footnote{刘公荣:刘昶,字公荣。参前\CJKunderwave{任诞}第4则注。},阮谓王曰:“偶有二斗美酒,当与君共饮,彼公荣者无预焉。”二人交觞酬酢\footnote{交觞酬酢:轮流举杯敬酒应酬。},公荣遂不得一杯\footnote{遂:终。},而言语谈戏,三人无异。或有问之者,阮答曰:“胜公荣者,不得不与饮酒,不如公荣者,不可不与饮酒;唯公荣可不与饮酒。”{\fzxk\zihao{6}\textcolor{red}{\CJKunderwave{晋阳秋}曰:“戎年十五,随父浑在郎舍,阮籍见而说焉。每适浑,俄顷,辄在戎室,久之,乃谓浑:‘濬冲清尚,非卿伦也。’戎尝诣籍共饮,而刘昶在坐,不与焉,昶无恨色。既而戎问籍曰:‘彼为谁也?’曰:‘刘公荣也。’濬冲曰:‘胜公荣,故与酒;不如公荣,不可不与酒;唯公荣者,可不与酒。’”\CJKunderwave{竹林七贤论}曰:“初,籍与戎父浑俱为尚书郎,每造浑,坐未安,辄曰:‘与卿语不如与阿戎语。’就戎,必日夕而返。籍长戎二十岁,相得如时辈。刘公荣通士,性尤好酒。籍与戎酬酢终日,而公荣不蒙一杯,三人各自得也。戎为物论所先,皆此类。”}}

{\cangkai\zihao{5}【评】这则故事形象生动,其中阮籍及刘昶的心理刻画颇为细腻。故事是调侃与幽默合二为一,浑然一体,而绝无轻慢讽刺之恶。前\CJKunderwave{任诞}第4则公荣曾说:“胜公荣者,不可不与饮;不如公荣者,亦不可不与饮;是公荣辈者,又不可不与饮。”这里则纯用公荣语来加以调侃,所异者“唯公荣可不与饮酒”一句。如王世懋所评:“即以公荣语翻出更妙,滑稽之雄。”于此不仅见阮籍的诙谐风趣,同时又见刘昶之雅量。他虽不得一杯饮,但却不以为忤,照常“言语谈戏”,了无异色。刘昶是一个真正懂得幽默人生的人。}

\lettrine{24.3} 锺士季\myidx{锺会}精有才理\footnote{锺士季:锺会,字士季。颍川长社(今属河南)人。繇子、毓弟。官至司隶校尉。参前\CJKunderwave{言语}第12则注。精:极,甚。才理:才情理致。},先不识嵇康\myidx{嵇康},锺要于时贤隽之士\footnote{要:邀。贤隽之士:贤达秀俊的杰出士人。},俱往寻康。康方大树下锻\footnote{方:正在。锻:冶锻,这里指打铁。},向子期\myidx{向秀}为佐,鼓排\footnote{向子期:向秀字子期,河内怀(今属河南)人。为佐:当助手。鼓排:拉风箱。鼓,鼓风;排,通“𩎻”(bài),皮制风箱。}。康扬槌不辍\footnote{不辍:不停止。},傍若无人,移时不交一言\footnote{移时:过了一段时间。}。锺起去,康曰:“何所闻而来?何所见而去?”锺曰:“闻所闻而来,见所见而去。”{\fzxk\zihao{6}\textcolor{red}{\CJKunderwave{文士传}曰:“康性绝巧,能锻铁。家有盛柳树,乃激水以圜之,夏天甚清凉,恒居其下傲戏,乃身自锻。家虽贫,有人就锻者,康不受直。唯亲旧以鸡酒往,与共饮啖,清言而已。”\CJKunderwave{魏氏春秋}曰:“锺会为大将军兄弟所昵,闻康名而造焉。会,名公子,以才能贵幸。乘肥衣轻,宾从如云。康方箕踞而锻,会至,不为之礼,会深衔之。后因吕安事,而遂谮康焉。”}}

{\cangkai\zihao{5}【评】锺会是贵公子,不择手段,野心勃勃,成为司马集团夺权的急先锋。而嵇康则对当时统治者的虚伪礼教极端厌恶。嵇、锺二人,道不同不相为谋。锺多次试探嵇,欲因其言行之失而致之死地。以此,嵇从不假锺以脸色,其傲对锺会,“移时不交一言”,正是一种不合作的蔑视。“何所闻而来?”与“闻所闻而来”诸语,对话生动,闪烁着刀光剑影,在委婉的修辞艺术中,却同时埋伏了抗争与杀机。日后嵇康终被锺会借故谗杀,因不肯低下那高傲的头颅而付出了生命的代价。统治者对简傲的容忍,是有限度的。如阮籍终老家门,是侥幸,只可一,不可再。嵇康的“玉碎”就是血淋淋的教训。}

\lettrine{24.4} 嵇康\myidx{嵇康}与吕安\myidx{吕安}善\footnote{吕安:字仲悌。东平(今属山东)人。与嵇康、向秀等竹林七贤中人友善,后与嵇康一起被司马氏杀害。},每一相思,千里命驾\footnote{千里命驾:不顾路途遥远,立即驱车前行。喻友情之深厚。}。{\fzxk\zihao{6}\textcolor{red}{\CJKunderwave{晋阳秋}曰:“安字仲悌,东平人。冀州刺史招之第二子。志量开旷,有拔俗风气。”干宝\CJKunderwave{晋纪}曰:“初,安之交康也,其相思则率尔命驾。”}} 安后来,值康不在\footnote{值:正好,恰巧。},喜\myidx{嵇喜}出户延之\footnote{喜:嵇喜,康兄。},不入。{\fzxk\zihao{6}\textcolor{red}{\CJKunderwave{晋百官名}曰:“嵇喜,字公穆。历扬州刺史。康兄也。阮籍遭丧,往吊之。籍能为青白眼,见凡俗之士,以白眼对之。及喜往,籍不哭,见其白眼,喜不怿而退。康间(闻)之,乃赍酒挟琴而造之,遂相与善。”干宝\CJKunderwave{晋纪}曰:“安尝从康,或遇其行,康兄喜拭席而待之,弗顾,独坐车中,康母就设酒食。求康儿共语戏,良久则去。其轻贵如此。”}} 题门上作“凤”字而去。喜不觉,犹以为欣\footnote{欣:欢欣,高兴。}。故作“凤”字,凡鸟也\footnote{“凤字”句:繁体字“凤”,义符为“鸟”,声符为“凡”。吕安以此讽刺嵇喜为庸俗之人。}。{\fzxk\zihao{6}\textcolor{red}{许慎\CJKunderwave{说文}曰:“凤,神鸟也。从鸟,凡声。”}}

{\cangkai\zihao{5}【评】吕安名在竹林七贤之外,但论其精神风度,则与嵇、阮诸贤息息相通。简傲者是非分明,吕安也是性情中人。于其所善,有所相思即“千里命驾”;于其不喜,则题门作“凤”字,刺其庸俗凡鸟。吕安之爱恶是非分明,实是阮籍青白眼之再版。以此见魏末晋初社会风气之一斑。}

\lettrine{24.5} 陆士衡\myidx{陆机}初入洛\footnote{陆士衡:陆机,字士衡。吴郡(今江苏苏州)人。西晋著名文学家。参前\CJKunderwave{言语}第26则注。洛:指西晋京师洛阳。},咨张公\myidx{张华}所宜诣\footnote{咨:咨询,讨教。张公:指张华,官至司空。参前\CJKunderwave{德行}第12则注。诣:拜访。},刘道真\myidx{刘宝}是其一\footnote{刘道真:刘宝字道真。高平(今属山东)人。官吏部郎、御史中丞、安北将军。曾侍皇太子讲\CJKunderwave{汉书}。}。陆既往,刘尚在哀制中\footnote{哀制:礼制规定的守丧期间。}。性嗜酒,礼毕,初无它言\footnote{初无:一点没有,全无。},唯问:“东吴有长柄壶卢\footnote{东吴:指三国时吴国所处江东地区。长柄壶卢:长把葫芦。},卿得种来不?”陆兄弟殊失望\footnote{殊:非常,甚。},乃悔往。

{\cangkai\zihao{5}【评】陆机、陆云兄弟初入洛,当在晋武帝太康十年(289)应征之年。张华推荐二陆前去拜望刘宝,自有道理。二陆要在京师立住脚,图发展,如果没有中原士人的支持,是难以实现的。刘宝是当朝名士,与王衍并为士林清议代表人物,经其品评,以定士人优劣。但作为中原士族之英,刘宝对于刚投降归顺不久的东吴士人,实是心存歧视。刘宝本身虽是治丧礼的专家,但却任诞而行,不顾礼仪,故意只问酒事,而不及其他,令二陆兄弟难堪。其内心独白是:要我品评推扬,免开尊口。二陆兄弟何等聪明之人,岂能不明白?二陆当时不满三十,正当年轻气盛之时,深感中原士人唯我独尊而目中无人,因而心生悔往之痛。当时南、北士人的情绪对立,于此可见一斑。}

\lettrine{24.6} 王平子\myidx{王澄}出为荆州\footnote{王平子:王澄字平子。琅邪人。西晋太尉王衍之弟。参前\CJKunderwave{德行}第23则注。出为荆州:出任荆州刺史。},{\fzxk\zihao{6}\textcolor{red}{\CJKunderwave{晋阳秋}曰:“惠帝时,太尉王夷甫言于选者,以弟澄为荆州刺史,从弟敦为青州刺史。澄、敦俱诣太尉辞,太尉谓曰:‘今王室将卑,故使弟等居齐、楚之地,外可以建霸业,内足以匡帝室,所望于二弟也。’”}} 王太尉\myidx{王澄}及时贤送者倾路\footnote{王太尉:指王衍,字夷甫。官至太尉,故称。参前\CJKunderwave{言语}第23则注。倾路:挤满道路。}。时庭中有大树,上有鹊巢,平子脱衣巾,径上树取鹊子,凉衣拘阂树枝\footnote{凉衣:贴身内衣。拘阂(hé合):挂碍。},便复脱去。得鹊子还下弄\footnote{鹊子:小喜鹊。弄:玩耍。},神色自若,傍若无人。{\fzxk\zihao{6}\textcolor{red}{邓粲\CJKunderwave{晋纪}曰:“澄放荡不拘,时谓之达。”}}

{\cangkai\zihao{5}【评】王澄刺荆,据\CJKunderwave{通鉴}当在怀帝永嘉元年(307)。时经八王之乱,“五胡”混战中原,正是志士仁人挽狂澜于既倒之时。但是,史称“澄既至镇,日夜纵酒,不亲庶事,虽寇戎急务,亦不以为怀”。此虽后来之事,但却有助于说明他赴任时的具体心态。澄学竹林诸贤作达,貌似阮籍嗜饮狂放,而内心全不一样。如李慈铭所评:“王澄一生,绝无可取。狂且恃贵,轻侻丧身。既无当世之才,亦绝片言之善。虚叨疆寄,致乱逃归。……观于此举,脱衣上树,裸体探雏,直是无赖妄人,风狂乞相。以为简傲,何啻寱言!”澄时膺任方伯疆寄,职责何等重大,赴任祖送之际,仪式非常隆重。但澄之裸裎为戏,当众凌傲百官,既不自重,更是践踏他人尊严。澄虽具勃勃野心,实乏治国之才,惟以门阀傲人,故刘琨谓其“以此处世,难得其死”。晋之不竞,于此见其端倪。}

\lettrine{24.7} 高坐道人\myidx{高坐道人}于丞相\myidx{王导}坐\footnote{高坐道人:两晋间时西域和尚,永嘉时来中土。原名尸黎密。参前\CJKunderwave{言语}第39则注。丞相:指王导。},恒偃卧其侧\footnote{恒:经常,总是。}。见卞令\myidx{卞壸}\footnote{卞令:指卞壸,字望之。曾官尚书令,故称。},肃然改容\footnote{肃然:形容脸色严肃。},云:“彼是囗囗(礼法)人\footnote{彼是囗囗人:宋本“人”上二字模糊不清,据袁本当作“礼法”。注文亦有糊字,据袁本辨之,加括号标识。}。”{\fzxk\zihao{6}\textcolor{red}{\CJKunderwave{高坐传}曰:“王公曾诣和上,和上(解带偃)伏,悟言神解。见尚书令卞望之,便(敛衿)饰容,时(叹)皆得其所。”}}

{\cangkai\zihao{5}【评】入乡随俗,虽佛学高僧也不能免。和尚要在世俗中宣传佛理,就必须先与世俗打交道,说世俗人听得懂的话。具体做法因人而异,以便取得最佳效果,这是宣扬佛学教义的需要,也是佛学东渐必经的中国化过程。但后人或不明此理,如陶珙讥评高坐云:“直是依人而施,尚得谓为出世高僧耶?”超越时间和环境,言虽正而无的放矢。至于王与卞态度之别,形成对照,则因其所持玄学与儒学之异。王之宽容随和,更是东晋初立国“统战”的需要。}

\lettrine{24.8} 桓宣武\myidx{桓温}作徐州\footnote{桓宣武:指桓温。作徐州:任徐州刺史。时徐州刺史治所在京口(今江苏镇江)。},时谢奕\myidx{谢奕}为晋陵\footnote{谢奕:字无奕。谢安长兄。参前\CJKunderwave{德行}第33则注。为晋陵:任晋陵郡太守。晋陵郡治所在丹徒(今属江苏)。},{\fzxk\zihao{6}\textcolor{red}{\CJKunderwave{中兴书}曰:“奕自吏部郎,出为晋陵太守。”}} 先粗经虚怀\footnote{粗经虚怀:粗叙寒温。粗,大致。虚怀,心怀。},而乃无异常。及桓迁荆州\footnote{迁:升任。荆州:州名,治所在江陵。},将西之间,意气甚笃\footnote{意气:情意。笃:深厚。},奕弗之疑。唯谢虎子\myidx{谢据}妇王\myidx{王绥}悟其旨\footnote{谢虎子:谢据小字虎子。奕弟,安之二兄。妇王:谢据妻王氏,名绥,王韬女。悟:领悟,明白。旨:意图,用意。},{\fzxk\zihao{6}\textcolor{red}{虎子,谢据小字,奕弟也。其妻王氏,已见。}} 每曰:“桓荆州用意殊异\footnote{殊异:很不一般。},必与晋陵\myidx{谢奕}俱西矣\footnote{晋陵:指谢奕。}。”俄而引奕为司马\footnote{俄:不久。引:援引,荐举。司马:幕府中掌兵事的官员。}。奕既上,犹推布衣交\footnote{布衣交:不论地位的贫贱之交。},在温坐,岸帻啸咏\footnote{岸帻(zé责):掀起头巾,露出额头,以示不拘礼节之洒脱。帻:包发头巾。啸咏:即歌啸,撮唇鼓舌运气发声,犹如今之口哨歌吟,表现魏晋士人悠然逸态。},无异常日。宣武每曰:“我方外司马\footnote{方外:世俗之外。}。”遂因酒转无朝夕礼\footnote{朝夕礼:日常一般礼节。},桓舍入内\footnote{舍:躲开。内:内宅。},奕辄复随去。后至奕醉,温往主许避之\footnote{主:公主,指桓温妻南康长公主。许:住处。}。主曰:“君无狂司马,我何由得相见!”

{\cangkai\zihao{5}【评】谢奕之狂,不仅出于个性,而且自恃门第身份,傲对上司,因为桓氏家族兵家出身,非华丽家族子弟。不过与王澄诸人不同,谢奕狂得可爱,并不盛气凌人,他对桓温也是一片真情,醉态天真烂漫,纯是出于本性之自然。至于桓温与公主,则是政治婚姻。皇家要世族的支持,以维护政权;世族需皇家的庇荫,以求飞黄腾达。公主与驸马的婚姻,以“利”维系。一旦形势失衡,则其感情危机,立即显现。桓温刺荆,掌控长江中上游军事力量,权势迅速膨胀,因此借故冷落公主。一门之内,夫妻长期不相见。“君无狂司马,我何由得见?”公主愤懑之色,呈现脸上,声吻之间,略带几分悲怆。公主尚且如此,更何况是一般妇女的命运!}

\lettrine{24.9} 谢万\myidx{谢万}在兄前\footnote{谢万:字万石。奕、据、安之弟。曾任西中郎将、豫州刺史,于穆帝升平中奉命率军北伐,败归。参前\CJKunderwave{言语}第77则注。},欲起索便器\footnote{索:要,找。便器:如便壶之类的器具。}。于时阮思旷\myidx{阮裕}在坐\footnote{于时:当时。阮思旷:阮裕字思旷。东晋名士,屡辞征辟。参前\CJKunderwave{德行}第32则注。},曰:“新出门户\footnote{新出门户:新兴暴发家族。},笃而无礼\footnote{笃:笃诚,真率。}。”

{\cangkai\zihao{5}【评】谢万初经世面,少不更事,自恃门第,于众坐前公然索便器,傲诞无礼,轻慢于人。其兄长有奕、据、安三人,具体谓谁?待考。奕、据早逝,安晚出,其时谢家尚未发展至顶峰。在西晋间,陈郡阳夏谢家,声誉未著,难与琅邪王家争衡。后来,经由谢衡至谢鲲,由儒入玄,适应了时代潮流之变化,谢氏家族声名始兴。发展至谢尚、谢奕辈,则日渐发达,至谢安及奕子玄辈,更是发扬光大,始与琅邪王氏齐名,并称王谢家族。万在谢家尚未发达之时,简傲旧日门阀,故为阮裕所斥。裕出于陈留阮家,阮瑀为建安七子之一,阮籍与兄子咸并为竹林七贤中人。阮氏家族自魏晋以来,声满人间。阮裕以“笃而无礼”讥万,一语中的;但轻其“新出门户”,讥为暴发户,则仍属门阀偏见。}

\lettrine{24.10} 谢中郎\myidx{谢万}是王蓝田\myidx{王述}女婿\footnote{谢中郎:即谢万。万曾任抚军从事中郎,故称。王蓝田:王述字怀祖,封蓝田侯,故称。参前\CJKunderwave{文学}第22则注。},{\fzxk\zihao{6}\textcolor{red}{\CJKunderwave{谢氏谱}曰:“万取太原王述女,名荃。”}} 尝箸白纶巾\footnote{纶巾:又称诸葛巾,丝织头巾。},肩舆径至扬州听事见王\footnote{肩舆:用人力抬的轻便轿子。扬州听事:扬州府厅堂。王述时任扬州刺史。},直言曰\footnote{直:直率。}:“人言君侯痴,君侯信自痴\footnote{信自:诚然,确实。君侯:敬称,王述袭爵为侯,故称。}。”蓝田曰:“非无此论,但晚令耳\footnote{但:只是。晚令:迟到的美名。}。”{\fzxk\zihao{6}\textcolor{red}{\CJKunderwave{述别传}曰:“述少真独退静,人未尝知,故有晚令之言。”}}

{\cangkai\zihao{5}【评】谢万直上扬州府堂,当面讥讽老丈人痴呆。对长辈尚且如此,对平辈或下级又当如何?其简慢狂傲可想而知。后来他做统兵将帅,挥如意直指帐下诸将为劲卒,思想言行一以贯之。因此而惹怒众将,如果不是谢安出面安抚,就差点为此而付出生命的代价。这是惨重的人生教训,但万至死不悟。至于王述,其晚到的令名,正是优秀家风的承传发扬。史称其祖父湛,“阖门守静,不交当世,冲素简淡”,故人以为痴,晋武帝也在庙堂之上谓其痴。但湛好学深思,精于\CJKunderwave{易}理,实非常人所能及。其父承,“清虚寡欲,无所修尚”,渡江名臣王导、庾亮诸人皆出其下,“为中兴第一”。发展至述,则“安夷守约,不求闻达”,王导评其不痴,曰:“怀祖清贞简贵,不减祖、父。”面对女婿的无礼嘲弄,他却以晚到令名作答,语极幽默,正见其宽广胸怀和雅量。}

\lettrine{24.11} 王子猷\myidx{王徽之}作桓车骑\myidx{桓冲}骑兵参军\footnote{王子猷:王徽之,字子猷。见前注。桓车骑:指桓冲,温弟,曾任荆州刺史、车骑将军,故称。骑兵参军:官名,掌马匹供给诸事。}。桓问曰:“卿何署\footnote{署:衙门。}?”答曰:“不知何署,时见牵马来,似是马曹\footnote{马曹:徽之戏言,时无马曹而有骑曹。}。”{\fzxk\zihao{6}\textcolor{red}{\CJKunderwave{中兴书}曰:“桓冲引徽之为参军,蓬首散带,不综知其府事。”}} 桓又问:“官有几马?”答曰:“不问马,何由知其数?”{\fzxk\zihao{6}\textcolor{red}{\CJKunderwave{论语}曰:“厩焚,孔子退朝,曰:‘伤人乎?’不问马。”注:“贵人贱畜,故不问也。”}} 又问:“马比死多少?”\footnote{比:近来。}答曰:“未知生,焉知死。”{\fzxk\zihao{6}\textcolor{red}{\CJKunderwave{论语}曰:“子路问死。孔子曰:‘未知生,焉知死!’”马融注曰:“死事难明,语之无益,故不答。”}}

{\cangkai\zihao{5}【评】故事颇似小说,对话传神有味。王徽之自恃门阀高贵,傲慢世人,连顶头上司也概莫能外。桓冲当时镇守荆州,与谢安并为朝廷倚仗的重臣,其谦虚爱士,尽忠国家,史有令名。但徽之却因其出身兵家,身份微贱而蔑视之。作为桓冲下属,他以不务世事为高。桓冲有问,又顾左右而言他,故意摆谱。他在荆州幕府担任什么官职,怎么可能不知道呢?如真不知,又岂能走马上任?“不问马”、“未知生,焉知死”诸语,巧用\CJKunderwave{论语}中孔子之言以对,虽是强词夺理,却也表现了读书活用的几分聪明。只是这些学问不是用在正道,而尽成戏言慢语,话中带刺,听来令人不舒服。故事发生在淝水之战前夕,正是国家民族危急存亡之秋,如此“不婴世务”,岂能有益苍生社稷!}

\lettrine{24.12} 谢公\myidx{谢安}尝与谢万\myidx{谢万}共出西\footnote{谢公:指谢安。谢万:见前注。共出西:东晋都建康,会稽在其东方,二谢兄弟居会稽,故称赴京师为“出西”。},过吴郡\footnote{吴郡:郡名,治所在今苏州。},阿万欲相与共萃王恬\myidx{谢万}共出西\footnote{谢公:指谢安。谢万:见前注。共出西:东晋都建康,会稽在其东方,二谢兄弟居会稽,故称赴京师为“出西”。},过吴郡\footnote{吴郡:郡名,治所在今苏州。},阿万欲相与共萃王恬许\footnote{萃:聚集。王恬}许\footnote{萃:聚集。王恬:字敬豫,小字螭虎。王导子。官至中书郎、中军将军、会稽内史。多才气,善隶书,兼擅围棋。},{\fzxk\zihao{6}\textcolor{red}{恬,已见,时为吴郡太守。}} 太傅云:“恐伊不必酬汝\footnote{伊:他。不必:不一定。酬汝:与你应酬,指酒食招待等。},意不足尔\footnote{不足尔:不值得如此。}。”万犹苦要\footnote{要:通“邀”。},太傅坚不回\footnote{不回:不改变主意。},万乃独往。坐少时,王便入问(门)内\footnote{问内:袁本作“门内”,是。},谢殊有欣色,以为厚待己。良久,乃沐头散发而出,亦不坐,仍据胡床\footnote{胡床:轻便交椅。},在中庭晒头\footnote{中庭:庭中。晒头:晾晒头发。},神气傲迈\footnote{傲迈:傲慢。},了无相酬对意\footnote{了无相酬对意:完全没有一点招待客人的意思。}。谢于是乃还。未至船,逆呼太傅安\footnote{逆呼:迎头大喊。太傅安:袁本无“安”字。},安曰:“阿螭\myidx{王恬}不作尔\footnote{阿螭:指王恬,小字螭虎。不作尔:不交往应酬。}。”{\fzxk\zihao{6}\textcolor{red}{王恬小字螭虎。}}

{\cangkai\zihao{5}【评】谢万一贯自高门第清华而傲慢他人。但这次王恬却以其人之道而还治其人之身,终于让谢万也尝到了傲慢与偏见的滋味,因为王恬出于琅邪王家,丞相王导之子,门第更高贵,更有傲人的“资本”。如余嘉锡所评:“江左王、谢齐名,实在安立功名以后。此时谢氏兄弟甫有盛名,而其先本非世族,故阮裕讥为新兴门户。王恬贵游子弟,宜其不礼谢万也。”二谢兄弟在发达之前,万急于往上爬,巴结比自己高贵的人物;而安则有自知之明,决不自讨没趣而前往受辱。安、万相较,内心素质之优劣自现。安后来建功立业,与万之失败被废,在年轻时就见其端倪。}

\lettrine{24.13} 王子猷\myidx{王徽之}作桓车骑\myidx{桓冲}参军\footnote{王子猷:王徽之。桓车骑:桓冲。参军:具体指骑兵参军。}。桓谓王曰:“卿在府久,比当相料理\footnote{比:近来。料理:安排,照顾,治事。}。”初不答,直高视,以手版拄颊\footnote{手版:笏,古时官员上朝或见官长时所执,用以记事备忘。},云:“西山朝来,致有爽气\footnote{朝来:早晨。致:送来。}。”

{\cangkai\zihao{5}【评】刻画贵游子弟傲慢神气,活灵活现。答非所问,自说自话,正见其不撄世务的迈往不羁之态。此则与本篇第11则为姐妹篇,主人公皆为王徽之,当并读而细加体悟,则意味愈深。}

\lettrine{24.14} 谢万\myidx{谢万}北征\footnote{北征:晋穆帝升平二年(358),谢万为西中郎将、豫州刺史,奉命率师北伐前燕。次年于寿春败废。},常以啸咏自高\footnote{啸咏:即歌啸,撮唇鼓舌运气发声,犹如今之口哨歌吟,表现魏晋士人悠然逸态。},未尝抚慰众士。谢公\myidx{谢安}甚器爱万\footnote{谢公:指谢安。万之三兄。},而审其必败\footnote{审:明白,知道。},乃俱行,从容谓万曰:“汝为元帅,宜数唤诸将宴会,以悦众心。”万从之。因召集诸将,都无所说,直以如意指四坐云\footnote{直:只。如意:原为抓痒用器,因其可如人意,故名。后清谈者常持以比画助兴。}:“诸君皆是劲卒\footnote{劲卒:劲健士兵。\CJKunderwave{通鉴}胡注云:“凡奋身行伍者,以兵与卒为讳。既为将矣,而称之为卒,所以益恨也。”}。”诸将甚忿恨之。谢公欲深箸恩信,自队主将帅以下\footnote{队主:队长。},无不身造\footnote{身造:亲身拜访。},厚相逊谢\footnote{厚相逊谢:诚恳道歉。}。及万事败,军中因欲除之,复云:“当为隐士\footnote{隐士:时谢安未仕,故称隐士。}。”故幸而得免\footnote{幸而得免:侥幸生还,贷其一死。}。{\fzxk\zihao{6}\textcolor{red}{万败事已见上。}}

{\cangkai\zihao{5}【评】东晋士林望族称王、谢。但真正第一簪缨世家,则非琅邪王家莫属。“简傲”篇共17则故事,与琅邪王家有关的共8则,占全篇的近一半。其中,王子猷(徽之)独占3则,合占1则,位居全篇之首。这与琅邪王家门阀第一的地位相符。其次即为陈郡谢家,共5则。其中谢万一人独占2则,合占2则,在全篇所占地位,当与王徽之并列第一。这也与谢家这一新兴门户迅速暴发的形势相称。谢万之傲慢,自恃门第高华,视人如粪土,见其狂悖之性,而毫无遮饰。从文学形象角度言,刘辰翁谓“甚得騃态”,甚是。但以之治国安邦,帅师作战,则犹如儿戏一般,岂能不败?试想,麾下诸将皆是刀头舔血之人,在公开场合,视为“劲卒”,也即最低等士兵。“卒”者亡也,“兵”者音近于“殡”,也与死亡相关。行伍出身之人,视此为不吉祥的谶语,故诸将有受侮之怒,也是事出自然。作为军事统帅的谢万,在决战前夕,公开侮辱将士,不仅直接损害国家,而且简直拿自己的生命开玩笑。战前王羲之寄信诫万,要他收敛其“迈往不屑”之气,与属下同其甘苦,言中其病。而万不改其狂傲之性,虽安之智而无救其败,呜呼哀哉!}

\lettrine{24.15} 王子敬\myidx{王献之}兄弟见郗公\myidx{郗愔}\footnote{王子敬兄弟:子敬,王献之字。王羲之与郗璿共育七子,据考,依次是玄之、凝之、涣之、肃之、徽之、操之、献之。七子中献之最贤,故以之为王家子弟的代表。郗公:郗愔,字方回,高平金乡人。郗璿之弟,子敬兄弟母舅。参前\CJKunderwave{品藻}第29则注。},蹑履问讯\footnote{蹑履:穿着鞋。当时正式着装,以示敬。问讯:问候,问安。},甚修外生礼。及嘉宾\myidx{郗超}死\footnote{嘉宾:郗超,字景兴,一字嘉宾。愔子。东晋名士,卓荦不群。官至司徒左长史。曾为权臣桓温谋主,甚获宠信,权重一时。参前\CJKunderwave{言语}第59则注。},皆箸高屐\footnote{高屐(jī基):高齿木底鞋。魏晋贵族穿高屐属便装,示人悠游闲暇。但在正式场合或见长辈,著高屐则属轻慢无礼。},仪容轻慢\footnote{仪容轻慢:神态轻率傲慢。}。命坐,皆云:“有事不暇坐。”既去,郗公慨然曰:“使嘉宾不死,鼠辈敢尔!”{\fzxk\zihao{6}\textcolor{red}{愔子超,有盛名,且获宠于桓温,故为超敬愔。}}

{\cangkai\zihao{5}【评】这是一则很有意味的古代小小说,刘辰翁评谓“备极世态”,而给人以启迪。从文学角度言,故事的细节生动,形象栩栩如生而“慢态可掬”(王世懋语)。同是一双脚,一“蹑屐”,毕恭毕敬;一“著高屐”,昂首阔步,前恭后倨,何其傲慢无礼。其言语对话,活脱传达了人物内心的秘密,其心理刻画惟妙惟肖。但是,透过成功的艺术帷幕,见其言约旨远的幕后话语。}

{\cangkai\zihao{5}郗超虽为桓温腹心谋主,威吓朝廷,但其父愔并不知晓,他一生忠于晋王朝。故王家兄弟简慢舅父,并非为政治立场,而是琅邪王家贵游子弟那根深蒂固的门阀观念所致。超死之后,郗家顿失支撑,形势发生变化。琅邪王家自视为天下第一贵族,连陈郡谢家也不在话下,更何况是高平郗家呢?因此子敬兄弟目空一切,不给娘舅脸色,当时母亲郗璿健在,为了门第“尊严”,连母舅都在傲视简慢之列。门阀意识,摧残了贵族的人性,于此可见一斑。郗愔痛斥亲外甥,称“鼠辈敢尔”!“鼠辈”袁本改作“儿辈”,误。应以宋本为是,非如此无以见老人激动愤慨之色。}

\lettrine{24.16} 王子猷\myidx{王徽之}尝行过吴中\footnote{吴中:吴郡地区,今江苏苏州一带。},见一士大夫家极有好竹。主已知子猷当往,乃洒揥(扫)施设\footnote{洒揥:诸本作“洒埽”,是。“埽”通“扫”。洒扫即清洁庭院。施设:即具饮馔。},在听事坐相待\footnote{听事:厅堂。相待:等待。}。王肩舆径造竹下\footnote{径造:直达,径直。},讽啸良久,主已失望,犹冀还当通\footnote{冀:希望。通:通问,通报。}。遂直欲出门\footnote{遂:竟然。直:直接。},主人大不堪\footnote{不堪:无法忍受。},便令左右闭门,不听出\footnote{不听出:不让出门。}。王更以此赏主人,乃留坐,尽欢而去。

{\cangkai\zihao{5}【评】徽之爱竹有名,本是风雅之事。他爱竹只为风神俊赏,人竹浑然一体,实是欣赏自我的一种艺术人生,而与竹园主人是否热情无关。见竹而不见人,贱视主人而毫不顾惜,正见其只知自我的张狂。但当主人怒其失礼而闭门不听出时,徽之却突然发现了主人之个性自我,惺惺相惜,乃留坐尽欢。这一戏剧性的波澜变化,让人感到这一贵游子弟尚存童真可爱的另一面。王乾开云:“风流亦多,猖狂太甚。”所评颇为贴切。}

\lettrine{24.17} 王子敬\myidx{王献之}自会稽经吴\footnote{王子敬:王献之,字子敬。羲之幼子,尚主。官至尚书令。著名书法家。参前\CJKunderwave{德行}第39则注。},闻顾辟彊\myidx{顾辟彊}{\fzxk\zihao{6}\textcolor{red}{\CJKunderwave{顾氏谱}曰:“辟彊,吴郡人,历郡功曹、平北参军。”}}有名园\footnote{名园:顾辟彊在苏州的园林,是现知江南最早的私家园林。},先不识主人,径往其家。值顾方集宾友酣燕\footnote{值:正巧,碰上。燕:通“宴”,酒宴。},而王游历既毕,指麾好恶\footnote{指麾好恶:评论优劣是非。指麾,同“指挥”,指点议论。},傍若无人。顾勃然不堪,曰:“傲主人,非礼也;以贵骄人,非道也。失此二者,不足齿人,伧耳\footnote{不足齿人,伧耳:袁本改“人”作“之”,连为“不足齿之伧耳”,亦通。但宋本“不足齿人”断,“伧耳”单独成句,文句不仅通畅,而且更富感情色彩。伧,当时南人对中原人的蔑称。}!”便驱其左右出门。王独在舆上,回转顾望,左右移时不至。然后令送箸门外,怡然不屑。

{\cangkai\zihao{5}【评】子猷、子敬兄弟,故事相似而结果不同。子猷所遇士大夫,欣慕之心溢于言表,经过相打相识,二情相通。子敬遭遇,则没有那么幸运,名园主人顾辟彊亦非等闲之辈,个性倔强,得理而不饶人。王家子弟以其第一贵族出身而傲人,入顾氏园,不请自至,旁若无人,瞎发议论,而自以为高明,对主人及其宾友又不屑一顾,大大刺伤了别人的感情,主人勃然变色,下逐客令,也在情理之中。顾氏一段议论,愤形于色,掷地有声,很有力量。子敬左右被驱出门,他仍坐在肩舆上而不肯下轿走人。羞人者反被羞,所谓“怡然不屑”,只不过是贵游子弟死要面子的自我解嘲而已,悲哉!}





%%% Local Variables:
%%% mode: latex
%%% TeX-engine: xetex
%%% TeX-master: "../Main"
%%% End:
