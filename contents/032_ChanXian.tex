%% -*- coding: utf-8 -*-
%% Time-stamp: <Chen Wang: 2025-12-09 21:10:02>

% ○ ◎ ‧ 「 」 『 』 々 ( ) “ ” ■ ^[一-龥]
% 【\([^】][^】][^】]+\)】 → {\\fzxk\\zihao{6}\\textcolor{red}{\1}}
% \(【评】.*\) → {\\cangkai\\zihao{5}\1}
% \(【题解】.*\) → {\\cangkai\\zihao{5}\1}
% 《\([^》]+\)》 → \\CJKunderwave{\1}
% ^\([0-9]+.[0-9]+\) → \\lettrine{\1}
% {\\fzxk\\zihao{6}\\textcolor{red}{[^o]*}}

\setlength{\parindent}{0pt}


\chapter{谗险第三十二}




{\cangkai\zihao{5}【题解】 谗险者,谗言诽谤,阴险中伤之谓也。本门所述,多是进谗言陷害贤能的故事。小人以势利相交,有小人则无以立君子。谗害贤良,尔虞我诈以争权夺利,是小人的看家本领,自古已然。\CJKunderwave{左传}哀公十六年载叶公子高之言:“以险侥幸者,其求无餍。”为一己之私利,阴险设计谗害贤良之事,历史上永无休止,上自国家朝廷,下至君子士庶,无不蒙受其害。故\CJKunderwave{诗经·小雅}有\CJKunderwave{巧言}之篇,说是“乱之又生,君子信谗”。写的是统治者因听信小人谗言而祸国殃民,后患无穷。因此,诗人大声疾呼:“取彼谗人,投畀豺虎!”(\CJKunderwave{小雅·巷伯})只有让老天爷来惩罚这些谗害忠良的小人。本门故事所载,进谗与反谗的斗争,形势错综复杂,反映了努力靠拢政治漩涡中心各色人物的种种微妙心态,从另一个侧面为当时的名士绘形写照。}

\lettrine{32.1} 王平子\myidx{王澄}形甚散朗\footnote{王平子:王澄字平子,西晋名士,衍弟。官荆州刺史。参前\CJKunderwave{德行}第23则注。散朗:洒脱疏朗。},内实劲侠\footnote{劲侠:刚愎褊狭。侠,通“狭”,狭隘。}。{\fzxk\zihao{6}\textcolor{red}{邓粲\CJKunderwave{晋纪}云:“刘琨尝谓澄曰:‘卿汧(形)虽散朗,而内实劲侠,以此处世,难得其死。’澄默然无以答。后果为王敦所害。刘琨闻之曰:‘自取死耳。’”}}

{\cangkai\zihao{5}【评】王澄是太尉衍弟,也称西京名士。其外形虽然洒脱爽朗,日夜纵酒,而不亲庶事;而内心实是恃才傲物,高自标榜而眼中无物,心胸极其狭隘。作为政治家,“如此处世,难得其死”,后来果为王敦所杀,刘琨不幸言中。但是,故事与\CJKunderwave{谗险}关系不大,改入\CJKunderwave{识鉴}门似乎更为妥帖。另外,也可能刊刻遗漏,今之所见,故事未完。此事待考。}

\lettrine{32.2} 袁悦\myidx{袁悦}有口才\footnote{袁悦:参刘注。\CJKunderwave{晋书}作“袁悦之”。},能短长说\footnote{短长说:纵横家的游说。},亦有精理\footnote{精理:深刻思考。}。始作谢玄\myidx{谢玄}参军\footnote{谢玄:奕子、安侄,卒赠车骑将军。},颇被礼遇。后丁艰\footnote{丁艰:父母丧在家守制。},服除还都\footnote{服除:三年守丧毕除去孝服。},唯赍\CJKunderwave{战国策}而已\footnote{赍(jī基):携带。\CJKunderwave{战国策}:战国史书,汉刘向编,内容多说客纵横之辞。}。语人曰:“少年时读\CJKunderwave{论语}、\CJKunderwave{老子}\footnote{\CJKunderwave{论语}:记载孔子及其学生言行思想的书。\CJKunderwave{老子}:即老子\CJKunderwave{道德经}。},又看\CJKunderwave{庄}、\CJKunderwave{易}\footnote{\CJKunderwave{庄}:指\CJKunderwave{庄子}。\CJKunderwave{易}:指\CJKunderwave{周易},也称\CJKunderwave{易经}。},此皆是病痛事\footnote{病痛事:喻人生常有的小灾难。},当何所益邪?天下要物,正有\CJKunderwave{战国策}\footnote{正有:只有。}。”既下\footnote{下:下都,既回到京师。},说司马孝文王\myidx{司马道子}\footnote{司马孝文王:“孝文王”,当作“文孝王”,指会稽王司马道子。},大见亲待,几乱机轴\footnote{机轴:机,弩牙;轴,车轴。此皆物之要者,故谓机要用事为机轴。此喻朝廷。}。俄而见诛\footnote{俄:不久。}。{\fzxk\zihao{6}\textcolor{red}{\CJKunderwave{袁氏谱}曰:“悦字元礼,陈郡阳夏人。父朗,给事中。仕至骠骑咨议。太元中,悦有宠于会稽王,每劝专览朝权,王颇纳其言。王粲(恭)闻其说,言于孝武,乃托以他罪,杀悦于市中。既而朋党同异之声,播于朝野矣。”}}

{\cangkai\zihao{5}【评】魏晋之时,思想界以玄、佛与儒三教并立,而袁悦则持\CJKunderwave{战国策}纵横家之游说,这是乱世争权的路数,重在一时的政治权益,而弃置根本的思想修养。袁悦的存在,也说明了当时思想自由争鸣的情况。袁悦虽能言善辩而“有精理”,但其心思,主要花在捞取实际政治利益之上,唯利是图而不顾国家与民族利益。此则如与\CJKunderwave{赏誉}第153则故事合读互参,则袁悦小人形象毕现。阴险谗害人者最终害己,政治斗争是残酷的,袁悦被推上刑场,正是自我作孽的结果。}

\lettrine{32.3} 孝武\myidx{司马曜}甚亲数王国宝\myidx{王国宝}、王雅\myidx{王雅}\footnote{孝武帝:司马曜,简文帝子,在位共二十四年,年号为宁康和太元。亲数:袁本作“敬亲”。王国宝:王坦之第三子,历官中书令,尚书左仆射,卒赠光禄大夫。},{\fzxk\zihao{6}\textcolor{red}{\CJKunderwave{雅别传}曰:“雅字茂建,东海沂人。少知名。”\CJKunderwave{晋安帝纪}曰:“雅之为侍中,孝武甚信而重之。王珣、王恭特以地望见礼,至于亲幸,莫及雅者。上每置酒燕集,或召雅未至,上不先举觞。时议谓珣、恭宜傅东宫,而雅以宠幸,超授太傅、尚书左仆射。”}} 雅荐王珣\myidx{王珣}于帝\footnote{王珣:字元琳,祖导父洽。官至尚书令,爵东亭侯。},帝欲见之。尝夜与国宝及雅相对,帝微有酒色,令唤珣,垂至\footnote{垂至:将到。},已闻卒传声。国宝自知才出珣下,恐倾夺其宠\footnote{倾夺:剥夺。},因曰:“王珣当今名流,陛下不宜有酒色见之,自可别诏召也\footnote{别诏:另发诏令。}。”帝然其言,心以为忠,遂不见珣。

{\cangkai\zihao{5}【评】王国宝与王雅,二人于孝武帝朝,俱有“佞幸之目”。但究其事实,则二王本性有异。王国宝出身于太原晋阳王氏家族,宰相谢安之婿,一心只想往上爬,眼睛只看天,是个典型的反复无常谗险小人。他少无士操,丈人谢安恶其倾侧,抑而不用,他即在司马道子面前“间毁安焉”。其母舅范宁疾其阿谀,他即“劝孝武帝黜之”。因此,国宝谗毁王珣,不过是万千小事中的一桩,关键是设计周巧,令帝“心以为忠”——谗间忠良的同时又获“忠”名,可谓一箭双雕,其“高明”在此,于此也可看出当日朝廷的腐败。至于王雅,虽然谨慎胆小,但当孝武欲重用王恭和殷仲堪时,雅谏“恭等无当时之才,不可大任”,又批评王恭“执自是之操,无守节之志”,批评仲堪“虽谨于细行……亦无弘量,且干略不长”,见\CJKunderwave{晋书}雅传。这不是谗毁,而是实事求是之言。但其言不用,后恭与仲堪果然双双落败,给国家政局造成了极大的动荡与破坏。王雅有知人之明,而非谗险小人,性质与王国宝不同。}

\lettrine{32.4} 王绪\myidx{王绪}数谗殷荆州\myidx{殷仲堪}于王国宝\myidx{王国宝}\footnote{王绪:字仲业,太原晋阳人。官至会稽王从事中郎,与王国宝弄权,后被诛。参前\CJKunderwave{规箴}第26则注。殷荆州:殷仲堪,时任荆州刺史,故称。},殷甚患之,求术于王东亭\myidx{王珣}\footnote{王东亭:王珣,封东亭侯,故称。}。曰:“卿但数诣王绪\myidx{王绪}\footnote{数诣:经常去拜访。},往辄屏人\footnote{屏人:屏退左右。},因论他事,如此,则二王之好离矣。”殷从之。国宝见王绪,问曰:“比与仲堪屏人何所道\footnote{比:近来。}?”绪云:“故是常往来\footnote{故:确实。常往来:一般的交往。},无他所论\footnote{无他所论:没有议论什么。}。”国宝谓绪于己有隐,果情好日疏,谗言以息。{\fzxk\zihao{6}\textcolor{red}{按国宝得宠于会稽王,由绪获进,同恶相求,有如市贾,终至诛夷,曾不携贰。岂有仲堪微间而成离隙?}}

{\cangkai\zihao{5}【评】此则可与\CJKunderwave{规箴}第26则相互参考。谗毁与反谗毁是一场复杂而微妙的政治斗争。进谗者的心计与手段,常是令人防不胜防。这就为反谗斗争增加了难度,也就是说,必须巧施智慧来加以反抗。孝武一朝,王国宝、王绪固然是为利舍义的典型谗险小人,但被谗者殷仲堪也不是什么成熟进步的政治家,更非贤良君子,史称其“精心事神,不吝财贿;而怠行仁义,啬于周急”,其败固不足惜,故王雅讥其“干略不长”,自取灭亡。但王珣的反谗之计,则深入对手的内心世界,打一战术心理攻防之战,体现了智慧与技巧,也算是名士才气的一种特殊体现。}







%%% Local Variables:
%%% mode: latex
%%% TeX-engine: xetex
%%% TeX-master: "../Main"
%%% End:
