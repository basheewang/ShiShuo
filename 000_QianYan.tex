%% -*- coding: utf-8 -*-
%% Time-stamp: <Chen Wang: 2025-11-24 17:04:47>

% ○ ◎ ‧ 「 」 『 』 々 ( ) “ ”


\chapter{前言}


\section{一}


学习或研究魏晋六朝文学,有几部重要著作是必备之书:一是文学总集\CJKunderline{萧统}\CJKunderwave{文选},南朝齐梁以前的中国古典优秀作品,大多精选在册,一目了然;一是古典文论名著\CJKunderline{刘勰}\CJKunderwave{文心雕龙}和\CJKunderline{锺嵘}\CJKunderwave{诗品},其理论影响至今仍是痕迹宛然;一是\CJKunderline{刘义庆}\CJKunderwave{世说新语}\footnote{关于\CJKunderwave{世说新语}之撰著者究竟是\CJKunderline{刘义庆},还是其门客所为,研究者有不同意见,这里不拟讨论。}。三者鼎足而立,可见其文学价值和重要历史地位。

\CJKunderwave{世说新语}是一部优秀的古典笔记小说,它网罗诸多魏晋士人的遗闻逸事和文坛佳话,在历史的动态发展中活脱地展现了魏晋时代的社会风情和士人内在的心灵世界,构成了一幅活动的魏晋社会人生的形象历史画卷。它不仅是了解和研究魏晋清谈及其玄学思考必读的活资料,更重要的是,它是一部艺术上成功的文学名著,其语言精练简约,蕴藉隽永,妙语如珠而风趣横生,在幽微的哲理思考中,蕴涵了深沉的人生慨叹,从而给后世以无尽的启迪。作者在一个思想奔逸、玄思深邃的玄学时代中,塑造了翩翩来去的五百多个风流人物,个个鲜活地跃动于字里行间,其俊逸风流的艺术形象,无不栩栩如生,给人留下了玩味不尽的深刻印象。\CJKunderwave{世说新语}以其独特的艺术魅力及其文化影响,为我国古典文学宝库留下了一笔丰厚的文化遗产。

这样一部名著的文本,宋代以后国内流传的,主要是淳熙十五年\CJKunderline{陆游}刻本和淳熙十六年湘中刻本。陆刻本今佚,但其成果幸赖明\CJKunderline{袁褧}“嘉趣堂”翻刻本得以保存;湘中本清初犹见,有\CJKunderline{徐乾学}“传是楼”藏本,后来又有\CJKunderline{沈宝砚}的\CJKunderwave{校语},从此略知其貌。但此本今已不知去向,\CJKunderwave{中国古籍善本书目}未见登录。淳熙两宋本相继亡佚,实为可叹。但幸运的是,更早一些的绍兴八年\CJKunderline{董弅}的刻本却仍存世间。该刻本自南宋末年流入日本,便为国人所未见,先存“金泽文库”,后入前田氏的“尊经阁”,上世纪初才以珂罗版影印传回国内,1956年文学古籍刊行社影印刊行了\CJKunderline{王利器}先生的“断句校订”本,原本现仍藏于日本“尊经阁文库”\footnote{见\CJKunderwave{日藏汉籍善本书录},\CJKunderline{严绍璗}编著,中华书局2007年3月版1253页。董刻绍兴本,另有一本今藏日本宫内厅,但缺\CJKunderline{汪藻}\CJKunderwave{叙录},并非完本。}。此前国内长期流行的宋刻本是\CJKunderline{陆游}、湘中两种,尽管用这两种刻本作为蓝本的研究成果及重雕本最为丰富,多有优胜处,但就版本本身的价值看,\CJKunderline{余嘉锡}先生断言:“三种宋刻本,以第一种\CJKunderline{董弅}本最佳。”\footnote{见\CJKunderwave{世说新语笺疏·凡例},上海古籍出版社1993年版。}\CJKunderline{朱一玄}先生更确切地说明:“现存最早的接近\CJKunderline{刘孝标}注本的最完整的本子,是宋绍兴八年\CJKunderline{董弅}刻本。”\footnote{见\CJKunderwave{世说新语汇校集注·序言},上海古籍出版社2002年版。}

\CJKunderline{刘义庆}的\CJKunderwave{世说新语}(以下简称\CJKunderwave{世说})原著,自有注、抄、刻始,便被删削改易。在今天能见到的最近真的版本当是唐写本,惜为残卷,只有\CJKunderwave{规箴}、\CJKunderwave{捷悟}、\CJKunderwave{夙惠}、\CJKunderwave{豪爽}四门51则,其次,便是宋绍兴年间的董刻本了。

\CJKunderline{董弅},字令升,东平(今属山东)人,两宋之交学者、书画理论家\CJKunderline{董逌}之子。其郡望自谓广川,“盖欲附\CJKunderline{仲舒}裔耳”(\CJKunderwave{四库全书总目})。绍兴七年(1137)始知严州。在知严州期间,颇富文化建设之功,修\CJKunderwave{新定严州志};又编刻\CJKunderwave{严陵集},许多没有专集作者的诗作依赖此集而得以存其梗概;而其\CJKunderwave{世说}刻本则尤有价值。这一刻本不仅近真而完整,而且其后附有\CJKunderline{汪藻}的\CJKunderwave{叙录},传本和\CJKunderwave{叙录}一道,堪称是南宋初年对\CJKunderwave{世说}版本及\CJKunderwave{世说}文本本身研究的一次总结。

前辈学人十分珍视董刻本,运用其校勘成果以求真,近年也引起了时贤的关注,多有研究成果,但对这样一部传世的具有独特价值的传本本身之面貌、特色仍有进一步说明的必要。


\section{二}


这里我们从两个方面概要叙述董刻本近真的面貌。

(一)从文句看,董刻本所保留的用语或更合于原本面貌,或于义更胜。

以今见文献参校,及以\CJKunderwave{世说}本身之内在蕴涵的“理校”,不难体味,从文句看,董刻本所保留的用语或更合于原本面貌,或于义更胜。兹取以下诸例说明之:

\CJKunderwave{政事}(18)“王、刘与林公看何骠骑,骠骑看文书不顾之。王谓何曰:‘我今故与林公来相看,望卿摆拨常务,应对共言,那得方低头看此邪?’……” “共言”,后来的\CJKunderline{袁褧}、\CJKunderline{凌濛初}刻本等皆作“玄言”。这里当以董刻本为是。考之\CJKunderwave{世说},“共言”为当时谈玄的常用说法。时人不直接说“玄言”、“谈玄”,而说“共谈”、“口谈”、“共论”、“共语”、“清言”、“言理”、“微言”、“论理”、“往反”、“言”,等等。董刻本作“共言”更合于原本面貌。

\CJKunderwave{品藻}(3)\CJKunderline{庞士元}语“陶冶世俗,与时浮沈,吾不如子;论王霸之馀策,览倚伏之要害,吾似有一日之长”。 “倚伏”,\CJKunderline{袁褧}刻本作“倚仗”。这里明显以董刻本为是。“倚伏”,在此语境中为\CJKunderline{庞士元}所引\CJKunderwave{老子}典故,“祸兮福之所倚;福兮祸之所伏”(五十八章)。谓了悟祸福相互依存转化的玄理、要义;“要害”,谓枢机关键或规律。这样,\CJKunderline{庞士元}的思理高深就不言而喻了,所表现出来的正是名士风貌。倘作“倚仗”,则此言句义窒碍难通,不知作何解释了。故\CJKunderline{余嘉锡}\CJKunderwave{笺疏}以绍兴董刻本为是。

\CJKunderwave{品藻}(12)“王大将军在西朝时,见周侯辄扇障面,不得住。后度江左,不能复尔。三叹曰:‘不知我进伯仁退。’”“三叹曰”,\CJKunderline{袁褧}、\CJKunderline{凌濛初}刻本作“王叹曰”。此当以董刻本近真而于义更胜。“王”仅指姓氏,前面已有“王大将军”,作为主语可通贯而下,后面省却姓氏,无碍文义。而作“三叹”则不同,愈发见出\CJKunderline{王敦}南渡得势后,骄矜得意的神态,使人物性格鲜明醒目。

\CJKunderwave{任诞}(15)“阮仲容先幸姑家鲜卑婢。及居母丧,姑当远移,初云当留婢,既发,定将去。仲容借客驴,箸重服,自追之,累骑而返,曰:‘人种不可失!’即遥集之母也”。 “定”, \CJKunderline{袁褧}刻本不改,\CJKunderline{沈宝砚}校本作“迺”。作“迺”亦通,然而不如作“定”于义更长。在这里用“迺”,是一个副词,解作“竟然”,有出乎意料的意思,乍看去似有助故事波澜,增强戏剧性,故徐震堮先生认为 “迺”义为长(\CJKunderwave{世说新语校笺}),然而细味起来,则无如“定”隽永。“定”有终究、到底之意,说的是,其结果没有以阮咸原来的主观意志而改变,到底将婢带走了。这个结果包含了其姑对阮咸惠爱至深的苦心和深思熟虑后的决断。因为这是一个婢,而且为异种鲜卑,以阮咸的身份与之生情留恋,无疑是公然挑战世俗、礼法,妄取祸端,其后果不测自明。如此,则其姑对阮咸之爱及聪慧明智,便于一“定”字——她最后的行为中,深含无遗了。有了这一层,才愈见阮咸的“任诞”。他的惊世骇俗之举是动人的,但事实上的结果,正是因此而使他付出了长期沉沦闾巷,被摒于仕途之外的代价。这样看来,“迺”富于暂时的刺激,表达的只是阮咸的瞬间感受与激动果行,突出了故事的戏剧性,而作“定”则相反,它更有深味,更耐咀嚼。“初云”之允诺与“定”之翻悔相映带,深含了时代因素、其姑的矛盾、复杂心理和对事情因果关联的理性判断。在如此一个短则故事里,不止正面表演的主角阮咸,就是隐含幕后的其姑的形象也都丰满活跃了起来;因在他们的形象中,饱含着时代特点、个性特征等丰富的信息而使故事深富意味了。这正表明\CJKunderwave{世说}之品格不以波澜、悬念见长,而是以隽永、深刻独擅胜境,所以作“定”更为本色,更像\CJKunderwave{世说}的语言。

\CJKunderwave{捷悟}(7)“王东亭作宣武主簿,尝春月与石头兄弟乘马出郊。时彦同游者连镳俱进,唯东亭一人常在前,觉数十步,诸人莫之解。石头等既疲倦,俄而乘舆,向诸人皆似从官,唯东亭弈弈在前,其悟捷如此”。此则\CJKunderline{袁褧}刻本等皆作“石头等既疲倦,俄而乘舆回,诸人皆似从官,唯东亭弈弈在前,其悟捷如此”,相沿流传。董刻本、唐写本“回”作“向”,就文意说,当以“向”为是。倘为“回”,则难以解读,石头舍马乘舆,回车而返,这样原超越几十步而在前行的东亭,随方向回转,反而在后似从官了,喜剧意味不在“诸人”, 而落在了东亭,明显此非“捷悟”,反成了“笨伯”笑料,适与本则所要表述的意思相反。而作“向”,为表时间的副词,写出刚才从容马队的“诸人”,现在因石头兄弟舍骑乘舆而使得他们列队车后,跨马相随,形同“从官”了,此时“唯东亭弈弈在前”。这一对比,才见出东亭先见之明远在“时彦”之上的“捷悟”。可见唐写本、董刻本为是,保留了原本的真实面貌。故通行的现代诸本,徐震堮\CJKunderwave{校笺}以“作‘向’为是”,杨勇、张万起、刘尚慈诸先生也都采纳了唐写、董刻的用法。

\CJKunderwave{豪爽}(6)“王大将军始欲下都,更分树置,先遣参军告朝廷,讽旨时贤。祖车骑尚未镇寿春, 目厉声语使人曰:‘卿语阿黑:何敢不逊!催摄面去,须臾不尔,我将三千兵槊脚令上。’王闻之而止”。“更分”袁本作“处分”,而唐写本、董刻本、沈校本同作“更分”。“处分”为处置、处理、安排之义,与下文“树置”一起,是说\CJKunderline{王敦}要安排设置官员。实质上故事要表达的是,\CJKunderline{王敦}包藏祸心,拥兵威胁,按自己的意愿重新设置官员,安插党羽,即“更分者更动处分,有所树置也”\footnote{见\CJKunderwave{世说新语汇校集注},朱铸禹校注,上海古籍出版社2002年12月版512页。},其要害是更换现有官署的执事人员,以为其实现野心铺平道路。显然“处”显得平平,“更”比“处”表义更深刻,更传神。故徐震堮、杨勇等先生亦皆以唐写、董刻为是。

以上都说明董刻本颇能体味\CJKunderwave{世说},保留其本色,而这种保留也表达了\CJKunderline{董弅}对\CJKunderwave{世说}文本、风格的熟悉和他卓有识见的学养。从而使\CJKunderwave{世说}得以近真的面貌保留下来,原著作之胜情胜意不致被损害,不致因失之毫厘而谬以千里,诓哄了后来的读者,就文献价值说来,这恐怕是最为珍贵的。

(二) 董本的错误处,仍为近真传本的风貌。

可以用来作为董刻本近真品性的反证是,董刻本有些错误处,也是近真传本的面貌。它表明,\CJKunderline{董弅}雕本在未及深察时,并没有主观臆断,妄加改易,而是尽量一仍其旧,“故亦传疑,以俟通博”,忠于实情。兹亦取诸例,加以说明:

董刻本\CJKunderwave{文学}(80)习凿齿因忤旨“出为荥阳郡”,\CJKunderline{袁褧}刻本作“衡阳”,是。但\CJKunderline{朱铸禹}先生的\CJKunderwave{世说新语汇校集注}指出,\CJKunderwave{晋书}卷八十二习凿齿的本传也作“荥阳”,这显系当时写本就如此,非董刻致误。这或是\CJKunderline{董弅}对“旧语”在未及弄清楚的时候,“故亦传疑,以俟通博”,而保留了原样。

董刻本\CJKunderwave{尤悔}(3)“陆平原沙桥败,为卢志所谗,被诛”。“沙桥”,\CJKunderline{袁褧}刻本为“河桥”,\CJKunderline{王利器}考证:“案,作‘河桥’是,\CJKunderwave{通鉴}卷一一四\CJKunderwave{晋纪}三六注‘沙桥在江陵北’。据\CJKunderwave{晋书·陆机传}‘列军自朝歌至于河桥’则河桥在朝歌附近,与江陵之沙桥,地望之差何止千里。”\footnote{见\CJKunderwave{世说新语·校勘记},\CJKunderline{王利器}断句校订,文学古籍刊行社1956年6月版66、67页。}这一点,\CJKunderline{刘孝标}本则注引\CJKunderwave{陆机别传}也可以佐证:“及(陆)机于七里涧大败,(孟)玖诬(陆)机谋反所致,(司马)颖乃使牵秀斩(陆)机。”又\CJKunderwave{晋书·陆机传}:“(陆)机军大败,赴七里涧而死者如积,水为之不流……”七里涧,\CJKunderwave{通鉴}卷八十四\CJKunderwave{晋纪}六注:“\CJKunderwave{水经注}:鸿台陂在洛阳东北二十里,其水东流,左合七里涧。”朝歌在洛阳东,则陆机兵败之河桥(七里涧),在洛阳、朝歌之间,而近于朝歌。孝标所引\CJKunderwave{别传}是概说,“河桥”讲得更具体指实,是以个别代全体战场的借代手法。无论如何,是洛阳、朝歌间的“河桥”而非江陵之“沙桥”,\CJKunderline{王利器}、\CJKunderline{刘孝标}两说,皆可对之确证无疑。董刻本用“沙桥”,未及深察而致误,然并非只有董刻本误,后来\CJKunderline{沈宝砚}以淳熙十六年湘中刻本为底本校勘时也未出校,仍作“沙桥”。此亦为当时写本如此。

董刻本\CJKunderwave{言语}(108)\CJKunderline{刘孝标}注引\CJKunderwave{庄子·渔父篇}:“子修心守真,还以物与人,则无异矣。”“无异”,今本\CJKunderwave{庄子·渔父篇}作“无累”。作“无异”则此句无法释读,然而不独董本,诸本皆同,为原初诸传本之误。

董刻本\CJKunderwave{雅量}(40)\CJKunderline{刘孝标}注:“徐广\CJKunderwave{晋纪}曰:‘泰元二十年九月,有蓬星如粉絮,东南行,历须女至央星。’”“央星”,\CJKunderline{袁褧}刻本同,\CJKunderline{沈宝砚}校本作“哭星”。\CJKunderwave{晋书·天文志}作“哭星”;\CJKunderline{王利器}引\CJKunderwave{开元占经}卷八十六、\CJKunderwave{御览}卷八七五引\CJKunderwave{晋中兴书}载\CJKunderwave{世说}此事作“历女虚危,至哭星”\footnote{见\CJKunderwave{世说新语·校勘记},\CJKunderline{王利器}断句校订,文学古籍刊行社1956年6月版,33页。}。可见淳熙十五年\CJKunderline{陆游}的刻本也据所见原刻,误作“央星”,袁据此重刻时未及校改。

上举诸例,可以窥见一般情形,这些都说明当时传本之间差异甚大,\CJKunderline{董弅}雕本在未及深察时,并没有主观臆断,妄加改易,而是尽量传其原来面貌。

综合前述诸项,我们可以看到,\CJKunderline{董弅}刻本所保留的\CJKunderwave{世说}的近真的面貌,对\CJKunderwave{世说}的阅读、品味,对\CJKunderwave{世说}的研究都有其独特的价值,就版本意义上说,这在今天能见到的诸传本中,它近于本真面貌的文献价值是很值得玩味深思的。

\section{三}


董刻本另一个独特的价值,是来自它所保留的宋代文人\CJKunderline{汪藻}对\CJKunderwave{世说}的研究成果。

董刻本中保留的\CJKunderline{汪藻}\CJKunderwave{叙录},是今见最早对\CJKunderwave{世说}进行全面研究的成果,其贡献和价值,学界曾给予关注,这里在过去研究的基础上,从董刻本价值的角度,再做一点钩沉。

\CJKunderline{汪藻},“幼颖异,入太学”,\CJKunderline{宋徽宗}崇宁五年(1106)进士及第,因忤权贵,仕宦并不顺利,所以终生“博极群书,老不释卷……多著述”\footnote{见\CJKunderwave{宋史·文苑七}卷四百四十五。}。其中\CJKunderwave{世说叙录}便是极有价值的\CJKunderwave{世说新语}研究著述。它介绍了北宋年间\CJKunderwave{世说}的家藏和校勘情况,并厘正了此前纷扰难清的诸如名称、卷数、版本等问题。汪氏在胪列、辨析了诸家说法之后,做出了这样一些结论:关于书名“晁氏诸本皆作\CJKunderwave{世说新语},今以\CJKunderwave{世说新语}为正”;关于卷数“以九卷为正”;关于篇数“定以三十六篇为正”。这在形成绍兴八年\CJKunderline{董弅}刻本时起了作用,董刻本刊行时,不但名称、篇数与汪氏同,而且将三十六篇厘为上、中、下三卷。其后出现之诸本,大抵采用了这些做法。

\CJKunderline{汪藻}\CJKunderwave{叙录}中还有极具价值的\CJKunderwave{考异}一卷。该卷录五十一事,为今见最早的注本——齐梁间人\CJKunderline{敬胤}所注。其注文广引当时诸记,以明\CJKunderwave{世说}所记人物、事迹,这与后来\CJKunderline{刘孝标}注有相同之处,但其剪裁,不如孝标精要,显得芜杂,所引群书亦不如孝标广博,不过作为早于孝标之注,其开拓之功,功不可没,因而仍有值得重视之处。

其一,\CJKunderline{敬胤}引书保留了当时的史料,有些与\CJKunderline{孝标}不同,而仅见于\CJKunderline{敬胤}注,故为史家所珍视,并援引以征史实。

其二,\CJKunderline{汪藻}颇疑“\CJKunderline{敬胤}专录此,传疑纠谬”,而且所载五十一事,有三则为传本所无,其馀“悉重出”。汪氏所疑甚是。\CJKunderline{敬胤}之注多对\CJKunderline{刘义庆}之著“传疑纠谬”。而其“纠谬”,不仅勘正史实,且在叩问中将\CJKunderwave{世说}作为“小说”的性质,也不自觉地揭示了出来。如其第一则(原在\CJKunderwave{世说·言语}29)“元帝始过江”,\CJKunderline{敬胤}就颇疑此则故事的真实性。针对元帝说“寄人国土”及顾荣对元帝呼“陛下”,其纠谬云:“元帝之镇建业,于时天下虽乱,而朝廷尤存,经年之后,方还本国葬太妃,方伯述职,何谓为寄也?”诚然,司马睿当时虽镇建业,但不过方伯而已,并不是皇帝,何来以君主的姿态去感受出“寄人国土”之羞惭?\CJKunderline{敬胤}又驳:“元帝永嘉元年,以顾荣为安东军司。五年(元帝)进号镇东,荣为军司。其年荣卒。后七岁,元帝方为天子,岂得此时,便为‘陛下’,已曰‘迁都’邪?”\CJKunderline{刘义庆}撰述此事,传闻而已,顾荣并未及呼陛下,当时也没有称帝迁都之事。可见\CJKunderline{敬胤}勘正史实之用功。这点已为\CJKunderline{余嘉锡}先生的研究所采纳。由于史官文化的强劲惯性,即使魏晋时人,也还习惯以“真实”来看待当时的笔记小说,即使是干宝\CJKunderwave{搜神记}也概莫能外。因此,在这样的时代风气下,\CJKunderline{敬胤}以史家征实的态度去看待\CJKunderwave{世说},因而发谬叩问,也属自然之事。然正是这一驳问,见出了\CJKunderwave{世说}一书的真性质,它与当时的\CJKunderwave{语林}等旨趣相类,“要为远实用而近娱乐”,后来\CJKunderline{刘孝标}作注就不这样胶着,而是或驳或申,旨在“映带文本,增其隽永”\footnote{鲁迅\CJKunderwave{中国小说史略·〈世说新语〉与其前后},人民文学出版社1981年版。}。这一性质,已为后来用文学批评的眼光看待\CJKunderwave{世说}的批评家所自觉领悟,\CJKunderline{刘辰翁}等批评\CJKunderwave{世说}就略其玄黄而取其神韵,不再执着于征实,而是评点人物与文章神采。在后来的史家眼里,即\CJKunderline{刘孝标}也是文学家,而非史家\footnote{见唐长孺\CJKunderwave{魏晋南北朝史论拾遗},中华书局1983年版。}。由此可见,\CJKunderline{敬胤}这位早期的\CJKunderwave{世说新语}研究家的贡献和其成果的价值。

其三,\CJKunderline{敬胤}所录五十一事,实为专题研究,它们相对集中在当时影响甚大,几乎是左右东晋王朝命运的几个势族和豪帅人物身上。\CJKunderline{刘琨}、\CJKunderline{祖逖}为一组,凡八则。\CJKunderline{祖逖}、\CJKunderline{刘琨}无论从北伐健将,还是从北来流民豪帅的角度说,都是东晋一朝颇具内涵,引人注目的专题。\CJKunderline{王敦}一组,凡十六则;王导一组,凡二十三则。琅邪王氏是助成\CJKunderline{司马睿}东晋政权的最重要、最核心的势族之一,他们一方面要求巩固王权,另一方面又力争代表世家大族利益,使得王权和世族利益平衡发展,对此,他们所起的作用都是无人能代替的。无论\CJKunderline{王敦}的起兵威胁司马政权,还是王导以“网漏吞舟”之政维系江左政局,东晋享国百馀年,实非偶然。琅邪王氏的这两个人物,无疑都是东晋一朝最醒目的专题。不知\CJKunderline{敬胤}是否如\CJKunderline{孝标}全面注过\CJKunderwave{世说},但仅就此五十一则看,就足以证明\CJKunderline{敬胤}是以研究者的视角去面对\CJKunderwave{世说}的,并给我们留下了成果和启发\footnote{\CJKunderline{敬胤}是否全面注过\CJKunderwave{世说},研究者也有不同意见,此点我们别有说。}。

\CJKunderline{汪藻}除了别具慧眼保留了\CJKunderline{敬胤}注之\CJKunderwave{考异}一卷外,还对“凡\CJKunderwave{世说}人物可谱者”,做了谱牒。谱牒既是汪氏的研究成果,也是后人阅读\CJKunderwave{世说}的工具。它已经引起了当代史学家的重视和使用\footnote{见田余庆\CJKunderwave{东晋门阀政治},北京大学出版社1989年版。}。

最后,\CJKunderline{汪藻}又对“无谱者二十六族”的人物在\CJKunderwave{世说}中出现的一人之不同称谓做了索引排列,甚便\CJKunderwave{世说}的阅读。后来,明代\CJKunderline{凌濛初}刻本及一些校本,虽详列\CJKunderwave{世说}中一人之不同称谓及同姓名者、名与字同者,以期方便读者,但毫无疑问,论学术贡献汪氏的谱牒应记首功。

总之,董刻本所保留的\CJKunderline{汪藻}\CJKunderwave{叙录},可说是对\CJKunderwave{世说}文本及\CJKunderwave{世说}研究都进行了较难得的清理。仅就这一点看,董刻本的文献价值便为其他诸刻所无法代替。

\section{四}

如前所述,董刻本的价值是重要而独特的,之所以会出现这种情况,其深层原因在于,其实不止\CJKunderline{晏殊}、\CJKunderline{汪藻}等学者,即\CJKunderline{董弅}本人也是以研究者的态度和学养去对待\CJKunderwave{世说新语}的,这在其刊本的跋语中说得很明确\footnote{\CJKunderline{朱一玄}\CJKunderwave{世说新语汇校集注·序言}:今见的\CJKunderline{董弅}刻本“后边所附的宋\CJKunderline{汪藻}\CJKunderwave{世说叙录}末尾残缺,董氏的跋语也就因而不可得见。幸而这个跋语能在宋淳熙戊申\CJKunderline{陆游}重刻本上保存下来,才使我们藉以了解到宋绍兴本的校刻经过”。} :“晋人雅尚清谈,唐初史臣修书,率意窜定,多非旧语,尚赖此书(按,指\CJKunderline{晏殊}手校本)以传后世。然字有伪舛,语有难解,以它书正之,间有可是正处。而注亦比晏本时为增损。至于所疑,则不敢妄下雌黄,故亦传疑,以俟通博。”以上说法,所可注意者应有三点:

(一)\CJKunderline{董弅}对“旧语”,也就是\CJKunderwave{世说}原本的本真面貌特别关注,不满于“唐初史臣修书,率意窜定”。他面对当时诸种传本以及包括唐初史臣所修之正史在内对\CJKunderwave{世说}的运用,都报以审慎的批判、筛选的态度,而其批判所建立的标准,就是力求保持\CJKunderwave{世说}原貌。尽量还原其旧,不能“率意”对待,更要剔出其“窜定”成分,所以\CJKunderline{董弅}选取了当时所见更为近真而可信的\CJKunderline{晏殊}手校本作为底本。就底本的选择本身看,便说明了\CJKunderline{董弅}曾做过一番研究,而这种研究所守持的是求真、求实的科学态度,他企望以\CJKunderwave{世说}的真面貌传世。至少,这是他重刻\CJKunderwave{世说}时心中所追求的目标。

(二)在选取了\CJKunderline{晏殊}手校本后,\CJKunderline{董弅}的工作并未了结,而是进一步研究,“以它书正之”。所谓“正之”就是在当时所能见到的传本中去对比研究,做了甄别、剔抉的还原工作。仅以\CJKunderline{汪藻}\CJKunderwave{叙录}所列十馀种传本的情况观察,便可以说明,\CJKunderline{董弅}当时所见传本,不乏足资参考、是正的有价值的资料。在这样的基础上,他做了慎重的甄别、剔抉,不是全面改造,而是对“字有伪舛,语有难解”处,“间有”校订。所为“增损”者,有其依据,这点恐是不容置疑的。经过这样一些努力,拿出的底本,就成为一个相对详善的传本。以其所追求的目标看,这个传本,的确更为近真。

(三)\CJKunderline{董弅}在处理这样一个底本时,遵循了“多闻阙疑”的治学原则。“至于所疑,则不敢妄下雌黄,故亦传疑,以俟通博。”这在前文的介绍中,已经看到了,对于传本异说,对于未及弄清楚的地方,他没有强作解人,而是一仍其旧,保留了原貌,提供给后来者去作自己的思考。

\CJKunderline{董弅}的夫子自道,除明显属雕工形近误刻及漏刻者外,证之以版本全貌,基本是可信的。

另一方面,也是因为\CJKunderline{董弅}以研究者的姿态和学养对待\CJKunderwave{世说},使他能识别并推重当时杰出研究成果,不止版本取优,再加勘订,而且能将\CJKunderline{汪藻}的成果一并刻入书中,充分显示了宋代的研究水平,足以泽惠后学。\CJKunderline{董弅}的这种态度和工作,使得这一刻本成为至南宋初年为止,众多学者对\CJKunderwave{世说}传本及\CJKunderwave{世说}文本本身研究的一次全面的总结,并将其成果以新刊版本的形式保留了下来。

董刻本自身经过了这样较为严谨的处理,又采择当时传本的众多研究成果,使其具有近真的本色,力求切近\CJKunderwave{世说}文本的历史原貌,在今天,它所显示的文献价值和艺术价值,正如\CJKunderline{余嘉锡}先生所评,在诸传本中堪称“最佳”了。

以上事例,说明了董刻本的学术价值。本注评,即选取绍兴董刻本为底本,尽量保持其原貌,同时也从诸校本吸取其校订成果,并在注中予以说明,以期尽可能完整、准确地把握文本。在此基础上,挖掘其内涵,剔抉其神韵。

我们深知解读\CJKunderwave{世说新语}并非易事,虽已经有很多学者对之校勘、笺识,也有诸多今注今译本行世,还有对当时语言的专门研究著作,但对每则故事的发覆索隐,征求神韵,赏会妙处,仍然需要诸多努力。\CJKunderwave{世说}受魏晋玄风的影响,其写作重在“得意忘言”,因而语言简约精练,文字隽永有味,作者的深邃思考和思想意趣,常是意在言外,而尽得风流。解读这样一部著作,就需要做许多\CJKunderwave{世说}之外的功夫,体会当时特殊的历史文化、独有的思想特色,以及由此而造就的一系列独特的历史人物,他们的生动、灵妙,都需在这一背景中去求证,因而,我们在探求故事的本事及言外之义、味外之旨的时候,首先深入到当时的历史文化背景中去,尽可能求取一个近真的面貌,尽可能准确地把握人物,将其风流韵味做还原的评点。同时,我们不仅注意到它的言外之味在当时的作用和影响以及历史意义,也关注到它给予今天人们的新的启发,将社会、人生、审美等连贯起来,既做史的求证,又做审美赏会,进一步引发了现代联想,突出经典重读那温故知新的意义。忠实于原作,求得其神韵,赏会其妙处,追寻其启迪意义是我们的努力和期望所在。

总之,阅读\CJKunderwave{世说},最难的不是表面的语言文字的训诂解释,而是在结合时代以还原历史的综合能力和整体把握,以便进一步解开寓藏在语言文字背后的“言外之意”——即故事的精神实质,作为历史的借鉴。我们尽量把故事安放在一定的历史文化背景之中,力求还历史的真实面貌,让读者知其然,明白“是什么”的问题,这是颇有难度的;进一步,还必须让读者知其所以然,明白作者之用心,他这样的写作,用意何在?这是弄清“为什么”的问题;最后,还必须让读者明其所当然,也就是从故事的启悟来思考人生,站在时代的高度做现代反思,明白自己应该如何行动与实践,弄清楚我们应该“怎么想”的问题。用精约的语言,力求揭示上述三个“W”(即“是什么”、“为什么”和“怎么想”),让读者不仅知其然,还要知其所以然,进一步还要明其所当然。环环相扣,步步深入,任重道远,可不勉哉!

以下简略介绍一下注评的体例。

1.对版本、校勘的处理。本注评以宋绍兴八年\CJKunderline{董弅}刻本为底本,吸收前贤的校勘成果,并在注中标明。本书侧重评点,因而注释尽量简约。

2.对古今字、异体字、正俗字的处理。因文字改为简体,古今字、异体字、正俗字,无法一一依照董刻复原,祈谅。今后若出繁体文本,当力求存原貌而少作改动。因为原作多用俗体字,见其从俗倾向,在文学史的雅、俗之争中,有助于明了笔记小说的走向。

3.评点从每则故事本身的特点出发,或评人物,或评思想特色,或揭示当时风俗、风貌,或评点写作特色,或评言语韵味,在评点中对已往评点的成果尽量吸收,择善而从。由于评点为每人独立写作,作者的行文风格不强做统一。

本书的写作,\CJKunderline{蒋凡}撰写上卷的\CJKunderwave{德行},中卷的\CJKunderwave{规箴}、\CJKunderwave{捷悟}、\CJKunderwave{夙惠}、\CJKunderwave{豪爽}以及下卷的全部注评;\CJKunderline{李笑野}撰写\CJKunderwave{前言}和上卷的\CJKunderwave{言语}、\CJKunderwave{政事}、\CJKunderwave{文学}注评;\CJKunderline{白振奎}撰写中卷的\CJKunderwave{方正}、\CJKunderwave{雅量}、\CJKunderwave{识鉴}、\CJKunderwave{赏誉}、\CJKunderwave{品藻}注评。全书由\CJKunderline{蒋凡}审阅。

对\CJKunderwave{世说新语}作全面的现代视角的评点,对我们说来,是一次大胆的尝试,缺点、不足之处在所难免,敬祈读者批评指正。


%%% Local Variables:
%%% mode: latex
%%% TeX-engine: xetex
%%% TeX-master: "Main"
%%% End:
