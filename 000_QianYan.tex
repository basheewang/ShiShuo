%% -*- coding: utf-8 -*-
%% Time-stamp: <Chen Wang: 2025-11-24 16:18:40>

% ○ ◎ ‧ 「 」 『 』 々 ( ) “ ”


\chapter{前言}


\section{一}


学习或研究魏晋六朝文学,有几部重要著作是必备之书:一是文学总集萧统《文选》,南朝齐梁以前的中国古典优秀作品,大多精选在册,一目了然;一是古典文论名著刘勰《文心雕龙》和锺嵘《诗品》,其理论影响至今仍是痕迹宛然;一是刘义庆《世说新语》 [1] 。三者鼎足而立,可见其文学价值和重要历史地位。

《世说新语》是一部优秀的古典笔记小说,它网罗诸多魏晋士人的遗闻逸事和文坛佳话,在历史的动态发展中活脱地展现了魏晋时代的社会风情和士人内在的心灵世界,构成了一幅活动的魏晋社会人生的形象历史画卷。它不仅是了解和研究魏晋清谈及其玄学思考必读的活资料,更重要的是,它是一部艺术上成功的文学名著,其语言精练简约,蕴藉隽永,妙语如珠而风趣横生,在幽微的哲理思考中,蕴涵了深沉的人生慨叹,从而给后世以无尽的启迪。作者在一个思想奔逸、玄思深邃的玄学时代中,塑造了翩翩来去的五百多个风流人物,个个鲜活地跃动于字里行间,其俊逸风流的艺术形象,无不栩栩如生,给人留下了玩味不尽的深刻印象。《世说新语》以其独特的艺术魅力及其文化影响,为我国古典文学宝库留下了一笔丰厚的文化遗产。

这样一部名著的文本,宋代以后国内流传的,主要是淳熙十五年陆游刻本和淳熙十六年湘中刻本。陆刻本今佚,但其成果幸赖明袁褧“嘉趣堂”翻刻本得以保存;湘中本清初犹见,有徐乾学“传是楼”藏本,后来又有沈宝砚的《校语》,从此略知其貌。但此本今已不知去向,《中国古籍善本书目》未见登录。淳熙两宋本相继亡佚,实为可叹。但幸运的是,更早一些的绍兴八年董弅的刻本却仍存世间。该刻本自南宋末年流入日本,便为国人所未见,先存“金泽文库”,后入前田氏的“尊经阁”,上世纪初才以珂罗版影印传回国内,1956年文学古籍刊行社影印刊行了王利器先生的“断句校订”本,原本现仍藏于日本“尊经阁文库” [2] 。此前国内长期流行的宋刻本是陆游、湘中两种,尽管用这两种刻本作为蓝本的研究成果及重雕本最为丰富,多有优胜处,但就版本本身的价值看,余嘉锡先生断言:“三种宋刻本,以第一种董弅本最佳。” [3] 朱一玄先生更确切地说明:“现存最早的接近刘孝标注本的最完整的本子,是宋绍兴八年董弅刻本。” [4]

刘义庆的《世说新语》(以下简称《世说》)原著,自有注、抄、刻始,便被删削改易。在今天能见到的最近真的版本当是唐写本,惜为残卷,只有《规箴》、《捷悟》、《夙惠》、《豪爽》四门51则,其次,便是宋绍兴年间的董刻本了。

董弅,字令升,东平(今属山东)人,两宋之交学者、书画理论家董逌之子。其郡望自谓广川,“盖欲附仲舒裔耳”(《四库全书总目》)。绍兴七年(1137)始知严州。在知严州期间,颇富文化建设之功,修《新定严州志》;又编刻《严陵集》,许多没有专集作者的诗作依赖此集而得以存其梗概;而其《世说》刻本则尤有价值。这一刻本不仅近真而完整,而且其后附有汪藻的《叙录》,传本和《叙录》一道,堪称是南宋初年对《世说》版本及《世说》文本本身研究的一次总结。

前辈学人十分珍视董刻本,运用其校勘成果以求真,近年也引起了时贤的关注,多有研究成果,但对这样一部传世的具有独特价值的传本本身之面貌、特色仍有进一步说明的必要。

[1] 关于《世说新语》之撰著者究竟是刘义庆,还是其门客所为,研究者有不同意见,这里不拟讨论。

[2] 见《日藏汉籍善本书录》,严绍璗编著,中华书局2007年3月版1253页。董刻绍兴本,另有一本今藏日本宫内厅,但缺汪藻《叙录》,并非完本。

[3] 见《世说新语笺疏·凡例》,上海古籍出版社1993年版。

[4] 见《世说新语汇校集注·序言》,上海古籍出版社2002年版。

%%% Local Variables:
%%% mode: latex
%%% TeX-engine: xetex
%%% TeX-master: "Main"
%%% End:
