%% -*- coding: utf-8 -*-
%% Time-stamp: <Chen Wang: 2025-11-24 17:04:47>

% ○ ◎ ‧ 「 」 『 』 々 ( ) “ ”


\part{上卷}


\chapter{德行第一}


{\cangkai\zihao{5}\noindent{} 【题解】今本《世说新语》(以下简称《世说》)共三十六门类,人称以《论语·先进》所载孔门四科(德行、言语、政事、文学)冠其首。此话不假。但若论其区分门类的标准及其精神实质,则因作者及所录人物的生活年代不同,已具有魏晋六朝时代的新鲜风貌,而不必与先秦两汉传统儒家观念尽皆相同。魏晋六朝的时代思潮,玄风炽煽,释理禅义,熏染了一代士人。因而魏晋士人的思想面貌,道德标准及其言语行事,既继往,又开来,在旧模式中又具有新内容和新突破。本门所称“德行”,当作如是观。汉郑玄曾云:“德行,内外之称,在心为德,施之为行。”(《周礼·地官·师氏》注)所谓“德行”,顾名思义,是道德与品行,指人们的道德观念及其行为实践。但如“言皆玄远,未尝臧否人物”的阮籍一类人物,已成为魏晋士人心仪之典型,正见当时道德观念的微妙变化及其时代复杂性。}

1.1 陈仲举言为士则 [1] ,行为世范 [2] ,登车揽辔 [3] ,有澄清天下之志。《汝南先贤传》曰:“陈蕃字仲举,汝南平舆人。有室荒芜,不扫除。曰:‘大丈夫当为国家扫天下。’值汉桓之末,阉竖用事,外戚豪横。及拜太傅,与大将军窦武谋诛宦官,反为所害。” 为豫章太守 [4] ,《海内先贤传》曰:“蕃为尚书,以忠正忤贵戚,不得在台,迁豫章太守。” 至,便问徐孺子所在 [5] ,欲先看之 [6] 。谢承《后汉书》曰:“徐穉字孺子,豫章南昌人。清妙高时(他本作‘跱’),超世绝俗。前后为诸公所辟,虽不就,及其死,万里赴吊。常预炙鸡一只,以绵渍酒中,暴干以裹鸡,径到所赴冢隧外,以水渍绵,斗米饭,白茅为藉,以鸡置前。酹酒毕,留谒即去,不见丧主。” 主簿白 [7] :“群情欲府君先入廨 [8] 。”陈曰:“武王式商容之闾 [9] ,席不暇 [10] 。许叔重曰:“商容,殷之贤人,老子师也。”车上 曰式。 吾之礼贤,有何不可!”袁宏《汉纪》曰:“蕃在豫章,为穉独设一榻,去则悬之,见礼如此。”

【评】

据《后汉书·陈蕃传》,同一故事,是东安太守礼遇周璆,与《世说》不同。然唐王勃《滕王阁序》有“徐孺下陈蕃之榻”的名言,充分说明了《世说》的影响及魅力。礼贤下士,是当时人们称颂的美德,与士人们匡时救国的理想密切相关。史称陈蕃性“方峻”,不交非类。但对贤人高士则不因其地位卑微而轻之,悬榻示敬,正见其评价人物以道德为先,而非以功名爵禄为准。本则故事,言约旨远,正气凛然。“澄清天下”诸语,掷地铿然有声,见后汉志士仁人力挽狂澜理想之远大。思想是行为的指南,陈蕃最后明知必死,而慷慨赴义,终成一代士人之典范。

%%% Local Variables:
%%% mode: latex
%%% TeX-engine: xetex
%%% TeX-master: "Main"
%%% End:
